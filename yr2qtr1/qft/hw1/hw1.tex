%%%%%%%%%%%%%%%%%%%%%%%%%%%%%%%%%%%%%%%%%
% Short Sectioned Assignment
% LaTeX Template
% Version 1.0 (5/5/12)
%
% This template has been downloaded from:
% http://www.LaTeXTemplates.com
%
% Original author:
% Frits Wenneker (http://www.howtotex.com)
%
% License:
% CC BY-NC-SA 3.0 (http://creativecommons.org/licenses/by-nc-sa/3.0/)
%
%%%%%%%%%%%%%%%%%%%%%%%%%%%%%%%%%%%%%%%%%

%----------------------------------------------------------------------------------------
%	PACKAGES AND OTHER DOCUMENT CONFIGURATIONS
%----------------------------------------------------------------------------------------

\documentclass[fontsize=11pt]{scrartcl} % 11pt font size

\usepackage[T1]{fontenc} % Use 8-bit encoding that has 256 glyphs
\usepackage[english]{babel} % English language/hyphenation
\usepackage{amsmath,amsfonts,amsthm} % Math packages
\usepackage{mathrsfs}

\usepackage[margin=1in]{geometry}

\usepackage{sectsty} % Allows customizing section commands
\allsectionsfont{\centering \normalfont\scshape} % Make all sections centered, the default font and small caps

\usepackage{fancyhdr} % Custom headers and footers
\pagestyle{fancyplain} % Makes all pages in the document conform to the custom headers and footers
\fancyhead{} % No page header - if you want one, create it in the same way as the footers below
\fancyfoot[L]{} % Empty left footer
\fancyfoot[C]{} % Empty center footer
\fancyfoot[R]{\thepage} % Page numbering for right footer
\renewcommand{\headrulewidth}{0pt} % Remove header underlines
\renewcommand{\footrulewidth}{0pt} % Remove footer underlines
\setlength{\headheight}{13.6pt} % Customize the height of the header

\numberwithin{equation}{section} % Number equations within sections (i.e. 1.1, 1.2, 2.1, 2.2 instead of 1, 2, 3, 4)
\numberwithin{figure}{section} % Number figures within sections (i.e. 1.1, 1.2, 2.1, 2.2 instead of 1, 2, 3, 4)
\numberwithin{table}{section} % Number tables within sections (i.e. 1.1, 1.2, 2.1, 2.2 instead of 1, 2, 3, 4)

\newcommand{\R}{\mathbb{R}}
\newcommand{\Q}{\mathbb{Q}}
\newcommand{\N}{\mathbb{N}}
\newcommand{\C}{\mathbb{C}}
\newcommand{\Z}{\mathbb{Z}}

\newcommand{\ad}{a^{\dagger}}

\newtheorem{lemma}{Lemma}
%----------------------------------------------------------------------------------------
%	TITLE SECTION
%----------------------------------------------------------------------------------------

\newcommand{\horrule}[1]{\rule{\linewidth}{#1}} % Create horizontal rule command with 1 argument of height

\title{	
\normalfont \normalsize 
\textsc{QFT} \\ [25pt] % Your university, school and/or department name(s)
\horrule{0.5pt} \\[0.4cm] % Thin top horizontal rule
\huge Homework 1 \\ % The assignment title
\horrule{2pt} \\[0.5cm] % Thick bottom horizontal rule
}

\author{Daniel Halmrast} % Your name

\date{\normalsize\today} % Today's date or a custom date

\begin{document}

\maketitle % Print the title

% Problems
\section*{Problem 1}
Show that multiparticle nonrelativistic quantum can be recovered from QFT.
Namely, define
\[
    H = \int d^3x\ad(x)\left( \frac{-\hbar^2}{2m}\nabla^2 + U(x)
    \right)a(x) + \int d^3xd^3yV(x-y)\ad(x)\ad(y)a(y)a(x)
\]
and
\[
    |\psi,t\rangle = \int d^3x_1\dots d^3x_n
    \psi(x_1,\dots,x_n;t)\ad(x_1)\dots\ad(x_n)|0\rangle
\]
We will show that $|\psi,t\rangle$ satisfies the abstract Schrodinger equation
if and only if $\psi$ satisfies the Schrodinger equation
\[
    i\partial_t\psi = H\psi
\]
for
\[
    H = \sum_{i=1}^n \frac{-\hbar^2}{2m}\nabla^2_i + U(x_i) +
    \sum_{j=1}^n\sum_{i=1}^{j-1}V(x_i-x_j)
\]

We calculate $H|\psi,t\rangle$ directly in three parts. That is:
\[
    \begin{aligned}
    H|\psi,t\rangle &= 
    \int d^3\ad(x)\frac{-\hbar^2}{2m}\nabla^2a(x)|\psi,t\rangle\\
        &+ \int d^3x \ad(x)U(x)a(x)|\psi,t\rangle\\
        &+ \int d^3xd^3yV(x-y)\ad(x)\ad(y)a(y)a(x)|\psi,t\rangle\\
    \end{aligned}
\]
For this calculation, we'll use the fact that
\[
    a(x)\ad(x_1)\dots\ad(x_n) =
    \sum_{i=1}^n(-1)^{i-1}_{\pm}\delta(x-x_i)\ad(x_1)\dots\hat{\ad(x_i)}\dots\ad(x_n) +
    (-1)^n_{\pm}
    \ad(x_1)\dots\ad(x_n)a(x)
\]
where $(-1)^n_{\pm}$ indicates the $(-1)^n$ only appears in the fermionic
calculation, and $\hat{\ad(x_i)}$ indicates that the $i$th creation operator is
omitted. This is calculated by iterated application of the rule
\[
    a(x)\ad(y) = \pm\ad(y)a(x) + \delta(x-y)
\]
with a plus sign for bosonic calculations, and a minus sign for fermionic
calculations.

We can now calculate directly the three components of $H|\psi,t\rangle$ as
\[
    \begin{aligned}
        \int d^3x\ad(x)\frac{-\hbar^2}{2m}\nabla^2a(x)|\psi,t\rangle
        &= 
        \int d^3x_1\dots d^3x_n
        d^3x\ad(x)\frac{-\hbar^2}{2m}\nabla^2a(x)\psi\ad(x_1)\dots\ad(x_n)|0\rangle\\
        &=
        \int d^3x_1\dots d^3x_n
        d^3x\ad(x)\frac{-\hbar^2}{2m}\nabla^2\psi\\
        &\left( 
            \sum_{i=1}^n(-1)^{i-1}_{\pm}\delta(x-x_i)\ad(x_1)\dots\hat{\ad(x_i)}\dots\ad(x_n)
            + (-1)^n_{\pm}
            \ad(x_1)\dots\ad(x_n)a(x)
        \right)|0\rangle\\
        &=
        \int d^3x_1\dots d^3x_n
        d^3x\ad(x)\frac{-\hbar^2}{2m}\nabla^2\psi\\
        &\left( 
            \sum_{i=1}^n(-1)^{i-1}_{\pm}\delta(x-x_i)\ad(x_1)\dots\hat{\ad(x_i)}\dots\ad(x_n) 
        \right)|0\rangle\\
        &=
        \int d^3x_1\dots d^3x_n
        \sum_{i=1}^n\ad(x_i)\frac{-\hbar^2}{2m}\nabla_i^2\psi\\
        &\left( 
            (-1)^{i-1}_{\pm}\ad(x_1)\dots\hat{\ad(x_i)}\dots\ad(x_n)
        \right)|0\rangle\\
        &=
        \int d^3x_1\dots d^3x_n
        \sum_{i=1}^n\frac{-\hbar^2}{2m}\nabla_i^2\psi
        \left( 
            (-1)^{i-1}_{\pm}\ad(x_i)\ad(x_1)\dots\hat{\ad(x_i)}\dots\ad(x_n)
        \right)|0\rangle\\
        &=
        \int d^3x_1\dots d^3x_n
        \sum_{i=1}^n\frac{-\hbar^2}{2m}\nabla_i^2\psi
        \left( 
            \ad(x_1)\dots\ad(x_i)\dots\ad(x_n)
        \right)|0\rangle\\
    \end{aligned}
\]
Note that we integrated over $x$, and the $\delta$ factors in the sum mean that
integrating over $x$ just swaps $x$ with $x_i$. We also used integration by
parts implicitly to go from line 4 and 5 to line 6 and 7 by shifting over the
$\nabla^2$ onto $\ad(x)$ before integrating out the $x$, changing it to a
$\nabla_i^2\ad(x_i)$, then integrating by parts again to move it back to $\psi$.
Finally, in the fermionic calculation, we note that the factor of
$(-1)^{i-1}$ disappears when we move $\ad(x_i)$ across to the $i$th position, as
it will pick up an extra factor of $(-1)^{i-1}$.

We carry the exact same calculations for the second integral, without the
integration by parts, to find
\[
    \begin{aligned}
        \int d^3x \ad(x)U(x)a(x)|\psi,t\rangle
        &= \int d^3x_1\dots d^3x_n \sum_{i=1}^nU(x_i)\psi \ad(x_1)\dots\ad(x_n)
    \end{aligned}
\]

And finally, we calculate the third integral as 
\[
\begin{aligned}
        \int d^3xd^3yV(x-y)\ad(x)\ad(y)a(y)a(x)|\psi,t\rangle
        &= \int d^3x_1\dots d^3x_n d^3x d^3y \psi V(x-y)\ad(x)\ad(y)a(y)\\
        &\left( 
            \sum_{i=1}^n(-1)^{i-1}_{\pm}\delta(x-x_i)\ad(x_1)\dots\hat{\ad(x_i)}\dots\ad(x_n)
        \right)|0\rangle\\
        &= \int d^3x_1\dots d^3x_nd^3y \sum_{i=1}^n\psi V(x_i-y)\ad(x_i)\ad(y)a(y)\\
        &\left( 
            (-1)^{i-1}_{\pm}\ad(x_1)\dots\hat{\ad(x_i)}\dots\ad(x_n)
        \right)|0\rangle\\
        &= \int d^3x_1\dots d^3x_nd^3y \sum_{i=1}^n\psi V(x_i-y)(-1)_{\pm}\ad(y)\ad(x_i)a(y)\\
        &\left( 
            (-1)^{i-1}_{\pm}\ad(x_1)\dots\hat{\ad(x_i)}\dots\ad(x_n)
        \right)|0\rangle\\
        &= \int d^3x_1\dots d^3x_nd^3y \sum_{i=1}^n\psi V(x_i-y)\ad(y)(a(y)\ad(x_i) \pm \delta(x_i-y)\\
        &\left( 
            (-1)^{i-1}_{\pm}\ad(x_1)\dots\hat{\ad(x_i)}\dots\ad(x_n)
        \right)|0\rangle\\
\end{aligned}
\]

The first calculation was done identical to the other two integrals, and the
second calculation was done by direct evaluation of the commutators. Now, if we
integrate out $y$ in the part of the integral with $\delta(x_i-y)$, we get a
factor of $V(0) = 0$, and so that integral goes to zero. Thus, we have
\[
    \begin{aligned}
        &= \int d^3x_1\dots d^3x_nd^3y \sum_{i=1}^n\psi V(x_i-y)\ad(y)a(y)\ad(x_i)\\
        &\left( 
            (-1)^{i-1}_{\pm}\ad(x_1)\dots\hat{\ad(x_i)}\dots\ad(x_n)
        \right)|0\rangle\\
        &= \int d^3x_1\dots d^3x_nd^3y \sum_{i=1}^n\psi V(x_i-y)\ad(y)a(y)\\
        &\left( 
            \ad(x_1)\dots\ad(x_i)\dots\ad(x_n)
        \right)|0\rangle\\
    \end{aligned}
\]
We then carry the exact same calculation out for $y$, integrating out the
factors of $\delta$ to get
\[
    \begin{aligned}
        &= \int d^3x_1\dots d^3x_n\sum_{i=1}^n\sum_{j=1}^n\psi V(x_i-x_j)\\
        &\left( 
            \ad(x_1)\dots\ad(x_n)
        \right)|0\rangle\\
    \end{aligned}
\]
as desired. Thus, since
\[
    i\partial_t|\psi,t\rangle = \int d^3x_1\dots d^3x_n i\partial_t\psi
    \ad(x_1)\dots\ad(x_n)
\]
we equate both sides to get
\[
    \begin{aligned}
        &\int d^3x_1\dots d^3x_ni\partial_t\psi \ad(x_1)\dots\ad(x_n)\\
        &= \int d^3x_1\dots d^3x_n
        \left(
 \sum_{i=1}^n \frac{-\hbar^2}{2m}\nabla^2_i + U(x_i) +
    \sum_{j=1}^n\sum_{i=1}^{j-1}V(x_i-x_j)
\right)\psi \ad(x_1)\dots\ad(x_n)
    \end{aligned}
\]
Equating integrands yields the desired result. Thus, the abstract Schrodinger
equation is solved if and only if $\psi$ solves the regular Schrodinger
equation.

\newpage
\section*{Problem 2}
Show that the infinitesimal Lorentz transformations are antisymmetric.

We expand directly using $\Lambda^{\mu}_{\nu} = \delta^{\mu}_{\nu} + \delta
\omega^{\mu}_{\nu}$ and omitting products of infinitesimals as
\[
    \begin{aligned}
        g_{\mu\nu}\Lambda^{\mu}_{\rho}\Lambda^{\nu}_{\sigma} = g_{\rho\sigma}\\
        g_{\mu\nu}\left( 
            \delta^{\mu}_{\rho}\delta^{\nu}_{\sigma} +
            \delta^{\mu}_{\rho}\delta\omega^{\nu}_{\sigma} +
            \delta^{\nu}_{\sigma}\delta\omega^{\mu}_{\rho}
        \right) = g_{\rho\sigma}\\
        g_{\rho\sigma} + g_{\rho\nu}\delta\omega^{\nu}_{\sigma} +
        g_{\mu\sigma}\delta\omega^{\mu}_{\rho} = g_{\rho\sigma}\\
        g_{\rho\sigma} + \delta\omega_{\rho\sigma} + \delta\omega_{\sigma\rho} =
        g_{\rho\sigma}
    \end{aligned}
\]
and so it must be that $\delta\omega_{\rho\sigma} + \delta\omega_{\sigma\rho} =
0$ as desired.


\newpage
\section*{Problem 3}
Derive the commutation relations on momentum creation and annihilation operators
from the canonical commutation relations.

We know
\[
    a(k) = \int d^3x\exp(-ikx)\left( i\partial_t\phi(x) + \omega\phi(x) \right)
\]
and
\[
    \Pi(x) = \partial_t\phi(x)
\]
along with the canonical commutation relations on $\phi$ and $\Pi$.

So, we calculate directly
\[
    \begin{aligned}
        a(k)a(k') &= \int d^3x\exp(-ikx)\left( i\partial_t\phi(x) + \omega\phi(x) \right)
        \int d^3x\exp(-ik'x)\left( i\partial_t\phi(x) + \omega\phi(x) \right)\\
        &=\int d^3x d^3x' \exp(-ikx)(i\Pi(x) +
        \omega\phi(x))\exp(-ik'x')(i\Pi(x') + \omega\phi(x'))\\
        &=\int d^3x d^3x'\exp(-ikx)\exp(-ik'x')(i\Pi(x)i\Pi(x') +
            i\Pi(x)\omega'\phi(x') + \omega\phi(x)i\Pi(x') +
        \omega\omega'\phi(x)\phi(x'))\\
        &=\int d^3x d^3x'\exp(-ikx)\exp(-ik'x')(i\Pi(x')i\Pi(x) +
            i\Pi(x')\omega\phi(x) +i\omega\delta(x-x')\\
            &+ \omega'\phi(x')i\Pi(x) +
            i\omega'\delta(x-x') +
        \omega\omega'\phi(x')\phi(x))\\
        &=a(k')a(k) + \int d^3x d^3x'\exp(-ikx)\exp(-ik'x')(i(\omega+\omega')\delta(x-x'))\\
        &= a(k')a(k) + \int d^3x \exp(-ikx)\exp(-ik'x)i(\omega+\omega')\\
        &= a(k')a(k)
    \end{aligned}
\]
where the last integral disappeared because $\exp(i(k+k')x)$ disappears on a
symmetric domain. The exact same argument is made for $\ad(k)\ad(k')$ just by
complex conjugation.

Thus, all we have to show is the final commutator. We calculate
\[
\begin{aligned}
    a(k)\ad(k') &= \int d^3x d^3x' \exp(-ikx)\exp(ik'x')(i\Pi(x) +
    \omega\phi(x))(-i\Pi(x') + \omega'\phi(x'))\\
    &=\int d^3x d^3x' \exp(-ikx)\exp(ik'x')(i\Pi(x)(-i\Pi(x')) \\
        &+ i\Pi(x)\omega'\phi(x') + \omega\phi(x)(-i\Pi(x')) +
            \omega\phi(x)\omega'\phi(x'))\\
    &=\int d^3x d^3x' \exp(-ikx)\exp(ik'x')(-i\Pi(x')i\Pi(x) \\
        &+ \omega'\phi(x')i\Pi(x) - i\omega'i\delta(x-x') +
        (-i\Pi(x'))\omega\phi(x) - i\omega i\delta(x-x') +
            \omega'\phi(x')\omega\phi(x))\\
    &=\int d^3x d^3x' \exp(-ikx)\exp(ik'x')(-i\Pi(x')i\Pi(x) \\
    &+ \omega'\phi(x')i\Pi(x) + \omega'\delta(x-x') +
    (-i\Pi(x'))\omega\phi(x) + \omega\delta(x-x') +
        \omega'\phi(x')\omega\phi(x))\\
        &=\ad(k')a(k) + \int d^3x
        d^3x'\exp(-ikx)\exp(ik'x')(\omega+\omega')\delta(x-x')\\
        &=\ad(k')a(k) + \int d^3x \exp(i(k'-k)x)(\omega'+\omega)\\
        &=\ad(k')a(k) + (2\pi)^32\omega\delta(k-k')
\end{aligned}
\]
as desired.


\newpage
\section*{Problem 4}
\subsection*{Part A}
Find an expression for $[\phi(x),P^i]$.

We carry out two expansions here on $a$. First:
\[
    \phi(x-a) = \phi(x) - \partial_{\mu}\phi(x)a^{\mu} + O(a^2)
\]
and
\[
    T(a) = I + iP^{\mu}a_{\mu}
\]

From the expression for the $T(a)$ action on $\phi(x)$ we know that (by killing
$O(a^2)$ terms)
\[
    \begin{aligned}
        \phi(x)T(a) &= T(a)\phi(x-a)\\
        \phi(x)T(a) &= T(a)\left( \phi(x) -\partial_{\mu}\phi(x)a^{\mu}
        \right)\\
        \phi(x)(I + iP^{\mu}a_{\mu}) &= (I + iP^{\mu}a_{\mu})\phi(x) - (I +
        iP^{\mu}a_{\mu})\partial_{\nu}\phi(x)a^{\nu}\\
        \phi(x)iP^{\mu}a_{\mu} &=iP^{\mu}a_{\mu}\phi(x) -
        \partial^{\mu}\phi(x)a_{\mu}\\
        \phi(x)P^{\mu}a_{\mu} &= P^{\mu}a_{\mu}\phi(x) +
        i\partial^{\mu}\phi(x)a_{\mu}\\
    \end{aligned}
\]
and so
\[
    [\phi(x),P^{\mu}] = i\partial^{\mu}\phi(x)
\]
as desired.


\subsection*{Part B}
Show the time component of the calculation above is the Heisenberg equation of
motion.

This follows immediately, noting that $P^0 = H$, and so
\[
    [\phi(x),H] = i\partial^0\phi(x) = i\partial_t\phi(x)
\]
as desired.


\subsection*{Part C}
Derive the Klein-Gordon equation from the Heisenberg equation for a free
particle.

We'll use the Hamiltonian
\[
    H = \int \tilde{dk}\omega \ad(k)a(k)
\]
along with the free-field expression
\[
    \phi(x) = \int \tilde{dk}(a(k)\exp(ikx) + \ad(k)\exp(-ikx))
\]
as derived in the text. Thus,
\[
    \begin{aligned}
        i\dot{\phi}(x) &= [\phi(x),H]\\
        &= \phi(x)H - H\phi(x)\\
        &= \int\tilde{dk}a(k)\exp(ikx) +
        \ad(k)\exp(-ikx))\int\tilde{dk}\omega\ad(k)a(k)\\
        &- \int\tilde{dk}\omega\ad{k}a(k)\int\tilde{dk}(a(k)\exp(ikx) +
        \ad(k)\exp(-ikx))\\
        &=\int
        \tilde{dk}\tilde{dk'}\omega\exp(ikx)(a(k)\ad(k')a(k')-\ad(k')a(k')a(k))\\
            &+\omega\exp(-ikx)(\ad(k)\ad(k')a(k') - \ad(k')a(k')\ad(k))\\
        &=\int
        \tilde{dk}\tilde{dk'}\omega\exp(ikx)(a(k)\ad(k')a(k')-\ad(k')a(k)a(k'))\\
            &+\omega\exp(-ikx)(\ad(k')\ad(k)a(k') - \ad(k')a(k')\ad(k))\\
        &=\int
        \tilde{dk}\tilde{dk'}\omega\exp(ikx)(a(k)\ad(k')-\ad(k')a(k))a(k')\\
        &+\omega\exp(-ikx)\ad(k')(\ad(k)a(k') - a(k')\ad(k))\\
        &=\int
        \tilde{dk}\tilde{dk'}\omega\exp(ikx)([a(k),\ad(k')])a(k')\\
        &+\omega\exp(-ikx)\ad(k')([\ad(k),a(k')])\\
        &=\int
        \tilde{dk}\tilde{dk'}\omega\exp(ikx)((2\pi)^32\omega\delta(k-k'))a(k')\\
        &+\omega\exp(-ikx)\ad(k')(-(2\pi)^3)2\omega\delta(k-k')\\
        &=\int
        \tilde{dk}\omega\exp(ikx)((2\pi)^32\omega)a(k)\\
        &-\omega\exp(-ikx)\ad(k)((2\pi)^3)2\omega\\
        &=(2\pi)^32\omega^2\left( \int \tilde{dk}(a(k)\exp(ikx) -
            \ad(k)\exp(-ikx) \right)
    \end{aligned}
\]
and so
\[
    \dot{\phi}(x) = (2\pi)^32\omega^2\left( \int \tilde{dk}(-ia(k)\exp(ikx) +
        i\ad(k)\exp(-ikx) \right)
\]
we can carry the exact same calculations out to find that
\[
    \begin{aligned}
        i\ddot{\phi}(x) &= (2\pi)^32\omega^2(
            \int\tilde{dk}\tilde{dk'}\omega\exp(ikx)([-ia(k),\ad(k')])a(k')\\
        &+\omega\exp(-ikx)\ad(k')([i\ad(k),a(k')]))\\
        &=
        (2\pi)^32\omega^2(
            \int\tilde{dk}\tilde{dk'}\omega\exp(ikx)(-i(2\pi)^32\omega\delta(k-k'))a(k')\\
        &+\omega\exp(-ikx)\ad(k')(-i(2\pi)^32\omega\delta(k-k')))\\
        &=
        (2\pi)^32\omega^2(
            \int\tilde{dk}\omega\exp(ikx)(-i(2\pi)^32\omega)a(k)\\
        &+\omega\exp(-ikx)\ad(k)(-i(2\pi)^32\omega))\\
        &=-i(2\pi)^64\omega^4\phi(x)
    \end{aligned}
\]
and so
\[
    \begin{aligned}
        \ddot{\phi} &= -(2\pi)^64\omega^4\phi(x)\\
        &= -(2\pi)^64(k^2 + m^2)^2\phi(x)\\
        &= -((2\pi)^3)^24(k^4 + m^4 + 2k^2m^2)\phi(x)\\
    \end{aligned}
\]
which (I think) reduces to the Klein-Gordon equation.


\subsection*{Part D}
We compute the commutator directly:
\[
    \begin{aligned}
        \phi(x)\int d^3x' \Pi(x')\nabla\phi(x')
        &= \int d^3x' \phi(x)\Pi(x')\nabla\phi(x')\\
        &= \int d^3x' (\Pi(x')\phi(x) + i\delta(x-x'))\nabla\phi(x')\\
        &= \int d^3x' (\Pi(x')\phi(x)\nabla\phi(x') +
            i\delta(x-x')\nabla\phi(x'))\\
            &= \int d^3x'\Pi(x')\nabla\phi(x')\phi(x) + i\nabla\phi(x)\\
            &=P\phi(x) + i\nabla\phi(x)
    \end{aligned}
\]
and so
\[
    [\phi(x),P^i] = i\partial^i\phi(x)
\]
as desired.

\subsection*{Part E}
Express $P$ in terms of momentum state creation and annihilation operators.

We do this directly.
\[
    \begin{aligned}
        P &= \int d^3x\Pi(x)\nabla\phi(x)\\
        &= \int \tilde{dk}\tilde{dk'}d^3x(-i\omega a(k)\exp(ikx) +
        i\omega\ad(k)\exp(-ikx))(ik' a(k')\exp(ik'x) -ik' \ad(k')\exp(-ik'x))\\
        &= \int \tilde{dk}\tilde{dk'}d^3x(\omega k a(k)a(k')\exp(i(k+k')x) -
            \omega k a(k)\ad(k')\exp(i(k-k')x)\\ 
            &-\omega k
        \ad(k)a(k')\exp(i(k'-k)x) + \omega k \ad(k)\ad(k')\exp(-i(k+k')x))\\
        &= \int \tilde{dk}\tilde{dk'}(\omega k a(k)a(k')(2\pi)^3\delta(k+k') -
            \omega k a(k)\ad(k')(2\pi)^3\delta(k-k')\\ 
            &-\omega k
        \ad(k)a(k')(2\pi)^3\delta(k-k') + \omega k
    \ad(k)\ad(k')(2\pi)^3\delta(k+k'))\\
        &= \int \tilde{dk}(\omega k a(k)a(-k)(2\pi)^3 -
            \omega k a(k)\ad(k)(2\pi)^3\\ 
            &-\omega k
        \ad(k)a(k)(2\pi)^3+ \omega k
    \ad(k)\ad(-k)(2\pi)^3)\\
        &= \int \tilde{dk}(-
            \omega k a(k)\ad(k)(2\pi)^3\\ 
            &-\omega k
        \ad(k)a(k)(2\pi)^3)\\
        &=\int \tilde{dk}\omega k(2\pi)^3 (-a(k)\ad(k) + \ad(k)a(k)
    \end{aligned}
\]


\end{document}
