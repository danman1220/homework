%%%%%%%%%%%%%%%%%%%%%%%%%%%%%%%%%%%%%%%%%
% Short Sectioned Assignment
% LaTeX Template
% Version 1.0 (5/5/12)
%
% This template has been downloaded from:
% http://www.LaTeXTemplates.com
%
% Original author:
% Frits Wenneker (http://www.howtotex.com)
%
% License:
% CC BY-NC-SA 3.0 (http://creativecommons.org/licenses/by-nc-sa/3.0/)
%
%%%%%%%%%%%%%%%%%%%%%%%%%%%%%%%%%%%%%%%%%

%----------------------------------------------------------------------------------------
%	PACKAGES AND OTHER DOCUMENT CONFIGURATIONS
%----------------------------------------------------------------------------------------

\documentclass[fontsize=11pt]{scrartcl} % 11pt font size

\usepackage[T1]{fontenc} % Use 8-bit encoding that has 256 glyphs
\usepackage[english]{babel} % English language/hyphenation
\usepackage{amsmath,amsfonts,amsthm} % Math packages
\usepackage{mathrsfs}

\usepackage[margin=1in]{geometry}

\usepackage{sectsty} % Allows customizing section commands
\allsectionsfont{\centering \normalfont\scshape} % Make all sections centered, the default font and small caps

\usepackage{fancyhdr} % Custom headers and footers
\pagestyle{fancyplain} % Makes all pages in the document conform to the custom headers and footers
\fancyhead{} % No page header - if you want one, create it in the same way as the footers below
\fancyfoot[L]{} % Empty left footer
\fancyfoot[C]{} % Empty center footer
\fancyfoot[R]{\thepage} % Page numbering for right footer
\renewcommand{\headrulewidth}{0pt} % Remove header underlines
\renewcommand{\footrulewidth}{0pt} % Remove footer underlines
\setlength{\headheight}{13.6pt} % Customize the height of the header

\numberwithin{equation}{section} % Number equations within sections (i.e. 1.1, 1.2, 2.1, 2.2 instead of 1, 2, 3, 4)
\numberwithin{figure}{section} % Number figures within sections (i.e. 1.1, 1.2, 2.1, 2.2 instead of 1, 2, 3, 4)
\numberwithin{table}{section} % Number tables within sections (i.e. 1.1, 1.2, 2.1, 2.2 instead of 1, 2, 3, 4)

\newcommand{\R}{\mathbb{R}}
\newcommand{\Q}{\mathbb{Q}}
\newcommand{\N}{\mathbb{N}}
\newcommand{\C}{\mathbb{C}}
\newcommand{\Z}{\mathbb{Z}}

\newcommand{\ad}{a^{\dagger}}

\newtheorem{lemma}{Lemma}
%----------------------------------------------------------------------------------------
%	TITLE SECTION
%----------------------------------------------------------------------------------------

\newcommand{\horrule}[1]{\rule{\linewidth}{#1}} % Create horizontal rule command with 1 argument of height

\title{	
\normalfont \normalsize 
\textsc{QFT} \\ [25pt] % Your university, school and/or department name(s)
\horrule{0.5pt} \\[0.4cm] % Thin top horizontal rule
\huge Homework 1 \\ % The assignment title
\horrule{2pt} \\[0.5cm] % Thick bottom horizontal rule
}

\author{Daniel Halmrast} % Your name

\date{\normalsize\today} % Today's date or a custom date

\begin{document}

\maketitle % Print the title

% Problems
\section*{Problem 1}
Show that multiparticle nonrelativistic quantum can be recovered from QFT.
Namely, define
\[
    H = \int d^3x\ad(x)\left( \frac{-\hbar^2}{2m}\nabla^2 + U(x)
    \right)a(x) + \int d^3xd^3yV(x-y)\ad(x)\ad(y)a(y)a(x)
\]
and
\[
    |\psi,t\rangle = \int d^3x_1\dots d^3x_n
    \psi(x_1,\dots,x_n;t)\ad(x_1)\dots\ad(x_n)|0\rangle
\]
We will show that $|\psi,t\rangle$ satisfies the abstract Schrodinger equation
if and only if $\psi$ satisfies the Schrodinger equation
\[
    i\partial_t\psi = H\psi
\]
for
\[
    H = \sum_{i=1}^n \frac{-\hbar^2}{2m}\nabla^2_i + U(x_i) +
    \sum_{j=1}^n\sum_{i=1}^{j-1}V(x_i-x_j)
\]

We calculate $H|\psi,t\rangle$ directly in three parts. That is:
\[
    \begin{aligned}
    H|\psi,t\rangle &= 
    \int d^3\ad(x)\frac{-\hbar^2}{2m}\nabla^2a(x)|\psi,t\rangle\\
        &+ \int d^3x \ad(x)U(x)a(x)|\psi,t\rangle\\
        &+ \int d^3xd^3yV(x-y)\ad(x)\ad(y)a(y)a(x)|\psi,t\rangle\\
    \end{aligned}
\]
For this calculation, we'll use the fact that
\[
    a(x)\ad(x_1)\dots\ad(x_n) =
    \sum_{i=1}^n\delta(x-x_i)\ad(x_1)\dots\hat{\ad(x_i)}\dots\ad(x_n) +
    \ad(x_1)\dots\ad(x_n)a(x)
\]
where $\hat{\ad(x_i)}$ indicates that the $i$th creation operator is omitted.

\end{document}
