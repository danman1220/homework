%%%%%%%%%%%%%%%%%%%%%%%%%%%%%%%%%%%%%%%%%
% Short Sectioned Assignment
% LaTeX Template
% Version 1.0 (5/5/12)
%
% This template has been downloaded from:
% http://www.LaTeXTemplates.com
%
% Original author:
% Frits Wenneker (http://www.howtotex.com)
%
% License:
% CC BY-NC-SA 3.0 (http://creativecommons.org/licenses/by-nc-sa/3.0/)
%
%%%%%%%%%%%%%%%%%%%%%%%%%%%%%%%%%%%%%%%%%

%----------------------------------------------------------------------------------------
%	PACKAGES AND OTHER DOCUMENT CONFIGURATIONS
%----------------------------------------------------------------------------------------

\documentclass[fontsize=11pt]{scrartcl} % 11pt font size

\usepackage[T1]{fontenc} % Use 8-bit encoding that has 256 glyphs
\usepackage[english]{babel} % English language/hyphenation
\usepackage{amsmath,amsfonts,amsthm} % Math packages
\usepackage{mathrsfs}

\usepackage[margin=1in]{geometry}

\usepackage{sectsty} % Allows customizing section commands
\allsectionsfont{\centering \normalfont\scshape} % Make all sections centered, the default font and small caps

\usepackage{fancyhdr} % Custom headers and footers
\pagestyle{fancyplain} % Makes all pages in the document conform to the custom headers and footers
\fancyhead{} % No page header - if you want one, create it in the same way as the footers below
\fancyfoot[L]{} % Empty left footer
\fancyfoot[C]{} % Empty center footer
\fancyfoot[R]{\thepage} % Page numbering for right footer
\renewcommand{\headrulewidth}{0pt} % Remove header underlines
\renewcommand{\footrulewidth}{0pt} % Remove footer underlines
\setlength{\headheight}{13.6pt} % Customize the height of the header

\numberwithin{equation}{section} % Number equations within sections (i.e. 1.1, 1.2, 2.1, 2.2 instead of 1, 2, 3, 4)
\numberwithin{figure}{section} % Number figures within sections (i.e. 1.1, 1.2, 2.1, 2.2 instead of 1, 2, 3, 4)
\numberwithin{table}{section} % Number tables within sections (i.e. 1.1, 1.2, 2.1, 2.2 instead of 1, 2, 3, 4)

\newcommand{\R}{\mathbb{R}}
\newcommand{\Q}{\mathbb{Q}}
\newcommand{\N}{\mathbb{N}}
\newcommand{\C}{\mathbb{C}}
\newcommand{\Z}{\mathbb{Z}}

\newtheorem{lemma}{Lemma}
\newtheorem*{problem}{Problem}
%----------------------------------------------------------------------------------------
%	TITLE SECTION
%----------------------------------------------------------------------------------------

\newcommand{\horrule}[1]{\rule{\linewidth}{#1}} % Create horizontal rule command with 1 argument of height

\title{	
\normalfont \normalsize 
\textsc{QFT} \\ [25pt] % Your university, school and/or department name(s)
\horrule{0.5pt} \\[0.4cm] % Thin top horizontal rule
\huge Homework 2 \\ % The assignment title
\horrule{2pt} \\[0.5cm] % Thick bottom horizontal rule
}

\author{Daniel Halmrast} % Your name

\date{\normalsize\today} % Today's date or a custom date

\begin{document}

\maketitle % Print the title

% Problems
\section*{Problem 1}
Recall the $\varphi^4$ Lagrangian is given by
\[
    \mathscr{L} = -\frac{1}{2}\partial^{\mu}\varphi\partial_{\mu}\varphi -
    \frac{1}{2}m^2\varphi^2 - \frac{1}{3!}g\varphi^3 -
    \frac{1}{4!}\lambda\varphi^4
\]
and has an energy-momentum tensor
\[
    T^{\mu\nu} = \partial^{\mu}\varphi\partial^{\nu}\varphi +
    g^{\mu\nu}\mathscr{L}
\]
\subsection*{Part A}
\begin{problem}
    Derive the equation of motion for $\varphi$ subject to the $\varphi^4$
    Lagrangian.
\end{problem}

To calculate the equation of motion for $\varphi$, we just have to find the
stationary points of
\[
    S = \int d^4x\mathscr{L} = \int d^4x
    \left(\frac{-1}{2}\partial^{\mu}\varphi\partial_{\mu}\varphi -
    \frac{1}{2}m^2\varphi^2 - \frac{1}{3!}g\varphi^3 -
\frac{1}{4!}\lambda\varphi^4\right)
\]

That is, we find when $\delta S=0$. To do so, we calculate
\[
    \begin{aligned}
        \delta S &= \int d^4x \delta \mathscr{L}\\
        &=\int d^4x 
        \left(  
            \frac{-1}{2}\delta(\partial^{\mu}\varphi\partial_{\mu}\varphi) -
            \frac{1}{2}m^2\delta(\varphi^2) - \frac{1}{3!}g\delta(\varphi^3) -
            \frac{1}{4!}\lambda\delta(\varphi^4)
        \right)\\
        &=\int d^4x 
        \left(  
            \frac{-1}{2}(\partial^{\mu}\delta\varphi\partial_{\mu}\varphi +
            \partial^{\mu}\varphi\partial_{\mu}\delta\varphi) -
            m^2\varphi\delta\varphi - \frac{1}{2}g\varphi^3\delta\varphi -
            \frac{1}{3!}\lambda\varphi^3\delta\varphi
        \right)\\
        &=\int d^4x 
        \left(  
            \partial^2\varphi\delta\varphi -
            m^2\varphi\delta\varphi - \frac{1}{2}g\varphi^3\delta\varphi -
            \frac{1}{3!}\lambda\varphi^3\delta\varphi
        \right)\\
        &=\int d^4x 
        \left(  
            \partial^2\varphi -
            m^2\varphi - \frac{1}{2}g\varphi^3 -
            \frac{1}{3!}\lambda\varphi^3
        \right)\delta\varphi\\
    \end{aligned}
\]
Which is zero for arbitrary variation if $
\left(  \partial^2\varphi - m^2\varphi - \frac{1}{2}g\varphi^3 - 
\frac{1}{3!}\lambda\varphi^3 \right)=0$. Thus, this is the equation of motion
for $\varphi$.

\newpage

\subsection*{Part B}

\begin{problem}
    Show that the energy-momentum tensor $T^{\mu\nu}$ satisfies
    $\partial_{\mu}T^{\mu\nu}=0$.
\end{problem}

This is just an exercise in direct calculation:
\[
    \begin{aligned}
        \partial_{\mu}T^{\mu\nu} &=
        \partial_{\mu}\left( \partial^{\mu}\varphi\partial^{\nu}\varphi \right)
        +\partial_{\mu}g^{\mu\nu}\mathscr{L}\\
        &=
        \partial_{\mu}\partial^{\mu}\varphi\partial^{\nu}\varphi
        +\partial^{\mu}\varphi\partial_{\mu}\partial^{\nu}\varphi
        +\partial^{\nu}
        \left(\frac{-1}{2}\partial^{\mu}\varphi\partial_{\mu}\varphi -
        \frac{1}{2}m^2\varphi^2 - \frac{1}{3!}g\varphi^3 -
        \frac{1}{4!}\lambda\varphi^4\right)\\
        &=
        \partial_{\mu}\partial^{\mu}\varphi\partial^{\nu}\varphi
        +\partial^{\mu}\varphi\partial_{\mu}\partial^{\nu}\varphi
        -\frac{1}{2}\partial^{\nu}\partial^{\mu}\varphi\partial_{\mu}\varphi 
        -\frac{1}{2}\partial^{\mu}\varphi\partial^{\nu}\partial_{\mu}\varphi
        -m^2\varphi\partial^{\nu}\varphi
        -\frac{1}{2}g\varphi^2\partial^{\nu}\varphi
        -\frac{1}{3!}\lambda\varphi^3\partial^{\nu}\varphi\\
        &=
        \partial_{\mu}\partial^{\mu}\varphi\partial^{\nu}\varphi
        -m^2\varphi\partial^{\nu}\varphi
        -\frac{1}{2}g\varphi^2\partial^{\nu}\varphi
        -\frac{1}{3!}\lambda\varphi^3\partial^{\nu}\varphi\\
        &=
        \left(
            \partial^2\varphi
            -m^2\varphi
            -\frac{1}{2}g\varphi^2
            -\frac{1}{3!}\lambda\varphi^3
        \right)\partial^{\nu}\varphi
    \end{aligned}
\]
Which is clearly zero if $\varphi$ follows its equation of motion.


\newpage
\section*{Problem 2}
Consider a complex scalar field $\varphi$ governed by the Lagrangian
\[
    \mathscr{L} = -\partial^{\mu}\varphi^{\dagger}\partial_{\mu}\varphi -
    m^2\varphi^{\dagger}\varphi + \Omega_0
\]
\subsection*{Part A}
\begin{problem}
    Show $\varphi$ obeys the Klein-Gordon equation.
\end{problem}
We calculate the variation in $S = \int d^4x\mathscr{L}$ directly:
\[
\begin{aligned}
    \delta S &= \int d^4x \delta\mathscr{L}\\
    &=\int d^4x \left(
    -\delta(\partial^{\mu}\varphi^{\dagger})\partial_{\mu}\varphi
    -\partial^{\mu}\phi^{\dagger}\delta(\partial_{\mu}\varphi)
    -m^2\varphi\delta\varphi^{\dagger}
    -m^2\varphi^{\dagger}\delta\varphi\right)\\
    &=\int d^4x \left(
    -\partial^{\mu}\delta\varphi^{\dagger}\partial_{\mu}\varphi
    -\partial^{\mu}\phi^{\dagger}\partial_{\mu}\delta\varphi
    -m^2\varphi\delta\varphi^{\dagger}
    -m^2\varphi^{\dagger}\delta\varphi\right)\\
    &=\int d^4x \left(
    \delta\varphi^{\dagger}\partial_{2}\varphi
    +\partial^{2}\phi^{\dagger}\delta\varphi
    -m^2\varphi\delta\varphi^{\dagger}
    -m^2\varphi^{\dagger}\delta\varphi\right)\\
    &=\int d^4x \left(
        \left( \partial^2\varphi-m^2\varphi \right)\delta\varphi^{\dagger}
        +\left( \partial^2\varphi^{\dagger}-m^2\varphi^{\dagger}
        \right)\delta\varphi
    \right)\\
\end{aligned}
\]
Which is zero for arbitrary variations when both $\varphi$ and
$\varphi^{\dagger}$ follow the Klein-Gordon equation.


\newpage

\subsection*{Part B}
\begin{problem}
    Find the conjugate momenta for $\varphi$ and $\varphi^{\dagger}$, and write down
    the Hamiltonian in terms of these.
\end{problem}

We can read off the conjugate momenta easily:
\[
    \begin{aligned}
        \pi(x) &= \frac{\partial\mathscr{L}}{\partial\dot{\varphi}}\\
        &=
        \frac{\partial}{\partial\dot{\varphi}}
        (\partial_{\mu}\varphi^{\dagger}(x)\partial_{\mu}\varphi(x) -
        m^2\varphi^{\dagger}(x)\varphi(x) + \Omega_0)\\
        &= \dot{\varphi^{\dagger}}(x)
    \end{aligned}
\]
and similarly
\[
    \pi^{\dagger}(x) = \dot{\varphi}(x)
\]
We write down the Hamiltonian density as
\[
    \mathscr{H} = \pi(x)\dot{\varphi}(x) +
    \pi^{\dagger}(x)\dot{\varphi^{\dagger}}(x) - \mathscr{L}
\]
and calculate
\[
    \begin{aligned}
        \mathscr{H} &= \pi(x)\dot{\varphi}(x)
        + \pi^{\dagger}(x)\dot{\varphi^{\dagger}}(x) 
        + \left( \partial^{\mu}\varphi^{\dagger}(x)\partial_{\mu}\varphi(x)
        + m^2 \varphi^{\dagger}(x)\varphi(x) - \Omega_0\right)\\
        &=\pi(x)\pi^{\dagger}(x) + \pi^{\dagger}(x)\pi(x)
        +\partial^{0}\varphi^{\dagger}(x)\partial_{0}\varphi(x)
        +\partial^{i}\varphi^{\dagger}(x)\partial_{i}\varphi(x)
        +m^2\varphi^{\dagger}(x)\varphi(x) - \Omega_0\\
        &=\pi(x)\pi^{\dagger}(x) + \pi^{\dagger}(x)\pi(x)
        -\pi(x)\pi^{\dagger}(x)
        +\partial^{i}\varphi^{\dagger}(x)\partial_{i}\varphi(x)
        +m^2\varphi^{\dagger}(x)\varphi(x) - \Omega_0\\
        &=\pi^{\dagger}(x)\pi(x)
        +\partial^{i}\varphi^{\dagger}(x)\partial_{i}\varphi(x)
        +m^2\varphi^{\dagger}(x)\varphi(x) - \Omega_0\\
    \end{aligned}
\]
Which gives the Hamiltonian for the system as
\[
    H = \int d^4x \left( \pi^{\dagger}(x)\pi(x)
        +\partial^{i}\varphi^{\dagger}(x)\partial_{i}\varphi(x)
        +m^2\varphi^{\dagger}(x)\varphi(x) - \Omega_0 \right)
\]

\subsection*{Part C}
\begin{problem}
    Expanding $\varphi$ as
    \[
        \varphi(x) = \int \tilde{d^3k}\left( a(k)\exp(ikx) +
        b^{\dagger}(k)\exp(-ikx) \right)
    \]
    solve for expressions of $a(k)$ and $b(k)$ in terms of $\varphi(x)$,
    $\varphi^{\dagger}(x)$ and their time derivatives.
\end{problem}

We'll evaluate the integrals
\[
    \begin{aligned}
        \int d^3x\exp(-ikx)\varphi(x)\\
        \int d^3x\exp(-ikx)\partial_t\varphi(x)\\
        \int d^3x\exp(-ikx)\varphi^{\dagger}(x)\\
        \int d^3x\exp(-ikx)\partial_t\varphi^{\dagger}(x)\\
    \end{aligned}
\]

So we first derive an expression for $\partial_t\varphi(x)$ and its conjugate.
\[
\begin{aligned}
    \partial_t\varphi(x) &= \int \tilde{d^3k}\partial_t\left( a(k)\exp(ikx) +
    b^{\dagger}(k)\exp(-ikx) \right)\\
    &= \int \tilde{d^3k}\left( a(k)(-i\omega)\exp(ikx) -
    b^{\dagger}(k)(-i\omega)\exp(-ikx) \right)\\
    &=\int \tilde{d^3k}(-i\omega)\left( a(k)\exp(ikx) - b^{\dagger}(k)\exp(-ikx) \right)
\end{aligned}
\]

And similarly,
\[
    \partial_t\varphi^{\dagger}(x) = \int \tilde{d^3k}(-i\omega)\left(
        b(k)\exp(ikx) - a^{\dagger}(k)\exp(-ikx)
    \right)
\]

Now, we can calculate those four integrals. One will be done explicitly, and the
other three are done using the exact same calculation.
\[
    \begin{aligned}
        \int d^3x\exp(-ikx)\varphi(x) 
        &=
        \int d^3x\frac{d^3k'}{(2\pi)^32\omega}\exp(-ikx)\left(
        a(k')\exp(ik'x) + b^{\dagger}(k')\exp(-ik'x)\right)\\
        &=
        \int d^3x\frac{d^3k'}{(2\pi)^32\omega}\left(
        a(k')\exp(i(k'-k)x) + b^{\dagger}(k')\exp(-i(k'+k)x)\right)\\
        &=
        \int d^3x\frac{d^3k'}{(2\pi)^32\omega}(
        a(k')\exp(i(k'-k)^ix_i + i(\omega'-\omega)t)\\
        &+ b^{\dagger}(k')\exp(-i(k'+k)^ix_i +
        -i(\omega'+\omega_t))\\
        &=
        \int \frac{d^3k'}{(2\pi)^32\omega}(
        a(k')(2\pi)^3\delta(k'-k)\exp(i(\omega'-\omega)t)\\
        &+ b^{\dagger}(k')(2\pi)^3\delta(k'+k)\exp(-i(\omega'+\omega)t))\\
        &=
        \frac{1}{2\omega}(
        a(k) + b^{\dagger}(-k)\exp(-i(2\omega)t))\\
    \end{aligned}
\]
Following the same tactic, we find
\[
    \begin{aligned}
        \int d^3x\exp(-ikx)\varphi(x) 
        &= \frac{1}{2\omega}(a(k) + b^{\dagger}(-k)\exp(-i(2\omega)t))\\
        \int d^3x\exp(-ikx)\partial_t\varphi(x)
        &= \frac{-i}{2}(a(k) - b^{\dagger}(-k)\exp(-i(2\omega)t))\\
        \int d^3x\exp(-ikx)\varphi^{\dagger}(x)
        &= \frac{1}{2\omega}(b(k) + a^{\dagger}(-k)\exp(-i(2\omega)t))\\
        \int d^3x\exp(-ikx)\partial_t\varphi^{\dagger}(x)
        &= \frac{-i}{2}(b(k) - a^{\dagger}(-k)\exp(-i(2\omega)t))\\
    \end{aligned}
\]

So,
\[
    a(k) = \int d^3x\exp(-ikx)\left( \omega\varphi(x) + i\partial_t\varphi(x)\right)
\]
and
\[
    b(k) =\int d^3x\exp(-ikx)\left( \omega\varphi^{\dagger}(x) +
    i\partial_t\varphi^{\dagger}(x)\right)
\]

\newpage
\subsection*{Part D}
\begin{problem}
    Derive the commutation relations for $a(k)$ and $b(k)$ and their conjugates.
\end{problem}

By conjugating, we find that
\[
    \begin{aligned}
        a^{\dagger}(k) &= \int d^3x\exp(ikx)\left( \omega\varphi^{\dagger}(x) -
        i\partial_t\varphi^{\dagger}(x)\right)\\
                b^{\dagger}(k) &= \int d^3x\exp(ikx)\left( \omega\varphi(x) -
                        i\partial_t\varphi(x)\right)
    \end{aligned}
\]

So now we can calculate the commutators directly. Let's first rewrite the
creation and annihilation operators in terms of $\varphi$ and $\pi$ instead:
\[
    \begin{aligned}
        a(k) &= \int d^3x\exp(-ikx)\left( \omega\varphi(x) 
        + i\pi^{\dagger}(x)\right)\\
        b(k) &= \int d^3x\exp(-ikx)\left( \omega\varphi^{\dagger}(x)
            + i\pi(x)\right)\\
        a^{\dagger}(k) &= \int d^3x\exp(ikx)\left( \omega\varphi^{\dagger}(x) -
            i\pi(x)\right)\\
        b^{\dagger}(k) &= \int d^3x\exp(ikx)\left( \omega\varphi(x) -
        i\pi^{\dagger}(x)\right)
    \end{aligned}
\]

\end{document}
