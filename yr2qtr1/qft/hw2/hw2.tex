%%%%%%%%%%%%%%%%%%%%%%%%%%%%%%%%%%%%%%%%%
% Short Sectioned Assignment
% LaTeX Template
% Version 1.0 (5/5/12)
%
% This template has been downloaded from:
% http://www.LaTeXTemplates.com
%
% Original author:
% Frits Wenneker (http://www.howtotex.com)
%
% License:
% CC BY-NC-SA 3.0 (http://creativecommons.org/licenses/by-nc-sa/3.0/)
%
%%%%%%%%%%%%%%%%%%%%%%%%%%%%%%%%%%%%%%%%%

%----------------------------------------------------------------------------------------
%	PACKAGES AND OTHER DOCUMENT CONFIGURATIONS
%----------------------------------------------------------------------------------------

\documentclass[fontsize=11pt]{scrartcl} % 11pt font size

\usepackage[T1]{fontenc} % Use 8-bit encoding that has 256 glyphs
\usepackage[english]{babel} % English language/hyphenation
\usepackage{amsmath,amsfonts,amsthm} % Math packages
\usepackage{mathrsfs}

\usepackage[margin=1in]{geometry}

\usepackage{sectsty} % Allows customizing section commands
\allsectionsfont{\centering \normalfont\scshape} % Make all sections centered, the default font and small caps

\usepackage{fancyhdr} % Custom headers and footers
\pagestyle{fancyplain} % Makes all pages in the document conform to the custom headers and footers
\fancyhead{} % No page header - if you want one, create it in the same way as the footers below
\fancyfoot[L]{} % Empty left footer
\fancyfoot[C]{} % Empty center footer
\fancyfoot[R]{\thepage} % Page numbering for right footer
\renewcommand{\headrulewidth}{0pt} % Remove header underlines
\renewcommand{\footrulewidth}{0pt} % Remove footer underlines
\setlength{\headheight}{13.6pt} % Customize the height of the header

\numberwithin{equation}{section} % Number equations within sections (i.e. 1.1, 1.2, 2.1, 2.2 instead of 1, 2, 3, 4)
\numberwithin{figure}{section} % Number figures within sections (i.e. 1.1, 1.2, 2.1, 2.2 instead of 1, 2, 3, 4)
\numberwithin{table}{section} % Number tables within sections (i.e. 1.1, 1.2, 2.1, 2.2 instead of 1, 2, 3, 4)

\newcommand{\R}{\mathbb{R}}
\newcommand{\Q}{\mathbb{Q}}
\newcommand{\N}{\mathbb{N}}
\newcommand{\C}{\mathbb{C}}
\newcommand{\Z}{\mathbb{Z}}

\newtheorem{lemma}{Lemma}
\newtheorem*{problem}{Problem}
%----------------------------------------------------------------------------------------
%	TITLE SECTION
%----------------------------------------------------------------------------------------

\newcommand{\horrule}[1]{\rule{\linewidth}{#1}} % Create horizontal rule command with 1 argument of height

\title{	
\normalfont \normalsize 
\textsc{QFT} \\ [25pt] % Your university, school and/or department name(s)
\horrule{0.5pt} \\[0.4cm] % Thin top horizontal rule
\huge Homework 2 \\ % The assignment title
\horrule{2pt} \\[0.5cm] % Thick bottom horizontal rule
}

\author{Daniel Halmrast} % Your name

\date{\normalsize\today} % Today's date or a custom date

\begin{document}

\maketitle % Print the title

% Problems
\section*{Problem 1}
Recall the $\varphi^4$ Lagrangian is given by
\[
    \mathscr{L} = -\frac{1}{2}\partial^{\mu}\varphi\partial_{\mu}\varphi -
    \frac{1}{2}m^2\varphi^2 - \frac{1}{3!}g\varphi^3 -
    \frac{1}{4!}\lambda\varphi^4
\]
and has an energy-momentum tensor
\[
    T^{\mu\nu} = \partial^{\mu}\varphi\partial^{\nu}\varphi +
    g^{\mu\nu}\mathscr{L}
\]
\subsection*{Part A}
\begin{problem}
    Derive the equation of motion for $\varphi$ subject to the $\varphi^4$
    Lagrangian.
\end{problem}

To calculate the equation of motion for $\varphi$, we just have to find the
stationary points of
\[
    S = \int d^4x\mathscr{L} = \int d^4x
    \left(\frac{-1}{2}\partial^{\mu}\varphi\partial_{\mu}\varphi -
    \frac{1}{2}m^2\varphi^2 - \frac{1}{3!}g\varphi^3 -
\frac{1}{4!}\lambda\varphi^4\right)
\]

That is, we find when $\delta S=0$. To do so, we calculate
\[
    \begin{aligned}
        \delta S &= \int d^4x \delta \mathscr{L}\\
        &=\int d^4x 
        \left(  
            \frac{-1}{2}\delta(\partial^{\mu}\varphi\partial_{\mu}\varphi) -
            \frac{1}{2}m^2\delta(\varphi^2) - \frac{1}{3!}g\delta(\varphi^3) -
            \frac{1}{4!}\lambda\delta(\varphi^4)
        \right)\\
        &=\int d^4x 
        \left(  
            \frac{-1}{2}(\partial^{\mu}\delta\varphi\partial_{\mu}\varphi +
            \partial^{\mu}\varphi\partial_{\mu}\delta\varphi) -
            m^2\varphi\delta\varphi - \frac{1}{2}g\varphi^3\delta\varphi -
            \frac{1}{3!}\lambda\varphi^3\delta\varphi
        \right)\\
        &=\int d^4x 
        \left(  
            \partial^2\varphi\delta\varphi -
            m^2\varphi\delta\varphi - \frac{1}{2}g\varphi^3\delta\varphi -
            \frac{1}{3!}\lambda\varphi^3\delta\varphi
        \right)\\
        &=\int d^4x 
        \left(  
            \partial^2\varphi -
            m^2\varphi - \frac{1}{2}g\varphi^3 -
            \frac{1}{3!}\lambda\varphi^3
        \right)\delta\varphi\\
    \end{aligned}
\]
Which is zero for arbitrary variation if $
\left(  \partial^2\varphi - m^2\varphi - \frac{1}{2}g\varphi^3 - 
\frac{1}{3!}\lambda\varphi^3 \right)=0$. Thus, this is the equation of motion
for $\varphi$.

\newpage

\subsection*{Part B}

\begin{problem}
    Show that the energy-momentum tensor $T^{\mu\nu}$ satisfies
    $\partial_{\mu}T^{\mu\nu}=0$.
\end{problem}

This is just an exercise in direct calculation:
\[
    \begin{aligned}
        \partial_{\mu}T^{\mu\nu} &=
        \partial_{\mu}\left( \partial^{\mu}\varphi\partial^{\nu}\varphi \right)
        +\partial_{\mu}g^{\mu\nu}\mathscr{L}\\
        &=
        \partial_{\mu}\partial^{\mu}\varphi\partial^{\nu}\varphi
        +\partial^{\mu}\varphi\partial_{\mu}\partial^{\nu}\varphi
        +\partial^{\nu}
        \left(\frac{-1}{2}\partial^{\mu}\varphi\partial_{\mu}\varphi -
        \frac{1}{2}m^2\varphi^2 - \frac{1}{3!}g\varphi^3 -
        \frac{1}{4!}\lambda\varphi^4\right)\\
        &=
        \partial_{\mu}\partial^{\mu}\varphi\partial^{\nu}\varphi
        +\partial^{\mu}\varphi\partial_{\mu}\partial^{\nu}\varphi
        -\frac{1}{2}\partial^{\nu}\partial^{\mu}\varphi\partial_{\mu}\varphi 
        -\frac{1}{2}\partial^{\mu}\varphi\partial^{\nu}\partial_{\mu}\varphi
        -m^2\varphi\partial^{\nu}\varphi
        -\frac{1}{2}g\varphi^2\partial^{\nu}\varphi
        -\frac{1}{3!}\lambda\varphi^3\partial^{\nu}\varphi\\
        &=
        \partial_{\mu}\partial^{\mu}\varphi\partial^{\nu}\varphi
        -m^2\varphi\partial^{\nu}\varphi
        -\frac{1}{2}g\varphi^2\partial^{\nu}\varphi
        -\frac{1}{3!}\lambda\varphi^3\partial^{\nu}\varphi\\
        &=
        \left(
            \partial^2\varphi
            -m^2\varphi
            -\frac{1}{2}g\varphi^2
            -\frac{1}{3!}\lambda\varphi^3
        \right)\partial^{\nu}\varphi
    \end{aligned}
\]
Which is clearly zero if $\varphi$ follows its equation of motion.


\newpage
\section*{Problem 2}
Consider a complex scalar field $\varphi$ governed by the Lagrangian
\[
    \mathscr{L} = -\partial^{\mu}\varphi^{\dagger}\partial_{\mu}\varphi -
    m^2\varphi^{\dagger}\varphi + \Omega_0
\]
\subsection*{Part A}
\begin{problem}
    Show $\varphi$ obeys the Klein-Gordon equation.
\end{problem}
We calculate the variation in $S = \int d^4x\mathscr{L}$ directly:
\[
\begin{aligned}
    \delta S &= \int d^4x \delta\mathscr{L}\\
    &=\int d^4x \left(
    -\delta(\partial^{\mu}\varphi^{\dagger})\partial_{\mu}\varphi
    -\partial^{\mu}\phi^{\dagger}\delta(\partial_{\mu}\varphi)
    -m^2\varphi\delta\varphi^{\dagger}
    -m^2\varphi^{\dagger}\delta\varphi\right)\\
    &=\int d^4x \left(
    -\partial^{\mu}\delta\varphi^{\dagger}\partial_{\mu}\varphi
    -\partial^{\mu}\phi^{\dagger}\partial_{\mu}\delta\varphi
    -m^2\varphi\delta\varphi^{\dagger}
    -m^2\varphi^{\dagger}\delta\varphi\right)\\
    &=\int d^4x \left(
    \delta\varphi^{\dagger}\partial_{2}\varphi
    +\partial^{2}\phi^{\dagger}\delta\varphi
    -m^2\varphi\delta\varphi^{\dagger}
    -m^2\varphi^{\dagger}\delta\varphi\right)\\
    &=\int d^4x \left(
        \left( \partial^2\varphi-m^2\varphi \right)\delta\varphi^{\dagger}
        +\left( \partial^2\varphi^{\dagger}-m^2\varphi^{\dagger}
        \right)\delta\varphi
    \right)\\
\end{aligned}
\]
Which is zero for arbitrary variations when both $\varphi$ and
$\varphi^{\dagger}$ follow the Klein-Gordon equation.


\newpage

\subsection*{Part B}
\begin{problem}
    Find the conjugate momenta for $\varphi$ and $\varphi^{\dagger}$, and write down
    the Hamiltonian in terms of these.
\end{problem}

We can read off the conjugate momenta easily:
\[
    \begin{aligned}
        \pi(x) &= \frac{\partial\mathscr{L}}{\partial\dot{\varphi}}\\
        &=
        \frac{\partial}{\partial\dot{\varphi}}
        (\partial_{\mu}\varphi^{\dagger}(x)\partial_{\mu}\varphi(x) -
        m^2\varphi^{\dagger}(x)\varphi(x) + \Omega_0)\\
        &= \dot{\varphi^{\dagger}}(x)
    \end{aligned}
\]
and similarly
\[
    \pi^{\dagger}(x) = \dot{\varphi}(x)
\]
We write down the Hamiltonian density as
\[
    \mathscr{H} = \pi(x)\dot{\varphi}(x) +
    \pi^{\dagger}(x)\dot{\varphi^{\dagger}}(x) - \mathscr{L}
\]
and calculate
\[
    \begin{aligned}
        \mathscr{H} &= \pi(x)\dot{\varphi}(x)
        + \pi^{\dagger}(x)\dot{\varphi^{\dagger}}(x) 
        + \left( \partial^{\mu}\varphi^{\dagger}(x)\partial_{\mu}\varphi(x)
        + m^2 \varphi^{\dagger}(x)\varphi(x) - \Omega_0\right)\\
        &=\pi(x)\pi^{\dagger}(x) + \pi^{\dagger}(x)\pi(x)
        +\partial^{0}\varphi^{\dagger}(x)\partial_{0}\varphi(x)
        +\partial^{i}\varphi^{\dagger}(x)\partial_{i}\varphi(x)
        +m^2\varphi^{\dagger}(x)\varphi(x) - \Omega_0\\
        &=\pi(x)\pi^{\dagger}(x) + \pi^{\dagger}(x)\pi(x)
        -\pi(x)\pi^{\dagger}(x)
        +\partial^{i}\varphi^{\dagger}(x)\partial_{i}\varphi(x)
        +m^2\varphi^{\dagger}(x)\varphi(x) - \Omega_0\\
        &=\pi^{\dagger}(x)\pi(x)
        +\partial^{i}\varphi^{\dagger}(x)\partial_{i}\varphi(x)
        +m^2\varphi^{\dagger}(x)\varphi(x) - \Omega_0\\
    \end{aligned}
\]
Which gives the Hamiltonian for the system as
\[
    H = \int d^4x \left( \pi^{\dagger}(x)\pi(x)
        +\partial^{i}\varphi^{\dagger}(x)\partial_{i}\varphi(x)
        +m^2\varphi^{\dagger}(x)\varphi(x) - \Omega_0 \right)
\]

\subsection*{Part C}
\begin{problem}
    Expanding $\varphi$ as
    \[
        \varphi(x) = \int \tilde{d^3k}\left( a(k)\exp(ikx) +
        b^{\dagger}(k)\exp(-ikx) \right)
    \]
    solve for expressions of $a(k)$ and $b(k)$ in terms of $\varphi(x)$,
    $\varphi^{\dagger}(x)$ and their time derivatives.
\end{problem}

We'll evaluate the integrals
\[
    \begin{aligned}
        \int d^3x\exp(-ikx)\varphi(x)\\
        \int d^3x\exp(-ikx)\partial_t\varphi(x)\\
        \int d^3x\exp(-ikx)\varphi^{\dagger}(x)\\
        \int d^3x\exp(-ikx)\partial_t\varphi^{\dagger}(x)\\
    \end{aligned}
\]

So we first derive an expression for $\partial_t\varphi(x)$ and its conjugate.
\[
\begin{aligned}
    \partial_t\varphi(x) &= \int \tilde{d^3k}\partial_t\left( a(k)\exp(ikx) +
    b^{\dagger}(k)\exp(-ikx) \right)\\
    &= \int \tilde{d^3k}\left( a(k)(-i\omega)\exp(ikx) -
    b^{\dagger}(k)(-i\omega)\exp(-ikx) \right)\\
    &=\int \tilde{d^3k}(-i\omega)\left( a(k)\exp(ikx) - b^{\dagger}(k)\exp(-ikx) \right)
\end{aligned}
\]

And similarly,
\[
    \partial_t\varphi^{\dagger}(x) = \int \tilde{d^3k}(-i\omega)\left(
        b(k)\exp(ikx) - a^{\dagger}(k)\exp(-ikx)
    \right)
\]

Now, we can calculate those four integrals. One will be done explicitly, and the
other three are done using the exact same calculation.
\[
    \begin{aligned}
        \int d^3x\exp(-ikx)\varphi(x) 
        &=
        \int d^3x\frac{d^3k'}{(2\pi)^32\omega}\exp(-ikx)\left(
        a(k')\exp(ik'x) + b^{\dagger}(k')\exp(-ik'x)\right)\\
        &=
        \int d^3x\frac{d^3k'}{(2\pi)^32\omega}\left(
        a(k')\exp(i(k'-k)x) + b^{\dagger}(k')\exp(-i(k'+k)x)\right)\\
        &=
        \int d^3x\frac{d^3k'}{(2\pi)^32\omega}(
        a(k')\exp(i(k'-k)^ix_i + i(\omega'-\omega)t)\\
        &+ b^{\dagger}(k')\exp(-i(k'+k)^ix_i +
        -i(\omega'+\omega_t))\\
        &=
        \int \frac{d^3k'}{(2\pi)^32\omega}(
        a(k')(2\pi)^3\delta(k'-k)\exp(i(\omega'-\omega)t)\\
        &+ b^{\dagger}(k')(2\pi)^3\delta(k'+k)\exp(-i(\omega'+\omega)t))\\
        &=
        \frac{1}{2\omega}(
        a(k) + b^{\dagger}(-k)\exp(-i(2\omega)t))\\
    \end{aligned}
\]
Following the same tactic, we find
\[
    \begin{aligned}
        \int d^3x\exp(-ikx)\varphi(x) 
        &= \frac{1}{2\omega}(a(k) + b^{\dagger}(-k)\exp(-i(2\omega)t))\\
        \int d^3x\exp(-ikx)\partial_t\varphi(x)
        &= \frac{-i}{2}(a(k) - b^{\dagger}(-k)\exp(-i(2\omega)t))\\
        \int d^3x\exp(-ikx)\varphi^{\dagger}(x)
        &= \frac{1}{2\omega}(b(k) + a^{\dagger}(-k)\exp(-i(2\omega)t))\\
        \int d^3x\exp(-ikx)\partial_t\varphi^{\dagger}(x)
        &= \frac{-i}{2}(b(k) - a^{\dagger}(-k)\exp(-i(2\omega)t))\\
    \end{aligned}
\]

So,
\[
    a(k) = \int d^3x\exp(-ikx)\left( \omega\varphi(x) + i\partial_t\varphi(x)\right)
\]
and
\[
    b(k) =\int d^3x\exp(-ikx)\left( \omega\varphi^{\dagger}(x) +
    i\partial_t\varphi^{\dagger}(x)\right)
\]

\newpage
\subsection*{Part D}
\begin{problem}
    Derive the commutation relations for $a(k)$ and $b(k)$ and their conjugates.
\end{problem}

By conjugating, we find that
\[
    \begin{aligned}
        a^{\dagger}(k) &= \int d^3x\exp(ikx)\left( \omega\varphi^{\dagger}(x) -
        i\partial_t\varphi^{\dagger}(x)\right)\\
                b^{\dagger}(k) &= \int d^3x\exp(ikx)\left( \omega\varphi(x) -
                        i\partial_t\varphi(x)\right)
    \end{aligned}
\]

So now we can calculate the commutators directly. Let's first rewrite the
creation and annihilation operators in terms of $\varphi$ and $\pi$ instead:
\[
    \begin{aligned}
        a(k) &= \int d^3x\exp(-ikx)\left( \omega\varphi(x) 
        + i\pi^{\dagger}(x)\right)\\
        b(k) &= \int d^3x\exp(-ikx)\left( \omega\varphi^{\dagger}(x)
            + i\pi(x)\right)\\
        a^{\dagger}(k) &= \int d^3x\exp(ikx)\left( \omega\varphi^{\dagger}(x) -
            i\pi(x)\right)\\
        b^{\dagger}(k) &= \int d^3x\exp(ikx)\left( \omega\varphi(x) -
        i\pi^{\dagger}(x)\right)
    \end{aligned}
\]

Now we can use the canonical commutation relations to derive expressions for
$[a(k),a^{\dagger}(k')]$ and $[b(k),b^{\dagger}(k')$. We calculate:
\[
    \begin{aligned}
        a(k)a^{\dagger}(k') &= \int d^3xd^3y\exp(-ikx)\exp(ik'y)\left( 
            \omega\varphi(x) + i\pi^{\dagger}(x)
        \right)
        \left( 
        \omega'\varphi^{\dagger}(y) -i\pi(y)\right)\\
        &= 
        \int d^3xd^3y\exp(-ikx)\exp(ik'y)\left(
            \omega\varphi(x)\omega'\varphi^{\dagger}(y)
            -\omega\varphi(x)i\pi(y)
            +i\pi^{\dagger}(x)\omega'\varphi^{\dagger}(y)
            -i\pi^{\dagger}(x)i\pi(y)
        \right)
    \end{aligned}
\]
At this point, we invoke the rules
\[
    \begin{aligned}
        \left[\varphi(x),\varphi^{\dagger}(y)\right] = \left[
        \pi(x),\pi^{\dagger}(y) \right]&= 0 &\text{Independence of fields}\\
        [\varphi(x),\varphi(y)]=[\pi(x),\pi(y)]&=0 &\text{Canonical commutation
        relation}\\
        [\varphi(x),\pi(y)] = [\varphi^{\dagger}(x),\pi^{\dagger}(y)] &=
        i\delta(x-y) &\text{Canonical commutation relation}\\
    \end{aligned}
\]
to commute the $\varphi$ and $\pi$ fields (and their conjugates) past each
other.
Thus, we find that
\[
    \begin{aligned}
        a(k)a^{\dagger}(k')
        &= 
        \int d^3xd^3y\exp(-ikx)\exp(ik'y)\left(
            \omega\varphi(x)\omega'\varphi^{\dagger}(y)
            -\omega\varphi(x)i\pi(y)
            +i\pi^{\dagger}(x)\omega'\varphi^{\dagger}(y)
            -i\pi^{\dagger}(x)i\pi(y)
        \right)\\
        &= 
        \int d^3xd^3y\exp(-ikx)\exp(ik'y)\\
        &(
            \omega'\varphi^{\dagger}(y)\omega\varphi(x)
            -(i\pi(y)\omega\varphi(x) + i\omega(i\delta(x-y)))
            +(\omega'\varphi^{\dagger}(y)i\pi^{\dagger}(x) -
            i\omega'(i\delta(x-y)))
            -i\pi(y)i\pi^{\dagger}(x)
        )\\
        &= a^{\dagger}(k')a(k) + \int d^3xd^3y\exp(-ikx)\exp(ik'y)
        (
            +\omega\delta(x-y) + \omega'\delta(x-y)
        )\\
        &= a^{\dagger}(k')a(k) + \int d^3x\exp(-ikx)\exp(ik'x)
        (\omega + \omega')\\
        &= a^{\dagger}(k')a(k) + (2\pi)^{3}2\omega\delta(k-k')
    \end{aligned}
\]
and so
\[
    \left[ a(k),a^{\dagger}(k') \right] = (2\pi)^{3}2\omega\delta(k-k')
\]

Carrying out the exact same calculation for $\left[ b(k),b^{\dagger}(k) \right]$,
we find that (by interchanging $\varphi \leftrightarrow \varphi^{\dagger}$ and
$\pi\leftrightarrow\pi^{\dagger}$)
\[
    \left[ b(k),b^{\dagger}(k') \right] = (2\pi)^{3}2\omega\delta(k-k')
\]
as well (since the commutators of $\varphi,\pi$ and their conjugates are
identical).

\newpage
\subsection*{Part E}
\begin{problem}
    Express the Hamiltonian in terms of $a,a^{\dagger},b,b^{\dagger}$.
\end{problem}

We just need to substitute the expressions for the creation and annihilation
operators into the expression for $H$. Namely, we have the following
substitutions:
\[
    \begin{aligned}
        \varphi(x) &= \int \tilde{dk}\left( a(k)\exp(ikx) +
            b^{\dagger}(k)\exp(-ikx) \right)\\
        \varphi^{\dagger}(x) &= \int\tilde{dk}\left( b(k)\exp(ikx) +
            a^{\dagger}(k)\exp(-ikx) \right)\\
        \pi(x) &= \int \tilde{dk}(-i\omega)\left( b(k)\exp(ikx) -
            a^{\dagger}(k)\exp(-ikx) \right)\\
        \pi^{\dagger}(x) &= \int \tilde{dk}(-i\omega)\left( a(k)\exp(ikx) -
            b^{\dagger}(k)\exp(-ikx) \right)\\
        \partial^i\varphi(x) &= \int \tilde{dk}(ik^i)\left( b(k)\exp(ikx) -
            a^{\dagger}(k)\exp(-ikx)\right)\\
            \partial^i\varphi^{\dagger}(x) &= \int \tilde{dk}(ik^i)\left( a(k)\exp(ikx) -
            b^{\dagger}(k)\exp(-ikx)\right)\\
    \end{aligned}
\]

We can now express $H$ as
\[
    \begin{aligned}
        H &= \int d^3x\mathscr{H}\\
        &=\int d^3x\left( 
        \pi^{\dagger}(x)\pi(x)
        +\partial^{i}\varphi^{\dagger}(x)\partial_{i}\varphi(x)
        +m^2\varphi^{\dagger}(x)\varphi(x) - \Omega_0
        \right)\\
        &=\int d^3x\tilde{dk}\tilde{dk'}\\
            &((-i\omega)\left( 
                a(k)\exp(ikx) - b^{\dagger}(k)\exp(-ikx)
            \right)
            (-i\omega')\left( 
                b(k')\exp(ik'x) - a^{\dagger}(k')\exp(-ik'x)
            \right)\\
            &+(ik^i)\left( 
                a(k)\exp(ikx) - b^{\dagger}(k)\exp(-ikx)
            \right)
            (ik_i')\left( 
                b(k')\exp(ik'x) - a^{\dagger}(k')\exp(-ik'x)
            \right)\\
            &+m^2\left( 
                a(k)\exp(ikx)+b^{\dagger}(k)\exp(-ikx)
            \right)
            \left( 
                b(k')\exp(ik'x) + a^{\dagger}(k')\exp(-ik'x)
            \right)
        )
    \end{aligned}
\]
We collect like terms to get
\[
    \begin{aligned}
        H &=
        \int d^3x\tilde{dk}\tilde{dk'}\\
            &(\exp(i(k+k')x)\left( 
                (-i\omega)a(k)(-i\omega')b(k')
                +(ik^i)a(k)(ik_i')b(k')
                +m^2a(k)b(k')
            \right)\\
            &+\exp(i(k-k')x)\left( 
                (-i\omega)a(k)(i\omega')a^{\dagger}(k)
                +(ik^i)a(k)(-ik_i')a^{\dagger}(k')
                +m^2a(k)a^{\dagger}(k')
            \right)\\
            &+\exp(i(k'-k)x)\left( 
                (i\omega)b^{\dagger}(k)(-i\omega')b(k')
                +(-ik^i)b^{\dagger}(k)(ik_i')b(k')
                +m^2b^{\dagger}(k)b(k')
            \right)\\
            &+\exp(-i(k+k')x)\left( 
                (i\omega)b^{\dagger}(k)(i\omega')a^{\dagger}(k')
                +(-ik^i)b^{\dagger}(k)(-ik_i')a^{\dagger}(k')
                +m^2b^{\dagger}(k)a^{\dagger}(k')
            \right)
        )
    \end{aligned}
\]
Integrating out the $d^3x$ yields
\[
    \begin{aligned}
        H &=
        \int \tilde{dk}\tilde{dk'}\\
        &(2\pi)^3(\delta(k+k')\exp(i(\omega + \omega')t)a(k)b(k')\left( 
            -\omega\omega' - k^ik_i' + m^2
            \right)\\
            &+\delta(k-k')\exp(i(\omega-\omega')t)a(k)a^{\dagger}(k')\left( 
                \omega\omega' + k^ik_i' + m^2
            \right)\\
            &+\delta(k'-k)\exp(i(\omega'-\omega)t)b^{\dagger}(k)b(k')\left( 
                \omega\omega' + k^ik_i' + m^2
            \right)\\
            &+\delta(-k-k')\exp(-i(\omega +
            \omega')t)b^{\dagger}(k)a^{\dagger}(k')\left( 
                -\omega\omega' - k^ik_i' + m^2
            \right)
    \end{aligned}
\]
We then integrate out $dk'$ yielding
\[
\begin{aligned}
    H &= \int \tilde{dk}\frac{1}{2\omega}\\
    &\exp(2i\omega t)a(k)b(-k)\left( 
        -\omega^2 + k^2 + m^2
    \right)\\
    &+a(k)a^{\dagger}(k)\left( 
        \omega^2 + k^2 + m^2
    \right)\\
    &+b^{\dagger}(k)b(k)\left( 
        \omega^2 + k^2 + m^2
    \right)\\
    &+\exp(-2i\omega t)b^{\dagger}(k)a^{\dagger}(-k)\left( 
    -\omega^2 + k^2 + m^2
    \right)
\end{aligned}
\]
Since $\omega^2 = k^2 + m^2$, this simplifies greatly to
\[
    H = \int \tilde{dk}\omega\left( a(k)a^{\dagger}(k) + b^{\dagger}(k)b(k) \right)
\]
We've omitted the offset factor $\Omega_0$, but it can be safely added in to the
final result. Of course, if we want the ground state to have zero energy, we
should move the $a$ operator to the front, yielding
\[
    H = \int \tilde{dk}\omega\left( a^{\dagger}(k)a(k) + b^{\dagger}(k)b(k) +
    (2\pi)^32\omega \delta(0) \right)
\]
and if we set $\Omega_0$ to cancel with this value, we get
\[
    H = \int \tilde{dk}\omega\left( a^{\dagger}(k)a(k) + b^{\dagger}(k)b(k)
    \right)
\]
as desired.


\newpage

\section*{Problem 3}
\begin{problem}
    From the previous problem, we have a conserved current
    \[
        J^{\mu} = -i\varphi^{\dagger}\partial^{\mu}\varphi +
        i\varphi\partial^{\mu}\varphi^{\dagger}
    \]

    Express the charge
    \[
        Q = \int d^3x J^{0}
    \]
    in terms of $a,b$ and their conjugates.
\end{problem}

We use the substitutions
\[
    \begin{aligned}
        \varphi(x) &= \int \tilde{dk}\left( a(k)\exp(ikx) +
        b^{\dagger}(k)\exp(-ikx) \right)\\
        \varphi^{\dagger}(x) &= \int\tilde{dk}\left( b(k)\exp(ikx) +
        a^{\dagger}(k)\exp(-ikx) \right)\\
        \partial^{\mu}\varphi(x) &= \int \tilde{dk}(ik^{\mu})\left(
            a(k)\exp(ikx) - b^{\dagger}(k)\exp(-ikx) \right)\\
            \partial^{\mu}\varphi^{\dagger}(x) &= \int\tilde{dk}(ik^{\mu})
            \left( b(k)\exp(ikx) - a^{\dagger}(k)\exp(-ikx) \right)\\
    \end{aligned}
\]

and calculate directly
\[
    \begin{aligned}
        Q &= \int d^3x J^0\\
        &= \int d^3x\tilde{dk}\tilde{dk'}\\
        &(
            (-i)(b(k)\exp(ikx) + a^{\dagger}(k)\exp(-ikx))
            (i\omega')(a(k')\exp(ik'x) - b^{\dagger}(k')\exp(-ik'x))\\
            &+(i)(a(k)\exp(ikx) + b^{\dagger}(k)\exp(-ikx))
            (i\omega')(b(k')\exp(ik'x) - a^{\dagger}(k')\exp(-ik'x))
        )\\
        &= \int d^3x\tilde{dk}\tilde{dk'}\\
        &(
            \omega'(b(k)\exp(ikx) + a^{\dagger}(k)\exp(-ikx))
            (a(k')\exp(ik'x) - b^{\dagger}(k')\exp(-ik'x))\\
            &-(\omega')(a(k)\exp(ikx) + b^{\dagger}(k)\exp(-ikx))
            (b(k')\exp(ik'x) - a^{\dagger}(k')\exp(-ik'x))
        )\\
        &= \int d^3x\tilde{dk}\tilde{dk'}\\
        &\omega'(
            \exp(i(k+k')x)(b(k)a(k') - a(k)b(k'))\\
            &-\exp(i(k-k')x)(b(k)b^{\dagger}(k') - a(k)a^{\dagger}(k'))\\
            &+\exp(i(k'-k)x)(a^{\dagger}(k)a(k') - b^{\dagger}(k)b(k'))\\
            &-\exp(-i(k+k')x)(a^{\dagger}(k)b^{\dagger}(k') -
            b^{\dagger}(k)a^{\dagger}(k'))
        )
    \end{aligned}
\]

Integrating out $d^3x$ yields
\[
    \begin{aligned}
        Q &= \int \tilde{dk}\tilde{dk'}\\
        &\omega'(2\pi)^3(
            \delta(k+k')\exp(i(\omega+\omega')t)(b(k)a(k') - a(k)b(k'))\\
            &-\delta(k-k')\exp(i(\omega-\omega')t)(b(k)b^{\dagger}(k') - a(k)a^{\dagger}(k'))\\
            &+\delta(k'-k)\exp(i(\omega'-\omega)t)(a^{\dagger}(k)a(k') - b^{\dagger}(k)b(k'))\\
            &-\delta(-k-k')\exp(-i(\omega+\omega')t)(a^{\dagger}(k)b^{\dagger}(k')
                -
            b^{\dagger}(k)a^{\dagger}(k'))
        )
    \end{aligned}
\]
and finally integrating out $d^3k$ we get
\[
    \begin{aligned}
        Q &= \int \tilde{dk}\\
        &\frac{1}{2}(
            \exp(i(2\omega)t)(b(k)a(-k) - a(k)b(-k))\\
            &-(b(k)b^{\dagger}(k) - a(k)a^{\dagger}(k))\\
            &+(a^{\dagger}(k)a(k) - b^{\dagger}(k)b(k))\\
            &-\exp(-i(2\omega)t)(a^{\dagger}(k)b^{\dagger}(-k) -
            b^{\dagger}(k)a^{\dagger}(-k))
        )\\
        &= \int \tilde{dk}
        \frac{1}{2}(
            (a(k)a^{\dagger}(k) - b(k)b^{\dagger}(k))
            +(a^{\dagger}(k)a(k) - b^{\dagger}(k)b(k))
        )\\
        &= \int \tilde{dk}
        (
            (a^{\dagger}(k)a(k) - b^{\dagger}(k)b(k))
        )
    \end{aligned}
\]
as desired. We note that this can be expressed with the number operators $N_a$
and $N_b$ as
\[
    Q = N_a - N_b
\]
and so $a$ particles and $b$ particles have opposite equal charges.

\newpage

\section*{Problem 4}
\begin{problem}
    Consider the Lagrangian
    \[
        \mathscr{L} = -\frac{1}{2}\partial^{\mu}A^{\nu}\partial_{\mu}A_{\nu}
        +\frac{1}{2}k\partial^{\nu}A^{\mu}\partial_{\mu}A_{\nu}
        -\frac{1}{2}m^2A^{\nu}A_{\nu} + J^{\nu}A_{\nu}
    \]
\end{problem}
\subsection*{Part A}
\begin{problem}
    Compute the variation of the action, and find the equation of motion for
    $A^{\nu}$.
\end{problem}

We'll compute the variation of the action directly, noting that
\[
    \delta(\partial^{\mu}A^{\nu}) = \partial^{\mu}\delta A^{\nu}
\]

We find
\[
    \begin{aligned}
        \delta S &= \int d^4x\delta\mathscr{L}\\
        &=\int d^4x
        (
            -\frac{1}{2}\delta(\partial^{\mu}A^{\nu}\partial_{\mu}A_{\nu})
            +\frac{1}{2}k\delta(\partial^{\nu}A^{\mu}\partial_{\mu}A_{\nu})
            -\frac{1}{2}m^2\delta(A^{\nu}A_{\nu}) + J^{\nu}\delta A_{\nu}
        )\\
        &=\int d^4x
        (
            -\frac{1}{2}\partial^{\mu}\delta A^{\nu}\partial_{\mu}A_{\nu}
            -\frac{1}{2}\partial^{\mu}A^{\nu}\partial_{\mu}\delta A_{\nu}\\
            &+\frac{1}{2}k\partial^{\nu}\delta A^{\mu}\partial_{\mu}A_{\nu}
            +\frac{1}{2}k\partial^{\nu}A^{\mu}\partial_{\mu}\delta A_{\nu}\\
            &-\frac{1}{2}m^2\delta A^{\nu}A_{\nu} - \frac{1}{2}m^2A^{\nu}\delta
            A_{\nu} + J^{\nu}\delta A_{\nu}
        )
    \end{aligned}
\]
The first two terms combine if we raise/lower indices on the second term, as
well as the fifth and sixth terms. We do so, and we also interchange the indices
on the third term to make $\delta A^{\mu}\to \delta A^{\nu}$.
\[
    \begin{aligned}
        \delta S
        &=\int d^4x
        (
            -\partial^{\mu}\delta A^{\nu}\partial_{\mu}A_{\nu}
            +\frac{1}{2}k\partial^{\mu}\delta A^{\nu}\partial_{\nu}A_{\mu}
            +\frac{1}{2}k\partial^{\nu}A^{\mu}\partial_{\mu}\delta A_{\nu}
            -m^2\delta A^{\nu}A_{\nu} + J^{\nu}\delta A_{\nu}
        )
    \end{aligned}
\]
Now, if we raise/lower indices on the third term, we get
\[
    \begin{aligned}
        \delta S
        &=\int d^4x
        (
            -\partial^{\mu}\delta A^{\nu}\partial_{\mu}A_{\nu}
            +k\partial^{\mu}\delta A^{\nu}\partial_{\nu}A_{\mu}
            -m^2\delta A^{\nu}A_{\nu} + J^{\nu}\delta A_{\nu}
        )
    \end{aligned}
\]
Integrating by parts yields
\[
    \begin{aligned}
        \delta S
        &=\int d^4x
        (
            \delta A^{\nu}\partial^{\mu}\partial_{\mu}A_{\nu}
            -k\delta A^{\nu}\partial^{\mu}\partial_{\nu}A_{\mu}
            -m^2\delta A^{\nu}A_{\nu} + J^{\nu}\delta A_{\nu}
        )\\
        &=\int d^4x
        (
            \partial^2A_{\nu} - k\partial_{\nu}\partial^{\mu}A_{\mu} -
            m^2A_{\nu} + J_{\nu}
        )\delta A^{\nu}
    \end{aligned}
\]
Which holds for arbitrary variations if the term in the parentheses is always
zero. Thus, we have
\[
    -\partial^2 A_{\nu} + k\partial_{\nu}\partial^{\mu}A_{\mu} + m^2A_{\nu} =
    J_{\nu}
\]
Multiplying each side by $g^{\nu\rho}$ we get
\[
    \begin{aligned}
        g^{\nu\rho}(-\partial^2 A_{\nu} + m^2A_{\nu}) + 
        kg^{\nu\rho}\partial_{\nu}\partial^{\mu}A_{\mu} &= g^{\nu\rho}J_{\nu}\\
        g^{\nu\rho}(-\partial^2 A_{\nu} + m^2A_{\nu}) + 
        kg^{\mu\rho}\partial_{\mu}\partial^{\nu}A_{\nu} &= J^{\rho}\\
        \left[ g^{\nu\rho}(-\partial^2  + m^2) + 
        k\partial^{\rho}\partial^{\nu}\right]A_{\nu} &= J^{\rho}\\
    \end{aligned}
\]
 as desired.
\newpage
 \subsection*{Part B}
 \begin{problem}
     Find the equation of motion for $\partial_{\mu}A^{\mu}$.
 \end{problem}

We follow the hint and act on the equation of motion for $A$ with (changing
$\mu$ with $\rho$)
$\partial_{\mu}$
\[
    \begin{aligned}
        \partial_{\mu}\left( g^{\nu\mu}(-\partial^2  + m^2)A_{\nu} + 
        k\partial^{\mu}\partial^{\nu}A_{\nu}\right) &= \partial_{\mu}J^{\mu}\\
        (-\partial^2 + m^2)\partial^{\nu}A_{\nu} +
        k\partial_{\mu}\partial^{\mu}\partial^{\nu}A_{\nu} &=
        \partial_{\mu}J^{\mu}
    \end{aligned}
\]
Where the last equality was obtained by observing that derivatives commute with
each other and the metric, and $g^{\mu\nu}\partial_{\mu} = \partial^{\nu}$.
Thus, substituting $\varphi = \partial^{\nu}A_{\nu}$, we get
\[
    (-\partial^2 + m^2)\varphi + k\partial^2\varphi = \partial_{\mu}J^{\mu}
\]
and if we set $k=1$, we get the equation
\[
    m^2\varphi = \partial_{\mu}J^{\mu}
\]
as desired.

\end{document}
