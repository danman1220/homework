%%%%%%%%%%%%%%%%%%%%%%%%%%%%%%%%%%%%%%%%%
% Short Sectioned Assignment
% LaTeX Template
% Version 1.0 (5/5/12)
%
% This template has been downloaded from:
% http://www.LaTeXTemplates.com
%
% Original author:
% Frits Wenneker (http://www.howtotex.com)
%
% License:
% CC BY-NC-SA 3.0 (http://creativecommons.org/licenses/by-nc-sa/3.0/)
%
%%%%%%%%%%%%%%%%%%%%%%%%%%%%%%%%%%%%%%%%%

%----------------------------------------------------------------------------------------
%	PACKAGES AND OTHER DOCUMENT CONFIGURATIONS
%----------------------------------------------------------------------------------------

\documentclass[fontsize=11pt]{scrartcl} % 11pt font size

\usepackage[T1]{fontenc} % Use 8-bit encoding that has 256 glyphs
\usepackage[english]{babel} % English language/hyphenation
\usepackage{amsmath,amsfonts,amsthm} % Math packages
\usepackage{mathrsfs}

\usepackage[margin=1in]{geometry}

\usepackage{sectsty} % Allows customizing section commands
\allsectionsfont{\centering \normalfont\scshape} % Make all sections centered, the default font and small caps

\usepackage{fancyhdr} % Custom headers and footers
\pagestyle{fancyplain} % Makes all pages in the document conform to the custom headers and footers
\fancyhead{} % No page header - if you want one, create it in the same way as the footers below
\fancyfoot[L]{} % Empty left footer
\fancyfoot[C]{} % Empty center footer
\fancyfoot[R]{\thepage} % Page numbering for right footer
\renewcommand{\headrulewidth}{0pt} % Remove header underlines
\renewcommand{\footrulewidth}{0pt} % Remove footer underlines
\setlength{\headheight}{13.6pt} % Customize the height of the header

\numberwithin{equation}{section} % Number equations within sections (i.e. 1.1, 1.2, 2.1, 2.2 instead of 1, 2, 3, 4)
\numberwithin{figure}{section} % Number figures within sections (i.e. 1.1, 1.2, 2.1, 2.2 instead of 1, 2, 3, 4)
\numberwithin{table}{section} % Number tables within sections (i.e. 1.1, 1.2, 2.1, 2.2 instead of 1, 2, 3, 4)

\newcommand{\R}{\mathbb{R}}
\newcommand{\Q}{\mathbb{Q}}
\newcommand{\C}{\mathbb{C}}

\newtheorem{lemma}{Lemma}

%----------------------------------------------------------------------------------------
%	TITLE SECTION
%----------------------------------------------------------------------------------------

\newcommand{\horrule}[1]{\rule{\linewidth}{#1}} % Create horizontal rule command with 1 argument of height

\title{	
\normalfont \normalsize 
\textsc{Topology} \\ [25pt] % Your university, school and/or department name(s)
\horrule{0.5pt} \\[0.4cm] % Thin top horizontal rule
\huge Problem Set 3\\ % The assignment title
\horrule{2pt} \\[0.5cm] % Thick bottom horizontal rule
}

\author{Daniel Halmrast} % Your name

\date{\normalsize\today} % Today's date or a custom date

\begin{document}

\maketitle % Print the title

%----------------------------------------------------------------------------------------
%	PROBLEM 1
%----------------------------------------------------------------------------------------
\section*{Problem 1}
Prove that the 1-norm on $\R^n$ defines a metric on $\R^n$ that is equivalent to the
standard 2-norm metric on $\R^n$.
\\
\begin{proof}
Let $d_1$ be the metric induced by the 1-norm on $\R^n$. Clearly, $d_1$ is positive definite,
since it comes from a norm. So, let's show it satisfies the triangle inequality.

In proving the triangle inequality, we first state a general property of norms. The
so-called triangle inequality of norms is given as
\[
|x+y|\leq |x| + |y|
\]
which is true for any normed space.

Let $x,y,z$ be distinct points in $\R^n$ with coordinates $x^i,y^i,z^i$.
Then we have that
\[
\begin{aligned}
d(x,z)  &= \sum_i |x^i-z^i|\\
        &= \sum_i |x^i-z^i+y^i-y^i|\\
        &= \sum_i |(x^i-y^i) + (y^i-z^i)|\\
        &\leq\sum_i |x^i-y^i| + |z^i-y^i|\\
        &=\sum_i |x^i-y^i| + \sum_i |z^i-y^i|\\
        &= d(x,y) + d(y,z)
\end{aligned}
\]
and thus the metric satisfies the axioms for a metric.

Now, let's show that the metric is equivalent to the standard 2-norm metric on $\R^n$.
To do this, we will show that each point in a standard $n$-ball has a 1-norm ball
contained in the $n$-ball, and vice versa.

So, without loss of generality (via translation) let $B_r(0)$ be the open ball of radius $r$
around $0$, and let $x\in B_r(0)$. In particular, there is some $\delta>0$ such that
$d(x,0)<r-\delta$. Now, take $C_{\delta}$ to be the 1-norm ball
of radius $\delta$.
Now, if $y\in C_{\delta}$, then we have that
\[
\begin{aligned}
d(x,y)  &= \sum_i|x^i-y^i|\\
        &< \delta\\
\implies (\sum_i|x^i-y^i|)^2 &< \delta^2\\
\implies \sum_i(|x^i-y^i|)^2 &< \delta^2\\
\implies d_2(x,y) &< \delta
\implies d_2(y,0) &< d_2(x,0) + d_2(x,y)\\
                    &< r-\delta + \delta\\
                    &< r
\end{aligned}
\]
so, the 1-ball of radius $\delta$ is contained in $B_r(0)$ as desired.
Thus, since each $x\in B_r(0)$ has a neighborhood (in 1-norm) contained in $B_r(0)$,
$B_r(0)$ is open in the 1-norm topology.

For the other way, we first prove the more general fact about norms on $\R^n$.
\begin{lemma}
There exists a constant $C$ such that for all $x\in\R^n$,
\[
||x||_1 \leq C||x||_2
\]
\end{lemma}
\begin{proof}
We first observe the basic fact that, for $x_1,x_2\in\R^+$, we have
\[
2x_1x_2\leq x_1^2+x_2^2
\]
Now, it follows quickly that
\[
\begin{aligned}
||x||^2 = \left(\sum_{i=1}^n |x_i|\right) &= \sum_{i=1}^n |x_i|^2 + \sum_{i\neq j}2|x_i||x_j|\\
                                        &\leq \sum_{i=1}^n |x_i|^2 +(n-1)\sum_{i=1}^n|x_i|^2\\
                                        &= n\sum_{i=1}^n |x_i|^2
\end{aligned}
\]
Thus $\sqrt{n}$ is a constant for which the lemma holds.
\end{proof}
Now, since we have a bound on the norms, we can prove that a 1-norm ball is open in the 2-norm.
To do so, let $\Delta_r(0)$ be the 1-norm ball of radius $r$ at zero, and let
$x\in\Delta_r(0)$. In particular, we have that there exists a $\delta$ such that 
$d_1(x,0)<r-\delta$. Now, let $\varepsilon = \frac{\delta}{\sqrt{n}}$, and consider
the 2-norm ball $V_{\varepsilon}(x)$. Then, we will show that $V_{\epsilon}(x)\subset\Delta_r(0)$.
To do so, let $y\in V_{\epsilon}(x)$, and observe that
\[
\begin{aligned}
d_1(x,y) &<\sqrt{n}d_2(x,y)\\
         &<\sqrt{n}\frac{\delta}{\sqrt{n}}\\
         &=\delta
\end{aligned}
\]
and
\[
\begin{aligned}
d_1(0,y) &\leq d_1(0,x) + d_1(x,y)\\
          &\leq r-\delta + \delta\\
            &= r
\end{aligned}
\]
as desired.
\end{proof}
%----------------------------------------------------------------------------------------
\pagebreak
%----------------------------------------------------------------------------------------
%	PROBLEM 2
%----------------------------------------------------------------------------------------

%----------------------------------------------------------------------------------------
\pagebreak
%----------------------------------------------------------------------------------------
%	PROBLEM 3
%----------------------------------------------------------------------------------------

%----------------------------------------------------------------------------------------
\pagebreak
%----------------------------------------------------------------------------------------
%	PROBLEM 4
%----------------------------------------------------------------------------------------

%----------------------------------------------------------------------------------------
\pagebreak
%----------------------------------------------------------------------------------------
%	PROBLEM 5
%----------------------------------------------------------------------------------------

%----------------------------------------------------------------------------------------
\pagebreak
\end{document}
