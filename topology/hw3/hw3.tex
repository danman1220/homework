%%%%%%%%%%%%%%%%%%%%%%%%%%%%%%%%%%%%%%%%%
% Short Sectioned Assignment
% LaTeX Template
% Version 1.0 (5/5/12)
%
% This template has been downloaded from:
% http://www.LaTeXTemplates.com
%
% Original author:
% Frits Wenneker (http://www.howtotex.com)
%
% License:
% CC BY-NC-SA 3.0 (http://creativecommons.org/licenses/by-nc-sa/3.0/)
%
%%%%%%%%%%%%%%%%%%%%%%%%%%%%%%%%%%%%%%%%%

%----------------------------------------------------------------------------------------
%	PACKAGES AND OTHER DOCUMENT CONFIGURATIONS
%----------------------------------------------------------------------------------------

\documentclass[fontsize=11pt]{scrartcl} % 11pt font size

\usepackage[T1]{fontenc} % Use 8-bit encoding that has 256 glyphs
\usepackage[english]{babel} % English language/hyphenation
\usepackage{amsmath,amsfonts,amsthm} % Math packages
\usepackage{mathrsfs}

\usepackage[margin=1in]{geometry}

\usepackage{sectsty} % Allows customizing section commands
\allsectionsfont{\centering \normalfont\scshape} % Make all sections centered, the default font and small caps

\usepackage{fancyhdr} % Custom headers and footers
\pagestyle{fancyplain} % Makes all pages in the document conform to the custom headers and footers
\fancyhead{} % No page header - if you want one, create it in the same way as the footers below
\fancyfoot[L]{} % Empty left footer
\fancyfoot[C]{} % Empty center footer
\fancyfoot[R]{\thepage} % Page numbering for right footer
\renewcommand{\headrulewidth}{0pt} % Remove header underlines
\renewcommand{\footrulewidth}{0pt} % Remove footer underlines
\setlength{\headheight}{13.6pt} % Customize the height of the header

\numberwithin{equation}{section} % Number equations within sections (i.e. 1.1, 1.2, 2.1, 2.2 instead of 1, 2, 3, 4)
\numberwithin{figure}{section} % Number figures within sections (i.e. 1.1, 1.2, 2.1, 2.2 instead of 1, 2, 3, 4)
\numberwithin{table}{section} % Number tables within sections (i.e. 1.1, 1.2, 2.1, 2.2 instead of 1, 2, 3, 4)

\newcommand{\R}{\mathbb{R}}
\newcommand{\Q}{\mathbb{Q}}
\newcommand{\C}{\mathbb{C}}

\newtheorem{lemma}{Lemma}

%----------------------------------------------------------------------------------------
%	TITLE SECTION
%----------------------------------------------------------------------------------------

\newcommand{\horrule}[1]{\rule{\linewidth}{#1}} % Create horizontal rule command with 1 argument of height

\title{	
\normalfont \normalsize 
\textsc{Topology} \\ [25pt] % Your university, school and/or department name(s)
\horrule{0.5pt} \\[0.4cm] % Thin top horizontal rule
\huge Problem Set 3\\ % The assignment title
\horrule{2pt} \\[0.5cm] % Thick bottom horizontal rule
}

\author{Daniel Halmrast} % Your name

\date{\normalsize\today} % Today's date or a custom date

\begin{document}

\maketitle % Print the title

%----------------------------------------------------------------------------------------
%	PROBLEM 1
%----------------------------------------------------------------------------------------
\section*{Problem 1}
Prove that the 1-norm on $\R^n$ defines a metric on $\R^n$ that is equivalent to the
standard 2-norm metric on $\R^n$.
\\
\begin{proof}
Let $d_1$ be the metric induced by the 1-norm on $\R^n$. Clearly, $d_1$ is positive definite,
since it comes from a norm. So, let's show it satisfies the triangle inequality.

In proving the triangle inequality, we first state a general property of norms. The
so-called triangle inequality of norms is given as
\[
|x+y|\leq |x| + |y|
\]
which is true for any normed space.

Let $x,y,z$ be distinct points in $\R^n$ with coordinates $x^i,y^i,z^i$.
Then we have that
\[
\begin{aligned}
d(x,z)  &= \sum_i |x^i-z^i|\\
        &= \sum_i |x^i-z^i+y^i-y^i|\\
        &= \sum_i |(x^i-y^i) + (y^i-z^i)|\\
        &\leq\sum_i |x^i-y^i| + |z^i-y^i|\\
        &=\sum_i |x^i-y^i| + \sum_i |z^i-y^i|\\
        &= d(x,y) + d(y,z)
\end{aligned}
\]
and thus the metric satisfies the axioms for a metric.

Now, let's show that the metric is equivalent to the standard 2-norm metric on $\R^n$.
To do this, we will show that each point in a standard $n$-ball has a 1-norm ball
contained in the $n$-ball, and vice versa.

So, without loss of generality (via translation) let $B_r(0)$ be the open ball of radius $r$
around $0$, and let $x\in B_r(0)$. In particular, there is some $\delta>0$ such that
$d(x,0)<r-\delta$. Now, take $C_{\delta}$ to be the 1-norm ball
of radius $\delta$.
Now, if $y\in C_{\delta}$, then we have that
\[
\begin{aligned}
d(x,y)  &= \sum_i|x^i-y^i|\\
        &< \delta\\
\implies (\sum_i|x^i-y^i|)^2 &< \delta^2\\
\implies \sum_i(|x^i-y^i|)^2 &< \delta^2\\
\implies d_2(x,y) &< \delta
\implies d_2(y,0) &< d_2(x,0) + d_2(x,y)\\
                    &< r-\delta + \delta\\
                    &< r
\end{aligned}
\]
so, the 1-ball of radius $\delta$ is contained in $B_r(0)$ as desired.
Thus, since each $x\in B_r(0)$ has a neighborhood (in 1-norm) contained in $B_r(0)$,
$B_r(0)$ is open in the 1-norm topology.

For the other way, we first prove the more general fact about norms on $\R^n$.
\begin{lemma}
There exists a constant $C$ such that for all $x\in\R^n$,
\[
||x||_1 \leq C||x||_2
\]
\end{lemma}
\begin{proof}
We first observe the basic fact that, for $x_1,x_2\in\R^+$, we have
\[
2x_1x_2\leq x_1^2+x_2^2
\]
Now, it follows quickly that
\[
\begin{aligned}
||x||^2 = \left(\sum_{i=1}^n |x_i|\right) &= \sum_{i=1}^n |x_i|^2 + \sum_{i\neq j}2|x_i||x_j|\\
                                        &\leq \sum_{i=1}^n |x_i|^2 +(n-1)\sum_{i=1}^n|x_i|^2\\
                                        &= n\sum_{i=1}^n |x_i|^2
\end{aligned}
\]
Thus $\sqrt{n}$ is a constant for which the lemma holds.
\end{proof}
Now, since we have a bound on the norms, we can prove that a 1-norm ball is open in the 2-norm.
To do so, let $\Delta_r(0)$ be the 1-norm ball of radius $r$ at zero, and let
$x\in\Delta_r(0)$. In particular, we have that there exists a $\delta$ such that 
$d_1(x,0)<r-\delta$. Now, let $\varepsilon = \frac{\delta}{\sqrt{n}}$, and consider
the 2-norm ball $V_{\varepsilon}(x)$. Then, we will show that $V_{\epsilon}(x)\subset\Delta_r(0)$.
To do so, let $y\in V_{\epsilon}(x)$, and observe that
\[
\begin{aligned}
d_1(x,y) &<\sqrt{n}d_2(x,y)\\
         &<\sqrt{n}\frac{\delta}{\sqrt{n}}\\
         &=\delta
\end{aligned}
\]
and
\[
\begin{aligned}
d_1(0,y) &\leq d_1(0,x) + d_1(x,y)\\
          &\leq r-\delta + \delta\\
            &= r
\end{aligned}
\]
as desired.
\end{proof}
%----------------------------------------------------------------------------------------
\pagebreak
%----------------------------------------------------------------------------------------
%	PROBLEM 2
%----------------------------------------------------------------------------------------
\section*{Problem 2}
\subsection*{Munkres Problem 4}
Consider the box, uniform, and product topologies on $\R^{\omega}$.
\subsubsection*{Part a}
In which topologies are the following functions continuous?
\[
    \begin{aligned}
        f(t) &= (t,2t,3t,\ldots)\\
        g(t) &= (t,t,t,\ldots)\\
        h(t) &= (t,\frac{1}{2}t,\frac{1}{3}t,\ldots)
    \end{aligned}
\]

\begin{proof}
We first note that the universal property of product spaces guarantees that a
    function $f$ is continuous in the product topology if and only if its
    component functions $\pi_i\circ f$ are continuous.  Since this is true for
    all three of $f,g,h$, it follows that they are all continuous in the product
    topology.

For the remainder of this problem, we will use the pointwise definition of
    continuity. That is, given a point $x\in\R$, $f$ is convergent at $x$ if and
    only if for each neighborhood $U$ of $f(x)$, we have that $f^{-1}(U)$
    contains a neighborhood of $x$.

For $f(t)$, let's consider the basic open neighborhood $U_t$ in
    $\R^{\omega}$ around $f(t)$ in the uniform topology, which looks like
    \[
        U_t = \bigcup_{\delta<\varepsilon} \prod_i V_{\delta}(it)
    \]
    which has an inverse image of
    \[
        \begin{aligned}
            f_i^{-1}(U_t) &= \bigcup_{\delta<\varpsilon}
            f_i^{-1}(V_{\delta}(it))
                        &= \bigcup_{\delta<\varepsilon}V_{\frac{\delta}{i}}(t)
        \end{aligned}
    \]
    which goes to $\{t\}$ as $i$ goes to infinity. Thus, the inverse image is
    just $\{t\}$, which cannot contain an open set, so $f$ is not continuous
    in the uniform topology. Then, since the box topology is finer than
    the uniform topology, $f$ is not continuous in the box topology either.

    Now, consider $g$, which we will show is continuous in the uniform
    topology, but not in the box topology.

    To see this, consider in the uniform topology, the neighborhood $U_t$
    around $g(t)$, which is given as
    \[
        U_t = \bigcup_{\delta<\varepsilon} \prod_i V_{\delta}(t)
    \]
    Now, the inverse image of this is just
    \[
    g^{-1}(U_t) = V_{\varepsilon}(t)
    \]
    which is open in $\R$, so $g$ is continuous in the uniform topology.

    However, in the box topology, we have the neighborhood
    \[
        U_t = \prod_i V_{\frac{\varepsilon}{i}}(t)
    \]
    whose inverse image (as shown above) is just $\{t\}$, so it cannot contain
    an open set, and $g$ is not continuous in the box topology.

    Now, consider $h$, which is also continuous in the uniform
    topology, but not in the box topology.
    
    To see this, consider again a neighborhood in the uniform topology
    \[
        U_t = \bigcup_{\delta<\varepsilon} \prod_i V_{\delta}(\frac{t}{i})
    \]
    which has an inverse image of
    \[
    h_i^{-1}(U_t) = \bigcup_{\delta<\varepsilon} V_{i\delta}(t)
    \]
    and a composite inverse image of
    \[
    h^{-1}(U_t) = \bigcup_{\delta<\varepsilon} V_{\delta}(t)
    \]
    Which certainly contains an open neighborhood of $t$, so $h$ is open
    in the uniform topology.

    However, in the box topology, the neighborhood
    \[
    U_t = \prod_i V_{\frac{\varepsilon}{i^2}}(\frac{t}{i})
    \]
    which has an inverse image of just $\{t\}$, so $h$ is not open in the box
    topology
\end{proof}

\subsubsection*{Part b}
In which topologies do the sequences $w,x,y,z$ (definitions omitted) converge?
\\
\begin{proof}
We note that all sequences converge in each coordinate, so they converge in the
product topology.

Now, the $w$ sequence does not converge in the uniform topology, since for
    $\varepsilon = 1$, the open set $\bigcup_{\delta<\varepsilon} V_{\delta}(0)$
    never eventually contains the sequence, since the terms far enough down the
    coordinates keep growing. Thus, it does not converge in the box topology
    either.

    The $x$ sequence does converge in the uniform topology, since for every
    $\varepsilon>0$, the sequence eventually gets to where every term is below
    $\frac{1}{n}$ for any $n$, so the sequence eventually fits in the
    neighborhood
    \[
    U_0 = \prod_i V_{\varepsilon}(0)
    \]
    However, in the box topology, this sequence does not converge. This is
    because the neighborhood
    \[
        U_0 = \prod_i V_{\frac{1}{i^2}}
    \]
    does not eventually contain the sequence. In particular, the $i\th$
    coordinate of the $i\th$ term in the sequence is always $\frac{1}{i}$,\
    which is never in $V_{\frac{1}{i^2}}(0)$.

    The $y$ sequence has the same properties as the $x$ sequence described
    above, so it does converge in the uniform topology. However, since the
    diagonal elements are always $\frac{1}{i}$, the neighborhood
    \[
        U_0 = \prod_i V_{\frac{1}{i^2}}
    \]
    never eventually contains the sequence.

    Now, the $z$ sequence does converge in the box topology.
    To see this, we note that the $z$ sequence lies in the subspace
    $\R^2\times\prod\{0\}$, which (by an earlier homework assignment) is
    homeomorphic to $\R^2$. Now, in $\R^2$, the product topology and the
    box topology coincide, so by the observation that the $z$ sequence converges
    in the product topology, it must also converge in the box topology and
    the uniform topology as well.
\end{proof}

\subsection*{Munkres Problem 5}
What is $\overline{\R^{\infty}}$ in the uniform topology on $\R^{\omega}$?
\\
\begin{proof}
    We will show that the closure of $\R^{\infty}$ in the uniform topology
    is the set of all sequences which converge (in norm) to zero. This is clear,
    since for any sequence $(x_i)$ which did not converge to zero, there must be some
    $\varepsilon$ such that the sequence is eventually $\varepsilon - \delta$ away from
    zero.
    Then, the open neighborhood around $(x_i)$ given as
    \[
        U = \bigcup_{\delta<\varepsilon}\prod_i V_{\delta}(x_i)
    \]
    will have infinitely many terms whose projections do not intersect zero,
    but any sequence in $\R^{\infty}$ must eventually be constantly zero, thus
    will eventually leave the neighborhood.

    However, for any sequence $(y_i)$ that converges to zero, any neighborhood
    of $(y_i)$ must intersect $\R^{\infty}$. To see this, we consider that
    since $(y_i)$ converges to zero, for any $\varepsilon$, the sequence
    $(y_i)$ must eventually be within $\varepsilon$ of zero. Thus, for any
    neighborhood
    \[
        U = \bigcup_{\delta<\varepsilon}\prod V_{\delta}(y_i)
    \]
    it must be that for some $N>0$ and for all $n>N$, $V_{\delta}(y_n$
    intersects zero. Thus, the element $(y_1,\ldots,y_N,0,\ldots)\in\R^{\infty}$
    is also in $U$.
    Therefore, the closure is all sequences that converge to zero in norm.
\end{proof}
\subsection*{Munkres Problem 6}

%----------------------------------------------------------------------------------------
\pagebreak
%----------------------------------------------------------------------------------------
%	PROBLEM 3
%----------------------------------------------------------------------------------------

%----------------------------------------------------------------------------------------
\pagebreak
%----------------------------------------------------------------------------------------
%	PROBLEM 4
%----------------------------------------------------------------------------------------

%----------------------------------------------------------------------------------------
\pagebreak
%----------------------------------------------------------------------------------------
%	PROBLEM 5
%----------------------------------------------------------------------------------------

%----------------------------------------------------------------------------------------
\pagebreak
\end{document}
