%%%%%%%%%%%%%%%%%%%%%%%%%%%%%%%%%%%%%%%%%
% Short Sectioned Assignment
% LaTeX Template
% Version 1.0 (5/5/12)
%
% This template has been downloaded from:
% http://www.LaTeXTemplates.com
%
% Original author:
% Frits Wenneker (http://www.howtotex.com)
%
% License:
% CC BY-NC-SA 3.0 (http://creativecommons.org/licenses/by-nc-sa/3.0/)
%
%%%%%%%%%%%%%%%%%%%%%%%%%%%%%%%%%%%%%%%%%

%----------------------------------------------------------------------------------------
%	PACKAGES AND OTHER DOCUMENT CONFIGURATIONS
%----------------------------------------------------------------------------------------

\documentclass[fontsize=11pt]{scrartcl} % 11pt font size

\usepackage[T1]{fontenc} % Use 8-bit encoding that has 256 glyphs
\usepackage[english]{babel} % English language/hyphenation
\usepackage{amsmath,amsfonts,amsthm, amssymb} % Math packages
\usepackage{mathrsfs}

\usepackage[margin=1in]{geometry}

\usepackage{sectsty} % Allows customizing section commands
\allsectionsfont{\centering \normalfont\scshape} % Make all sections centered, the default font and small caps

\usepackage{fancyhdr} % Custom headers and footers
\pagestyle{fancyplain} % Makes all pages in the document conform to the custom headers and footers
\fancyhead{} % No page header - if you want one, create it in the same way as the footers below
\fancyfoot[L]{} % Empty left footer
\fancyfoot[C]{} % Empty center footer
\fancyfoot[R]{\thepage} % Page numbering for right footer
\renewcommand{\headrulewidth}{0pt} % Remove header underlines
\renewcommand{\footrulewidth}{0pt} % Remove footer underlines
\setlength{\headheight}{13.6pt} % Customize the height of the header

\numberwithin{equation}{section} % Number equations within sections (i.e. 1.1, 1.2, 2.1, 2.2 instead of 1, 2, 3, 4)
\numberwithin{figure}{section} % Number figures within sections (i.e. 1.1, 1.2, 2.1, 2.2 instead of 1, 2, 3, 4)
\numberwithin{table}{section} % Number tables within sections (i.e. 1.1, 1.2, 2.1, 2.2 instead of 1, 2, 3, 4)

\newcommand{\R}{\mathbb{R}}
\newcommand{\Q}{\mathbb{Q}}
\newcommand{\N}{\mathbb{N}}
\newcommand{\C}{\mathbb{C}}

\newtheorem{lemma}{Lemma}
%----------------------------------------------------------------------------------------
%	TITLE SECTION
%----------------------------------------------------------------------------------------

\newcommand{\horrule}[1]{\rule{\linewidth}{#1}} % Create horizontal rule command with 1 argument of height

\title{	
\normalfont \normalsize 
\textsc{Topology} \\ [25pt] % Your university, school and/or department name(s)
\horrule{0.5pt} \\[0.4cm] % Thin top horizontal rule
\huge Problem Set 5 \\ % The assignment title
\horrule{2pt} \\[0.5cm] % Thick bottom horizontal rule
}

\author{Daniel Halmrast} % Your name

\date{\normalsize\today} % Today's date or a custom date

\begin{document}

\maketitle % Print the title

\section*{Problem 1}
Show that for $(X,\mathscr{T})$ a compact Hausdorff space, any topology
$\mathscr{T}\subsetneq\mathscr{T}'$ makes $(X,\mathscr{T}')$ no longer compact
Hausdorff. Furthermore, for $\mathscr{T}'\subsetneq\mathscr{T}$, the space
$(X,\mathscr{T}')$ is not compact Hausdorff.
\\
\begin{proof}
    For this proof, we will show that for $\mathscr{T},\mathscr{T}'$ topologies
    on $X$ with $\mathscr{T}\subset\mathscr{T}'$ and the property that $X$ is
    compact Hausdorff under both these topologies, then they must be equal.

    To do so, we consider the identity function
    $id:(X,\mathscr{T}')\to(X,\mathscr{T})$, which is continuous since
    $\mathscr{T}\subset\mathscr{T}'$. Now, for any $C$ closed in
    $(X,\mathscr{T}')$, since $(X,\mathscr{T}')$ is compact and $id$ is
    continuous, $id(C)=C$ is compact in $(X,\mathscr{T})$. However, since
    $(X,\mathscr{T})$ is Hausdorff, $C$ must also be closed. Thus, $id$ is a
    closed map, and thus a homeomorphism, and the two topologies must be equal.
\end{proof}

\section*{Problem 2}
Prove that for compact subsets $A,B$ of $X$, their union $A\cup B$ is compact as
well.
\\
\begin{proof}
    Let $\mathscr{O}$ be an open cover of $A\cup B$. In particular,
    $\mathscr{O}$ covers $A$, and has a finite subset $\mathscr{O}_A$ that
    covers $A$. Similarly, there is a finite subset $\mathscr{O}_B$ that
    covers $B$. Their union $\mathscr{O}_A\cup\mathscr{O}_B$ covers $A\cup B$,
    and is finite, as it is the union of finite sets. It is also a subcover,
    since both $\mathscr{O}_A$ and $\mathscr{O}_B$ are subsets of
    $\mathscr{O}$. Thus. $\mathscr{O}$ has a finite subcover, and $A\cup B$ is
    compact as desired.
\end{proof}

\section*{Problem 3}
Suppose $A$ is a subset of a metric space. Show that $A$ is compact implies that
$A$ is closed and bounded. Give an example where the converse does not hold.
\\
\begin{proof}
    Suppose $A$ is a compact subset of a metric space $(X,d)$. Since all metric
    spaces are Hausdorff, it follows immediately that $A$ is closed (since it is
    a compact subset of a Hausdorff space).

    Now, fix $x\in A$, and consider the open cover $\mathscr{O} = \{V_n(x)\ |\
    n\in\N\}$ of balls centered at $x$ with radius $n$. This has a finite
    subcover, by compactness of $A$. So, let $N$ be the largest radius in the
    finite subcover. For each $V_m(x)$ in the finite subcover, then, we have
    that $V_m(x)\subset V_N(x)$. Thus, since the subcover covers $A$, it follows
    that $V_N(x)$ covers $A$, and thus $A$ is bounded.

    For an example where the converse fails, consider the metric space
    $(\R,d_{LA})$ of the real numbers with the discrete (Los Angeles) metric.
    The set $\R$ itself is closed and bounded (since everything is at most
    distance $1$ away from any particular point), but is clearly not compact,
    since it is an uncountable discrete set of points.
\end{proof}

\section*{Problem 4}
Show that for $X$ Hausdorff, $A,B$ disjoint compact subsets of $X$, then there
exist disjoint open sets $U$ and $V$ containing $A$ and $B$ respectively.
\\
\begin{proof}
Let $a\in A$, and $b\in B$. Since $A$ and $B$ are disjoint, it must be that
    $a\neq b$. Thus, there exist disjoint open sets $U_{ab},V_{ab}$ such that $a\in
    U_{ab}$ and $b\in V_{ab}$. Now, since $B$ is compact, the open cover
    $\mathscr{O}_B = \{V_{ab}\ |\ b\in B\}$ has a finite subcover
    $\{V_{ab_i}\}_{i=1}^n$. Furthermore, the open set $U_a =
    \bigcap_{i=1}^nU_{ab_i}$ does not intersect the open set $V_a =
    \bigcup_{i=1}^nV_{ab_i}$ which covers $B$,

    So, consider the open cover $\mathscr{O}_A = \{U_a\ |\ a\in A\}$. By
    compactness of $A$, this has a finite subcover $\{U_{a_j}\}_{j=1}^m$. Now,
    we have that the open set $V = \bigcap_{j=1}^mV_{a_j}$ (which covers $B$, since
    each $V_a$ covers $B$) does not intersect the open set $U =
    \bigcup_{j=1}^mU_{a_j}$ which covers $A$.

    Thus, $U$ and $V$ are disjoint open sets containing $A$ and $B$
    respectively, as desired.
\end{proof}

\section*{Problem 5}
Show that for $X$ and $Y$ topological spaces with $Y$ compact, the projection
$\pi_x:X\times Y\to X$ is a closed map.
\\
\begin{proof}
    Let $C\subset X\times Y$ be closed, and let $C_X = \pi_X(C)$. Now, for $x\in
    X\setminus C_X$, we know that the slice $\{x\}\times Y$ does not intersect
    $C$, and thus $\{x\}\times Y\subset X\times Y\setminus C$, which is an open
    subset of $X\times Y$. Since the slice $\{x\}\times Y$ is contained in this
    open set, and $Y$ is compact, we have by the tube lemma that there is some
    neighborhood $U_x$ of $x$ such that $U_x\times Y$ is completely contained in
    $X\times Y\setminus C$. Therefore, since $U_\times Y\cap C = \emptyset$, it
    must be that $U_x\cap C_x = \emptyset$ as well. Thus, each point $x$ in the
    complement of $C_x$ has a neighborhood that is also in the complement of
    $C_x$, and so the complement of $C_x$ is open, and $C_x$ is closed, as
    desired.
\end{proof}

\section*{Problem 6}
Suppose $Y$ is compact Hausdorff, and $f:X\to Y$ a function of sets. Prove that
$f$ is continuous if and only if its graph is a closed subset of $X\times Y$.
\\
\begin{proof}
    ($\implies$)
    Suppose $f$ is continuous. Specifically, we know that for every net $x_{\alpha}$
    in $X$ that converges to some point $x\in X$, the net $f(x_{\alpha})$
    in $Y$ converges to $f(x)\in Y$.

    So, consider a net $(x_{\alpha},f(x_{\alpha}))$ in the graph $\{(x,f(x))\ |\
    x\in X\}$ that converges to some $(x,y)\in X\times Y$. Now, since the
    projection maps $\pi_x,\pi_y$ are continuous, they preserve nets, so
    \[
        \pi_x(x_{\alpha},f(x_{\alpha})) = x_{\alpha} \to \pi_x(x,y)=x
    \]
    and
    \[
        \pi_y(x<{\alpha},f(x_{\alpha})) = f(x_{\alpha}) \to \pi_y(x,y) = y
    \]
    but we know that $f(x_{\alpha})\to f(x)$ by continuity of $f$, and since $Y$
    is Hausdorff, nets have unique limits, so $y=f(x)$, and the net
    $(x_{\alpha},f(x_{\alpha}))$ converges to $(x,f(x))\in \{(x,f(x)\ |\ x\in
    X\}$, and so the graph of $f$ is closed, as desired.
    \\
    \\
    ($\impliedby$)
    Suppose $f$ is not continuous. That is, suppose there exists some net
    $x_{\alpha}$ in $X$ converging to some $L\in X$ such that
    $f(x_{\alpha})$ does not converge to $f(L)$.

    Now, consider the net $(x_{\alpha},f(x_{\alpha}))$ in $X\times Y$. In
    particular, consider the projection $\pi_y((x_{\alpha},f(x_{\alpha}))) =
    f(x_{\alpha})$. Since $Y$ is compact, it follows that this net has a
    convergent subnet $f(x_{\alpha_{\beta}})\to y$ for some $y\in Y$.

    Now, in the product space, we have the net
    \[
        (x_{\alpha_{\beta}}, f(x_{\alpha_{\beta}}))
    \]
    which converges in the first coordinate to some subset $L'\subset L$, and in
    the second coordinate to $y$. Now, suppose $y\not\in f(L)$. Then, we have
    that
    \[
        (x_{\alpha_{\beta}}, f(x_{\alpha_{\beta}}))\to L'\times \{y\}
    \]
    but since $y\not\in f(L)$, it follows that
    \[
        L'\times \{y\}\not\subset \{(x,f(x))\ |\ x\in X\}
    \]
    and thus the graph is not closed.

    Now, suppose that $y\in f(L)$. Then, it follows that, since
    $f(x_{\alpha})\not\to f(L)$, there is some subnet $f(x_{\alpha_{\gamma}})$
    for which $y$ is not an accumulation point. In particular, this subnet has a
    convergent sub-subnet which does not converge to $y$. Applying the argument
    above to this sub-subnet yields a net in the graph that does not converge in
    the graph, as desired.
\end{proof}

\end{document}
