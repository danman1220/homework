%%%%%%%%%%%%%%%%%%%%%%%%%%%%%%%%%%%%%%%%%
% Short Sectioned Assignment
% LaTeX Template
% Version 1.0 (5/5/12)
%
% This template has been downloaded from:
% http://www.LaTeXTemplates.com
%
% Original author:
% Frits Wenneker (http://www.howtotex.com)
%
% License:
% CC BY-NC-SA 3.0 (http://creativecommons.org/licenses/by-nc-sa/3.0/)
%
%%%%%%%%%%%%%%%%%%%%%%%%%%%%%%%%%%%%%%%%%

%----------------------------------------------------------------------------------------
%	PACKAGES AND OTHER DOCUMENT CONFIGURATIONS
%----------------------------------------------------------------------------------------

\documentclass[fontsize=11pt]{scrartcl} % 11pt font size

\usepackage[T1]{fontenc} % Use 8-bit encoding that has 256 glyphs
\usepackage[english]{babel} % English language/hyphenation
\usepackage{amsmath,amsfonts,amsthm} % Math packages
\usepackage{mathrsfs}
\usepackage{tikz-cd}

\usepackage[margin=1in]{geometry}

\usepackage{sectsty} % Allows customizing section commands
\allsectionsfont{\centering \normalfont\scshape} % Make all sections centered, the default font and small caps

\usepackage{fancyhdr} % Custom headers and footers
\pagestyle{fancyplain} % Makes all pages in the document conform to the custom headers and footers
\fancyhead{} % No page header - if you want one, create it in the same way as the footers below
\fancyfoot[L]{} % Empty left footer
\fancyfoot[C]{} % Empty center footer
\fancyfoot[R]{\thepage} % Page numbering for right footer
\renewcommand{\headrulewidth}{0pt} % Remove header underlines
\renewcommand{\footrulewidth}{0pt} % Remove footer underlines
\setlength{\headheight}{13.6pt} % Customize the height of the header

\numberwithin{equation}{section} % Number equations within sections (i.e. 1.1, 1.2, 2.1, 2.2 instead of 1, 2, 3, 4)
\numberwithin{figure}{section} % Number figures within sections (i.e. 1.1, 1.2, 2.1, 2.2 instead of 1, 2, 3, 4)
\numberwithin{table}{section} % Number tables within sections (i.e. 1.1, 1.2, 2.1, 2.2 instead of 1, 2, 3, 4)

\newcommand{\R}{\mathbb{R}}
\newcommand{\Q}{\mathbb{Q}}
\newcommand{\N}{\mathbb{N}}
\newcommand{\C}{\mathbb{C}}

\newtheorem*{lemma}{Lemma}
%----------------------------------------------------------------------------------------
%	TITLE SECTION
%----------------------------------------------------------------------------------------

\newcommand{\horrule}[1]{\rule{\linewidth}{#1}} % Create horizontal rule command with 1 argument of height

\title{	
\normalfont \normalsize 
\textsc{Topology} \\ [25pt] % Your university, school and/or department name(s)
\horrule{0.5pt} \\[0.4cm] % Thin top horizontal rule
\huge Problem Set 2\\ % The assignment title
\horrule{2pt} \\[0.5cm] % Thick bottom horizontal rule
}

\author{Daniel Halmrast} % Your name

\date{\normalsize\today} % Today's date or a custom date

\begin{document}

\maketitle % Print the title

%----------------------------------------------------------------------------------------
%	PROBLEM 1
%----------------------------------------------------------------------------------------
\section*{Problem 1}
Prove that there is an embedding of $X$ into $X\times Y$.
\\
\begin{proof}
For this proof, $\{\bullet\}$ will represent the one-point set.

To start with, we will prove the following lemma:
\begin{lemma}
For $X$ any topological space, $X\cong X\times\{\bullet\}$.
\end{lemma}
\begin{proof}
By the definition of the product space, the projection maps
\[
\begin{tikzcd}[column sep=tiny]
 & X\times\{\bullet\} \arrow[ld,"\pi_x"] \arrow[rd, "\pi_{\bullet}" swap]\\
X & & \{\bullet\}
\end{tikzcd}
\]
exist and are continuous open maps. Now, all we need to show is that $\pi_x$ is
injective, and it will follow immediately that it is a homeomorphism.

To see this, let $x\in X$ and consider $\pi_x^{-1}(\{x\}) = \{(x,\bullet)\}$. Since
the inverse image of a singleton is again a singleton, the function is injective.

Thus, $X$ is homeomorphic to $X\times\{\bullet\}$.
\end{proof}

Now, let $f:\{\bullet\}\to Y$ be a continuous function. Consider the diagram:
\[
\begin{tikzcd}[column sep=small, row sep=large]
 & X \arrow[leftrightarrow]{d}{\cong} & \\
 &X\times\{\bullet\} \arrow[d, dashed, "id\times f" description] \arrow[bend right=20, swap]{ddl}{id} \arrow[bend left=20]{ddr}{f}& \\
 &X\times Y \arrow{dl}{\pi_x} \arrow[swap]{dr}{\pi_y} & \\
X & & Y
\end{tikzcd}
\]
where $id$ and $f$ are the obvious extensions $id(x,\bullet) = id(x) = x$ and $f(x,\bullet) = f(\bullet)$.
Here, the product map $id\times f$ is continuous by the universal property of products.
Now, we just need to show that $id\times f$ is injective with a continuous inverse on its image. 

To see that $id\times f$ is injective, consider a point $(id(x),f(\bullet))$ in the image
of $id\times f$, and consider its preimage:
\[
(id\times f)^{-1}(\{(id(x),f(\bullet))\}) = \{(x,\bullet)\}
\]
Since the preimage of any singleton is again a singleton, the function $id\times f$ is
injective.

Now, lets consider the diagram
\[
\begin{tikzcd}[column sep=small, row sep=large]
 &X\times Y \arrow[d, dashed, "\pi_x \times c" description] \arrow[bend right=20, swap]{ddl}{\pi_x} \arrow[bend left=20]{ddr}{c}& \\
 &X\times \{\bullet\} \arrow{dl}{\pi_x} \arrow[swap]{dr}{pi_{\bullet}} & \\
X & & \{\bullet\}
\end{tikzcd}
\]
where $c$ is unique constant function from $Y$ to the terminal object $\{\bullet\}$.

Here, the dashed arrow $\pi_x\times c$ is continuous by the universal property of products.
It is easy to see that $\pi_x\times c|_{(id\times f)(X\times\{\bullet\})}$ is the
inverse of $id\times f$ on the image of $id\times f$.

Hence, since the inverse of $id\times f$ is continuous, $id\times f$ is an embedding
of $X\cong X\times \{\bullet\}$ into $X\times Y$.
\end{proof}
%----------------------------------------------------------------------------------------
%----------------------------------------------------------------------------------------
%	PROBLEM 2
%----------------------------------------------------------------------------------------
\section*{Problem 2}
Prove that every open interval in $\R$ is homeomorphic to $\R$.
\\
\begin{proof}
Consider an open interval $(a,b)\subset \R$. It is easy to see that $(a,b)\cong(-1,1)$,
since the operations of scaling and translation are continuous functions with continuous
inverses.

Thus, all we need to prove is that $(-1,1)\cong\R$. To see this, consider the function
\[
\tan(\frac{\pi}{2} x)
\]
defined on $(-1,1)$,
which is a continuous bijection with continuous inverse. (proofs for the continuity of $\tan$ and $\arctan$
are easily given by basic analysis arguments, and will not be reproduced here.)
\end{proof}
%----------------------------------------------------------------------------------------
%----------------------------------------------------------------------------------------
%	PROBLEM 3
%----------------------------------------------------------------------------------------
\section*{Problem 3}
Give an example of a function from $\R$ to $\R$ that is continuous at exactly one point.
\\
\begin{proof}
The function
\[
\begin{aligned}
f:\R&\to\R\\
f(x) &=x\chi_{\Q}(x)
\end{aligned}
\]
is continuous only at zero. To see this, we will use the neighborhood definition of
continuity. That is, $f$ is continuous at $x$ if for each neighborhood of $f(x)$, its
preimage contains a neighborhood of $x$.

First, we will prove that $f$ is continuous at zero. It suffices to show that each basic
open neighborhood of $f(x)$ has a preimage that contains an open neighborhood of $x$.
So, let $(-\varepsilon,\varepsilon)$ be a basic neighborhood of $f(0)=0$. Then,
\[
f^{-1}((-\varepsilon,\varepsilon)) = \R\setminus\Q\cup(-\varepsilon,\varepsilon)
\]
which contains $(-\varepsilon,\varepsilon)$ an open neighborhood of $0$ as desired.

Now, let $x \neq 0$. We will show that $f$ is not continuous at $x$. If $x$ is irrational,
then $f(x) = 0$. Now, choose $\varepsilon$ so that $x\not\in(-\varepsilon,\varepsilon)$.
Then, by the above calculation, we have
\[
f^{-1}((-\varepsilon,\varepsilon)) = \R\setminus\Q\cup(-\varepsilon,\varepsilon)
\]
which does not contain any neighborhood of $x$.

If $x$ is rational, then $f(x) = x$. Choose $\varepsilon$ such that $0\not\in V_{\varepsilon}(x)$.
Then,
\[
f^{-1}(V_{\varepsilon}(x)) = V_{\varepsilon}(x)\cap\Q
\]
which does not contain any open neighborhood of $x$ (This is easily seen by observing
that any neighborhood of $x$ must intersect $\R\setminus\Q$, but the inverse image contains
only rational points).
\end{proof}
%----------------------------------------------------------------------------------------
%----------------------------------------------------------------------------------------
%	PROBLEM 4
%----------------------------------------------------------------------------------------
\section*{Problem 4}
Suppose $Y$ is Hausdorff, and
\begin{tikzcd}
X \arrow[shift left]{r}{f} \arrow[shift right]{r}[swap]{g} & Y
\end{tikzcd}
are continuous. If $f|_A = g|_A$ for a dense subset $A\subset Y$, prove that
$f=g$.
\\
\begin{proof}
Let $f$ and $g$ be parallel morphisms that satisfy the assumptions.

Now, let $y\in Y$. Since $A$ is dense, $y\in\overline{A}$, so there exists some
net $\{y_{\alpha}\}$ such that $y_{\alpha}\in A$ for all $\alpha$ and $y_{\alpha}\to y$.
In particular, since $Y$ is Hausdorff, this net converges to the unique limit $y$.

By the hypothesis, $f(y_{\alpha}) = g(y_{\alpha})\ \forall\alpha$, and since both $f$
and $g$ are continuous, they preserve limits. That is $f(y_{\alpha})\to f(y)$ and
$g(y_{\alpha})\to g(y)$. Since $f(y_{\alpha}) = g(y_{\alpha})$ for all $\alpha$ and 
limits of nets in $Y$ are unique, they must converge to the same element, and $f(y)=g(y)$.

Since this works for all $y\in Y$, $f=g$.
\end{proof}

%----------------------------------------------------------------------------------------
%----------------------------------------------------------------------------------------
%	PROBLEM 5
%----------------------------------------------------------------------------------------
\section*{Problem 5}
Prove that if $A_{\alpha}$ is a closed subset of $X_{\alpha}$ for all $\alpha$, then
$\prod A_{\alpha}$ is closed in $\prod X_{\alpha}$.
\\
\begin{proof}
To show that $\prod A_{\alpha}$ is closed, we need to show that it contains its limit points.
To do so, let $\{a_{\gamma}\}$ be a convergent net in the product $\prod A_{\alpha}$. In
particular, each of its projections $\pi_{\alpha}(a_\gamma)$ is also a net in $A_{\alpha}$,
and since $A_{\alpha}$ is closed, this net converges to elements in $A_{\alpha}$.

Thus, each coordinate $\alpha$ of the net $\{a_{\gamma}\}$ converges in $A_{\alpha}$, so
any limit point must have coordinates in the $A_{\alpha}$ as well. That is, if $a$ is a
limit point of $\{a_{\gamma}\}$, then for each $\alpha$, $\pi_{\alpha}(a)\in A_{\alpha}$,
which means that $a\in\prod A_{\alpha}$ as desired.

Since $\prod A_{\alpha}$ contains all its limit points, it is closed.
\end{proof}
%----------------------------------------------------------------------------------------
%----------------------------------------------------------------------------------------
%	PROBLEM 6
%----------------------------------------------------------------------------------------
\section*{Problem 6}
Let $y\in\prod X_{\alpha}$, and $\{x_n\}$ a sequence of points in $\prod X_{\alpha}$.
Show that $x_n\to y$ if and only if $\pi_{\alpha}(x_n)\to\pi_{\alpha}(y)$ for all $\alpha$.
\\
\begin{proof}
(=>) For the first direction, assume that $x_n\to y$. Since each $\pi_{\alpha}$ is continuous,
they preserve limits. Thus, for each $\alpha$, $\pi_{\alpha}(x_n)\to\pi_{\alpha}(y)$ as
desired.

(<=) For the other direction, let $\{x_n\}$ be such that for all $\alpha$, $\pi_{\alpha}(x_n)\to\pi_{\alpha}(y)$.
In particular, this means that the filter $\mathscr{F} = \{A\subset \prod X_{\alpha}\ |\ \exists n\in\N:\ x_m\in A\ \forall m>n\}$
pushes forward along each $\pi_{\alpha}$ to a filter that converges to $\pi_{\alpha}(y)$.

Now, we just need to show that $\mathscr{F}$ converges to $y$ (Equivalently, that $\mathscr{F}$
contains each neighborhood of $y$). To do so, we will show that each neighborhood of
$y$ contains an element of $\mathscr{F}$, then since $\mathscr{F}$ is a filter,
it is closed under supersets and contains each neighborhood of $y$.

So, let $U$ be a neighborhood of $y$. In particular, there exists a basis element
\[
B = V_1\times V_2\times \ldots \times V_n\times X\times X\ldots \subset U
\]
Now, since the push-forward of $\mathscr{F}$ along each projection is a convergent filter,
$N_{\alpha}\in\pi_{\alpha *}(\mathscr{F})$ for each neighborhood $N_{\alpha}$of $\pi_{\alpha}(y)$.

In particular, $V_i\in\pi_{\alpha *}(\mathscr{F})$, which means that $\pi^{-1}_{\alpha}(V_i)\in\mathscr{F}$.
Now, we can write $B$ as
\[
B = \bigcap_{i=1}^n \pi_{\alpha}^{-1}(V_i)
\]
which is a finite intersection of elements of $\mathscr{F}$, so $B\in\mathscr{F}$. Thus,
$U\supset B$ is in $\mathscr{F}$ as well. Since $U$ was any neighborhood of $y$, the
neighborhood filter $\mathscr{N}_y\subset\mathscr{F}$ and $\mathscr{F}\to y$ as desired.
\end{proof}
%----------------------------------------------------------------------------------------
%----------------------------------------------------------------------------------------
%	PROBLEM 7
%----------------------------------------------------------------------------------------
\section*{Problem 7}
Let $\R^{\omega}$ be the space of sequences of real numbers, and let $\R^{\infty}$ be
the space of sequences that are eventually zero. What is $\overline{\R^{\infty}}\subset\R^{\omega}$?

%----------------------------------------------------------------------------------------
%----------------------------------------------------------------------------------------
%	PROBLEM 8
%----------------------------------------------------------------------------------------

%----------------------------------------------------------------------------------------

\end{document}
