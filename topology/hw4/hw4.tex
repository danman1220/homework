%%%%%%%%%%%%%%%%%%%%%%%%%%%%%%%%%%%%%%%%%
% Short Sectioned Assignment
% LaTeX Template
% Version 1.0 (5/5/12)
%
% This template has been downloaded from:
% http://www.LaTeXTemplates.com
%
% Original author:
% Frits Wenneker (http://www.howtotex.com)
%
% License:
% CC BY-NC-SA 3.0 (http://creativecommons.org/licenses/by-nc-sa/3.0/)
%
%%%%%%%%%%%%%%%%%%%%%%%%%%%%%%%%%%%%%%%%%

%----------------------------------------------------------------------------------------
%	PACKAGES AND OTHER DOCUMENT CONFIGURATIONS
%----------------------------------------------------------------------------------------

\documentclass[fontsize=11pt]{scrartcl} % 11pt font size

\usepackage[T1]{fontenc} % Use 8-bit encoding that has 256 glyphs
\usepackage[english]{babel} % English language/hyphenation
\usepackage{amsmath,amsfonts,amsthm} % Math packages
\usepackage{mathrsfs}
\usepackage{tikz-cd}

\usepackage[margin=1in]{geometry}

\usepackage{sectsty} % Allows customizing section commands
\allsectionsfont{\centering \normalfont\scshape} % Make all sections centered, the default font and small caps

\usepackage{fancyhdr} % Custom headers and footers
\pagestyle{fancyplain} % Makes all pages in the document conform to the custom headers and footers
\fancyhead{} % No page header - if you want one, create it in the same way as the footers below
\fancyfoot[L]{} % Empty left footer
\fancyfoot[C]{} % Empty center footer
\fancyfoot[R]{\thepage} % Page numbering for right footer
\renewcommand{\headrulewidth}{0pt} % Remove header underlines
\renewcommand{\footrulewidth}{0pt} % Remove footer underlines
\setlength{\headheight}{13.6pt} % Customize the height of the header

\numberwithin{equation}{section} % Number equations within sections (i.e. 1.1, 1.2, 2.1, 2.2 instead of 1, 2, 3, 4)
\numberwithin{figure}{section} % Number figures within sections (i.e. 1.1, 1.2, 2.1, 2.2 instead of 1, 2, 3, 4)
\numberwithin{table}{section} % Number tables within sections (i.e. 1.1, 1.2, 2.1, 2.2 instead of 1, 2, 3, 4)

\newcommand{\R}{\mathbb{R}}
\newcommand{\Q}{\mathbb{Q}}
\newcommand{\N}{\mathbb{N}}
\newcommand{\C}{\mathbb{C}}

\newtheorem{lemma}{Lemma}
%----------------------------------------------------------------------------------------
%	TITLE SECTION
%----------------------------------------------------------------------------------------

\newcommand{\horrule}[1]{\rule{\linewidth}{#1}} % Create horizontal rule command with 1 argument of height

\title{	
\normalfont \normalsize 
\textsc{Topology} \\ [25pt] % Your university, school and/or department name(s)
\horrule{0.5pt} \\[0.4cm] % Thin top horizontal rule
\huge Problem Set 4 \\ % The assignment title
\horrule{2pt} \\[0.5cm] % Thick bottom horizontal rule
}

\author{Daniel Halmrast} % Your name

\date{\normalsize\today} % Today's date or a custom date

\begin{document}

\maketitle % Print the title

% Problems
\section*{Problem 1}
Prove that $[0,1)$ is not homeomorphic to $(0,1)$.
\\
\begin{proof}
    Let $f$ be a bijective function from $[0,1)$ to $(0,1)$. Since $f$ is
    bijective, it follows that $f^{-1}((0,1))=[0,1)$. However, $(0,1)$ is open,
    but $[0,1)$ is not. Thus, $f$ is not continuous.

    Since no continuous bijections exist from $[0,1)$, to $(0,1)$, they are not
    homeomorphic.
\end{proof}

\section*{Problem 2}
Prove that $\R$ is not homeomorphic to $\R^2$.
\\
\begin{proof}
    Suppose there existed a homeomorphism $f:\R\to\R^2$. Now, consider
    restricting the domain of $f$ to $\R\setminus\{0\}$. This yields a
    homeomorphism between $\R\setminus\{0\}$ and $\R^2\setminus\{f(0)\}$.
    However, such a homeomorphism cannot exist, since $\R\setminus\{0\}$ is not
    connected, but $\R^2\setminus\{f(0)\}$ is connected, and connectedness is a
    topological property.
\end{proof}

\newpage

\section*{Problem 3}
Prove that every continuous function $f:[0,1]\to [0,1]$ has a fixed point.
\\
\begin{proof}
    Suppose there existed a function $f:I\to I$ (with $I=[0,1]$) such that $f$
    has no fixed points. In particular, this defines a (continuous) retract
    $r:I\to \partial I$ given by
    \[
        r(x) =
        \begin{cases}
            1, &\textrm{if } x > f(x)\\
            0, &\textrm{if } x < f(x)
        \end{cases}
    \]
    Now, we see clearly that $r^{-1}(1)\neq \emptyset$, since at $x=1$, $f(1)$
    cannot be greater than $1$, and thus must be less than $1$, forcing
    $r(1)=1$. Similarly, $r^{-1}(0)\neq \emptyset$, since at $x=0$, $f(0)$
    cannot be less than $0$, and thus must be greater than $0$, forcing
    $r(0)=0$.

    Therefore , $r$ is a continuous function from $I$ to the two-point set, and
    defines a separation of $I$. But $I$ is connected, so no such separation can
    exist. Thus, such a retract cannot exist, and $f$ must have a fixed point.
\end{proof}

\section*{Problem 4}
Prove that $X\times X$ is connected if and only if $X$ is.
\\
\begin{proof}
    ($\implies$)
    Suppose $X$ is not connected. In particular, there exists a continuous
    surjection from $X$ to the two-point set.

    Thus, we have the diagram
    \[
        \begin{tikzcd}[column sep=small]
            &X\times X
            \arrow[two heads]{dl}[swap]{\pi_1}
            \arrow[two heads]{dr}{\pi_2}
            &\\
            X
            \arrow[two heads]{dr}[swap]{s_1} 
            &&
            X
            \arrow[two heads]{dl}{s_2}\\
            &\{0,1\}&
        \end{tikzcd}
    \]
    In particular, the composition $s_1\circ\pi_1$ is a surjection from
    $X\times X$ onto $\{0,1\}$, and defines a separation of $X\times
    X$. Thus, $X\times X$ is separated.

    ($\impliedby$)
    Suppose $X\times X$ is not connected. In particular, there exists a
    continuous surjection $s:X\times X\to\{0,1\}$. Now, for each $x_{\alpha}\in
    X$, we know that the map $i_{\alpha}:X\to X\times X$ given by
    $i(x)=(x,x_{\alpha})$ is an embedding of $X$ into $X\times X$. I assert that
    there exists some $x_0$ for which $s^{-1}(\{0\})\cap X\times\{x_0\}\neq
    \emptyset$ and $s^{-1}(\{1\})\cap X\times\{x_0\}\neq \emptyset$.

\end{proof}

\end{document}
