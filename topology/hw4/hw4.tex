%%%%%%%%%%%%%%%%%%%%%%%%%%%%%%%%%%%%%%%%%
% Short Sectioned Assignment
% LaTeX Template
% Version 1.0 (5/5/12)
%
% This template has been downloaded from:
% http://www.LaTeXTemplates.com
%
% Original author:
% Frits Wenneker (http://www.howtotex.com)
%
% License:
% CC BY-NC-SA 3.0 (http://creativecommons.org/licenses/by-nc-sa/3.0/)
%
%%%%%%%%%%%%%%%%%%%%%%%%%%%%%%%%%%%%%%%%%

%----------------------------------------------------------------------------------------
%	PACKAGES AND OTHER DOCUMENT CONFIGURATIONS
%----------------------------------------------------------------------------------------

\documentclass[fontsize=11pt]{scrartcl} % 11pt font size

\usepackage[T1]{fontenc} % Use 8-bit encoding that has 256 glyphs
\usepackage[english]{babel} % English language/hyphenation
\usepackage{amsmath,amsfonts,amsthm} % Math packages
\usepackage{mathrsfs}
\usepackage{tikz-cd}

\usepackage[margin=1in]{geometry}

\usepackage{sectsty} % Allows customizing section commands
\allsectionsfont{\centering \normalfont\scshape} % Make all sections centered, the default font and small caps

\usepackage{fancyhdr} % Custom headers and footers
\pagestyle{fancyplain} % Makes all pages in the document conform to the custom headers and footers
\fancyhead{} % No page header - if you want one, create it in the same way as the footers below
\fancyfoot[L]{} % Empty left footer
\fancyfoot[C]{} % Empty center footer
\fancyfoot[R]{\thepage} % Page numbering for right footer
\renewcommand{\headrulewidth}{0pt} % Remove header underlines
\renewcommand{\footrulewidth}{0pt} % Remove footer underlines
\setlength{\headheight}{13.6pt} % Customize the height of the header

\numberwithin{equation}{section} % Number equations within sections (i.e. 1.1, 1.2, 2.1, 2.2 instead of 1, 2, 3, 4)
\numberwithin{figure}{section} % Number figures within sections (i.e. 1.1, 1.2, 2.1, 2.2 instead of 1, 2, 3, 4)
\numberwithin{table}{section} % Number tables within sections (i.e. 1.1, 1.2, 2.1, 2.2 instead of 1, 2, 3, 4)

\newcommand{\R}{\mathbb{R}}
\newcommand{\Q}{\mathbb{Q}}
\newcommand{\N}{\mathbb{N}}
\newcommand{\C}{\mathbb{C}}

\newtheorem{lemma}{Lemma}
%----------------------------------------------------------------------------------------
%	TITLE SECTION
%----------------------------------------------------------------------------------------

\newcommand{\horrule}[1]{\rule{\linewidth}{#1}} % Create horizontal rule command with 1 argument of height

\title{	
\normalfont \normalsize 
\textsc{Topology} \\ [25pt] % Your university, school and/or department name(s)
\horrule{0.5pt} \\[0.4cm] % Thin top horizontal rule
\huge Problem Set 4 \\ % The assignment title
\horrule{2pt} \\[0.5cm] % Thick bottom horizontal rule
}

\author{Daniel Halmrast} % Your name

\date{\normalsize\today} % Today's date or a custom date

\begin{document}

\maketitle % Print the title

% Problems
\section*{Problem 1}
Prove that $[0,1)$ is not homeomorphic to $(0,1)$.
\\
\begin{proof}
    Let $f$ be a bijective function from $[0,1)$ to $(0,1)$. Since $f$ is
    bijective, it follows that $f^{-1}((0,1))=[0,1)$. However, $(0,1)$ is open,
    but $[0,1)$ is not. Thus, $f$ is not continuous.

    Since no continuous bijections exist from $[0,1)$, to $(0,1)$, they are not
    homeomorphic.
\end{proof}

\section*{Problem 2}
Prove that $\R$ is not homeomorphic to $\R^2$.
\\
\begin{proof}
    Suppose there existed a homeomorphism $f:\R\to\R^2$. Now, consider
    restricting the domain of $f$ to $\R\setminus\{0\}$. This yields a
    homeomorphism between $\R\setminus\{0\}$ and $\R^2\setminus\{f(0)\}$.
    However, such a homeomorphism cannot exist, since $\R\setminus\{0\}$ is not
    connected, but $\R^2\setminus\{f(0)\}$ is connected, and connectedness is a
    topological property.
\end{proof}

\newpage

\section*{Problem 3}
Prove that every continuous function $f:[0,1]\to [0,1]$ has a fixed point.
\\
\begin{proof}
    Suppose there existed a function $f:I\to I$ (with $I=[0,1]$) such that $f$
    has no fixed points. In particular, this defines a (continuous) retract
    $r:I\to \partial I$ given by
    \[
        r(x) =
        \begin{cases}
            1, &\textrm{if } x > f(x)\\
            0, &\textrm{if } x < f(x)
        \end{cases}
    \]
    Now, we see clearly that $r^{-1}(1)\neq \emptyset$, since at $x=1$, $f(1)$
    cannot be greater than $1$, and thus must be less than $1$, forcing
    $r(1)=1$. Similarly, $r^{-1}(0)\neq \emptyset$, since at $x=0$, $f(0)$
    cannot be less than $0$, and thus must be greater than $0$, forcing
    $r(0)=0$.

    Therefore , $r$ is a continuous function from $I$ to the two-point set, and
    defines a separation of $I$. But $I$ is connected, so no such separation can
    exist. Thus, such a retract cannot exist, and $f$ must have a fixed point.
\end{proof}

\section*{Problem 4}
Prove that $X\times X$ is connected if and only if $X$ is.
\\
\begin{proof}
    ($\implies$)
    Suppose $X$ is not connected. In particular, there exists a continuous
    surjection from $X$ to the two-point set.

    Thus, we have the diagram
    \[
        \begin{tikzcd}[column sep=small]
            &X\times X
            \arrow[two heads]{dl}[swap]{\pi_1}
            \arrow[two heads]{dr}{\pi_2}
            &\\
            X
            \arrow[two heads]{dr}[swap]{s_1} 
            &&
            X
            \arrow[two heads]{dl}{s_2}\\
            &\{0,1\}&
        \end{tikzcd}
    \]
    In particular, the composition $s_1\circ\pi_1$ is a surjection from
    $X\times X$ onto $\{0,1\}$, and defines a separation of $X\times
    X$. Thus, $X\times X$ is separated.
    \\
    \\
    ($\impliedby$)
    Suppose $X$ is connected. We wish to show that any function $f:X\times
    X\to\{0,1\}$ is constant.

    So, let $f:X\times X\to \{0,1\}$ be a continuous function,
    Now, since $X$ is connected, and for all $x\in X$, $X\cong \{x\}\times X 
    \cong X\times \{x\}$, it must be that $f$ is constant on each of these
    fibers.

    So, for arbitrary, $(x_1,y_1)$ and $(x_2,y_2)$, we have that
    \[
        f((x_1,y_1)) = f((x_1,y_2)) = f((x_2,y_2))
    \]
    where in each equality we used that $f$ is constant on the fibers
    $\pi_1^{-1}(x_1)$ and $\pi_2^{-1}(y_2)$.

    Thus, $f$ is constant, and $X\times X$ cannot be separated.
\end{proof}

\newpage

\section*{Problem 5}
Suppose that $A_{\alpha}$ are all path connected subspaces of a space $X$, and
that $\cap A_{\alpha}\neq \emptyset$. Prove that $\cup A_{\alpha}$ is
path-connected.
\\
\begin{proof}
    Let $x,y$ be distinct points in $\cup A_{\alpha}$. In particular, $x\in A_x$
    for some $A_x$, and $y\in A_y$ for some $A_y$. Now, let $x_0\in \cap
    A_{\alpha}$. Then, since $A_x$ is path-connected, there is a path $\gamma_x$
    from $x$ to $x_0$. Similarly for $y$, there is a path $\gamma_y$ from $y$ to
    $x_0$. Then, the path $\gamma_x - \gamma_y$ is a continuous path from $x$ to
    $y$. 

    Since this can be done for any $x,y\in \cup A_{\alpha}$, it follows that
    $\cup A_{\alpha}$ is path-connected.
\end{proof}

\section*{Problem 6}
Prove that $\R^2\setminus \Q^2$ is path-connected.
\\
\begin{proof}
    Let $(x_1,y_1),(x_2,y_2)$ be distinct points in $\R^2\setminus \Q^2$, and
    without loss of generality, let $x_1\neq x_2$. Now, fix an irrational point
    $x_0$ such that $x_1<x_0<x_2$, and consider the set $\{x_0\}\times \R\subset
    \R^2\setminus\Q^2$. Now, this set is clearly uncountable, since it is
    homeomorphic to $\R$. Furthermore, each point $y$ in $\{x_0\}\times \R$ defines
    a path in $\R^2$ by taking the straight line from $(x_1,y_1)$ to $(x_0,y)$,
    then the straight line from $(x_0,y)$ to $(x_2,y_2)$.

    Now, each path must necessarily intersect unique rationals, since if two
    paths intersected the same rational point (and since they originated from
    the same position) by linearity they must be the same line.

    So, since there are only countably many rationals, but uncountably many
    paths defined in this manner, there must be a path that does not intersect
    any rationals.

    Thus, there is a path in $\R^2\setminus \Q^2$ that connects $(x_1,y_1)$ and
    $(x_2,y_2)$, and since these points were arbitrary, it must be that
    $\R^2\setminus \Q^2$ is path-connected, as desired.
\end{proof}

\section*{Problem 7}
Prove that every connected open subset of $\R^2$ is path-connected.
\\
\begin{proof}
Suppose that an open subset $U\subset\R^2$ is not
    path-connected. In particular, let $A_{\alpha}$ be the path-components of
    $U$. Since $U$ is open as a subset of $\R^2$, it follows that every point of
    $U$ has an $\varepsilon$-ball around it contained in $U$. Since
    $\varepsilon$-balls are path-connected in $\R^2$, it follows that for $a\in
    A_{\alpha}$, the $\varepsilon$-ball around $a$ is contained in $A_{\alpha}$
    (since $A_{\alpha}$ is a path-component). Thus, since each point in
    $A_{\alpha}$ has a basic
    open neighborhood in $A_{\alpha}$, it follows that $A_{\alpha}$ is open.

    Since each path-component is disjoint and open, and they union to $U$, it
    follows that they form a separation of $U$, and thus $U$ is not connected.
    
    Therefore, if $U$ is connected, then $U$ is path-connected.
\end{proof}

\end{document}
