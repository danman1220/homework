%%%%%%%%%%%%%%%%%%%%%%%%%%%%%%%%%%%%%%%%%
% Short Sectioned Assignment
% LaTeX Template
% Version 1.0 (5/5/12)
%
% This template has been downloaded from:
% http://www.LaTeXTemplates.com
%
% Original author:
% Frits Wenneker (http://www.howtotex.com)
%
% License:
% CC BY-NC-SA 3.0 (http://creativecommons.org/licenses/by-nc-sa/3.0/)
%
%%%%%%%%%%%%%%%%%%%%%%%%%%%%%%%%%%%%%%%%%

%----------------------------------------------------------------------------------------
%	PACKAGES AND OTHER DOCUMENT CONFIGURATIONS
%----------------------------------------------------------------------------------------

\documentclass[fontsize=11pt]{scrartcl} % 11pt font size

\usepackage[T1]{fontenc} % Use 8-bit encoding that has 256 glyphs
\usepackage[english]{babel} % English language/hyphenation
\usepackage{amsmath,amsfonts,amsthm} % Math packages
\usepackage{mathrsfs}

\usepackage[margin=1in]{geometry}

\usepackage{sectsty} % Allows customizing section commands
\allsectionsfont{\centering \normalfont\scshape} % Make all sections centered, the default font and small caps

\usepackage{fancyhdr} % Custom headers and footers
\pagestyle{fancyplain} % Makes all pages in the document conform to the custom headers and footers
\fancyhead{} % No page header - if you want one, create it in the same way as the footers below
\fancyfoot[L]{} % Empty left footer
\fancyfoot[C]{} % Empty center footer
\fancyfoot[R]{\thepage} % Page numbering for right footer
\renewcommand{\headrulewidth}{0pt} % Remove header underlines
\renewcommand{\footrulewidth}{0pt} % Remove footer underlines
\setlength{\headheight}{13.6pt} % Customize the height of the header

\numberwithin{equation}{section} % Number equations within sections (i.e. 1.1, 1.2, 2.1, 2.2 instead of 1, 2, 3, 4)
\numberwithin{figure}{section} % Number figures within sections (i.e. 1.1, 1.2, 2.1, 2.2 instead of 1, 2, 3, 4)
\numberwithin{table}{section} % Number tables within sections (i.e. 1.1, 1.2, 2.1, 2.2 instead of 1, 2, 3, 4)

\newcommand{\R}{\mathbb{R}}
\newcommand{\Q}{\mathbb{Q}}
\newcommand{\N}{\mathbb{N}}
\newcommand{\C}{\mathbb{C}}

\newtheorem{lemma}{Lemma}
%----------------------------------------------------------------------------------------
%	TITLE SECTION
%----------------------------------------------------------------------------------------

\newcommand{\horrule}[1]{\rule{\linewidth}{#1}} % Create horizontal rule command with 1 argument of height

\title{	
\normalfont \normalsize 
\textsc{Analysis} \\ [25pt] % Your university, school and/or department name(s)
\horrule{0.5pt} \\[0.4cm] % Thin top horizontal rule
\huge Problem Set 8 \\ % The assignment title
\horrule{2pt} \\[0.5cm] % Thick bottom horizontal rule
}

\author{Daniel Halmrast} % Your name

\date{\normalsize\today} % Today's date or a custom date

\begin{document}

\maketitle % Print the title

\section*{Problem 1}
Show that $\frac{\sin(x)}{x}$ is not in $L^1((0,\infty),\lambda^1)$.
\\
\begin{proof}
We wish to evaluate
    \[
        \int_{(0,\infty)}\frac{|\sin(x)|}{x}d\lambda^1(x)
    \]
    and show that it diverges. To do so, we split the integral into half-cycles
    \[
        \int_{(0,\infty)}\frac{|\sin(x)|}{x}d\lambda^1(x) = 
        \sum_{n=0}^{\infty} \int_{(n\pi,(n+1)\pi)}\frac{|\sin(x)|}{x}d\lambda^1(x)
    \]
    Now, we know that on each half-cycle,
    \[
        \frac{|\sin(x)|}{x} \geq \frac{|\sin(x)|}{(n+1)\pi}
    \]
    so we have a lower bound for the integral:
    \[
        \sum_{n=0}^{\infty}
        \int_{(n\pi,(n+1)\pi)}\frac{|\sin(x)|}{x}d\lambda^1(x)\geq
        \sum_{n=0}^{\infty}
        \int_{(n\pi,(n+1)\pi)}\frac{|\sin(x)|}{(n+1)\pi}d\lambda^1(x)
    \]
    Now, since each half-cycle is either entirely positive or entirely negative,
    we know that
    \[
        \int_{(n\pi,(n+1)\pi)}\frac{|\sin(x)|}{(n+1)\pi}d\lambda^1(x)=
        \left|\int_{(n\pi,(n+1)\pi)}\frac{\sin(x)}{(n+1)\pi}d\lambda^1(x)\right|
    \]
    And finally, we can evaluate the integral directly:
    \[
        \begin{aligned}
        \sum_{n=0}^{\infty}
        \left|\int_{(n\pi,(n+1)\pi)}\frac{\sin(x)}{(n+1)\pi}d\lambda^1(x)\right|
            &=
        \sum_{n=0}^{\infty}
            \left|\frac{1}{(n+1)\pi}[\cos(x)]_{n\pi}^{(n+1)\pi}\right|\\
            &=
        \sum_{n=0}^{\infty}
            \frac{2}{(n+1)\pi}\\
            &=\infty
        \end{aligned}
    \]
    Thus, since 
    \[
        \int_{(0,\infty)}\frac{|\sin(x)|}{x}d\lambda^1(x)\geq
        \sum_{n=0}^{\infty}
        \left|\int_{(n\pi,(n+1)\pi)}\frac{\sin(x)}{(n+1)\pi}d\lambda^1(x)\right|
        = \infty
    \]
    the integral diverges, and $\frac{\sin(x)}{x}$ is not in
    $L^1((0,\infty),\lambda^1)$.
\end{proof}

\newpage

\section*{Problem 2}
Prove that $L^p$ for $1\leqp\leq\infty$ is complete. Note that case $p=1$ has
already been covered in class.
\\
\begin{proof}
To begin with, let $p<\infty$. 
\end{proof}


\end{document}
