%%%%%%%%%%%%%%%%%%%%%%%%%%%%%%%%%%%%%%%%%
% Short Sectioned Assignment
% LaTeX Template
% Version 1.0 (5/5/12)
%
% This template has been downloaded from:
% http://www.LaTeXTemplates.com
%
% Original author:
% Frits Wenneker (http://www.howtotex.com)
%
% License:
% CC BY-NC-SA 3.0 (http://creativecommons.org/licenses/by-nc-sa/3.0/)
%
%%%%%%%%%%%%%%%%%%%%%%%%%%%%%%%%%%%%%%%%%

%----------------------------------------------------------------------------------------
%	PACKAGES AND OTHER DOCUMENT CONFIGURATIONS
%----------------------------------------------------------------------------------------

\documentclass[fontsize=11pt]{scrartcl} % 11pt font size

\usepackage[T1]{fontenc} % Use 8-bit encoding that has 256 glyphs
\usepackage[english]{babel} % English language/hyphenation
\usepackage{amsmath,amsfonts,amsthm} % Math packages
\usepackage{mathrsfs}

\usepackage[margin=1in]{geometry}

\usepackage{sectsty} % Allows customizing section commands
\allsectionsfont{\centering \normalfont\scshape} % Make all sections centered, the default font and small caps

\usepackage{fancyhdr} % Custom headers and footers
\pagestyle{fancyplain} % Makes all pages in the document conform to the custom headers and footers
\fancyhead{} % No page header - if you want one, create it in the same way as the footers below
\fancyfoot[L]{} % Empty left footer
\fancyfoot[C]{} % Empty center footer
\fancyfoot[R]{\thepage} % Page numbering for right footer
\renewcommand{\headrulewidth}{0pt} % Remove header underlines
\renewcommand{\footrulewidth}{0pt} % Remove footer underlines
\setlength{\headheight}{13.6pt} % Customize the height of the header

\numberwithin{equation}{section} % Number equations within sections (i.e. 1.1, 1.2, 2.1, 2.2 instead of 1, 2, 3, 4)
\numberwithin{figure}{section} % Number figures within sections (i.e. 1.1, 1.2, 2.1, 2.2 instead of 1, 2, 3, 4)
\numberwithin{table}{section} % Number tables within sections (i.e. 1.1, 1.2, 2.1, 2.2 instead of 1, 2, 3, 4)

\newcommand{\R}{\mathbb{R}}
\newcommand{\Q}{\mathbb{Q}}
\newcommand{\N}{\mathbb{N}}
\newcommand{\C}{\mathbb{C}}

\newtheorem{lemma}{Lemma}
%----------------------------------------------------------------------------------------
%	TITLE SECTION
%----------------------------------------------------------------------------------------

\newcommand{\horrule}[1]{\rule{\linewidth}{#1}} % Create horizontal rule command with 1 argument of height

\title{	
\normalfont \normalsize 
\textsc{Analysis} \\ [25pt] % Your university, school and/or department name(s)
\horrule{0.5pt} \\[0.4cm] % Thin top horizontal rule
\huge Problem Set 8 \\ % The assignment title
\horrule{2pt} \\[0.5cm] % Thick bottom horizontal rule
}

\author{Daniel Halmrast} % Your name

\date{\normalsize\today} % Today's date or a custom date

\begin{document}

\maketitle % Print the title

\section*{Problem 1}
Show that $\frac{\sin(x)}{x}$ is not in $L^1((0,\infty),\lambda^1)$.
\\
\begin{proof}
We wish to evaluate
    \[
        \int_{(0,\infty)}\frac{|\sin(x)|}{x}d\lambda^1(x)
    \]
    and show that it diverges. To do so, we split the integral into half-cycles
    \[
        \int_{(0,\infty)}\frac{|\sin(x)|}{x}d\lambda^1(x) = 
        \sum_{n=0}^{\infty} \int_{(n\pi,(n+1)\pi)}\frac{|\sin(x)|}{x}d\lambda^1(x)
    \]
    Now, we know that on each half-cycle,
    \[
        \frac{|\sin(x)|}{x} \geq \frac{|\sin(x)|}{(n+1)\pi}
    \]
    so we have a lower bound for the integral:
    \[
        \sum_{n=0}^{\infty}
        \int_{(n\pi,(n+1)\pi)}\frac{|\sin(x)|}{x}d\lambda^1(x)\geq
        \sum_{n=0}^{\infty}
        \int_{(n\pi,(n+1)\pi)}\frac{|\sin(x)|}{(n+1)\pi}d\lambda^1(x)
    \]
    Now, since each half-cycle is either entirely positive or entirely negative,
    we know that
    \[
        \int_{(n\pi,(n+1)\pi)}\frac{|\sin(x)|}{(n+1)\pi}d\lambda^1(x)=
        \left|\int_{(n\pi,(n+1)\pi)}\frac{\sin(x)}{(n+1)\pi}d\lambda^1(x)\right|
    \]
    And finally, we can evaluate the integral directly:
    \[
        \begin{aligned}
        \sum_{n=0}^{\infty}
        \left|\int_{(n\pi,(n+1)\pi)}\frac{\sin(x)}{(n+1)\pi}d\lambda^1(x)\right|
            &=
        \sum_{n=0}^{\infty}
            \left|\frac{1}{(n+1)\pi}[\cos(x)]_{n\pi}^{(n+1)\pi}\right|\\
            &=
        \sum_{n=0}^{\infty}
            \frac{2}{(n+1)\pi}\\
            &=\infty
        \end{aligned}
    \]
    Thus, since 
    \[
        \int_{(0,\infty)}\frac{|\sin(x)|}{x}d\lambda^1(x)\geq
        \sum_{n=0}^{\infty}
        \left|\int_{(n\pi,(n+1)\pi)}\frac{\sin(x)}{(n+1)\pi}d\lambda^1(x)\right|
        = \infty
    \]
    the integral diverges, and $\frac{\sin(x)}{x}$ is not in
    $L^1((0,\infty),\lambda^1)$.
\end{proof}

\newpage

\section*{Problem 2}
Prove that $L^p$ for $1\leq p\leq\infty$ is complete. Note that case $p=1$ has
already been covered in class.
\\
\begin{proof}
To begin with, let $p<\infty$. For this proof, we will show that an absolutely
    convergent series will converge to a function in $L^p$. So, let $\{f_n\}$ be
    such that
    \[
        \sum_{n=1}^{\infty}\|f_n\|_p = M<\infty
    \]
    So, let $g_n = \sum_{k=1}^n |f_k|$. We know by Minkowski's inequality that
    \[
        \|g_n\|_p \leq \sum_{k=1}^{n}\|f_n\|_p \leq M
    \]
    Now, $g_n$ converges pointwise to some $g$, since it's a monotonically
    increasing bounded sequence.

    Now, we know that
    \[
        \int g_n^p \leq M^p
    \]
    and by Fatou's lemma,
    \[
        \int g^p \leq M^p
    \]
    as well.

    Thus, the series
    \[
        \sum_{n=1}^{\infty}|f_n(x)|
    \]
    converges pointwise for $\mu$-almost every $x$, so the series
    \[
        \sum_{n=1}^{\infty}f_n
    \]
    converges pointwise as well. Call the limit $f$.
    
    Now, let $s_n(x)$ denote the $n^{th}$ partial sum of
    $\sum_{n=1}^{\infty}f_n(x)$. We know that $|s_n(x)| \leq g_n(x)$ and
    thus $|f|(x)\leq g(x)$.

    So, by the dominated convergence theorem, we have that
    \[
    \lim \|s_n\|_p =\|\lim s_n\|_p = \|f\|_p
    \]
    as desired.

    Now, suppose $p=\infty$. Let $\{f_n\}$ be Cauchy in $L^{\infty}$. Then, we
    know that for $\mu$-almost every $x$, the sequence $\{f_n(x)\}$ is Cauchy as well, and by
    the completeness of $\R$, has a limit. Call this pointwise limit $f(x)$.

    Now, we know that $\forall \varepsilon >0$, we have some $N_{\varepsilon}$
    such that $\forall x$ ($\mu$-almost all $x$),
    \[
        |f_k(x) - f_j(x)| < \varepsilon
    \]
        for $j,k > N_{\varepsilon}$. Thus, taking a limit as $j$ goes to
        infinity, we have that
        \[
            |f_k(x)-f(x)| < \varepsilon
        \]
        for all $x$, as desired.
\end{proof}

\section*{Problem 3}
\subsection*{Part 1: Notes 3.11}
Prove that $\ell^p_n$, $\ell^p$, $1\leq p\leq \infty$ are Banach. Prove that
$\ell^{p_1}\subset \ell^{p_2}$ for $1\leq p_1\leq p_2\leq \infty$, and that
\[
    \|x\|_{p_2}\leq \|x\|_{p_1}
\]

\begin{proof}
    We first note that $\ell^p_n$ is isomorphic (as vector spaces) with $\R^n$,
    by the canonical identification $(x^i)\mapsto (x^i)$ (where $x^i$ is the
    $i^{th}$ point in the sequence, and the $i^{th}$ component of the vector).
    Furthermore, since all norms on a finite dimensional vector space are
    equivalent, the $\ell^p$ norm applied to $\ell^p_n\cong \R^n$ is equivalent
    to the standard 2-norm on $\R^n$. Now, since $\R^n$ is complete with this
    norm, it follows that $\ell^p_n$ is complete as well.

    For the case of $\ell^p$, we note that $\ell^p = L^p(\N,2^{\N},\mu_c)$. By
    Problem 2, we know that $L^p$ spaces are complete, so it follows that
    $\ell^p$ is complete as well.

    Now, we will prove the norm inequality. Without loss of generality, we will
    let $(x_n)\in\ell^{p_1}$ such that $\|(x_n)\|_{p_1} = 1$ (i.e. scale the
    sequence by its norm, which will not change the inequality).

    Now, we wish to show that
    \[
        \left(\sum_{n=1}^{\infty}|x_n|^{p_2}\right)^{\frac{1}{p_2}} \leq
        \left(\sum_{n=1}^{\infty}|x_n|^{p_1}\right)^{\frac{1}{p_1}} (=1)
    \]

    We observe first that since $\|(x_n\|_{p_1} = 1$ and $p_1\geq 1$, it must be
    that each $x_n$ is less than $1$. Thus, we have that for each $n$,
    \[
        |x_n|^{p_2}\leq |x_n|^{p_1}
    \]
    since $p_2 \geq p_1$, and each term $x_n < 1$.

    Thus, we have that
    \[
        \begin{aligned}
            \sum_{n=1}^{\infty}|x_n|^{p_2} &\leq
            \sum_{n=1}^{\infty}|x_n|^{p_1}\\
            &= 1\\
            \implies 
        \left(\sum_{n=1}^{\infty}|x_n|^{p_2}\right)^{\frac{1}{p_2}} \leq 1
        \end{aligned}
    \]
    as desired.

    Thus, if $(x_n)\in\ell^{p_1}$, we have that
    $\|(x_n)\|_{p_2}\leq\|(x_n)\|_{p_1} < \infty$, and so $(x_n)$ is in
    $\ell^{p_2}$ as well. Thus, $\ell^{p_1}\subset\ell^{p_2}$ as desired.
\end{proof}

\subsection*{Part 2: Notes 3.13}
Prove that $L^{p_1}(\Omega,\mu)\subset L^{p_2}(\Omega,\mu)$ when
$\mu(\Omega)<\infty$. To do so, establish the inequality for the average
integral
\[
    \|f\|_{\bar{p_1}} \leq \|f\|_{\bar{p_2}}
\]
where the barred norm is the average norm defined in the notes.

Furthermore, prove that this does not hold in the case $\mu(\Omega)=\infty$.
\\
\begin{proof}
We first prove the average norm inequality.

    \[
    \begin{aligned}
        \int |u|^{p_1}\frac{1}{\mu(\Omega)}d\mu &\leq \left(\int
        (|u|^{p_1})^{\frac{p_2}{p_1}}d\mu\right)^{p_1}{p_2}
        \left(\int
        (\frac{1}{\mu(\Omega)})^{\frac{p_2}{p_2-p_1}}d\mu\right)^{\frac{p_2-p_1}{p_2}}\\
        &=\|u\|_{p_2}^{p_1}((\frac{1}{\mu(\Omega)})^{\frac{p_2}{p_2-p_1}}\mu(\Omega))^{\frac{p_2-p_1}{p_2}}\\
        \implies (\int |u|^{p_1}\frac{1}{\mu(\Omega)}d\mu)^{\frac{1}{p}}
        &\leq
        \|u\|_{p_2}(\frac{1}{\mu(\Omega)}\mu(\Omega)^{1-\frac{p_1}{p_2}})^{\frac{1}{p_1}}\\
        &= \|u\|_{p_2}(\frac{1}{\mu(\Omega)})^{\frac{1}{p_2}}
    \end{aligned}
    \]
    as desired.

    Now, it follows immediately from this inequality that if $\|f\|_{p_2}
    <\infty$, it must be that $\|f\|_{p_1} < \infty$ as well, and thus $f\in
    L^{p_2}$ implies $f\in L^{p_1}$.

    However, for the measure space $\Omega = (0,\infty)$ with the Lebesgue
    measure, this fails. This is clear by considering the function $f(x) =
    \frac{1}{x^{\frac{1}{p_1}}}$, which is in $L^{p_2}$ since
    $\frac{1}{x^{\frac{p_2}{p_1}}}$ is integrable ($\frac{p_2}{p_1} > 1$), but
    $\frac{1}{x}$ is not.
\end{proof}

\subsection*{Part 3: Notes 3.15}
Prove that $\textrm{ess sup}f = \inf\{C : f \leq C \mu\textrm{-a.e.}$.
Prove that $\inf$ can be replaced with $\min$.
Prove that $\textrm{ess sup}|f| = M \iff |f|\leq M \mu\textrm{-a.e.}, \mu\{f
>M-\varepsilon\}\neq 0 \forall\varepsilon>0$.
\\
\begin{proof}
    The first statement to be proved follows immediately from the definitions,
    since the set definition is just a restatement of $\mu\{f>C\} = 0$. Thus,
    the sets the $\inf$ is taken over are the same, so the $\inf$ itself is the
    same.

    The second statement follows immediately, since if $M=\inf\{C:
    \mu\{f>C\}=0\}$, then there is some monotonically decreasing sequence $C_n$ in the set that
    approaches $M$. Now, we have that $\{f>M\} = \lim \{f>C_n\}$ which is a
    monotonic limit of sets, which the measure preserves. So,
    \[
        \mu\{f>M\} = \lim \mu\{f>C_n\} = \lim 0 = 0
    \]
    and thus $M$ is in the set the $\inf$ is taken from.]

    The last statement also follows immediately, since the first property states
    that $M$ is an essential upper bound for $|f|$, and the second property says
    that $M$ is the least of such upper bounds. That is, $M$ is the $\inf$ of
    all upper bounds on $|f|$. This characterizes the essential supremum, so the
    definitions are equivalent.
\end{proof}

\subsection*{Part 4: Notes 3.17}
For a sequence $\{f_n\}$ of measurable functions, prove that
$\|f_j-f\|_{\infty}\to0$ if and only if $f_j\to f$ uniformly outside a set of
measure zero.
\\
\begin{proof}
We know that $\forall n>0$, there is some $A_n$ such that $\forall m>A_n$,
    \[
        \|f_m-f\|<\frac{1}{n}
    \]
    or,
    \[
        \mu\{|f_m-f|>\frac{1}{n}\} = 0
    \]
    Define $E_{mn} = \{|f_m-f|>\frac{1}{n}\}$. Now, we know that
    \[
        \sup|f_m-f| <\frac{1}{n}
    \]
    on $\Omega\setminus \bigcup_{m=A_n}^{\infty}E_{mn}$.

    Thus, since $\mu (\bigcup_{n=1}^{\infty}\bigcup_{m=A_n}^{\infty}E_{mn}) =0$,
    we know that
    \[
        \sup|f_n-f|\to 0
    \]
    on $\Omega\setminus \bigcup_{n=1}^{\infty}\bigcup_{m=A_n}^{\infty}E_{mn}$,
    as desired.

    Now, for the other direction, we know that $\sup|f_n-f|\to 0$ on a $\Omega$
    minus a set $E$ of measure zero, so change $f_n$ and $f$ to be zero on $E$,
    and thus
    \[
        \sup|[f_n]-[f]| \to 0
    \]
    on all of $\Omega$, and thus $\|f_j-f\|_{\infty}\to 0$ as desired.
\end{proof}

\subsection*{Part 5: Notes 3.19}
Prove Holder and Minkowski inequalities. Prove that $L^{p_2}\subset L^{p_1}$ for
finite measure spaces. Prove this does not hold for infinite measure spaces.
\\
\begin{proof}
We begin by proving the lemma:
\begin{lemma}
For real numbers $a,b$,
    \[
        ab \leq \frac{a^p}{p} + \frac{b^{q}}{q}
    \]
    for $p,q > 1$.
\end{lemma}
    \begin{proof}
    We know that
        \[
            \begin{aligned}
                ab &= \exp(\log a + \log b)\\
                    &=\exp\left(\frac{1}{p}\log a^p + \frac{1}{q}\log b^q\right)
            \end{aligned}
        \]
        Now, by the convexity of the exponential, we have that 
        \[
            \exp\left(\frac{1}{p}\log a^p + \frac{1}{q}\log b^q\right) \leq
            \frac{1}{p}\exp(\log a^p) \frac{1}{q}\exp(\log b^q)
            = \frac{a^p}{p} + \frac{b^q}{q}
        \]
        as desired.
    \end{proof}

    Now, we are ready to prove Holder's inequality.

    Let $f,g\in L^p$, and without loss of generality let $\|f\|_p = \|g\|_q =
    1$. Now, we have that 
    \[
    \begin{aligned}
        \int |fg|d\mu &\leq \int |\frac{f^p}{p}+\frac{g^q}{q}|d\mu\\
                    &\leq \frac{1}{p}\int |f^p|d\mu + \frac{1}{q}\int
                    |g|^qd\mu\\
                    &= \frac{1}{p} + \frac{1}{q}\\
                    &= 1\\
                    &= \|f\|_p \|g\|_q
    \end{aligned}
    \]
    as desired.

    Now, for the $L^{\infty}$ case, we have that
    \[
        \int |fg|d\mu \leq \|g\|_{\infty}\int |f|d\mu = \|g\|_{\infty}\|f\|_1
    \]
    as desired.

    Now, we prove the Minkowski inequality. To do so, we will use the Holder
    inequality.
    \[
        \begin{aligned}
            \int |f+g|^pd\mu &= \int |f+g|^{p-1}|f+g|d\mu\\
            &\leq (\int (|f+g|^{p-1})^{\frac{p}{p-1}}d\mu)^{\frac{p-1}{p}}(\int
            (|f|+|g|)^pd\mu)^{\frac{1}{p}}\\
            &\leq \|f+g\|_{p}^{p-1}(\|f\|_p+\|g\|_p)
        \end{aligned}
    \]
    so we have that
    \[
        \|f+g\|_p^p \leq \|f+g\|_p^{p-1}(\|f\|_p+\|g\|_p)
    \]
    and thus
    \[
        \|f+g\|_p \leq \|f\|_p + \|g\|_p
    \]
    as desired.

    The case $p=1$ falls out of the triangle inequality, and the linearity of
    the integral, and the case $p=\infty$ comes from the subadditivity of the
    supremum.

    The second half of this problem is a restatement of Notes 3.13, which was
    proved earlier.
\end{proof}

\end{document}
