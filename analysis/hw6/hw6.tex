%%%%%%%%%%%%%%%%%%%%%%%%%%%%%%%%%%%%%%%%%
% Short Sectioned Assignment
% LaTeX Template
% Version 1.0 (5/5/12)
%
% This template has been downloaded from:
% http://www.LaTeXTemplates.com
%
% Original author:
% Frits Wenneker (http://www.howtotex.com)
%
% License:
% CC BY-NC-SA 3.0 (http://creativecommons.org/licenses/by-nc-sa/3.0/)
%
%%%%%%%%%%%%%%%%%%%%%%%%%%%%%%%%%%%%%%%%%

%----------------------------------------------------------------------------------------
%	PACKAGES AND OTHER DOCUMENT CONFIGURATIONS
%----------------------------------------------------------------------------------------

\documentclass[fontsize=11pt]{scrartcl} % 11pt font size

\usepackage[T1]{fontenc} % Use 8-bit encoding that has 256 glyphs
\usepackage[english]{babel} % English language/hyphenation
\usepackage{amsmath,amsfonts,amsthm} % Math packages
\usepackage{mathrsfs}

\usepackage[margin=1in]{geometry}

\usepackage{sectsty} % Allows customizing section commands
\allsectionsfont{\centering \normalfont\scshape} % Make all sections centered, the default font and small caps

\usepackage{fancyhdr} % Custom headers and footers
\pagestyle{fancyplain} % Makes all pages in the document conform to the custom headers and footers
\fancyhead{} % No page header - if you want one, create it in the same way as the footers below
\fancyfoot[L]{} % Empty left footer
\fancyfoot[C]{} % Empty center footer
\fancyfoot[R]{\thepage} % Page numbering for right footer
\renewcommand{\headrulewidth}{0pt} % Remove header underlines
\renewcommand{\footrulewidth}{0pt} % Remove footer underlines
\setlength{\headheight}{13.6pt} % Customize the height of the header

\numberwithin{equation}{section} % Number equations within sections (i.e. 1.1, 1.2, 2.1, 2.2 instead of 1, 2, 3, 4)
\numberwithin{figure}{section} % Number figures within sections (i.e. 1.1, 1.2, 2.1, 2.2 instead of 1, 2, 3, 4)
\numberwithin{table}{section} % Number tables within sections (i.e. 1.1, 1.2, 2.1, 2.2 instead of 1, 2, 3, 4)

\newcommand{\R}{\mathbb{R}}
\newcommand{\Q}{\mathbb{Q}}
\newcommand{\N}{\mathbb{N}}
\newcommand{\C}{\mathbb{C}}

\newtheorem{lemma}{Lemma}
%----------------------------------------------------------------------------------------
%	TITLE SECTION
%----------------------------------------------------------------------------------------

\newcommand{\horrule}[1]{\rule{\linewidth}{#1}} % Create horizontal rule command with 1 argument of height

\title{	
\normalfont \normalsize 
\textsc{Analysis} \\ [25pt] % Your university, school and/or department name(s)
\horrule{0.5pt} \\[0.4cm] % Thin top horizontal rule
\huge Problem Set 6\\ % The assignment title
\horrule{2pt} \\[0.5cm] % Thick bottom horizontal rule
}

\author{Daniel Halmrast} % Your name

\date{\normalsize\today} % Today's date or a custom date

\begin{document}

\maketitle % Print the title

%----------------------------------------------------------------------------------------
%	PROBLEM 1
%----------------------------------------------------------------------------------------
\section*{Problem 1}
Prove that in a normed space, a sequence can have at least one strong limit.
Prove that a strongly convergent sequence is Cauchy.
\\
\begin{proof}
    To begin with, we note that every normed space is necessarily Hausdorff,
    and that every Hausdorff space has the property that sequences have at most
    one limit. 

    To see that a normed space is Hausdorff, consider an arbitrary normed space
    $X$, and two distinct points $x,y$. Now, by the definition of a norm, it
    must be that $||x-y||>0$. So, let
    \[
        ||x-y||=\varepsilon
    \]
    Then, the open sets $B(x,\frac{\varepsilon}{2})$ and
    $B(Y,\frac{\varepsilon}{2})$ separate $x$ and $y$. Thus, normed spaces are
    Hausdorff.

    Furthermore, it is clear that a sequence converges to at most one limit in
    Hausdorff spaces. To see this, we recall the topological definition of
    convergence, which states that a sequence $(x_n)$ converges to $x$ if and
    only if every neighborhood of $x$ eventually contains the sequence.

    Suppose for a contradiction that $(x_n)$ had two limits, $x$, and $y$.
    Since $x$ and $y$ are distinct points in a Hausdorff space, there must be
    some neighborhood $V_x$ of $x$ and $V_y$ of $y$ such that $V_x\cap
    V_y=\emptyset$.

    However, this contradicts the sequence converging to both $x$ and $y$, since
    for $(x_n)$ to converge to $x$, it must eventually be in $V_x$, which means
    it is eventually outside $V_y$, and thus cannot converge to $y$ as well.

    Now, let $(x_n)$ be a convergent sequence in a normed space $X$, and let $x$
    be its limit. We will show that this sequence is Cauchy.

    To do this, let $\varepsilon>0$ be arbitrary. Now, we know that there exists
    some $N$ such that for any $n>N$,
    \[
        ||x_n - x||<\frac{\varepsilon}{2}
    \]
    
    Furthermore, for any $m,n>N$ we have that
    \[
        \begin{aligned}
            ||x_m-x_n|| &= ||x_m-x+x-x_n||\\
                        &\leq ||x_m-x|| + ||x_n-x||\\
                        &\leq \frac{\varepsilon}{2}+\frac{\varepsilon}{2}\\
                        &=\varepsilon
        \end{aligned}
    \]
    and thus the sequence is Cauchy.
\end{proof}
%----------------------------------------------------------------------------------------
\newpage
%----------------------------------------------------------------------------------------
%	PROBLEM 2
%----------------------------------------------------------------------------------------
\section*{Problem 2}
Show that the closure of the ball $B(a,r)$ is the closed ball $\overline{B}(a,r)=\{x\ |\ |x-a|\leq r\}$
\\
\begin{proof}
Suppose $x$ is in $\overline{B}(a,r)$. Then, consider the sequence
\[
(x_n) = (x-a)(1-\frac{1}{n}) + a
\]
Clearly, this sequence converges to $x$, and each term is in $B(a,r)$.
To see this, we observe that
\[
\begin{aligned}
||x_n-a|| &= ||(x-a)(1-\frac{1}{n}) + a - a||\\
          &= ||x-a||(1-\frac{1}{n}\\
            &\leq r(1-\frac{1}{n}\\
            &\leq r
\end{aligned}
\]
Thus, each point in $\overline{B}(a,r)$ is a limit point of $B(a,r)$, and since
$\overline{B}(a,r)$ is closed, it follows that it is the closure of $B(a,r)$ (since the 
closure of $B(a,r)$ is the smallest closed set containing it.)
\end{proof}

%----------------------------------------------------------------------------------------
\newpage
%----------------------------------------------------------------------------------------
%	PROBLEM 3
%----------------------------------------------------------------------------------------
\section*{Problem 3}
Prove that for a linear operator $A$ between normed spaces $V,W$, the following
are equivalent:
\begin{enumerate}
    \item $A$ is continuous at every $p\in V$.
    \item $A$ is continuous at $0_V$.
    \item $A$ is bounded in the sense that 
        \[
            \sup_{||x||=1} ||Ax|| < \infty
        \]
    \item $A$ is bounded in the sense that for some $M>0$,
        \[
            ||Ax||\leq M||x||
        \]
        for all $x\in V$.
\end{enumerate}
Furthermore, prove that the set $\mathscr{B}(V,W)$ of the bounded linear
operators from $V$ to $W$ is a normed space.
\\
\begin{proof}
    ($1\implies 2$)
    This follows immediately from the statement of $1$.

    ($2\implies 1$)
    Suppose $A$ is continuous at zero. Now, let $p\in V$. We will show that $A$
    is continuous at $p$. So, let $\varepsilon>0$ be arbitrary.
    Since $A$ is continuous at zero, there is some $\delta>0$ such that
    \[
        ||x||<\delta\implies||Ax||<\frac{\varepsilon}{2}
    \]
    Now, suppose $x$ is such that
    \[
    ||x-p||<\delta
    \]
    Then, we have that
    \[
        \begin{aligned}
        \end{aligned}
    \]
\end{proof}

%----------------------------------------------------------------------------------------
%----------------------------------------------------------------------------------------
%	PROBLEM 4
%----------------------------------------------------------------------------------------

%----------------------------------------------------------------------------------------
%----------------------------------------------------------------------------------------
%	PROBLEM 5
%----------------------------------------------------------------------------------------

%----------------------------------------------------------------------------------------
%----------------------------------------------------------------------------------------
%	PROBLEM 6
%----------------------------------------------------------------------------------------

%----------------------------------------------------------------------------------------
%----------------------------------------------------------------------------------------
%	PROBLEM 7
%----------------------------------------------------------------------------------------

%----------------------------------------------------------------------------------------
%----------------------------------------------------------------------------------------
%	PROBLEM 8
%----------------------------------------------------------------------------------------

%----------------------------------------------------------------------------------------
%----------------------------------------------------------------------------------------
%	PROBLEM 9
%----------------------------------------------------------------------------------------

%----------------------------------------------------------------------------------------
%----------------------------------------------------------------------------------------
%	PROBLEM 10
%----------------------------------------------------------------------------------------
\section*{Problem 10}
Show that $\ell^1$ is not complete in the $\ell^{\infty}$ norm.
\\
\begin{proof}
We will show that the sequence $(x_m)_k = (\frac{1}{m^{1+\frac{1}{k}}})$ converges to
the function $x_m = \frac{1}{m}$, which is not in $\ell^1$, even though each term in the
sequence is in $\ell^1$.

To show convergence, we wish to show that the functions
\[
f_k(n) = \frac{1}{n^{1+\frac{1}{k}}} 
\]
converges uniformly to $f(n)=\frac{1}{n}$.

To do so, we will consider instead the extended functions
\[
f_k(x) = x^{1+\frac{1}{k}},\ x\in[0,1]
\]
which clearly converges pointwise to $f(x) = x$. Now, since $f$ is defined on a compact
domain, it must also uniformly converge to $f(x) = x$. Thus, the restriction $f_k(x)|_{\{\frac{1}{n}\}}$
also converges uniformly to $f(n)=\frac{1}{n}$.

Thus,the sequence of sequences $(x_m)_k$ converges uniformly (in $\ell^{\infty}$) to $(x_m)$ as
desired. Furthermore, since each $(x_m)_k$ is in $\ell^1$, but $(x_m)$ is not, it follows that
$\ell^1$ is not complete in the $\ell^{\infty}$ norm.
\end{proof}

%----------------------------------------------------------------------------------------

\end{document}
