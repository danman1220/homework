%%%%%%%%%%%%%%%%%%%%%%%%%%%%%%%%%%%%%%%%%
% Short Sectioned Assignment
% LaTeX Template
% Version 1.0 (5/5/12)
%
% This template has been downloaded from:
% http://www.LaTeXTemplates.com
%
% Original author:
% Frits Wenneker (http://www.howtotex.com)
%
% License:
% CC BY-NC-SA 3.0 (http://creativecommons.org/licenses/by-nc-sa/3.0/)
%
%%%%%%%%%%%%%%%%%%%%%%%%%%%%%%%%%%%%%%%%%

%----------------------------------------------------------------------------------------
%	PACKAGES AND OTHER DOCUMENT CONFIGURATIONS
%----------------------------------------------------------------------------------------

\documentclass[fontsize=11pt]{scrartcl} % 11pt font size

\usepackage[T1]{fontenc} % Use 8-bit encoding that has 256 glyphs
\usepackage[english]{babel} % English language/hyphenation
\usepackage{amsmath,amsfonts,amsthm} % Math packages
\usepackage{mathrsfs}

\usepackage[margin=1in]{geometry}

\usepackage{sectsty} % Allows customizing section commands
\allsectionsfont{\centering \normalfont\scshape} % Make all sections centered, the default font and small caps

\usepackage{fancyhdr} % Custom headers and footers
\pagestyle{fancyplain} % Makes all pages in the document conform to the custom headers and footers
\fancyhead{} % No page header - if you want one, create it in the same way as the footers below
\fancyfoot[L]{} % Empty left footer
\fancyfoot[C]{} % Empty center footer
\fancyfoot[R]{\thepage} % Page numbering for right footer
\renewcommand{\headrulewidth}{0pt} % Remove header underlines
\renewcommand{\footrulewidth}{0pt} % Remove footer underlines
\setlength{\headheight}{13.6pt} % Customize the height of the header

\numberwithin{equation}{section} % Number equations within sections (i.e. 1.1, 1.2, 2.1, 2.2 instead of 1, 2, 3, 4)
\numberwithin{figure}{section} % Number figures within sections (i.e. 1.1, 1.2, 2.1, 2.2 instead of 1, 2, 3, 4)
\numberwithin{table}{section} % Number tables within sections (i.e. 1.1, 1.2, 2.1, 2.2 instead of 1, 2, 3, 4)

\newcommand{\R}{\mathbb{R}}
\newcommand{\Q}{\mathbb{Q}}
\newcommand{\N}{\mathbb{N}}
\newcommand{\C}{\mathbb{C}}

\theoremstyle{definition}
\newtheorem{lemma}{Lemma}
%----------------------------------------------------------------------------------------
%	TITLE SECTION
%----------------------------------------------------------------------------------------

\newcommand{\horrule}[1]{\rule{\linewidth}{#1}} % Create horizontal rule command with 1 argument of height

\title{	
\normalfont \normalsize 
\textsc{Analysis} \\ [25pt] % Your university, school and/or department name(s)
\horrule{0.5pt} \\[0.4cm] % Thin top horizontal rule
\huge Problem Set 5\\ % The assignment title
\horrule{2pt} \\[0.5cm] % Thick bottom horizontal rule
}

\author{Daniel Halmrast} % Your name

\date{\normalsize\today} % Today's date or a custom date

\begin{document}

\maketitle % Print the title

%----------------------------------------------------------------------------------------
%	PROBLEM 1
%----------------------------------------------------------------------------------------
\section*{Problem 1}
Prove the linearity of the general Lebesgue integral for complex-valued functions.
\\
\begin{proof}
To begin with, we prove a sequence of lemmas concerning the linearity of easier functions.
\begin{lemma}[Linearity of Positive Real Functions]
For $f,g$ positive, real-valued measurable functions from a measure space $(\Omega,\mu)$,
the integral is linear with respect to pointwise addition and positive scalar
multiplication. That is, for $\alpha, \beta \geq 0$,
\[
\int_{\Omega} \alpha f + \beta g d\mu = \alpha\int_{\Omega} fd\mu + \beta\int_{\Omega} gd\mu
\]
\end{lemma}
\begin{proof}
This has been proven in the notes, and will not be replicated here.
\end{proof}

\begin{lemma}
For $f$ a positive real-valued measurable function from a measure space $(\Omega,\mu)$,
the integral respects general scalar multiplication. That is, for $\alpha\in\R$, 
\[
\int_{\Omega}\alpha fd\mu = \alpha\int_{\Omega} fd\mu
\]
\end{lemma}
\begin{proof}
This lemma has already been proven for $\alpha\geq 0$. So, suppose $\alpha < 0$.
In particular, $-\alpha > 0$, and thus by straightforward application of the definition of
the Lebesgue integral along with lemma 1, we have
\[
\begin{aligned}
\int_{\Omega}\alpha fd\mu   &= \int_{\Omega}-(-\alpha f)d\mu\\
                            &= -\int_{\Omega}-\alpha fd\mu\\
                            &= -(-\alpha)\int_{\Omega} fd\mu\\
                            &= \alpha\int_{\Omega} fd\mu
\end{aligned}
\]
which is the desired result.
\end{proof}

\begin{lemma}
For $f_1,f_2,g_1,g_2$ positive real-valued measurable functions from a measure space
$(\Omega,\mu)$ such that $f_1-f_2 = g_1-g_2$,
\[
\int_{\Omega} f_1d\mu - \int_{\Omega} f_2d\mu = \int_{\Omega} g_1d\mu - \int_{\Omega}g_2d\mu
\]
\end{lemma}

\begin{proof}
This follows immediately from integrating $f_1-f_2$ and $g_1-g_2$, and applying lemma 2
and lemma 1.
\end{proof}
We are now ready to begin the proof of the linearity of the Lebesgue integral.

Let $f$ and $g$ be complex-valued measurable functions from a measure space $(\Omega,\mu)$,
and let $\alpha,\beta\in\C$. We wish to compute the integral
\[
\int_{\Omega}\alpha f + \beta gd\mu
\]
Now, letting $f = u_f + iv_f$, and $g = u_g + iv_g$, we have that
\[
\begin{aligned}
\int_{\Omega}\alpha f + \beta gd\mu &= \int_{\Omega}(\Re(\alpha) +i\Im(\alpha))(u_f+iv_f) + (\Re(\beta)+i\Im(\beta))(u_g+iv_g)d\mu\\
    &=\int_{\Omega}(\Re(\alpha)u_f -\Im(\alpha)v_f + \Re(\beta)u_g -\Im(\beta)v_g) + i(\Re(\alpha)v_f +\Im(\alpha)u_f +\Re(\beta)v_g+\Im(\beta)u_g)d\mu\\
\end{aligned}
\]
\end{proof}

%----------------------------------------------------------------------------------------
%----------------------------------------------------------------------------------------
%	PROBLEM 2
%----------------------------------------------------------------------------------------

%----------------------------------------------------------------------------------------
%----------------------------------------------------------------------------------------
%	PROBLEM 3
%----------------------------------------------------------------------------------------

%----------------------------------------------------------------------------------------
%----------------------------------------------------------------------------------------
%	PROBLEM 4
%----------------------------------------------------------------------------------------

%----------------------------------------------------------------------------------------
%----------------------------------------------------------------------------------------
%	PROBLEM 5
%----------------------------------------------------------------------------------------

%----------------------------------------------------------------------------------------
%----------------------------------------------------------------------------------------
%	PROBLEM 6
%----------------------------------------------------------------------------------------

%----------------------------------------------------------------------------------------
%----------------------------------------------------------------------------------------
%	PROBLEM 7
%----------------------------------------------------------------------------------------

%----------------------------------------------------------------------------------------

\end{document}
