%%%%%%%%%%%%%%%%%%%%%%%%%%%%%%%%%%%%%%%%%
% Short Sectioned Assignment
% LaTeX Template
% Version 1.0 (5/5/12)
%
% This template has been downloaded from:
% http://www.LaTeXTemplates.com
%
% Original author:
% Frits Wenneker (http://www.howtotex.com)
%
% License:
% CC BY-NC-SA 3.0 (http://creativecommons.org/licenses/by-nc-sa/3.0/)
%
%%%%%%%%%%%%%%%%%%%%%%%%%%%%%%%%%%%%%%%%%

%----------------------------------------------------------------------------------------
%	PACKAGES AND OTHER DOCUMENT CONFIGURATIONS
%----------------------------------------------------------------------------------------

\documentclass[fontsize=11pt]{scrartcl} % 11pt font size

\usepackage[T1]{fontenc} % Use 8-bit encoding that has 256 glyphs
\usepackage[english]{babel} % English language/hyphenation
\usepackage{amsmath,amsfonts,amsthm} % Math packages
\usepackage{mathrsfs}

\usepackage[margin=1in]{geometry}

\usepackage{sectsty} % Allows customizing section commands
\allsectionsfont{\centering \normalfont\scshape} % Make all sections centered, the default font and small caps

\usepackage{fancyhdr} % Custom headers and footers
\pagestyle{fancyplain} % Makes all pages in the document conform to the custom headers and footers
\fancyhead{} % No page header - if you want one, create it in the same way as the footers below
\fancyfoot[L]{} % Empty left footer
\fancyfoot[C]{} % Empty center footer
\fancyfoot[R]{\thepage} % Page numbering for right footer
\renewcommand{\headrulewidth}{0pt} % Remove header underlines
\renewcommand{\footrulewidth}{0pt} % Remove footer underlines
\setlength{\headheight}{13.6pt} % Customize the height of the header

\numberwithin{equation}{section} % Number equations within sections (i.e. 1.1, 1.2, 2.1, 2.2 instead of 1, 2, 3, 4)
\numberwithin{figure}{section} % Number figures within sections (i.e. 1.1, 1.2, 2.1, 2.2 instead of 1, 2, 3, 4)
\numberwithin{table}{section} % Number tables within sections (i.e. 1.1, 1.2, 2.1, 2.2 instead of 1, 2, 3, 4)

\newcommand{\R}{\mathbb{R}}
\newcommand{\Q}{\mathbb{Q}}
\newcommand{\N}{\mathbb{N}}
\newcommand{\C}{\mathbb{C}}

\newtheorem{lemma}{Lemma}
%----------------------------------------------------------------------------------------
%	TITLE SECTION
%----------------------------------------------------------------------------------------

\newcommand{\horrule}[1]{\rule{\linewidth}{#1}} % Create horizontal rule command with 1 argument of height

\title{	
\normalfont \normalsize 
\textsc{Analysis} \\ [25pt] % Your university, school and/or department name(s)
\horrule{0.5pt} \\[0.4cm] % Thin top horizontal rule
\huge Final Exam \\ % The assignment title
\horrule{2pt} \\[0.5cm] % Thick bottom horizontal rule
}

\author{Daniel Halmrast} % Your name

\date{\normalsize\today} % Today's date or a custom date

\begin{document}

\maketitle % Print the title

\section*{Problem 1}
For every $n\in\N$, let $\mu_n$ be a measure on $(\Omega,\mathscr{A})$ with
$\mu_n(\Omega)=1$. For every $E\in\mathscr{A}$, define
\[
    \mu(E) = \sum_{n=1}^{\infty}\frac{\mu_n(E)}{2^n}
\]
Give a careful proof that $\mu$ is a measure on $(\omega,\mathscr{A})$ with
$\mu(\Omega)=1$.
\\
\begin{proof}
    We wish to prove that $\mu$ is a measure on $(\Omega,\mathscr{A})$. That is,
    we wish to show that that $\mu(\emptyset) = 0$, that $\mu(E)\geq 0$ for all
    $E\in\mathscr{A}$, and that for a countable collection of disjoint sets
    $\{E_j\}_{j=1}^{\infty}$ for which $E_j\in\mathscr{A}$ for all $j$,
    \[
        \mu\left(\bigcup_{j=1}^{\infty}E_j\right) = \sum_{j=1}^{\infty}\mu(E_j)
    \]

    To begin with, we note that since each $\mu_n$ is a measure, we have that
    $\mu_n(\emptyset) = 0$.
    Thus,
    \[
        \begin{aligned}
            \mu(\emptyset)  &= \sum_{n=1}^{\infty}\frac{\mu_n(\emptyset)}{2^n}\\
                            &= \sum_{n=1}^{\infty}\frac{0}{2^n}\\
                            &=0
        \end{aligned}
    \]
    as desired.

    Next, we note that since each $\mu_n$ is a measure, $\mu_n(E)\geq 0$ for all
    $E\in\mathscr{A}$. Thus, since both $\mu_n(E)$ and $2^n$ are greater than
    zero for each $n$, it must be that 
    \[
        \mu(E) = \sum_{n=1}^{\infty}\frac{\mu_n(E)}{2^n} \geq 0
    \]

    To show that $\mu$ is additive, let $\{E_j\}_{j=1}^{\infty}$ be a countable
    collection of disjoint measurable sets.
\end{proof}


\end{document}
