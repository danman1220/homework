%%%%%%%%%%%%%%%%%%%%%%%%%%%%%%%%%%%%%%%%%
% Short Sectioned Assignment
% LaTeX Template
% Version 1.0 (5/5/12)
%
% This template has been downloaded from:
% http://www.LaTeXTemplates.com
%
% Original author:
% Frits Wenneker (http://www.howtotex.com)
%
% License:
% CC BY-NC-SA 3.0 (http://creativecommons.org/licenses/by-nc-sa/3.0/)
%
%%%%%%%%%%%%%%%%%%%%%%%%%%%%%%%%%%%%%%%%%

%----------------------------------------------------------------------------------------
%	PACKAGES AND OTHER DOCUMENT CONFIGURATIONS
%----------------------------------------------------------------------------------------

\documentclass[fontsize=11pt]{scrartcl} % 11pt font size

\usepackage[T1]{fontenc} % Use 8-bit encoding that has 256 glyphs
\usepackage[english]{babel} % English language/hyphenation
\usepackage{amsmath,amsfonts,amsthm} % Math packages
\usepackage{mathrsfs}

\usepackage[margin=1in]{geometry}

\usepackage{sectsty} % Allows customizing section commands
\allsectionsfont{\centering \normalfont\scshape} % Make all sections centered, the default font and small caps

\usepackage{fancyhdr} % Custom headers and footers
\pagestyle{fancyplain} % Makes all pages in the document conform to the custom headers and footers
\fancyhead{} % No page header - if you want one, create it in the same way as the footers below
\fancyfoot[L]{} % Empty left footer
\fancyfoot[C]{} % Empty center footer
\fancyfoot[R]{\thepage} % Page numbering for right footer
\renewcommand{\headrulewidth}{0pt} % Remove header underlines
\renewcommand{\footrulewidth}{0pt} % Remove footer underlines
\setlength{\headheight}{13.6pt} % Customize the height of the header

\numberwithin{equation}{section} % Number equations within sections (i.e. 1.1, 1.2, 2.1, 2.2 instead of 1, 2, 3, 4)
\numberwithin{figure}{section} % Number figures within sections (i.e. 1.1, 1.2, 2.1, 2.2 instead of 1, 2, 3, 4)
\numberwithin{table}{section} % Number tables within sections (i.e. 1.1, 1.2, 2.1, 2.2 instead of 1, 2, 3, 4)

\newcommand{\R}{\mathbb{R}}
\newcommand{\Q}{\mathbb{Q}}
\newcommand{\C}{\mathbb{C}}

\newcommand{\supp}{\textrm{supp}}
%----------------------------------------------------------------------------------------
%	TITLE SECTION
%----------------------------------------------------------------------------------------

\newcommand{\horrule}[1]{\rule{\linewidth}{#1}} % Create horizontal rule command with 1 argument of height

\title{	
\normalfont \normalsize 
\textsc{Analysis} \\ [25pt] % Your university, school and/or department name(s)
\horrule{0.5pt} \\[0.4cm] % Thin top horizontal rule
\huge Problem Set 4\\ % The assignment title
\horrule{2pt} \\[0.5cm] % Thick bottom horizontal rule
}

\author{Daniel Halmrast} % Your name

\date{\normalsize\today} % Today's date or a custom date

\begin{document}

\maketitle % Print the title

%----------------------------------------------------------------------------------------
%	PROBLEM 1
%----------------------------------------------------------------------------------------
\section*{Problem 1}
For $E\subset\Omega$ measurable, prove the implication
\[
\int_E fd\mu = 0\ \forall f\geq 0 \implies \mu(E) = 0
\]

\begin{proof}
This follows immediately by letting $f = \chi_E$, and observing that
\[
\begin{aligned}
\int_E \chi_E(x)d\mu(x) &=\int_{\Omega}\chi_E(x)\chi_E(x)d\mu(x)\\
                        &=\int_{\Omega}\chi_E(x)d\mu(x)\\
                        &=\mu(E)
\end{aligned}
\]
Which is zero by the hypothesis. Thus, $\mu(E) = 0$ as desired.
\end{proof}
%----------------------------------------------------------------------------------------
\pagebreak
%----------------------------------------------------------------------------------------
%	PROBLEM 2
%----------------------------------------------------------------------------------------
\section*{Problem 2}
For $f\geq 0$ measurable on $\Omega$ with $\mu(\Omega) > 0$, show that
\[
\int_{\Omega}f(x)d\mu(x) = 0 \implies [f] = 0
\]
(i.e. $f$ is zero $\mu$-almost everywhere).
\\
\begin{proof}
Consider the equivalence class $[f]$ in $L^1(\Omega,\mu)$. In particular,
since $\int_{\Omega}|f|d\mu =||f||_1 = 0$, we must have that $[f]=0$, which means
$f$ agrees with the $0$ function $\mu$-almost everywhere. It follows immediately, then,
that $f$ is zero $\mu$-almost everywhere.
\end{proof}
%----------------------------------------------------------------------------------------
%----------------------------------------------------------------------------------------
%	PROBLEM 3
%----------------------------------------------------------------------------------------
\section*{Problem 3}
Use Fatou's lemma to show that for a sequence $\{f_n\}$ of positive measurable functions,
the inequality
\[
\int_{\Omega}\liminf f_nd\mu \leq \liminf\int_{\Omega} f_nd\mu
\]

\begin{proof}
We note first that the inequality is vacuously true if $\liminf\int_{\Omega}f_nd\mu = \infty$.

So, assume that $\liminf\int_{\Omega}f_nd\mu = M$ for some positive number $M$. Then,
consider the family of subsequences
\[
\{f_{n_i}\}_{\epsilon} = \{f_n\ |\ \int_{\Omega} f_nd\mu < M+\epsilon\}
\]
Now, for any $\epsilon$, this defines an infinite subsequence, since if the integrals of
the sequence were not frequently below $M+\epsilon$, then $M+\epsilon$ would be an
eventual lower bound higher than $M$, which contradicts $M$ being the $\liminf$ of the
integrals.

Now, we apply Fatou's lemma by observing that for each $f_{n_i}$ we have that
\[
\int_{\Omega} f_{n_i}d\mu < M+\epsilon
\]
which gives us the upper bound
\[
\int_{\Omega}\liminf f_{n_i}d\mu \leq M+\epsilon
\]
Now, a basic property of the $\liminf$ is that for a sequence $x_n$ with a subsequence
$x_{n_i}$,
\[
\liminf x_n \leq \liminf x_{n_i}
\]
Thus, we also have that
\[
\int_{\Omega} \liminf f_nd\mu \leq \int_{\Omega}\liminf f_{n_i} d\mu\leq M+\epsilon
\]
However, since this is true for all $\epsilon > 0$, it must be that
\[
\int_{\Omega} \liminf f_nd\mu \leq \int_{\Omega}\liminf f_{n_i} d\mu\leq M
\]
And by the definition of $M$, we have the desired inequality
\[
\int_{\Omega} \liminf f_nd\mu \leq M = \liminf\int_{\Omega}f_nd\mu
\]

\end{proof}

%----------------------------------------------------------------------------------------
\pagebreak
%----------------------------------------------------------------------------------------
%	PROBLEM 4
%----------------------------------------------------------------------------------------
\section*{Problem 4}
\subsection*{Notes 2-13}
Prove that $f\in L^1\implies |f|<\infty$ $\mu$-almost everywhere, and describe the 
spaces $L^1(\N,\mu_c)$ and $L^1(\Omega,\delta_p$.
\\
\begin{proof}

\end{proof}
\subsection*{Notes 2-12}
%----------------------------------------------------------------------------------------
%----------------------------------------------------------------------------------------
%	PROBLEM 5
%----------------------------------------------------------------------------------------

%----------------------------------------------------------------------------------------
\end{document}
