%%%%%%%%%%%%%%%%%%%%%%%%%%%%%%%%%%%%%%%%%
% Short Sectioned Assignment
% LaTeX Template
% Version 1.0 (5/5/12)
%
% This template has been downloaded from:
% http://www.LaTeXTemplates.com
%
% Original author:
% Frits Wenneker (http://www.howtotex.com)
%
% License:
% CC BY-NC-SA 3.0 (http://creativecommons.org/licenses/by-nc-sa/3.0/)
%
%%%%%%%%%%%%%%%%%%%%%%%%%%%%%%%%%%%%%%%%%

%----------------------------------------------------------------------------------------
%	PACKAGES AND OTHER DOCUMENT CONFIGURATIONS
%----------------------------------------------------------------------------------------

\documentclass[fontsize=11pt]{scrartcl} % 11pt font size

\usepackage[T1]{fontenc} % Use 8-bit encoding that has 256 glyphs
\usepackage[english]{babel} % English language/hyphenation
\usepackage{amsmath,amsfonts,amsthm} % Math packages
\usepackage{mathrsfs}

\usepackage[margin=1in]{geometry}

\usepackage{sectsty} % Allows customizing section commands
\allsectionsfont{\centering \normalfont\scshape} % Make all sections centered, the default font and small caps

\usepackage{fancyhdr} % Custom headers and footers
\pagestyle{fancyplain} % Makes all pages in the document conform to the custom headers and footers
\fancyhead{} % No page header - if you want one, create it in the same way as the footers below
\fancyfoot[L]{} % Empty left footer
\fancyfoot[C]{} % Empty center footer
\fancyfoot[R]{\thepage} % Page numbering for right footer
\renewcommand{\headrulewidth}{0pt} % Remove header underlines
\renewcommand{\footrulewidth}{0pt} % Remove footer underlines
\setlength{\headheight}{13.6pt} % Customize the height of the header

\numberwithin{equation}{section} % Number equations within sections (i.e. 1.1, 1.2, 2.1, 2.2 instead of 1, 2, 3, 4)
\numberwithin{figure}{section} % Number figures within sections (i.e. 1.1, 1.2, 2.1, 2.2 instead of 1, 2, 3, 4)
\numberwithin{table}{section} % Number tables within sections (i.e. 1.1, 1.2, 2.1, 2.2 instead of 1, 2, 3, 4)

\newcommand{\R}{\mathbb{R}}
\newcommand{\Q}{\mathbb{Q}}
\newcommand{\N}{\mathbb{N}}
\newcommand{\C}{\mathbb{C}}

\newtheorem{lemma}{Lemma}
%----------------------------------------------------------------------------------------
%	TITLE SECTION
%----------------------------------------------------------------------------------------

\newcommand{\horrule}[1]{\rule{\linewidth}{#1}} % Create horizontal rule command with 1 argument of height

\title{	
\normalfont \normalsize 
\textsc{Geometry} \\ [25pt] % Your university, school and/or department name(s)
\horrule{0.5pt} \\[0.4cm] % Thin top horizontal rule
\huge Problem Set 5 \\ % The assignment title
\horrule{2pt} \\[0.5cm] % Thick bottom horizontal rule
}

\author{Daniel Halmrast} % Your name

\date{\normalsize\today} % Today's date or a custom date

\begin{document}

\maketitle % Print the title


\section*{Problem 1} %Lee 8-10
Let $M$ be the open submanifold of $\R^2$ with both coordinates positive, and
define $F:M\to M$ as $F(x,y) = (xy,\frac{y}{x})$. Show that $F$ is a
diffeomorphism, and compute $dFX$ and $dFY$ for
\[
    \begin{aligned}
        X &= x\partial_x + y\partial_y\\
        Y &= y\partial_x
    \end{aligned}
\]

\begin{proof}
To begin with, we observe that so long as $x\neq 0$, $F$ is in fact smooth.
    Furthermore, it has an inverse
    \[
        F^{-1}(x,y) = (\sqrt{\frac{x}{y}},\sqrt{xy})
    \]
    which is also smooth on $M$, and defined for all of $M$.
    
    Furthermore, we can calculate its Jacobian:
    \[
        J(F) = dF =
        \begin{bmatrix}
            y & x\\
            \frac{-y}{x^2} & \frac{1}{x}
        \end{bmatrix}
    \]
    which expresses $dF$ in the coordinates $\partial_x,\partial_y$ at every
    point.

    Now, we calculate $dF(X)$:
    \[
        \begin{aligned}
            dF(X) &= dF(x\partial_x + y\partial_y)\\
                &= xdF(\partial_x) + ydF(\partial_y)\\
                &= x(y\partial_x - \frac{y}{x^2}\partial_y) + y(x\partial_x +
                \frac{1}{x}\partial_y)\\
                &= xy\partial_x - \frac{y}{x}\partial_y + yx\partial_x
                +\frac{y}{x}\partial_y\\
                &= 2xy\partial_x
        \end{aligned}
    \]
    and $dF(Y)$:
    \[
        \begin{aligned}
            dF(Y) &= dF(x\partial_y)\\
                  &= xdF(\partial_y)\\
                  &= x(x\partial_x + \frac{1}{x}\partial_y)\\
                  &=x^2\partial_x + \partial_y
        \end{aligned}
    \]
\end{proof}

\section*{Problem 2} %Lee 8-15
Let $M$ be a smooth manifold, $S\subseteq M$ an embedded submanifold. Given
$X\in\mathcal{X}(S)$, show that there is a smooth vector field $Y$ on a
neighborhood of $S$ in $M$ such that $X=Y|_S$. Show that every such vector field
extends to all of $M$ if and only if $S$ is properly embedded.
\\
\begin{proof}
%TODO write this
    %lemma 5-34 for part 2
\end{proof}

\section*{Problem 3} %Lee 8-18
%TODO write this

\end{document}
