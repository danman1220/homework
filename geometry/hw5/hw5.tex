%%%%%%%%%%%%%%%%%%%%%%%%%%%%%%%%%%%%%%%%%
% Short Sectioned Assignment
% LaTeX Template
% Version 1.0 (5/5/12)
%
% This template has been downloaded from:
% http://www.LaTeXTemplates.com
%
% Original author:
% Frits Wenneker (http://www.howtotex.com)
%
% License:
% CC BY-NC-SA 3.0 (http://creativecommons.org/licenses/by-nc-sa/3.0/)
%
%%%%%%%%%%%%%%%%%%%%%%%%%%%%%%%%%%%%%%%%%

%----------------------------------------------------------------------------------------
%	PACKAGES AND OTHER DOCUMENT CONFIGURATIONS
%----------------------------------------------------------------------------------------

\documentclass[fontsize=11pt]{scrartcl} % 11pt font size

\usepackage[T1]{fontenc} % Use 8-bit encoding that has 256 glyphs
\usepackage[english]{babel} % English language/hyphenation
\usepackage{amsmath,amsfonts,amsthm} % Math packages
\usepackage{mathrsfs}
\usepackage{tikz-cd}

\usepackage[margin=1in]{geometry}

\usepackage{sectsty} % Allows customizing section commands
\allsectionsfont{\centering \normalfont\scshape} % Make all sections centered, the default font and small caps

\usepackage{fancyhdr} % Custom headers and footers
\pagestyle{fancyplain} % Makes all pages in the document conform to the custom headers and footers
\fancyhead{} % No page header - if you want one, create it in the same way as the footers below
\fancyfoot[L]{} % Empty left footer
\fancyfoot[C]{} % Empty center footer
\fancyfoot[R]{\thepage} % Page numbering for right footer
\renewcommand{\headrulewidth}{0pt} % Remove header underlines
\renewcommand{\footrulewidth}{0pt} % Remove footer underlines
\setlength{\headheight}{13.6pt} % Customize the height of the header

\numberwithin{equation}{section} % Number equations within sections (i.e. 1.1, 1.2, 2.1, 2.2 instead of 1, 2, 3, 4)
\numberwithin{figure}{section} % Number figures within sections (i.e. 1.1, 1.2, 2.1, 2.2 instead of 1, 2, 3, 4)
\numberwithin{table}{section} % Number tables within sections (i.e. 1.1, 1.2, 2.1, 2.2 instead of 1, 2, 3, 4)

\newcommand{\R}{\mathbb{R}}
\newcommand{\Q}{\mathbb{Q}}
\newcommand{\N}{\mathbb{N}}
\newcommand{\C}{\mathbb{C}}

\newtheorem{lemma}{Lemma}
%----------------------------------------------------------------------------------------
%	TITLE SECTION
%----------------------------------------------------------------------------------------

\newcommand{\horrule}[1]{\rule{\linewidth}{#1}} % Create horizontal rule command with 1 argument of height

\title{	
\normalfont \normalsize 
\textsc{Geometry} \\ [25pt] % Your university, school and/or department name(s)
\horrule{0.5pt} \\[0.4cm] % Thin top horizontal rule
\huge Problem Set 5 \\ % The assignment title
\horrule{2pt} \\[0.5cm] % Thick bottom horizontal rule
}

\author{Daniel Halmrast} % Your name

\date{\normalsize\today} % Today's date or a custom date

\begin{document}

\maketitle % Print the title


\section*{Problem 1} %Lee 8-10
Let $M$ be the open submanifold of $\R^2$ with both coordinates positive, and
define $F:M\to M$ as $F(x,y) = (xy,\frac{y}{x})$. Show that $F$ is a
diffeomorphism, and compute $dFX$ and $dFY$ for
\[
    \begin{aligned}
        X &= x\partial_x + y\partial_y\\
        Y &= y\partial_x
    \end{aligned}
\]

\begin{proof}
    To begin with, we observe that so long as $x\neq 0$, $F$ is in fact smooth.
    Furthermore, it has an inverse
    \[
        F^{-1}(x,y) = (\sqrt{\frac{x}{y}},\sqrt{xy})
    \]
    which is also smooth on $M$, and defined for all of $M$.

    Furthermore, we can calculate its Jacobian:
    \[
        J(F) = dF =
        \begin{bmatrix}
            y & x\\
            \frac{-y}{x^2} & \frac{1}{x}
        \end{bmatrix}
    \]
    which expresses $dF$ in the coordinates $\partial_x,\partial_y$ at every
    point.

    Now, we calculate $dF(X)$:
    \[
        \begin{aligned}
            dF(X) &= dF(x\partial_x + y\partial_y)\\
                &= xdF(\partial_x) + ydF(\partial_y)\\
                &= x(y\partial_x - \frac{y}{x^2}\partial_y) + y(x\partial_x +
                \frac{1}{x}\partial_y)\\
                &= xy\partial_x - \frac{y}{x}\partial_y + yx\partial_x
                +\frac{y}{x}\partial_y\\
                &= 2xy\partial_x
        \end{aligned}
    \]
    and $dF(Y)$:
    \[
        \begin{aligned}
            dF(Y) &= dF(x\partial_y)\\
                  &= xdF(\partial_y)\\
                  &= x(x\partial_x + \frac{1}{x}\partial_y)\\
                  &=x^2\partial_x + \partial_y
        \end{aligned}
    \]
\end{proof}

\section*{Problem 2} %Lee 8-15
Let $M$ be a smooth manifold, $S\subseteq M$ an embedded submanifold. Given
$X\in\mathcal{X}(S)$, show that there is a smooth vector field $Y$ on a
neighborhood of $S$ in $M$ such that $X=Y|_S$. Show that every such vector field
extends to all of $M$ if and only if $S$ is properly embedded.
\\
\begin{proof}
    Recall that a vector field $X\in\mathcal(X)(S)$ is a linear derivation of
    the algebra $C^{\infty}(M)$ over $\R$. That is, $X$ is a linear map
    $X:C^{\infty}(M)\to C^{\infty}(M)$ such that
    \[
        X(fg) = X(f)g + fX(g)
    \]

    Now, from an earlier assignment, we know that for $S\subseteq M$ an embedded
    submanifold of a manifold $M$, we have the existence of extensions of
    $C^{\infty}$ functions on $S$ to $C^{\infty}$ functions on the neighborhood
    $U\subseteq M$ of $S$. That is, the restriction function $r:C^{\infty}(U)\to
    C^{\infty}(S)$ has a section $e:C^{\infty}(S)\to C^{\infty}(U)$ such that
    $r\circ e = id$. Thus, we have the following diagram:
    \[
        \begin{tikzcd}
            C^{\infty}(S)\arrow{r}{X} &C^{\infty}(S)\arrow{d}{e}\\
            C^{\infty}(U)\arrow{u}{r} &C^{\infty}(U)
        \end{tikzcd}
    \]
    Now, we define $Y$ to be the linear map $Y:C^{\infty}(U)\to C^{\infty}(U)$
    that makes the diagram commute. That is:
    \[
        \begin{tikzcd}
            C^{\infty}(S)\arrow{r}{X} &C^{\infty}(S)\arrow{d}{e}\\
            C^{\infty}(U)\arrow{u}{r}\arrow[dashed]{r}{Y} &C^{\infty}(U)
        \end{tikzcd}
    \]
    We note that such a $Y$ is not unique, since the extension $e$ is not
    uniquely defined. In fact, all extensions differ by an element of
    $C^{\infty}(U)/{C^{\infty}(S)}$.

    $Y$ is clearly linear, since it is the composition of linear arrows, so all
    that we need to show is that $Y$ is a derivation. First, we observer that
    since $e$ is a section of $r$, it follows that $X\circ r = r\circ Y$. That
    is, the diagram
    \[
        \begin{tikzcd}
            C^{\infty}(S)\arrow{r}{X} &C^{\infty}(S)\arrow[shift left]{d}{e}\\
            C^{\infty}(U)\arrow{u}{r}\arrow[dashed]{r}{Y}
            &C^{\infty}(U)\arrow[shift left]{u}{r}
        \end{tikzcd}
    \]
    commutes. We note also that the restriction map $r$ is multiplicative. That
    is, $r(fg) = r(f)r(g)$. 

    Let $f,g\in
    C^{\infty}(U)$. We calculate
    \[
        \begin{aligned}
            Y(fg) &= e\circ X\circ r(fg)\\
                    &= e\circ X(r(f)r(g))\\
                    &= e(X(r(f))r(g) + r(f)X(r(g)))\\
                    &= e(r(Y(f))r(g) +r(f)r(Y(g))\\
                    &= (e\circ r)(Y(f)g + fY(g))
        \end{aligned}
    \]
    so if $e\circ r$ is the identity on $Y(f)g + fY(g)$, then $Y$ is a
    derivation. However, we recall that $e$ is only unique up to a factor of
    $C^{\infty}(U)/{C^{\infty}(S)}$. So, for each $f\in C^{\infty}(U)$, we
    define $e_f$... %TODO finish
\end{proof}

\section*{Problem 3} %Lee 8-18
Let $F:M\to N$ be a smooth submersion.
\subsection*{Part 1}
Show that if $\dim M = \dim N$, then every smooth vector field on $N$ has a
unique lift.
\\
\\
\begin{proof}
    Let $Y\in \mathcal{X}(N)$. Then, define $X_p = dF^{-1}_{F(p)}(Y_{F(p)})$, where
    $dF^{-1}_{F(p)}$ is an isomorphism, since $F$ is a submersion to a space of
    the same dimension. Clearly, $X$ and $Y$ are $F$-related, as desired.
\end{proof}

\subsection*{Part b}
Show that if $\dim M \neq \dim N$, then the lift is not unique.
\\
\\
\begin{proof}
    Theorem 4.26 guarantees the existence of local sections of $F$, so we can
    define $X_p = d\pi_{F(p)}(Y_{F(p)})$, and clearly $X$ is a lift of $Y$,
    since $d\pi\circ dF = id$. This lift will not be unique, since $\pi$ is not
    unique.
\end{proof}

\subsection*{Part c}
Suppose $F$ is surjective. Given $X\in\mathcal{X}(M)$, show that $X$ is the lift
of a smooth vector field on $N$ if and only if $dF_p(X_p) = dF_q(X_q)$ for all
$F(p) = F(q)$.
\\
\\
\begin{proof}
    Suppose $X$ is the lift of some $Y$ on $N$. Then, we know that for $F(p) =
    F(q)$,
    \[
        dF_p(X_p) = Y_{F(p)}=Y_{F(q)}=dF_q(X_q)
    \]
    as desired.

    Suppose instead that $dF_p(X_p)=dF_q(X_q)$ for each $F(p)=F(q)$. Then, we
    define
    \[
        Y_{F(p)} = dF_p(X_p)
    \]
    Clearly, $X$ is a lift of $Y$, and $Y$ is well-defined since $F$ is
    surjective, and if $F(p)=F(q)$, then $Y_{F(p)}=Y_{F(q)}$ by the condition of
    $X$.
\end{proof}

\subsection*{Part d}
Suppose in addition that $F$ has connected fibers. Show that
$X\in\mathcal{X}(M)$ is a lift of a smooth vector field on $N$ if and only if
$[V,X]$ is vertical whenever $V$ is a vertical vector field on $M$.
\\
\\
\begin{proof}
Suppose $X$ is a lift of some vector field $Y$ in $N$. Then,
we know that $X$ and $Y$ are $F$-related, and $V$ and $0$ are $F$-related, so
    $[V,X]$ and $[0,Y]=0$ are $F$-related, and thus $[V,X]$ is vertical.

    Suppose instead that $[V,X]$ is vertical. That is $dF([V,X])=0$.
    Then,
    \[
        \begin{aligned}
            dF([V,X])(f) &= 0\\
                        &= (VX-XV)(f\circ F)\\
                        &=VX(f\circ F) - XV(f\circ F)\\
                        &=VX(f\circ F) - X(0)\\
                        &=VX(f\circ F)
        \end{aligned}
    \]
    Thus, $V(X(f\circ F)) = 0$, which means that around a point $p$, $X(f\circ
    F) = C_p$ for some constant $C_p$. Now, for points $p$ and $q$ on the same
    (connected) fiber, it must be that $C_p=C_q$, and thus $dF_p(X_p)
    =dF_q(X_q)$ and by the previous result, $X$ is a lift of some vector field
    on $N$.
\end{proof}

\section*{Problem 4} %Lee 8-19
Show that $\R^3$ is a Lie algebra with the cross product.
\\
\begin{proof}
    The cross product is, by definition, bilinear and antisymmetric, so it suffices to check that the
    cross product satisfies the Jacobi identity.

    We proceed to calculate the Jacobi identity directly. Now, we know that
    \[
        ((A\times B)\times C)^i = \epsilon^i_{jk}\epsilon^j_{mn}A^mB^nC^k
    \]
    where $\epsilon^i_{jk}$ is the Levi-Civita symbol.
    So
    \[
        \begin{aligned}
            ((A\times B)\times C + (B\times C)\times A + (C\times A)\times B)^i &=
            \epsilon^i_{jk}\epsilon^j_{mn}(A^kB^mC^n + B^kC^mA^n + C^kA^mB^n)\\
            &=T^i_{kmn}V^{kmn}
        \end{aligned}
    \]
    where $T^i_{kmn}= \epsilon^i_{jk}\epsilon^j_{mn}$ and $V^{kmn} = A^kB^mC^n +
    B^kC^mA^n + C^kA^mB^n$.

    Now, $\epsilon^i_{jk}\epsilon^j_{mn} = T^i_{kmn}$is a symbol of rank $(1,3)$ whose
    components can be calculated directly. It is easy to see that this symbol is
    antisymmetric in the first and last two components. Furthermore, the only
    nonzero terms (up to antisymmetry) is $T^i_{kki} = 1$. Thus
    \[
        \begin{aligned}
            ((A\times B)\times C + (B\times C)\times A + (C\times A)\times B)^i &=
            T^i_{kmn}V^{kmn}\\
            &=T^i_{kki}V^{kki} + T^i_{kik}V^{kik} &\textrm{for fixed }i,k\\
            &=T^i_{kki}V^{kki} - T^i_{kki}V^{kki} &\textrm{by antisymmetry of
            }T\\
            &=0
        \end{aligned}
    \]
    as desired.

    Thus, the cross product is a bilinear map that satisfies the Jacobi
    identity, and $\R^3$ with this product is a Lie algebra.
\end{proof}

\section*{Problem 5} %Lee 8-20
Let $A\subseteq \mathcal{X}(\R^3)$ be the subspace spanned by the vector fields
\[
    \begin{aligned}
        X &= y\partial_z - z\partial_y\\
        Y &= z\partial_x - x\partial_z\\
        Z &= x\partial_y - y\partial_x\\
    \end{aligned}
\]
Show that $A$ is a Lie subalgebra of $\mathcal{X}(\R^3)$.
\\
\begin{proof}
    We can calculate
    \[
        \begin{aligned}
            {[}X,Y{]}^i &= X^j\partial_jY^i - Y^j\partial_jX^i\\
            [X,Y]^x &= X^j\partial_jY^x - Y^j\partial_jX^x\\
                    &= X^j\partial_jz - 0\\
                    &= X^z = y\\
                    \\
            [X,Y]^y &= X^j\partial_jY^y - Y^j\partial_jX^y\\
                    &= 0 - Y^j\partial_j(-z)\\
                    &= Y^z = -x\\
                    \\
            [X,Y]^z &= X^j\partial_jY^z - Y^j\partial_jX^z\\
                    &= X^j\partial_j(-x) - Y^j\partial_j(y)\\
                    &=-X^x - Y^y = 0
        \end{aligned}
    \]
    Thus, ${[}X,Y{]} = y\partial_x - x\partial_y = -Z$. However, the
    exact same calculation on ${[}Y,Z{]}$ by permuting the indices reveals that
    ${[}Y,Z{]} = -X$, and ${[}Z,X{]} = -Y$.

    So, the Lie algebra isomorphism will be
    \[
        \begin{aligned}
            X &\mapsto e_1\\
            Y &\mapsto -e_2\\
            Z &\mapsto e_3
        \end{aligned}
    \]
    and since this isomorphism preserves the bracket of each, it is a Lie
    algebra isomorphism.
\end{proof}

\section*{Problem 6}
Construct a non-vanishing vector field on a general odd-dimensional sphere.
\\
\\
\begin{proof}
    Let $S^{2n-1}$ be embedded in $\C^n$. Furthermore,
    $\gamma_i(t) = \exp(it)(0,\ldots,1,0,\ldots,0)$ where the $1$ is in the
    $i^{th}$ coordinate. Then, the vector field
    \[
        \sum_{i=1}^n \dot{\gamma}_i(t)
    \]
    is a nowhere-vanishing vector field on the sphere.
\end{proof}

\section*{Problem 7} %Lee 9-1
Suppose $M$ is a smooth manifold, $X\in \mathcal{X}(M)$, and $\gamma$ is a
maximal integral curve of $X$.
\subsection*{Part a}
Show that exactly one of the following holds:
\begin{itemize}
    \item $\gamma$ is constant.
    \item $\gamma$ is injective.
    \item $\gamma$ is periodic and nonconstant.
\end{itemize}
\begin{proof}
    If $\gamma$ is constant, In particular, $\gamma$ cannot be injective, since
    its domain is uncountable, but each maps to the same point. Furthermore,
    clearly if $\gamma$ is constant, $\gamma$ cannot be nonconstant.

    If $\gamma$ is injective, then it cannot be constant, since the domain has
    more than one point. Furthermore, it cannot be periodic, since if $\gamma(t)
    = \gamma(t+T)$ for some $T>0$, then $\gamma$ sends $t$ and $t+T$ to the
    same point.

    If $\gamma$ is periodic and nonconstant, then $\gamma$ is clearly not
    constant, and $\gamma$ is not injective, since $\gamma(t) = \gamma(t+T)$.
\end{proof}

\subsection*{Part b}
Suppose $\gamma$ is periodic and nonconstant. Show that there exists a unique
positive number $T$ for which $\gamma(t)=\gamma(t')$ if and only if $t'-t=kT$.
\\
\\
\begin{proof}
    Let $P = \{T\ |\ \gamma(t) = \gamma(t+T)\}$ the set of all periodicity
    constants of $\gamma$, and let $T' = \inf P$. Now, it should be clear that
    $P$ is closed, since for any $\tilde{T}\not\in P$, we have that 
    \[
        \gamma(t) \neq \gamma(t+\tilde{T})
    \]
    and since $\gamma$ is continuous, this must  be true on some neighborhood of
    $t$ as well. Thus, there is a neighborhood of $\tilde{T}$ for which
    \[
        \gamma(t)\neq\gamma(t+\tilde{T}-\epsilon)
    \]
    and so $P^c$ is open and $P$ is closed.

    Thus, $P$ contains its $\inf$, and thus $T'\in P$ and $T'$ is the period of
    $\gamma$.
\end{proof}

\subsection*{Part c}
Show that the image of $\gamma$ is an immersed submanifold of $M$ diffeomorphic
to $\R, S^1,$ or $\R^0$.
\\
\\
\begin{proof}
    If $\gamma$ is injective, then it is an injective smooth immersion, and thus
    its image is diffeomorphic to its domain (which is $\R$).

    If $\gamma$ is constant, then it is trivially diffeomorphic to $\R^0$.

    If $\gamma$ is periodic, then we can factor $\gamma$ through the quotient
    $\R/{T\R}\cong S^1$, and $\gamma$ is an injective smooth immersion from
    $S^1$ into $M$, and thus its image is diffeomorphic to its domain $S^1$.
\end{proof}

\section*{Problem 8} %Lee 9-3
Calculate the flows of the following
\subsection*{Part b}
$W=x\partial_x + 2y\partial_y$.
\\
\\
\begin{proof}
    We have the differential equation
    \[
        W = \dot{\theta}(p,t)
    \]
    which can be written as
    \[
        \begin{aligned}
            \dot{x}(t) &= x(t)\\
            \dot{y}(t) &=2y(t)
        \end{aligned}
    \]
    or (with $x$ being the vector quantity $(x,y)$)
    \[
        \dot{x}(t) =
        \begin{bmatrix}
        1&0\\
        0&2
        \end{bmatrix}
        x(t)
    \]
    Which has a general solution
    \[
        x(t) =
        \begin{bmatrix}
        \exp(t)&0\\
            0&\exp(2t)
        \end{bmatrix}
        x_0
    \]
    and so the flow is
    \[
        \theta(p,t) = (p_x\exp(t),p_y\exp(2t))
    \]
\end{proof}

\subsection*{Part b}
$Y=x\partial_y + y\partial_x$.
\\
\\
\begin{proof}
    This has the same form of solution as above, except now the equation is
    \[
        \dot{x}(t) =
        \begin{bmatrix}
        0&1\\
            1&0
        \end{bmatrix}
        x(t)
    \]
    Now, we can diagonalize the matrix to get
    \[
        \begin{bmatrix}
        0&1\\
            1&0
        \end{bmatrix}
        = M = SJS^{-1}
    \]
    for
    \[
        S=
        \begin{bmatrix}
            -1&1\\
            1&1
        \end{bmatrix}
    \]
    and
    \[
        J=
        \begin{bmatrix}
            -1&0\\
            0&1
        \end{bmatrix}
    \]
    which has solution
    \[
        x(t) = S\exp(Jt)S^{-1}x_0
            = S
            \begin{bmatrix}
                \exp(-t)&0\\
                0&\exp(t)
            \end{bmatrix}
            S^{-1}x_0
    \]
\end{proof}

\section*{Problem 9}
Prove the escape lemma.
\\
\\
\begin{proof}

\end{proof}

\section*{Problem 10}
Show that $\textrm{Diff}(M)$ acts transitively on $M$ for $M$ a connected smooth
manifold.
\\
\\
\begin{proof}

\end{proof}

\section*{Problem 11}
Give an example of smooth vector fields $V$, $\tilde{V}$ and $W$ on $\R^2$ so
that $V=\tilde{V}=\partial_x$ on the $x$ axis, but $L_VW\neq L_{\tilde{V}}W$ at
the origin.
\\
\\
\begin{proof}
    The vector fields
    \[ 
        \begin{aligned}
            V &=\partial_x\\
            \tilde{V} &=\partial_x + y\partial_y\\
            W &=\partial_y
        \end{aligned}
    \]
    are such that
    \[
        [V,W] = 0
    \]
    but
    \[
        \begin{aligned}
            [\tilde{V},W]^x &= \tilde{V}W^x - W\tilde{V}^x\\
                            &= 0\\
            \\
            [\tilde{V},W]^y &= \tilde{V}W^y - W\tilde{V}^y\\
                            &= -1\neq 0
        \end{aligned}
    \]
    as desired.
\end{proof}

\section*{Problem 12}
Let
\[
    \begin{aligned}
        X &= x\partial_x - y\partial_y\\
        Y &= x\partial_y + y\partial_x
    \end{aligned}
\]
Compute the flows and verify they do not commute.
\\
\\
\begin{proof}
    
\end{proof}

\section*{Problem 13}
\subsection*{Part a}
If $z = f(x,y)$ solves the system $z_x = g, z_y=h$. Find the compatibility
condition for $g$ and $h$.
\\
\\
\begin{proof}
    We know from the compatibility condition that $\partial_x\partial_y f =
    \partial_y\partial_x f$, which means that
    \[
        \partial_x z_y = \partial_y z_x
    \]
    or,
    \[
        \partial_x h = \partial_y g
    \]
\end{proof}

\subsection*{Part b}
Show that this is equivalent to $[X,Y] = 0$.
\\
\\
\begin{proof}
    The vector fields are
    \[
        \begin{aligned}
            X &= \partial_x + g\partial_z\\
            Y &= \partial_y + h\partial_z
        \end{aligned}
    \]
    We can calculate the Lie bracket
    \[
        \begin{aligned}
            {[}X,y{]} &= (XY^j-YX^j)\partial_j\\
                    &= 0\partial_x + 0\partial_y + (Xh-Yg)\partial_z\\
                    &= \partial_xh + g\partial_zh - (\partial_yg +
                    h\partial_zg)\partial_z\\
                    &=\partial_xh+\partial_xz\partial_zh -
                    (\partial_yg+\partial_yz\partial_zg)\\
                    &=\partial_xh - \partial_yg
        \end{aligned}
    \]
    and for this to be zero, we must have that $\partial_xh=\partial_yg$, which
    is the compatibility condition.
\end{proof}

\end{document}
