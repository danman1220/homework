\documentclass{article}

\usepackage{amsmath, amssymb, amsthm}

\newtheorem*{definition}{Definition}
\newtheorem*{rmk}{Remark}

\author{Daniel Halmrast}

\title{Differential Geometry Notes}

\date{October 6, 2017}

\begin{document}

\maketitle

\section*{Smooth Functions}

\begin{definition}
Given a smooth manifold $M$, a function $f:M\to\mathbb{R}$ is
said to be \em smooth\em (or $C^{\infty}$) iff:\\
$\forall (U,\phi)$ charts on $M$, $f\circ\phi^{-1}:\mathbb{R}^n\to\mathbb{R}$
is a smooth ($C^{\infty}$) function.
\end{definition}

\begin{rmk}
Note that you only have to check smoothness against one atlas,
not necessarily all in the smooth structure.
\end{rmk}

\section*{Smooth Maps}
Now, let's generalize this to maps between manifolds.

\begin{definition}
Given two smooth manifolds $M$ and $N$, a map $f:M\to N$ is
said to be \em smooth\em iff:\\
$\forall p\in M$, there exist charts $(U,\phi)$ on $M$ 
and $(V,\tau)$ on $N$ such that $p\in U$, $f(p)\in V$,
and the map $\tau\circ f \circ \phi^{-1}:\mathbb{R}^m\to\mathbb{R}^n$
is smooth.
\end{definition}
Basically, $f$ has to be compatible with the coordinate charts at $p$ and $f(p)$.

\end{document}
