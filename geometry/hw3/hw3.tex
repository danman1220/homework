%%%%%%%%%%%%%%%%%%%%%%%%%%%%%%%%%%%%%%%%%
% Short Sectioned Assignment
% LaTeX Template
% Version 1.0 (5/5/12)
%
% This template has been downloaded from:
% http://www.LaTeXTemplates.com
%
% Original author:
% Frits Wenneker (http://www.howtotex.com)
%
% License:
% CC BY-NC-SA 3.0 (http://creativecommons.org/licenses/by-nc-sa/3.0/)
%
%%%%%%%%%%%%%%%%%%%%%%%%%%%%%%%%%%%%%%%%%

%----------------------------------------------------------------------------------------
%	PACKAGES AND OTHER DOCUMENT CONFIGURATIONS
%----------------------------------------------------------------------------------------

\documentclass[fontsize=11pt]{scrartcl} % 11pt font size

\usepackage[T1]{fontenc} % Use 8-bit encoding that has 256 glyphs
\usepackage[english]{babel} % English language/hyphenation
\usepackage{amsmath,amsfonts,amsthm} % Math packages
\usepackage{mathrsfs}

\usepackage[margin=1in]{geometry}

\usepackage{sectsty} % Allows customizing section commands
\allsectionsfont{\centering \normalfont\scshape} % Make all sections centered, the default font and small caps

\usepackage{fancyhdr} % Custom headers and footers
\pagestyle{fancyplain} % Makes all pages in the document conform to the custom headers and footers
\fancyhead{} % No page header - if you want one, create it in the same way as the footers below
\fancyfoot[L]{} % Empty left footer
\fancyfoot[C]{} % Empty center footer
\fancyfoot[R]{\thepage} % Page numbering for right footer
\renewcommand{\headrulewidth}{0pt} % Remove header underlines
\renewcommand{\footrulewidth}{0pt} % Remove footer underlines
\setlength{\headheight}{13.6pt} % Customize the height of the header

\numberwithin{equation}{section} % Number equations within sections (i.e. 1.1, 1.2, 2.1, 2.2 instead of 1, 2, 3, 4)
\numberwithin{figure}{section} % Number figures within sections (i.e. 1.1, 1.2, 2.1, 2.2 instead of 1, 2, 3, 4)
\numberwithin{table}{section} % Number tables within sections (i.e. 1.1, 1.2, 2.1, 2.2 instead of 1, 2, 3, 4)

\newcommand{\R}{\mathbb{R}}
\newcommand{\Q}{\mathbb{Q}}
\newcommand{\C}{\mathbb{C}}

%----------------------------------------------------------------------------------------
%	TITLE SECTION
%----------------------------------------------------------------------------------------

\newcommand{\horrule}[1]{\rule{\linewidth}{#1}} % Create horizontal rule command with 1 argument of height

\title{	
\normalfont \normalsize 
\textsc{Geometry} \\ [25pt] % Your university, school and/or department name(s)
\horrule{0.5pt} \\[0.4cm] % Thin top horizontal rule
\huge Problem Set 3\\ % The assignment title
\horrule{2pt} \\[0.5cm] % Thick bottom horizontal rule
}

\author{Daniel Halmrast} % Your name

\date{\normalsize\today} % Today's date or a custom date

\begin{document}

\maketitle % Print the title

%----------------------------------------------------------------------------------------
%	PROBLEM 1
%----------------------------------------------------------------------------------------
\section*{Problem 1} %Lee 4-5
\subsection*{Part a}
Show that the quotient map $\pi: \C^{n+1}\setminus \{0\}\to\C P^n$ is a surjective
smooth submersion.
\\
\begin{proof}
Surjectivity follows almost immediately from the definition of the map, since any
equivalence class $[z]$ is mapped to by $\pi(z)$. Now, we must show it is a smooth submersion.

We will do this using the Global Rank Theorem (Theorem 4.14), which states that a surjection
with constant rank is a smooth submersion.

Now, the coordinate charts on $\C^{n+1}$ are just the standard rectangular coordinates
\[
(z^1,\ldots,z^{n+1}) \mapsto (\Re(z^1),\Im(z^1),\ldots,\Re(z^{n+1}),\Im(z^{n+1}))
\]
And for each $i\in\{1,\ldots,n+1\}$, we have a local coordinate chart on $\C P^n$ around
where $z^i$ is not zero, given as
\[
(z^1:\ldots:z^i:\ldots:z^{n+1}) \mapsto \frac{1}{z^i}(z^1,\ldots,z^{i-1},z^{i+1},\ldots,z^{n+1})
\]
(Here, this coordinate chart actually maps to $\C^n$, which has a global coordinate chart
as defined above, and the composition then defines a coordinate chart to $\R^{2n}$.)

So now we consider the coordinate representation of $\pi$ around a point $z_0$ for which
$z_0^i$ is not zero, which (as complex coordinates) is given as
\[
\tilde{\pi}(z^1,\ldots,z^i,\ldots,z^{n+1}) = \frac{1}{z^i}(z^1,\ldots,z^{i-1},z^{i+1},\ldots,z^{n+1})
\]
This is clearly smooth from $\C^{n+1}\to\C^n$, so $\pi$ is smooth as well.
Now, let's find what the rank at this point is.

To do so, we compute the Jacobian of $d\pi$. This is given as 

\end{proof}

\subsection*{Part b}
Show that $\C P^1 \cong S^2$.
\\
\begin{proof}

\end{proof}
%----------------------------------------------------------------------------------------
%----------------------------------------------------------------------------------------
%	PROBLEM 2
%----------------------------------------------------------------------------------------
\section*{Problem 2} %Lee 4-6
For $M$ a nonempty smooth compact manifold, show that there is no
smooth submersion $F:M\to\R^k$ for any $k>0$.
\\
\begin{proof}

\end{proof}
%----------------------------------------------------------------------------------------
%----------------------------------------------------------------------------------------
%	PROBLEM 3
%----------------------------------------------------------------------------------------
\section*{Problem 3} %Lee 4-12
Use the covering map $\varepsilon^2:\R^2\to\mathbb{T}^2$ to show that the immersion $\chi:\R^2\to\R^3$
descends to a smooth embedding of $\mathbb{T}^2$ into $\R^3$. Specifically, show that $\chi$
passes to the quotient to define a smooth map $\tilde{\chi}:\mathbb{T}^2\to\R^3$, then show
that $\tilde{\chi}$ is a smooth embedding whose image is the given surface of revolution.
\\
\begin{proof}

\end{proof}
%----------------------------------------------------------------------------------------
%----------------------------------------------------------------------------------------
%	PROBLEM 4
%----------------------------------------------------------------------------------------

%----------------------------------------------------------------------------------------
%----------------------------------------------------------------------------------------
%	PROBLEM 5
%----------------------------------------------------------------------------------------

%----------------------------------------------------------------------------------------

\end{document}
