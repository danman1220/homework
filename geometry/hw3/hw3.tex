%%%%%%%%%%%%%%%%%%%%%%%%%%%%%%%%%%%%%%%%%
% Short Sectioned Assignment
% LaTeX Template
% Version 1.0 (5/5/12)
%
% This template has been downloaded from:
% http://www.LaTeXTemplates.com
%
% Original author:
% Frits Wenneker (http://www.howtotex.com)
%
% License:
% CC BY-NC-SA 3.0 (http://creativecommons.org/licenses/by-nc-sa/3.0/)
%
%%%%%%%%%%%%%%%%%%%%%%%%%%%%%%%%%%%%%%%%%

%----------------------------------------------------------------------------------------
%	PACKAGES AND OTHER DOCUMENT CONFIGURATIONS
%----------------------------------------------------------------------------------------

\documentclass[fontsize=11pt]{scrartcl} % 11pt font size

\usepackage[T1]{fontenc} % Use 8-bit encoding that has 256 glyphs
\usepackage[english]{babel} % English language/hyphenation
\usepackage{amsmath,amsfonts,amsthm} % Math packages
\usepackage{mathrsfs}
\usepackage{tikz-cd}

\usepackage[margin=1in]{geometry}

\usepackage{sectsty} % Allows customizing section commands
\allsectionsfont{\centering \normalfont\scshape} % Make all sections centered, the default font and small caps

\usepackage{fancyhdr} % Custom headers and footers
\pagestyle{fancyplain} % Makes all pages in the document conform to the custom headers and footers
\fancyhead{} % No page header - if you want one, create it in the same way as the footers below
\fancyfoot[L]{} % Empty left footer
\fancyfoot[C]{} % Empty center footer
\fancyfoot[R]{\thepage} % Page numbering for right footer
\renewcommand{\headrulewidth}{0pt} % Remove header underlines
\renewcommand{\footrulewidth}{0pt} % Remove footer underlines
\setlength{\headheight}{13.6pt} % Customize the height of the header

\numberwithin{equation}{section} % Number equations within sections (i.e. 1.1, 1.2, 2.1, 2.2 instead of 1, 2, 3, 4)
\numberwithin{figure}{section} % Number figures within sections (i.e. 1.1, 1.2, 2.1, 2.2 instead of 1, 2, 3, 4)
\numberwithin{table}{section} % Number tables within sections (i.e. 1.1, 1.2, 2.1, 2.2 instead of 1, 2, 3, 4)

\newcommand{\R}{\mathbb{R}}
\newcommand{\Q}{\mathbb{Q}}
\newcommand{\C}{\mathbb{C}}

%----------------------------------------------------------------------------------------
%	TITLE SECTION
%----------------------------------------------------------------------------------------

\newcommand{\horrule}[1]{\rule{\linewidth}{#1}} % Create horizontal rule command with 1 argument of height

\title{	
\normalfont \normalsize 
\textsc{Geometry} \\ [25pt] % Your university, school and/or department name(s)
\horrule{0.5pt} \\[0.4cm] % Thin top horizontal rule
\huge Problem Set 3\\ % The assignment title
\horrule{2pt} \\[0.5cm] % Thick bottom horizontal rule
}

\author{Daniel Halmrast} % Your name

\date{\normalsize\today} % Today's date or a custom date

\begin{document}

\maketitle % Print the title

%----------------------------------------------------------------------------------------
%	PROBLEM 1
%----------------------------------------------------------------------------------------
\section*{Problem 1} %Lee 4-5
\subsection*{Part a}
Show that the quotient map $\pi: \C^{n+1}\setminus \{0\}\to\C P^n$ is a surjective
smooth submersion.
\\
\begin{proof}
Surjectivity follows almost immediately from the definition of the map, since any
equivalence class $[z]$ is mapped to by $\pi(z)$. Now, we must show it is a smooth submersion.

We will do this using the Global Rank Theorem (Theorem 4.14), which states that a surjection
with constant rank is a smooth submersion.

Now, the coordinate charts on $\C^{n+1}$ are just the standard rectangular coordinates
\[
(z^1,\ldots,z^{n+1}) \mapsto (\Re(z^1),\Im(z^1),\ldots,\Re(z^{n+1}),\Im(z^{n+1}))
\]
And for each $i\in\{1,\ldots,n+1\}$, we have a local coordinate chart on $\C P^n$ around
where $z^i$ is not zero, given as
\[
(z^1:\ldots:z^i:\ldots:z^{n+1}) \mapsto \frac{1}{z^i}(z^1,\ldots,z^{i-1},z^{i+1},\ldots,z^{n+1})
\]
(Here, this coordinate chart actually maps to $\C^n$, which has a global coordinate chart
as defined above, and the composition then defines a coordinate chart to $\R^{2n}$.)

So now we consider the coordinate representation of $\pi$ around a point $z_0$ for which
$z_0^i$ is not zero, which (as complex coordinates) is given as
\[
\tilde{\pi}(z^1,\ldots,z^i,\ldots,z^{n+1}) = \frac{1}{z^i}(z^1,\ldots,z^{i-1},z^{i+1},\ldots,z^{n+1})
\]
This is clearly smooth from $\C^{n+1}\to\C^n$, so $\pi$ is smooth as well.
Now, let's find what the rank at this point is.

To do so, we compute the Jacobian of $d\pi$, $J_j^k = \partial_j(\tilde{\pi}^k)$.
This is a somewhat complicated Jacobian, so we consider first the minor to the
left of the $i^{th}$ column.  On this minor, we have that
\[
J_j^k = \frac{1}{z^i}\delta_j^k
\]
Which is just $\frac{1}{z^i}*I$ for the $i-1\times i-1$ identity matrix.

Now, to the right of the $i^{th}$ column, we have
\[
J_j^k = \frac{1}{z^i}\delta_{(j+1)}^k
\] 
And finally, for the $i^{th}$ column itself, we have that
\[
J_i^k = \partial_i(\tilde{\pi}^k) = \frac{-1}{z^i}\tilde{\pi}^k
\]
So the whole matrix looks like
\[
\begin{bmatrix}
\frac{1}{z^i} & 0   &\ldots &0 &\frac{-z^1}{(z^i)^2} & 0 & \ldots & 0\\
0 & \frac{1}{z^i}   &\ldots &0 &\frac{-z^2}{(z^i)^2} & 0 & \ldots & 0\\
\vdots & \vdots     &\ddots &\vdots &\vdots          & \vdots & \ddots & \vdots\\
0 & 0               &\ldots &\frac{1}{z^i} &\frac{-z^{i-1}}{(z^i)^2} & 0 &\ldots& 0\\
0 & 0               &\ldots &0 &\frac{-z^{i+1}}{(z^i)^2} & \frac{1}{z^i} &\ldots &0\\
\vdots &\vdots      &\ddots &\vdots &\vdots                   &\vdots &\ddots &\vdots\\
0 & 0               &\ldots &0 &\frac{-z^{n+1}}{(z^i)^2} &0 &\ldots &\frac{1}{z^i}\\ 
\end{bmatrix}
\]
Which, since $z^i$ is not zero, has full rank. Thus, since this is true for any point
in $\C^{n+1}$, it follows that $\pi$ has full rank at every point, and is a smooth submersion.
\end{proof}

\subsection*{Part b}
Show that $\C P^1 \cong S^2$.
\\
\begin{proof}
To do this, we will consider the two projection maps
\[
\begin{tikzcd}[column sep=small]
 &S^3 \arrow[swap]{dl}{\pi_1} \arrow{dr}{\pi_2}& \\
\C P^1 & & S^2
\end{tikzcd}
\]
Given by $\pi_1$ the canonical projection map from $S^3$ to $S^3/{U(1)}$,
and $\pi_2$ is the Hopf fibration, which was shown to be a smooth surjection in the last
homework assignment. (Here, we will use the fibration map $\pi_2(z_1,z_2) = (2z_1\overline{z_2},|z_1|^2-|z_2|^2)$).

A small calculation shows that the Hopf fibration is, in fact, a surjective smooth
submersion, since in rectangular coordinates, we have
\[
\pi_2(x,y,w,v) = (2(xw+yv),2(yw-xv),X^2+y^2-v^2-w^2)
\]
Which has a differential of
\[
d\pi_2 = 2
\begin{bmatrix}
w & v & y & x\\
-v &w &-x & y\\
x & y &-v &-w
\end{bmatrix}
\]
Which is clearly full rank, and thus surjective.

We note first the fibers of these maps coincide. This is clear since the fiber associated
with $\pi_1([(z_1,z_2)]) = \{\lambda (z_1,z_2)\}$ for all $\lambda\in U(1)$, which all map to the same
point via $\pi_2$, since 
\[
\begin{aligned}
\pi_2(\lambda(z_1,z_2)) &= (2\lambda z_1\overline{\lambda z_2},|\lambda z_1|^2-|\lambda z_2|^2)\\
                        &= (2\lambda\overline{\lambda}z_1\overline{z_2}, |z_1|^2 - |z_2|^2)\\
                        &= (2z_1\overline{z_2},|z_1|^2-|z_2|^2)
\end{aligned}
\]
Thus, a fiber of $\pi_1$ is a fiber of $\pi_2$.

Now, let's look at the fibers of $\pi_2$. We know that
\[
\pi_2^{-1}(z,r) = \{(z_1,z_2)\in S^3\ |\ z_1\overline{z_2} = z,\ |z_1|^2-|z_2|^2 = r\}
\]
Now, the last condition, along with the condition that $z_1,z_2\in S^3$, fix the norms of
$z_1$ and $z_2$. Thus, for any other point $(z_1',z_2')$ in the fiber, we must have
that $z_1' = \lambda_1 z_1$ and $z_2' = \lambda_2 z_2$ with $\lambda_1,\lambda_2\in U(1)$.
Furthermore, the restriction
that $z_1\overline z_2 = z$ forces $\lambda_1 = \lambda_2$. Thus,
$\pi_1(\pi_2^{-1}(z,r))$, which quotients by $\lambda \in U(1)$, sends the fiber 
to a single point.

Thus, each smooth surjection is constant on each other's fibers, which gives rise to the
diffeomorphism
\[
\begin{tikzcd}[column sep=small]
 &S^3 \arrow[swap]{dl}{\pi_1} \arrow{dr}{\pi_2}& \\
\C P^1 \arrow[dashed, hook, two heads]{rr}{F}& & S^2
\end{tikzcd}
\]
as desired.
\end{proof}
%----------------------------------------------------------------------------------------
\newpage
%----------------------------------------------------------------------------------------
%	PROBLEM 2
%----------------------------------------------------------------------------------------
\section*{Problem 2} %Lee 4-6
For $M$ a nonempty smooth compact manifold, show that there is no
smooth submersion $F:M\to\R^k$ for any $k>0$.
\\
\begin{proof}
We note two things immediately. First, since $F$ is continuous, its image must be compact 
(and by the Heine-Borel theorem, must be closed).
Second, since $F$ is a smooth submersion, it is an open map. Thus, the image of $F$ is open
in $\R^k$. The only nonempty set in $R^k$ that is both closed and open is $\R^k$ itself,
but this cannot be the image of $F$, since the image must be compact. So, such an $F$ cannot
exist.
\end{proof}
%----------------------------------------------------------------------------------------
%----------------------------------------------------------------------------------------
%	PROBLEM 3
%----------------------------------------------------------------------------------------
\section*{Problem 3} %Lee 4-12
Use the covering map $\varepsilon^2:\R^2\to\mathbb{T}^2$ to show that the immersion $\chi:\R^2\to\R^3$
descends to a smooth embedding of $\mathbb{T}^2$ into $\R^3$. Specifically, show that $\chi$
passes to the quotient to define a smooth map $\tilde{\chi}:\mathbb{T}^2\to\R^3$, then show
that $\tilde{\chi}$ is a smooth embedding whose image is the given surface of revolution.
\\
\begin{proof}
We first show that $\chi$ passes smoothly through the quotient by showing it is constant
on the fibers of $\epsilon^2$. To see this, we observe that $\chi$ is periodic in integer
multiples of its arguments, and that $\epsilon^2$ is similarly periodic. That is, for
$(\exp(2\pi i\theta_1),\exp(2\pi i\theta_2)),$ its fiber is just $(n\theta_1,m\theta_2)$.
But since $\chi$ is periodic, it must map all of this fiber to a point. Thus, $\chi$ is
constant on the fibers of $\epsilon^2$, and can be passed through the quotient to
obtain $\tilde{\chi}$ which makes
\[
\begin{tikzcd}[column sep=small]
 &\R^2 \arrow[swap]{dl}{\chi} \arrow[two heads]{dr}{\epsilon^2}& \\
\R^3 & & \mathbb{T}^2\arrow[dashed]{ll}{\tilde{\chi}}
\end{tikzcd}
\]
commute.

Now, $\tilde{\chi}$ is clearly injective, since $\chi$ is only periodic on the fibers of $\epsilon^2$.
So, we just need to show that $\tilde{\chi}$ is an immersion, and Proposition 4.22 will
give us that $\tilde{\chi}$ is an embedding.

To see that $\tilde{\chi}$ is an immersion, we observe the following diagram:
\[
\begin{tikzcd}
T\R^2\arrow[bend right, hook]{rr}{d\chi} \arrow[hook, two heads]{r}{d\epsilon^2} & T\mathbb{T}^2 \arrow{r}{d\tilde{\chi}} & T\R^3\\
\end{tikzcd}
\]
Where $d\epsilon^2$ is locally bijective since $\epsilon^2$ is a covering map, and
$d\chi$ is injective since $\chi$ is an immersion. Therefore, $d\tilde{\chi}$ must be
injective, and the result follows immediately.
\end{proof}
%----------------------------------------------------------------------------------------
\newpage
%----------------------------------------------------------------------------------------
%	PROBLEM 4
%----------------------------------------------------------------------------------------
\section*{Problem 4} %Lee 4-13
Let $F:S^2\to\R^4$ by
\[
F(x,y,z) = (x^2-y^2,xy,xz,yz)
\]
descends along the quotient of $S^2$ by $O(1)$ to a smooth embedding of $\R P^2$ into
$\R^4$.
\\
\begin{proof}
TO begin with, we cite that the quotient map from $S^2$ to $\R P^2$ is given as a smooth
covering map from problem 4-10. Thus, we have the following diagram
\[
\begin{tikzcd}[column sep=small]
 &S^2 \arrow[swap]{dl}{F} \arrow{dr}{q}& \\
\R^4 & & \R P^2
\end{tikzcd}
\]
We note that $F$ is constant on the fibers of $q$ (where we observe that $q^{-1}(\{[x]\}) = \{x,-x\}$)
since
\[
F(-x,-y,-z) = ((-x)^2 - (-y)^2,(-x)(-y),(-x)(-z),(-y)(-z)) = F(x,y,z)
\]
Thus, we apply Theorem 4.30 to construct $\tilde{F}$ such that
\[
\begin{tikzcd}[column sep=small]
 &S^2 \arrow[swap]{dl}{F} \arrow{dr}{q}& \\
\R^4 & & \R P^2\arrow[dashed]{ll}{\tilde{F}}
\end{tikzcd}
\]
commutes.

Now, since $\R P^2$ is compact, it suffices to show that $\tilde{F}$ is an injective
immersion, and Proposition 4.22 gives us that $\tilde{F}$ is a smooth embedding.

To see that $\Tilde{F}$ is an injective immersion, we first show that $\tilde{F}$ is 
injective.

A simple diagram chase of
\[
\begin{tikzcd}[column sep=small]
 &S^2 \arrow[swap]{dl}{F} \arrow[two heads]{dr}{q}& \\
\R^4 & & \R P^2\arrow{ll}{\tilde{F}}
\end{tikzcd}
\]
Will prove this. To do so, let $[x],[y]$ be distinct points in $\R P^2$. Then, by injectivity
of $q$, it follows that there are points $x,y\in S^2$ such that $q(x) = [x]$ and $q(y) = [y]$.
Now, since $[x]\neq[y]$, we have that $x\neq -y$. Thus, it is clear that $F(x)\neq F(y)$.
This can be seen by the following argument.

Suppose $F(x,y,z) = (q,r,s,t)$ for some constants $q,r,s,t$. In particular, we have that
\[
\begin{aligned}
x^2 &= \frac{rs}{t}\\
y^2 &= \frac{rt}{s}\\
z^2 &= \frac{st}{r}
xy  &= r\\
xz  &= s\\
yz  &= t 
\end{aligned}
\]
It follows immediately that if $F(x',y',z') = (q,r,s,t)$, then $(x',y',z') = (x,y,z)$ or $-(x,y,z)$.

Thus, $F(x)\neq F(y)$, and we have that
\[
\begin{aligned}
F(x) = \tilde{F}(q(x)) &= \tilde{F}([x])\\
F(y) &= \tilde{F}([x])\\
F(x)\neq F(y) \implies \tilde{F}([x])\neq\tilde{F}([y])
\end{aligned}
\]
which shows that $\tilde{F}$ is injective.

Similarly, it is clear that $d\tilde{F}$ is injective. To see this, we note that
since $q$ is a covering map, it is a local diffeomorphism, and thus $dq$ is locally
a bijection.

Then, observe the diagram
\[
\begin{tikzcd}
TS^2\arrow[bend right]{rr}{dF} \arrow[hook, two heads]{r}{dq} & T\R P^2 \arrow{r}{d\tilde{F}} & \R^4\\
\end{tikzcd}
\]
Now, a routine calculation of the Jacobian $dF$ shows that it is injective locally,
which implies by the above diagram (since $dq$ is locally bijective) that $d\tilde{F}$ is
locally injective. 

Thus, $\tilde{F}$ is an injective smooth immersion from a compact manifold, and is
a smooth embedding.
\end{proof}

%----------------------------------------------------------------------------------------
\newpage
%----------------------------------------------------------------------------------------
%	PROBLEM 5
%----------------------------------------------------------------------------------------
\section*{Problem 5}

\subsection*{Part a}
Show that an immersion between two manifolds of the same dimension is an open mapping.
\\
\begin{proof}
To see this, let $f:M\to N$ be an immersion between $M$ and $N$ of the same dimension $n$,
and let $U$ be open in $M$.

Now, since $f$ is an immersion, it must be that for each point $x\in U$, there is a neighborhood
of $x$ contained in $U$ for which the coordinate representative of $f$ is the identity. Thus,
on that neighborhood, $f$ is an open map. However, since $U$ can be written as the union of such
neighborhoods, and $f$ respects unions, it follows that $f(U)$ is the union of open sets, and is
open.
\end{proof}
\subsection*{Part b}
Use this to show there is no immersion from $S^n$ to $\R^n$.
\\
\begin{proof}
Suppose there did exist such an $f:S^n\to\R^n$. Since $f$ is continuous, it must
map the compact set $S^n$ to a compact set. However, $S^n$ is also open, and $f$ is an open
map, so the image $f(S^n)$ must also be open. Since there are no open compact sets in $\R^n$,
such a map cannot exist.
\end{proof}
%----------------------------------------------------------------------------------------

\end{document}
