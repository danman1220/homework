%%%%%%%%%%%%%%%%%%%%%%%%%%%%%%%%%%%%%%%%%
% Short Sectioned Assignment
% LaTeX Template
% Version 1.0 (5/5/12)
%
% This template has been downloaded from:
% http://www.LaTeXTemplates.com
%
% Original author:
% Frits Wenneker (http://www.howtotex.com)
%
% License:
% CC BY-NC-SA 3.0 (http://creativecommons.org/licenses/by-nc-sa/3.0/)
%
%%%%%%%%%%%%%%%%%%%%%%%%%%%%%%%%%%%%%%%%%

%----------------------------------------------------------------------------------------
%	PACKAGES AND OTHER DOCUMENT CONFIGURATIONS
%----------------------------------------------------------------------------------------

\documentclass[fontsize=11pt]{scrartcl} % 11pt font size

\usepackage[T1]{fontenc} % Use 8-bit encoding that has 256 glyphs
\usepackage[english]{babel} % English language/hyphenation
\usepackage{amsmath,amsfonts,amsthm} % Math packages
\usepackage{mathrsfs}

\usepackage[margin=1in]{geometry}

\usepackage{sectsty} % Allows customizing section commands
\allsectionsfont{\centering \normalfont\scshape} % Make all sections centered, the default font and small caps

\usepackage{fancyhdr} % Custom headers and footers
\pagestyle{fancyplain} % Makes all pages in the document conform to the custom headers and footers
\fancyhead{} % No page header - if you want one, create it in the same way as the footers below
\fancyfoot[L]{} % Empty left footer
\fancyfoot[C]{} % Empty center footer
\fancyfoot[R]{\thepage} % Page numbering for right footer
\renewcommand{\headrulewidth}{0pt} % Remove header underlines
\renewcommand{\footrulewidth}{0pt} % Remove footer underlines
\setlength{\headheight}{13.6pt} % Customize the height of the header

\numberwithin{equation}{section} % Number equations within sections (i.e. 1.1, 1.2, 2.1, 2.2 instead of 1, 2, 3, 4)
\numberwithin{figure}{section} % Number figures within sections (i.e. 1.1, 1.2, 2.1, 2.2 instead of 1, 2, 3, 4)
\numberwithin{table}{section} % Number tables within sections (i.e. 1.1, 1.2, 2.1, 2.2 instead of 1, 2, 3, 4)

\newcommand{\R}{\mathbb{R}}
\newcommand{\Q}{\mathbb{Q}}
\newcommand{\Z}{\mathbb{Z}}
\newcommand{\C}{\mathbb{C}}

\newcommand{\supp}{\textrm{supp}}

%----------------------------------------------------------------------------------------
%	TITLE SECTION
%----------------------------------------------------------------------------------------

\newcommand{\horrule}[1]{\rule{\linewidth}{#1}} % Create horizontal rule command with 1 argument of height

\title{	
\normalfont \normalsize 
\textsc{Geometry} \\ [25pt] % Your university, school and/or department name(s)
\horrule{0.5pt} \\[0.4cm] % Thin top horizontal rule
\huge Problem Set 2\\ % The assignment title
\horrule{2pt} \\[0.5cm] % Thick bottom horizontal rule
}

\author{Daniel Halmrast} % Your name

\date{\normalsize\today} % Today's date or a custom date

\begin{document}

\maketitle % Print the title

%----------------------------------------------------------------------------------------
%	PROBLEM 1
%----------------------------------------------------------------------------------------
\section*{Problem 2-1}
For $f$ the Heaviside step function (with $f(0)=1$), show that $\forall x \in\R$, there
exist smooth charts $(U,\phi)$ around $x$ and $(V,\psi)$ around $f(x)$ such that 
$\psi\circ f\circ\phi^{-1}$ is smooth as a map from its domain to its image, but $f$ is not
smooth in a smooth manifold sense.
\\
\begin{proof}
For $x\neq 0$, neighborhoods avoiding zero can be chosen, and identity charts make $f$
locally smooth. For $x=0$, set $U = (-\epsilon,\epsilon)$, $V = (1-\epsilon,1+\epsilon)$
and have $\phi_U = \psi_V = \textrm{id}$.
Then, on $U\cap f^{-1}(V) = [0,\epsilon)$ we have $\psi\circ f\circ\phi^{-1}(x) = 1$
which is smooth. But this fails the test in proposition 2.5, so $f$ is not smooth in a
manifold sense.
\end{proof}
%----------------------------------------------------------------------------------------
\pagebreak
%----------------------------------------------------------------------------------------
%	PROBLEM 2
%----------------------------------------------------------------------------------------
\section*{Problem 2-3}
For each of the following maps, show that the map is smooth via computation through coordinate
representations.
\subsection*{Part a}
The power map $p_n:S^1\to S^1$ defined as $p_n(z) = z^n$.
\\
\begin{proof}
For this problem, we will use two coordinate charts on $S^1$. First, let's parameterize
the circle by $\theta$, so that the point $\theta$ is identified with $\exp(i\theta)$ in the
standard embedding of the circle into $\C$. Then, the first coordinate chart will be for
$\theta\in(0,2\pi)$ given as $\phi(\theta) = \theta$. The second coordinate chart will
be for $\theta\in(-\pi,\pi)$ (where $2\pi\theta \sim\theta$) given as $\psi(\theta) = \theta$.

Now, the transition maps can easily be verified to be smooth. To see this, let $\theta_0$
be a point in the intersection of the two charts. Then, if $\theta\in(0,\pi)$, we have
\[
\begin{aligned}
\phi(\theta) &= \theta\\
\psi(\theta) &= \theta\\
\end{aligned}
\]
Which are easily verified to be smooth and compatible with each other.

Suppose, then, that $\theta\in (\pi,2\pi)$. Then, we have that
\[
\begin{aligned}
\phi(\theta) &= \theta\\
\psi(\theta) &= \theta - 2\pi
\end{aligned}
\]
With transition charts
\[
\begin{aligned}
\phi\circ\psi^{-1}(\theta) = \theta + 2\pi\\
\psi\circ\phi^{-1}(\theta) = \theta - 2\pi
\end{aligned}
\]
which are clearly smooth.

Now, we just have to check that the power function, which can be thought of in terms of
our parameterization as $p_n(\theta) = n\theta (\mod 2\pi)$, is smooth.

So, let's compute some coordinate representations. We have a total of four to check.
\[
\begin{aligned}
\phi\circ p_n\circ\phi^{-1}(\theta) &= n\theta (\mod 2\pi)\\
\psi\circ p_n\circ\psi^{-1}(\theta) &= n(\theta+2\pi)  (\mod 2\pi)- 2\pi\\
\phi\circ p_n\circ\psi^{-1}(\theta) &= n(\theta +2\pi) (\mod 2\pi)\\
\psi\circ p_n\circ\phi^{-1}(\theta) &= n\theta  (\mod 2\pi)- 2\pi
\end{aligned}
\]
Now, addition of a scalar is a smooth operation, so we just have to check that the
function $p_n$ is smooth as a function of $\theta$. 

Now, we observe that $p_n$ is continuous as a function of $\theta$ by viewing $p_n:[0,2\pi)\to\R$
as a continuous function $\theta\mapsto n\theta$, and passing through the quotient $\R/{2\pi\Z}$.
Since the derivative $p_n' = np_{n-1}$ is also of the same form, it is continuous as well,
and by induction each derivative of $p_n$ is continuous, so $p_n$ is smooth.

Thus, the composition maps defined above are smooth, and $p_n$ is a smooth function from
$S^1$ to itself.

Alternately, utilizing the Lie group structure of $S^1$ defined in problem 3-4,
we have that the map $l_{\theta}$, left multiplication by $\theta$, is smooth.
Since the power map is just repeated application of $l_{\theta}$ to itself,
it is a composition $n$ times of $l_{\theta}$, and thus is a composition of smooth
functions and is smooth.
\end{proof}

\subsection*{Part b}
The antipodal map $\alpha:S^n\to S^n$ by $\alpha(x) = -x$.
\\
\begin{proof}
Consider the stereographic projection charts $\sigma$ and $\tilde{\sigma}$, where
$\tilde{\sigma}(x) = -\sigma(-x)$. Let's compute some coordinate representations:
\[
\begin{aligned}
\sigma\circ\alpha\circ\sigma^{-1}(x) &= \sigma(-\sigma^{-1}(x))\\
\tilde{\sigma}\circ\alpha\circ\tilde{\sigma}^{-1}(x) &= \tilde{\sigma}(-\tilde{\sigma}^{-1}(x))\\
\sigma\circ\alpha\circ\tilde{\sigma}^{-1}(x) &= \sigma(-\tilde{\sigma}^{-1}(x))\\
\tilde{\sigma}\circ\alpha\circ\sigma^{-1}(x) &= \tilde{\sigma}(-\sigma^{-1}(x))
\end{aligned}
\]
Now, these are all compositions of smooth functions, which are smooth as well.
Thus, the antipodal map is a smooth function. 
\end{proof}

\subsection*{Part c}
Show that the map $F:S^3\to S^2$ defined as $F(w,z) = (z\bar{w} + w\bar{z}, iw\bar{z} - iz\bar{w}, z\bar{z} - w\bar{w})$,
is smooth.
\\
\begin{proof}
To show that this map is smooth, we will show it is smooth in the ambient space $\C^2\setminus\{0\}$
and $\R^3\setminus\{0\}$. 

Now, $F$ is smooth as a map from the ambient spaces, which is clear when viewing it as a
map from $\R^4\setminus\{0\}\to\R^3\setminus\{0\}$. Using this, we have that
\[
F(x^1,x^2,x^3,x^4) = (2(x^1x^3 + x^2x^4),2(x^2x^3 - x^1x^4), (x^1)^2 + (x^2)^2 - (x^3)^2 - (x^4)^2)
\] 
which is clearly smooth. Now, since $F$ is smooth in the ambient space, it must also be smooth
when restricted to $S^3\subset \C^2$, since $S^3$ is an embedded submanifold, and $S^2=F(S^3)$
is also an embedded submanifold. This is clear from the definition of the coordinate charts
of embedded submanifolds, which are slices of coordinate charts on the ambient space.
\end{proof}
%----------------------------------------------------------------------------------------
\pagebreak
%----------------------------------------------------------------------------------------
%	PROBLEM 3
%----------------------------------------------------------------------------------------
\section*{Problem 2-7}
Show that for $M$ a nonempty smooth $n$-manifold, with $n\geq 1$, the vector space $C^{\infty}(M)$
is infinite dimensional.
\\
\begin{proof}
Let $\{U_i\}$ be a set of open subsets of $M$ that are all pairwise disjoint, and consider
the set of $C^{\infty}$ functions $\{f_i\}$ on $M$ such that $\supp(f_i)\subset U_i$.
Such a construction is done using partitions of unity subordinate to a carefully chosen
open cover of $M$.

Now, it is easy to see each $f_i$ is linearly independent of the others. To see this,
suppose for a contradiction that for some $f_0\in\{f_i\}$, $f_0 = \sum_{i\neq 0} a_if_i$.
Let $x\in\supp(f_0)$. In particular, we have $f_0(x)\neq 0$. However, since the supports
of $\{f_i\}$ are all pairwise disjoint, it must be that $f_i(x)=0$ for all $f_i\neq f_0$.
Thus we have
\[
\begin{aligned}
f_0(x)  &= \sum_{i\neq 0}a_if_i(x)\\
        &= \sum_{i\neq 0}a_i (0)\\
        &= 0
\end{aligned}
\]
which contradicts the fact that $f_0(x)\neq 0$.

Now, since an arbitrary number of disjoint open sets can be constructed on $M$, it follows
that there are arbitrarily many linearly independent functions in $C^{\infty}(M)$,
so it is infinite dimensional.
\end{proof}
%----------------------------------------------------------------------------------------
\pagebreak
%----------------------------------------------------------------------------------------
%	PROBLEM 4
%----------------------------------------------------------------------------------------
\section*{Problem 2-10}
Consider the algebra $C(M)$ of continuous functions on $M$, and observe that a map$f:M\to N$
induces a map $f^*:C(N)\to C(M)$ via pre-composition.
\subsection*{Part a}
Show that $f^*$ is linear.
\\
\begin{proof}
Let $g,h\in C(N)$, and $\alpha,\beta\in\R$. Now,
\[
\begin{aligned}
f^*(\alpha g+\beta h)(x)    &= (\alpha g + \beta h)\circ f(x)\\
                            &= \alpha g(f(x) + \beta h(f(x))\\
                            &= \alpha f^*(g) + \beta f^*(h)
\end{aligned}
\]
Thus, $f^*$ is linear.
\end{proof}

\subsection*{Part b}
Show that $f$ is smooth if and only if $f^*(C^{\infty}(N))\subseteq C^{\infty}(M)$.
\\
\begin{proof}
(=>)
Assume that $f:M\to N$ is smooth. Then, for any $g\in C^{\infty}(N)$, we have
$f^*(g) = g\circ f$, which is the composition of smooth functions, and thus
$f^*(g)\in C^{\infty}(M)$. Therefore, $f^*(C^{\infty}(N))\subseteq C^{\infty}(M)$ as desired. 

(<=)
Now, suppose $f$ is such that $f^*(C^{\infty}(N))\subseteq C^{\infty}(M)$. In particular,
for any coordinate chart $\phi$ on $N$, we have $f^*(\phi)\in C^{\infty}(M)$. That is,
for any chart $\psi$ on $M$, we have
\[
\begin{aligned}
\phi\circ f &\in C^{\infty}(M)\\
\implies \phi\circ f\circ\psi^{-1} &\in C^{\infty}(\R)
\end{aligned}
\]
Since this works for any $\phi$ on $N$ and $\psi$ on $M$, it follows that $f$ is smooth.
\end{proof}

\subsection*{Part c}
Given a homeomorphism $f:M\to N$, show that $f$ is a diffeomorphism if and only if
$f^*$ restricts to an isomorphism $f^*:C^{\infty}(N)\to C^{\infty}(M)$
\\
\begin{proof}
Observe first that since $f$ is a homeomorphism, $f^{-1}$ is well-defined and continuous.

(=>) Suppose $f$ is a diffeomorphism. In particular, this means $f$ and $f^{-1}$ are smooth.
By the previous result, we have that
\[
\begin{aligned}
f^*(C^{\infty}(N)) &\subseteq C^{\infty}(M)\\
f^{-1^*}(C^{\infty}(M)) &\subseteq C^{\infty}(N)
\end{aligned}
\]
In particular, we have that $f^*$ and $f^{-1^*}$ are surjective by the following argument.

Let $g\in C^{\infty}(M)$. Then, $f^{-1^*}(g) = g\circ f^{-1} \in C^{\infty}(N)$, and
$f^*(f^{-1^*}(g) = g\circ f^{-1}\circ f = g$. Thus, $f^*$ is surjective (more specifically,
$(f^{-1})^* = f^{-1^*}$ on $C^{\infty}(N)$).

By the same argument, $f^{-1^*}$ is surjective and the inverse of $f^*$. Thus,
$f^*$ is an isomorphism as desired.

(<=)
Now, suppose $f^*$ restricts to an isomorphism between $C^{\infty}(N)$ and $C^{\infty}(M)$.
In particular, this means that
$f^*(C^{\infty}(N)) \subseteq C^{\infty}(M)$, which implies $f$ is smooth. Now, the above
argument suggests that the same argument for $f^{-1^*} = (f^{-1})^*$ shows that $f^{-1}$ is smooth
as well. Thus, $f$ and $f^{-1}$ are smooth, and $f$ is a diffeomorphism.
\end{proof}

%----------------------------------------------------------------------------------------
\pagebreak
%----------------------------------------------------------------------------------------
%	PROBLEM 5
%----------------------------------------------------------------------------------------
\section*{Problem 2-14}
For $A$ and $B$ disjoint closed subsets of a smooth manifold $M$, show that there
exists $f\in C^{\infty}$ such that $0\leq f\leq 1$, $f^{-1}(0) = A$, and $f^{-1}(1) = B$.
\\
\begin{proof}
Since $A$ and $B$ are disjoint, there exists an open neighborhood $V$ such
that $B\subset V$ and $V\cap A=\emptyset$.

Now, let $f_A$ be a function constructed as in theorem 2.29. In particular, it is positive, and
$f_A^{-1}(0) = A$. Now, construct another function $\psi_B$ to be a smooth bump
function for $B$ on $V$. In particular, it is positive, $\psi^{-1}(1) = B$ and $\supp(\psi) \subset V$.

Now, consider the function
\[
f(x) = \frac{f_A(x) + \psi_B(x)}{f_A(x)+1}
\]
which is defined everywhere, since $f_A$ is positive. This function is zero only
when $f_A$ and $\psi_B$ are identically zero, which is only on $A$ by construction of
$f_A$, and $f(x) = 1$ only when $f_A(x) + \psi_B(x) = f_A(x) + 1$, or when $\psi_B(x) = 1$,
which is only on $B$.

Thus, $f$ fulfills the properties desired.
\end{proof}
%----------------------------------------------------------------------------------------
\pagebreak
%----------------------------------------------------------------------------------------
%	PROBLEM 6
%----------------------------------------------------------------------------------------
\section*{Problem 3-5}
Let $S^1\subset\R^2$, and let $K=\partial [-1,1]^2\subset\R^2$. Show that there
is a homeomorphism $F:\R^2\to\R^2$ such that $F(S^1) = K$, but there is no diffeomorphism
with the same property.
\\
\begin{proof}
To begin with, we show that there exists such a homeomorphism. Define $F$ to be
the function which moves the point $(x,y)$ along its direction vector in proportion
to its norm such that one coordinate is $1$. Such a speed can be chosen for each
direction, and since the distance from $S^1$ to $K$ varies continuously, the speeds
at which the points move vary continuously as well, and $F$ is a homeomorphism.

Now, we will show that there is no such diffeomorphism. Let $F$ be any homeomorphism that
sends $S^1$ to $K$. In particular, there exists a neighborhood $U$ of a point $p\in S^1$ 
such that $F(U)$ maps to a corner $F(p)$ in $K$. Now, consider a centered coordinate system
at $p$ in $U$, which defines a smooth curve $\gamma$ passing through $p$ at time zero.
In particular, $F$ maps $\gamma$ to a smooth curve on $K$ passing through the corner $F(p)$
at time zero.

Let's consider what $F$ does to $\gamma'(0)$.
\[
\begin{aligned}
dF(\gamma'(0)) &= \partial_t|_0 F(\gamma(0))
\end{aligned}
\]
But on the square, we know that (supposing without loss of generality that $F(\gamma)$ moves
counterclockwise and passes through the first quadrant corner at time zero) the tangent
vector to $F(\gamma)$ before the corner has a $x$ component of zero, and a nonzero $y$ component,
but after the corner has a $y$ component of zero, and a nonzero $x$ component. Since
the velocity vector can never be identically zero, it cannot be continuous at the corner.
Thus, $F(\gamma)$ is not a smooth curve, which is a contradiction.
\end{proof}
%----------------------------------------------------------------------------------------
\pagebreak
%----------------------------------------------------------------------------------------
%	PROBLEM 7
%----------------------------------------------------------------------------------------
\section*{Problem 3-6}
For $z^1, z^2$ in $S^3$, let $\gamma_z:\R\to S^3$ be a curve defined by
$\gamma_z(t) = (\exp(it)z^1,\exp(it)z^2)$. Show that $\gamma_z$ is a smooth curve whose
velocity is never zero.
\\
\begin{proof}
To begin with, we observe that $\gamma_z(t) = \exp(it)(z^1,z^2)$. Now, since $\exp(it)$ is
a smooth function from $\R$ to $S^1$, and $S^1\subset S^3$, and the group operation
of multiplication on $S^3$ is smooth (since $S^3$ is a Lie group, namely the unit quaternionic
sphere, with quaternion multiplication as the group operation), the composition 
(which is just $\exp(it)(z^1,z^2)$)
is smooth as well.

Now to show that $\gamma'$ is never zero. To do so, let $x,y,u,v$ be coordinates
on $S^3$. Then, the differential $d\gamma$ is given in matrix form as
\[
\partial_t \gamma^i = \left(x\cos(t) - y\sin(t),\ x\sin(t)+y\cos(t),\ u\cos(t) - v\sin(t),\ u\sin(t) + v\cos(t)\right)^T
\]
Which, since we have that $x^2 + y^2 = u^2 + v^2 = 1$, it follows that $x$ and $y$ are
never both identically zero, along with $u$ and $v$. Thus, it is never the 
case that the pushforward $d\gamma(\partial_t) = \gamma'$ is zero.
\end{proof}
%----------------------------------------------------------------------------------------
\pagebreak
%----------------------------------------------------------------------------------------
%	PROBLEM 8
%----------------------------------------------------------------------------------------
\section*{Problem 3-7}
Show that the map $\Phi:\mathscr{D}_p\to T_pM$ given by $\Phi(v)(f) = v([f]_p)$ is an
isomorphism. ($\mathscr{D}_p$ is the vector space of linear derivations of germs of
functions at $p$).
\\
\begin{proof}
To begin with, we observe that the map $\Phi$ is clearly linear. Furthermore, it is injective. 
This is clear, since
if we have that $\Phi(x)(f)=\Phi(y)(f)$ for $x,y$ in $\mathscr{D}_p$ and all $f\in C^{\infty}(M)$,
then it follows that $x([f]_p) = y([f]_p)$ for all $f$. In particular, it holds for all
equivalence classes, so $x$ must equal $y$.

$\Phi$ is also clearly surjective. Let $x\in T_pM$. In particular, the linear derivation
$\tilde{x}$, operating on germs by $\tilde{x}([f]_p) = x(f)$ gets mapped by $\Phi$
as $\Phi(\tilde{x}) = x$. Now, $\tilde{x}$ is well defined, by a straightforward application
of Proposition 3.8.

Thus, $\Phi$ is a linear isomorphism, as desired.
\end{proof}

%----------------------------------------------------------------------------------------
\pagebreak
%----------------------------------------------------------------------------------------
%	PROBLEM 9
%----------------------------------------------------------------------------------------
\section*{Problem 3-8}
For $M$ a smooth manifold, and $p\in M$, let $\mathscr{V}_pM$ be the set of equivalence
classes of smooth curves starting at $p$ under the relation $\gamma_1 \sim \gamma_2$ if
for all $f\in C^{\infty}(M)$, $(f\circ\gamma_1)'(0) = (f\circ\gamma_2)'(0)$. Show that
the map $\Psi:\mathscr{V}_pM\to T_pM$ defined as $\Psi[\gamma] = \gamma'(0)$ is
well defined and bijective.
\\
\begin{proof}
To begin with, we show that this map is well defined. To do so, let $\gamma_1$ and $\gamma_2$
be equivalent in the sense defined above. In particular, this means that $d\gamma_1(\partial_t|_0)(f) = d\gamma_2(\partial_t|_0)$
for all $f$ in $\C^{\infty}(M)$. Thus, since the differentials are functions on $C^{\infty}(M)$
that are identical for all $f$, we have that $d\gamma_1(\partial_t|_0) = d\gamma_2(\partial_t|_0)$
which implies $\gamma_1'(0) = \gamma_2'(0)$ as desired.

Now, let's show that this is bijective. To do so, we will first show $\Psi$ is surjective.
Let $v$ be some vector in $T_pM$. In particular, $v = v^i\frac{\partial}{\partial x^i}|_p$
for some coordinates $x^i$ centered at $p$. Now, define a curve $\gamma:[0,1]\to M$ as $\gamma^i(t) = tv^i$.
It is clear that $\gamma'(0) = v$, since 
$\gamma'^i(0) = v^i$, which implies $\gamma'(0) = v^i\partial_i = v$ as desired.

Second, we will show $\Psi$ is injective. This is immediate from the definition of the 
equivalence relation, since by the argument for well-definedness if $\gamma_1'(0)=\gamma_2'(0)$,
then $\gamma_1\sim\gamma_2$.

Thus, $\Psi$ is bijective, as desired.
\end{proof}
%----------------------------------------------------------------------------------------
\pagebreak
%----------------------------------------------------------------------------------------
%	PROBLEM 10
%----------------------------------------------------------------------------------------
\section*{Problem 3-4}
Show $TS^1 \cong S^1\times\R$.
\\
\begin{proof}
To prove this, we first note that there is a natural group structure on $S^1$ when thought
of as a subset of $\C^*$, namely the multiplicative structure from $\C^*$. This is clearly
a Lie group, since the map $(\theta,\phi)\mapsto\theta\phi^{-1}$ is smooth. To see this,
consider the fact that, in $\C^*$, the map $(z_1,z_2)\mapsto z_1z_2^{-1}$ from $\C^*$ to
itself is clearly smooth, since multiplication, and inversion are smooth operations.
Thus, $S^1$ is a Lie group under this operation.

Consider the space $\mathfrak{g}$, the set of all left-invariant vector fields on a Lie group $G$.
Here, a vector field on a Lie group $G$ is said to be \em left-invariant\em\ if for all $\sigma\in G$,
we have that 
\[
dl_{\sigma}\circ X = X\circ l_{\sigma}
\]
for $l_{\sigma}$ the operation of left-multiplication by $\sigma$.
Clearly, this forms a vector space, with addition and scalar multiplication inherited
from the tangent spaces. It is clearly closed under these operations, since
\[
\begin{aligned}
(X + Y)\circ l_{\sigma} &= X\circ l_{\sigma} + Y\circ l_{\sigma}\\
                        &= dl_{\sigma}\circ X + dl_{\sigma}\circ Y\\
                        &= dl_{\sigma}\circ(X + Y)
\end{aligned}
\]
And, for $r\in\R$,
\[
(rX)\circ l_{\sigma} = r(X\circ l_{\sigma}) = r(dl_{\sigma}\circ X) = dl_{\sigma}\circ rX
\]
Thus, $\mathfrak{g}$ is a real vector space.

Now, we establish an isomorphism between $\mathfrak{g}$ and
the tangent space $T_eG$ given by $\alpha:\mathfrak{g}\to T_eG$, $\alpha(X) = X(e)$.

Now, $\alpha$ is clearly linear, so we just need to show it is injective and surjective.
To see this, let $\alpha(X) = \alpha(Y)$. Then, for each $\theta\in G$, we have
\[
\begin{aligned}
X(\theta)   &= dl_{\theta}X(e)\\
            &= dl_{\theta}Y(e)\\
            &= Y(\theta)
\end{aligned}
\]
Thus, $\alpha(X)=\alpha(Y)$ implies that $X=Y$, so $\alpha$ is injective.

To show surjectivity, let $x\in T_eG$. Then, define a vector field $X$ to be $X(\sigma) = dl_{\sigma}(x)$.
Clearly, $X$ is left-invariant, since for all $\theta,\sigma \in G$, we have
\[
X(l_{\sigma}\theta) = X(\theta\sigma) = dl_{\sigma\theta}(x) = dl_{\sigma}dl_{\theta}(x) = dl_{\sigma}X(\theta)
\]
Here, we used the functoriality of $d$ to split $dl_{\sigma\theta} = dl_{\sigma}dl_{\theta}$.

Now, it is clear that $\alpha(X) = X(e) = x$, so $\alpha$ is surjective as well.
Therefore, the tangent space $T_eG$ is isomorphic to the set $\mathfrak{g}$ of left-invariant
vector fields on $G$. 

This establishes the basic isomorphism we will use. Define $\Phi:G\times T_eG\to TG$
by 
\[
\Phi(\sigma,x) = dl_{\sigma}\alpha^{-1}(x)
\] 
That is, for a vector $x\in T_eG$, identify
it with the left-invariant vector field $X\in\mathfrak{g}$ by $\alpha(X) = x$. Then,
$\Phi$ takes the tangent vector $x$ and sends it to the tangent vector $X(\sigma)$.

$\Phi$ can be shown to be a smooth bijection. First, we will show it is surjective and
injective, then we will show it is smooth.

First, let $\Phi(\theta_1,x_1) = \Phi(\theta_2,x_2)$. Clearly, $\theta_1 = \theta_2$, since
if $\Phi(\theta_1,x_1) = \Phi(\theta_2,x_2)$, then its projections back to $G$ must
be equal as well. Thus$\theta_1 =\theta_2$.
Now, let $X_i = \alpha^{-1}(x_i)$. Then, we have that $X_1(\theta)=X_2(\theta)$.
Since $X_i$ is left-invariant, we must have that
\[
X_1(e) = dl_{\theta^{-1}}\circ X_1(\theta) = dl_{\theta^{-1}}\circ X_2(\theta) = X_2(e)
\]
So $x_1 = x_2$ and $\Phi$ is injective.

Second, let $(\sigma,x)\in TG$. Clearly, $\Phi(\sigma,x) = X(\sigma) = (\sigma,x)$ by
the definition of $\Phi$, so $\Phi$ is surjective as well.

Now we can see also that $\Phi$ is smooth. To do so, let's choose a coordinate chart $(U,\phi)$
centered at $e$ given as $(x_1,\ldots,x_n)$ (which naturally gives a basis for $T_eG$
as $\{\partial_1|_e,\ldots,\partial_n|_e\}$).
This chart induces a chart at $\theta$ given on $l_{\theta}(U)$ by $\phi\circ l_{\theta^{-1}}$,
and induces a basis on $T_{\theta}G$ by pushing forward $\partial_i|_e$ along $dl_{\theta}$
to get $\partial_i|_{\theta}$.

So, for any $(\theta, x)\in G\times T_eG$, we have the coordinate chart $(l_{\theta}U\times T_eG,\tilde{\phi})$
given as
\[
\widetilde{\phi}(\sigma,x^i\partial_i|_e) = (\phi(l_{\theta^{-1}}(\sigma)), x^i)
\]
Recall also that we need a coordinate chart on $TG$, but this is induced from the
coordinate chart defined above. In particular, (for $\pi$ the standard projection map
from $TG$ to $G$) on $\pi^{-1}(l_{\theta}(U))$ we have the
chart:
\[
\widetilde{\varphi}(\sigma,x^i\partial_i|_{\sigma}) = (\phi(l_{\theta^{-1}}(\sigma)),x^i)
\]
We note that this chart is smooth, since the basis $\partial_i|_{\sigma}$ arises from
the left-invariant vector field given by $\alpha^{-1}(\partial_i|_e)$, which smoothly varies
across the manifold.

Now, let's compute the transition map $\widetilde{\varphi}\circ\Phi\circ\widetilde{\phi}^{-1}$.
For $\sigma$ in the coordinatized neighborhood of $\theta$, we have
\[
\begin{aligned}
\widetilde{\varphi}\circ\Phi\circ\widetilde{\phi}^{-1}(\phi(l_{\theta^{-1}}(\sigma)),x^i) &= \widetilde{\varphi}\circ\Phi(\sigma,x^i\partial_i|_e)\\
                &=\widetilde{\varphi}(\sigma,dl_{\sigma}(x^i\partial_i|_e))\\
                &=\widetilde{\varphi}(\sigma,(x^i\partial_i|_{\sigma}))\\
                &=(\phi(l_{\theta^{-1}}(\sigma)),x^i)
\end{aligned}
\]
Which is a smooth function, so $\Phi$ is a diffeomorphism.
Here, we used the fact that $dl_{\sigma}(\partial_i|_e) = \partial_i|_{\sigma}$.

Therefore, the tangent bundle of a Lie group is trivial.
Applying this to the special case of $G=S^1$, we have that $TS^1\cong S^1\times\R$ as
desired.
\end{proof}
%----------------------------------------------------------------------------------------
\pagebreak
%----------------------------------------------------------------------------------------
%	PROBLEM 11
%----------------------------------------------------------------------------------------
\section*{Problem 11}
Let $F:M\to M$ be the identity function on $M$. Show that, for two coordinate systems
$\phi = (x_i)$ and $\psi = (y_i)$ of a point $p$, find the change of basis matrix $dF_p$, and
show that the two charts give rise to compatible charts on the tangent space.
\\
\begin{proof}
To begin with, we note what $dF_p$ does to the basis elements $\partial_{x^j}$.
Since $F$ is the identity, it must be that each basis element gets sent to itself.
However, we must now express the basis vector in the $y^i$ coordinate system. To do
so, we push forward $\partial_{x^j}$ along the $y^i$ coordinate via $dy^i$.
Thus,
\[
\begin{aligned}
\partial_{x^j} &= dy^i(\partial_{x^j})\partial_{y^i}\\
                &= \partial_{x^j}(y^i)\partial_{y^i}
\end{aligned}
\]
so the transformation matrix is just $\partial_{x^j}(y^i)$.

Now to show that the charts are smooth in $TM$. To do so, we compute the transition
chart
\[
\tilde{\phi}\circ\tilde{\psi^{-1}}
\]
Which is given as,
\[
\begin{aligned}
\tilde{\phi}\circ\tilde{\psi^{-1}}(y(p),dy(v))  &= \tilde{\phi}(p,v)\\
                                                &= (x(p), dx(v))\\
\end{aligned}
\]
So, we have that
\[
\tilde{\phi}\circ\tilde{\psi^{-1}} = (\phi\circ\psi^{-1}, \partial_{x^j}(y^i))
\]
which is smooth as desired.

By symmetry of the problem, the reverse transition chart $\tilde{\psi}\circ\tilde{\phi^{-1}}$
is smooth as well.
\end{proof}
%----------------------------------------------------------------------------------------
\end{document}
