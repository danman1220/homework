%%%%%%%%%%%%%%%%%%%%%%%%%%%%%%%%%%%%%%%%%
% Short Sectioned Assignment
% LaTeX Template
% Version 1.0 (5/5/12)
%
% This template has been downloaded from:
% http://www.LaTeXTemplates.com
%
% Original author:
% Frits Wenneker (http://www.howtotex.com)
%
% License:
% CC BY-NC-SA 3.0 (http://creativecommons.org/licenses/by-nc-sa/3.0/)
%
%%%%%%%%%%%%%%%%%%%%%%%%%%%%%%%%%%%%%%%%%

%----------------------------------------------------------------------------------------
%	PACKAGES AND OTHER DOCUMENT CONFIGURATIONS
%----------------------------------------------------------------------------------------

\documentclass[fontsize=11pt]{scrartcl} % 11pt font size

\usepackage[T1]{fontenc} % Use 8-bit encoding that has 256 glyphs
\usepackage[english]{babel} % English language/hyphenation
\usepackage{amsmath,amsfonts,amsthm} % Math packages
\usepackage{mathrsfs}

\usepackage[margin=1in]{geometry}

\usepackage{sectsty} % Allows customizing section commands
\allsectionsfont{\centering \normalfont\scshape} % Make all sections centered, the default font and small caps

\usepackage{fancyhdr} % Custom headers and footers
\pagestyle{fancyplain} % Makes all pages in the document conform to the custom headers and footers
\fancyhead{} % No page header - if you want one, create it in the same way as the footers below
\fancyfoot[L]{} % Empty left footer
\fancyfoot[C]{} % Empty center footer
\fancyfoot[R]{\thepage} % Page numbering for right footer
\renewcommand{\headrulewidth}{0pt} % Remove header underlines
\renewcommand{\footrulewidth}{0pt} % Remove footer underlines
\setlength{\headheight}{13.6pt} % Customize the height of the header

\numberwithin{equation}{section} % Number equations within sections (i.e. 1.1, 1.2, 2.1, 2.2 instead of 1, 2, 3, 4)
\numberwithin{figure}{section} % Number figures within sections (i.e. 1.1, 1.2, 2.1, 2.2 instead of 1, 2, 3, 4)
\numberwithin{table}{section} % Number tables within sections (i.e. 1.1, 1.2, 2.1, 2.2 instead of 1, 2, 3, 4)

\newcommand{\R}{\mathbb{R}}
\newcommand{\Q}{\mathbb{Q}}
\newcommand{\Z}{\mathbb{Z}}
\newcommand{\C}{\mathbb{C}}

%----------------------------------------------------------------------------------------
%	TITLE SECTION
%----------------------------------------------------------------------------------------

\newcommand{\horrule}[1]{\rule{\linewidth}{#1}} % Create horizontal rule command with 1 argument of height

\title{	
\normalfont \normalsize 
\textsc{Geometry} \\ [25pt] % Your university, school and/or department name(s)
\horrule{0.5pt} \\[0.4cm] % Thin top horizontal rule
\huge Problem Set 2\\ % The assignment title
\horrule{2pt} \\[0.5cm] % Thick bottom horizontal rule
}

\author{Daniel Halmrast} % Your name

\date{\normalsize\today} % Today's date or a custom date

\begin{document}

\maketitle % Print the title

%----------------------------------------------------------------------------------------
%	PROBLEM 1
%----------------------------------------------------------------------------------------
\section*{Problem 2-1}
For $f$ the Heaviside step function (with $f(0)=1$), show that $\forall x \in\R$, there
exist smooth charts $(U,\phi)$ around $x$ and $(V,\psi)$ around $f(x)$ such that 
$\psi\circ f\circ\phi^{-1}$ is smooth as a map from its domain to its image, but $f$ is not
smooth in a smooth manifold sense.
\\
\begin{proof}
For $x\neq 0$, neighborhoods avoiding zero can be chosen, and identity charts make $f$
locally smooth. For $x=0$, set $U = (-\epsilon,\epsilon)$, $V = (1-\epsilon,1+\epsilon)$
and have $\phi_U = \psi_V = \textrm{id}$.
Then, on $U\cap f^{-1}(V) = [0,\epsilon)$ we have $\psi\circ f\circ\phi^{-1}(x) = 1$
which is smooth. But this fails the test in proposition 2.5, so $f$ is not smooth in a
manifold sense.
\end{proof}
%----------------------------------------------------------------------------------------

%----------------------------------------------------------------------------------------
%	PROBLEM 2
%----------------------------------------------------------------------------------------
\section*{Problem 2-3}
For each of the following maps, show that the map is smooth via computation through coordinate
representations.
\subsection*{Part a}
The power map $p_n:S^1\to S^1$ defined as $p_n(z) = z^n$.
\\
\begin{proof}
For this problem, we will use two coordinate charts on $S^1$. First, let's parameterize
the circle by $\theta$, so that the point $\theta$ is identified with $\exp(i\theta)$ in the
standard embedding of the circle into $\C$. Then, the first coordinate chart will be for
$\theta\in(0,2\pi)$ given as $\phi(\theta) = \theta$. The second coordinate chart will
be for $\theta\in(-\pi,\pi)$ (where $2\pi\theta \sim\theta$) given as $\psi(\theta) = \theta$.

Now, the transition maps can easily be verified to be smooth. To see this, let $\theta_0$
be a point in the intersection of the two charts. Then, if $\theta\in(0,\pi)$, we have
\[
\begin{aligned}
\phi(\theta) &= \theta\\
\psi(\theta) &= \theta\\
\end{aligned}
\]
Which are easily verified to be smooth and compatible with each other.

Suppose, then, that $\theta\in (\pi,2\pi)$. Then, we have that
\[
\begin{aligned}
\phi(\theta) &= \theta\\
\psi(\theta) &= \theta - 2\pi
\end{aligned}
\]
With transition charts
\[
\begin{aligned}
\phi\circ\psi^{-1}(\theta) = \theta + 2\pi\\
\psi\circ\phi^{-1}(\theta) = \theta - 2\pi
\end{aligned}
\]
which are clearly smooth.

Now, we just have to check that the power function, which can be thought of in terms of
our parameterization as $p_n(\theta) = n\theta (\mod 2\pi)$, is smooth.

So, let's compute some coordinate representations. We have a total of four to check.
\[
\begin{aligned}
\phi\circ p_n\circ\phi^{-1}(\theta) &= n\theta (\mod 2\pi)\\
\psi\circ p_n\circ\psi^{-1}(\theta) &= n(\theta+2\pi)  (\mod 2\pi)- 2\pi\\
\phi\circ p_n\circ\psi^{-1}(\theta) &= n(\theta +2\pi) (\mod 2\pi)\\
\psi\circ p_n\circ\phi^{-1}(\theta) &= n\theta  (\mod 2\pi)- 2\pi
\end{aligned}
\]
Now, addition of a scalar is a smooth operation, so we just have to check that the
function $p_n$ is smooth as a function of $\theta$. 

Now, we observe that $p_n$ is continuous as a function of $\theta$ by viewing $p_n:[0,2\pi)\to\R$
as a continuous function $\theta\mapsto n\theta$, and passing through the quotient $\R/{2\pi\Z}$.
Since the derivative $p_n' = np_{n-1}$ is also of the same form, it is continuous as well,
and by induction each derivative of $p_n$ is continuous, so $p_n$ is smooth.

Thus, the composition maps defined above are smooth, and $p_n$ is a smooth function from
$S^1$ to itself.
\end{proof}

\subsection*{Part b}
The antipodal map $\alpha:S^n\to S^n$ by $\alpha(x) = -x$.
\\
\begin{proof}
Consider the stereographic projection charts $\sigma$ and $\tilde{\sigma}$, where
$\tilde{\sigma}(x) = -\sigma(-x)$. Let's compute some coordinate representations:
\[
\begin{aligned}
\sigma\circ\alpha\circ\sigma^{-1}(x) &= \sigma(-\sigma^{-1}(x))\\
\tilde{\sigma}\circ\alpha\circ\tilde{\sigma}^{-1}(x) &= \tilde{\sigma}(-\tilde{\sigma}^{-1}(x))\\
\sigma\circ\alpha\circ\tilde{\sigma}^{-1}(x) &= \sigma(-\tilde{\sigma}^{-1}(x))\\
\tilde{\sigma}\circ\alpha\circ\sigma^{-1}(x) &= \tilde{\sigma}(-\sigma^{-1}(x))
\end{aligned}
\]
Now, these are all compositions of smooth functions, which are smooth as well.
Thus, the antipodal map is a smooth function. 
\end{proof}

\subsection*{Part c}
Show that the map $F:S^3\to S^2$ defined as $F(w,z) = (z\bar{w} + w\bar{z}, iw\bar{z} - iz\bar{w}, z\bar{z} - w\bar{w})$,
is smooth.
\\
\begin{proof}
To show that this map is smooth, we will show it is smooth in the ambient space $\C^2\setminus\{0\}$
and $\R^3\setminus\{0\}$. 

Now, $F$ is smooth as a map from the ambient spaces, which is clear when viewing it as a
map from $\R^4\setminus\{0\}\to\R^3\setminus\{0\}$. Using this, we have that
\[
F(x^1,x^2,x^3,x^4) = (2(x^1x^3 + x^2x^4),2(x^2x^3 - x^1x^4), (x^1)^2 + (x^2)^2 - (x^3)^2 - (x^4)^2)
\] 
which is clearly smooth. Now, since $F$ is smooth in the ambient space, it must also be smooth
when restricted to $S^3\subset \C^2$.
\end{proof}
%----------------------------------------------------------------------------------------
%----------------------------------------------------------------------------------------
%	PROBLEM 3
%----------------------------------------------------------------------------------------
\section*{Problem 2-7}
Show that for $M$ a nonempty smooth $n$-manifold, with $n\geq 1$, the vector space $C^{\infty}(M)$
is infinite dimensional.
\\
\begin{proof}
%TODO show that functions with disjoint support are independent, then build a bunch of
%bumps.
\end{proof}
%----------------------------------------------------------------------------------------
%----------------------------------------------------------------------------------------
%	PROBLEM 4
%----------------------------------------------------------------------------------------
\section*{Problem 2-10}
Consider the algebra $C(M)$ of continuous functions on $M$, and observe that a map$f:M\to N$
induces a map $f^*:C(N)\to C(M)$ via pre-composition.
\subsection*{Part a}
Show that $f^*$ is linear.

\subsection*{Part b}
Show that $f$ is smooth if and only if $f^*(C^{\infty}(N))\subseteq C^{\infty}(M)$.

\subsection*{Part c}
Given a homeomorphism $f:M\to N$, show that $f$ is a diffeomorphism if and only if
$f^*$ restricts to an isomorphism $f^*:C^{\infty}(N)\to C^{\infty}(M)$
%----------------------------------------------------------------------------------------
%----------------------------------------------------------------------------------------
%	PROBLEM 5
%----------------------------------------------------------------------------------------
\section*{Problem 2-14}
For $A$ and $B$ disjoint closed subsets of a smooth manifold $M$, show that there
exists $f\in C^{\infty}$ such that $0\leq f\leq 1$, $f^{-1}(0) = A$, and $f^{-1}(1) = B$.
%----------------------------------------------------------------------------------------
%----------------------------------------------------------------------------------------
%	PROBLEM 6
%----------------------------------------------------------------------------------------
\section*{Problem 3-5}
%----------------------------------------------------------------------------------------
%----------------------------------------------------------------------------------------
%	PROBLEM 7
%----------------------------------------------------------------------------------------
\section*{Problem 3-6}
%----------------------------------------------------------------------------------------
%----------------------------------------------------------------------------------------
%	PROBLEM 8
%----------------------------------------------------------------------------------------
\section*{Problem 3-7}
%----------------------------------------------------------------------------------------
%----------------------------------------------------------------------------------------
%	PROBLEM 9
%----------------------------------------------------------------------------------------
\section*{Problem 3-8}
For $M$ a smooth manifold, and $p\in M$, let $\mathscr{V}_pM$ be the set of equivalence
classes of smooth curves starting at $p$ under the relation $\gamma_1 \sim \gamma_2$ if
for all $f\in C^{\infty}(M)$, $(f\circ\gamma_1)'(0) = (f\circ\gamma_2)'(0)$. Show that
the map $\Psi:\mathscr{V}_pM\to T_pM$ defined as $\Psi[\gamma] = \gamma'(0)$ is
well defined and bijective.
\\
\begin{proof}
To begin with, we show that this map is well defined. To do so, let $\gamma_1$ and $\gamma_2$
be equivalent in the sense defined above. In particular, this means that $d\gamma_1(\partial_t|_0)(f) = d\gamma_2(\partial_t|_0)$
for all $f$ in $\C^{\infty}(M)$. Thus, since the differentials are functions on $C^{\infty}(M)$
that are identical for all $f$, we have that $d\gamma_1(\partial_t|_0) = d\gamma_2(\partial_t|_0)$
which implies $\gamma_1'(0) = \gamma_2'(0)$ as desired.

Now, let's show that this is bijective. To do so, we will first show $\Psi$ is surjective.
Let $v$ be some vector in $T_pM$. In particular, $v = v^i\frac{\partial}{\partial x^i}|_p$
for some coordinates $x^i$ centered at $p$. Now, define a curve $\gamma:[0,1]\to M$ as $\gamma^i(t) = tv^i$.
It is clear that $\gamma'(0) = v$, since 
$\gamma'^i(0) = v^i$, which implies $\gamma'(0) = v^i\partial_i = v$ as desired.

Second, we will show $\Psi$ is injective. This is immediate from the definition of the 
equivalence relation, since by the argument for well-definedness if $\gamma_1'(0)=\gamma_2'(0)$,
then $\gamma_1\sim\gamma_2$.

Thus, $\Psi$ is bijective, as desired.
\end{proof}
%----------------------------------------------------------------------------------------
%----------------------------------------------------------------------------------------
%	PROBLEM 10
%----------------------------------------------------------------------------------------
\section*{Problem 3-4}
Show $TS^1 \cong S^1\times\R$.
\\
\begin{proof}

\end{proof}
%----------------------------------------------------------------------------------------
%----------------------------------------------------------------------------------------
%	PROBLEM 11
%----------------------------------------------------------------------------------------

%----------------------------------------------------------------------------------------
\end{document}
