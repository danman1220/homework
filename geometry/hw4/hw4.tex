%%%%%%%%%%%%%%%%%%%%%%%%%%%%%%%%%%%%%%%%%
% Short Sectioned Assignment
% LaTeX Template
% Version 1.0 (5/5/12)
%
% This template has been downloaded from:
% http://www.LaTeXTemplates.com
%
% Original author:
% Frits Wenneker (http://www.howtotex.com)
%
% License:
% CC BY-NC-SA 3.0 (http://creativecommons.org/licenses/by-nc-sa/3.0/)
%
%%%%%%%%%%%%%%%%%%%%%%%%%%%%%%%%%%%%%%%%%

%----------------------------------------------------------------------------------------
%	PACKAGES AND OTHER DOCUMENT CONFIGURATIONS
%----------------------------------------------------------------------------------------

\documentclass[fontsize=11pt]{scrartcl} % 11pt font size

\usepackage[T1]{fontenc} % Use 8-bit encoding that has 256 glyphs
\usepackage[english]{babel} % English language/hyphenation
\usepackage{amsmath,amsfonts,amsthm} % Math packages
\usepackage{mathrsfs}

\usepackage[margin=1in]{geometry}

\usepackage{sectsty} % Allows customizing section commands
\allsectionsfont{\centering \normalfont\scshape} % Make all sections centered, the default font and small caps

\usepackage{fancyhdr} % Custom headers and footers
\pagestyle{fancyplain} % Makes all pages in the document conform to the custom headers and footers
\fancyhead{} % No page header - if you want one, create it in the same way as the footers below
\fancyfoot[L]{} % Empty left footer
\fancyfoot[C]{} % Empty center footer
\fancyfoot[R]{\thepage} % Page numbering for right footer
\renewcommand{\headrulewidth}{0pt} % Remove header underlines
\renewcommand{\footrulewidth}{0pt} % Remove footer underlines
\setlength{\headheight}{13.6pt} % Customize the height of the header

\numberwithin{equation}{section} % Number equations within sections (i.e. 1.1, 1.2, 2.1, 2.2 instead of 1, 2, 3, 4)
\numberwithin{figure}{section} % Number figures within sections (i.e. 1.1, 1.2, 2.1, 2.2 instead of 1, 2, 3, 4)
\numberwithin{table}{section} % Number tables within sections (i.e. 1.1, 1.2, 2.1, 2.2 instead of 1, 2, 3, 4)

\newcommand{\R}{\mathbb{R}}
\newcommand{\Q}{\mathbb{Q}}
\newcommand{\N}{\mathbb{N}}
\newcommand{\C}{\mathbb{C}}

\newcommand{\im}{\textrm{im}}

\newtheorem{lemma}{Lemma}
%----------------------------------------------------------------------------------------
%	TITLE SECTION
%----------------------------------------------------------------------------------------

\newcommand{\horrule}[1]{\rule{\linewidth}{#1}} % Create horizontal rule command with 1 argument of height

\title{	
\normalfont \normalsize 
\textsc{Geometry} \\ [25pt] % Your university, school and/or department name(s)
\horrule{0.5pt} \\[0.4cm] % Thin top horizontal rule
\huge Problem Set 4 \\ % The assignment title
\horrule{2pt} \\[0.5cm] % Thick bottom horizontal rule
}

\author{Daniel Halmrast} % Your name

\date{\normalsize\today} % Today's date or a custom date

\begin{document}

\maketitle % Print the title

% Problems
\section*{Problem 1} %Lee 5-3
Prove that if any of the following hold for $S$ an immersed submanifold of a
manifold $M$, then $S$ is embedded:
\begin{itemize}
    \item $S$ has codimension $0$ in $M$.
    \item The inclusion map $S\subseteq M$ is proper.
    \item $S$ is compact.
\end{itemize}

\begin{proof}
    Suppose $S$ has codimension $0$ in $M$. Then, we must show that $S$ is open
    with respect to $M$, and since every open subset of a manifold with
    codimension zero is an embedded submanifold, $S$ will be embedded.

    From the previous homework, we know that an immersion between two manifolds
    of the same dimension is an open mapping, so the image of $S$ under the
    inclusion map must be open. Thus, $S$ is open with respect to $M$, and is an
    embedded submanifold.

    Suppose the inclusion map is proper. Then, by proposition 4.22, since the
    inclusion map is an injective smooth immersion which is proper, $S$ is
    embedded.

    Suppose $S$ is compact. Then, by Proposition 4.22, since the inclusion map
    is an injective smooth immersion from a compact set, $S$ is embedded.
\end{proof}

\newpage

\section*{Problem 2} %Lee 5-4
Show that the image of the curve $\beta:(-\pi,\pi)\to\R^2$ given by
$\beta(t) = (\sin(2t),\sin(t))$ is not an embedded submanifold.
\\
\begin{proof}
To show that this is not an embedded submanifold, we will show that it does not
    get the subspace topology from $\R^2$. To see this, we consider the
    neighborhood $(-1,1)$ of zero, which is open in $\im(\beta)$. In particular,
    this neighborhood cannot be an intersection of an open set in $\R^2$ with
    $\im(\beta)$. This is evident, since any neighborhood of zero in $\R^2$ must
    intersect $\beta(-\pi+\varepsilon)$ and $\beta(\pi-\varepsilon)$ for
    sufficiently small $\varepsilon$.

    So, since any open set in $\R^2$ containing zero also contains
    $\beta(\pi-\varepsilon)$ for all $\varepsilon$ sufficiently small, every
    open neighborhood of $\beta(0)$ in the subspace topology must contain
    $\beta(\pi-\varepsilon)$ for all sufficiently small $\varepsilon$. Since the
    neighborhood $\beta(-1,1)$ does not contain $\beta(\pi-\varepsilon)$, it is
    not open in the subspace topology, and since $(-1,1)$ is open with respect
    to the manifold structure on $\im(\beta)$, it must be that $\im(\beta)$ is
    not an embedded submanifold.
\end{proof}

\newpage

\section*{problem 3} %Lee 5-9
Show that the boundary of the unit square does not have a topology and a smooth
structure for which it is an immersed submanifold of $\R^2$.
\\
\begin{proof}
Suppose for a contradiction that the boundary $\partial I^2$ did admit a smooth
structure for which it is an immersed submanifold. In particular, we know that
    at the corner $(0,0)$, the tangent space $T_0\partial I^2$ is a
    one-dimensional subspace of $T_0\R^2\cong\R^2$. So, by Proposition 5.35,
    there must be a smooth curve $\gamma:(-\varepsilon,\varepsilon)\to\partial
    I^2$ such that $\gamma(0) = (0,0)$ and $\gamma'(0)\neq 0$. Now, if we write
    $\gamma(t) = (x(t),y(t))$ we know that $x(t)$ takes a global minimum at
    $(0,0)$, and thus $x'(0)=0$. Similarly, $y(t)$ attains a global minimum at
    $(0,0)$, so $y'(0)=0$. However, this contradicts the assertion that
    $\gamma'(0)\neq 0$, so no such submanifold structure can exist.
\end{proof}

\newpage

\section*{Problem 4}
Show that for any Lie group $G$, the multiplication map $\mu:G\times G\to G$ is
a smooth submersion.
\\
\begin{proof}
Since $G$ is a Lie group, it follows that $\mu$ is smooth. Now, all that is
    needed to show is that $\mu$ is a submersion.

    From the next problem, we know that
    \[
        dm|_{(e,e)}(X,Y) = X+Y
    \]
    Now, since every vector can be written as the sum of two vectors (trivially
    $X=X+0$), it follows that $dm$ is surjective at the identity. But what about
    elsewhere? Consider that
    \[
        \begin{aligned}
            dm|_{(\sigma,\sigma)}(X,Y) &= X|_{\sigma} + Y_{\sigma}\\
                                &= dl_{\sigma}(X_e + Y_e)\\
                                &=dl_{\sigma}(dm|_{(e,e)}(X,Y)
        \end{aligned}
    \]
    Thus, $dm|_{(\sigma,\sigma)} = dl_{\sigma}\circ dm|_{(e,e)}$ is the
    composition of a surjection and a bijection from
    $T_{(\sigma,\sigma)}G\times G$ to $T_{\sigma}G$, and is therefore surjective
    onto $T_{\sigma}G$.

    Thus, the multiplication map defines a smooth map whose differential is
    everywhere surjective, and is thus a submersion.
\end{proof}

\newpage

\section*{Problem 5}
Let $G$ be a Lie group.
\subsection*{Part a}
for $m:G\times G\to G$ the multiplication map, show that the differential
$dm_{(e,e)}$ is given by
\[
    dm_{(e,e)}(X,Y) = X+Y
\]

\begin{proof}
    Let $s_1$ be the section $s_1:G\to G\times G$ given by $s_1(\tau) =
    (\tau,e)$, and $s_2$ the section given by $s_2(\tau) = (e,\tau)$.
    Furthermore, define the isomorphism
    \[
        \begin{aligned}
            \psi: &(T_eG)^2\to T_e(G\times G)\\
                  &(X,Y) \mapsto ds_1|_eX + ds_2|_eY
        \end{aligned}
    \]
    We note finally that $m\circ s_1 = m\circ s_2 = id_G$, by the definition of
    the section maps.

    Now, let's calculate $dm|_{(e,e)}(\psi(X,Y))(f)$ for some test function $f$.
    \[
        \begin{aligned}
            dm|_{(e,e)}(\psi(X,Y))(f) &= \psi(X,Y)|_e(f\circ m)\\
                    &=(ds_1|_eX+ds_2|_eY)(f\circ m)\\
                    &=ds_1|_eX(f\circ m) + ds_2|_eY(f\circ m)\\
                    &=X(f\circ m\circ s_1) + Y(f\circ m\circ s_2)\\
                    &=Xf + Yf\\
                    &=(X+Y)f
        \end{aligned}
    \]
    Thus, $dm|_{(e,e)}(X,Y)=X+Y$ as desired.
\end{proof}

\subsection*{Part b}
Show that the inversion map $i:G\to G$ given by $i(\tau)=\tau^{_1}$ has a
differential $di|_e(X)=-X$.
\\
\begin{proof}
    We note first that $m(\tau,i(\tau))=e$ for all $\tau$ by the definition of
    inversion. Thus, differentiating both sides (at the identity) yields
    \[
        d(m(\bullet,i(\bullet))) = 0
    \]
    but what is the left hand side? Lets denote the function
    $\tau\mapsto(\tau,i(\tau))$ as $id\times i$. Then, we can
    use the chain rule to evaluate (for $X\in T_eG$)
    \[
        \begin{aligned}
            d(m\circ id\times i)X &= dm\circ d(id\times i)X\\
                                &= dm\circ (id\times di)X\\
                                &= dm(X,di(X))\\
                                &= X+di(X)\\
        \end{aligned}
    \]
    Now, since it must hold for all $X$ that $d(m\circ id\times i)X =0$,
    we must have that $X+di(X) =0$, or $di(X) = -X$.

    In this problem, we used the fact that the differential distributes over
    products of maps, which has been proven in an earlier homework, but follows
    from the functoriality of the differential and the universal property of
    products.
\end{proof}

\newpage

\section*{Problem 6}
Show that if $G$ is a smooth manifold with a group structure so that the
multiplication $m:G\times G\to G$ is smooth, then $G$ is a Lie group.
\\
\begin{proof}
For this problem, we only have to show that the inversion map is smooth. To do
    so, we first show that the map $F:G\times G\to G\times G$ given by
    \[
        F(\tau,\sigma)=(\tau,\tau\sigma)
    \]
    is a bijective smooth local diffeomorphism, and is thus a diffeomorphism.

    To see this, we note first that $F$ is clearly bijective, since for any
    $(x,y)\in G\times G$, we have a unique point $(x,x^{-1}y)$ such that
    $F(x,x^{-1}y) = (x,y)$. Thus, $F$ is bijective.

    Now, we need to show $F$ is smooth. To do this, we observe that $F$ is the
    composition of smooth functions. Specifically,
    \[
        F = (id\times m)\circ (\Delta\times id)
    \]
    Where $\Delta:G\to G\times G$ is given by $\Delta(\tau) = (\tau,\tau)$ and
    is clearly smooth.

    Thus, since $m$, $id$, and $\Delta$ are all smooth, so are their products
    and compositions, and so $F$ is smooth as well.

    Now, let's show that $dF$ is bijective at the identity. To do this, we will
    use the earlier decomposition of $F$ to calculate its differential.
    \[
        \begin{aligned}
            dF &= d(id\times m)\circ d(\Delta\times id)\\
            &= (id\times dm)\circ (d\Delta\times id)\\
            dF(X,Y) &= (X,X+Y)
        \end{aligned}
    \]
    where the last equality was obtained by observing that $dm(X,Y) = X+Y$ at
    the identity. Since the map $(X,Y)\mapsto(X,X+Y)$ is bijective, it must be
    that $dF$ is bijective at the identity.

    Now, since left-multiplication is a diffeomorphism, it follows (by a very
    similar argument to the one in problem 4) that $dF$ is everywhere bijective.
    Thus, $F$ is a local diffeomorphism, and since $F$ is bijective, it is also
    a diffeomorphism.

    Now, the section $s:G\to G\times G$ given by $S(\tau)=(\tau,e)$ is smooth,
    and thus the composition
    \[
        \begin{aligned}
            \pi_2\circ F^{-1}\circ s\\
            \tau\mapsto(\tau,e)\mapsto(\tau,\tau^{-1})\mapsto\tau^{-1}
        \end{aligned}
    \]
    is smooth as well, and is equal to $i$, the inversion map. Thus, $i$ is
    smooth.
\end{proof}

\newpage

\section*{Problem 7}
For $\det:GL_n(\R)\to\R$, compute the differential in the following steps:
\subsection*{Part a}
Show that
\[
    \partial_t|_0 \det(I+tA) = \textrm{tr}(A)
\]
for any $A$.
\\
\begin{proof}
We begin by noting that the determinant can be expressed as
    \[
        \det(I+tA) = \sum_{\sigma\in
        S_n}\epsilon(\sigma)\prod_{i=1}^{n}(I+tA)_i^{\sigma(i)}
    \]
    Now, if we single out the linear term in the product by multiplying by $tA$
    once and then by $I$ the rest of the time, we end up with
    \[
        \begin{aligned}
        \textrm{lin}(\det(I+tA) &= \sum_{\sigma\in S_n}
        \epsilon(\sigma)\sum_{i=1}^m(\prod_{j\neq
        i}I_j^{\sigma(j)})A_i^{\sigma(i)}t\\
            &= \sum_{i=1}^nA_i^it\\
            &=t\textrm{tr}(A)
        \end{aligned}
    \]
    and thus, the derivative at zero is $\textrm{tr}(A)$, as desired.
\end{proof}

\subsection*{Part b}
For $X\in GL_n(\R)$ and $B\in T_XGL_n(\R)$, show that
\[
    d(\det)_X(B) = \det(X)\textrm{tr}(X^{-1}B)
\]

\begin{proof}
For this problem, we will identify $B$ with the derivative of the curve
    \[
        \gamma(t) = X+tB
    \]
    at zero.

    Then
    \[
        \begin{aligned}
            d(\det)_X(\gamma'(0)) &= \partial_t|_0 \det(\gamma(t))\\
                            &= \partial_t|_0(\det(X+tB))\\
                            &= \partial_t|_0(\det(X)\det(I+tX^{-1}B))\\
                            &= \det(X)\partial_t|_0(\det(I+tX^{-1}B))\\
                            &= \det(X)\textrm{tr}(x^{-1}B)
        \end{aligned}
    \]  
    as desired. Here we used the identity provided in the problem, stating that
    $\det(X+tB) = \det(X)\det(I+tX^{-1}B)$.
\end{proof}

\newpage

\section*{Problem 8}
Show that the Hopf action of $S^1$ on $S^{2n+1}$ defined by
\[
    z\cdot w = zw
\]
for $z\in S^1$ and $w\in S^{2n+1}\subset \C^{n+1}$ is a smooth action with
orbits that are disjoint unit circles in $S^{2n+1}$ that union to all of
$S^{2n+1}$.

\begin{proof}
Clearly, this is a smooth map, since it is just complex multiplication, in
$\C^{n+1}\setminus\{0\}$, which is a Lie group, and thus has smooth
multiplicative structure.

Furthermore, its orbits are disjoint, since the orbits of any group action are
disjoint.
Now, fixing $w\in S^{2n+1}$, we have the orbit of $w$  as the image of the map
$z\mapsto wz$ for $z\in S^1$. Since left-multiplication by $w$ is an isometric
    diffeomorphism (since $w$ is of unit length), it follows that $S^1$ is diffeomorphic to its image, which is
the orbit of $w$. Thus each orbit is a copy of $S^1$. Furthermore, since each
$w\in S^{2n+1}$ is in its own orbit, it follows that the union of all the orbits
is $S^{2n+1}$ itself, as desired.
\end{proof}

\end{document}
