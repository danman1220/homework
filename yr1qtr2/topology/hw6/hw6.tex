%%%%%%%%%%%%%%%%%%%%%%%%%%%%%%%%%%%%%%%%%
% Short Sectioned Assignment
% LaTeX Template
% Version 1.0 (5/5/12)
%
% This template has been downloaded from:
% http://www.LaTeXTemplates.com
%
% Original author:
% Frits Wenneker (http://www.howtotex.com)
%
% License:
% CC BY-NC-SA 3.0 (http://creativecommons.org/licenses/by-nc-sa/3.0/)
%
%%%%%%%%%%%%%%%%%%%%%%%%%%%%%%%%%%%%%%%%%

%----------------------------------------------------------------------------------------
%	PACKAGES AND OTHER DOCUMENT CONFIGURATIONS
%----------------------------------------------------------------------------------------

\documentclass[fontsize=11pt]{scrartcl} % 11pt font size

\usepackage[T1]{fontenc} % Use 8-bit encoding that has 256 glyphs
\usepackage[english]{babel} % English language/hyphenation
\usepackage{amsmath,amsfonts,amsthm} % Math packages
\usepackage{mathrsfs}
\usepackage{bbm}
\usepackage{tikz-cd}

\usepackage[margin=1in]{geometry}

\usepackage{sectsty} % Allows customizing section commands
\allsectionsfont{\centering \normalfont\scshape} % Make all sections centered, the default font and small caps

\usepackage{fancyhdr} % Custom headers and footers
\pagestyle{fancyplain} % Makes all pages in the document conform to the custom headers and footers
\fancyhead{} % No page header - if you want one, create it in the same way as the footers below
\fancyfoot[L]{} % Empty left footer
\fancyfoot[C]{} % Empty center footer
\fancyfoot[R]{\thepage} % Page numbering for right footer
\renewcommand{\headrulewidth}{0pt} % Remove header underlines
\renewcommand{\footrulewidth}{0pt} % Remove footer underlines
\setlength{\headheight}{13.6pt} % Customize the height of the header

\numberwithin{equation}{section} % Number equations within sections (i.e. 1.1, 1.2, 2.1, 2.2 instead of 1, 2, 3, 4)
\numberwithin{figure}{section} % Number figures within sections (i.e. 1.1, 1.2, 2.1, 2.2 instead of 1, 2, 3, 4)
\numberwithin{table}{section} % Number tables within sections (i.e. 1.1, 1.2, 2.1, 2.2 instead of 1, 2, 3, 4)

\newcommand{\R}{\mathbb{R}}
\newcommand{\Q}{\mathbb{Q}}
\newcommand{\N}{\mathbb{N}}
\newcommand{\C}{\mathbb{C}}
\newcommand{\Z}{\mathbb{Z}}


\newtheorem{lemma}{Lemma}
%----------------------------------------------------------------------------------------
%	TITLE SECTION
%----------------------------------------------------------------------------------------

\newcommand{\horrule}[1]{\rule{\linewidth}{#1}} % Create horizontal rule command with 1 argument of height

\title{	
\normalfont \normalsize 
\textsc{Homotopy Theory} \\ [25pt] % Your university, school and/or department name(s)
\horrule{0.5pt} \\[0.4cm] % Thin top horizontal rule
\huge Problem Set 6 \\ % The assignment title
\horrule{2pt} \\[0.5cm] % Thick bottom horizontal rule
}

\author{Daniel Halmrast} % Your name

\date{\normalsize\today} % Today's date or a custom date

\begin{document}

\maketitle % Print the title

% Problems
\section*{Problem 1}
Show that $M_g$ does not retract onto $C$ the circle that separates $M_g$ into
two compact surfaces $M_h'$ and $M_k'$ with $M_i'=M_i\setminus D^2$. In
particular, show $M_h'$ does not retract onto its boundary $C$. However, show
$M_g$ does retract onto $C'$ the nonseparating circle.
\\
\\
\begin{proof}
    Suppose for a contradiction that $M_h'$ did retract onto $C$. Then, we would
    have the following diagram:
    \[
\begin{tikzcd}
    C\arrow{r}{i}\arrow[bend right]{rr}{\mathbbm{1}} &M_h'\arrow{r}{r} &C\\
\end{tikzcd}
    \]
    Now, let's calculate some fundamental groups. We know that $\pi_1(C) \cong
    \Z$. Furthermore, if we regard $M_h'$ by its cell structure, we see that
    $M_h' = M_h\setminus D^2$ is homotopic to its $1$-skeleton, which is just a
    wedge of $2h$ circles. Thus, $\pi_1(M_h')\cong \star_{i=1}^{2h}\Z$.

    Applying the $\pi_1$ functor to the diagram above, we reach the conclusion
    that
    \[
\begin{tikzcd}
    \Z\arrow{r}{i_*}\arrow[bend right]{rr}{\mathbbm{1}}
    &\star^{2h}\Z\arrow{r}{r_*} &\Z\\
\end{tikzcd}
    \]
    Now, we let $[f]$ be the generator for $\pi_1(C)$. By considering again
    $M_h'$ as (homotopic to) the $1$-skeleton of $M_h$ (which has a cell
    structure of a regular $2h$-gon with opposite sides identified), we see that
    $i_*([f])$ gets sent to a loop that travels exactly once counterclockwise
    around the $2h$-gon. Reading off this loop as the concatenation of
    generating loops $a_1,a_2,\dots,a_{2h}$ we see that
    \[
        i_*([f]) = [a_1][a_2]\dots[a_{2h}][a_1]^{-1}[a_2]^{-1}\dots[a_{2h}]^{_1}
    \]
    Now, this is a nontrivial map, but if we abelianize the diagram, we see that
    $i_*{}_{Ab}$ is actually the trivial map! Thus, if a retract were to exist,
    we would have
    \[
\begin{tikzcd}
    \Z\arrow{r}{0}\arrow[bend right]{rr}{\mathbbm{1}}
    &\Z^{2h}\arrow{r}{r_*} &\Z\\
\end{tikzcd}
    \]
    which cannot happen. Thus, it follows that $M_g$ as well cannot retract onto
    $C$.

    Now, $M_g$ can retract onto the nonseparating circle $C'$. Consider again
    $M_g$ as a cell complex, namely a $2g$-gon with opposite sides identified.
    The loop $C'$ in this complex is a single pair of opposite sides. We can
    explicitly construct a retract of this complex by deforming it into a
    rectangle whose horizontal edges are $C'$, then sending a point $x$ to its
    projection onto $C'$.
\end{proof}

\section*{Problem 2}
Consider the two arcs $\alpha,\beta$ embedded in $D^2\times I$ as shown in the
book. Prove that the loop $\gamma$ going once around $D^2$ is not nullhomotopic
in $X = D^2\times I\setminus(\alpha\cup\beta)$.
\\
\\
\begin{proof}
    We proceed by explicit calculation of the fundamental group of $X$.
    
    To use van Kampen's theorem, we need to cover $X$ with two open sets. Let
    $A$ be the open set covering $X$ up to $\gamma$ plus a little bit more, so
    that $A$ deformation retracts onto the closed space of $X$ to the left of
    $\gamma$. Similarly, $B$ is the open set covering the right half of $X$ up
    to $\gamma$. Their intersection $A\cap B$ deformation retracts onto the
    slice of $X$ that $\gamma$ outlines.

    Now, we need to calculate the fundamental groups of $A,B$ and $A\cap B$. We
    begin with the fundamental group of $A$. Careful inspection of the space
    $X$ reveals that $A$ is a cylinder $D^2\times I$ with two lines (the two
    halves of $\alpha$) removed, as well as an arc (the slice of $\beta$). If we
    homotope the arc so that one of its endpoints stays fixed while the other
    goes to the other side of the cylinder, we end up with a cylinder with three
    parallel lines removed. This deformation retracts via the straight line
    homotopy to $D^2\times\{1\}\setminus\{a_1,a_2,b\}$ where $a_1,a_2,b$ are the three
    endpoints of the deleted lines corresponding to the two halves of $\alpha$
    and $\beta$ respectively.. We've already calculated the fundamental
    group of this space, however. $\pi_1(A)$ is the free product on
    three generators. Specifically, by letting $[a_1],[a_2],[b]$ be the generators
    of loops around $a_1,a_2,b$ respectively, $\pi_1(A) = \langle
    [a_1],[a_2],[b]|\rangle$.

    We can do the same thing to $B$, noting that here we have $\pi_1(B) =
    \langle [b_1],[b_2],[a]|\rangle$ for $b_1,b_2,a$ being the endpoints of the
    two halves of $\beta$ and $\alpha$ respectively.

    Now, we need to calculate the fundamental group of $A\cap B$. This space,
    however, is clearly just a disk with the four points $a_1,a_2,b_1,b_2$
    removed. Thus, $\pi_1(A\cap B) = \langle [a_1],[a_2],[b_1],[b_2]|\rangle$.

    Let's examine how each of these include into $A$ and $B$. Clearly,
    $i_A([a_1]) = [a_1]$, $i_A([a_2]) = [a_2]$, and similarly for $B$.
    Furthermore, $i_A([b_1]) = [b]$ (if we let $b_1$ be the endpoint fixed by
    the homotopy of $A$), and careful inspection reveals that $i_A([b_2]) =
    -[b]$. The results for $B$ follows similarly. That is
    \[
\begin{aligned}
    i_A([a_1]) &= [a_1]\\
    i_A([a_2]) &= [a_2]\\
    i_A([b_1]) &= [b]\\
    i_A([b_2]) &= -[b]\\
    i_B([a_1]) &= [a]\\
    i_B([a_2]) &= -[a]\\
    i_B([b_1]) &= [b_1]\\
    i_B([b_2]) &= [b_2]\\
\end{aligned}
    \]

    Thus, we have a full presentation of the fundamental group. 
    \[
        \pi_1(X) = \langle [a_1],[a_2],[b],[b_1],[b_2],[a]|
        [a_1]=[a],[a_2]=-[a],[b]=[b_1],-[b]=[b_2]\rangle
    \]
    which reduces to
    \[
        \pi_1(X) = \langle [a],[b]|\rangle
    \]
    Now, careful inspection of $\gamma$ reveals that it is homotopic to
    \[
        [\gamma] = [b_1][a_1][b_2][a_2] = [b][a][b]^{-1}[a]^{-1}
    \]
    which is clearly nontrivial. Thus, $[\gamma] \neq 0$ as desired.
\end{proof}


\section*{Problem 3}
Show that $\pi_1(\R^2\setminus\Q^2)$ is uncountable.
\\
\\
\begin{proof}
    Recall from an earlier homework we proved that for two points $x,y$ in
    $\R^2\setminus \Q^2$, there are uncountably many paths from $x$ to $y$. I
    assert that none of these paths are homotopic. To see this, suppose
    $\gamma_1,\gamma_2$ are two distinct paths from $x$ to $y$. Then,
    $\gamma_1\bar{\gamma}_2$ is a loop based at $x$. In particular, this loop
    defines the boundary of a disk $D^2$ in $\R^2$. Since $\gamma_1\neq
    \gamma_2$, this disk has some interior, and since $\Q^2$ is dense, the
    interior must contain some point in $\Q^2$. Therefore,
    $\gamma_1\bar{\gamma}_2$ cannot be nullhomotopic, and thus
    $\gamma_1\not\simeq\gamma_2$ as desired.

    Thus, for any choice of two paths from $x$ to $y$,we obtain a loop based at
    $x$ which is not homotopic to any other loop of paths from $x$ to $y$. This
    defines an injection from the set of (distinct) pairs of paths from $x$ to
    $y$ (which is uncountable) into $\pi_1(\R^2\setminus\Q^2)$, which implies
    that $\pi_1(\R^2\setminus\Q^2)$ is uncountable as desired.
\end{proof}<++>

\section*{Problem 4}
Let $p:\tilde{X}\to X$ be a covering space, with subspace $A\subset X$, and let
$\tilde{A}=p^{-1}(A)$. Show that the restriction $p:\tilde{A}\to A$ is a
covering space.
\\
\\
\begin{proof}
    We wish to show that $A$ is evenly covered by $p$. So, let $x\in A$, and let
    $U\subset X$ be an open neighborhood such  that $p^{-1}(U)$ evenly covers
    $U$. In particular, $U\cap A$ is open in $A$, and $p^{-1}(U\cap
    A)=p^{-1}(U)\cap p^{-1}(A)$ evenly covers $U\cap A$. This follows, since
    $p^{-1}(U)$ is a union of disjoint open sets, each of which is homeomorphic
    to $U$. Thus, $p^{-1}(U\cap A) = p^{-1}(U)\cap \tilde{A}$ is a union of
        disjoint open sets in $\tilde{A}$ (with the subspace topology) which are
        homeomorphic to $U\cap A$.

        Thus, $p:\tilde{A}\to A$ is a covering space, as desired.
\end{proof}

\section*{Problem 5}
Show that if $p_1:\tilde{X}_1\to X_1$ and $p_2:\tilde{X}_2\to X_2$ are covering
spaces, so is their product.
\\
\\
\begin{proof}
    Let $(x_1,x_2)\in X_1\times X_2$, and let $U_1$, $U_2$ be open neighborhoods
    of $x_1,x_2$ which are evenly covered by $p_1,p_2$ respectively. I assert
    that the product $U_1\times U_2$ is evenly covered by $p_1\times p_2$.

    Let $\tilde{U_i}_{\alpha}$ be an open set in $\tilde{X}_1$ homeomorphic to
    $U_i$ ($p_i(\tilde{U_i}_{\alpha}) = U_i$). Now, since each
    $\tilde{U_i}_{\alpha}$ is disjoint from the others, we know that the
    products will be disjoint as well. That is, the subset
    \[
        \bigcup_{\alpha,\beta}\tilde{U_1}_{\alpha}\times\tilde{U_2}_{\beta}
    \]
    is a union of disjoint products
    $\tilde{U_1}_{\alpha}\times\tilde{U_2}_{\beta}$. Furthermore, each of these
    is homeomorphic to $U_1\times U_2$. This is clear, since
    $p_i|_{\tilde{U_i}_{\alpha}}$ is a homeomorphism, and thus the product
    $p_1\times p_2$ restricted to
    $\tilde{U_1}_{\alpha}\times\tilde{U_2}_{\beta}$ is a homeomorphism as well.

    Thus, $U_1\times U_2$ is evenly covered by $p_1\times p_2$, and $p_1\times
    p_2:\tilde{X}_1\times \tilde{X}_2$ is a covering space.
\end{proof}

\section*{Problem 6}
Let $p:\tilde{X}\to X$ be a covering space with finite fibers. Show that
$\tilde{X}$ is compact Hausdorff if and only if $X$ is.
\\
\\
\begin{proof}
    ($\implies$) Suppose $\tilde{X}$ is compact Hausdorff. Then, since $X$ is
    the image of $\tilde{X}$ under the continuous map $p$, it follows
    immediately that $X$ is compact as well. Now we just need to show $X$ is
    Hausdorff. Now, if $x\in X$, then there exists a neighborhood $U$ of $x$
    which is evenly covered by $p$. In particular, this means that $U$ is
    homeomorphic to a subset $\tilde{U}$ of $\tilde{X}$, which is Hausdorff.
    Thus, it follows that $U$ is Hausdorff, and thus $X$ is locally Hausdorff.
    
    Now, suppose $x,y$ are distinct points in $X$, with Hausdorff neighborhoods
    $U_x$ and $U_y$. If $U_x$ and $U_y$ are disjoint, we have separated $x$ and
    $y$ with open sets, and are done. So, suppose $U_x$ and $U_y$ have nonempty
    intersection. Since $X$ is compact, it is possible to construct an open set
    $V_x\subset U_x$ containing $x$ such that $V_x$ does not intersect $U_x\cap
    U_y$. Thus, $V_x$ and $U_y$ are disjoint, and $x$ and $y$ are separated by
    open sets as desired.
    \\
    \\
    ($\impliedby$)
    Suppose $X$ is compact Hausdorff. Now, it follows immediately that
    $\tilde{X}$ is Hausdorff. To see this, let $x,y$ be distinct points in
    $\tilde{X}$. Now, if $p(x) = p(y)$, then $x$ and $y$ are in the same fiber,
    and since fibers of covering maps are discrete, there exist open sets $U$
    and $V$ that separate $x$ and $y$. 

    Now, suppose $x$ and $y$ are not in the same fiber. Then, let $U,V$ be open
    sets in $X$ separating $p(x)$ and $p(y)$. Then, the inverse images
    $p^{-1}(U)$ and $p^{-1}(V)$ contain $x$ and $y$ respectively, and are
    disjoint, since $U$ and $V$ are disjoint. Thus, $x$ and $y$ are separated,
    and $\tilde{X}$ is Hausdorff as desired.

    Now, we finally prove that $\tilde{X}$ is compact. Let $\mathscr{O} =
    \{U_{\alpha}\}$ be an open cover of $\tilde{X}$. Furthermore, let $U_x$ be
    an evenly covered neighborhood of $x$ for each $x\in X$. Since $p$ is a
    finite covering, this lifts to a finite collection $\{V_x^i\}_{i=1}^n$ of
    subsets of $\tilde{X}$, where each $V_x^i$ is homeomorphic to $U_x$ via $p$.
    In particular, we can refine $\mathscr{O}$ to $\mathscr{O}' =
    \{U_{\alpha}'\}$ a cover for which every open
    set is contained in some $V_x^i$ (Let $U_{\alpha} = \cup_{x,i}(U_{\alpha}\cap
    V_x^i)$). Now, all we need to show is that this refinement has a finite
    subcover. 

    To do so, consider the projection of the refinement $\{p(U_{\alpha}')\}$,
    which forms an open cover of $X$. This has a finite subcover
    $\{p(U_j')\}_{j=1}^m$. Now, since each $p(U_j')$ is an evenly covered
    neighborhood, it follows that $p^{-1}(p(U_j')$ is a disjoint union of a
    finite number of elements of $\mathscr{O}'$. Thus, letting $\{W_{j_k}\}$ be
    these elements, we see that
    \[
        \bigcup_{j,k}W_{j_k} = \bigcup_j p^{-1}(p(U_j)) = p^{-1}(X) =
        \tilde{X}
    \]
    as desired.
\end{proof}

\section*{Problem 7}
Construct a simply connected covering space of the space $X\subset \R^3$  that
is the union of the sphere and a diameter. Do the same when $X$ is a sphere and
a circle intersecting at two points.
\\
\\
\begin{proof}
    For the first space, we can have $\tilde{X}$ be an infinite line of spheres
    joined by line segments. If $x$ is the point where the diameter of $S^2$
    meets $S^2$, then $p^{-1}(x)$ is the right endpoints of each of the line
    segments. This space is simply connected (as it is homotopic
    to the infinite wedge of $S^2$, which is simply connected), and covers $X$
    as desired.

    For the second space, we can have $\tilde{X}$ be the same space. This is
    easily verified to be a covering space of $X$ by noting that for $x,y$ the
    intersection of the circle with $S^2$, we let $p^{-1}(x)$ be the left
    endpoints of the line segments, and $p^{-1}(y)$ be the right endpoints of
    the line segments.
\end{proof}

\section*{Problem 8}
Let $X = \partial I^2 \cup \{\frac{1}{n}\}_{n=1}^{\infty}\times I$. Show that
for every covering space $p:\tilde{X}\to X$ there is some neighborhood of the
left edge of $X$ that lifts homeomorphically to $\tilde{X}$. Conclude that $X$
has no simply connected covering space.
\\
\\
\begin{proof}
    We note first that by the definition of a covering space, for $x$ on the
    left edge of $X$ there is a neighborhood $U$ of $x$ which lifts
    homeomorphically to some $\tilde{U}$ in $\tilde{X}$. I didn't have time to
    prove that the entire left side $\{0\}\times I$  has a neighborhood that
    maps homeomorphically to $\tilde{X}$, so we will assume it does.

    Let $U$ be the neighborhood of $\{0\}\times I$ that maps homeomorphically
    into $\tilde{X}$. Then, in particular there are nontrivial loops in $U$ that
    get lifted to nontrivial loops in $\tilde{X}$ (since $p_*$ is injective).
    Thus, $\tilde{X}$ cannot be simply connected.
\end{proof}

\section*{Problem 9}
Let $X$ be the Hawaiian earring, and let $\tilde{X}$ be its covering space
defined in the book. Find a two-sheeted covering space $Y$ of $\tilde{X}$ such
that the composition $Y\to \tilde{X}\to X$ is not a covering space.
\\
\\
\begin{proof}
    Let $Y$ be the space obtained by taking two copies of $\tilde{X}$ with the
    loops facing each other, and joining them by splitting the outermost loop at
    the peak and joining the edges. Now, let the covering map $p$ be as follows.
    From the center moving outward (symmetrical in both directions) the $n$th
    copy of the Hawaiian earring has the $n$th nested loop mapped to the loop
    joining the two copies of $\tilde{X}$.

    Now, this is clearly a two-sheeted covering, so all that remains is to show
    that $Y$ is not a cover of $X$ as a composition. To see this, note that any
    open neighborhood of the origin (the intersection of the circles)
    necessarily contains some loop, which eventually gets mapped to the loop
    joining the two copies of the Hawaiian earring. But the interior of the loop
    in $X$ contains a single copy of $X$, whereas the interior of the loop in
    $Y$ contains two copies! Thus, $Y$ cannot be a covering, since it cannot
    evenly cover the origin.
\end{proof}

\end{document}
