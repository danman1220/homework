%%%%%%%%%%%%%%%%%%%%%%%%%%%%%%%%%%%%%%%%%
% Short Sectioned Assignment
% LaTeX Template
% Version 1.0 (5/5/12)
%
% This template has been downloaded from:
% http://www.LaTeXTemplates.com
%
% Original author:
% Frits Wenneker (http://www.howtotex.com)
%
% License:
% CC BY-NC-SA 3.0 (http://creativecommons.org/licenses/by-nc-sa/3.0/)
%
%%%%%%%%%%%%%%%%%%%%%%%%%%%%%%%%%%%%%%%%%

%----------------------------------------------------------------------------------------
%	PACKAGES AND OTHER DOCUMENT CONFIGURATIONS
%----------------------------------------------------------------------------------------

\documentclass[fontsize=11pt]{scrartcl} % 11pt font size

\usepackage[T1]{fontenc} % Use 8-bit encoding that has 256 glyphs
\usepackage[english]{babel} % English language/hyphenation
\usepackage{amsmath,amsfonts,amsthm} % Math packages
\usepackage{mathrsfs}
\usepackage{bbm}
\usepackage{tikz-cd}

\usepackage[margin=1in]{geometry}

\usepackage{sectsty} % Allows customizing section commands
\allsectionsfont{\centering \normalfont\scshape} % Make all sections centered, the default font and small caps

\usepackage{fancyhdr} % Custom headers and footers
\pagestyle{fancyplain} % Makes all pages in the document conform to the custom headers and footers
\fancyhead{} % No page header - if you want one, create it in the same way as the footers below
\fancyfoot[L]{} % Empty left footer
\fancyfoot[C]{} % Empty center footer
\fancyfoot[R]{\thepage} % Page numbering for right footer
\renewcommand{\headrulewidth}{0pt} % Remove header underlines
\renewcommand{\footrulewidth}{0pt} % Remove footer underlines
\setlength{\headheight}{13.6pt} % Customize the height of the header

\numberwithin{equation}{section} % Number equations within sections (i.e. 1.1, 1.2, 2.1, 2.2 instead of 1, 2, 3, 4)
\numberwithin{figure}{section} % Number figures within sections (i.e. 1.1, 1.2, 2.1, 2.2 instead of 1, 2, 3, 4)
\numberwithin{table}{section} % Number tables within sections (i.e. 1.1, 1.2, 2.1, 2.2 instead of 1, 2, 3, 4)

\newcommand{\R}{\mathbb{R}}
\newcommand{\Q}{\mathbb{Q}}
\newcommand{\N}{\mathbb{N}}
\newcommand{\C}{\mathbb{C}}

\newtheorem{lemma}{Lemma}
%----------------------------------------------------------------------------------------
%	TITLE SECTION
%----------------------------------------------------------------------------------------

\newcommand{\horrule}[1]{\rule{\linewidth}{#1}} % Create horizontal rule command with 1 argument of height

\title{	
\normalfont \normalsize 
\textsc{Homotopy Theory} \\ [25pt] % Your university, school and/or department name(s)
\horrule{0.5pt} \\[0.4cm] % Thin top horizontal rule
\huge Problem Set 2 \\ % The assignment title
\horrule{2pt} \\[0.5cm] % Thick bottom horizontal rule
}

\author{Daniel Halmrast} % Your name

\date{\normalsize\today} % Today's date or a custom date

\begin{document}

\maketitle % Print the title

\section*{Preliminaries}
Before the homework begins, we prove a few useful lemmas.
\begin{lemma}
    Let $g,h:X\to Y$ be homotopic maps. Then, for any function $f:Y\to Z$ for
    arbitrary $Z$, the push-forwards $f_*g$ and $f_*h$ are also homotopic.
    Similarly, if instead we have $f:Z\to X$, then the pullbacks $f^*g$ and
    $f^*h$ are homotopic.
\end{lemma}
\begin{proof}
    Let $g_t:X\to Y$ be a homotopy with $g_0=g$ and $g_1=h$. Then, for $f:Y\to
    Z$, the homotopy $f_*g_t:X\to Z$ yields
    \[
        \begin{aligned}
            f_*g_0 &= f_*g\\
            f_*g_1 &= f_*h
        \end{aligned}
    \]
    and so $f_*g$ is homotopic to $f_*h$.

    Similarly, if $f:Z\to X$, the homotopy $f^*g_t:Z\to Y$ yields
    \[
        \begin{aligned}
            f^*g_0 &= f^*g\\
            f^*g_1 &= f^*h
        \end{aligned}
    \]
    and so $f^*g$ and $f^*h$ are homotopic.
\end{proof}

\begin{lemma}
    Let $X$ be a contractible space. Then, there exists a homotopy $f_t:X\to X$
    with $f_0=\mathbbm{1}_X$ and $f_1=x_0$ the constant function to some $x_0\in
    X$.
\end{lemma}
\begin{proof}
    Since $X$ is contractible, it has the homotopy type of a point.
    Specifically, there exists a homotopy equivalence $g:X\to\{\cdot\}$ with
    homotopy inverse $h:\{\cdot\}\to X$.

    Now, we will call $h(\{\cdot\}) = x_0$, since the image of $h$ has only one
    point. Since $h$ is the homotopy inverse of $g$, it follows that
    $hg\simeq\mathbbm{1}_X$ with homotopy $f_t:X\to X$ such that
    $f_0=\mathbbm{1}_X$ and $f_1 = hg$. But $hg$ is easily verified to be the
    constant map $x_0$, and the homotopy $f$ satisfies the conditions desired.
\end{proof}

\newpage

% Problems
\section*{Problem 1} %Hatcher 0.9
Show that the retract of a contractible space is contractible.
\\
\\
\begin{proof}
    Let $X$ retract onto $A$ via $r:X\to A$, with $X$ a contractible space.
    We wish to show $A$ is contractible. To do so, we will show that every
    map $g:A\to Y$ for arbitrary $Y$ is nullhomotopic (see next problem).

    Consider the commutative diagram
    \[
        \begin{tikzcd}
        A\arrow{r}{i}\arrow[bend right]{rr}{\mathbbm{1}_A} &X\arrow{r}{r}
            &A\arrow{r}{g} &Y
        \end{tikzcd}
    \]
    Now, $gr$ defines a map from $X$ to $Y$, which we know is nullhomotopic by
    the next problem. In particular, this means that $gr\simeq y_0$ for some
    constant $y_0$, and by Lemma 1, we know that $i^*(gr)\simeq i^*(y_0)=y_0$.
    But $i^*(gr)=gri=g$, and so $g\simeq y_0$.

    Thus, since every map from $A$ to $Y$ is nullhomotopic, it follows that $A$
    is contractible.

\end{proof}

\newpage

\section*{Problem 2} %Hatcher 0.10
Show that a space $X$ is contractible if and only if every map $f:X\to Y$ for
any $Y$ is nullhomotopic. Similarly, show that $X$ is contractible if and only
if every map $f:Y\to X$ for arbitrary $Y$ is nullhomotopic.
\\
\\
\begin{proof}
    This proof is broken into two parts, one for each iff statement.

    For the first statement, we first assume $X$ is contractible, and let
    $f:X\to Y$ be arbitrary. Now, since $x$ is contractible, we know that
    $\mathbbm{1}_X\simeq x_0$ for some constant function $x_0$ (Lemma 2). 
    Thus, it follows that $f_*\mathbbm{1}_X\simeq f_*x_0 = f(x_0)$ where
    $f(x_0)$ is the constant function from $X$ to the point $f(x_0)$. Thus,
    since $f_*\mathbbm{1}_X = f$, we have that $f$ is homotopic to a constant
    map, and is nullhomotopic.

    Now, suppose for every space $Y$ and every map $f:X\to Y$, $f$ is
    nullhomotopic. In particular, take $Y=X$ and $f=\mathbbm{1}_X$. Then, it
    follows immediately that $\mathbbm{1}_X$ is nullhomotopic, and $X$ is
    contractible.
    \\
    \\
    For the second statement, we first assume $X$ is contractible, and let
    $f:Y\to X$ be arbitrary. Now, since $x$ is contractible, we know that
    $\mathbbm{1}_X\simeq x_0$ for some constant function $x_0$ (Lemma 2). 
    Thus, it follows that $f^*\mathbbm{1}_X\simeq f^*x_0 = x_0$.
    Since $f^*\mathbbm{1}_X = f$, we have that $f$ is homotopic to a constant
    map, and is nullhomotopic.

    Conversely, suppose every map $f:Y\to X$ is nullhomotopic. Taking $Y=X$ and
    $f = \mathbbm{1}_X$, we see that $\mathbbm{1}_X$ is homotopic to the
    constant map, and thus $X$ is contractible.
\end{proof}

\newpage

\section*{Problem 3} %Hatcher 0.11
Show that $f:X\to Y$ is a homotopy equivalence if there exist maps $g,h:Y\to X$
such that $fg\simeq\mathbbm{1}_Y$ and $hf\simeq\mathbbm{1}_X$. More generally, show that
$f$ is a homotopy equivalence if $fg$ and $hf$ are homotopy equivalences.
\\
\\
\begin{proof}
    Suppose $f:X\to Y$ and there exist maps $g,h:Y\to X$ such that
    $fg\simeq\mathbbm{1}_X$ and $hf\simeq\mathbbm{1}_Y$.
    Now, from Lemma 1 we have that
    \[
        \begin{aligned}
            gfh &= h^*(gf) \simeq h^*(\mathbbm{1}_X) = h\\
            gfh &= g_*(fh) \simeq g_*(\mathbbm{1}_Y) = g
        \end{aligned}
    \]
    and so $g\simeq h$. Then, again by Lemma 1, we have
    \[
        \begin{aligned}
            fg &\simeq\mathbbm{1}_Y &\text{by hypothesis}\\
            gf &=f^*g \simeq f^*h = hf\simeq \mathbbm{1}_X &\text{by Lemma 1}
        \end{aligned}
    \]
    and so $f$ is a homotopy equivalence with homotopy inverse $g$.

    More generally, suppose $fg$ and $hf$ are homotopy equivalences. That is,
    that there exist functions $\gamma:Y\to Y$ and $\delta:X\to X$ such that
    $fg\gamma\simeq \gamma fg\simeq \mathbbm{1}_Y$ and $hf\delta\simeq \delta hf\simeq
    \mathbbm{1}_X$.

    Now, since $fg\gamma \simeq \mathbbm{1}_Y$, we have that (using Lemma 1)
    \[
        \begin{aligned}
            h_*fg\gamma &\simeq h_*\mathbbm{1}_Y\\
            hfg\gamma &\simeq h\\
            \delta_*hfg\gamma &\simeq \delta_*h\\
            \delta hfg\gamma &\simeq \delta h\\
            g\gamma &\simeq \delta h
        \end{aligned}
    \]
    Now, it is easily verified that $\delta hfg\gamma$ is a homotopy inverse for
    $f$. To see this, note that
    \[
        \begin{aligned}
            \delta hfg\gamma f &\simeq \delta hf &\simeq \mathbbm{1}_X\\
            f\delta hfg\gamma &\simeq f\delta h &\simeq fg\gamma\simeq
            \mathbbm{1}_Y
        \end{aligned}
    \]
    and so $f$ is a homotopy equivalence.
\end{proof}

\newpage

\section*{Problem 4}
Show that a homotopy equivalence $f:X\to Y$ induces a bijection between the set
of path components of $X$ and the set of path components of $Y$, and that $f$
restricts to a homotopy equivalence on each path component. Prove the same for
components instead of path components. Deduce that if the components of $X$
coincide with the path components of $X$, then the same is true for $Y$ homotopy
equivalent to $X$.
\\
\\
\begin{proof}
    There is a functor $Path:Top\to Set$ which takes a space $X$ to the set of
    its path components, and takes maps $f:X\to Y$ to induced maps
    $\tilde{f}:Path(X)\to Path(Y)$ by mapping a path component $X_p$ to the path
    component containing $f(X_p)$. This function is well-defined, since the
    image of a path connected space is path connected.

    This is functorial, since it sends the identity map in $Top$ to the identity
    in $Set$ (since the identity in $Top$ will necessarily map path components
    to themselves), and furthermore it respects composition. That is, for
    $g:X\to Y$ and $f:Y\to Z$, we have
    \[
        \widetilde{fg} = \tilde{f}\tilde{g}
    \]
    This is easily verified by considering a path-component $X_p$ of $X$. Now,
    we have that $\widetilde{fg}(X_p)$ is the path component containing
    $fg(X_p)$, and $\tilde{f}\tilde{g}(X_p)$ is the path component containing
    the image under $f$ of the path component containing $g(X_p)$. But this is
    simply the path component containing $fg(X_p)$, and thus
    $\widetilde{fg}=\tilde{f}\tilde{g}$ as desired.

    Now we just have to show that such a functor is homotopy invariant. That is,
    we wish to show that for $f\simeq g$, we have that $\tilde{f}=\tilde{g}$.
    So, let $f,g:X\to Y$ be two homotopic maps, and let $X_p$ be a path
    component of $X$. We wish to show that the path component containing
    $f(X_p)$ also contains $g(X_p)$. That is, we wish to find a path from
    $f(x_p)$ to $g(x_p)$ for all $x_p\in X_p$. 
    
    Let $F:X\times I\to Y$ be the
    homotopy from $f$ to $g$. Fixing $x_p$, we have a function $F_{x_p}:I\to Y$
    defined by $F_{x_p}(t) = F(x_p,t)$. This function has the property that
    $F_{x_p}(0)=f(x_p)$ and $F_{x_p}(1)=g(x_p)$ since $F$ is the homotopy from
    $f$ to $g$. Thus, $F_{x_p}$ defines a path from $f(x_p)$ to $g(x_p)$ and
    therefore $\tilde{f}(X_p)=\tilde{g}(X_p)$. Since $X_p$ was arbitrary, we
    have that $\tilde{f}=\tilde{g}$ as desired.

    Thus, the $Path$ functor is homotopy invariant. In particular, if $X$ is
    homotopy equivalent to $Y$ via a function $f:X\to Y$ with homotopy inverse
    $g:Y\to X$ (that is, $fg\simeq\mathbbm{1}_Y$ and $gf\simeq\mathbbm{1}_X$),
    we can apply $Path$ to find that 
    \[
        \tilde{f}\tilde{g} =
        \widetilde{fg}=\mathbbm{1}_{Path(Y)}
    \]
    and 
    \[
        \tilde{g}\tilde{f} = \widetilde{gf}=\mathbbm{1}_{Path(X)}
    \]
    and so $\tilde{f}$ is a bijection with inverse $\tilde{g}$ between the set
    of path components of $X$ and the set of path components of $Y$.
    \\
    \\
    We now wish to show that $f$ restricts to a homotopy equivalence between the
    corresponding path components of $X$ and $Y$. To see this, we first
    show that a homotopy $F$ restricts to the path components in
    a well-defined way. That is, if $F(x_p,0)$ is in a path component $X_p$,
    then $F(x_p,t_0)$ is in that path component for all $t_0\in I$.

    This is obvious, since the point $F(x_p,t_0)$ has a path from $F(x_p,0)$ to
    it: namely, the path given by $F(x_p,\frac{t}{t_0})$. Thus, the homotopy $F$
    is well-defined when restricted to a path component.

    So, given that $f$ is a homotopy equivalence with homotopy inverse $g$, we
    can consider the restriction of the homotopy $F$ between $gf$ and
    $\mathbbm{1}$ on a path component to get a homotopy $F|_{X_p}$ from $gf$
    restricted to $X_p$ and $\mathbbm{1}$ restricted to $X_p$.
    Similarly it can be shown that $fg$ restricts on the path component
    $Y_p$ to a map homotopic to $\mathbbm{1}_{Y_p}$, and thus we have that $f$
    restricted to a path component is a homotopy equivalence.
    \\
    \\
    We now wish to prove the same for the connected components of the space.
    We can define a similar functor $Conn:Top\to Set$ taking a space $X$ to its
    connected components, and a function $f:X\to Y$ mapping a connected
    component $X_c$ to the connected component containing $f(X_c)$. In a similar
    argument to the one above, it is easy to see that this is indeed a functor.
    We wish to show that it is homotopy invariant.

    Now, we argued in the previous part of this proof that if $f$ and $g$ are
    homotopic to each other, then $f(x_c)$ is path connected to $g(x_c)$. Since
    this is the case, it follows that $f(x_c)$ and $g(x_c)$ are in the same
    connected component, and thus for connected component $X_c$, $f(X_c)$ and
    $g(X_c)$ are in the same connected component. Thus, $\tilde{f}$ and
    $\tilde{g}$ are equal, and the functor is homotopy invariant.

    By a similar argument to the one made in the path component case, this means
    that for $f$ a homotopy equivalence between $X$ and $Y$, $\tilde{f}$ defines
    a bijection between $Conn(X)$ and $Conn(y)$, as desired.
    \\
    \\
    Now, we wish to show that $f$ restricts to a homotopy equivalence between
    the corresponding path components. Again, we just need to show that a
    homotopy $F$ restricts to components in a well-defined way. That is,
    $F(x_c,t)$ lies in the same component for all $t$.

    However, we have already shown that $F(x_c,t)$ lies in the same path
    component for all $t$, and it follows immediately that $F(x_c,t)$ lies in
    the same connected component as well. Thus, $F$ restricts to the components
    in a well-defined way.

    By the same argument as the one made in the path connected case, this means
    that the homotopy equivalence $f$ restricts to a homotopy equivalence
    between the connected components of $X$ and $Y$, as desired.
    \\
    \\
    Finally, we conclude that if $X$ and $Y$ are homotopy equivalent, and the
    components of $X$ coincide with the path components of $X$, then the same is
    true for $Y$. This is immediate by considering the fact that each component
    $X_c$ is homotopy equivalent to its corresponding component
    $\tilde{f}_c(X_c)$, and thus its path components are in bijection with each
    other. Since $X_c$ has only one path component, so does $\tilde{f}_c(X_c)$
    as desired.
\end{proof}

\newpage

\section*{Problem 5}
Show that two deformation retractions $r^0_t$ and $r^1_t$ from $X$ to $A\subset
X$ can be joined by a continuous family of deformation retractions $r^s_t$,
$s\in I$ of $X$ into $A$.
\\
\\
\begin{proof}
    We define $r_t^s$ to be
    \[
        r^s_t = 
        \begin{cases}
            r^0_{(1-2s)t}, &s\leq \frac{1}{2}\\
            r^1_{2(s-\frac{1}{2})t}, &s\geq \frac{1}{2}
        \end{cases}
    \]
    which is clearly continuous at $s=\frac{1}{2}$ since
    $r^0_{(1-2(\frac{1}{2}))t} = r^0_0 = r^1_0 =
    r^1_{2(\frac{1}{2}-\frac{1}{2})t}$.
\end{proof}

\newpage

\section*{Problem 6}
\subsection*{Part a}
Show that for a map $f:S^1\to S^1$, the mapping cylinder is a CW complex.
\\
\\
\begin{proof}
    We will show that the mapping cylinder is actually the 1-skeleton $S^1\wedge
    I\wedge S^1$ along with a single $2$-cell attached to it. To see this, we
    note that the mapping cylinder is actually
    \[
        S^1_1\times I \coprod S^1_2/{(x_1,1)\sim f(x_1)}
    \]
    Now, $S^1\times I$ is just $I\times I/{(0,t)\sim (1,t)}$ and so the total
    space is just
    \[
        \frac{I\times I \coprod S^1_2}{(0,t)\sim (1,t), (x_1,1)\sim f(x_1)}
    \]
    Which can be further decomposed into
    \[
        \frac{I_s\coprod S^1_1\coprod I\times I\coprod S^1_2}
        {(s)_{I_s}\sim(0,s)_{I\times I}, (x)_{S^1_1}\sim (x,0)_{I\times I},
        (0,t)_{I\times I}\sim (1,t)_{I\times I}, (x_1,1)_{I\times I}\sim
        f(x_1)_{S^1_2}}
    \]
    Here, it is clear that $I_s$, $S^1_1$ and $S^1_2$ form a 1-skeleton. Now, we
    can take the attaching map from $I\times I\cong D^2$ to the 1-skeleton as
    \[
        \begin{aligned}
            \phi^2&:\partial(I\times I)\to X^1\\
            \phi^2(0,s) &= s_{I_s}\\
            \phi^2(1,s) &= s_{I_s}\\
            \phi^2(x,0) &= x_{S^1_1}\\
            \phi^2(x,1) &= f(x)_{S^1_2}
        \end{aligned}
    \]

    Now, $I\times I$ is homeomorphic to $D^2$, and so we can consider $I\times
    I$ to be a 2-cell we attach to the 1-skeleton. Since $\phi$ is a valid
    attaching map,attaching $D^2$ along $\phi$ yields a 2-dimensional cell
    complex. However, this construction is identical to the construction we
    started with (constructing the mapping cylinder) and so the mapping cylinder
    is a CW complex.
\end{proof}

\subsection*{Part b}
Construct a space with both the Mobius band and the annulus $S^1\times I$ as
deformation retracts.
\\
\\
\begin{proof}
    Consider the space constructed as follows:
    Start with the Mobius band $M$, and glue a copy of $S^1\times I$ along the map
    that sends $S^1\times \{1\}$ via the identity to the equator $S^1\subset M$ of the
    Mobius band. Since both the Mobius band and the annulus are CW complexes,
    and the gluing is done between entire subcomplexes, it follows that this
    construction yields a CW complex.

    Now, the Mobius band deformation retracts onto its equator, and so this CW
    complex deformation retracts in the same way by sending $M$ to its equator
    and leaving $S^1\times I$ by itself. This yields just $S^1\times I$, as
    desired.

    $S^1\times I$ also deformation retracts onto $S^1\times \{1\}$, and so this
    CW complex deformation retracts in the same way by leaving $M$ along and
    sending $S^1\times I$ to $S^1\times \{1\}$. This yields just $M$, as
    desired.
\end{proof}

\newpage

\section*{Problem 7}
Show that a CW complex is path connected if and only if its 1-skeleton is path
connected.
\\
\\
\begin{proof}
    ($\implies$)
    Suppose first that a CW complex $X$ is path connected. We will induct on the
    dimension of $X$. The base case of $X$ being one dimensional means that
    $X=X^1$ and thus the 1-skeleton is path connected.

    Now suppose that this theorem holds for an $n-1$-dimensional CW complex. We
    will show that any path in an $n$-dimensional CW complex starting and ending
    on the $n-1$-skeleton is homotopic to a path in the $n-1$-skeleton rel the
    start and end points.

    To see this, we observe two key facts about paths through $n$-cells. First,
    any path in the disk $D^n$ for $n>1$ can be homotoped so as to avoid passing the
    center. Second, any path through the disk $D^n$ that avoids the center is
    homotopic to a path along the boundary $\partial D^n$. This homotopy is just
    the deformation retraction of $D^n\setminus\{0\}$ to $\partial D^n$ given by
    radial projection. 

    Thus, any path through an $n$-cell is homotopic to a path through the $n-1$
    skeleton. Therefore, for any two points $x,y$ in the $n-1$-skeleton, there exists
    a path from $x$ to $y$ (since $X$ is path connected) that stays in the $n-1$
    skeleton. Thus, the $n-1$-skeleton is path-connected, and by the inductive
    hypothesis, the $1$-skeleton is as well.
    \\
    \\
    ($\impliedby$)
    Suppose instead that for a CW complex $X$, its 1-skeleton is path connected.
    Again we will induct on the dimension of $X$. If $X$ is 1-dimensional, then
    $X^1=X$ and so $X$ is path connected.

    Now, suppose this holds for an $n-1$ dimensional CW complex. We will show
    that an $n$ dimensional CW complex with a path-connected 1-skeleton is path
    connected. To see this, note that since $X^1$ is path connected, so is
    $X^{n-1}$. So, all we have to do to get a path from $x$ to $y$ (for $x,y\in
    X$) is find a path from $x$ and $y$ to the $n-1$-skeleton.

    Now, if $x$ or $y$ are already in the $n-1$-skeleton, we are done. Suppose,
    however, that $x$ is not in the $n-1$-skeleton. Then, $x$ must be in the
    interior of some $n$-cell. However, since $D^n$ is path connected, there
    always exists a path from $x\in D^n$ to the boundary $\partial D^n$. This
    path sends $x$ to the $n-1$-skeleton, as desired.

    Thus, $X$ is path connected, as desired.
\end{proof}

\section*{Problem 8}
Show that a CW complex is locally compact if and only if each point has a
neighborhood that meets only finitely many cells.
\\
\\
\begin{proof}
    ($\implies$)
    Suppose that a CW complex $X$ is locally compact. It follows that for any
    point $x\in X$, there is a neighborhood $U$ of $x$ with compact closure.
    However, in proposition A1 of Hatcher, it is asserted that any compact
    subset of a CW complex meets only finitely many cells. Thus, the closure of
    $U$ (and therefore $U$ itself) must meet only finitely many cells. Such a
    neighborhood satisfies the properties desired.
    \\
    \\
    ($\impliedby$)
    Suppose instead that each point $x\in X$ has a neighborhood $U$ that meets
    only finitely many cells. In particular, this means that the closure of $U$
    also meets only finitely many cells. Thus, $\overline{U}$ is a subset of
    $X^n$ for some $n$. As a closed subset of a compact Hausdorff space,
    $\overline{U}$ is compact as well, as desired.
    
    So, each point has a neighborhood with compact closure, and $X$ is locally
    compact.
\end{proof}

\end{document}
