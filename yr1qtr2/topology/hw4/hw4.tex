%%%%%%%%%%%%%%%%%%%%%%%%%%%%%%%%%%%%%%%%%
% Short Sectioned Assignment
% LaTeX Template
% Version 1.0 (5/5/12)
%
% This template has been downloaded from:
% http://www.LaTeXTemplates.com
%
% Original author:
% Frits Wenneker (http://www.howtotex.com)
%
% License:
% CC BY-NC-SA 3.0 (http://creativecommons.org/licenses/by-nc-sa/3.0/)
%
%%%%%%%%%%%%%%%%%%%%%%%%%%%%%%%%%%%%%%%%%

%----------------------------------------------------------------------------------------
%	PACKAGES AND OTHER DOCUMENT CONFIGURATIONS
%----------------------------------------------------------------------------------------

\documentclass[fontsize=11pt]{scrartcl} % 11pt font size

\usepackage[T1]{fontenc} % Use 8-bit encoding that has 256 glyphs
\usepackage[english]{babel} % English language/hyphenation
\usepackage{amsmath,amsfonts,amsthm} % Math packages
\usepackage{mathrsfs}

\usepackage[margin=1in]{geometry}

\usepackage{sectsty} % Allows customizing section commands
\allsectionsfont{\centering \normalfont\scshape} % Make all sections centered, the default font and small caps

\usepackage{fancyhdr} % Custom headers and footers
\pagestyle{fancyplain} % Makes all pages in the document conform to the custom headers and footers
\fancyhead{} % No page header - if you want one, create it in the same way as the footers below
\fancyfoot[L]{} % Empty left footer
\fancyfoot[C]{} % Empty center footer
\fancyfoot[R]{\thepage} % Page numbering for right footer
\renewcommand{\headrulewidth}{0pt} % Remove header underlines
\renewcommand{\footrulewidth}{0pt} % Remove footer underlines
\setlength{\headheight}{13.6pt} % Customize the height of the header

\numberwithin{equation}{section} % Number equations within sections (i.e. 1.1, 1.2, 2.1, 2.2 instead of 1, 2, 3, 4)
\numberwithin{figure}{section} % Number figures within sections (i.e. 1.1, 1.2, 2.1, 2.2 instead of 1, 2, 3, 4)
\numberwithin{table}{section} % Number tables within sections (i.e. 1.1, 1.2, 2.1, 2.2 instead of 1, 2, 3, 4)

\newcommand{\R}{\mathbb{R}}
\newcommand{\Q}{\mathbb{Q}}
\newcommand{\N}{\mathbb{N}}
\newcommand{\C}{\mathbb{C}}

\newtheorem{lemma}{Lemma}
%----------------------------------------------------------------------------------------
%	TITLE SECTION
%----------------------------------------------------------------------------------------

\newcommand{\horrule}[1]{\rule{\linewidth}{#1}} % Create horizontal rule command with 1 argument of height

\title{	
\normalfont \normalsize 
\textsc{Homotopy Theory} \\ [25pt] % Your university, school and/or department name(s)
\horrule{0.5pt} \\[0.4cm] % Thin top horizontal rule
\huge Problem Set 4 \\ % The assignment title
\horrule{2pt} \\[0.5cm] % Thick bottom horizontal rule
}

\author{Daniel Halmrast} % Your name

\date{\normalsize\today} % Today's date or a custom date

\begin{document}

\maketitle % Print the title

% Problems
\section*{Problem 1}
Show that the composition of paths satisfies the following property: if 
$f_0\cdot g_0 \simeq f_1\cdot g_1$, and $g_0\simeq g_1$, then $f_0\simeq f_1$.
\\
\\
\begin{proof}
    Recall from homework 2 that pre-composition and post-composition respect
    homotopies. That is, for $f,g:X\to Y$ with $f\simeq g$, and $h_1:Y\to Z$ and
    $h_2:A\to X$, then $h_{1*}f\simeq h_{1*}g$ and $h_2^*f\simeq h_2^*g$.

    Now, since $g_0\simeq g_1$, we have that
    \[
        f_{0*}g_0\simeq f_{0*}g_1
    \]
    and since $f_0g_0\simeq f_1g_1$, we have that $f_1g_1\simeq f_0g_1$.

    Now, let $g_1^{-1}$ be the inverse path of $g_1$. That is, the path
    $g_1(1-t)$. We have that 
    \[
        \begin{aligned}
            g_1^{-1*}f_1g_1 &\simeq g_1^{-1*}f_0g_1\\
            f_1g_1g_1^{-1} &\simeq f_0g_1g_1^{-1}
        \end{aligned}
    \]
    now, since $g_1g_1^{-1}$ is homotopic to the identity, we have that
    \[
        f_0\simeq f_1
    \]
    as desired.
\end{proof}


\section*{Problem 2}
Show that the change of basepoint homomorphism $\beta_h$ depends only on the
homotopy class of $h$.
\\
\\
\begin{proof}
    Let $g,h$ be paths in a space $X$ with the same starting and ending points,
    and such that $g\simeq h$. We will show that $\beta_h=\beta_g$. In
    particular, we will show that conjugating by $h$ is homotopic to conjugating
    by $g$.

    So, let $f$ be a loop based at the endpoint of $g,h$. We will show that
    $g^{-1}fg \simeq h^{-1}fh$. This, however, is just a straightforward
    application of the pre-composition and post-composition lemma from homework
    2.
    
    To see this, we note that since $g\simeq h$, we have that $f_*g\simeq f_*h$.
    Thus, $fg\simeq fh$. Now, since $g^{-1}\simeq h^{-1}$, we can also write
    \[
        g^{-1}fg\simeq g^{-1}fh = (fh)^*g^{-1}\simeq (fh^*)h^{-1} = h^{-1}fh
    \]
    as desired
\end{proof}


\section*{Problem 3}
For a path-connected space $X$, show that $\pi_1(X)$ is Abelian if and only if
all basepoint-change homomorphisms $\beta_h$ depend only on the endpoints of
$h$.
\\
\\
\begin{proof}
    
\end{proof}

\end{document}
