%%%%%%%%%%%%%%%%%%%%%%%%%%%%%%%%%%%%%%%%%
% Short Sectioned Assignment
% LaTeX Template
% Version 1.0 (5/5/12)
%
% This template has been downloaded from:
% http://www.LaTeXTemplates.com
%
% Original author:
% Frits Wenneker (http://www.howtotex.com)
%
% License:
% CC BY-NC-SA 3.0 (http://creativecommons.org/licenses/by-nc-sa/3.0/)
%
%%%%%%%%%%%%%%%%%%%%%%%%%%%%%%%%%%%%%%%%%

%----------------------------------------------------------------------------------------
%	PACKAGES AND OTHER DOCUMENT CONFIGURATIONS
%----------------------------------------------------------------------------------------

\documentclass[fontsize=11pt]{scrartcl} % 11pt font size

\usepackage[T1]{fontenc} % Use 8-bit encoding that has 256 glyphs
\usepackage[english]{babel} % English language/hyphenation
\usepackage{amsmath,amsfonts,amsthm} % Math packages
\usepackage{mathrsfs}
\usepackage{tikz-cd}

\usepackage{enumitem}

\usepackage[margin=1in]{geometry}

\usepackage{sectsty} % Allows customizing section commands
\allsectionsfont{\centering \normalfont\scshape} % Make all sections centered, the default font and small caps

\usepackage{fancyhdr} % Custom headers and footers
\pagestyle{fancyplain} % Makes all pages in the document conform to the custom headers and footers
\fancyhead{} % No page header - if you want one, create it in the same way as the footers below
\fancyfoot[L]{} % Empty left footer
\fancyfoot[C]{} % Empty center footer
\fancyfoot[R]{\thepage} % Page numbering for right footer
\renewcommand{\headrulewidth}{0pt} % Remove header underlines
\renewcommand{\footrulewidth}{0pt} % Remove footer underlines
\setlength{\headheight}{13.6pt} % Customize the height of the header

\numberwithin{equation}{section} % Number equations within sections (i.e. 1.1, 1.2, 2.1, 2.2 instead of 1, 2, 3, 4)
\numberwithin{figure}{section} % Number figures within sections (i.e. 1.1, 1.2, 2.1, 2.2 instead of 1, 2, 3, 4)
\numberwithin{table}{section} % Number tables within sections (i.e. 1.1, 1.2, 2.1, 2.2 instead of 1, 2, 3, 4)

\newcommand{\R}{\mathbb{R}}
\newcommand{\Q}{\mathbb{Q}}
\newcommand{\N}{\mathbb{N}}
\newcommand{\C}{\mathbb{C}}

\newtheorem{lemma}{Lemma}
%----------------------------------------------------------------------------------------
%	TITLE SECTION
%----------------------------------------------------------------------------------------

\newcommand{\horrule}[1]{\rule{\linewidth}{#1}} % Create horizontal rule command with 1 argument of height

\title{	
\normalfont \normalsize 
\textsc{Homotopy Theory} \\ [25pt] % Your university, school and/or department name(s)
\horrule{0.5pt} \\[0.4cm] % Thin top horizontal rule
\huge Problem Set 4 \\ % The assignment title
\horrule{2pt} \\[0.5cm] % Thick bottom horizontal rule
}

\author{Daniel Halmrast} % Your name

\date{\normalsize\today} % Today's date or a custom date

\begin{document}

\maketitle % Print the title

\section*{Preliminaries}

\begin{lemma}
    For two paths $f,g:I\to X$ with $f\simeq g$ relative to $\partial I$, and
    $h:I\to X$ with $f(1)=g(1)=h(0)$, then $hf\simeq hg$, where $hf$ is the
    path that first traverses $f$ first, and then $h$ (similarly for $hg$).

    Similarly, if $h:I\to X$ with $h(1) = f(0)=g(0)$, then $fh\simeq gh$.
\end{lemma}

\begin{proof}
    Let $F:I\times I\to X$ be the homotopy from $f$ to $g$ relative to $\partial
    I$, and let $h:I\to X$ be such that $f(1)=g(1)=h(0)$. The homotopy between
    $hf$ and $hg$ is given by
    \[
        \begin{aligned}
            H:&I\times I\to X\\
            H(t,s) &=
        \begin{cases}
            F(2t,s), &\text{ if }t<\frac{1}{2}\\
            h(2(t-\frac{1}{2})), &\text { else}
        \end{cases}
        \end{aligned}
    \]
    That is, run the homotopy on the $f$ section of the path, and leave $h$
    alone. Since the homotopy $F$ fixes the endpoints, it follows that
    $F(1,s)=h(0)$ for all $s$, and the homotopy $H$ is well-defined. $H$ is
    clearly continuous, then, by the pasting lemma. Therefore, $H(t,0) =
    hf\simeq H(t,1)=hg$ as desired.

    Suppose instead that $h:I\to X$ is such that $h(1)=f(0)=g(0)$. Then, it
    follows that $\overline{h}(0) = \overline{f}(1)=\overline{g}(1)$ which
    satisfies the hypotheses for the previous result, and so
    $\bar{h}\bar{f}\simeq \bar{h}\bar{g}$, and so
    $\overline{fh}\simeq \overline{gh}$, which immediately implies that
    $fh\simeq gh$ as desired.
\end{proof}

% Problems
\section*{Problem 1}
Show that the composition of paths satisfies the following property: if 
$f_0\cdot g_0 \simeq f_1\cdot g_1$, and $g_0\simeq g_1$, then $f_0\simeq f_1$.
\\
\\
\begin{proof}
%    Recall from homework 2 that pre-composition and post-composition respect
%    homotopies. That is, for $f,g:X\to Y$ with $f\simeq g$, and $h_1:Y\to Z$ and
%    $h_2:A\to X$, then $h_{1*}f\simeq h_{1*}g$ and $h_2^*f\simeq h_2^*g$.
%
%    Now, since $g_0\simeq g_1$, we have that
%    \[
%        f_{0*}g_0\simeq f_{0*}g_1
%    \]
%    and since $f_0g_0\simeq f_1g_1$, we have that $f_1g_1\simeq f_0g_1$.
%
%    Now, let $g_1^{-1}$ be the inverse path of $g_1$. That is, the path
%    $g_1(1-t)$. We have that 
%    \[
%        \begin{aligned}
%            g_1^{-1*}f_1g_1 &\simeq g_1^{-1*}f_0g_1\\
%            f_1g_1g_1^{-1} &\simeq f_0g_1g_1^{-1}
%        \end{aligned}
%    \]
%    now, since $g_1g_1^{-1}$ is homotopic to the identity, we have that
%    \[
%        f_0\simeq f_1
%    \]
%    as desired.

    Since $g_0\simeq g_1$ (assumed to be relative to $\partial I$), we have that
    \[
        f_0g_0\simeq f_0g_1
    \]
    by Lemma 1. Now, since $f_1g_1\simeq f_0g_0$, it follows that $f_1g_1\simeq
    f_0g_1$ by transitivity of $\simeq$.

    Now, letting $\overline{g}$ be the inverse path of $g$, we have that
    \[
        \begin{aligned}
            f_1g_1&\simeq f_0g_1\\
            f_1g_1\overline{g_1} &\simeq f_0g_1\overline{g_1} &\text{by Lemma
            1}\\
            f_1 &\simeq f_0 &\text{by $g_1\overline{g_1}\simeq 0$ and Lemma 1}
        \end{aligned}
    \]
    as desired.
\end{proof}


\section*{Problem 2}
Show that the change of basepoint homomorphism $\beta_h$ depends only on the
homotopy class of $h$.
\\
\\
\begin{proof}
    Let $g,h$ be paths in a space $X$ with the same starting and ending points,
    and such that $g\simeq h$. We will show that $\beta_h=\beta_g$. In
    particular, we will show that conjugating by $h$ is homotopic to conjugating
    by $g$.

    So, let $f$ be a loop based at the endpoint of $g,h$. We will show that
    $\bar{g}fg \simeq \bar{h}fh$. This, however, is just a straightforward
    application of Lemma 1.
    
    To see this, we note that since $g\simeq h$, we have that $fg\simeq fh$.
    Now, since $bar{g}\simeq \bar{h}$, we can also write
    \[
        \bar{g}fg\simeq \bar{g}fh = \bar{g}(fh)\simeq \bar{h}(fh) = \bar{h}fh
    \]
    as desired
\end{proof}


\section*{Problem 3}
For a path-connected space $X$, show that $\pi_1(X)$ is Abelian if and only if
all basepoint-change homomorphisms $\beta_h$ depend only on the endpoints of
$h$.
\\
\\
\begin{proof}
    ($\implies$)
    Suppose that for a path-connected space $X$, we have that $\pi_1(x)$ is
    Abelian. Furthermore, let $h,g$ be two paths in $X$ such that
    $g(0)=h(0)=x_0$ and $g(1)=h(1)=x_1$. Furthermore, let $f$ be a loop based at
    $x_1$. We wish to show that $\beta_h([f]) = \beta_g([f])$ which is equivalent to
    showing that $\beta_{\bar{g}}\beta_h([f]) = [f]$. Now,
    $\beta_{\bar{g}}\beta_h([f])$ is just $[g\bar{h}fh\bar{g}]$. Note however that
    $h\bar{g}$, $g\bar{h}$, and $f$ are loops based at $x_1$. Since $\pi_1(X)$ is
    Abelian, it follows that
    \[
        \begin{aligned}
            [g\bar{h}fh\bar{g}] &= [g\bar{h}][f][h\bar{g}]\\
            &=[f][g\bar{h}][h\bar{g}]\\
            &=[f][g\bar{h}h\bar{g}]\\
            &=[f]
        \end{aligned}
    \]
    as desired.

    ($\impliedby$)
    Suppose that $X$ is such that for any two paths $h,g$ with $h(0)=g(0)=x_0$
    and $h(1)=g(1)=x_1$, we have that $\beta_h = \beta_g$. We wish to show that
    for any two elements $[f_1],[f_2]\in \pi_1(X)$, we have that $[f_1][f_2] =
    [f_2][f_1]$. Alternately, we can show that $[f_1][f_2][\bar{f_1}]=[f_2]$.
    This is obvious, though, since $f_1$ and $f_2$ satisfy the hypotheses for
    $h,g$, which implies that $\beta_{f_1} = \beta_{f_2}$. So,
    \[
        \begin{aligned}
            [f_1][f_2][\bar{f_1}] &= \beta_{\bar{f_1}}[f_2]\\
            &= \beta_{\bar{f_2}}[f_2]\\
        &= [f_2][f_2][f_2^{-1}]\\
        &=[f_2]
        \end{aligned}
    \]
    as desired.
\end{proof}

\section*{Problem 4}
Show that if a subspace $x\subset \R^n$ is locally star-shaped, then every path
in $X$ is homotopic in $X$ to a piecewise linear path. Show specifically this
holds when $X$ is open, and when $X$ is a union of finitely many closed convex
sets.
\\
\\
\begin{proof}
    To begin with, let $\gamma$ be a path in $X$. At each point $\gamma(t)\in
    X$, let $S_t$ be a star-shaped neighborhood around $\gamma(t)$. In
    particular, $\{S_t\}$ is an open cover of $\gamma(I)$, and since $\gamma(I)$
    is compact, it follows that there is some finite subcover $\{S_i\}_{i=1}^n$.
    In particular, we can take a finite subcover such that each point
    $\gamma(t)$ is in at most two open sets in the subcover, and the preimages
    $\gamma^{-1}(S_i)$ and $\gamma^{-1}(S_j)$ are distinct (neither contains the
    other). We further require that each open set in the subcover be an
    interval. Finally, order the subcover sequentially. That is, let $S_1$ be the
    open set containing $\gamma(0)$, and let $S_i$ be the open set that overlaps
    with $S_{i-1}$. Define a set of partition points $\{t_i\}_{i=1}^{n-1}$ such
    that $t_i$ lies in the intersection of $S_i$ and $S_{i+1}$.

    Now, for each $S_i$, let $x_i$ be the distinguished point in the star-shaped
    neighborhood. That is, for each $x\in S_i$, the line segment from $x$ to
    $x_i$ is in $S_i$.

    We are finally ready to describe the homotopy from $\gamma$ to a piecewise
    linear function. We note first that each $S_i$ is simply connected. In
    particular, all paths in $S_i$ with fixed endpoints are homotopic to each
    other.

    On $S_1$, we can homotope the segment of the path $\gamma([0,t_1])$ to the
    path obtained by taking the line segment from $\gamma(0)$ to $x_1$ and then
    the line segment from $x_1$ to $\gamma(t_1)$. Generally, on $S_i$, homotope
    the path $\gamma([t_{i-1},t_i])$ to the path from $\gamma(t_{i-1})$ to
    $x_i$, then from $x_i$ to $\gamma(t_i)$. Finally, in $S_n$, we homotope the
    path $\gamma([t_n,1])$ to the path from $\gamma(t_n)$ to $\gamma(1)$.

    On each open set, we homotoped to a piecewise linear path, and they agree on
    the intersection, and so the resulting path is piecewise linear as desired.
\end{proof}

\section*{Problem 5}
Show that for every space $X$, the following are equivalent:
\begin{enumerate}[label=(\alph*)]
    \item Every map $S^1\to X$ is homotopic to the constant map, with image a
        point.
    \item Every map $S^1\to X$ extends to a map $D^2\to X$.
    \item $\pi(X,x_0) = 0$ for all $x_0\in X$.
\end{enumerate}

\begin{proof}
    ((a)$\implies$(b))
    Suppose $f:S^1\to X$ is such that $f$ is homotopic to a constant map $x_0$.
    In particular, we have a homotopy $F:S^1\times I\to X$ with $F(x,0) = f(x)$
    and $F(x,1)=x_0$.

    Now, the disk $D^2$ is homeomorphic to the quotient $(S^1\times I)/{S^1\times
    \{1\}}$ via the homeomorphism 
    \[
        \begin{aligned}
            \phi&:D^2\to (S^1\times I)/{S^1\times\{0\}}
            \phi(r,\theta) &= [(\theta,r)]
        \end{aligned}
    \]
    Where the coordinates on $D^2$ are polar coordinates with $r \leq 1$,
    $\theta\in[0,2\pi)$, and $[(\theta,r)]$ is the equivalence class of the
    point $(\theta,r)\in S^1\times I$.

    Now, we can define $\tilde{F}:(S^1\times I)/{S^1\times \{1\}}\to X$ to be the
    unique map that makes the diagram
    \[
        \begin{tikzcd}[column sep=small]
            S^1\times I\arrow{rr}{F}\arrow[two heads]{rd}{q} &&X\\
            & (S^1\times I)/{S^1\times\{1\}}\arrow{ur}{\tilde{F}}
        \end{tikzcd}
    \]
    commute. Here, $q$ is the canonical quotient map. $\tilde{F}$ is
    well-defined, since $F$ is constant on the fibers of $q$. This is clear,
    since the only nontrivial fiber of $q$ is the subspace $S^1\times\{1\}$,
    which $F$ sends identically to $x_0$.

    Thus, using the homeomorphism above, we find the map $\tilde{F}\circ\phi$ to
    be an extension of $f$. This is evident, since
    $\tilde{F}\circ\phi|_{\partial D^2}$ is just $\tilde{F}|_{S^1\times\{0\}}$
    which is just $F(x,0)=f(x)$ as desired.

    ((b)$\implies$(c))
    Suppose that any map $f:S^1\to X$ can be extended to a map $\tilde{f}:D^2\to
    X$. Now, let $x_0\in X$ be arbitrary. We will show that $\pi_1(X,x_0)=0$. in
    particular, we will show that any loop based at $x_0$ is homotopic to the
    constant loop.

    So, let $f:S^1\to X$ be a loop such that $f(0) = x_0$. By (b), we know that
    such an $f$ extends to a $\tilde{f}:D^2\to X$. Now, via the homeomorphism
    above, we have a map 
    \[
        \tilde{F}:(S^1\times I)/{S^1\times \{1\}}\to X
    \]
    given by $\tilde{F} = \tilde{f}\circ\phi^{-1}$. Furthermore, for
    $q:S^1\times I\to (S^1\times I)/{S^1\times\{1\}}$ the canonical quotient
    map, we have a map
    \[
        F:S^1\times I\to X
    \]
    given by $F = \tilde{F}\circ q$. Now, 
    \[
        \begin{aligned}
            F|_{S^1\times\{0\}} &= \tilde{F}|_{S^1\times\{0\}}\\
            &= \tilde{f}|_{\partial D^2}\\
            &= f
        \end{aligned}
    \]
    and furthermore
    \[
        F|_{S^1\times\{1\}} = \tilde{F}|_{[S^1\times\{1\}]} = x_1
    \]
    for some $x_1\in X$. This is because $q(S^1\times\{1\}) = [S^1\times\{1\}]$
    is just a single point.

    Thus, $F$ defines a homotopy from $f$ to the constant map $x_1$ as desired.

    ((c)$\implies$(a))
    Suppose that $\pi_1(X,x_0)=0$ for all $x_0\in X$. Trivially, each loop in
    $X$ is homotopic to a constant loop, as desired.
\end{proof}

\end{document}
