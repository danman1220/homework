%%%%%%%%%%%%%%%%%%%%%%%%%%%%%%%%%%%%%%%%%
% Short Sectioned Assignment
% LaTeX Template
% Version 1.0 (5/5/12)
%
% This template has been downloaded from:
% http://www.LaTeXTemplates.com
%
% Original author:
% Frits Wenneker (http://www.howtotex.com)
%
% License:
% CC BY-NC-SA 3.0 (http://creativecommons.org/licenses/by-nc-sa/3.0/)
%
%%%%%%%%%%%%%%%%%%%%%%%%%%%%%%%%%%%%%%%%%

%----------------------------------------------------------------------------------------
%	PACKAGES AND OTHER DOCUMENT CONFIGURATIONS
%----------------------------------------------------------------------------------------

\documentclass[fontsize=11pt]{scrartcl} % 11pt font size

\usepackage[T1]{fontenc} % Use 8-bit encoding that has 256 glyphs
\usepackage[english]{babel} % English language/hyphenation
\usepackage{amsmath,amsfonts,amsthm} % Math packages
\usepackage{mathrsfs}

\usepackage[margin=1in]{geometry}

\usepackage{sectsty} % Allows customizing section commands
\allsectionsfont{\centering \normalfont\scshape} % Make all sections centered, the default font and small caps

\usepackage{fancyhdr} % Custom headers and footers
\pagestyle{fancyplain} % Makes all pages in the document conform to the custom headers and footers
\fancyhead{} % No page header - if you want one, create it in the same way as the footers below
\fancyfoot[L]{} % Empty left footer
\fancyfoot[C]{} % Empty center footer
\fancyfoot[R]{\thepage} % Page numbering for right footer
\renewcommand{\headrulewidth}{0pt} % Remove header underlines
\renewcommand{\footrulewidth}{0pt} % Remove footer underlines
\setlength{\headheight}{13.6pt} % Customize the height of the header

\numberwithin{equation}{section} % Number equations within sections (i.e. 1.1, 1.2, 2.1, 2.2 instead of 1, 2, 3, 4)
\numberwithin{figure}{section} % Number figures within sections (i.e. 1.1, 1.2, 2.1, 2.2 instead of 1, 2, 3, 4)
\numberwithin{table}{section} % Number tables within sections (i.e. 1.1, 1.2, 2.1, 2.2 instead of 1, 2, 3, 4)

\newcommand{\R}{\mathbb{R}}
\newcommand{\Q}{\mathbb{Q}}
\newcommand{\N}{\mathbb{N}}
\newcommand{\C}{\mathbb{C}}

\newtheorem{lemma}{Lemma}
%----------------------------------------------------------------------------------------
%	TITLE SECTION
%----------------------------------------------------------------------------------------

\newcommand{\horrule}[1]{\rule{\linewidth}{#1}} % Create horizontal rule command with 1 argument of height

\title{	
\normalfont \normalsize 
\textsc{Homotopy Theory} \\ [25pt] % Your university, school and/or department name(s)
\horrule{0.5pt} \\[0.4cm] % Thin top horizontal rule
\huge Problem Set 3 \\ % The assignment title
\horrule{2pt} \\[0.5cm] % Thick bottom horizontal rule
}

\author{Daniel Halmrast} % Your name

\date{\normalsize\today} % Today's date or a custom date

\begin{document}

\maketitle % Print the title

% Problems
\section*{Problem 1} %Hatcher 0.15
Enumerate all subcomplexes of $S^{\infty}$ with the cell structure on
$S^{\infty}$ that has $S^n$ as the $n$-skeleton.
\\
\\
\begin{proof}
    We notice at first that each $n$-skeleton is a subcomplex, and so $S^n$ is a
    subcomplex of $S^{\infty}$ for each $n$.

    There is another subcomplex in each dimension. Namely, by omitting one of
    the $n$-cells attaching to the $n-1$-skeleton, we obtain another subcomplex
    in the $n$th dimension that is the $n-1$ skeleton along with a single
    $n$-cell attached in the usual way. In fact, depending on which $n$-cell we
    omit, we can obtain two different subcomplexes.

    So far, we have three subcomplexes in each dimension. I assert that this is
    all the subcomplexes. Suppose there existed a subcomplex in $n$ dimensions
    that did not contain the entire $n-1$-skeleton. In particular, this means
    that the attaching map of the $n$-cell, which is bijective from $\partial
    D^n$ onto the entire
    $n-1$-skeleton, is not well-defined, and so no such subcomplex can be
    constructed.
\end{proof}


\section*{Problem 2} %Hatcher 0.16
Show $S^{\infty}$ is contractible.
\\
\\
\begin{proof}
    We will show that the $n$-skeleton of $X=S^{\infty}$ is contractible in
    $X^{n+1}$. To see this, consider the subcomplex $X^n$ along with a single
    disk $D^{n+1}$ attached in the usual way. In particular, $X^n$ is identified
    with $\partial D^{n+1}$, and since $D^{n+1}$ is contractible, it follows
    that $\partial D^{n+1}$ contracts to a point in
    $D^{n+1}$.

    Thus, each $X^n$ is contractible in $X$, and since
    $X=\cup_{n=0}^{\infty}X^n$, it follows that $X$ is contractible as well.
\end{proof}

\section*{Problem 3}
Show that $S^1\star S^1=S^3$. In general, show $S^m\star S^n= S^{m+n+1}$.
\\
\\
\begin{proof}
    We will prove the more general result that
    \[
        \bigstar_{i=1}^n S^0 = S^{n-1}
    \]
    along with the associativity of $\star$.

    To see the first result, we will use the interpretation of the star product
    as the set of all convex formal linear combinations of the two spaces. In
    particular, interpreting $S^0_i$ to be the two points $1,-1$ on the $i$th
    coodinate axis in $\R^n$, we see that join across all $n$ copies of $S^0$ is
    really all points $x=(x^1,\ldots,x^n)$ satisfying
    \[
        \begin{aligned}
            x^i\leq 1 \textrm{ and } x^i\geq -1\\
            \sum_i x^i = 1
        \end{aligned}
    \]
    which is just the convex hull of the $2n$ unit vectors along the coordinate
    axes of $/R^n$. In other words, it is the ball of radius $1$ in the
    $1$-norm. However, this is obviously homeomorphic to $S^n-1$ as desired.

    Now, we will show that $\star$ is associative. However, this falls almost
    immediately from the definition of $\star$ in terms of formal linear
    combinations.

    Thus, $S^n = \bigstar_{i=1}^{n+1}S^0$, and so
    \[
        \begin{aligned}
            S^m\star S^n &=
            (\bigstar_{i=1}^{m+1}S^0)\star(\bigstar_{i=1}^{n+1}S^0)\\
            &=\bigstar_{i=1}^{m+n+2}S^0\\
            &= S^{m+n+1}
        \end{aligned}
    \]
    as desired.
\end{proof}

\section*{Problem 4}
Show that the space obtained by attaching $n$ 2-cells along any collection of
$n$ circles in $S^2$ is homotopy equivalent to the wedge sum of $n+1$ 2-spheres.
\\
\\



\section*{Problem Extra}
In the proof of the Brouwer fixed point theorem, construct explicitly the
retraction from $D$ to $\partial D$ by assuming $f(x)\neq x$.

\end{document}
