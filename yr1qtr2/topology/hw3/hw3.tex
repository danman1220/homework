%%%%%%%%%%%%%%%%%%%%%%%%%%%%%%%%%%%%%%%%%
% Short Sectioned Assignment
% LaTeX Template
% Version 1.0 (5/5/12)
%
% This template has been downloaded from:
% http://www.LaTeXTemplates.com
%
% Original author:
% Frits Wenneker (http://www.howtotex.com)
%
% License:
% CC BY-NC-SA 3.0 (http://creativecommons.org/licenses/by-nc-sa/3.0/)
%
%%%%%%%%%%%%%%%%%%%%%%%%%%%%%%%%%%%%%%%%%

%----------------------------------------------------------------------------------------
%	PACKAGES AND OTHER DOCUMENT CONFIGURATIONS
%----------------------------------------------------------------------------------------

\documentclass[fontsize=11pt]{scrartcl} % 11pt font size

\usepackage[T1]{fontenc} % Use 8-bit encoding that has 256 glyphs
\usepackage[english]{babel} % English language/hyphenation
\usepackage{amsmath,amsfonts,amsthm} % Math packages
\usepackage{mathrsfs}

\usepackage[margin=1in]{geometry}

\usepackage{sectsty} % Allows customizing section commands
\allsectionsfont{\centering \normalfont\scshape} % Make all sections centered, the default font and small caps

\usepackage{fancyhdr} % Custom headers and footers
\pagestyle{fancyplain} % Makes all pages in the document conform to the custom headers and footers
\fancyhead{} % No page header - if you want one, create it in the same way as the footers below
\fancyfoot[L]{} % Empty left footer
\fancyfoot[C]{} % Empty center footer
\fancyfoot[R]{\thepage} % Page numbering for right footer
\renewcommand{\headrulewidth}{0pt} % Remove header underlines
\renewcommand{\footrulewidth}{0pt} % Remove footer underlines
\setlength{\headheight}{13.6pt} % Customize the height of the header

\numberwithin{equation}{section} % Number equations within sections (i.e. 1.1, 1.2, 2.1, 2.2 instead of 1, 2, 3, 4)
\numberwithin{figure}{section} % Number figures within sections (i.e. 1.1, 1.2, 2.1, 2.2 instead of 1, 2, 3, 4)
\numberwithin{table}{section} % Number tables within sections (i.e. 1.1, 1.2, 2.1, 2.2 instead of 1, 2, 3, 4)

\newcommand{\R}{\mathbb{R}}
\newcommand{\Q}{\mathbb{Q}}
\newcommand{\N}{\mathbb{N}}
\newcommand{\C}{\mathbb{C}}

\newtheorem{lemma}{Lemma}
%----------------------------------------------------------------------------------------
%	TITLE SECTION
%----------------------------------------------------------------------------------------

\newcommand{\horrule}[1]{\rule{\linewidth}{#1}} % Create horizontal rule command with 1 argument of height

\title{	
\normalfont \normalsize 
\textsc{Homotopy Theory} \\ [25pt] % Your university, school and/or department name(s)
\horrule{0.5pt} \\[0.4cm] % Thin top horizontal rule
\huge Problem Set 3 \\ % The assignment title
\horrule{2pt} \\[0.5cm] % Thick bottom horizontal rule
}

\author{Daniel Halmrast} % Your name

\date{\normalsize\today} % Today's date or a custom date

\begin{document}

\maketitle % Print the title

% Problems
\section*{Problem 1} %Hatcher 0.15
Enumerate all subcomplexes of $S^{\infty}$ with the cell structure on
$S^{\infty}$ that has $S^n$ as the $n$-skeleton.
\\
\\
\begin{proof}
    We notice at first that each $n$-skeleton is a subcomplex, and so $S^n$ is a
    subcomplex of $S^{\infty}$ for each $n$.

    There is another subcomplex in each dimension. Namely, by omitting one of
    the $n$-cells attaching to the $n-1$-skeleton, we obtain another subcomplex
    in the $n$th dimension that is the $n-1$ skeleton along with a single
    $n$-cell attached in the usual way. In fact, depending on which $n$-cell we
    omit, we can obtain two different subcomplexes.

    So far, we have three subcomplexes in each dimension. I assert that this is
    all the subcomplexes. Suppose there existed a subcomplex in $n$ dimensions
    that did not contain the entire $n-1$-skeleton. In particular, this means
    that the attaching map of the $n$-cell, which is bijective from $\partial
    D^n$ onto the entire
    $n-1$-skeleton, is not well-defined, and so no such subcomplex can be
    constructed.
\end{proof}


\section*{Problem 2} %Hatcher 0.16
Show $S^{\infty}$ is contractible.
\\
\\
\begin{proof}
    We will show that the $n$-skeleton of $X=S^{\infty}$ is contractible in
    $X^{n+1}$. To see this, consider the subcomplex $X^n$ along with a single
    disk $D^{n+1}$ attached in the usual way. In particular, $X^n$ is identified
    with $\partial D^{n+1}$, and since $D^{n+1}$ is contractible, it follows
    that $\partial D^{n+1}$ contracts to a point in
    $D^{n+1}$.

    Thus, each $X^n$ is contractible in $X$, and by running all of them
    sequentially (say, running the $n$th homotopy in time $[2^{-n},2^{-(n+1)}]$)
    we obtain a contraction of $S^{\infty}$.
\end{proof}

\section*{Problem 3}
Show that $S^1\star S^1=S^3$. In general, show $S^m\star S^n= S^{m+n+1}$.
\\
\\
\begin{proof}
    We will prove the more general result that
    \[
        \star_{i=1}^n S^0 = S^{n-1}
    \]
    along with the associativity of $\star$.

    To see the first result, we will use the interpretation of the star product
    as the set of all convex formal linear combinations of the two spaces. In
    particular, interpreting $S^0_i$ to be the two points $1,-1$ on the $i$th
    coodinate axis in $\R^n$, we see that join across all $n$ copies of $S^0$ is
    really all points $x=(x^1,\ldots,x^n)$ satisfying
    \[
        \begin{aligned}
            x^i\leq 1 \textrm{ and } x^i\geq -1\\
            \sum_i x^i = 1
        \end{aligned}
    \]
    which is just the convex hull of the $2n$ unit vectors along the coordinate
    axes of $/R^n$. In other words, it is the ball of radius $1$ in the
    $1$-norm. However, this is obviously homeomorphic to $S^n-1$ as desired.

    Now, we will show that $\star$ is associative. However, this falls almost
    immediately from the definition of $\star$ in terms of formal linear
    combinations.

    Thus, $S^n = \star_{i=1}^{n+1}S^0$, and so
    \[
        \begin{aligned}
            S^m\star S^n &=
            (\star_{i=1}^{m+1}S^0)\star(\star_{i=1}^{n+1}S^0)\\
            &=\star_{i=1}^{m+n+2}S^0\\
            &= S^{m+n+1}
        \end{aligned}
    \]
    as desired.
\end{proof}

\section*{Problem 4}
Show that the space obtained by attaching $n$ 2-cells along any collection of
$n$ circles in $S^2$ is homotopy equivalent to the wedge sum of $n+1$ 2-spheres.
\\
\\
\begin{proof}
    Consider the (potentially disconnected) graph formed by the $n$ circles on the surface $S^2$. This
    graph in particular can be homotoped to a tree relative to the points of
    intersection (the nodes of the graph). This is done by taking any loop (two
    distinct intervals connected on both endpoints) and sliding the two
    intervals to meet each other. After obtaining a tree, the tree can be
    homotoped to a point,

    Now, the key thing about these homotopies is that they define a homotopy on
    the space obtained by attaching $2$-cells along the $n$ circles on $S^2$ as
    well. This is clear, since the first homotopy does not move points of
    intersection (so as not to have discontinuities). Thus, this space obtained
    by attaching the $n$ $2$-cells to $S^2$ is homotopy equivalent to the space
    attaching $n$ $2$-cells to a point, which is just the wedge sum of $n+1$
    copies of $S^2$ as desired.
\end{proof}

\newpage

\section*{Problem 5}
Show that the subspace $X$ of $\R^3$ formed by a self-intersecting Klein bottle
is homotopy equivalent to $S^1\vee S^1\vee S^2$.
\\
\\

\newpage

\section*{Problem 6}
Show that a CW complex is contractible if its the union of contractible spaces
whose intersection is contractible.
\\
\\
\begin{proof}
    Let $X$, $Y$ be CW compexes with intersection $A$, and suppose these three
    are contractible. We wish to show that $X\cup Y$ is contractible. Now, we
    know that $X/{A}$ and $Y/{A}$ are contractible, since $A$ is contractible
    and thus $X\simeq X/{A}$ and $Y\simeq Y/{A}$. So, we also have that
    \[
        (X\cup Y)/{A} \simeq X\cup Y
    \]
    and so showing $(X\cup Y)/{A}$ is contractible is enough to prove the
    conjecture. 

    However, $(X\cup Y)/{A}$ is just $X/{A}\vee Y/{A}$ where they are attached
    at the quotient of $A$. Since both $X/{A}$ and $Y/{A}$ are contractible, it
    follows immediately that their wedge is contractible as well.
\end{proof}

\section*{Problem 7}
Let $X$ and $Y$ be CW complexes with $0$-cells $x_0$ and $y_0$. Show that the
quotient $X\star Y/(X\star\{y_0\}\cup \{x_0\star Y)$ and $S(X\wedge
Y)/S(\{x_0\}\wedge \{y_0\})$ are homeomorphic.
\\
\\
\begin{proof}
    
\end{proof}

\end{document}
