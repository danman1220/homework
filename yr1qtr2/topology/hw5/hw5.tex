%%%%%%%%%%%%%%%%%%%%%%%%%%%%%%%%%%%%%%%%%
% Short Sectioned Assignment
% LaTeX Template
% Version 1.0 (5/5/12)
%
% This template has been downloaded from:
% http://www.LaTeXTemplates.com
%
% Original author:
% Frits Wenneker (http://www.howtotex.com)
%
% License:
% CC BY-NC-SA 3.0 (http://creativecommons.org/licenses/by-nc-sa/3.0/)
%
%%%%%%%%%%%%%%%%%%%%%%%%%%%%%%%%%%%%%%%%%

%----------------------------------------------------------------------------------------
%	PACKAGES AND OTHER DOCUMENT CONFIGURATIONS
%----------------------------------------------------------------------------------------

\documentclass[fontsize=11pt]{scrartcl} % 11pt font size

\usepackage[T1]{fontenc} % Use 8-bit encoding that has 256 glyphs
\usepackage[english]{babel} % English language/hyphenation
\usepackage{amsmath,amsfonts,amsthm} % Math packages
\usepackage{mathrsfs}
\usepackage{tikz-cd}
\usepackage{bbm}

\usepackage[margin=1in]{geometry}

\usepackage{sectsty} % Allows customizing section commands
\allsectionsfont{\centering \normalfont\scshape} % Make all sections centered, the default font and small caps

\usepackage{fancyhdr} % Custom headers and footers
\pagestyle{fancyplain} % Makes all pages in the document conform to the custom headers and footers
\fancyhead{} % No page header - if you want one, create it in the same way as the footers below
\fancyfoot[L]{} % Empty left footer
\fancyfoot[C]{} % Empty center footer
\fancyfoot[R]{\thepage} % Page numbering for right footer
\renewcommand{\headrulewidth}{0pt} % Remove header underlines
\renewcommand{\footrulewidth}{0pt} % Remove footer underlines
\setlength{\headheight}{13.6pt} % Customize the height of the header

\numberwithin{equation}{section} % Number equations within sections (i.e. 1.1, 1.2, 2.1, 2.2 instead of 1, 2, 3, 4)
\numberwithin{figure}{section} % Number figures within sections (i.e. 1.1, 1.2, 2.1, 2.2 instead of 1, 2, 3, 4)
\numberwithin{table}{section} % Number tables within sections (i.e. 1.1, 1.2, 2.1, 2.2 instead of 1, 2, 3, 4)

\newcommand{\R}{\mathbb{R}}
\newcommand{\Q}{\mathbb{Q}}
\newcommand{\N}{\mathbb{N}}
\newcommand{\C}{\mathbb{C}}
\newcommand{\Z}{\mathbb{Z}}

\newtheorem{lemma}{Lemma}
%----------------------------------------------------------------------------------------
%	TITLE SECTION
%----------------------------------------------------------------------------------------

\newcommand{\horrule}[1]{\rule{\linewidth}{#1}} % Create horizontal rule command with 1 argument of height

\title{	
\normalfont \normalsize 
\textsc{Homotopy Theory} \\ [25pt] % Your university, school and/or department name(s)
\horrule{0.5pt} \\[0.4cm] % Thin top horizontal rule
\huge Problem Set 5 \\ % The assignment title
\horrule{2pt} \\[0.5cm] % Thick bottom horizontal rule
}

\author{Daniel Halmrast} % Your name

\date{\normalsize\today} % Today's date or a custom date

\begin{document}

\maketitle % Print the title

% Problems
\section*{Problem 1}
Define $f:S^1\times I\to S^1\times I$ by $f(\theta,s) = (\theta+2\pi s,s)$ so
that $f$ restricts to the identity on the two boundary circles of $S^1\times I$. 
Show that $f$ is homotopic to the identity by a homotopy $f_t$ that is
stationary on one of the boundary circles, but not by any homotopy $f_t$ that
is stationary on both boundary circles.
\\
\\
\begin{proof}
    We begin by explicitly constructing the homotopy that is stationary on
    $S^1\times\{0\}$. This is done via the homotopy
    \[
        f_t(\theta,s) = (\theta + t2\pi s,s)
    \]
    Clearly, $f_0 = \mathbbm{1}$ and $f_1 = f$. Furthermore, $f_t(\theta,0) =
    (\theta,0)$ and so $f_t$ is stationary on $S^1\times \{0\}$

\end{proof}


\section*{Problem 2}
Does the Borsuk-Ulam theorem hold for the torus? That is, for every map
$f:S^1\times S^1 \to \R^2$ does there exist a point $(x,y)$ for which
$f(x,y) = f(-x,-y)$?
\\
\\
\begin{proof}
    I assert that the Borsuk-Ulam theorem does not hold for the torus. To see
    this, we construct an explicit function from $T^2$ to $\R^2$ which does not
    have any antipodal points with the same value.

    Consider the function $f:T^2\to \R^2$ given as follows. First, let $T^2$ be
    embedded in $\R^3$. Then, consider the vector field
    $\frac{\partial}{\partial\phi}$ where $\phi$ runs parallel to the $x-y$
    plane. Since $T^2$ is embedded in $\R^3$, these vectors can be thought of as
    living in the tangent bundle to $\R^3$. Thus, for each vector, it makes
    sense to take its projection onto the $x-y$ plane. Now, since the original
    vector field $\frac{\partial}{\partial \phi}$ is smooth, and projection is a
    continuous operation, this defines a continuous map from $T^2$ to $\R^2$.

    In coordinates, this map is given as
    \begin{equation}
        f(\theta,\phi) = (\cos(\phi),\sin(\phi))
    \end{equation}
    and clearly, $f(\theta,\phi)\neq f(-\theta,-\phi)$ as desired.
\end{proof}

\section*{Problem 3}
From the isomorphism $\pi_1(X\times Y, (x_0,y_0))\cong
\pi_1(X,x_0)\times\pi_1\left( Y,y_0 \right)$ it follows that loops in
$X\times\{y_0\}$ and $\left\{ x_0 \right\}\times Y$ represent commuting elements
of $\pi_1\left( X\times Y,\left( x_0,y_0 \right) \right)$. Construct an explicit
homotopy demonstrating this.
\\
\\
\begin{proof}
    Let $f$ be a loop in $X\times\left\{ y_0 \right\}$, and let $g$ be a loop in
    $\left\{ x_0 \right\}\times Y$. We wish to show that $f\cdot g \simeq g\cdot
    f$. We can construct this homotopy explicitly by ``sliding'' $f$ along $g$.
    Concretely the homotopy is as follows.

    First, define $g_t = g\Big|_{[0,t]}$ and $g_{t'} = g\Big|_{[t,1]}$ so that
    $g_{t'}\cdot g_t = g$. That is, $g_t$ is the segment of $g$ from $g(0)$ to
    $g(t)$. Furthermore, let $f_y$ be the loop $f$ in the subspace
    $X\times\left\{ y \right\}$.

    Now, define a homotopy $h$ as
    \begin{equation}
        h_t = g_{t'}\cdot f_{\pi_y(g(t))}\cdot g_t
    \end{equation}
    Now, this is well-defined, since $g_t$ has an endpoint at
    $(x_0,\pi_y(g(t)))$, and $f$ starts and ends at $(x_0,\pi_y(g(t)))$, and
    $g_{t'}$ begins at $(x_0,\pi_y(g(t))$ and ends at $(x_0,y_0)$. We just have
        to make sure this is continuous. Since $h$ is a map into $X\times Y$, we
        just have to check that the corresponding map $H:S^1\times I \to X\times
        Y$ is continuous onto both $X$ and $Y$. That is, $H$ is continuous if
        and only if $\pi_X H$ and $\pi_Y H$ are continuous.

    However, it should be clear that $\pi_X H = f$, since $\pi_X(g(t)) = x_0$
    for all $t$. Thus, the path $g_{t'}\cdot f \cdot g_t$ projects down to just
    $x_0\cdot f\cdot x_0$, which is clearly continuous. 
    
    Similarly, $\pi_Y H = g_{t'}\cdot y_0\cdot g_t$, which is (after
    reparameterization) equal to $g_{t'}\cdot g_t = g$, which is clearly
    continuous for all $t$.

    Thus, $H$ is continuous in its projections, and by the universal property of
    products, is continuous in general.

    Now, $h_0$ is just $g_{0'}\cdot f\cdot g_0 = g\cdot f$, and $h_1 =
    g_{1'}\cdot f\cdot g_1 = f\cdot g$, and so $f\cdot g \simeq g\cdot f$ as
    desired.
\end{proof}

\section*{Problem 4}
Show that every homomorphism $\pi_1(S^1)\to \pi_1(S^1)$ can be realized as the
induced homomorphism $\varphi_*$ of a map $\varphi:S^1\to S^1$.
\\
\\
\begin{proof}
    Throughout this problem, I will identify $\pi_1(S^1)$ with $\Z$ by
    identifying $1$ with the loop that goes once around the circle
    counterclockwise.

    Since $\pi_1(S^1) \cong \Z$, each homomorphism is characterized by where it
    sends the generator $1$.

    So, suppose $\phi:\pi_1(S^1)\to\pi_1(S^1)$ is such that $\phi(1) = n$ for
    some integer $n$. Define $\varphi(z) = z^n$. I claim that $\varphi_* =
    \phi$.

    To see this, we just need to see that $\varphi_*(1) = n$. Now, the loop $1$
    is just the map $f(t) = \exp(2\pi i t)$, and the induced loop $\varphi_*(1)$ is the
    composition $\phi(f(t)) = \exp(2\pi i nt)$ which is easily seen to be
    homotopic to the loop $n$, as desired.
\end{proof}

\section*{Problem 5}
Show that there does not exist a retraction from $X$ to $A$ in the following
cases:
\subsection*{Part (a)}
$X=\R^3$ and $A$ is any subspace homeomorphic to $S^1$.
\begin{proof}
    Suppose such a retraction $r:X\to A$ existed. In particular, we would have
    that
    \[
\begin{tikzcd}
    A\arrow{r}{i}\arrow[bend right]{rr}{\mathbbm{1}} &X\arrow{r}{r} &A
\end{tikzcd}
    \]
    commutes. Now, applying the $\pi_1$ functor to this diagram yields
    \[
\begin{tikzcd}
    \Z\arrow[leftrightarrow]{d} &0\arrow[leftrightarrow]{d}
    &\Z\arrow[leftrightarrow]{d}\\
    \pi_1(A)\arrow{r}{i_*}\arrow[bend right]{rr}{\mathbbm{1}} &X\arrow{r}{r} &A
\end{tikzcd}
    \]
    Now, this diagram implies that $\mathbbm{1}$ on $\Z$ factors through zero,
    which cannot happen. Thus, no such $r$ exists.
\end{proof}
\subsection*{Part (b)}
$X=S^1\times D^2$ with $A$ the boundary torus $S^1\times S^1$.
\\
\\
\begin{proof}
    Following the same diagram as the one used in the previous problem, assuming
    such an $r:X\to A$ exists yields the diagram
    \[
\begin{tikzcd}
    \Z\times\Z\arrow{r}{i_*}\arrow[bend right]{rr}{\mathbbm{1}}
    &\Z\arrow{r}{r_*} &\Z\times \Z
\end{tikzcd}
    \]
    which commutes. However, for this to commute, it must be that $i_*$ is
    injective. This cannot happen, however, since $\Z\times \Z$ has two
    generators, but $\Z$ only has one. Since an injective map has to map
    generators to generators, $i_*$ cannot be injective, and the diagram cannot
    commute, a contradiction. Thus, no such $r$ exists.
\end{proof}
\subsection*{Part (c)}
$X=S^1\times D^2$ with $A$ the circle shown in the figure.
\\
\\
\begin{proof}
    Again, we assume that such a retraction exists, and form the diagram
    \[
\begin{tikzcd}
    \pi_1(A)\arrow{r}{i}\arrow[bend right]{rr}{\mathbbm{1}} $X\arrow{r}{r} &A
\end{tikzcd}
    \]
    which would force $i_*$ to be injective. However, since $A$ is contractible
    in $X$, for any loop $[f]\in\pi_1(A)$, $i_*[f]$ is homotopic to the constant
    loop. Thus, $i_*$ is actually the zero map, and in particular is not
    injective. Thus, no such retraction $r$ can exist.
\end{proof}
\subsection*{Part (d)}
$X=D^2\vee D^2$ with $A = \partial X$.
\\
\\
\begin{proof}
    Mirroring the approach in part (a), we show that $\pi_1(X)=0$, with
    $\pi_1(A)\neq 0$. This forces $i_*$ to fail to be injective, and thus no
    retraction can exist from $X$ to $A$.

    It should be clear that $\pi_1(D^2\vee D^2)$ is trivial, since the space is
    contractible via the contraction that sends each copy of $D^2$ to the single
    point in their intersection.

    Thus, since $\pi_1(X) = 0$ (and $\pi_1(A) = \Z\star\Z\neq 0$) the induced
    map $i_*$ is not injective, and thus cannot have a retract.
\end{proof}
\subsection*{Part (e)}
$X$ is a disk with two points on the boundary identified, and $A$ is the
boundary $S^1\vee S^1$.
\\
\\
\begin{proof}
    To begin with, we note that $\pi_1(X) = \Z$ (since $X\simeq S^1$) and that
    $\pi_1(A)=\Z\star\Z$ (calculated in Hatcher).

    Now, we will show that the induced map $i_*$ is not injective (for $i$ the
    canonical inclusion map of $A$ into $X$). Clearly, $i_*$ cannot be
    injective, since there is no injective map from $\Z\star\Z$ to $\Z$.
    
    To see this, note that any group homomorphism $\phi:\Z\star\Z\to\Z$ is
    completely determined by where it sends its generators. Let $a$ and $b$ be
    the generators for $\Z\star\Z$. Then, since $\Z$ only has one generator, it
    must be that $\phi(a) = n\phi(b)$ for some $n\in\Z$. However, this implies
    that $\phi(a) = \phi(nb)$ and since $\Z\star\Z$ has no relations between $a$
    and $b$, we know that $nb\neq a$. Thus, $\phi$ is not injective.

    Since $i_*$ is not injective, it cannot have a retract, and so no
    retraction $r:X\to A$ exists.
\end{proof}
\subsection*{Part (f)}
$X$ is the Mobius band, and $A$ is the boundary circle.
\\
\\
\begin{proof}
    We observe first that the boundary of the Mobius band loops twice around it
    before returning to its basepoint. Using this fact, we can calculate what
    the induced map $i_*$ is.

    Let $[1]_A$ be the generator of $\pi_1(A)$ going counterclockwise, and let
    $[1]_X$ be similarly defined for $X$. We only need to calculate what
    $i_*[1]_A$ is. However, it is clear from the observation above that
    $i_*[1]_A = [2]_X$.

    Now, suppose a retraction $r:X\to A$ existed. since $r\circ i  =
    \mathbbm{1}$, we know that $r_*\circ i_* = \mathbbm{1}$ as well. Thus,
    $r_*[2]_X = [1]_A$. But this is not a well-defined group homomorphism! In
    particular, $r_*[1]_X$ is not well-defined. Suppose $r_*[1]_X = [n]$. Then,
    $r_*[2]_X = 2r_*[1]_X = [2n] \neq [1]$ which contradicts $r_*[2]_X = [1]_A$.
    Thus, no such retraction can exist.
\end{proof}

\section*{Problem 6}
Use Lemma 1.15 to show that if a space $X$ is obtained from a path-connected
subspace $A$ by attaching a cell $e^n$ with $n\geq 2$, then the inclusion $A\to
X$ induces a surjection on $\pi_1$.
\\
\\
\begin{proof}
    Recall that Lemma 1.15 states that if a space $X$ is the union of a
    collection of path-connected open sets $A_{\alpha}$ each containing the
    basepoint $x_0$, and each pairwise intersection is path-connected, then
    every loop in $X$ at $x_0$ is homotopic to a product of loops each of which
    is contained in a single $A_{\alpha}$.

    Now, we can decompose $X$ into $A\cup D^n$ where $D^n$ is the $n$-cell $e^n$
    along with its boundary. Furthermore, since $A$ is path-connected, we can
    take the basepoint of $X$ without loss of generality to be on the
    intersection $A\cap D^n$. Thus, this decomposition satisfies the properties
    of Lemma 1.15. So, we just need to show that any loop in $X$ is the
    inclusion of a loop in $A$.

    Let $f$ be any loop in $X$. By Lemma 1.15, we know that $f$ is homotopic to
    a product of loops in $A$ and $D^n$. However, each loop in $D^n$ is
    nullhomotopic, and so $f$ is homotopic to a product of loops in $A$, which
    itself is a loop in $A$. Call this loop $\tilde{f}$. It should be clear,
    then, that $i_*[\tilde{f}] = [f]$, and thus $i_*$ is surjective.
\end{proof}
\subsection*{Part (a)}
Use this to show that the wedge sum $S^1\vee S^2$ has fundamental group $\Z$.
\\
\\
\begin{proof}
    We let $A=S^1$, and $X=S^1\vee S^2$. Then the inclusion map $i_*$ is a
    surjection. Furthermore, it should be clear that $i_*$ is injective as well,
    since the inclusion of a nontrivial loop in $A$ to $X$ will not be
    nullhomotopic.
    
    Thus, $i_*$ is actually an isomorphism, and $\pi_1(X) = \Z$ as desired.
\end{proof}
\subsection*{Part (b)}
Use this to show that for a path-connected CW complex $X$ the inclusion map
$X^1\to X$ of its $1$-skeleton induces a surjection $\pi_1(X^1)\to\pi_1(X)$.
\\
\\
\begin{proof}
    We first show this result holds in the finite case. To do so, we induct on
    the number of cells attached to the $1$-skeleton. In particular, we show
    that adding an $n$-cell for $n\geq 2$ keeps the inclusion map a surjection.

    This should be clear. For the base case, we note that $X^1$ along with the
    next cell to be attached $e_1$ satisfy the hypothesis for the main proof of
    this problem, and so the map $i_*:\pi_1(X^1)\to\pi_1(X_1)$ is surjective
    (here $X_1$ is just $X^1\cup e_1$, and $X_k$ will denote the space obtained
    by attaching the first $k$ cells to $X^1$).

    Now, suppose the proposition holds for $X_k$. That is, the inclusion
    $i_*:X^1\to X_k$ is surjective. Then, $X_k$ along with the
    next cell $e_{k+1}$ satisfy the hypotheses for the main proof of this
    problem, and the map $i_{k*}:\pi_1(X_k}\to \pi_1(X_{k+1})$ is surjective.
    Then, the composition $i_{k*}\circ i_*$ which is the inclusion of $X^1$ into
    $X_{k+1}$ is surjective as well.

    Thus, if a loop is in a finite number of cells, it is homotopic to a loop on
    the $1$-skeleton. However, Proposition $A.1$ guarantees that any loop in $X$
    (which is compact since it is the image of the interval) only passes through
    finitely many cells. So, every loop in $X$ is homotopic to a loop in $X^1$,
    and thus the inclusion map $i_*:\pi_1(X^1)\to \pi_1(X)$ is surjective as
    desired.
\end{proof}

\section*{Problem extra}
Find the standard form of $\Z_4\times\Z_6\times\Z_6$, and prove or disprove:
\[
    \Z_4\times\Z_6\times\Z_6\cong \Z_4\times\Z_4\times\Z_3\times\Z_3
\]

\end{document}
