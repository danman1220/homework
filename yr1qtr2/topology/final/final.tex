%%%%%%%%%%%%%%%%%%%%%%%%%%%%%%%%%%%%%%%%%
% Short Sectioned Assignment
% LaTeX Template
% Version 1.0 (5/5/12)
%
% This template has been downloaded from:
% http://www.LaTeXTemplates.com
%
% Original author:
% Frits Wenneker (http://www.howtotex.com)
%
% License:
% CC BY-NC-SA 3.0 (http://creativecommons.org/licenses/by-nc-sa/3.0/)
%
%%%%%%%%%%%%%%%%%%%%%%%%%%%%%%%%%%%%%%%%%

%----------------------------------------------------------------------------------------
%	PACKAGES AND OTHER DOCUMENT CONFIGURATIONS
%----------------------------------------------------------------------------------------

\documentclass[fontsize=11pt]{scrartcl} % 11pt font size

\usepackage[T1]{fontenc} % Use 8-bit encoding that has 256 glyphs
\usepackage[english]{babel} % English language/hyphenation
\usepackage{amsmath,amsfonts,amsthm} % Math packages
\usepackage{tikz-cd}
\usepackage{mathrsfs}
\usepackage{bbm}

\usepackage[margin=1in]{geometry}

\usepackage{sectsty} % Allows customizing section commands
\allsectionsfont{\centering \normalfont\scshape} % Make all sections centered, the default font and small caps

\usepackage{fancyhdr} % Custom headers and footers
\pagestyle{fancyplain} % Makes all pages in the document conform to the custom headers and footers
\fancyhead{} % No page header - if you want one, create it in the same way as the footers below
\fancyfoot[L]{} % Empty left footer
\fancyfoot[C]{} % Empty center footer
\fancyfoot[R]{\thepage} % Page numbering for right footer
\renewcommand{\headrulewidth}{0pt} % Remove header underlines
\renewcommand{\footrulewidth}{0pt} % Remove footer underlines
\setlength{\headheight}{13.6pt} % Customize the height of the header

\numberwithin{equation}{section} % Number equations within sections (i.e. 1.1, 1.2, 2.1, 2.2 instead of 1, 2, 3, 4)
\numberwithin{figure}{section} % Number figures within sections (i.e. 1.1, 1.2, 2.1, 2.2 instead of 1, 2, 3, 4)
\numberwithin{table}{section} % Number tables within sections (i.e. 1.1, 1.2, 2.1, 2.2 instead of 1, 2, 3, 4)

\newcommand{\R}{\mathbb{R}}
\newcommand{\Q}{\mathbb{Q}}
\newcommand{\N}{\mathbb{N}}
\newcommand{\C}{\mathbb{C}}
\newcommand{\Z}{\mathbb{Z}}

\newtheorem{lemma}{Lemma}
%----------------------------------------------------------------------------------------
%	TITLE SECTION
%----------------------------------------------------------------------------------------

\newcommand{\horrule}[1]{\rule{\linewidth}{#1}} % Create horizontal rule command with 1 argument of height

\title{	
\normalfont \normalsize 
\textsc{Homotopy Theory} \\ [25pt] % Your university, school and/or department name(s)
\horrule{0.5pt} \\[0.4cm] % Thin top horizontal rule
\huge Final Exam \\ % The assignment title
\horrule{2pt} \\[0.5cm] % Thick bottom horizontal rule
}

\author{Daniel Halmrast} % Your name

\date{\normalsize\today} % Today's date or a custom date

\begin{document}

\maketitle % Print the title

% Problems
\section*{Problem 1}
Carefully prove a CW complex is contractible if it is the union of two
contractible subcomplexes whose intersection is contractible.
\\
\\
\begin{proof}
    For this proof, we will begin by proving a couple of useful lemmas.
    \begin{lemma}
        The wedge sum is the coproduct in the category $Top_*$ of pointed
        topological spaces.
    \end{lemma}
    \begin{proof}
        To show this is the coproduct, we need to show that it satisfies the
        universal property for coproducts. That is for $(X,x_0),(Y,y_0)$ pointed
        topological spaces, and $(U,u_0)$ any other pointed topological space
        with arrows $f:(X,x_0)\to (U,u_0)$ and $g:(Y,y_0)\to (U,u_0)$,
        \[
            \begin{tikzcd}[column sep=tiny, row sep=large]
                (X,x_0)\ar[dr,"i_{x*}"]\ar[ddr,"f", bend right]
                &&(Y,y_0)\ar[dl,"i_{y*}"'] \ar[ddl,"g"', bend left]\\
                &(X,x_0)\vee (Y,y_0)\ar[d, "\exists !{(f,g)}",dashed]\\
                &(U,u_0)
\end{tikzcd}
        \]
    where $(f,g)$ is the unique arrow that makes the diagram commute. Now,
    recall that the wedge sum is defined as the quotient
    \[
        (X,x_0)\vee (Y,y_0) = X\coprod Y/{x_0\sim y_0}
    \]
    so we can bootstrap by using the fact that the disjoint union is the
    coproduct in $Top$. That is, we define $i_{x*}$ and $i_{y*}$ to be the
    unique maps given by the universal property
    \[
        \begin{tikzcd}[column sep=tiny, row sep=large]
    X\ar[dr,"i_x"]\ar[ddr, "i_{x*}",dashed, bend right] &&Y\ar[dl, "i_y"']\ar[ddl,
    "i_{y*}"',dashed, bend left]\\
    &X\coprod Y\ar[d,"q"]\\
    &X\vee Y
\end{tikzcd}
    \]
    Note that these maps preserve basepoints, since 
    \[
        i_{x*}(x_0) = q\circ i_x(x_0) = q(x_0)
    \]
    and similarly for $y_0$. 
    
    Now, suppose we have maps $f:(X,x_0)\to (U,u_0)$
    and $g:(Y,y_0)\to (U,u_0)$. Define $(f,g)$ to be the unique map
    \[
        (f,g):X\coprod Y\to U
    \]
    that makes the diagram
    \[
        \begin{tikzcd}[column sep=tiny, row sep=large]
    X\ar[dr,"i_x"]\ar[ddr, "f", bend right] &&Y\ar[dl, "i_y"']\ar[ddl,
    "g"', bend left]\\
    &X\coprod Y\ar[d,"{(f,g)}",dashed]\\
    &U
\end{tikzcd}
    \]
    commute. Now, since $f(x_0) = g(y_0) = u_0$, it follows that $(f,g)$ is
    constant on the fibers of $q:X\coprod Y\to X\vee Y$ and thus by the
    universal property of quotient maps factors through $X\vee Y$. That is,
    \[
        \begin{tikzcd}[column sep=small]
            &X\coprod Y\arrow[dl,"{(f,g)}"']\ar[dr,"q"]\\
    U &&X\vee Y\arrow[dashed]{ll}{(f,g)_*}
\end{tikzcd}
    \]
    commutes for a unique $(f,g)_*$. Thus, we have for $f,g$ arrows from
    $(X,x_0)$ and $(Y,y_0)$ to $(U,u_0)$ a unique arrow $(f,g)_*$ from $X\vee Y$
    which makes
    \[
        \begin{tikzcd}[column sep=tiny, row sep=large]
                (X,x_0)\ar[dr,"i_{x*}"]\ar[ddr,"f", bend right]
                &&(Y,y_0)\ar[dl,"i_{y*}"'] \ar[ddl,"g"', bend left]\\
                &(X,x_0)\vee (Y,y_0)\ar[d, "{(f,g)_*}",dashed]\\
                &(U,u_0)
        \end{tikzcd}
    \]
    commute as desired.
    \end{proof}

    \begin{lemma}
        The wedge product is stable with respect to (basepoint-preserving) homotopy. That is, if
        $X_1\simeq X_2$ and $Y_1\simeq Y_2$ relative to basepoints, then
        \[
            X_1\vee Y_1\simeq X_2\vee Y_2
        \]
    \end{lemma}
    \begin{proof}
        Since the wedge sum is the coproduct of pointed topological spaces, we
        construct the homotopy equivalence as follows:

        Let $f_x:X_1\to X_2$ and $g_x:X_2\to X_1$ be homotopy equivalences of
        $X_2$ and $X_1$. That is, $f_x$ and $g_x$ are homotopy inverses of each
        other. Furthermore, let $f_y:Y_1\to Y_2$ and $g_y:Y_2\to Y_1$ be
        homotopy inverses as well. Then, by the universal property of
        coproducts, we have the commutative diagram
        \[
\begin{tikzcd}
    &X_2\vee Y_2\arrow[dashed, shift left]{ddd}{\tilde{f}}\\
    X_2\arrow{ur}\arrow[shift left]{d}{g_x} &&Y_2\arrow{ul}\arrow[shift left]{d}{g_y}\\
    X_1\arrow{dr}\arrow[shift left]{u}{f_x} &&Y_1\arrow{dl}\arrow[shift left]{u}{f_y}\\
    &X_1\vee Y_1\arrow[dashed, shift left]{uuu}{\tilde{g}}
\end{tikzcd}
        \]
        where $\tilde{f}$ is defined in terms of the universal
        property of coproducts with respect to the compositions
        $\begin{tikzcd}X_1\arrow{r}{f_x} &X_2\arrow{r}&X_2\vee Y_2\end{tikzcd}$
        and
        $\begin{tikzcd}Y_1\arrow{r}{f_y} &Y_2\arrow{r}&_2\vee Y_2\end{tikzcd}$
        (and similarly for $\tilde{g}$).

        I assert that $\tilde{f}$ and $\tilde{g}$ are homotopy inverses. It
        should be clear that by symmetry of the problem, I need only check that
        $\tilde{f}\circ\tilde{g}\simeq\mathbbm{1}$. 

        Suppose $x\in X_1\vee Y_1$, and without loss of generality let $x$ be in
        the inclusion of $X_1$ to $X_1\vee Y_1$. Then, we have the diagram
        \[
\begin{tikzcd}
    &X_2\vee Y_2\arrow[dashed, shift left]{ddd}{\tilde{f}}\\
    X_2\arrow{ur}{i_2}\arrow[shift left]{d}{g_x}\\
    X_1\arrow[shift left]{dr}{i_1}\arrow[shift left]{u}{f_x}\\
    &X_1\vee Y_1\arrow[dashed, shift left]{uuu}{\tilde{g}}
    \arrow[dashed,shift left]{ul}{i_1^{-1}}
\end{tikzcd}
        \]
        (where $i_1^{-1}$ is only defined on the image of $i_1$, but we are
            assuming that $x\in i_1(X_1)$ for this diagram chase, so this arrow
        exists).
        From here, it is clear that $\tilde{g}(x) = i_2\circ f_x(i_1^{-1}(x))$,
        and if we identify $X_1$ and $X_2$ as subspaces of their wedge product,
        we have $\tilde{g}(x) = f_x(x)$. Similarly, we have $\tilde{f}(x) =
        g_x(x)$. Thus,
        \[
            \tilde{f}\circ\tilde{g}(x) = g\circ f(x)\simeq \mathbbm{1}(x)
        \]
        as desired.

        Thus, $\tilde{f}$ and $\tilde{g}$ are homotopy inverses, and $X_1\vee
        Y_1\simeq X_2\vee Y_2$ as desired.
    \end{proof}
    Now, we prove the main result.
    \\
    \\
    Let $Z$ be a CW complex which satisfies the hypotheses. In particular, let
    $X$ and $Y$ be such that $Z=X\cup Y$, $X$ and $Y$ are both contractible, and
    their intersection $A=X\cap Y$ is contractible as well.

    Since $A$ is a contractible subcomplex of $Z$, we know (via Hatcher prop
    0.16 and 0.17) that $Z$ is homotopy equivalent to $Z/{A}$. Thus, we only
    need to show that $Z/{A}$ is contractible.

    Let $q:Z\to Z/{A}$ be the canonical quotient map. Since $Z=X\cup Y$, we know
    that $q(Z) = Z/{A} = q(X)\cup q(Y)$. In particular, $q(X) = X/{A}$ and $q(Y)
    = Y/{A}$ (this follows from the definition of the quotient). Thus, $Z/{A} =
    X/{A}\cup Y/{A}$. Since the quotient map is the identity on $Z\setminus A$
    and collapses $A$ to a point, it follows that $X/{A}\cap Y/{A} = A/{A} =
    \{a_0\}$ where $a_0$ is the point $q(A)$. Thus, $Z/{A}$ is actually the wedge
    sum 
    \[
        Z/{A} = X/{A}\vee Y/{A}
    \]
    
    Now, since $A$ is contractible, we know that $X/{A}\simeq X$ and
    $Y/{A}\simeq Y$. In particular, since $X$ and $Y$ are contractible, so is
    $X/{A}$ and $Y/{A}$. Thus, $X/{A}\simeq Y/{A}\simeq \{\cdot\}$. By lemma 2,
    this implies that
    \[
        X/{A}\vee Y/{A} \simeq \{\cdot\}\vee \{\cdot\} = \{\cdot\}
    \]
    and thus, $X/{A}\vee Y/{A}$ is contractible. Thus, $Z/{A} = X/{A}\vee Y/{A}$
    is contractible as well, and $Z$ itself is contractible as desired.
\end{proof}

\newpage

\section*{Problem 2}
Prove that there is no retraction $r:D^n\to\partial D^n$ using the fact that
$\pi_n(S^n)=\Z$.
\\
\\
\begin{proof}
    Assume for a contradiction that there exists a retraction $r:D^n\to \partial
    D^n$. That is, assume that there is some $r$ such that the diagram
    \[
        \begin{tikzcd}
            \partial D^n \arrow[r, "i"] \arrow[rr, "\mathbbm{1}"', bend right] &
            D^n \arrow[r, "r"] & \partial D^n
        \end{tikzcd}
    \]
    commutes. Here $i$ is the inclusion map of $\partial D^n$ into $D^n$. (Since
    $r$ restricted to $\partial D^n$ is the identity by definition,
    $ri=\mathbbm{1}$).

    Since $\pi_n$ is a covariant functor (Hatcher states this on page 342), we
    can apply it to this diagram to get the commutative diagram
    \[
        \begin{tikzcd}
            \pi_n(\partial D^n)=\Z \arrow[r, "i_*"] \arrow[rr, "\mathbbm{1}"', bend
            right] & \pi_n(D^n)=0 \arrow[r, "r_*"] & \pi_n(\partial D^n)=\Z
        \end{tikzcd}
    \]
    which implies that the identity map from $\Z$ to itself factors through
    zero, which cannot be. Thus, our assumption that $r$ exists is false, and no
    retract of $D^n$ to its boundary exists.
\end{proof}

\newpage

\section*{Problem 3}
In each case, either prove a covering exists, or that no such covering exists.
\subsection*{Part a}
$p:S^2\to T^2$.
\\
\\
\begin{proof}
    I assert that no such covering map exists. Recall from Hatcher proposition
    1.37 that for $X$ a path-connected, locally path-connected space, and
    $p_1:\tilde{X}_1\to X$ and $p_2:\tilde{X}_2$ covering spaces, then
    $\tilde{X}_1$ and $\tilde{X}_2$ are isomorphic if and only if
    $p_{1*}(\pi_1(\tilde{X}_1)) = p_{2*}(\pi_1(\tilde{X}_2))$.

    Now, $T^2$ is path-connected and locally path-connected. Furthermore, it has
    a universal cover $\R^2$. To see this consider the action of $\Z\times \Z$
    on $\R^2$ given by $(m,n)(x,y) = (x+m,y+n)$. Clearly, this action is
    properly discontinuous, since each point moves a distance of at least $1$
    for the action of any nontrivial element of $\Z\times \Z$, and so for any
    point $(x,y)$ we can take the neighborhood $U= V_{\frac{1}{10}}((x,y))$, which
    easily satisfies the property $U\cap g(U) = \emptyset$ for nontrivial $g\in
    \Z\times\Z$.

    Thus, by proposition 1.40, we know that the quotient map $p_{\R^2}:\R^2\to
    \R^2/{\Z\times\Z}$ is a covering space map. It is easy to see that
    $\R^2/{\Z\times\Z} = T^2$, and so $\R^2$ is a simply-connected covering
    space of $T^2$.

    Now, since both $S^2$ and $\R^2$ are simply-connected, we know that
    \[
        p_*(\pi_1(S^2)) = p_{\R^2*}(\pi_1(\R^2)) = 0
    \]
    and so proposition 1.37 guarantees the existence of a covering space
    isomorphism between the two. However, $\R^2$ and $S^2$ are not homeomorphic,
    ($S^2$ is compact but $\R^2$ is not) and so no such isomorphism can exist.
    Thus, $S^2$ cannot cover $T^2$.
    
\end{proof}

\subsection*{Part b}
$p:T^2\to S^2$.
\\
\\
\begin{proof}
    Recall Hatcher proposition 1.31, which asserts that for a covering space
    $p:\tilde{X}\to X$, the induced map $p_*:\pi_1(\tilde{X})\to\pi_1(X)$ is
    injective. So, if such a covering $p:T^2\to S^2$ existed, it would induce an
    injection
    \[
        p_*:\pi_1(T^2)\to\pi_1(S^2)
    \]
    but $\pi_1(T^2) = \Z\times\Z$, and $\pi_1(S^2)=0$, so no such injection can
    exist! Thus, $T^2$ does not cover $S^2$.
\end{proof}

\subsection*{Part c}
$p:\R^2\to T^2$.
\\
\\
\begin{proof}
    This was constructed in part a as a covering map.
\end{proof}

\subsection*{Part d}
$p:\R^2\to S^2$.
\\
\\
\begin{proof}
    I assert that no such covering exists. This follows from the fact that $S^2$
    covers itself, and simply-connected covering spaces are unique up to
    isomorphism.

    Recall from Hatcher proposition 1.37 that for $X$ a
    path-connected, locally path-connected space, and $p_1:\tilde{X}_1\to X$ and
    $p_2:\tilde{X}_2$ covering spaces, then $\tilde{X}_1$ and $\tilde{X}_2$ are
    isomorphic if and only if $p_{1*}(\pi_1(\tilde{X}_1)) =
    p_{2*}(\pi_1(\tilde{X}_2))$.

    Since $S^2$ is path-connected and locally path-connected, it satisfies the
    hypotheses for this proposition. Furthermore, the two covering space maps
    $p:\R^2\to S^2$ and $\mathbbm{1}:S^2\to S^2$ both induce
    \[
        p_*(\pi_1(\R^2)) = \mathbbm{1}_*(\pi_1(S^2)) = 0
    \]
    and thus if such a $p$ existed, there would be a covering space isomorphism
    between $\R^2$ and $S^2$. However, these spaces are not homeomorphic, and
    thus cannot be isomorphic as covering spaces. So, no such $p$ exists.
\end{proof}

\newpage
%TODO problem 4!
\section*{Problem 5}
Given a sequence $f:\N\to(0, \infty)$ define $p_k = (0, 0, f(k))$ and $C_k$ is
the boundary of the triangle in the plane $\{(x, y, 0) \in \R^3\}$ with vertices
$0, v_{2k}, v_{2k+1}$ where $v_n = (\frac{1}{n}, \frac{1}{n^2},0)$.
For $k \in \N$ the set
$A_k = \{t\cdot q + (1 - t) \cdot p_k : q \in C_k, 0 \leq t \leq 1\}$
is the cone from $p_k$ to $C_k$. The subspace $X \subset \R^3$
is $X=\bigcup A_k$.

\subsection*{Part a}
Show $X$ is simply-connected if $f(k)=\frac{1}{k}$.
\\
\\
\begin{proof}
    We begin by considering a homotopy of $X$ defined as follows. For $A_1$, fix
    the subcone, say $B_1$, from $p_2$ to $C_1$ (the cone inside $A_1$ whose face along
    $A_2$ is the entirety of $A_2\cap A_1$), and perform a straight line
    homotopy of the rest of $A_1$ downwards into $B_1$. 

    The homotopy on $A_k$ is defined inductively. We retract $\bigcup_{i=1}^k
    A_k$ (where each $A_i$ has already been moved to have a vertex at $p_k$)
    onto the subcone from $p_{k+1}$ to $\bigcup_{i=1}^k C_k$ using the straight
    line homotopy.

    Finally, we stitch these homotopies together by letting the $k$th homotopy
    occur in the time $[1-\frac{1}{k}, 1-\frac{1}{k+1}]$. Now, for each $t\neq
    1$, this action is continuous, as each individual retract of $A_k$ is
    continuous as a straight line homotopy. Furthermore, this homotopy is
    continuous at $1$. To see this, consider any point $x\in X$, Clearly, this
    homotopy takes $x$ to its projection on the $x-y$ plane. So, if the height
    of $x$ is not zero,
    then $x$ is in some $A_k$ (and possibly $A_{k-1}$ if its on the boundary),
    and is thus at most a distance $\frac{1}{k}$ from its projection. In
    particular, since $x$ moves downwards from height $\frac{1}{i}$ to height
    $\frac{1}{i+1}$ in time $t\in [1-\frac{1}{i},1-\frac{1}{i+1}]$, the homotopy
    is clearly continuous at $x$. Finally, if the height of $x$ is zero, then
    $x$ is fixed by the homotopy, and the homotopy is continuous at $x$.

    Thus, $X$ is homotopic to $\bigcup_{k=1}^{\infty}\tilde{C}_k$ for
    $\tilde{C}_k$ the filled in triangle with boundary $C_k$. This space is
    clearly simply connected, so $X$ is simply connected as well.
\end{proof}

\newpage

\subsection*{Part b}
Show $X$ is not simply-connected if $f(k) = k$.
\\
\\
\begin{proof}
    We proceed by constructing a nontrivial loop in $X$. To do so, consider the
    loop $\gamma$ which goes around $C_k$ in time
    $[1-\frac{1}{k},1-\frac{1}{k+1}]$. This loop is continuous on $[0,1)$, since
    at each $t\in [0,1)$, $t$ is in some interval
    $[1-\frac{1}{i},1-\frac{1}{i+1}]$ and is thus part of the loop around $C_i$.
    Furthermore, this loop is continuous at $t=1$. To see this, observe that any
    neighborhood $U$ of $\gamma(1)$ necessarily contains all but finitely many
    $C_k$. Thus, for $n$ the smallest integer for which $C_n$ is completely
    contained in $U$. The image $\gamma((1-\frac{1}{n},1])$ of the open set
    $(1-\frac{1}{n},1]$ is equal to $\bigcup_{k=n}^{\infty}C_k$ which is
    entirely in $U$ by definition of $n$.

    Thus, this defines a loop in $X$. However, this loop is nontrivial, since
    the only way to homotope the part of the loop around $C_k$ to the constant
    loop is by taking it over $p_k$. Since the heights $p_k$ go to infinity as
    $k$ does, it follows that there is no continuous way to homotope all of
    $\gamma$ to the constant loop (the speeds of the homotopy would have to
    diverge, which cannot be continuous).

    Thus, $X$ is not simply connected.
\end{proof}


\end{document}
