%%%%%%%%%%%%%%%%%%%%%%%%%%%%%%%%%%%%%%%%%
% Short Sectioned Assignment
% LaTeX Template
% Version 1.0 (5/5/12)
%
% This template has been downloaded from:
% http://www.LaTeXTemplates.com
%
% Original author:
% Frits Wenneker (http://www.howtotex.com)
%
% License:
% CC BY-NC-SA 3.0 (http://creativecommons.org/licenses/by-nc-sa/3.0/)
%
%%%%%%%%%%%%%%%%%%%%%%%%%%%%%%%%%%%%%%%%%

%----------------------------------------------------------------------------------------
%	PACKAGES AND OTHER DOCUMENT CONFIGURATIONS
%----------------------------------------------------------------------------------------

\documentclass[fontsize=11pt]{scrartcl} % 11pt font size

\usepackage[T1]{fontenc} % Use 8-bit encoding that has 256 glyphs
\usepackage[english]{babel} % English language/hyphenation
\usepackage{amsmath,amsfonts,amsthm} % Math packages
\usepackage{mathrsfs}
\usepackage{tikz-cd}

\usepackage[margin=1in]{geometry}

\usepackage{sectsty} % Allows customizing section commands
\allsectionsfont{\centering \normalfont\scshape} % Make all sections centered, the default font and small caps

\usepackage{fancyhdr} % Custom headers and footers
\pagestyle{fancyplain} % Makes all pages in the document conform to the custom headers and footers
\fancyhead{} % No page header - if you want one, create it in the same way as the footers below
\fancyfoot[L]{} % Empty left footer
\fancyfoot[C]{} % Empty center footer
\fancyfoot[R]{\thepage} % Page numbering for right footer
\renewcommand{\headrulewidth}{0pt} % Remove header underlines
\renewcommand{\footrulewidth}{0pt} % Remove footer underlines
\setlength{\headheight}{13.6pt} % Customize the height of the header

\numberwithin{equation}{section} % Number equations within sections (i.e. 1.1, 1.2, 2.1, 2.2 instead of 1, 2, 3, 4)
\numberwithin{figure}{section} % Number figures within sections (i.e. 1.1, 1.2, 2.1, 2.2 instead of 1, 2, 3, 4)
\numberwithin{table}{section} % Number tables within sections (i.e. 1.1, 1.2, 2.1, 2.2 instead of 1, 2, 3, 4)

\newcommand{\R}{\mathbb{R}}
\newcommand{\Q}{\mathbb{Q}}
\newcommand{\N}{\mathbb{N}}
\newcommand{\C}{\mathbb{C}}
\newcommand{\Z}{\mathbb{Z}}

\newcommand{\Hom}{\textrm{Hom}}

\newtheorem{lemma}{Lemma}
%----------------------------------------------------------------------------------------
%	TITLE SECTION
%----------------------------------------------------------------------------------------

\newcommand{\horrule}[1]{\rule{\linewidth}{#1}} % Create horizontal rule command with 1 argument of height

\title{	
\normalfont \normalsize 
\textsc{Homotopy Theory} \\ [25pt] % Your university, school and/or department name(s)
\horrule{0.5pt} \\[0.4cm] % Thin top horizontal rule
\huge Problem Set 7 \\ % The assignment title
\horrule{2pt} \\[0.5cm] % Thick bottom horizontal rule
}

\author{Daniel Halmrast} % Your name

\date{\normalsize\today} % Today's date or a custom date

\begin{document}

\maketitle % Print the title

% Problems
\section*{Problem 1}
Show that if a path-connected, locally path-connected space $X$ has $\pi_1$
finite, then every map $X\to S^1$ is nullhomotopic.
\\
\\
\begin{proof}
    Let $f\in C(X,S^1)$. Since $\pi_1(X)$ is finite, we know that
    $f_*:\pi_1(X)\to\pi_1(S^1)$ is trivial, since $\pi_1(S^1) = \Z$ and
    $\pi_1(X)$ is finite. Thus, the conditions are satisfied for $f$ to lift to
    the universal cover $\R$ of $S^1$.

    Thus, we have
    \[
\begin{tikzcd}
    &\R\arrow{d}{p}\\
    X\arrow{r}{f}\arrow[dashed]{ru}{\tilde{f}} &S^1
\end{tikzcd}
    \]

    Now, since $\R$ is simply connected, $\tilde{f}$ is nullhomotopic. Thus, the
    homotopy of $\tilde{f}$ with a constant map descends via $p$ to a homotopy
    of $f$ with a constant map. Thus, $f$ is nullhomotopic as desired.
\end{proof}

\section*{Problem 2}
Construct finite graphs $X_1,X_2$ which have a common finite-sheeted covering
space, but such that there is no space having both $X_1$ and $X_2$ as a covering
space.
\\
\\
\\
\\
\newpage

\section*{Problem 3}
Let $a,b$ be the generators of $\pi_1(S^1\vee S^1)$. Draw a picture of the
covering space of $S^1\wedge S^1$ corresponding to the normal subgroup generated
by $a^2,b^2,(ab)^4$, and prove that this is the correct space.
\\
\\
\newpage

\section*{Problem 4}
Let $\tilde{X}$ be a simply connected covering space of $X$, and $A\subset X$ be
a path-connected, locally path-connected subspace with $\tilde{A}$ a path
component of $p^{-1}(A)$. Show that $p:\tilde{A}\to A$ is a covering space
corresponding to the kernel of $i_*:\pi_1(A)\to\pi_1(X)$.
\\
\\
\begin{proof}
    First, we show that $p_*\pi_1(\tilde{A})\subseteq \ker i_*$. To see this,
    note that the diagram
    \[
\begin{tikzcd}
    \tilde{A}\arrow{d}{p}\arrow{r}{i} &\tilde{X}\arrow{d}{p}\\
    A\arrow{r}{i} &X
\end{tikzcd}
    \]
    commutes. This implies that the induced diagram
    \[
\begin{tikzcd}
    \pi_1(\tilde{A})\arrow{d}{p_*}\arrow{r}{i_*} &\pi_1(\tilde{X})=0\arrow{d}{p_*}\\
    \pi_1(A)\arrow{r}{i_*} &\pi_1(X)
\end{tikzcd}
    \]
    and a simple diagram chase shows that $p_*\pi_1(\tilde{A})\subseteq\ker i_*$
    as desired.

    Now we show the reverse. Let $[f]$ be a loop in $A$ with $i_*([f])=0$. Then,
    by the lifting property of $p:\tilde{X}\to X$, we have
    \[
\begin{tikzcd}
    &&\tilde{X}\arrow{d}{p}\\
    S^1\arrow{r}{f}\arrow[dashed]{urr}{\tilde{f}} &A\arrow{r}{i} &X
\end{tikzcd}
    \]
    where we fix $\tilde{f}(0)$ to be in $\tilde{A}$.
    Now, since this commutes, this implies that $\tilde{f}$ stays in
    $\tilde{A}$, and defines a loop in $\tilde{A}$ that projects to $f$. Thus,
    $p_*([\tilde{f}]) = [f]$ and thus $\ker i_* \subseteq p_*(\tilde{A})$ as
    desired.
\end{proof}

\newpage

\section*{Problem 5}
Given covering space actions $G_1,G_2$ on $X_1,X_2$, show that the product
action $G_1\times G_2$ on $X_1\times X_2$ is a covering space action isomorphic
to $X_1/{G_1}\times X_2/{G_2}$.
\\
\\
\begin{proof}
    Since $G_i$ is a covering space action on $X_i$, the quotient map
    $q_i:X_i\to X_i/{G_i}$ is a covering space map. Now, we proved in the last
    homework set that cartesian products of covering spaces are again covering
    spaces with the product map (the category of covering spaces is cartesian
    closed?). Thus, 
    \[
        p_1\times p_2:X_1\times X_2\to X_1/{G_1}\times X_2/{G_2}
    \]
    is a covering space. Furthermore, note that the quotient map $p_1\times p_2$
    has image
    \[
        \frac{X_1\times X_2}{G_1\times G_2}
    \]
    by the definition of the action on $X_1\times X_2$. Thus, the two base
    spaces are actually isomorphic as desired.
    Furthermore, this action is a covering space action, since the action of
    $G_1\times G_2$ is a deck transformation of $X_1\times X_2$, and deck
    transformations are covering space actions.
\end{proof}

\newpage

\section*{Problem 6}
Show that if a group $G$ acts freely and properly discontinuously on a Hausdorff
space $X$, then $G$ is a covering space action.
\\
\\
\begin{proof}
    Let $x\in X$, and let $U$ be a neighborhood of $x$ such that $H_x = \{U\cap g(U)
    \neq \emptyset\}$ is finite. Then for each $g\in H_x$ with $g\neq e$, we
    know that $gx\neq x$, and since $X$ is Hausdorff, there exist disjoint open
    sets $U_{gx}$ and $V_{gx}$ containing $x$ and $gx$ respectively. Now, let 
    \[
        W = \bigcap_{g\in H_x,g\neq e}\left( U_{gx}\cap g^{-1}(V_{gx}) \right)
    \]
    This satisfies the conditions for a neighborhood of $x$ making $G$ act as a
    covering space action. To see this, let $g\in G$, and let $y\in W$. If
    $g\not\in H_x$, then $U\cap g(U) = \emptyset$ by definition, and so $W\cap
    g(W) = \emptyset$ as well.

    Now, suppose $g\in H_x$ with $g\neq e$. Then, we know $y\in U_{gx}$ and
    $y\in g^{-1}(V_{gx})$, so $gy\in V_{gx}$. In particular, this means that
    $gy\not\in U_{gx}$, and so $gy\not\in W$ as desired.

    Thus, $G$ acts as a covering space action.
\end{proof}

\end{document}
