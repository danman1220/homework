%%%%%%%%%%%%%%%%%%%%%%%%%%%%%%%%%%%%%%%%%
% Short Sectioned Assignment
% LaTeX Template
% Version 1.0 (5/5/12)
%
% This template has been downloaded from:
% http://www.LaTeXTemplates.com
%
% Original author:
% Frits Wenneker (http://www.howtotex.com)
%
% License:
% CC BY-NC-SA 3.0 (http://creativecommons.org/licenses/by-nc-sa/3.0/)
%
%%%%%%%%%%%%%%%%%%%%%%%%%%%%%%%%%%%%%%%%%

%----------------------------------------------------------------------------------------
%	PACKAGES AND OTHER DOCUMENT CONFIGURATIONS
%----------------------------------------------------------------------------------------

\documentclass[fontsize=11pt]{scrartcl} % 11pt font size

\usepackage[T1]{fontenc} % Use 8-bit encoding that has 256 glyphs
\usepackage[english]{babel} % English language/hyphenation
\usepackage{amsmath,amsfonts,amsthm} % Math packages
\usepackage{mathrsfs}
\usepackage{bbm}

\usepackage[margin=1in]{geometry}

\usepackage{sectsty} % Allows customizing section commands
\allsectionsfont{\centering \normalfont\scshape} % Make all sections centered, the default font and small caps

\usepackage{fancyhdr} % Custom headers and footers
\pagestyle{fancyplain} % Makes all pages in the document conform to the custom headers and footers
\fancyhead{} % No page header - if you want one, create it in the same way as the footers below
\fancyfoot[L]{} % Empty left footer
\fancyfoot[C]{} % Empty center footer
\fancyfoot[R]{\thepage} % Page numbering for right footer
\renewcommand{\headrulewidth}{0pt} % Remove header underlines
\renewcommand{\footrulewidth}{0pt} % Remove footer underlines
\setlength{\headheight}{13.6pt} % Customize the height of the header

\numberwithin{equation}{section} % Number equations within sections (i.e. 1.1, 1.2, 2.1, 2.2 instead of 1, 2, 3, 4)
\numberwithin{figure}{section} % Number figures within sections (i.e. 1.1, 1.2, 2.1, 2.2 instead of 1, 2, 3, 4)
\numberwithin{table}{section} % Number tables within sections (i.e. 1.1, 1.2, 2.1, 2.2 instead of 1, 2, 3, 4)

\newcommand{\R}{\mathbb{R}}
\newcommand{\Q}{\mathbb{Q}}
\newcommand{\N}{\mathbb{N}}
\newcommand{\C}{\mathbb{C}}

\newtheorem{lemma}{Lemma}
%----------------------------------------------------------------------------------------
%	TITLE SECTION
%----------------------------------------------------------------------------------------

\newcommand{\horrule}[1]{\rule{\linewidth}{#1}} % Create horizontal rule command with 1 argument of height

\title{	
\normalfont \normalsize 
\textsc{Homotopy Theory} \\ [25pt] % Your university, school and/or department name(s)
\horrule{0.5pt} \\[0.4cm] % Thin top horizontal rule
\huge Problem Set 1 \\ % The assignment title
\horrule{2pt} \\[0.5cm] % Thick bottom horizontal rule
}

\author{Daniel Halmrast} % Your name

\date{\normalsize\today} % Today's date or a custom date

\begin{document}

\maketitle % Print the title

% Problems
\section*{Problem 1} %Hatcher 0.2
Construct an explicit deformation retraction of $\R^n\setminus \{0\}$ to
$S^{n-1}$.
\\
\\
\begin{proof}
    The straight-line homotopy from $v$ to $\frac{v}{\|v\|}$ satisfies the
    criteria for a deformation retract. Namely, the retract is given by
    \[
        \begin{aligned}
            r:&\R^n\setminus\{0\}\to S^{n-1}\\
            r(v) &= \frac{v}{\|v\|}
        \end{aligned}
    \]
    With homotopy
    \[
        \begin{aligned}
            F:&\R^n\setminus\{0\}\times I\to S^{n-1}\\
            F(v,t) &= (1-t)v + t\frac{v}{\|v\|}
        \end{aligned}
    \]
\end{proof}

\newpage

\section*{Problem 2} %Hatcher 0.3c
Show that a map homotopic to a homotopy equivalence is a homotopy equivalence.
\\
\\
\begin{proof}
    Let $f:X\to Y$ be a map, which is homotopic to a homotopy equivalence
    $g:X\to Y$ with homotopy inverse $h:Y\to X$. That is, $g\circ h \simeq
    \mathbbm{1}_Y$
    and $h\circ g \simeq \mathbbm{1}_X$. Furthermore, let $F:X\times I\to Y$ be the
    homotopy between $f$ and $g$.

    First, let's consider the map $h\circ f:X\to X$. We wish to show
    $h\circ f\simeq \mathbbm{1}_X$. To do so, let's consider the
    homotopy 
    \[
        h\circ F:X\times I\to X
    \]
    This is the composition of two continuous functions, and so it is
    continuous. Furthermore, since $F(0,x) = f(x)$ and $F(1,x)=g(x)$, this is
    actually a homotopy between $h\circ f$ and $h\circ g$. Now, since $h\circ
    f\simeq h\circ g\simeq \mathbbm{1}_X$ and homotopy equivalence is an
    equivalence relation, it follows immediately that $h\circ f\simeq
    \mathbbm{1}_X$.

    Now, consider the map $f\circ h:Y\to Y$. We wish to show $f\circ h\simeq
    \mathbbm{1}_Y$. To do so, consider the homotopy 
    \[
        F\circ(h\times \mathbbm{1}_I): Y\times I\to Y
    \]
    It is easy to see this is a homotopy between $f\circ h$ and $g\circ h$, and
    so we have that $f\circ h \simeq g\circ h \simeq \mathbbm{1}_X$, and so
    $f\circ h \simeq \mathbbm{1}_X$, as desired.
\end{proof}

\newpage

\section*{Problem 3} %Hatcher 0.4
A deformation retraction in the weak sense of a space $X$ to a subspace $A$ is a
homotopy $f_t:X\to X$ such that $f_0 = \mathbbm{1}_X$, $f_1(X)\subset A$, and
$f_t(A)\subset A$ for all $t$. Show that if $X$ deformation retracts onto $A$ in
the weak sense, then the inclusion map $i:A\to X$ is a homotopy equivalence.
\\
\\
\begin{proof}

\end{proof}


\end{document}
