%%%%%%%%%%%%%%%%%%%%%%%%%%%%%%%%%%%%%%%%%
% Short Sectioned Assignment
% LaTeX Template
% Version 1.0 (5/5/12)
%
% This template has been downloaded from:
% http://www.LaTeXTemplates.com
%
% Original author:
% Frits Wenneker (http://www.howtotex.com)
%
% License:
% CC BY-NC-SA 3.0 (http://creativecommons.org/licenses/by-nc-sa/3.0/)
%
%%%%%%%%%%%%%%%%%%%%%%%%%%%%%%%%%%%%%%%%%

%----------------------------------------------------------------------------------------
%	PACKAGES AND OTHER DOCUMENT CONFIGURATIONS
%----------------------------------------------------------------------------------------

\documentclass[fontsize=11pt]{scrartcl} % 11pt font size

\usepackage[T1]{fontenc} % Use 8-bit encoding that has 256 glyphs
\usepackage[english]{babel} % English language/hyphenation
\usepackage{amsmath,amsfonts,amsthm} % Math packages
\usepackage{mathrsfs}
\usepackage{bbm}

\usepackage[margin=1in]{geometry}

\usepackage{sectsty} % Allows customizing section commands
\allsectionsfont{\centering \normalfont\scshape} % Make all sections centered, the default font and small caps

\usepackage{fancyhdr} % Custom headers and footers
\pagestyle{fancyplain} % Makes all pages in the document conform to the custom headers and footers
\fancyhead{} % No page header - if you want one, create it in the same way as the footers below
\fancyfoot[L]{} % Empty left footer
\fancyfoot[C]{} % Empty center footer
\fancyfoot[R]{\thepage} % Page numbering for right footer
\renewcommand{\headrulewidth}{0pt} % Remove header underlines
\renewcommand{\footrulewidth}{0pt} % Remove footer underlines
\setlength{\headheight}{13.6pt} % Customize the height of the header

\numberwithin{equation}{section} % Number equations within sections (i.e. 1.1, 1.2, 2.1, 2.2 instead of 1, 2, 3, 4)
\numberwithin{figure}{section} % Number figures within sections (i.e. 1.1, 1.2, 2.1, 2.2 instead of 1, 2, 3, 4)
\numberwithin{table}{section} % Number tables within sections (i.e. 1.1, 1.2, 2.1, 2.2 instead of 1, 2, 3, 4)

\newcommand{\R}{\mathbb{R}}
\newcommand{\Q}{\mathbb{Q}}
\newcommand{\N}{\mathbb{N}}
\newcommand{\C}{\mathbb{C}}

\newtheorem{lemma}{Lemma}
%----------------------------------------------------------------------------------------
%	TITLE SECTION
%----------------------------------------------------------------------------------------

\newcommand{\horrule}[1]{\rule{\linewidth}{#1}} % Create horizontal rule command with 1 argument of height

\title{	
\normalfont \normalsize 
\textsc{Homotopy Theory} \\ [25pt] % Your university, school and/or department name(s)
\horrule{0.5pt} \\[0.4cm] % Thin top horizontal rule
\huge Problem Set 1 \\ % The assignment title
\horrule{2pt} \\[0.5cm] % Thick bottom horizontal rule
}

\author{Daniel Halmrast} % Your name

\date{\normalsize\today} % Today's date or a custom date

\begin{document}

\maketitle % Print the title

% Problems
\section*{Problem 1} %Hatcher 0.2
Construct an explicit deformation retraction of $\R^n\setminus \{0\}$ to
$S^{n-1}$.
\\
\\
\begin{proof}
    The straight-line homotopy from $v$ to $\frac{v}{\|v\|}$ satisfies the
    criteria for a deformation retract. Namely, the retract is given by
    \[
        \begin{aligned}
            r:&\R^n\setminus\{0\}\to S^{n-1}\\
            r(v) &= \frac{v}{\|v\|}
        \end{aligned}
    \]
    With homotopy
    \[
        \begin{aligned}
            F:&\R^n\setminus\{0\}\times I\to S^{n-1}\\
            F(v,t) &= (1-t)v + t\frac{v}{\|v\|}
        \end{aligned}
    \]
\end{proof}

\newpage

\section*{Problem 2} %Hatcher 0.3c
Show that a map homotopic to a homotopy equivalence is a homotopy equivalence.
\\
\\
\begin{proof}
    Let $f:X\to Y$ be a map, which is homotopic to a homotopy equivalence
    $g:X\to Y$ with homotopy inverse $h:Y\to X$. That is, $g\circ h \simeq
    \mathbbm{1}_Y$
    and $h\circ g \simeq \mathbbm{1}_X$. Furthermore, let $F:X\times I\to Y$ be the
    homotopy between $f$ and $g$.

    First, let's consider the map $h\circ f:X\to X$. We wish to show
    $h\circ f\simeq \mathbbm{1}_X$. To do so, let's consider the
    homotopy 
    \[
        h\circ F:X\times I\to X
    \]
    This is the composition of two continuous functions, and so it is
    continuous. Furthermore, since $F(0,x) = f(x)$ and $F(1,x)=g(x)$, this is
    actually a homotopy between $h\circ f$ and $h\circ g$. Now, since $h\circ
    f\simeq h\circ g\simeq \mathbbm{1}_X$ and homotopy equivalence is an
    equivalence relation, it follows immediately that $h\circ f\simeq
    \mathbbm{1}_X$.

    Now, consider the map $f\circ h:Y\to Y$. We wish to show $f\circ h\simeq
    \mathbbm{1}_Y$. To do so, consider the homotopy 
    \[
        F\circ(h\times \mathbbm{1}_I): Y\times I\to Y
    \]
    It is easy to see this is a homotopy between $f\circ h$ and $g\circ h$, and
    so we have that $f\circ h \simeq g\circ h \simeq \mathbbm{1}_X$, and so
    $f\circ h \simeq \mathbbm{1}_X$, as desired.
\end{proof}

\newpage

\section*{Problem 3} %Hatcher 0.4
A deformation retraction in the weak sense of a space $X$ to a subspace $A$ is a
homotopy $f_t:X\to X$ such that $f_0 = \mathbbm{1}_X$, $f_1(X)\subset A$, and
$f_t(A)\subset A$ for all $t$. Show that if $X$ deformation retracts onto $A$ in
the weak sense, then the inclusion map $i:A\to X$ is a homotopy equivalence.
\\
\\
\begin{proof}
Let $f_t:X\to X$ be a deformation retraction in the weak sense of $X$ onto
$A$, and let $i$ be the inclusion map from $A$ to $X$. We will show that
    $i\circ f_1 \simeq \mathbbm{1}_X$ and that $f_1\circ i \simeq
    \mathbbm{1}_A$.
    
    Considering $i\circ f_1$, we note that this is actually equal to $f_1$,
    since the inclusion map is the identity on $A$, and $f_1$ maps into $A$.
    Now, $f_1$ is homotopic to $f_0$ which is equal to $\mathbbm{1}_X$, and so
    by transitivity of homotopy equivalence, $f_1\simeq \mathbbm{1}_X$.

    Now, let's consider $f_1\circ i$. We note first that this is equal to
    $f_1|_A$, since $i$ is the identity on $A$. Furthermore, the restrictions
    $f_t|_A$ define a homotopy from $f_1|_A$ to $f_0|_A$, and so we have that
    \[
        f_1\circ i = f_1|_A\simeq f_0|_A = \mathbbm{1}_X|_A = \mathbbm{1}_A
    \]
    as desired. Thus, $i$ is a homotopy equivalence with homotopy inverse $f_1$.
\end{proof}

\section*{Problem 4} %Hatcher 0.5
Show that if a space $X$ deformation retracts to a point $x\in X$, then for each
neighborhood $U$ of $x$ in $X$, there exists a neighborhood $V\subset U$ of $x$
such that the inclusion map $V\to U$ is nullhomotopic.
\\
\\
\begin{proof}
    Let $F:X\times I\to X$ be the deformation retraction of $X$ onto $x_0\in X$,
    and let $U$ be a neighborhood of $x_0$. Now, consider the open set
    \[
        F^{-1}(U)\subset X\times I
    \]
    Now, since $F(x_0,t) = x_0$ for all $t$, we know that $\{x_0\}\times I$ is
    in $F^{-1}(U)$. Applying the tube lemma, we find an open set $V$ containing
    $x_0$ for which $V\times I\subset F^{-1}(U)$. In particular, this means that
    for all $v\in V$, we have that $F(v,t)\in U$ for all $t$.

    Now we are ready to show that the inclusion map from $V$ to $U$ is
    nullhomotopic. We note that $F\circ(i\times \mathbbm{1}_I):V\times I\to U$
    defines a homotopy from $f_0\circ i = i$ to $f_1\circ i = c_{x_0}$, where
    $c_{x_0}$ is the constant function to $x_0$. This is clear, since the image
    of $i$ is $V\subset U$, and the image of $V$ under $F$ is always in $U$, as
    proved above. Thus, since the domain of $F$ matches the image of $i$, and
    the image of $F$ stays inside $U$, this is a well-defined homotopy.

    Therefore, $i\simeq c_{x_0}$ as desired.
\end{proof}

\section*{Problem 5} %Hatcher 0.6
Consider the subspace $X\subset \R^2$ defined as
\[
    X = [0,1]\times\{0\} \cup \bigcup_{r\in\Q}\{r\}\times[0,1-r]
\]
\subsection*{Part a}
Show that $X$ deformation retracts to any point in $[0,1]\times\{0\}$, but not
any other point.
\\
\\
\begin{proof}
    To construct the deformation retraction of $X$ to a point in the interval
    $[0,1]\times\{0\}$, we first note that $X$ deformation retracts onto
    $[0,1]\times\{0\}$ via the straight-line homotopy along the $y$-axis. It is
    clear also that the unit interval deformation retracts to any point on it
    via the straight-line homotopy along the $x$-axis. Running the first
    homotopy for the first half time, and running the second homotopy for the
    second half time yields a deformation retraction of $X$ onto a point in the
    interval $[0,1]\times\{x\}$.

    Now, consider a point $x$ not in the base interval. Consider also a
    neighborhood $U$ of $x$ that does not intersect the base interval. Any
    neighborhood $V\subset U$ containing $x$ will necessarily intersect at least
    one other stalk than the one $x$ is in (since the rationals are dense in
    $\R$), and these stalks will not be connected, since $V$ does not intersect
    the base interval. Thus, the inclusion map of $V$ into $U$ cannot be
    nullhomotopic, and by problem 4, we know that $X$ therefore cannot
    deformation retract onto $x$.
\end{proof}

\subsection*{Part b}
Let $Y$ be the subset of $\R^2$ that is the union of infinite copies of $X$ in a
zigzag pattern. Show that $Y$ is contractible, but does not deformation retract
onto any point.
\\
\\
\begin{proof}
    To show that $Y$ is contractible, we reference part c of this problem, which
    asserts the existence of a deformation retraction in the weak sense of $Y$
    onto the zigzag subspace $Z$. Now, problem 3 guarantees that if $Y$
    deformation retracts onto $Z$ in the weak sense, then the inclusion $i:Z\to
    Y$ is a homotopy equivalence, and thus $Y$ and $Z$ have the same homotopy
    type. However, $Z$ is homeomorphic to $\R$, which has the homotopy type of a
    point. Therefore, by transitivity of the homotopy equivalence, $Y$ has the
    homotopy type of a point as well.

    Now, we must show that $Y$ does not deformation retract onto any point. To
    do so, we look at any point $x$ in $Y$. If $x$ is not in $Z$, the same
    argument from part a can be applied to show that $Y$ cannot deformation
    retract onto $x$. If $x$ is in $Z$, we observe that $x$ is actually in a
    stalk of the copy of $X$ running parallel to the line segment of $Z$ that
    $x$ is on. Noting then that $x$ is on a stalk, we apply the same argument as
    the one in part a to see that $Y$ cannot deformation retract onto $x$.
\end{proof}

\subsection*{Part c}
Let $Z$ be the zigzag subspace of $Y$ homeomorphic to $\R$. Show that there is a
deformation retraction in the weak sense of $Y$ onto $Z$, but no true
deformation retraction. 
\\
\\
\begin{proof}
     We can construct a deformation retraction in the weak sense explicitly for
     $Y$ onto $Z$. For each stalk, we define its ``direction of motion'' to be
     towards $Z$, and on $Z$ we define its ``direction of motion'' to be towards
     the right. Now, the deformation retraction in the weak sense sends points
     at constant velocity $1$ along the direction of motion. Away from $Z$, this
     is clearly continuous, and on $Z$, we see that all points are moving at the
     same speed, so whatever stalks $Z$ is close to are retracting at the same
     speed $Z$ itself is moving. Thus, points stay close to each other, and the
     motion is continuous. This is a weak deformation retraction, since it does
     not fix any point in $Z$.

     However, there is no true deformation retraction of $Y$ onto $Z$, since if
     there were, it could be concatenated with a deformation retraction of $Z$
     onto a point in $Z$ to yield a deformation retraction of $Y$ onto a point
     in $Z\subset Y$. However, this would contradict part b, and so no such
     deformation retraction can exist.
\end{proof}

\end{document}
