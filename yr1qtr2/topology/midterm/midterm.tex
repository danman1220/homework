%%%%%%%%%%%%%%%%%%%%%%%%%%%%%%%%%%%%%%%%%
% Short Sectioned Assignment
% LaTeX Template
% Version 1.0 (5/5/12)
%
% This template has been downloaded from:
% http://www.LaTeXTemplates.com
%
% Original author:
% Frits Wenneker (http://www.howtotex.com)
%
% License:
% CC BY-NC-SA 3.0 (http://creativecommons.org/licenses/by-nc-sa/3.0/)
%
%%%%%%%%%%%%%%%%%%%%%%%%%%%%%%%%%%%%%%%%%

%----------------------------------------------------------------------------------------
%	PACKAGES AND OTHER DOCUMENT CONFIGURATIONS
%----------------------------------------------------------------------------------------

\documentclass[fontsize=11pt]{scrartcl} % 11pt font size

\usepackage[T1]{fontenc} % Use 8-bit encoding that has 256 glyphs
\usepackage[english]{babel} % English language/hyphenation
\usepackage{amsmath,amsfonts,amsthm} % Math packages
\usepackage{mathrsfs}

\usepackage[margin=1in]{geometry}

\usepackage{sectsty} % Allows customizing section commands
\allsectionsfont{\centering \normalfont\scshape} % Make all sections centered, the default font and small caps

\usepackage{fancyhdr} % Custom headers and footers
\pagestyle{fancyplain} % Makes all pages in the document conform to the custom headers and footers
\fancyhead{} % No page header - if you want one, create it in the same way as the footers below
\fancyfoot[L]{} % Empty left footer
\fancyfoot[C]{} % Empty center footer
\fancyfoot[R]{\thepage} % Page numbering for right footer
\renewcommand{\headrulewidth}{0pt} % Remove header underlines
\renewcommand{\footrulewidth}{0pt} % Remove footer underlines
\setlength{\headheight}{13.6pt} % Customize the height of the header

\numberwithin{equation}{section} % Number equations within sections (i.e. 1.1, 1.2, 2.1, 2.2 instead of 1, 2, 3, 4)
\numberwithin{figure}{section} % Number figures within sections (i.e. 1.1, 1.2, 2.1, 2.2 instead of 1, 2, 3, 4)
\numberwithin{table}{section} % Number tables within sections (i.e. 1.1, 1.2, 2.1, 2.2 instead of 1, 2, 3, 4)

\newcommand{\R}{\mathbb{R}}
\newcommand{\Q}{\mathbb{Q}}
\newcommand{\N}{\mathbb{N}}
\newcommand{\C}{\mathbb{C}}
\newcommand{\Aut}{\text{Aut}}
\newcommand{\im}{\text{im}}

\newtheorem{lemma}{Lemma}
%----------------------------------------------------------------------------------------
%	TITLE SECTION
%----------------------------------------------------------------------------------------

\newcommand{\horrule}[1]{\rule{\linewidth}{#1}} % Create horizontal rule command with 1 argument of height

\title{	
\normalfont \normalsize 
\textsc{<<CLASS>>} \\ [25pt] % Your university, school and/or department name(s)
\horrule{0.5pt} \\[0.4cm] % Thin top horizontal rule
\huge <<TITLE>> \\ % The assignment title
\horrule{2pt} \\[0.5cm] % Thick bottom horizontal rule
}

\author{Daniel Halmrast} % Your name

\date{\normalsize\today} % Today's date or a custom date

\begin{document}

\maketitle % Print the title

% Problems
\section*{Problem 1}
Show that for two nontrivial groups $G$ and $H$, their free product $G\star H$
has trivial center, and that the only elements of finite order are the
conjugates of finite order elements in $G$ and $H$.
\\
\\
\begin{proof}
    We first show that the free product has a trivial center. To show this, we
    will consider two cases of elements from $G\star H$. First, observe
    that for any element $g\in G$ with $g\neq e$, $g$ (or rather, the inclusion
    of $g$ into $G\star H$) cannot be in the center $Z(G\star H)$. To see this,
    let $h\in H$ with $h\neq e$, and note that the element $ghg^{-1}h^{-1}$ is a
    reduced word in the free group, and cannot be equal to the identity. Thus,
    $gh\neq hg$, and $g$ is not in $Z(G\star H)$.  Similarly, any element $h\in
    H$ is not in $Z(G\star H)$.

    Now, let $w$ be a reduced word in $G\star H$ that is not in $G$ or $H$.
    Without loss of generality, we can write
    \[
        w = g_1h_1g_2h_2\dots g_nh_n
    \]
    or
    \[
        w = g_1h_1g_2h_2\dots g_n
    \]
    for $g_i\in G$, and $h_i\in H$. Now, since $G$ and $H$ have no relations to
    each other, this word is not equal to any word beginning with an element of
    $H$. Suppose for a contradiction that
    \[
        w = h'_1g'_1h'_2g'_2\dots h'_mg'_m
    \]
    with $g'_m$ possibly equal to the identity.
    This would imply that
    \[
        g_1h_1g_2h_2\dots g_nh_n = h'_1g'_1h'_2g'_2\dots h'_mg'_m
    \]
    which would imply the relation
    \[
        g_1h_1g_2h_2\dots g_nh_n(g'_m)^{-1}(h'_m)^{-1}\dots
        (g'_1)^{-1}(h'_1)^{-1} = e
    \]
    Now, if $g'_m\neq e$, the word on the left hand side is already reduced, and
    thus cannot be equal to $e$. If $g'_m=e$, then the word could reduce
    further. However, note that this word only reduces further if adjacent
    elements from the same group cancel to the identity (that is, if adjacent
        elements from the same group do not cancel to the identity, then
    the word is reduced, and in particular is not equal to the identity). Now,
    careful counting of the elements of the word reveals that the (unreduced)
    word has one more element from $H$ than it does elements from $G$. So, the
    total number of elements in the unreduced word is odd. Clearly, then, we
    cannot cancel in pairs to get the identity. 
    Therefore, this relation cannot be true, and thus $w$ cannot be written as a
    word beginning with an element from $H$.

    It follows immediately, then, that $wh\neq hw$ for $h\in H$ a nontrivial
    element, since $wh$ is a word beginning with an element of $G$, and $hw$ is
    a word beginning with an element of $H$.

    Thus, no element in $G\star H$ is in the center, as desired.
    \\
    \\
    Next, we show that every element of finite order is the conjugate of a
    finite order element of $G$ or $H$. So, let $w$ be a word in $G\star H$ with
    $w^n = e$. Without loss of generality, let the first term of $w$ be from
    $G$.

    If $w = g_1h_1\dots g_nh_n$, then the concatenation $w^n$ is already reduced
    (since adjacent elements are from different groups) and in particular cannot
    be equal to the identity.

    So, suppose $w = g_1h_1\dots h_{n-1}g_{n}$, with $w^n = e$. Now, for this to
    be true, it must be that $g_n=g_1^{-1}$. If this were not the case, the
    concatenated word $w^n$ would be reduced already (by treating $g_ng_1$ as an
    element of $G$) since $g_ng_1$ is adjacent to only elements of $H$.
    Continuing the argument, we find that $h_{n-1} = h_1^{-1}$,
    $g_{n-1}=g_2^{_1}$ and so on, save for the central term (since $w$ has odd
    length). If $n$ is even, the central term is $h_{\frac{n}{2}}$, and if $n$
    is odd, the central term is $g_{\frac{n+1}{2}}$. For the sake of simplicity,
    we will assume that $n$ is odd, but the proof works the same way for $n$
    even.

    Now, we know that
    \[
        w = g_1h_1g_2h_2\dots h_{\frac{n+1}{2}-1}
        g_{\frac{n+1}{2}}h_{\frac{n+1}{2}-1}^{-1}\dots h_2^{-1}g_2^{-1}h_1^{-1}g_1^{-1}
    \]
    which is just
    \[
        w= (g_1h_1g_2h_2\dots
        h_{\frac{n+1}{2}-1})g_{\frac{n+1}{2}}(g_1h_1g_2h_2\dots
        h_{\frac{n+1}{2}-1})^{-1}
    \]
    and thus $w$ is a conjugate of an element of $G$. Now for $w^n = e$ it must
    be that 
    \[
        w^n = 
        (g_1h_1g_2h_2\dots
        h_{\frac{n+1}{2}-1})g^n_{\frac{n+1}{2}}(g_1h_1g_2h_2\dots
        h_{\frac{n+1}{2}-1})^{-1} = e
    \]
    which is only true if $g^n = e$.

    Thus, the elements of finite order in $G\star H$ are all conjugates of
    elements of finite order in $G$ or $H$.

    Finally, we observe that every element that is the conjugate of an element
    of finite order in $G$ (or $H$) is of finite order. Let $g\in G$ with $g^n =
    e$, and observe that for any $w\in G\star H$, we have that
    \[
        (wgw^{-1})^n = wg^nw^{-1} = ww^{-1} = e
    \]
    and so $wgw^{-1}$ has finite order as well.

    Thus the elements of finite order in $G\star H$ are exactly the conjugates
    of elements of finite order in $G$ or $H$, as desired.
\end{proof}

\newpage

\section*{Problem 2}
Let $X\subset \R^n$ be the union of convex open sets $X_1,\dots,X_n$ such that
$X_i\cap X_j\cap X_k\neq\emptyset$ for all $i,j,k$. Prove that $X$ is simply
connected.
\\
\\
\begin{proof}
    Without loss of generality, let the basepoint $x_0$ be in $X_1$. Let $f$ be
    a loop in $X$ based at $x_0$. We wish to show that $f$ is nullhomotopic
    relative to the basepoint.


\end{proof}<++>

\end{document}
