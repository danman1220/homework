%%%%%%%%%%%%%%%%%%%%%%%%%%%%%%%%%%%%%%%%%
% Short Sectioned Assignment
% LaTeX Template
% Version 1.0 (5/5/12)
%
% This template has been downloaded from:
% http://www.LaTeXTemplates.com
%
% Original author:
% Frits Wenneker (http://www.howtotex.com)
%
% License:
% CC BY-NC-SA 3.0 (http://creativecommons.org/licenses/by-nc-sa/3.0/)
%
%%%%%%%%%%%%%%%%%%%%%%%%%%%%%%%%%%%%%%%%%

%----------------------------------------------------------------------------------------
%	PACKAGES AND OTHER DOCUMENT CONFIGURATIONS
%----------------------------------------------------------------------------------------

\documentclass[fontsize=11pt]{scrartcl} % 11pt font size

\usepackage[T1]{fontenc} % Use 8-bit encoding that has 256 glyphs
\usepackage[english]{babel} % English language/hyphenation
\usepackage{amsmath,amsfonts,amsthm} % Math packages
\usepackage{mathrsfs}
\usepackage{tikz-cd}

\usepackage[margin=1in]{geometry}

\usepackage{sectsty} % Allows customizing section commands
\allsectionsfont{\centering \normalfont\scshape} % Make all sections centered, the default font and small caps

\usepackage{fancyhdr} % Custom headers and footers
\pagestyle{fancyplain} % Makes all pages in the document conform to the custom headers and footers
\fancyhead{} % No page header - if you want one, create it in the same way as the footers below
\fancyfoot[L]{} % Empty left footer
\fancyfoot[C]{} % Empty center footer
\fancyfoot[R]{\thepage} % Page numbering for right footer
\renewcommand{\headrulewidth}{0pt} % Remove header underlines
\renewcommand{\footrulewidth}{0pt} % Remove footer underlines
\setlength{\headheight}{13.6pt} % Customize the height of the header

\numberwithin{equation}{section} % Number equations within sections (i.e. 1.1, 1.2, 2.1, 2.2 instead of 1, 2, 3, 4)
\numberwithin{figure}{section} % Number figures within sections (i.e. 1.1, 1.2, 2.1, 2.2 instead of 1, 2, 3, 4)
\numberwithin{table}{section} % Number tables within sections (i.e. 1.1, 1.2, 2.1, 2.2 instead of 1, 2, 3, 4)

\newcommand{\R}{\mathbb{R}}
\newcommand{\Q}{\mathbb{Q}}
\newcommand{\N}{\mathbb{N}}
\newcommand{\C}{\mathbb{C}}
\newcommand{\Z}{\mathbb{Z}}
\newcommand{\Aut}{\text{Aut}}
\newcommand{\im}{\text{im}}

\newtheorem{lemma}{Lemma}
%----------------------------------------------------------------------------------------
%	TITLE SECTION
%----------------------------------------------------------------------------------------

\newcommand{\horrule}[1]{\rule{\linewidth}{#1}} % Create horizontal rule command with 1 argument of height

\title{	
\normalfont \normalsize 
\textsc{Homotopy Theory} \\ [25pt] % Your university, school and/or department name(s)
\horrule{0.5pt} \\[0.4cm] % Thin top horizontal rule
\huge Midterm \\ % The assignment title
\horrule{2pt} \\[0.5cm] % Thick bottom horizontal rule
}

\author{Daniel Halmrast} % Your name

\date{\normalsize\today} % Today's date or a custom date

\begin{document}

\maketitle % Print the title

I pledge by Frank W. Warner's ``Foundations of Differentiable Manifolds and Lie
Groups'' this is my own work. Signed:
\\
\\
\\
\\

% Problems
\section*{Problem 1}
Show that for two nontrivial groups $G$ and $H$, their free product $G\star H$
has trivial center, and that the only elements of finite order are the
conjugates of finite order elements in $G$ and $H$.
\\
\\
\begin{proof}
    We first show that the free product has a trivial center. To show this, we
    will consider two cases of elements from $G\star H$. First, observe
    that for any element $g\in G$ with $g\neq e$, $g$ (or rather, the inclusion
    of $g$ into $G\star H$) cannot be in the center $Z(G\star H)$. To see this,
    let $h\in H$ with $h\neq e$, and note that the element $ghg^{-1}h^{-1}$ is a
    reduced word in the free group, and cannot be equal to the identity. Thus,
    $gh\neq hg$, and $g$ is not in $Z(G\star H)$.  Similarly, any element $h\in
    H$ is not in $Z(G\star H)$.

    Now, let $w$ be a reduced word in $G\star H$ that is not in $G$ or $H$.
    Without loss of generality, we can write
    \[
        w = g_1h_1g_2h_2\dots g_nh_n
    \]
    or
    \[
        w = g_1h_1g_2h_2\dots g_n
    \]
    for $g_i\in G$, and $h_i\in H$. Now, since $G$ and $H$ have no relations to
    each other, this word is not equal to any word beginning with an element of
    $H$. Suppose for a contradiction that
    \[
        w = h'_1g'_1h'_2g'_2\dots h'_mg'_m
    \]
    with $g'_m$ possibly equal to the identity.
    This would imply that
    \[
        g_1h_1g_2h_2\dots g_nh_n = h'_1g'_1h'_2g'_2\dots h'_mg'_m
    \]
    which would imply the relation
    \[
        g_1h_1g_2h_2\dots g_nh_n(g'_m)^{-1}(h'_m)^{-1}\dots
        (g'_1)^{-1}(h'_1)^{-1} = e
    \]
    Now, if $g'_m\neq e$, the word on the left hand side is already reduced, and
    thus cannot be equal to $e$. If $g'_m=e$, then the word could reduce
    further. However, note that this word only reduces further if adjacent
    elements from the same group cancel to the identity (that is, if adjacent
        elements from the same group do not cancel to the identity, then
    the word is reduced, and in particular is not equal to the identity). Now,
    careful counting of the elements of the word reveals that the (unreduced)
    word has one more element from $H$ than it does elements from $G$. So, the
    total number of elements in the unreduced word is odd. Clearly, then, we
    cannot cancel in pairs to get the identity. 
    Therefore, this relation cannot be true, and thus $w$ cannot be written as a
    word beginning with an element from $H$.

    It follows immediately, then, that $wh\neq hw$ for $h\in H$ a nontrivial
    element, since $wh$ is a word beginning with an element of $G$, and $hw$ is
    a word beginning with an element of $H$.

    Thus, no element in $G\star H$ is in the center, as desired.
    \\
    \\
    Next, we show that every element of finite order is the conjugate of a
    finite order element of $G$ or $H$. So, let $w$ be a word in $G\star H$ with
    $w^n = e$. Without loss of generality, let the first term of $w$ be from
    $G$.

    If $w = g_1h_1\dots g_nh_n$, then the concatenation $w^n$ is already reduced
    (since adjacent elements are from different groups) and in particular cannot
    be equal to the identity.

    So, suppose $w = g_1h_1\dots h_{n-1}g_{n}$, with $w^n = e$. Now, for this to
    be true, it must be that $g_n=g_1^{-1}$. If this were not the case, the
    concatenated word $w^n$ would be reduced already (by treating $g_ng_1$ as an
    element of $G$) since $g_ng_1$ is adjacent to only elements of $H$.
    Continuing the argument, we find that $h_{n-1} = h_1^{-1}$,
    $g_{n-1}=g_2^{-1}$ and so on, save for the central term (since $w$ has odd
    length). If $n$ is even, the central term is $h_{\frac{n}{2}}$, and if $n$
    is odd, the central term is $g_{\frac{n+1}{2}}$. For the sake of simplicity,
    we will assume that $n$ is odd, but the proof works the same way for $n$
    even.

    Now, we know that
    \[
        w = g_1h_1g_2h_2\dots h_{\frac{n+1}{2}-1}
        g_{\frac{n+1}{2}}h_{\frac{n+1}{2}-1}^{-1}\dots h_2^{-1}g_2^{-1}h_1^{-1}g_1^{-1}
    \]
    which is just
    \[
        w= (g_1h_1g_2h_2\dots
        h_{\frac{n+1}{2}-1})g_{\frac{n+1}{2}}(g_1h_1g_2h_2\dots
        h_{\frac{n+1}{2}-1})^{-1}
    \]
    and thus $w$ is a conjugate of an element of $G$. Now for $w^n = e$ it must
    be that 
    \[
        w^n = 
        (g_1h_1g_2h_2\dots
        h_{\frac{n+1}{2}-1})g^n_{\frac{n+1}{2}}(g_1h_1g_2h_2\dots
        h_{\frac{n+1}{2}-1})^{-1} = e
    \]
    which is only true if $g^n = e$.

    Thus, the elements of finite order in $G\star H$ are all conjugates of
    elements of finite order in $G$ or $H$.

    Finally, we observe that every element that is the conjugate of an element
    of finite order in $G$ (or $H$) is of finite order. Let $g\in G$ with $g^n =
    e$, and observe that for any $w\in G\star H$, we have that
    \[
        (wgw^{-1})^n = wg^nw^{-1} = ww^{-1} = e
    \]
    and so $wgw^{-1}$ has finite order as well.

    Thus the elements of finite order in $G\star H$ are exactly the conjugates
    of elements of finite order in $G$ or $H$, as desired.
\end{proof}

\newpage
\section*{Problem 2}
Let $X\subset \R^n$ be the union of convex open sets $X_1,\dots,X_n$ such that
$X_i\cap X_j\cap X_k\neq\emptyset$ for all $i,j,k$. Prove that $X$ is simply
connected.
\\
\\
\begin{proof}
    This proof will be done inductively on the number of convex open sets used
    to union to $X$. Now, convex sets have a distinct property which makes them
    nice for this problem. Namely, if $X$ is a convex set, and $\gamma$ is a
    path in $X$, $\gamma$ is homotopic to the line segment from $\gamma(0)$ to
    $\gamma(1)$. This is clear; since convex sets are simply connected, all
    paths with fixed endpoints are homotopic to each other, and since each
    convex set is convex, it contains the straight line path from $\gamma(0)$ to
    $\gamma(1)$.

    Now, let's first prove the base case of $n=2$. For this, let $f$ be a loop
    in $X$ based at $x_0$. Without loss of generality, we let $x_0\in X_1$. Now,
    if $f$ stays in $X_1$, then it is nullhomotopic since $X_1$ is simply
    connected. Now, suppose $f$ enters $X_2$ at some time $t_1$, exits $X_1$ at
    some $t>t_1$, and exits $X_2$ at $t_2>t$. Then, the segment $f|_{[t_1,t_2]}$
    is homotopic to the straight line from $f(t_1)$ to $f(t_2)$, which is in
    $X_1$. Repeating this for the (finite) number of times $f$ exits $X_1$, we
    see that $f$ is homotopic to a loop that stays in $X_1$, and is thus
    nullhomotopic.

    Now suppose the theorem holds for the union of $n-1$ convex sets with the
    triple intersection property, and let $X$ be the union of $n$ convex sets
    with the same property. Let $f$ be a loop in $X$ based at $x_0\in X_1$. Now,
    if $f$ stays in $\cup_{i=1}^{n-1}X_i$ (that is, if $f$ avoids being in only
    $X_n$), then $f$ is a loop in the union of $n-1$ convex sets with the triple
    intersection property, and is nullhomotopic by the inductive hypothesis.
    
    So, suppose $f$ enters $X_n$ at
    some time $t_1$ from a set $X_i$, leaves $X_i$ at some $t>t_1$, then leaves
    $X_n$ in $X_j\cap X_n$ at some $t_2$. Now, by the triple intersection
    property, we know that $X_i\cap X_j\cap X_n \neq \emptyset$, so let $x$ be a
    point in the common intersection. Now, since $f(t_1)$, $f(t_2)$, and $x$ are
    all in $X_n$, so is the path taking the straight line from $f(t_1)$ to $x$,
    then the straight line from $x$ to $f(t_2)$. Then, the segment
    $f|_{[t_1,t_2]}$ is homotopic to this path from $f(t_1)$ through $x$ to
    $f(t_2)$. However, since $f(t_1)$ and $x$ are in $X_i$, so is the straight
    line connecting them. Furthermore, since $x$ and $f(t_2)$ are in $X_j$, so
    is the straight line connecting them. Thus, $f|_{[t_1,t_2]}$ is homotopic to
    a path that stays in $X_i$ and $X_j$. Repeating this process for the
    (finite) number of times $f$ stays in $X_n$ alone, we see that $f$ is
    homotopic to a loop that stays in $\cup_{i=1}^{n-1}X_i$, and by the
    inductive hypothesis, $f$ is nullhomotopic as desired.

\end{proof}

\newpage
\section*{Problem 3}
Show that the complement of a finite set of points in $\R^n$ is simply connected
if $n\geq 3$.
\\
\\
\begin{proof}
    Let $X$ be a set of $m$ points in $\R^n$. We wish to show that
    $\pi_1(\R^n\setminus X)$ is trivial. Now, since $X$ is a discrete set of
    points, there is some $\varepsilon>0$ for which the balls
    $B_{\varepsilon}(p)$ of radius $\varepsilon$ centered at $p\in X$ do not
    intersect. We can deformation retract $\R^n\setminus X$ to $\R^n\setminus
    (\bigcup_{p\in X}B_{\varepsilon}(p))$ by fixing all $x\not\in
    B_{\varepsilon}(p)$ and sending $x\in B_{\varepsilon}(p)$ along the radial
    line to the boundary. This is well-defined, since the central point $p$ is
    not in the domain of this homotopy.

    Now, we form a new space $Y$ as follows: for each $B_{\varepsilon}(p)$,
    attach an $n$-cell along the boundary $\partial B_{\varepsilon}(p)$ (which
    is in $\R^n\setminus(\bigcup_{p\in X}B_{\varepsilon}(p))$). Clearly, $Y =
    \R^n$, which has trivial fundamental group. However, proposition 1.26
    guarantees that there is an isomorphism between $\pi_1(\R^n\setminus
    (\bigcup_{p\in x}B_{\varepsilon}(p)))$ and $\pi_1(Y)$. Thus, it follows that
    $\pi_1(\R^n\setminus(\bigcup_{p\in x}B_{\varepsilon}(p)))$ is trivial as
    well, and thus since $\R^n\setminus X$ deformation retracts to this space,
    it has trivial fundamental group as well, as desired.
\end{proof}

\newpage
\section*{Problem 4}
Let $X\subset \R^3$ be the union of $n$ lines through the origin. Compute
$\pi_1(\R^3\setminus X)$.
\\
\\
\begin{proof}
    To begin with, we deformation retract $\R^3\setminus X$ onto
    $S^2\setminus\tilde{X}$, where $\tilde{X}$ is the collection of $2n$ points
    $S^2\cap X$. This deformation retraction is the standard radial retraction
    \[
        f_t(x) = (1-t)x + t\frac{x}{\|x\|}
    \]
    which sends the lines in $X$ to their intersection points in
    $\tilde{X}$. This is well-defined, since $0\in X$ and so $0\not\in
    \R^3\setminus X$.

    Now, we can designate one of these points as the north pole $N$, and use the
    homeomorphism from $\R^2$ to $S^2\setminus \{N\}$ to construct a
    homeomorphism between $S^2\setminus \tilde{X}$ and $\R^2\setminus Y$ where
    $Y$ is the collection of $2n-1$ points that is the image of
    $\tilde{X}\setminus \{N\}$ under the homeomorphism.

    Finally, we draw a finite graph on $\R^2$ with the property that there are
    exactly $2n-1$ bounded complementary components, each containing exactly one
    point of $Y$.
    Then, since each bounded complementary component is homeomorphic to a disk
    with a point removed, we can deformation retract each one onto its boundary.
    Finally, we deformation retract the unbounded complementary component to the
    outside boundary of the graph. Now, we can apply problem 5 to see that the
    fundamental group of this space is the free group on $2n-1$ generators.

    Since $\R^3\setminus X$ can be homotoped to this graph via the homotopies
    described, we have that the fundamental group of $\R^3\setminus X$ is the
    free group on $2n-1$ generators.

\end{proof}

\newpage
\section*{Problem 5}
Let $X\subset \R^2$ be a connected graph that is the union of a finite number of
straight line segments. Show that $\pi_1(X)$ is free with a basis consisting of
loops formed by the boundaries of the bounded complementary regions of $X$,
joined to a basepoint by suitably chosen paths in $X$.
\\
\\
\begin{proof}
    Consider the following construction. For each bounded component of
    $\R^2\setminus X$, attach a $2$-cell $e^2_i$ along the boundary of the
    component (which is in $X$). Call the space obtained this way $Y$. Now, $Y$ is
    obtained from $X$ in the way necessary to apply proposition 1.26. Namely, we
    know that there is a surjection $\pi_1(X,x_0)\to \pi_1(Y,x_0)$ whose kernel
    is the normal subgroup generated by the (conjugates of the) attaching maps.

    Now, $Y$ can easily be seen to have trivial fundamental group, since if it
    did have a nontrivial loop, that loop would encircle some amount of bounded
    components of $\R^2\setminus X$. But we already attached $2$-cells across
    each of these components, and so the loop can homotope across the bounded
    components, and must actually be nullhomotopic.
    Thus, it must be that the normal subgroup $N$ which is the kernel of
    the induced surjection is the whole group $\pi_1(X,x_0)$. So it suffices to
    describe the subgroup $N$.

    Now, per proposition 1.26, $N$ is the normal subgroup generated by the
    loops around the bounded components of $\R^2\setminus X$, joined to the
    basepoint by some path in $X$. Furthermore, these loops have no relations to
    each other. Consider the homotopy from $X$ to the wedge of $n$ circles,
    where $n$ is the number of bounded components of $\R^2\setminus X$ (Hatcher
    example 0.7 guarantees the existence of such a homotopy). The image of the
    loop going once around the $i$th bounded component is a loop going once
    around the $i$th wedge of circles. Thus, this homotopy sends the generators
    of $N$ to the generators of $\pi_1(\vee_nS^1)$, which is the free group on
    $n$ generators. Since the homotopy induces an isomorphism, it must be that
    $N$ is free on these generators as well.

    Thus, the fundamental group of $X$ is the free group on $n$ generators,
    where the generators are the loops going once around a bounded complementary
    region of $X$ joined to the basepoint by a suitably chosen path, as desired.
\end{proof}

\newpage
\section*{Problem 6}
Use Proposition 1.26 to show that the complement of a closed discrete subspace
of $\R^n$ is simply connected if $n\geq 3$.
\\
\\
\begin{proof}
    Let $D$ denote the discrete subspace of $\R^n$. Now,
    since $D$ is discrete and closed, it follows that for each point $p$ in $D$,
    there is a ball $B_{\varepsilon_p}(p)$ with radius
    $\varepsilon_p$ such that $B_{\varepsilon_p}(p)\cap D = p$ (if this were
    not the case, then $p$ would be a limit point of $D$, which cannot
    happen since $D$ is discrete).

    We now follow the same strategy as the one used in problem 3. Namely, we
    first note that $\R^n\setminus D$ deformation retracts onto $\R^n\setminus
    (\bigcup_{p\in D}B_{\varepsilon_p}(p))$ through the radial homotopy in each
    $B_{\varepsilon_p}(p)$ to the boundary.

    Now, we construct the space $Y$ by attaching an $n$-cell for each
    $B_{\varepsilon_p}(p)$ along its boundary. Clearly, $Y=\R^n$, which has
    trivial fundamental group. However, proposition 1.26 guarantees that $Y$
    and $\R^n\setminus(\bigcup_{p\in D}B_{\varepsilon_p}(p))$ have the same
    fundamental group (the trivial group). Thus, since $R^n\setminus D$ is
    homotopic to $\R^n\setminus(\bigcup_{p\in D}B_{\varepsilon_p}(p))$, they
    have the same (trivial) fundamental group, and so $\R^n\setminus D$ is
    simply connected, as desired.
\end{proof}

\newpage
\section*{Problem 7}
Compute the fundamental group of the space obtained from two tori $S^1\times
S^1$ by identifying a circle $S^1\times \{x_0\}$ in one torus with the
corresponding circle in the other torus.
\\
\\
\begin{proof}
    Let $T_1$ and $T_2$ denote the tori being joined, and let $X$ be the
    resulting space.
    Furthermore, let $U_i$ be a neighborhood of the circle $S^1\times \{x_0\}$
    in $T_i$ which deformation retracts onto the circle (for example, $U_i =
    S^1\times (x_0-\varepsilon,x_0+\varepsilon)$).

    Now, $X$ is the union of $A_1 = T_1\cup U_2$ and $A_2 = T_2\cup U_1$, and the
    intersection $(T_1\cup U_2) \cap (T_2\cup U_1)$ is just $U_1\cup U_2$, which
    is path-connected. Letting the basepoint be $x_0$, which is in each $U_i$,
    we note that these two open sets satisfy the hypotheses for van Kampen's
    theorem. Thus, the fundamental group $\pi_1(X)$ is the pushout of
    \[
        \begin{tikzcd}
            \pi_1(A_1\cap A_2)\arrow{d}{i_1}\arrow{r}{i_2} &\pi_1(A_2)\\
            \pi_1(A_1)
        \end{tikzcd}
    \]
    which is just the free product of $\pi_1(A_1)$ with $\pi_1(A_2)$ modded out
    by the normal subgroup generated by the relation $i_1([f]) = i_2([f])$ for
    $[f]\in \pi_1(A_1\cap A_2)$.

    In particular, $A_1$ deformation retracts to $T_1$ by construction of
    $U_2$, and similarly $A_2$ deformation retracts to $T_2$. Thus,
    \[
        \pi_1(A_1) = \pi_1(A_2) = \Z\oplus\Z
    \]
    In particular, $A_1$ is the free abelian group
    generated by the loops $f_1$ going once around $\{x_0\}\times S^1$ and $g_1$
    going once around $S^1\times \{x_0\}$ in
    $T_1$, and $\pi_1(A_2)$ is the free abelian group generated by the loops $f_2$
    going once around $\{x_0\}\times S^1$ and $g_2$ going once around $S^1\times
    \{x_0\}$ in $T_2$. 

    Now, $\pi_1(A_1\cap A_2) = \Z$, since $A_1\cap A_2$ deformation retracts to
    the common circle $S^1\times \{x_0\}$. This group is generated by the loop
    $g$ going once around $S^1\times \{x_0\}$. Furthermore, $i_1([g]) = [g_1]$
    and $i_2([g]) = [g_2]$. Thus, the normal subgroup we need to quotient by is
    the one generated by the relation $[g_1] = [g_2]$.

    So, the fundamental group of $X$ is
    \[
        \pi_1(X) = (FrAb(f_1,g_1)\star FrAb(f_2,g_2))/{(g_1=g_2)}
    \]
    Where $FrAb(f_i,g_i)$ denotes the free abelian group on two generators
    $f_i,g_i$.
\end{proof}

\end{document}
