%%%%%%%%%%%%%%%%%%%%%%%%%%%%%%%%%%%%%%%%%
% Short Sectioned Assignment
% LaTeX Template
% Version 1.0 (5/5/12)
%
% This template has been downloaded from:
% http://www.LaTeXTemplates.com
%
% Original author:
% Frits Wenneker (http://www.howtotex.com)
%
% License:
% CC BY-NC-SA 3.0 (http://creativecommons.org/licenses/by-nc-sa/3.0/)
%
%%%%%%%%%%%%%%%%%%%%%%%%%%%%%%%%%%%%%%%%%

%----------------------------------------------------------------------------------------
%	PACKAGES AND OTHER DOCUMENT CONFIGURATIONS
%----------------------------------------------------------------------------------------

\documentclass[fontsize=11pt]{scrartcl} % 11pt font size

\usepackage[T1]{fontenc} % Use 8-bit encoding that has 256 glyphs
\usepackage[english]{babel} % English language/hyphenation
\usepackage{amsmath,amsfonts,amsthm} % Math packages
\usepackage{mathrsfs}

\usepackage[margin=1in]{geometry}

\usepackage{sectsty} % Allows customizing section commands
\allsectionsfont{\centering \normalfont\scshape} % Make all sections centered, the default font and small caps

\usepackage{fancyhdr} % Custom headers and footers
\pagestyle{fancyplain} % Makes all pages in the document conform to the custom headers and footers
\fancyhead{} % No page header - if you want one, create it in the same way as the footers below
\fancyfoot[L]{} % Empty left footer
\fancyfoot[C]{} % Empty center footer
\fancyfoot[R]{\thepage} % Page numbering for right footer
\renewcommand{\headrulewidth}{0pt} % Remove header underlines
\renewcommand{\footrulewidth}{0pt} % Remove footer underlines
\setlength{\headheight}{13.6pt} % Customize the height of the header

\numberwithin{equation}{section} % Number equations within sections (i.e. 1.1, 1.2, 2.1, 2.2 instead of 1, 2, 3, 4)
\numberwithin{figure}{section} % Number figures within sections (i.e. 1.1, 1.2, 2.1, 2.2 instead of 1, 2, 3, 4)
\numberwithin{table}{section} % Number tables within sections (i.e. 1.1, 1.2, 2.1, 2.2 instead of 1, 2, 3, 4)

\newcommand{\R}{\mathbb{R}}
\newcommand{\Q}{\mathbb{Q}}
\newcommand{\N}{\mathbb{N}}
\newcommand{\C}{\mathbb{C}}

\newcommand{\im}{\textrm{im}}

\newtheorem{lemma}{Lemma}
%----------------------------------------------------------------------------------------
%	TITLE SECTION
%----------------------------------------------------------------------------------------

\newcommand{\horrule}[1]{\rule{\linewidth}{#1}} % Create horizontal rule command with 1 argument of height

\title{	
\normalfont \normalsize 
\textsc{Analysis} \\ [25pt] % Your university, school and/or department name(s)
\horrule{0.5pt} \\[0.4cm] % Thin top horizontal rule
\huge Problem Set 3 \\ % The assignment title
\horrule{2pt} \\[0.5cm] % Thick bottom horizontal rule
}

\author{Daniel Halmrast} % Your name

\date{\normalsize\today} % Today's date or a custom date

\begin{document}

\maketitle % Print the title

% Problems

\section*{Problem 1}
Show that for $A\in \mathscr{B}(X,Y)$, $\|A\| = \|A^*\|$. Furthermore, if $A$ is
invertible, show that $A^*$ is invertible with inverse $(A^*)^{-1} =
(A^{-1})^*$.
\\
\\
\begin{proof}
    We know that
    \[
        \begin{aligned}
            \|A^*\| &= \sup_{\phi\in Y^*,\|\phi\|=1}\|A^*\phi\|\\
            &= \sup_{\phi\in Y^*,\|\phi\|=1}\sup_{x\in X,\|x\|=1}\|A^*\phi(x)\|
        \end{aligned}
    \]
    Now, by the definition of the norm, we have that
    \[
        \phi(Ax)\leq \|\phi\|\|Ax\|
    \]
    and so
    \[
        \begin{aligned}
            \|A^*\| &= \sup_{\phi\in Y^*,\|\phi\|=1}\sup_{x\in X,\|x\|=1}\|A^*\phi(x)\|
                    &\leq\sup_{\phi\in Y^*,\|\phi\|=1}\sup_{x\in
                    X,\|x\|=1}\|\phi\|\|Ax\|\\
                    &= \sup_{x\in X,\|x\|=1} \|Ax\|\\
                    &= \|A\|
        \end{aligned}
    \]
    and so $\|A^*\|\leq \|A\|$ as desired.

    For the other inequality, we note that $\|A\| = \sup_{x\in
    X,\|x\|=1}\|Ax\|$, and since $\|Ax\| = \sup_{\phi\in
    Y^*,\|\phi\|=1}|\phi(Ax)|$ (proved in HW2 by explicit construction of
    $\phi$ that attains the norm), we have that
    \[
        \|A\| = \sup_{x\in X,\|x\|=1}\sup_{\phi\in Y^*,\|\phi\|=1} |\phi(Ax)|
    \]
    The definition of the norm $\|x\| = \|x\|_{X^{**}}$ shows that
    \[
        x(A^*\phi) \leq \|x\|\|A^*\phi\|
    \]
    Now, by a similar argument to the last inequality, we have
    \[
        \begin{aligned}
            \|A\| &= \sup_{x\in X,\|x\|=1}\sup_{\phi\in Y^*,\|\phi\|=1}
            |\phi(Ax)|\\
            &= \sup_{x\in X,\|x\|=1}\sup_{\phi\in Y^*,\|\phi\|=1}
            |x(A^*\phi)|\\
            &\leq \sup_{x\in X,\|x\|=1}\sup_{\phi\in Y^*,\|\phi\|=1}
            \|x\|\|A^*\phi\|\\
            &=\sup_{\phi\in Y^*,\|\phi\|=1}\|A^*\phi\|\\
            &=\|A^*\|
        \end{aligned}
    \]
    and so $\|A\|\leq \|A^*\|$. Combining both inequalities, we have that
    \[
        \|A\|=\|A^*\|
    \]
    as desired.
    \\
    \\
    if $A$ is invertible, it is easily shown that $(A^{-1})^*$ inverts $A^*$.
    That is, we wish to show that for all $\phi\in Y^*$, we have that
    \[
        (A^{-1})^*A^*\phi = \phi
    \]
    In particular, we wish to show that for all $y\in Y$,
    \[
        (A^{-1})^*A^*\phi(y) = \phi(y)
    \]
    This is clear, however, since
    \[
        \begin{aligned}
            (A^{-1})^*A^*\phi(y) &= A^*\phi(A^{-1}y)\\
                &= \phi(AA^{-1}y)\\
                &=\phi(y)
        \end{aligned}
    \]
    as desired.
\end{proof}

\newpage

\section*{Problem 2}
Prove the Fredholm Theorem.
\\
\\
\begin{proof}
    Recall from the previous homework that for a subspace $V$ of a Banach space
    $X$, we have that
    \[
        \overline{V} = \cap_{\phi \text{s.t. } V\subset\ker\phi}\ker\phi
    \]

    Letting $V=\im A$, we see that
    \[
        \overline{\im A} = \cap_{\phi \text{s.t. } \im A\subset\ker\phi}\ker\phi
    \]
    Now, the right hand side is just the set of all $y\in Y$ for which
    $\phi(y)=0$ for any $\phi$ such that $\phi(Ax) = 0$ for all $x$. That is,
    $\phi$ is such that $A^*\phi(x) = 0$ for all $x$, so $A^*\phi = 0$. That is,
    the right hand side is the set of all $y\in Y$ for which $\phi(y) = 0$ for
    all $\phi$ in $\ker A^*$, as desired.
\end{proof}

\newpage

\section*{Problem 3}
Explain the difference between the weak-$*$ convergence of a sequence $(\phi_j)$ in
$X^*$ and the weak convergence of $(\phi_j)$ in $Y=X^*$. State the relations
between the strong, weak, and weak-$*$ convergences on $X^*$.
\\
\\
\begin{proof}
    If $(\phi_j)$ converges in weak-$*$ to $\phi$, this means that for all $x\in
    X$, $\phi_j(x) \to \phi(x)$. That is, $\phi_j$ converges pointwise to
    $\phi$. Specifically, the weak-$*$ topology is the weak topology with
    respect to $i_{can}X\subset X^{**}$.

    On the other hand, if $(\phi_j)$ converges to $\phi$ weakly, then for all
    $x\in X^{**}$, $x(\phi_j)\to x(\phi)$. In particular, the weak topology is
    the weak topology with respect to $X^{**}$. That is to say, the weak
    topology utilizes the entirety of $X^{**}$ to detect convergence, while the
    weak-$*$ convergence only uses $i_{can}(X)\subset X^{**}$.

    It should be clear, however, that strong convergence implies weak
    convergence, which implies weak-$*$ convergence. To see this, let $\phi_j$
    be such that $\|\phi_j-\phi\|\to 0$. Then, for any $x\in X^{**}$, since $x$
    is continuous with respect to the norm on $X^*$, we have that
    \[
        x(\phi_j)\to x(0)
    \]
    which is the condition for weak convergence. The fact that weak convergence
    implies weak-$*$ convergence is clear, since $i_{can}(X)\subset X^{**}$, and
    so if for all $x\in X^{**}$, $x(\phi_j)\to x(\phi)$, then clearly for all
    $x\in X$, $\phi_j(x)\to \phi(x)$.
\end{proof}

\newpage

\section*{Problem 4}
Prove that the sequence of standard basis vectors $e_n\in \ell^p$, $1<p<\infty$
converges weakly but not strongly.
\\
\\
\begin{proof}
    We first show that $(e_n)$ does not converge strongly. This is clear, since
    the sequence is not Cauchy, that is
    \[
        \begin{aligned}
            \|e_n-e_m\|^p_p &= \sum_{i=1}^{\infty}(|(e_n-e_m)_i|)^p\\
                            &= 1^p+1^p\neq 0
        \end{aligned}
    \]
    Thus, it does not converge in norm.

    Now, we show that $(e_n)$ converges weakly. To see this, let $x\in \ell^{p*}
    = \ell^q$. Then, we need to show that $|x(e_n)|\to 0$. This is clear,
    however, since
    \[
        |x(e_n)| = \sum_{i=1}^{\infty}|x_i(e_n)_i| = |x_n|
    \]
    and since $x\in\ell^q$, $|x_i|\to 0$ (since $\sum_{i=1}^{\infty}|x_i|^q <
    \infty$), it follows that
    \[
        |x(e_n)|\to 0
    \]
    as desired.
\end{proof}

\newpage

\section*{Problem 5}
Prove that $\frac{\epsilon}{\pi(x^2+\epsilon^2)}d\lambda^1(x)\to\delta_0$ in
weak-$*$ as
measures in $C([-1,1])^*$.
Prove that
$\frac{1}{2\epsilon}\chi_{-\epsilon,\epsilon}(x)d\lambda^1(x)\to\delta_0$ in
weak-$*$ as measures in $C([-1,1])^*$.
\\
\\
\begin{proof}
    First, we show that for all $f\in C([-1,1])$,
    $\int_{[-1,1]}f(x)\frac{\epsilon}{\pi(x^2+\epsilon^2)}d\lambda^1(x)$ goes to
    $f(0)$.

    To see this, we can estimate the integral away from zero as well as at zero
    by
    \[
        \begin{aligned}
            \int_{[-1,1]}f(x)\frac{\epsilon}{\pi(x^2+\epsilon^2)}d\lambda^1(x)
            &= \int_{[-1,-\delta]\cup[\delta,1]}
            f(x)\frac{\epsilon}{\pi(x^2+\epsilon^2)}dx + \int_{-\delta}^{\delta}
            f(x)\frac{\epsilon}{\pi(x^2+\epsilon^2)}dx\\
        \end{aligned}
    \]
    Letting $M$ be the bound on $|f(x)|$ (since $f$ is on a compact set, this is
    defined), we can bound the first integral above and below by
    \[
        \begin{aligned}
            \int_{[-1,-\delta]\cup[\delta,1]}f(x)\frac{\epsilon}{\pi(x^2+\epsilon^2)}dx
            &\leq M\frac{\epsilon}{\pi\delta^2}2(1-\delta)\\
            \int_{[-1,-\delta]\cup[\delta,1]}f(x)\frac{\epsilon}{\pi(x^2+\epsilon^2)}dx
            &\geq -M\frac{\epsilon}{\pi\delta^2}2(1-\delta)
        \end{aligned}
    \]
    which goes to zero as delta gets small, and can safely be ignored so long as
    $\delta$ shrinks slower than $\epsilon$. 

    For the second integral, we use the fact that $|f(x)-f(0)| < \epsilon'$ for
    $|x|<\delta'$ to estimate
    \[
        \begin{aligned}
        \int_{-\delta}^{\delta}f(x)\frac{\epsilon}{\pi(x^2+\epsilon^2)}dx
        &\leq
        \int_{-\delta}^{\delta}(f(0)+\epsilon')\frac{\epsilon}{\pi(x^2+\epsilon^2)}dx\\
        &= 2(f(0)+\epsilon')\frac{1}{\pi}\tan^{-1}(\frac{\delta}{\epsilon})
        \int_{-\delta}^{\delta}f(x)\frac{\epsilon}{\pi(x^2+\epsilon^2)}dx
        &\geq
        \int_{-\delta}^{\delta}(f(0)-\epsilon')\frac{\epsilon}{\pi(x^2+\epsilon^2)}dx\\
        &= 2(f(0)-\epsilon')\frac{1}{\pi}\tan^{-1}(\frac{\delta}{\epsilon})
        \end{aligned}
    \]
    which are both equal to $(f(0)\pm\epsilon')$ so long as
    $\frac{\delta}{\epsilon}$ goes to infinity, or $\delta$ shrinks slower than
    $\epsilon$.

    So, fixing a sequence of $\delta$ for which the integral goes to
    $f(0)\pm\epsilon'\to 0$, and letting $\epsilon$ go to zero faster than
    $\delta$, we see that the measure converges to the delta measure centered at
    zero, as desired.
    
    Thus, since this holds for any $f\in C([-1,1])$, it follows (by Riesz
    representation for $C([-1,1])^*$ into Borel measures on $[-1,1]$) that the
    measure given converges in weak-$*$ to the delta measure.
    \\
    \\
    For the second statement, we wish to show that for all $f\in C([-1,1])$,
    \[
        \int_{[-1,1]}f(x)\frac{1}{2\epsilon}\chi_{[-\epsilon,\epsilon]}dx = f(0)
    \]
    This is easily done.
    \[
        \begin{aligned}
        \int_{[-1,1]}f(x)\frac{1}{2\epsilon}\chi_{[-\epsilon,\epsilon]}dx
            &= \frac{1}{2\epsilon}\int_{-\epsilon}^{\epsilon}f(x)dx
        \end{aligned}
    \]
    Now, we use continuity of $f$ to get a sequence of $\delta_n$ for which
    $|x|<\delta_n$ implies $|f(x)-f(0)|<\frac{1}{n}$. Then, it follows that (by
    setting $\epsilon = \delta_n$)
    \[
        \begin{aligned}
        \int_{[-1,1]}f(x)\frac{1}{2\delta_n}\chi_{[-\delta_n,\delta_n]}dx
            &= \frac{1}{2\delta_n}\int_{-\delta_n}^{\delta_n}f(x)dx
            &\leq
            \frac{1}{2\delta_n}\int_{-\delta_n}^{\delta_n}(f(0)+\frac{1}{n})dx\\
            &=f(0)+\frac{1}{n}
        \end{aligned}
    \]
    and similarly,
    \[
        \begin{aligned}
        \int_{[-1,1]}f(x)\frac{1}{2\delta_n}\chi_{[-\delta_n,\delta_n]}dx
            &\geq
            \frac{1}{2\delta_n}\int_{-\delta_n}^{\delta_n}(f(0)-\frac{1}{n})dx\\
            &=f(0)-\frac{1}{n}
        \end{aligned}
    \]
    which goes to $f(0)$ as $n$ goes to infinity.
\end{proof}

\newpage

\section*{Problem 6}
Prove that a finite dimensional vector space is reflexive. Find an expression
for the matrix form of $A^*$ given the matrix form of $A$.
\\
\\
\begin{proof}
    To show that $V$ is reflexive, we only need to show that the canonical
    injection $i$ is surjective. However, since $V$ and $V^*$ have the same
    dimension, so does $V^*$ and $V^{**}$, and so since $i$ is injective into a
    space of the same dimension, $i$ is surjective as well, as desired.

    Now, we wish to find $A^*$ as a matrix in the dual basis. We wish to show
    \[
        \langle Ax,y\rangle = \langle x,A^{\dagger}y\rangle
    \]
    for all $x\in V$ and $y\in W$ for $A:V\to W$ (anticipating that $A^* =
    A^{\dagger}$.)

    To do so, let's express $A$ in local coordinates.
    \[
        \begin{aligned}
            \langle Ax,y\rangle &= A_{ij}x^j\overline{y^i}\\
            &= \overline{\overline{A_{ij}}y^i}x^j\\
            &= x^j\overline{\overline{A_{ji}^T}y^i}\\
            &= \langle x, \overline{A^T}y\rangle
        \end{aligned}
    \]
    and so $A^* = A^{\dagger}$, as desired.
\end{proof}

\end{document}
