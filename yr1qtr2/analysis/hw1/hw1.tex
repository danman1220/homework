%%%%%%%%%%%%%%%%%%%%%%%%%%%%%%%%%%%%%%%%%
% Short Sectioned Assignment
% LaTeX Template
% Version 1.0 (5/5/12)
%
% This template has been downloaded from:
% http://www.LaTeXTemplates.com
%
% Original author:
% Frits Wenneker (http://www.howtotex.com)
%
% License:
% CC BY-NC-SA 3.0 (http://creativecommons.org/licenses/by-nc-sa/3.0/)
%
%%%%%%%%%%%%%%%%%%%%%%%%%%%%%%%%%%%%%%%%%

%----------------------------------------------------------------------------------------
%	PACKAGES AND OTHER DOCUMENT CONFIGURATIONS
%----------------------------------------------------------------------------------------

\documentclass[fontsize=11pt]{scrartcl} % 11pt font size

\usepackage[T1]{fontenc} % Use 8-bit encoding that has 256 glyphs
\usepackage[english]{babel} % English language/hyphenation
\usepackage{amsmath,amsfonts,amsthm} % Math packages
\usepackage{mathrsfs}

\usepackage[margin=1in]{geometry}

\usepackage{sectsty} % Allows customizing section commands
\allsectionsfont{\centering \normalfont\scshape} % Make all sections centered, the default font and small caps

\usepackage{fancyhdr} % Custom headers and footers
\pagestyle{fancyplain} % Makes all pages in the document conform to the custom headers and footers
\fancyhead{} % No page header - if you want one, create it in the same way as the footers below
\fancyfoot[L]{} % Empty left footer
\fancyfoot[C]{} % Empty center footer
\fancyfoot[R]{\thepage} % Page numbering for right footer
\renewcommand{\headrulewidth}{0pt} % Remove header underlines
\renewcommand{\footrulewidth}{0pt} % Remove footer underlines
\setlength{\headheight}{13.6pt} % Customize the height of the header

\numberwithin{equation}{section} % Number equations within sections (i.e. 1.1, 1.2, 2.1, 2.2 instead of 1, 2, 3, 4)
\numberwithin{figure}{section} % Number figures within sections (i.e. 1.1, 1.2, 2.1, 2.2 instead of 1, 2, 3, 4)
\numberwithin{table}{section} % Number tables within sections (i.e. 1.1, 1.2, 2.1, 2.2 instead of 1, 2, 3, 4)

\newcommand{\R}{\mathbb{R}}
\newcommand{\Q}{\mathbb{Q}}
\newcommand{\N}{\mathbb{N}}
\newcommand{\C}{\mathbb{C}}

\newtheorem{lemma}{Lemma}
%----------------------------------------------------------------------------------------
%	TITLE SECTION
%----------------------------------------------------------------------------------------

\newcommand{\horrule}[1]{\rule{\linewidth}{#1}} % Create horizontal rule command with 1 argument of height

\title{	
\normalfont \normalsize 
\textsc{Analysis} \\ [25pt] % Your university, school and/or department name(s)
\horrule{0.5pt} \\[0.4cm] % Thin top horizontal rule
\huge Problem Set 1 \\ % The assignment title
\horrule{2pt} \\[0.5cm] % Thick bottom horizontal rule
}

\author{Daniel Halmrast} % Your name

\date{\normalsize\today} % Today's date or a custom date

\begin{document}

\maketitle % Print the title

% Problems

\section*{Problem 1}
Let $\mathscr{F}$ be the set of all measurable functions which are finite
$\mu$-a.e. on $\Omega$.

\subsection*{Part a}
Prove $\mathscr{F}$ is a vector space.
\\
\\
\begin{proof}
    We first note that $\mathscr{F}$ is a subset of the vector space of all
    measurable functions (modulo functions zero $\mu$-a.e.). We just need to
    show, then, that $\mathscr{F}$ is closed under addition and scalar
    multiplication.

    So, let $f$ and $g$ be measurable functions that are finite $\mu$-a.e. on
    $\Omega$. Now, let's consider $f+g$. If $x\in\Omega$ is such that $f(x)$ and
    $g(x)$ are finite, then the sum $f(x)+g(x)$ is finite. Let $E$ be the set of all
    such $x$. We will show that $\Omega\setminus E$ has measure zero, so that
    $f+g$ is finite $\mu$-a.e.

    To see that $\Omega\setminus E$ has measure zero, we note that
    $\Omega\setminus E =
    \{|f| = \infty\}\cup \{|g| = \infty\}$. Now, since $f$ and $g$ are finite
    $\mu$-a.e., we know that each of these sets has measure zero, and so the
    union $\Omega\setminus E$ has measure zero as well. Thus, $f+g$ is finite
    $\mu$-a.e.

    It is immediately clear as well that for arbitrary scalar $\alpha$, we have
    that $\alpha f$ is also finite $\mu$-a.e., since for $x\in \{|f|<\infty\}$,
    we have that $|f(x)|<\infty$, which implies that $|\alpha f(x)|<\infty$ as
    well.

    Thus, $\mathscr{F}$ is an algebraically closed subspace of a vector space,
    and is a vector space itself.
\end{proof}

\subsection*{Part b}
Prove that $\mathscr{F}$ is a metric space with the metric
\[
    d(f,g) = \int_{\Omega}\frac{|f-g|}{1+|f-g|}d\mu
\]
\begin{proof}
    To show that $d$ is a metric, we need to show that $d(f,f)=0$, $d(f,g)>0$
    for $f\neq g$, and the triangle inequality $d(f,h)\leq d(f,g)+d(g,h)$.

    It is clear that $d(f,f)=0$, since this amounts to
    \[
        \begin{aligned}
            d(f,f) &= \int_{\Omega}\frac{|f-f|}{1+|F-f|}d\mu\\
                    &=\int_{\Omega}\frac{0}{1}d\mu\\
                    &=0
        \end{aligned}
    \]

    Now, suppose $f$ and $g$ differ on a positive-measure set $E$. In other
    words, $|f-g|\neq 0$ on $E$. Now, since $|f-g|$ is positive on $E$,
    $\frac{|f-g|}{1+|f-g|}$ is as well. Thus, $\frac{|f-g|}{1+|f-g|}\neq 0$ in
    $L^1$, and so $\|\frac{|f-g|}{1+|f-g|}-0\|_{L^1}
    =\int_{\Omega}\frac{|f-g|}{1+|f-g|}d\mu >0$ as desired.

    Finally, we wish to prove the triangle inequality. This will follow from the
    convexity and monotonicity of the function $\frac{x}{1+x}$. That is, for
    $f,g,h\in\mathscr{F}$, we have that

    \[

    \]
\end{proof}

\subsection*{Part c}
Show that $d$ metrizes the convergence in measure.
\\
\\
\begin{proof}
Suppose first that $f_n\to f$ in $\mu$. That is, for all $t>0$,
    \[
        \mu(\{|f_n-f|>t\})\to 0
    \]
    Now, consider the integral
    \[
        \begin{aligned}
            \int_{\Omega}\frac{|f_n-f|}{1+|f_n-f|}d\mu
            &= \int_{\{|f_n-f|\leq t\}}\frac{|f_n-f|}{1+|f_n-f|}d\mu +
            \int_{\{|f_n-f|>t\}}\frac{|f_n-f|}{1+|f_n-f|}d\mu\\
            &\leq \int_{\{|f_n-f|\leq t\}}\frac{t}{1+t}d\mu +
            \int_{\{|f_n-f|>t\}}1d\mu\\
        \end{aligned}
    \]
    Now, the first term goes to zero as $t$ goes to zero, and the second term
    is just $\mu(\{|f_n-f|>t\})$, which goes to zero as $n$ goes to infinity.
    Thus, the expression goes to zero, and $d(f_n,f)$ goes to zero, as desired.

    Now, suppose $d(f_n,f)$ goes to zero. We wish to prove that for all $t>0$,
    $\mu(\{|f_n-f| > t\})\to 0$. To do so, we consider
    \[
        \begin{aligned}
            d(f_n,f) &= \int_{\Omega}\frac{|f_n-f|}{1+|f_n-f|}d\mu\\
            &= \int_{\{|f_n-f|>t\}}\frac{|f_n-f|}{1+|f_n-f|}d\mu +
                \int_{\{|f_n-f|\leq t\}}\frac{|f_n-f|}{1+|f_n-f|}d\mu\\
            &\geq\int_{\{|f_n-f|>t\}}\frac{|f_n-f|}{1+|f_n-f|}d\mu\\
            &\geq\int_{\{|f_n-f|>t\}}\frac{t}{1+t}d\mu
            &=\frac{t}{1+t}\mu(\{|f_n-f|>t\})
        \end{aligned}
    \]
    and since $d(f_n,f)$ goes to zero, so does
    $\frac{t}{1+t}\mu(\{|f_n-f|>t\})$, which implies that for fixed $t>0$,
    \[
        \mu(\{|f_n-f|>t\})\to 0
    \]
    as well.
\end{proof}

\section*{Problem 2}
\subsection*{Part a}

%TODO more

\section*{Problem 5}
Construct a sequence of $L^1$ functions that converge $\mu$-a.e. to an $L^1$
function, but do not converge in $L^1$ norm. Show that such a sequence is not
uniformly integrable, and describe the concentration phenomenon.
\\
\\
\begin{proof}
    Consider the sequence
    \[
        f_n = n\chi_{[0,\frac{1}{n}]}
    \]
    which converges $\mu$-a.e. to the zero function. However, in $L^1$ norm, we
    have that
    \[
        \begin{aligned}
            \|f_n-0\|_{L^1} &= \int_{\R}|f_n-0|d\lambda^1\\
                            &= \int_{\R}n\chi_{[0,\frac{1}{n}]}d\lambda^1\\
                            &= \int_{[0,\frac{1}{n}]}nd\lambda^1\\
                            &=1
        \end{aligned}
    \]
    which does not go to zero as $n$ goes to infinity.

    %TODO finish
\end{proof}


\end{document}
