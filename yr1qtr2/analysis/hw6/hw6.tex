%%%%%%%%%%%%%%%%%%%%%%%%%%%%%%%%%%%%%%%%%
% Short Sectioned Assignment
% LaTeX Template
% Version 1.0 (5/5/12)
%
% This template has been downloaded from:
% http://www.LaTeXTemplates.com
%
% Original author:
% Frits Wenneker (http://www.howtotex.com)
%
% License:
% CC BY-NC-SA 3.0 (http://creativecommons.org/licenses/by-nc-sa/3.0/)
%
%%%%%%%%%%%%%%%%%%%%%%%%%%%%%%%%%%%%%%%%%

%----------------------------------------------------------------------------------------
%	PACKAGES AND OTHER DOCUMENT CONFIGURATIONS
%----------------------------------------------------------------------------------------

\documentclass[fontsize=11pt]{scrartcl} % 11pt font size

\usepackage[T1]{fontenc} % Use 8-bit encoding that has 256 glyphs
\usepackage[english]{babel} % English language/hyphenation
\usepackage{amsmath,amsfonts,amsthm} % Math packages
\usepackage{mathrsfs}

\usepackage[margin=1in]{geometry}

\usepackage{sectsty} % Allows customizing section commands
\allsectionsfont{\centering \normalfont\scshape} % Make all sections centered, the default font and small caps

\usepackage{fancyhdr} % Custom headers and footers
\pagestyle{fancyplain} % Makes all pages in the document conform to the custom headers and footers
\fancyhead{} % No page header - if you want one, create it in the same way as the footers below
\fancyfoot[L]{} % Empty left footer
\fancyfoot[C]{} % Empty center footer
\fancyfoot[R]{\thepage} % Page numbering for right footer
\renewcommand{\headrulewidth}{0pt} % Remove header underlines
\renewcommand{\footrulewidth}{0pt} % Remove footer underlines
\setlength{\headheight}{13.6pt} % Customize the height of the header

\numberwithin{equation}{section} % Number equations within sections (i.e. 1.1, 1.2, 2.1, 2.2 instead of 1, 2, 3, 4)
\numberwithin{figure}{section} % Number figures within sections (i.e. 1.1, 1.2, 2.1, 2.2 instead of 1, 2, 3, 4)
\numberwithin{table}{section} % Number tables within sections (i.e. 1.1, 1.2, 2.1, 2.2 instead of 1, 2, 3, 4)

\newcommand{\R}{\mathbb{R}}
\newcommand{\Q}{\mathbb{Q}}
\newcommand{\N}{\mathbb{N}}
\newcommand{\C}{\mathbb{C}}

\newtheorem{lemma}{Lemma}
%----------------------------------------------------------------------------------------
%	TITLE SECTION
%----------------------------------------------------------------------------------------

\newcommand{\horrule}[1]{\rule{\linewidth}{#1}} % Create horizontal rule command with 1 argument of height

\title{	
\normalfont \normalsize 
\textsc{Analysis} \\ [25pt] % Your university, school and/or department name(s)
\horrule{0.5pt} \\[0.4cm] % Thin top horizontal rule
\huge Problem Set 6 \\ % The assignment title
\horrule{2pt} \\[0.5cm] % Thick bottom horizontal rule
}

\author{Daniel Halmrast} % Your name

\date{\normalsize\today} % Today's date or a custom date

\begin{document}

\maketitle % Print the title

% Problems
\section*{Problem 1}
Show that the weak and weak-$*$ topologies are Hausdorff.
\\
\\
\begin{proof}
    We start by showing the weak topology is Hausdorff. Recall from the midterm
    that $X^*$ separates points of $X$. So, for $x,y$ distinct points in $X$,
    let $\phi$ separate them. Choose $U,V\subset \R$ disjoint open sets
    containing $\phi(x)$ and $\phi(y)$ respectively. Then, since $\phi$ is
    continuous in the weak topology, $\phi^{-1}(U)$ and $\phi^{-1}(V)$ are
    disjoint open sets in $X$ that contain $x,y$ respectively, and thus $X$ is
    Hausdorff.

    For the weak-$*$ topology, note that all we have to show is that $X$
    separates the functionals in $X^*$, and we can apply the same argument as
    above. So, suppose $\phi,\psi$ are distinct elements of $X^*$. Then, by
    definition, there is some $x\in X$ for which $\phi(x)\neq \psi(x)$. Thus,
    $X$ separates points in $X^*$, and $X^*$ with the weak-$*$ topology is
    Hausdorff.
\end{proof}

\newpage

\section*{Problem 2}
Find the weak closure of the unit sphere $S$.
\\
\\
\begin{proof}
    Note first that we have proven already the unit ball is closed. Now, since
    $S$ is contained in the unit ball $B$, we know that $\overline{S}\subset B$
    as well.

    Now, let $x\in B$. Recall that any basic open set centered at $x$ contains a
    translation of a subspace through $x$. In particular, any open set $U$
    containing $x$ must also contain a subspace which goes through $x$. Now,
    since $x$ is inside $B$, and $B$ is bounded, it must be that $U$ intersects
    the boundary $S$. 

    Thus, for every point $x\in B$, every neighborhood of $x$ intersects $S$,
    and thus $x\in\overline{S}$, and in particular $B\subset\overline{S}$.

    Thus, $\overline{S} = B$.
\end{proof}

\newpage

\section*{Problem 3}
Show that the sets $W(\phi;x_1,\dots,x_n)$ form a basis for a topology on $X^*$.
Furthermore, convergence in this topology is the same as convergence in weak-$*$
topology.
\\
\\
\begin{proof}
    First, we show that these form a basis for a topology. First, it is clear
    that every element $\phi\in X^*$ is contained in a basic open set. Namely,
    $\phi\in W(\phi;x)$ for any $x$.

    Next, we show that every point in the intersection of two basic open sets
    has a basic open set containing it that is contained in the intersection.

    So, let $W(\phi_1;x_1,\dots,x_n)$ and $W(\phi_2;y_1,\dots,y_m)$ be basic
    open sets, and let $\psi$ be in their intersection. Then, the basic open set
    $U = W(\psi, C_1x_1,\dots,C_nx_n,D_1y_1,\dots,D_my_m)$ contains $\psi$, and for
    suitable choice of constants will be contained in the intersection. To see
    this, let $\varepsilon_i$ be such that
    $\|(\phi_1-\psi)(x_i)\|<1-\varepsilon_i$, and $\delta_i$ be such that
    $\|(\phi_2-\psi)(y_i)\|<1-\delta_i$. If we set $C_i =
    \frac{1}{\varepsilon_i}$ and $D_i = \frac{1}{\delta_i}$, we note that for
    any $\varphi\in U$, we have
    \[
\begin{aligned}
    \|(\varphi -\phi_1)(x_i)\| &= \|(\varphi - \psi + \psi - \phi_1)(x_i)\|\\
    &\leq \|(\varphi - \psi)(x_i)\| + \|(\psi - \phi_1)(x_i)\|\\
    &= \varepsilon_i\|(\varphi-\psi)(C_ix_i)\| + \|(\psi - \phi_1)(x_i)\|\\
    &\leq \varepsilon_i(1) + (1-\varepsilon_i) = 1
\end{aligned}
    \]
    and similarly for $\phi_2$. Thus, for any $\varphi\in U$, we have that
    $\varphi\in W(\phi_1;x_1,\dots,x_n)$ and $\varphi\in
    W(\phi_2;y_1,\dots,y_m)$ as desired.

    Thus, these sets form a basis for a topology.
    \\
    \\
    Now, let's show that convergence in this topology is equivalent to weak-$*$
    convergence.

    We first make the following observation. The sets $W(\phi;x_1,\dots,x_n)$
    are generated by finite intersections of the subbasis $W(\phi;x)$.
    Furthermore, this is equivalent to the subbasis given by
    \[
        \{\{\psi\in X^*\ |\ \|(\psi - \phi)(x) < r\}, \phi\in X^*,x\in X, r\in
        \R\}
    \]
    (since $\|(\psi-\phi)(x)\| < r$ is the same as
    $\|(\psi-\phi)(\frac{x}{r})<1$.)

    This leads to the neighborhood subbasis of zero given by
    \[
        \{\{\psi\in X^*\ |\ \|\psi(x)\|<r\}, x\in X, r\in \R\}
    \]
    which is equivalent to 
    \[
        \{x^{-1}(U)\ |\ U\subset \R \text{open}, 0\in U\}
    \]

    Now, without loss of generality, let's show that $\phi_n\to 0$ in weak-$*$ if and
    only if $\phi_n\to 0$ in this topology.

    Now, clearly each $x: X^*\to \R$ is continuous at zero, and thus is
    continuous. So, if $\phi_n\to 0$ in the topology, then $x(\phi_n)\to x(0)=0$
    for all $x$ (since $x$ is continuous, it preserves limits). Thus, $\phi_n\to
    0$ in weak-$*$ as well.

    Suppose instead that $\phi_n\to 0$ in weak-$*$. It suffices to show that
    every basic open neighborhood of $0$ eventually contains the sequence
    $\phi_n$. So, let $B = \cap_{i=1}^m x_i^{-1}(U_i)$ for $U_i$ an open
    neighborhood of zero. Then, it follows that since $\phi_n(x_i)\to 0$ for all
    $i$, then for each $i$ there is an $N_i$ such that $\phi_n(x_i)\in U_i$ for
    all $n>N_i$.

    Set $N = \max(N_i)$. Then, for each $n>N$, we have
    \[
        \begin{aligned}
            \phi_n&\in x_i^{-1}(U_i)\\
            \implies \phi_n\in \cap_{i=1}^m x_i^{-1}(U_i) = B
        \end{aligned}
    \]
    and thus $B$ eventually contains the sequence $\phi_n$, and so $\phi_n\to 0$
    in the topology as desired.
\end{proof}

\newpage
\section*{Problem 4}
Let $X$ be a seperable Banach space with countable dense subset $x_n$. Show that
the weak topology restricted to the unit ball coincides with the metric topology
given by 
\[
    d(\phi,\psi) =
    \sum_{n=1}^{\infty}2^{-n}\frac{\|(\phi-\psi)(x_n)\|}{1+\|(\phi-\psi)(x_n)\|}
\]

\begin{proof}
    First, we show that the identity map from $X$ with the weak topology to $X$
    with the metric topology is continuous. For this, it suffices to show that
    the ball $B_{\varepsilon}(0)$ contains a basic open set of the weak topology
    centered at zero (intersected with the ball).

    To see this, we consider the set $W = W(0;y_1,\dots,y_m)$ where $y_i =
    \frac{x_i}{\varepsilon}$, and $m$ is large enough so that
    $2^{-m}<\varepsilon$. Then, we have for $\varphi\in W$,
    \[
\begin{aligned}
    d(\varphi,0) &=
\sum_{n=1}^{\infty}2^{-n}\frac{\|\varphi(x_n)\|}{1+\|\varphi(x_n)\|}\\
&\leq \sum_{n=1}^m 2^{-n}\|\varphi(x_n)\| + \sum_{n=m+1}^{\infty}2^{-m}\\
&\leq 2\sum_{n=1}^m \varepsilon\|\varphi(y_n)\| + \varepsilon\\
&\leq 2m\varepsilon + \varepsilon
\end{aligned}
    \]
    and thus (for a more suitable choice of $y_i$ and $m$) the ball
    $B_{\varepsilon}(0)$ contains the basic open set $W$, and thus the identity
    map is continuous from the weak topology to the metric topology.

    Now, we know two important facts. One, the unit ball in the weak topology is
    compact. Two, the metric topology is always Hausdorff. So, any closed set in
    a compact space is compact, and images of compact spaces under continuous
    maps are compact. Finally, compact subsets of Hausdorff spaces are always
    closed. Thus, the image of any closed set under the identity is closed.
    So, for $U$ open in the weak topology, its complement $U^c$ is closed, and
    thus is closed in the metric topology as well. Thus, $U$ is open in the
    metric topology as well.

    Thus, the two topologies coincide, as desired.
\end{proof}
\end{document}
