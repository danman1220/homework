%%%%%%%%%%%%%%%%%%%%%%%%%%%%%%%%%%%%%%%%%
% Short Sectioned Assignment
% LaTeX Template
% Version 1.0 (5/5/12)
%
% This template has been downloaded from:
% http://www.LaTeXTemplates.com
%
% Original author:
% Frits Wenneker (http://www.howtotex.com)
%
% License:
% CC BY-NC-SA 3.0 (http://creativecommons.org/licenses/by-nc-sa/3.0/)
%
%%%%%%%%%%%%%%%%%%%%%%%%%%%%%%%%%%%%%%%%%

%----------------------------------------------------------------------------------------
%	PACKAGES AND OTHER DOCUMENT CONFIGURATIONS
%----------------------------------------------------------------------------------------

\documentclass[fontsize=11pt]{scrartcl} % 11pt font size

\usepackage[T1]{fontenc} % Use 8-bit encoding that has 256 glyphs
\usepackage[english]{babel} % English language/hyphenation
\usepackage{amsmath,amsfonts,amsthm} % Math packages
\usepackage{mathrsfs}

\usepackage[margin=1in]{geometry}

\usepackage{sectsty} % Allows customizing section commands
\allsectionsfont{\centering \normalfont\scshape} % Make all sections centered, the default font and small caps

\usepackage{fancyhdr} % Custom headers and footers
\pagestyle{fancyplain} % Makes all pages in the document conform to the custom headers and footers
\fancyhead{} % No page header - if you want one, create it in the same way as the footers below
\fancyfoot[L]{} % Empty left footer
\fancyfoot[C]{} % Empty center footer
\fancyfoot[R]{\thepage} % Page numbering for right footer
\renewcommand{\headrulewidth}{0pt} % Remove header underlines
\renewcommand{\footrulewidth}{0pt} % Remove footer underlines
\setlength{\headheight}{13.6pt} % Customize the height of the header

\numberwithin{equation}{section} % Number equations within sections (i.e. 1.1, 1.2, 2.1, 2.2 instead of 1, 2, 3, 4)
\numberwithin{figure}{section} % Number figures within sections (i.e. 1.1, 1.2, 2.1, 2.2 instead of 1, 2, 3, 4)
\numberwithin{table}{section} % Number tables within sections (i.e. 1.1, 1.2, 2.1, 2.2 instead of 1, 2, 3, 4)

\newcommand{\R}{\mathbb{R}}
\newcommand{\Q}{\mathbb{Q}}
\newcommand{\N}{\mathbb{N}}
\newcommand{\C}{\mathbb{C}}

\newtheorem{lemma}{Lemma}
%----------------------------------------------------------------------------------------
%	TITLE SECTION
%----------------------------------------------------------------------------------------

\newcommand{\horrule}[1]{\rule{\linewidth}{#1}} % Create horizontal rule command with 1 argument of height

\title{	
\normalfont \normalsize 
\textsc{Analysis} \\ [25pt] % Your university, school and/or department name(s)
\horrule{0.5pt} \\[0.4cm] % Thin top horizontal rule
\huge Problem Set 2 \\ % The assignment title
\horrule{2pt} \\[0.5cm] % Thick bottom horizontal rule
}

\author{Daniel Halmrast} % Your name

\date{\normalsize\today} % Today's date or a custom date

\begin{document}

\maketitle % Print the title

% Problems
\section*{Problem 1}
Show that the function sending $\phi$ to $\phi^{-1}(\{1\})$ is a bijection
between nonzero bounded linear functionals and hyperplanes not containing $0$.
\\
\\
\begin{proof}
    We first show that for arbitrary bounded linear functional $\phi$, the set
    $\phi^{-1}(\{1\})$ is a closed hyperplane.

    To see this, let $\phi\in X^*\setminus \{0\}$. In particular, this means
    that $H:=\phi^{-1}(\{1\})$ is nonempty. So, let $x_0\in H$. Then, we have
    $\phi(x_0) = 1$. Now, let $x\in X$ be arbitrary, and consider
    \[
        \phi(x-\phi(x)x_0) = \phi(x)-\phi(x)\phi(x_0)=0
    \]
    This implies that $y:=x-\phi(x)x_0$ is in the kernel of $\phi$. Solving for
    $x$ yields
    \[
        x = y+\phi(x)x_0
    \]
    and so $X=\ker\phi\oplus\text{span}(x_0)$.

    Thus, we know that $\ker\phi$ has codimension $1$. In particular, we know
    that for $y\in\ker\phi$ we have that $\phi(y+x_0) = 1$ and so the hyperplane
    obtained by translating $\ker\phi$ is in $\phi^{-1}(\{1\})$. Thus, since
    $\phi^{-1}(\{1\})$ does not contain zero, it follows that it is the
    translate of $\ker\phi$ by $x_0$. The hyperplane is closed because it is the
    inverse image of a point.

    Now, let's start with a hyperplane $H$ that misses zero, and show that it is the
    inverse image of $\{1\}$ by some function. To begin with, we note that a
    hyperplane missing zero is just a hyperplane containing zero translated by
    some constant vector. In particular,for each hyperplane missing zero, there
    is a unique hyperplane through zero that is the translate of it. Let $H'$ be
    the hyperplane through zero that is the translate of $H$. Then, since $H'$
    is closed with codimension $1$, we can write $X=H'\oplus \text{span}(v)$ for
    some $v$. Define $\phi$ to be
    \[
        \phi(x) = \phi(h+\lambda v) = \lambda
    \]
    for $\lambda\in\C$.

    Now, $H$ is just $H'+c$ for some $c\in X$, and so $H'$ is the set
    \[
        \{x:\ \phi(x-c)=0\}
    \]
    or,
    \[
        \{x:\ \phi(x)=\phi(c)\}
    \]
    letting $c=h+\eta v$ for $\eta\in\C$, we see that $\phi(x) = \eta$ defines
    the hyperplane.

    Now, define $\phi'(x) = \frac{\phi(x)}{\eta}$, and note that
    \[
        H' = \{x:\ \phi'(x)=1
    \]
    as desired.

    Now, let's show that distinct functionals generate distinct hyperplanes. To
    see this, let $\phi,\psi$ be such that
    $\phi^{-1}(\{1\})=\psi^{-1}(\{1\})=H$. Now, consider the bounded linear
    functional defined by
    \[
        \theta(x) = \phi(x)-\psi(x)
    \]
    Clearly, $H\subset\ker\theta$, and so since $H$ has codimension $1$ and does
    not contain zero, it follows immediately that $\ker\theta=X$ and so $\phi(x)
    = \psi(x)$ for all $x$ as desired.
\end{proof}


\section*{Problem 2}
For $K\subset L$ an absorbing convex set, define
\[
    \mu_K(x) = \inf\{t>0|\ x\in tK\}
\]
Show that $\mu_K$ is convex, and show that if $\phi\in L'$, then $\phi\leq \mu_K$
implies $\phi|K\leq 1$ and vice versa.
\\
\\
\begin{proof}
    We first show that $\mu_K$ is homogeneous. To see this, we note that
    \[
        \begin{aligned}
            \mu_K(ax) &= \inf\{t>0|\ ax\in tK\}\\
                    &= \inf\{t>0|\ x\in \frac{t}{a}K\}\\
                    &=a\inf\{t>0|\ x\in tK\}]\\
                    &= a\mu_K(x)
        \end{aligned}
    \]
    as desired. Now, we show that $\mu_K$ is subadditive. To see this, consider
    \[
        \mu_K(x) + \mu_K(y)
    \]
    and let $t^* = \max(\mu_K(x),\mu_K(y))$. Now, we know that $d(x,t^*K) =
    d(y,t^*K)=0$, and by the triangle inequality, this forces $d(x+y,t^*K)$ to
    be zero as well. Thus, $\mu_K(x+y)\geq t^*$. It follows immediately that
    \[
        \mu_K(x+y)\leq \mu_K(x)+\mu_K(y)
    \]
    since $\mu_K$ is positive.

    Now to show the second statement.
    \\
    ($\implies$)
    Suppose $\phi$ is such that $\phi\leq \mu_K$. It follows immediately from
    the definition that $\mu_K|_K\leq 1$, and so$\phi|_K\leq 1$ as well.

    ($\impliedby$)
    Suppose $\phi|_K\leq 1$. We wish to show (by homogeneity) that
    $\phi(\frac{x}{\mu_K(x)})\leq\mu_K(\frac{x}{\mu_K(x)})=1$. Now, since $K$ is
    closed, $\frac{x}{\mu_K(x)}$ is in $K$ (since $\mu_K(x)$ is the $\inf$ of
    all $t$ such that $\frac{x}{t}$ is in $K$). Now, it follows by the
    hypothesis that
    \[
        \phi(\frac{x}{\mu_K(x)})\leq 1=\mu(\frac{x}{\mu_K(x)})
    \]
    as desired.
\end{proof}

\section*{Problem 3}
Prove that for every $e\in X$ not equal to zero, there is some $\phi\in V^*$
such that $\phi(e) = \|e\|$ and $\|\phi\|_*=1$.
\\
\\
\begin{proof}
    We note first that there is a bounded linear functional $\rho(x)$ on
    $\text{span}(e)$ given by $\rho(\lambda e)=\lambda\|e\|$. 
    Hahn-Banach then guarantees the
    existence of $\phi\in X^*$ with $\phi|_{\text{span}(e)} = \rho$ and
    $\|\phi\|_*=\|\rho\|$. This is the same as saying $\phi(e) =\|e\|$ and
    $\|\phi\|_*=1$ as desired.
\end{proof}

\section*{Problem 4}
Let $X$, $Y$ be normed spaces, and $Z=X\oplus Y$ with $\|z\| = \|x\|+\|y\|$.
What is $Z^*$?
\\
\\
\begin{proof}
    I assert that $Z^* = X^*\oplus Y^*$ with the (linear) isomorphism
    \[
        \phi_x\times \phi_y \mapsto (z=x+y\mapsto \phi_x(x)+\phi_y(y))
    \]
    Now, clearly this map is injective, and the resulting functional is bounded
    (since $\phi_x$ and $\phi_y$ are bounded, and projections are bounded). So,
    all we need to show is that this map is surjective.

    However, consider some $\psi\in Z^*$. This can be written as
    \[
        \begin{aligned}
            \psi(z) &= \psi(x+y)\\
                    &= \psi(x) + \psi(y)\\
                    &= \psi|_X(x) + \psi|_Y(y)
        \end{aligned}
    \]
    and since $\psi|_X$ and $\psi|_Y$ are in $X^*$ and $Y^*$ respectively, it
    follows that the isomorphism maps $\psi|_X\times \psi|_Y$ to $\psi$, and the
    isomorphism is surjective.

    Thus, the spaces are equal.
\end{proof}

\section*{Problem 5}
Show that if $f$ is a positive linear functional on $l^{\infty}$, then $f$ is
bounded.
\\
\\
\begin{proof}
    Suppose $f$ is a positive linear functional, and let $x\in l^{\infty}$.
    Then, we know that $-x + \|x\|_{l^{\infty}}(1,1,\ldots) \geq 0$. So,
    \[
        \begin{aligned}
            f(-x + \|x\|_{\infty}(1,1,\ldots)) &\geq 0\\
            f(-x)\geq -f((1,1,\ldots))\|x\|_{\infty}\\
            f(x) \leq f((1,1,\ldots))\|x\|_{\infty}\\
            |f(x)| \leq f((1,1,\ldots))\|x\|_{\infty}\\
        \end{aligned}
    \]
    and thus $f$ is bounded.
\end{proof}

\section*{Problem 6}
Let $Y$ be a subspace of $X$. Show that the closure of $Y$ is the intersection
of all kernels containing $Y$.
\\
\\
\begin{proof}
    We first prove the hint: for $Y$ a closed subspace and $x_0\not\in Y$, there
    exists a function $\phi\in X^*$ such that $\phi(x_0) = 1$ and $\phi|_Y=0$.

    This is clear by considering the bounded linear functional $\xi$ on
    $Y\oplus\text{span}(x_0)$ given by $\xi(y+\lambda x_0) = \lambda$.
    Hahn-Banach guarantees this extends to a bounded linear functional on all of
    $X$.

    Now, clearly $\overline{y}\subset \bigcap_{f\ |\ \ker f\supset Y}\ker f$,
    since $\overline{Y}$ is the intersection of all closed subsets containing
    it, and kernels are closed.

    Now, we wish to show the other containment. To do so, we will show that for
    any point not in $\overline{Y}$ is outside of some kernel containing $Y$. To
    see this, we apply the hint on $x_0\not\in \overline{Y}$ to obtain $\phi\in X^*$ with
    $\phi(x_0)=1$ and $\phi|_Y=0$. Thus, $\ker\phi$ contains $Y$ but does not
    contain $x_0$. Thus, $x_0$ is not in the intersection.

    Therefore, the two sets are equal, as desired.
\end{proof}


\end{document}
