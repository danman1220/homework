%%%%%%%%%%%%%%%%%%%%%%%%%%%%%%%%%%%%%%%%%
% Short Sectioned Assignment
% LaTeX Template
% Version 1.0 (5/5/12)
%
% This template has been downloaded from:
% http://www.LaTeXTemplates.com
%
% Original author:
% Frits Wenneker (http://www.howtotex.com)
%
% License:
% CC BY-NC-SA 3.0 (http://creativecommons.org/licenses/by-nc-sa/3.0/)
%
%%%%%%%%%%%%%%%%%%%%%%%%%%%%%%%%%%%%%%%%%

%----------------------------------------------------------------------------------------
%	PACKAGES AND OTHER DOCUMENT CONFIGURATIONS
%----------------------------------------------------------------------------------------

\documentclass[fontsize=11pt]{scrartcl} % 11pt font size

\usepackage[T1]{fontenc} % Use 8-bit encoding that has 256 glyphs
\usepackage[english]{babel} % English language/hyphenation
\usepackage{amsmath,amsfonts,amsthm} % Math packages
\usepackage{mathrsfs}

\usepackage[margin=1in]{geometry}

\usepackage{sectsty} % Allows customizing section commands
\allsectionsfont{\centering \normalfont\scshape} % Make all sections centered, the default font and small caps

\usepackage{fancyhdr} % Custom headers and footers
\pagestyle{fancyplain} % Makes all pages in the document conform to the custom headers and footers
\fancyhead{} % No page header - if you want one, create it in the same way as the footers below
\fancyfoot[L]{} % Empty left footer
\fancyfoot[C]{} % Empty center footer
\fancyfoot[R]{\thepage} % Page numbering for right footer
\renewcommand{\headrulewidth}{0pt} % Remove header underlines
\renewcommand{\footrulewidth}{0pt} % Remove footer underlines
\setlength{\headheight}{13.6pt} % Customize the height of the header

\numberwithin{equation}{section} % Number equations within sections (i.e. 1.1, 1.2, 2.1, 2.2 instead of 1, 2, 3, 4)
\numberwithin{figure}{section} % Number figures within sections (i.e. 1.1, 1.2, 2.1, 2.2 instead of 1, 2, 3, 4)
\numberwithin{table}{section} % Number tables within sections (i.e. 1.1, 1.2, 2.1, 2.2 instead of 1, 2, 3, 4)

\newcommand{\R}{\mathbb{R}}
\newcommand{\Q}{\mathbb{Q}}
\newcommand{\N}{\mathbb{N}}
\newcommand{\C}{\mathbb{C}}

\newtheorem{lemma}{Lemma}
%----------------------------------------------------------------------------------------
%	TITLE SECTION
%----------------------------------------------------------------------------------------

\newcommand{\horrule}[1]{\rule{\linewidth}{#1}} % Create horizontal rule command with 1 argument of height

\title{	
\normalfont \normalsize 
\textsc{Analysis} \\ [25pt] % Your university, school and/or department name(s)
\horrule{0.5pt} \\[0.4cm] % Thin top horizontal rule
\huge Problem Set 4 \\ % The assignment title
\horrule{2pt} \\[0.5cm] % Thick bottom horizontal rule
}

\author{Daniel Halmrast} % Your name

\date{\normalsize\today} % Today's date or a custom date

\begin{document}

\maketitle % Print the title

% Problems
\section*{Problem 1}
Show that if a smooth function $f$ is such that for all $x$, there is some $N_x$
such that the $n$th derivative of $f$ vanishes at $x$ for all $n>N_x$, then $f$
is a polynomial.
\\
\\
\begin{proof}
Let $G$ be the set of all $x\in\R$ such that there exists some neighborhood of
$x$ for which $f$ is a polynomial on that neighborhood, and let $F=G^c$.

Now, we first observe that $F$ has no isolated points.
Indeed, suppose $c\in\R$ were such that $f$ was a polynomial on $(a,c)$ and
$(c,b)$ for some $a,b\in\R$. Then, by observation 2 made in class, since $f$ at
$c$ has derivatives vanishing beyond $N_c$ and $f$ is a polynomial on $(a,c)$
and $(c,b)$, we know that $f$ is the same polynomial on $(a,c]$ and $[c,b)$ and
so $c$ cannot be in $F$ (since it is now in $G$).

Now, we wish to show that $F=\emptyset$. So, suppose for contradiction $F$ is
nonempty. Now, since $G$ is open (by definition) $F$ is closed, and is thus a
complete metric space as a closed subset of $\R$.

Define $E_n = \{x\in F\ |\ f^{(j)}(x) = 0 \forall j>n\}$. Now, $F = \cup E_n$
and so at least one of them $E_{n_0}$ must contain an interval
$(x_0-r,x_0+r)\cap F$ for $x_0\in F$ (by Baire's lemma).

Now, for all $x\in (x_0-r,x_0+r)\cap F$, we have that $f^{(j)}(x) = 0$ for all
$j>n_0$. If $(x_0-r,x_0+r)$ does not intersect $G$, then $f^{(j)}(x) = 0$ for
all $j>n_0$ and all $x\in (x_0-r,x_0+r)$ and so $x_0\in G$, a contradiction.

So, if $I=(a,b)$ is an interval in $G$ contained in $(x_0-r,x_0+r)$, then $b\in
F$. But $b$ is such that $f^{(j)}(b) = 0$, and so $f$ is a polynomial on
$[a,b]$, and this leads to a contradiction on $(x_0-r,x_0+r)$ containing points
of $F$.

Thus, $F$ is empty, and $f$ is a polynomial on all of $\R$.
\end{proof}

\section*{Problem 2}
Prove that if a vector space is Banach with respect to two norms then the
topologies induced by the norms are either equivalent or incomparable.
\\
\\
\begin{proof}
    We will show that for two norms $\|\cdot\|_1$ and $\|\cdot\|_2$ on a
    vector space $V$ that is complete with respect to the induced
    topologies $\tau_1,\tau_2$, if $\tau_1\subset \tau_2$, then $\tau_1=\tau_2$
    which completes the proof.

    To see this, suppose $\tau_1\subset\tau_2$. Then, the ball $B^1(0,1)$ with
    respect to the $1$ norm is open in $\tau_2$, and thus $0$ has a ball
    $B^2(0,\epsilon)\subset B^1(0,1)$. Thus, by homogeneity of vector spaces,
    every $1$ norm ball contains a $2$ norm ball centered at the same point.

    Now, let $V_1$ denote $V$ under the $\|\cdot\|_1$ norm. Now, we know that $V
    = \cup_{x\in V_1}B^2(x,1)$ for $B^2$ the balls in the $2$ norm. Now, Baire's
    lemma guarantees that for some $x_0$, there is a ball
    $B^1(x_0,\delta)\subset  B^2(x_0,1)$. Then, translating these to the
    origin yields $B^1(0,\delta) \subset B^2(0,1)$. Thus, every $2$ norm ball
    contains a $1$ norm ball centered at the same point.

    Thus, the two topologies are equivalent.
\end{proof}

\section*{Problem 3}
Let $A\in B(X,Y)$ for $X,Y$ Banach spaces. Suppose that for every $f\in Y$, the
equation $Au=f$ is solvable. Prove that there exists $C<\infty$ such that for
every $f$ in $y$, one can find a solution to $Au=f$ with $\|u\|_X < C\|f\|_Y$.
\\
\\
\begin{proof}
    Since $Au=f$ is solvable for any $f$, $A$ is injective and satisfies the
    hypotheses for the open mapping theorem. Thus, $A$ is an open map.
    This means that $A(B_X(0,1))$ is open in $Y$, and contains zero. Thus, there
    is some $\epsilon$ for which $B_Y(0,\epsilon)\subset A(B_X(0,1))$. Thus, for
    $f$ such that $\|f\|<\epsilon$, and for $u$ such that $Au=f$, we have
    $\|u\|<1$.

    So, let $f$ be arbitrary. Then, let $\tilde{f} = \frac{\epsilon f}{2\|f\|}$
    with solution $\tilde{u} = \frac{\epsilon u}{2\|f\|}$. Now, $\|\tilde{f}\| <
    \epsilon$ and so $\|\tilde{u}\|<1$. Thus, $\|u\| < \frac{2}{\epsilon}\|f\|$
    as desired.
\end{proof}

\section*{Problem 4}
Let $X$ be a normed space, $E\subset X$. Suppose that for every $\phi\in X^*$
the set $\{\phi(x)\ |\ x\in E\}$ is bounded. Prove that $E$ is strongly bounded
in $X$.
\\
\\
\begin{proof}
    Let's start by embedding $E$ into $X^{**}$ via the canonical mapping. Then,
    we notice that the hypothesis states that for all $\phi\in X^*$, the set
    $\{x(\phi)\ |\ x\in E\}$ is bounded. In particular, this means that for all
    $\phi\in X^*$, we have that $\sup_{x\in E} |x(\phi)|$ is finite. Since $X^*$
    is complete, we can apply the uniform boundedness principle to the
    collection $E$ of linear functionals on $X^*$ (as operators from $X^*$ to
    $\R$) to get the result
    \[
        \sup_{x\in E}\|x\| < \infty
    \]
    and thus since $E$ was isometrically embedded into $X^{**}$, it follows that
    $E$ is bounded as desired.
\end{proof}

\section*{Problem 5}
Let $X$ be a Banach space, and let $E\subset X^*$. Suppose also that for every
$x\in X$, the set $\{\phi(x)\ |\ \phi\in E\}$ is bounded. Prove that $E$ is
strongly bounded in $X^*$.
\\
\\
\begin{proof}
    We note first that the condition that $\{\phi(x)\ |\ \phi\in E\}$ is bounded
    means that $\sup_{\phi\in E} |\phi(x)|$ is bounded for each $x$. Thus, we
    can apply the uniform boundedness principle to the family of operators $E$
    from $X$ to $\R$ to get the result
    \[
        \sup_{\phi\in E} \|\phi\|<\infty
    \]
    as desired.

    If $X$ is not complete, then we cannot use the uniform boundedness
    principle, and the proof falls apart.
\end{proof}

\section*{Problem 6}
Let $X$ be a Banach space decomposed as $X=X_1\oplus X_2$ with closed subspaces
$X_i$.
Define the projection operators $P_i:X\to X_i$ and prove they are bounded.
\\
\\
\begin{proof}
    Without loss of generality, we will define $P_1$ and show it is bounded. The
    argument works the same for $P_2$.

    Define $P_1(x) = P_1(x_1+x_2) = x_1$ for the (unique) decomposition
    $x=x_1+x_2$. Now, clearly $P_1$ is linear, so all we have to do is show it
    is bounded.

    Since $P_1:X\to X_1$ is a map between Banach spaces, we only need to show
    its graph is closed. So, let $(x_k, P_1(x_k))\to (x,y)$ for some $x\in X$
    and $y\in X_1$. Now, for each $x_k$, we can decompose it as
    \[
        x_k = (x_k)_1 + (x_k)_2
    \]
    for $(x_k)_i\in X_i$. Now, since $x_k\to x$ and $P_1(x_k) = (x_k)_1\to y$,
    it must be that $(x_k)_2\to z$ for some $z\in X_2$. Thus, $x = y+z$ and
    $P_1(x) = y$ as desired. Thus, the graph of $P_1$ is closed, and $P_1$ is
    bounded as desired.
\end{proof}


\end{document}
