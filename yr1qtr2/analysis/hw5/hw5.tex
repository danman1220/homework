%%%%%%%%%%%%%%%%%%%%%%%%%%%%%%%%%%%%%%%%%
% Short Sectioned Assignment
% LaTeX Template
% Version 1.0 (5/5/12)
%
% This template has been downloaded from:
% http://www.LaTeXTemplates.com
%
% Original author:
% Frits Wenneker (http://www.howtotex.com)
%
% License:
% CC BY-NC-SA 3.0 (http://creativecommons.org/licenses/by-nc-sa/3.0/)
%
%%%%%%%%%%%%%%%%%%%%%%%%%%%%%%%%%%%%%%%%%

%----------------------------------------------------------------------------------------
%	PACKAGES AND OTHER DOCUMENT CONFIGURATIONS
%----------------------------------------------------------------------------------------

\documentclass[fontsize=11pt]{scrartcl} % 11pt font size

\usepackage[T1]{fontenc} % Use 8-bit encoding that has 256 glyphs
\usepackage[english]{babel} % English language/hyphenation
\usepackage{amsmath,amsfonts,amsthm} % Math packages
\usepackage{mathrsfs}

\usepackage[margin=1in]{geometry}

\usepackage{sectsty} % Allows customizing section commands
\allsectionsfont{\centering \normalfont\scshape} % Make all sections centered, the default font and small caps

\usepackage{fancyhdr} % Custom headers and footers
\pagestyle{fancyplain} % Makes all pages in the document conform to the custom headers and footers
\fancyhead{} % No page header - if you want one, create it in the same way as the footers below
\fancyfoot[L]{} % Empty left footer
\fancyfoot[C]{} % Empty center footer
\fancyfoot[R]{\thepage} % Page numbering for right footer
\renewcommand{\headrulewidth}{0pt} % Remove header underlines
\renewcommand{\footrulewidth}{0pt} % Remove footer underlines
\setlength{\headheight}{13.6pt} % Customize the height of the header

\numberwithin{equation}{section} % Number equations within sections (i.e. 1.1, 1.2, 2.1, 2.2 instead of 1, 2, 3, 4)
\numberwithin{figure}{section} % Number figures within sections (i.e. 1.1, 1.2, 2.1, 2.2 instead of 1, 2, 3, 4)
\numberwithin{table}{section} % Number tables within sections (i.e. 1.1, 1.2, 2.1, 2.2 instead of 1, 2, 3, 4)

\newcommand{\R}{\mathbb{R}}
\newcommand{\Q}{\mathbb{Q}}
\newcommand{\N}{\mathbb{N}}
\newcommand{\C}{\mathbb{C}}

\newtheorem{lemma}{Lemma}
%----------------------------------------------------------------------------------------
%	TITLE SECTION
%----------------------------------------------------------------------------------------

\newcommand{\horrule}[1]{\rule{\linewidth}{#1}} % Create horizontal rule command with 1 argument of height

\title{	
\normalfont \normalsize 
\textsc{Analysis} \\ [25pt] % Your university, school and/or department name(s)
\horrule{0.5pt} \\[0.4cm] % Thin top horizontal rule
\huge Problem Set 5 \\ % The assignment title
\horrule{2pt} \\[0.5cm] % Thick bottom horizontal rule
}

\author{Daniel Halmrast} % Your name

\date{\normalsize\today} % Today's date or a custom date

\begin{document}

\maketitle % Print the title

% Problems

\section{Problem 1}
Show that for a Banach space $X$, $X^*$ is weak-$*$ sequentially complete.
Formulate an analagous statement for $X$ being weak sequentially complete, and
show where the proof collapses.
\\
\\
\begin{proof}
    We first show that $X^*$ is weak-$*$ sequentially complete. So, let $\phi_j$
    be a sequence for which $\langle \phi_j,x\rangle$ converges for all $x\in X$. We can
    define the limit to be $\phi := x\mapsto \lim \phi_j(x)$. Now, we just have
    to show that $\phi$ is linear and bounded.

    Linearity of $\phi$ follows directly from its definition, and the linearity
    of limits. That is, for $x,y\in X$,
    \[
\begin{aligned}
    \phi(x+y) &= \lim \phi_j(x+y)\\
    &= \lim (\phi_j(x) + \phi_j(y))\\
    &= \lim \phi_j(x) + \lim \phi_j(y)\\
    &= \phi(x) + \phi(y)
\end{aligned}
    \]
    and for $\alpha\in\C$,
    \[
\begin{aligned}
    \phi(\alpha x) &= \lim \phi_j(\alpha x)\\
    &= \lim \alpha\phi_j(x)\\
    &= \alpha\lim\phi_j(x)\\
    &=\alpha\phi(x)
\end{aligned}
    \]
    and thus $\phi$ is linear.

    Now, we just need to show that $\phi$ is bounded. This is clear from the
    uniform boundedness principle applied to the family $\phi_j$ of operators
    from $X$ to $\C$. That is, since for each $x\in X$, we have that
    \[
        \sup_j \|\phi_j(x)\| < \infty
    \]
    (because $\phi_j(x)$ converges for each fixed $x$), it follows then by the
    principle of uniform boundedness that
    \[
        \sup_j \{\|\phi_j\|\} < \infty
    \]
    as well. Let $C$ be the uniform bound on the norms of $\phi_j$. Then, it
    follows that for all $x$ and for all $j$,
    \[
        \|\phi_j(x)\|\leq C\|x\|
    \]
    Thus, it follows that the limit is also bound by this inequality
    \[
        \|\phi(x)\| = \|\lim\phi_j(x)\| = \lim\|\phi_j(x)\| \leq C\|x\|
    \]
    and $\phi$ is bounded as desired.

    Thus, $\phi_j$ converges in weak-$*$ to $\phi$, which is in $X^*$ as we just
    proved.
    \\
    \\
    Now, the analagous statement for $X$ being weak sequentially complete is as
    follows:
    Let $x_j$ be a sequence in $X$ for which $\phi(x_j)$ converges for all
    $\phi\in X^*$. For $X$ to be weak sequentially complete, there must be some
    $x\in X$ for which $x_j$ converges weakly to $x$.

    The proof fails when we attempt to apply PUB. In particular, we can define
    the limit of $x_j$ by treating it as a sequence of linear functionals on
    $X^*$, but we are not guaranteed that the limit of this sequence stays in
    the embedding of $X$ to $X^{**}$. For example, taking $X = L^1(\R)$, and
    $x_j = j\chi_{[-\frac{1}{j},\frac{1}{j}]}$, this converges weakly to the
    delta distribution, which is not an $L^1$ function.
\end{proof}

\newpage
\section{Problem 2}
Let $X$ be Banach. Prove that a sequence $\phi_j$ in $X^*$ converges weak-$*$ if
and only if it is strongly bounded and there exists a dense subset $E\subset X$
for which $\phi_j(u)$ converges for all $u\in X$.
\\
\\
\begin{proof}
    ($\implies$)
    Suppose first that $\phi_j\to\phi$ in weak-$*$. Then, the principle of
    uniform boundedness guarantees that the set $\{\phi_j\}$ is strongly bounded.
    Furthermore, since $\phi_j(x)\to\phi(x) < \infty$ for all $x$, setting $E=X$
    finishes the proof.

    ($\impliedby$)
    Suppose $\phi_j$ is such that it is strongly bounded, and there exists a
    dense subset $E\subset X$ for which $\phi_j(u)$ converges for all $u\in E$.
    All we need to show is that $\phi(x)$ converges for all $x\in X$, and by
    problem 1 we will know that $\phi_j$ converges weakly to some $\phi\in X^*$.
    
    So, we will show that $\phi_j(x)$ is a Cauchy sequence for all $x$. So, fix
    $x\in X$, and let $u_n$ be a sequence in $E$ converging to $x$. Then, for
    fixed $\epsilon>0$, fix $m$ so that $|u_m - x|<\epsilon$, and fix $N$ so
    that for all $j,k > N$, $|\phi_j(u_m)-\phi_k(u_m)|<\epsilon$. Finally, let
    $C$ be the uniform bound on $\|\phi_j\|$. Now, we have
    that
    \[
\begin{aligned}
    |\phi_j(x) - \phi_k(x)| &= |\phi_j(x)-\phi_j(u_m)+\phi_j(u_m) -\phi_k(x) +
    \phi_k(u_m) - \phi_k(u_m)|\\
    &\leq |\phi_j(x)-\phi_j(u_m)| + |\phi_k(x)-\phi_k(u_m)| + |\phi_j(u_m) -
    \phi_k(u_m)|\\
    &= |\phi_j(x-u_m)| + |\phi_k(x-u_m)| + |\phi_j(u_m)-\phi_k(u_m)|\\
&\leq (\|\phi_j\| + \|\phi_k\|)|x-u_m| + \epsilon\\
&\leq (\|\phi_j\| + \|\phi_k\|)\epsilon + \epsilon\\
&\leq 2C\epsilon + \epsilon
\end{aligned}
    \]
    and so the sequence is Cauchy, as desired. Thus, $\phi_j(x)$ converges for
    all $x$, and so $\phi_j$ converges in weak-$*$ as desired.
\end{proof}

\newpage
\section*{Problem 3}
Let $d\phi_n(x) = \cos(\pi n x)d\lambda^1(x)$. Prove that
\[
    \int_I gd\phi_n \to 0
\]
for all $g\in C^1(I)$. Prove that $\phi_n$ converges in weak-$*$ as measures in
$C(I)^*$.
\\
\\
\begin{proof}
    For the first part, we integrate by parts to find that
    \[
\begin{aligned}
    |\int_I g\cos(n\pi x)dx| &= |\frac{g\sin(\pi nx)}{n\pi}|_{\partial I} -
    \int_I \frac{\sin(\pi nx)}{n\pi}g'dx|\\
    &=|\frac{1}{\pi n}\int_I\sin(\pi nx)g'dx|\\
    &\leq |\frac{1}{n\pi}\int_I g'dx|
\end{aligned}
    \]
    which clearly goes to zero, since $\int_Ig'dx$ is constant.
    \\
    \\
    For the second part, note that the Weierstrauss approximation theorem says
    that the polynomials on $I$ are dense in $C(I)$. Now, since every polynomial
    is in $C^1(I)$, we know that for every polynomial,
    \[
        \int_Ip(x)d\phi_n(x) \to 0
    \]
    from the first part of this problem. Furthermore, $\{\phi_n\}$ is strongly
    bounded, since
    \[
        \|\phi_n(f)\| = |\int_If(x)\cos(n\pi x)dx| \leq |\int_I f(x)dx| = \|f\|
    \]
    so applying problem 2, we see that $\phi_n\to 0$ weakly, since they are
    strongly bounded, and converge on a dense subset of $C(I)$. 
\end{proof}

\newpage
\section*{Problem 4}
Let $(x_n)$ be a sequence in $\ell^1$ such that $x_n\to x$ weakly, and
$\|x_n\|\to\|x\|$. Prove that $x_n\to x$ strongly.
\\
\\
\begin{proof}
    For notation, we will use $f_n := x_n$ and $f:=x$. Now, we first show that
    $f_n\to f$ pointwise. This is clear, since $f_n\to f$ weakly implies that
    for each $y\in\ell^{\infty}$, $y(f_n)\to y(f)$. Specifically, by letting $y$
    be the basis vectors $e_m = \delta(x-m)$, we see that
    \[
        e_m(f_n) = f_n(m) \to e_m(f) = f(m)
    \]
    for all $m$. Thus, $f_n$ converges pointwise to $f$.

    Now, we let $g_n = |f|+|f_n| - |f_n-f|$. Note that $g_n$ is positive by the
    triangle inequality. Now, Fatou's lemma implies that
    \[
        \int\lim\inf g_n \leq \lim\inf\int g_n
    \]
    and so
    \[
\begin{aligned}
    \int\lim\inf(|f|+|f_n|+|f_n-f|) &\leq \lim\inf\int|f| + \lim\inf\int|f_n| +
    \lim\inf(-\int|f_n-f|)\\
    \int(|f|+|f|) &\leq |f| + |f| - \lim\sup(\int|f_n-f|)\\
    0&\leq-\lim\sup\|f_n-f\|
\end{aligned}
    \]
    and since $\|f_n-f\|$ is always positive, $-\lim\sup\|f_n-f\|\leq 0$ which
    forces $\lim\sup\|f_n-f\| = 0$ and thus $f_n$ converges strongly to $f$.
\end{proof}

\newpage
\section*{Problem 5}
Show that the closed unit ball $\bar{B}(X)$ in a Banach space $X$ is closed
in the weak topology. Show that the closed unit ball $\bar{B}(X^*)$ in $X^*$ is
closed in the weak-$*$ topology.
\\
\\
\begin{proof}
    In particular, we have to show that any net in $\bar{B}(X)$ that converges,
    converges in $\bar{B}(X)$. So, let $x_{\alpha}$ be a net in $\bar{B}(X)$,
    and suppose $x_{\alpha}\to x$ in the weak topology. We know, then, that for
    all $\phi\in X^*$, $\phi(x_{\alpha})\to\phi(x)$. In particular,
\[
    |\phi(x_{\alpha})|\to|\phi(x)|
\]
    for all $\phi$. Thus,
    \[
        |\phi(x_{\alpha})|\leq\|\phi\|\|x_{\alpha}\|\leq \|\phi\|
    \]
    and since $|\phi(x_{\alpha})|\to|\phi(x)|$, it follows that
    \[
        |\phi(x)|\leq \|\phi\|
    \]
    Now, we also know that
    \[
        \|x\| = \sup_{\|\phi\| = 1}|\phi(x)|
    \]
    and since $|\phi(x)|\leq 1$ for $\|\phi\|=1$, we know that
    \[
    \|x\|\leq 1
    \]
    as well. Thus, any limit point of a net in $\bar{B}(X)$ is also in
    $\bar{B}(X)$, and $\bar{B}(X)$ is closed in the weak topology.
    \\
    \\
    The second part follows in exactly the same way as the first, by noting that
    \[
        \|\phi\| = \sup_{\|x\|=1}|\phi(x)|
    \]
    and that the weak-$*$ topology is the weak topology with respect to $X$.
\end{proof}

\end{document}
