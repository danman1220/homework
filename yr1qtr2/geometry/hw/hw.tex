%%%%%%%%%%%%%%%%%%%%%%%%%%%%%%%%%%%%%%%%%
% Short Sectioned Assignment
% LaTeX Template
% Version 1.0 (5/5/12)
%
% This template has been downloaded from:
% http://www.LaTeXTemplates.com
%
% Original author:
% Frits Wenneker (http://www.howtotex.com)
%
% License:
% CC BY-NC-SA 3.0 (http://creativecommons.org/licenses/by-nc-sa/3.0/)
%
%%%%%%%%%%%%%%%%%%%%%%%%%%%%%%%%%%%%%%%%%

%----------------------------------------------------------------------------------------
%	PACKAGES AND OTHER DOCUMENT CONFIGURATIONS
%----------------------------------------------------------------------------------------

\documentclass[fontsize=11pt]{scrartcl} % 11pt font size

\usepackage[T1]{fontenc} % Use 8-bit encoding that has 256 glyphs
\usepackage[english]{babel} % English language/hyphenation
\usepackage{amsmath,amsfonts,amsthm} % Math packages
\usepackage{mathrsfs}
\usepackage{tikz-cd}

\usepackage[margin=1in]{geometry}

\usepackage{sectsty} % Allows customizing section commands
\allsectionsfont{\centering \normalfont\scshape} % Make all sections centered, the default font and small caps

\usepackage{fancyhdr} % Custom headers and footers
\pagestyle{fancyplain} % Makes all pages in the document conform to the custom headers and footers
\fancyhead{} % No page header - if you want one, create it in the same way as the footers below
\fancyfoot[L]{} % Empty left footer
\fancyfoot[C]{} % Empty center footer
\fancyfoot[R]{\thepage} % Page numbering for right footer
\renewcommand{\headrulewidth}{0pt} % Remove header underlines
\renewcommand{\footrulewidth}{0pt} % Remove footer underlines
\setlength{\headheight}{13.6pt} % Customize the height of the header

\numberwithin{equation}{section} % Number equations within sections (i.e. 1.1, 1.2, 2.1, 2.2 instead of 1, 2, 3, 4)
\numberwithin{figure}{section} % Number figures within sections (i.e. 1.1, 1.2, 2.1, 2.2 instead of 1, 2, 3, 4)
\numberwithin{table}{section} % Number tables within sections (i.e. 1.1, 1.2, 2.1, 2.2 instead of 1, 2, 3, 4)

\newcommand{\R}{\mathbb{R}}
\newcommand{\Q}{\mathbb{Q}}
\newcommand{\N}{\mathbb{N}}
\newcommand{\C}{\mathbb{C}}
\newcommand{\Z}{\mathbb{Z}}

\newcommand{\Hom}{\text{Hom}}

\newtheorem{lemma}{Lemma}
%----------------------------------------------------------------------------------------
%	TITLE SECTION
%----------------------------------------------------------------------------------------

\newcommand{\horrule}[1]{\rule{\linewidth}{#1}} % Create horizontal rule command with 1 argument of height

\title{	
\normalfont \normalsize 
\textsc{Geometry} \\ [25pt] % Your university, school and/or department name(s)
\horrule{0.5pt} \\[0.4cm] % Thin top horizontal rule
\huge Homework \\ % The assignment title
\horrule{2pt} \\[0.5cm] % Thick bottom horizontal rule
}

\author{Daniel Halmrast} % Your name

\date{\normalsize\today} % Today's date or a custom date

\begin{document}

\maketitle % Print the title

% Problems
\section*{Problem 1}
Prove that $V^{**}\cong V$ for a finite dimensional vector space $V$.
\\
\\
\begin{proof}
    The isomorphism is given by
    \[
        \begin{aligned}
        \Phi:V\to V^{**}\\
        \Phi(v) = ev_v = (\phi\mapsto \phi(v))
    \end{aligned}
    \]
    First, we observe that this map is linear. Indeed, for $v_1,v_2\in V$ and
    $\alpha,\beta$ a scalar, we have
    \[
\begin{aligned}
    \Phi(\alpha v_1 + \beta v_2)(\phi) &= \phi(\alpha v_1 + \beta v_2)\\
    &=\alpha \phi(v_1) + \beta\phi(v_2)\\
    &=\alpha\Phi(v_1)(\phi) + \beta\Phi(v_2)(\phi)
\end{aligned}
    \]
    as desired.

    We need to show this is surjective and injective. Injectivity of $\Phi$ is
    easily shown by examining the kernel of $\Phi$. Suppose $v$ is such that
    $ev_v(\phi)= \phi(v)=0$ for all $\phi\in V^*$. Then, since $V^*$ separates points of
    $V$, it follows that $v=0$. Thus, the kernel is trivial, as desired.

    Now, we need to show this map is surjective. To do so, we appeal to $v$
    being finite-dimensional, and let $\{e_i\}$ be a basis for $V$ with dual
    basis $\{\omega^i\}$. Then, let $x\in V^{**}$. Define a corresponding vector
    $\tilde{x} = \sum_ix(\omega^i)e_i$. Then
    \[
\begin{aligned}
    \Phi(\tilde{x})(\phi) &= \phi(\tilde{x})\\
    &= \sum_i x(\omega^i)\phi(e_i)\\
    &= \sum_i x(\omega^i\phi(e_i))\\
    &= x(\phi)
\end{aligned}
    \]
    as desired. Thus, $\Phi$ is a linear isomorphism.
\end{proof}

\section*{Problem 2}
Prove that for $V$ a vector space with basis $v_i$ and dual basis $v^i$, the set
\[
    \{v^i\otimes v^j\ |\ 1\leq i,j\leq n\}
\]
forms a basis for $V^*\otimes V^*$.
\\
\\
\begin{proof}
    For this, we will show that the vector space
\[
    W = \text{span}(\{v^i\otimes v^j\ |\ 1\leq i,j\leq n\})
\]
satisfies the universal property of
    tensor products. That is, we wish to show that for each bilinear map
    $h:V^*\times V^*\to U$ for some vector space $U$, there is a unique linear
    map $\tilde{h}:W\to U$ such that the diagram
    \[
        \begin{tikzcd}
            V^*\times V^* \arrow[d, "\otimes"] \arrow[r, "h"] & U \\
            W \arrow[ru, "\tilde{h}", dotted] & 
        \end{tikzcd}
    \]
    commutes. We will guess that $\otimes:V^*\times V^*\to W$ is given as
    \[
        \otimes(a_iv^i,b_jv^j) = a_ib_jv^i\otimes v^j
    \]

    So, let $h:V^*\times V^*\to U$ be a bilinear map. Define $\tilde{h}:W\to U$
    as 
    \[
        \tilde{h}(\sum_{i,j}a_{ij}v^i\otimes v^j)
        =a_{ij}h(v^i,v^j)
    \]
    Then it is clear that the diagram
    \[
        \begin{tikzcd}
            V^*\times V^* \arrow[d, "\otimes"] \arrow[r, "h"] & U \\
            W \arrow[ru, "\tilde{h}", dotted] & 
        \end{tikzcd}
    \]
    commutes, since
    \[
\begin{aligned}
    h(a_iv^i,b_jv^j) &= a_ib_jh(v^i,v^j)\\
    \tilde{h}\circ\otimes(a_iv^i,b_jv^j) &= \tilde{h}(a_ib_jv^i\otimes v^j)\\
    &= a_ib_jh(v^i,v^j)
\end{aligned}
    \]
    It should be clear from construction that $\tilde{h}$ is unique.

    Thus, since every bilinear map from $V^*\times v^*$ factors through $W$, $W$
    satisfies the universal property of tensor products, and is isomorphic to
    $V^*\otimes V^*$. Thus, the set $\{v^i\otimes v^j\}$ forms a basis for
    $V^*\otimes V^*$ as desired.
\end{proof}

\section*{Problem 3}
Show that $dvol = \wedge_i\omega^i = \sqrt{|g|}dx^n$.
\\
\\
\begin{proof}
    Recall that $dx^n = \wedge_{i=1}^ndx^i$, and our manifold is
    $n$-dimensional.

    Recall that for an $n$-fold wedge product $\wedge_{i=1}^n v^i$, we have (for
    $v^i = a_j^i\omega^j$ in an orthonormal frame)
    \[
        \wedge_{i=1}^n v^i = |\det(a_j^1,\dots,a_j^n)|\wedge_{i=1}^n\omega^i
        =\sqrt{\det(A^tA)}\wedge_{i=1}^n\omega^i
    \]
    for $A$ the matrix with columns $a^i$. We can apply this to $v^i = dx^i$ to
    get the desired result.
\end{proof}

\section*{Problem 4}
Show that the definition of the integral of a top degree form on a single chart
is independent of choice of coordinates.
\\
\\
\begin{proof}
    Recall the definition of the integral of a top degree differential form on a
    compact set $K$ in a single coordinate frame is
    \[
        \int_K\omega = \int_{\phi(K)}f\circ\phi^{-1}dx^n
    \]
    for $\omega = fdx^1\wedge\dots\wedge dx^n$.

    We also have the change-of-coordinates formula for a diffeomorphism
    $F:\Omega_1\to\Omega_2$ as
    \[
        \int_{\Omega_2}fdy^n = \int_{\Omega_1}f\circ F|J_F|dx^n
    \]
    where $J_F$ is the jacobian of $F$.

    So, let $\phi:M\to\R^n$ be the original coordinate system, and let
    $\psi:M\to\R^n$ be another coordinate system covering $K$. Then, we have the
    diffeomorphism $F = \psi\circ\phi^{-1}$ and we can apply this to get
    \[
        \int_{\psi(K)}g\circ psi^{-1}dy^n = \int_{\phi(K)} g\circ\psi^{-1}\circ F|J_F|dx^n
    \]
    which is just
    \[
        \begin{aligned}
            \int_{\phi(K)} g\circ\psi^{-1}\circ F|J_F|dx^n &= \int_{\phi(K)}
            g\circ \psi^{-1}\circ 
            \psi\circ\phi^{-1}|J_F|dx^n\\
            &= \int_{\phi(K)} g\circ \phi^{-1} |J_F|dx^n\\
            &= \int_K g|J_F|dx^n\\
            &= \int_K\omega
        \end{aligned}
    \]
    where we used the fact that $\omega = g dy^1\wedge\dots\wedge dy^n =
    g|J_F|dx^1\wedge\dots\wedge dx^n$ since $|J_F|=\det(J_F)$ on a two
    positively oriented charts.

    Thus, the two integrals agree.
\end{proof}

\section*{Problem 5}
Prove that a manifold is orientable if and only if it admits a nowhere vanishing
top degree form.
\\
\\
\begin{proof}
    Suppose $M$ is an orientable manifold. That is, there exists an atlas
    $\{U_{\alpha},\phi_{\alpha}\}$ of $M$ for which the Jacobian of each
    transition map has positive determinant. Let $\{\psi_{\alpha}\}$ be a
    partition of unity subordinate to the atlas $U_\alpha$. Then, define
    \[
        \omega = \sum_{\alpha}\psi_{\alpha}dx_{\alpha}^i\wedge\dots\wedge
        dx_{\alpha}^n
    \]
    where $x_{\alpha}^i$ are the coordinate functions on $U_\alpha$. We claim
    that $\omega$ is a nowhere-vanishing form. Clearly, if $p$ is a point in $M$
    contained in only one chart, then $\omega_p = dx_p^1\wedge\dots\wedge
    dx_p^n$ and does not vanish. If $p$ is such that it is contained in more
    than one chart, then $\omega$ at $p$ is the sum of positive terms (since
        each coordinate system is positive, we have $dx^n = \det(J)dy^n$ and
    $\det(J)$ is always positive) and does not vanish.
    \\
    \\
    Suppose instead that $M$ admits a nowhere-vanishing top degree form
    $\omega$. Then, let $\{U_{\alpha},\phi_{\alpha}\}$ be an atlas of $M$. For
    $x_{\alpha}^i$ the coordinate functions for $\phi_{\alpha}$, define a new
    coordinate system to be such that if $\omega$ is expressed as
    \[
    \omega = fdx_{\alpha}^1\wedge\dots\wedge dx_{\alpha}^n
    \]
    then $f$ is positive. This is done by setting $x_{\alpha}^1$ to its negative
    if $f$ is negative on that chart. Note that since $\omega$ never vanishes,
    we know that $f$ will be either entirely positive or entirely negative on a
    chart. Thus, such a choice can be made consistently.

    Then, the modified atlas is a positive coordinate chart for $M$, which is
    easily verified, since
    \[
    \omega = fdx^n = f\det(J)dy^n
    \]
    and $\omega$ always has positive coefficient, so $\det(J)$ is positive.
\end{proof}

\section*{Problem 6}
Show that the topology of $M$ coincides with the metric topology 
\[
    d_g(x,y) = \inf_{\gamma\in C^{\infty}(I,M)} \{L(\gamma)\ |\
    \gamma(0)=x,\gamma(1)=y\}
\]
where $L(\gamma)$ is the total length of $\gamma$ defined by
\[
    L(\gamma) = \int_Ig(\gamma',\gamma')dt
\]

\begin{proof}
    First, we observe the following: for $g$ a Riemannian metric, and $\gamma$ a
    curve contained entirely in a single coordinate chart $\phi$, there exist
    constants $c,C$ such that
    \[
        cL_{\R^n}(\gamma) \leq L_g(\gamma) \leq CL_{\R^n}(\gamma)
    \]
    where $L_{\R^n}(\gamma)$ is the length of $\phi\circ\gamma$ using the
    euclidean metric on $\R^n$.  This follows from the fact that the
    metric induced by $g$ along $\phi^{-1}$ defines a norm on $\R^n$, and all
    norms are equivalent. That is,
    \[
        k\|v\|_{\R^n} \leq \|v\|_{\phi^{-1*}g}\leq K\|v\|_{\R^n}
    \]
    for constants $k,K$. Thus, the lengths (defined in terms of integrals of the
    metric) follow the same inequality. 

    It should also be clear that the metrics induced by $\R^n$ and $g$ are
    equivalent as well. To see this, note that for any $x,y\in \R^n$, we have
    \[
        \begin{aligned}
        cd_{\R^n}(x,y) &= \inf_{\gamma(0)=x,\gamma(1)=y}cL_{\R^n}(\gamma)\\
        &\leq \inf_{\gamma(0)=x,\gamma(1)=y}L_g(\gamma)\\
        &=d_g(x,y)
    \end{aligned}
    \]
    (the first equality is proved in the next problem)
    and similarly for $d_g(x,y)\leq Cd_{\R^n}(x,y)$. This shows that the two
    topologies induced by the two metrics are equal.

    Equivalence of the two topologies follows immediately. We can show that for
    $U$ open in the manifold topology, and $x\in U$, there is a neighborhood $V$
    of $x$ in the metric topology contained in $U$. Simply take a coordinate
    ball $V_{\varepsilon}(x)$ of small enough radius to be contained in a single
    coordinate chart $\phi$. That is, the domain of $\phi$ contains
    $V_{\varepsilon}(x)$. Then, we know from the above observation that
    $\phi(V_{\varepsilon}(x))$ is open in $\R^n$ in the standard topology, and
    thus is open with respect to the pullback of $d_g$ along $\phi^{-1}$. Thus,
    $V_{\varepsilon}(x)$ is also open in the metric topology induced by $d_g$ on
    $M$. The same argument with the two topologies switched completes the
    argument that both topologies are equal.
\end{proof}

\section*{Problem 7}
Show that $\|a-b\|_{\R^n}$ is $d_{\R^n}(a,b) = \inf_{\gamma}L_{\R^n}(\gamma)$.
\\
\\
\begin{proof}
    This result follows from standard variational calculus on the functional
    $L_{\R^n}(\gamma)$. 

    Let's minimize the functional $L(\gamma)$ by varying the path $\gamma$. We
    do this by setting the variation to zero. That is, $L(\gamma)$ is maximized
    for $\gamma$ that makes $\delta L=0$. We calculate
    \[
\begin{aligned}
    \delta L &= \int \delta(\sqrt(g(\gamma',\gamma')))dt\\
    &=\int \frac{1}{2\sqrt{g(\gamma',\gamma')}}\delta g(\gamma',\gamma')dt
\end{aligned}
    \]
    Since arc length is independent of parameterization, we can take the unit
    speed parameterization of $\gamma$, so that $g(\gamma',\gamma')=0$. Then, we
    have
    \[
        \begin{aligned}
            \delta L &= \int \delta g(\gamma',\gamma')dt\\
            &= \int \delta (g_{ab}\partial_t\gamma^a\partial_t\gamma^b)dt\\
            &=\int g_{ab}\delta(\partial_t\gamma^a)\partial_t\gamma^b +
            g_{ab}\delta(\partial_t\gamma^b)\partial_t\gamma^a dt
            &= 2\int g_{ab}\partial_t(\delta\gamma^a)\partial_t\gamma^b dt
        \end{aligned}
    \]
    integrating by parts (and tossing boundary terms since the endpoints of
    $\gamma$ do not vary) and noting $g_{ab} = \delta_{ab}$ in $\R^n$ yields
    \[
        \delta L = -2\int \partial_t^2\gamma^a\delta\gamma^bdt
    \]
    which holds only if $\partial_t^2\gamma^a = 0$ for all $a$. Thus, the
    minimal length path from a point $p$ to a point $q$ is the straight line
    from $p$ to $q$.

    So, for $\gamma$ the straight line from $p$ to $q$, $L(\gamma) = \|\gamma\|
    = \|p-q\|$ as desired.
\end{proof}

\section*{Problem 8}
Define the Levi-Civita connection as the unique connection such that
\[
    \nabla_X Y-\nabla_YX = [X,Y]
\]
and
\[
    Xg(Y,Z) = g(\nabla_XY,Z) + g(Y,\nabla_XZ)
\]
Show that this is indeed a connection on $M$.
\\
\\
\begin{proof}
    For this proof, we will denote the Levi-Civita connection as $D$.

    We need to show that $D_XY$ is function linear in $X$, scalar linear in
    $Y$, and satisfies the Leibniz rule
    \[
        D_X(fY) = (Xf)Y + fD_XY
    \]

    Recall from class that by utilizing the two properties above, we see that
    \[
        2g(D_XY,Z) = Xg(Y,Z) + Yg(Z,X) - Zg(X,Y) + g([X,Y],Z) -g([Y,Z],X) -
        g([X,Z],Y)
    \]
    which uniquely determines the connection. Now, we just need to show that
    this definition satisfies the definition of a connection. To that end, let
    $f\in C^{\infty}(M)$. We calculate
    \[
\begin{aligned}
    2g(D_{fX}Y,Z) &= fXg(Y,Z) + Yg(Z,fX) - Zg(fX,Y)\\
        &+ g([fX,Y],Z) -g([Y,Z],fX) - g([fX,Z],Y)\\
        &=fXg(Y,Z) + Yg(Z,fX) - Zg(fX,Y)\\
        &- g([Y,fX],Z) - g([Y,Z],fX) + g([Z,fX],Y)\\
        &=fXg(Y,Z) + (Yf)g(Z,X) + fYg(Z,X) - (Zf)g(X,Y) - fZg(X,Y)\\
        &- g((Yf)X,Z) - g(f[Y,X],Z) - g([Y,Z],fX) + g((Zf)X,Y) + g(f[Z,X],Y)\\
        &=fXg(Y,Z)  + (Yf)g(Z,X) + fYg(Z,X) - (Zf)g(X,Y) - fZg(X,Y)\\
        &- g((Yf)X,Z) + g(f[X,Y],Z) - g([Y,Z],fX) + g((Zf)X,Y) - g(f[X,Z],Y)\\
        &=(fX)g(Y,Z) + (Yf)g(X,Z) - (Yf)g(X,Z) - (Zf)g(X,Y) + (Zf)g(X,Y)\\
        &+ fYg(X,Z) -fZg(X,Y) + g(f[X,Y],Z) -g([Y,Z],fX) -g(f[X,Z],Y)\\
        &= f\left\{ Xg(Y,Z) + Yg(X,Z) -Zg(X,Y) + g([X,Y],Z) - g([Y,Z],X) -
        g([X,Z],Y) \right\}\\
        &=fg(D_XY,Z) = g(fD_XY,Z)
\end{aligned}
    \]
    and thus $D$ is $C^{\infty}(M)$-linear in $X$.

    Linearity in $Y$ follows immediately from the fact that the Lie bracket and
    $g$ are both scalar linear.

    Finally, we show that this satisfies the Leibniz rule. This is done by
    direct calculation.
    \[
        \begin{aligned}
            2g(D_X(fY),Z) &= Xg(fY,Z) + fYg(Z,X) - Zg(X,fY)\\
            &+ g([X,fY],Z) - g([fY,Z],X) - g([X,Z],fY)\\
            &= (Xf)g(Y,Z) + fXg(Y,Z) + fYg(X,Z) - (Zf)g(X,Y)-fZg(X,Y)\\
            &+g(f[X,Y] +(Xf)Y,Z) - g(f[Y,Z] -(Zf)g(Y,X) - g([X,Z],fY)\\
            &= fXg(Y,Z) + fYg(X,Z) - fZg(X,Y)\\
            &+ fg([X,Y],Z) -fg([Y,Z],X) - fg([X,Z],Y)\\
            &+(Xf)g(Y,Z) + (Xf)g(Y,Z) - (Zf)g(X,Y) + (Zf)g(X,Y)\\
            &= 2fg(D_XY,Z) + 2(Xf)g(Y,Z)\\
            &= 2g((Xf)Y + D_XY,Z)
        \end{aligned}
    \]
    as desired.

    Thus, $D$ is a connection.
\end{proof}

\section*{Problem 9}
Construct a one dimensional smooth bump function on $\R$.
\\
\\
\begin{proof}
    Let
    \[
        f(x) =
        \begin{cases}
            \exp(\frac{-1}{t}), &t>0\\
            0 &\text{else}
        \end{cases}
    \]
    and let
    \[
        g(x) = \frac{f(x)}{f(x) + f(1-x)}
    \]
    which is $1$ for $x\geq 1$ and $0$ for $x\leq 0$. Finally, set
\[
    h(x) = g(x+2)g(2-x)
\]
which is zero outside of $[-2,2]$ and $1$ inside $[-1,1]$ as desired.
\end{proof}

\section*{Problem 10}
Show that the Christoffel symbols for the Levi-Civita connection are
\[
    \Gamma_{ij}^k = \frac{1}{2}g^{kl}\left( g_{il,j}+g_{jl,i}-g_{ij,l} \right)
\]

\begin{proof}
    Recall that a connection is completely described by the Christoffel symbols
    as
    \[
        \nabla_iv^k = \partial_iv^k + \Gamma_{ij}^kv^j
    \]
    and the conditions for a Levi-Civita connection are
    \[
        g(\nabla_XY,Z) = \frac{1}{2} \left( Xg(Y,Z) + Yg(Z,X) - Zg(X,Y)
        + g([X,Y],Z) -g([Y,Z],X) - g([X,Z],Y)\right)
    \]
    Setting $X=\partial_i$, $Y=\partial_j$, and $Z=\partial_k$ and noting that
    $[\partial_i,\partial_j]=0$ for all $i$ and $j$, we see that
    \[
        \begin{aligned}
        g(\nabla_i\partial_j,\partial_k) &= \frac{1}{2}\left(
            \partial_ig(\partial_j,\partial_k) +
            \partial_jg(\partial_k,\partial_i) -
            \partial_kg(\partial_i,\partial_j \right)\\
            &= \frac{1}{2}\left( 
            g_{jk,i} + g_{ik,j} - g_{ij,k}\right)\\
            g_{lk}(\nabla_i\partial_j)^l\partial_k^k 
            &= \frac{1}{2}\left( 
            g_{jk,i} + g_{ik,j} - g_{ij,k}\right)\\
            (\nabla_i\partial_j)^l &= \Gamma_{ij}^l
            = g^{lk}\frac{1}{2}\left( 
            g_{jk,i} + g_{ik,j} - g_{ij,k}\right)\\
        \end{aligned}
    \]
    as desired.
\end{proof}

\section*{Problem 11}
Prove that
\[
    \partial_tg(X,Y) = g(\nabla_tX,Y) + g(X,\nabla_tY)
\]

\begin{proof}
    Fix an orthonormal frame $P_i$ at $\gamma(0$ for $\gamma$ the curve we are
        differentiating against. This frame can be parallel transported along
        $\gamma$ to form a local orthonormal frame along $\gamma$. Then we have
        \[
        X = x^iP_i
        \]
        and
        \[
        Y = y^P_i
        \]
        from which it follows that
        \[
            \nabla_tX = \partial_tx^iP_i
        \]
        and similar for $Y$. Then
        \[
        \begin{aligned}
            g(\nabla_tX,Y)+g(X,\nabla_tY) &= \sum_i\left( \partial_tx^iy^i +
            \partial_ty^ix^i \right)\\
            &=\partial_t(\sum_i x^iy^i)\\
            &=\partial_tg(X,Y)
        \end{aligned}
        \]
        as desired.
\end{proof}

\section*{Problem 12}
Prove that for a smooth map $F:I^{2}\to M$ with first coordinate $t$ and second
coordinate $s$,
\[
    \nabla_t\partial_sF = \nabla_s\partial_tF
\]

\begin{proof}
    This proof is given explicitly in Do Carmo Chapter 3 lemma 3.4, and will not
    be replicated here.
\end{proof}

\section*{Problem 13}
Prove that
\[
    g((d\exp_p)_{\tilde{\gamma}(t)}(r(t)+n(t)),(d\exp_p)_{\tilde{\gamma}(t)}(r+n))
    = \|r(t)\|^2
\]
if and only if $\tilde{\gamma}(t)$ is radial.
\\
\\
\begin{proof}
    Recall that the Gauss lemma says that $\exp_p$ is an isometry on its normal
    ball. Thus, recalling from the notes that
    \[
    g((d\exp_p)_{\tilde{\gamma}(t)}(r(t)+n(t)),(d\exp_p)_{\tilde{\gamma}(t)}(r+n))
    = \|r(t)\|^2 + \|d\exp_p(n(t))\|^2
    \]
    we see that equality holds if and only if $\|d\exp_p(n(t))\|=0$. However,
    since $\exp_p$ is an isometry,
    \[
        \|d\exp_p(n(t))\| = \|n(t)\|
    \]
    which is zero for all time if and only if $n(t)$ is zero for all time. Thus,
    equality holds when $\tilde{\gamma}$ is a radial geodesic.
\end{proof}

\section*{Problem 14}
Find a counterexample to $\exp_p(B_r(0)) = B_r(p)$ for arbitrary $r$.
\\
\\
\begin{proof}
    %TODO do this
\end{proof}

\section*{Problem 15}
Show that $R_m$ is function linear in the first two components.
\\
\\
\begin{proof}
    It should be clear my the antisymmetry of $R_m$ in the first two components
    that we only need to show the first component is function linear, and the
    second follows immediately.

    So, we compute
    \[
\begin{aligned}
    R(fX,Y)Z &= -\nabla_{fX}\nabla_YZ + \nabla_Y\nabla_{fX}Z + \nabla_{[fX,Y]}Z\\
    &= -f\nabla_X\nabla_YZ + \nabla_Y(f\nabla_XZ) +\nabla_{f[X,Y] -(Yf)X}Z\\
    &= -f\nabla_X\nabla_YZ +\nabla_Y(f)\nabla_XZ + f\nabla_Y\nabla_XZ
    +f\nabla_{[X,Y]}Z - \nabla_{(Yf)X}Z\\
    &= fR(X,Y)Z + (Yf)\nabla_XZ - (Yf)\nabla_XZ\\
    &= fR(X,Y)Z
\end{aligned}
    \]
    as desired.
\end{proof}

\section*{Problem 16}
Show that in Riemannian normal coordinates at $p$, the christoffel symbols
vanish at $p$ and the first derivatives of the metric vanish at $p$. 
\\
\\
\begin{proof}
    We first show the Christoffel symbols at $p$ vanish. So, let $\gamma$ be a
    geodesic with $\gamma(0)=p$. This is given in normal coordinates as
    \[
        \gamma(t) = \exp_p(t(v_1,v_2,\cdots,v_n)) = t(x_1,x_2,\cdots,x_n)
    \]
    Now, we know the geodesics are the solution to the geodesic equation
    \[
        \partial_t^2\gamma^a(t)
        +\Gamma_{bc}^a\partial_t\gamma^b(t)\partial_t\gamma^c(t) = 0
    \]
    In particular, at zero we know that $\partial_t^2\gamma^a=0$, and so
    \[
        \Gamma_{bc}^a\partial_t\gamma^b\partial_t\gamma^c = 0
    \]

    But $\partial_t\gamma^c = x^c$ and so
    \[
        \Gamma_{bc}^ax^bx^c=0
    \]
    and this holds for all $x$ sufficiently small. Thus, $\Gamma_{bc}^a=0$ for
    all $a$, as desired.

    To show the first derivatives of the metric vanish, we use the metric
    compatibility condition of the connection. Namely,
    \[
        \nabla_ig_{jk}=0
    \]
    for all $i,j,k$. Thus, since the Christoffel symbols vanish, we know that
    \[
    \partial_ig_{jk}=0
    \]
    as desired.
\end{proof}

\section*{Problem 17}
Show that the induced inner product on two forms is independent of choice of
orthonormal basis.
\\
\\
\begin{proof}
    Suppose $\{e^j\}$ is an orthonormal basis, and $\{v^j\}$ is some other
    orthonormal basis. We just need to show that the matrix taking $e^i\wedge
    e^j$ to $v^i\wedge v^j$ is orthonormal. So, let $a$ be the matrix such that
    \[
        v^i = a^i_je^j
    \]
    Then, it is clear through rote calculation that $\langle v^i\wedge
    v^j, v^k\wedge v^l\rangle$ for $i<j,k<l$ is zero if $i\neq k,j\neq l$ and
    $1$ otherwise. Thus, the inner product does not depend on choice of
    orthonormal basis, as desired.

\end{proof}

\section*{Problem 18}
Prove that in the product manifold $S^1\times S^1$ the curvature tensor $R_m$ is
identically zero.
\\
\\
\begin{proof}
    Let $\theta$ parameterize the first $S^1$, and $\phi$ the second. The
    product metric is then given by
    \[
        g = d\theta\otimes d\theta + d\phi\otimes d\phi
    \]
    Thus $d\theta,d\phi$ form an orthonormal coframe (and
    $\partial_{\theta},\partial_{\phi}$ form an orthonormal frame) that is
    parallel in the neighborhood of some point $p$. Thus, since $R_m$ at $p$ is
    defined in terms of the covariant derivative around $p$, it follows that
    $R_m=0$ at $p$. Since this can be done at any $p$, the torus is indeed flat,
    as desired.
\end{proof}

\section*{Problem 19}
Prove that for $f:M\to \tilde{M}$ an isometric immersion, if all geodesics of
$M$ are also geodesics of $\tilde{M}$, then $f$ is totally geodesic. That is,
the second fundamental form vanishes.
\\
\\
\begin{proof}
    Suppose $f$ maps geodesics to geodesics. Then,
    \[
        \nabla_t\gamma' = 0 = \tilde{\nabla}_t\gamma'
    \]
    However, at a point, there are geodesics
    in every direction, and so at a point, the covariant derivatives agree.
    Thus, at every point, $B(X,Y)=0$ (since $\tilde{\nabla}_XY = \nabla_XY
    +B(X,Y)$) and so $f$ is totally geodesic.
\end{proof}

\section*{Do Carmo Problem 1.2}
Introduce a metric on $\R^n/{\Z^n}$ such that projection is a local isometry.
Show that this torus is isometric to the flat torus.
\\
\\
\begin{proof}
    Recall that the projection $\pi:\R^n\to\R^n/{\Z^n}$ is a smooth covering
    map. Furthermore, note that the Euclidean metric $g$ is invariant under the
    action of $\Z^n$. That is, for an action $f((x_1,\dots,x_n)) =
    (x_1+m_1,\dots,x_n+m_n)$ we know that
    \[
        f^*g = f^*(\sum_idx^i\otimes dx^i) = \sum_id(x^i\circ f)^2
        = d(x^i+m_i)^2 = x^i\otimes x^i
    \]

    Thus, we can define the metric as
    \[
        \tilde{g}_q(u,v) = g(d(\pi^{-1})_p(u),d(\pi^{-1})_p(v))
    \]
    which is clearly well-defined, since a fiber of $\pi^{-1}(p)$ consists of the
    orbit of $p$ under the action of $\Z^n$, which $g$ is invariant under.

    Consider the diffeomorphism (with $T^n$ parameterized as $\theta_i\in
    [0,1)$)
    \[
        \Phi:T^n\to \R^n/{\Z^n}
    \]
    given by $\Phi(\theta_1,\dots,\theta_n) = [(\theta_1,\dots,\theta_n)]$.

    This is clearly a diffeomorphism, since it has an inverse given by taking a
    point $[x]$ in $\R^n/{Z^n}$ to the element of its orbit in the unit square.
\end{proof}

\section{Do Carmo Problem 2.7}
Let $c$ be an arbitrary parallel of latitude on $S^2$, with $V_0$ a tangent
vector to $S^2$ at some point on $c$. Describe geometrically the parallel
transport of $V_0$ along $c$.
\\
\\
\begin{proof}
    We will show that parallel transport along $c$ in $S^2$ is the same as
    parallel transport along $c$ thought of as a curve in the cone $C$ that lies
    tangent to $S^2$ at $c$.

    In particular, note that the tangent spaces of $S^2$ and $C$ coincide on
    $c$. This means that projection of a vector on $c$ in $\R^3$ is the same
    whether it goes to $TS^2$ or $TC$. Furthermore, since the covariant
    derivative of a vector on $c$ is equal to the ordinary partial derivative in
    $\R^3$ followed by projection into the tangent space, it follows that the
    covariant derivative of $V_0$ along $c$ is the same whether taken in $S^2$
    or $C$. Thus, since parallel transport is defined in terms of the covariant
    derivative, the parallel transport of $V_0$ on $S^2$ coincides with the
    parallel transport of $V_0$ on $C$.

    Now, we note that $C$ is actually flat: by making a suitable radial cut, one
    may flatten $C$ so that it forms a disk with a slice missing, with the
    boundary of the disk coinciding with $c$. Here, parallel transport of $V_0$
    along $c$ is just ordinary translation in $C\subset \R^2$.

    Thus, we have a complete description of the parallel transport of $V_0$
    along $c$. We form the cone $C$ tangential to $c$, and make a cut so that
    $C$ can be isometrically embedded as a subset of $\R^2$. Then, identifying
    $V_0$ with its corresponding tangent vector on $c \subset \partial C$, we
    apply ordinary translation (parallel transport in $\R^2$) to $V_0$ along
    $c$. The result is the parallel vector field $V(t)$ along $c$ in $\R^2$,
    which is identified with the parallel vector field $V(t)\subset TC$.
    Finally, noting that $TC$ and $TS^2$ coincide on $c$, we see that
    $V(t)\subset TS^2$ is the parallel vector field of $V_0$ on $c$.
\end{proof}


\end{document}
