\documentclass[../main.tex]{subfiles}
\begin{document}
\section{Preliminaries}

\begin{hw}
    Prove that $V^{**}\cong V$ for finite-dimensional vector space $V$.
\end{hw}

From this, it is clear that $T_p^*M\otimes T_pM\cong \textrm{Hom}(T_pM,T_pM)$
for a manifold $M$.

Recall the tangent bundle $TM$ is defined as
\[
    TM = \coprod_{p\in M} T_pM
\]
and a vector field on the manifold $M$ is simply a section of the tangent bundle
projection
$\begin{tikzcd}TM\arrow{r}{\pi} &M\end{tikzcd}$. In other words, a vector field
    is a function $f:M\to TM$ such that $\pi\circ f = id$.
Requiring the section to be smooth makes it into a smooth vector field.

We can also do the same thing for the cotangent bundle $T^*M$ to obtain a
covector field.

Now, we can take the tensor product of copies of $TM$ and $T^*M$ to obtain our
tensor bundles, and tensor fields will be sections of these bundles.

Let $(U,\phi)$ be a smooth chart on $M$ with coordinate functions $x^i$,
coordinate vector fields $\partial_i$, and coordinate one-forms $dx^i$. Recall
that $dx^i$ is defined to be the dual basis to $\partial_i$, that is,
\[
    dx^i(\partial_j) = \delta^i_j
\]

Recall also that the exterior derivative of a function $df$ is defined as
\[
    df(v) = v(f)
\]
and this definition applied to the coordinate functions $x^i$ (yielding $dx^i$)
coincides with the definition above. Note that $\partial_i$ form a basis for
$T_pM$ and $dx^i$ form a basis for $T_p^*M$. Tensor products of them, then, form
a basis for the tensor product space.

\begin{hw}
    Prove that, for a vector space $V$ with basis $v_i$, dual basis $v^i$, the
    set
    \[
        \{v^i\otimes v^j\ |\ 1\leq i,j\leq n\}
    \]
    forms a basis for $V^*\otimes V^*$. Here $v^i\otimes v^j(u,v) =
    v^i(u)v^j(v)$.
\end{hw}

\section{Affine Connections}

\subsection{The Metric}

\begin{defn}
    Let $M^n$ be a smooth manifold of dimension $n$. A {\em Riemannian Metric}
    $g$ on $M$ is a rank $(0,2)$ tensor (a section of $T^*M\otimes T^*M$) that
    is symmetric and positive-definite. In other words, $g$ is a rank $(0,2)$
    tensor that restricts to an inner product on the tangent space at every
    point.
\end{defn}

We can express $g$ in local coordinates!
\[
    g_{ij} = g(\partial_i,\partial_j)
\]
or
\[
    g = g_{ij}dx^i\otimes dx^j
\]

\subsection{Integration of Top Degree Differential Forms}
Let $M^n$ be an orientable $n$-dimensional manifold, and $\omega\in
\Omega^n(M)$. Furthermore let $(U,\phi)$ be a positive coordinate chart. On $U$
we have that
\[
    \omega = fdx^1\wedge\ldots\wedge dx^n
\]
for some $f\in C^{\infty}(M)$.

Now, let $K\subset U$ be compact. We define
\[
    \begin{aligned}
        \int_K\omega &= \int_{\phi(K)}\phi^{-1*}\omega\\
                    &= \int_{\phi(K)}f\circ\phi^{-1}
                    \phi^{-1*}dx^1\wedge\ldots\wedge\phi^{-1*}dx^n\\
                    &=\int_{\phi(K)}f\circ\phi^{-1}dx^1\wedge\ldots\wedge dx^n
    \end{aligned}
\]
where the last integral is just the standard integral in $\R^n$.

Is this definition independent of choice of coordinates? Let's check.
Let $(V,\psi)$ be another coordinate chart containing $K$. Then, the integral
with respect to this coordinate system is
\[
    \int_K\omega = \int_{\psi(K)}g\circ\psi^{-1}dy^1\wedge\ldots\wedge dy^n
\]
for $g$ defined as
\[
    \omega = hdy^1\wedge\ldots\wedge dy^n
\]
with coordinate functions $y^i$. The claim is that these integrals are equal.

Consider the change-of-coordinates map $\psi\circ\phi^{-1}$ from the $x^i$ to
the $y^i$ coordinate system. Since $K$ is in both $U$ and $V$, its image
$\phi(K)$ lies in the domain of $\psi\circ\phi^{-1}$.

All that remains is to apply the change of variables to the integrals. Recall
that if one has a diffeomorphism $F:\Omega_1\to\Omega_2$ for compact $\Omega_i$,
one has that
\[
    \int_{\Omega_2}fdy^1\ldots dy^n = \int_{\Omega_1}f\circ F|J_F|dx^1\ldots
    dx^n
\]
where $|J_F|$ is the determinant of the Jacobian matrix for $F$.
\begin{hw}
    Check that the two integrals claimed to be equal are actually equal.
\end{hw}

Now we have an idea for how to integrate $\omega$ on a single chart, let's
extend this. Let $(\eta_i,U_i)$ be a partition of unity of $M$ where each $U_i$
is contained in a single chart on $M$. Then,
\[
    \omega = \sum\omega\eta_i
\]
and we can integrate by extending linearly
\[
    \int_K\omega = \sum\int_{K}\omega\eta_i
\]
where the right hand side has integrals over functions supported in a single
chart, and is well-defined. But is this independent of the choice of partition
of unity? Short answer: yes (Optional homework).

\subsection{Integration on an Orientable Smooth Riemannian Manifold}
Recall that a Riemannian manifold has a volume form
\[
    dvol = \sqrt{|g_{ij}|}dx^1\wedge\ldots\wedge dx^n
\]
which is obtained by taking an orthonormal frame $e_i$ and considering the dual
frame $\omega^i$ defined as
\[
    \omega^ie_j = \delta^i_j
\]
and letting
\[
    dvol = \omega^1\wedge\ldots\wedge\omega^n
\]
This construction is independent of choice of orthonormal frame.
\begin{proof}
Let $\epsilon_i$ be another orthonormal frame with dual frame $\alpha^i$. Then,
$\epsilon_i = a^j_ie_j$ and $\alpha^i = b^i_j\omega^j$ and so
    \[
        \begin{aligned}
            \alpha^1\wedge\ldots\wedge\alpha^n &=
            b^1_{j_1}\omega^{j_1}\wedge\ldots\wedge b^n_{j_n}\omega^{j_n}\\
            &= \sum_{\sigma\in S_n} b^1_{\sigma(1)}\ldots
            b^n_{\sigma(n)}sgn(\sigma)\omega^1\wedge\ldots\wedge\omega^n\\
            &= |b|\omega^1\wedge\ldots\wedge\omega^n\\
            &= \omega^1\wedge\ldots\wedge\omega^n\\
        \end{aligned}
    \]
    where the last line was obtained from the fact that $b$ is the orthogonal
    change-of-basis matrix from $e$ to $\epsilon$.
\end{proof}

Then, we define
\[
    \Vol(K) = \int_Kdvol
\]


\end{document}
