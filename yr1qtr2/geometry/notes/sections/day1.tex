\documentclass[../main.tex]{subfiles}
\begin{document}
\section{Preliminaries}

\begin{hw}
    Prove that $V^{**}\cong V$ for finite-dimensional vector space $V$.
\end{hw}

From this, it is clear that $T_p^*M\otimes T_pM\cong \textrm{Hom}(T_pM,T_pM)$
for a manifold $M$.

Recall the tangent bundle $TM$ is defined as
\[
    TM = \coprod_{p\in M} T_pM
\]
and a vector field on the manifold $M$ is simply a section of the tangent bundle
projection
$\begin{tikzcd}TM\arrow{r}{\pi} &M\end{tikzcd}$. In other words, a vector field
    is a function $f:M\to TM$ such that $\pi\circ f = id$.
Requiring the section to be smooth makes it into a smooth vector field.

We can also do the same thing for the cotangent bundle $T^*M$ to obtain a
covector field.

Now, we can take the tensor product of copies of $TM$ and $T^*M$ to obtain our
tensor bundles, and tensor fields will be sections of these bundles.

Let $(U,\phi)$ be a smooth chart on $M$ with coordinate functions $x^i$,
coordinate vector fields $\partial_i$, and coordinate one-forms $dx^i$. Recall
that $dx^i$ is defined to be the dual basis to $\partial_i$, that is,
\[
    dx^i(\partial_j) = \delta^i_j
\]

Recall also that the exterior derivative of a function $df$ is defined as
\[
    df(v) = v(f)
\]
and this definition applied to the coordinate functions $x^i$ (yielding $dx^i$)
coincides with the definition above. Note that $\partial_i$ form a basis for
$T_pM$ and $dx^i$ form a basis for $T_p^*M$. Tensor products of them, then, form
a basis for the tensor product space.

\begin{hw}
    Prove that, for a vector space $V$ with basis $v_i$, dual basis $v^i$, the
    set
    \[
        \{v^i\otimes v^j\ |\ 1\leq i,j\leq n\}
    \]
    forms a basis for $V^*\otimes V^*$. Here $v^i\otimes v^j(u,v) =
    v^i(u)v^j(v)$.
\end{hw}

\section{Affine Connections}

\subsection{The Metric}

\begin{defn}
    Let $M^n$ be a smooth manifold of dimension $n$. A {\em Riemannian Metric}
    $g$ on $M$ is a rank $(0,2)$ tensor (a section of $T^*M\otimes T^*M$) that
    is symmetric and positive-definite. In other words, $g$ is a rank $(0,2)$
    tensor that restricts to an inner product on the tangent space at every
    point.
\end{defn}

We can express $g$ in local coordinates!
\[
    g_{ij} = g(\partial_i,\partial_j)
\]
or
\[
    g = g_{ij}dx^i\otimes dx^j
\]

\begin{hw}
    Show that the two expressions for $dvol$, namely
    \[
        \begin{aligned}
            dvol &= \wedge_i\omega^i\\
            dvol &= \sqrt{|g|}dx^n
        \end{aligned}
    \]
\end{hw}

\subsection{Integration of Top Degree Differential Forms}
Let $M^n$ be an orientable $n$-dimensional manifold, and $\omega\in
\Omega^n(M)$. Furthermore let $(U,\phi)$ be a positive coordinate chart. On $U$
we have that
\[
    \omega = fdx^1\wedge\ldots\wedge dx^n
\]
for some $f\in C^{\infty}(M)$.

Now, let $K\subset U$ be compact. We define
\[
    \begin{aligned}
        \int_K\omega &= \int_{\phi(K)}\phi^{-1*}\omega\\
                    &= \int_{\phi(K)}f\circ\phi^{-1}
                    \phi^{-1*}dx^1\wedge\ldots\wedge\phi^{-1*}dx^n\\
                    &=\int_{\phi(K)}f\circ\phi^{-1}dx^1\wedge\ldots\wedge dx^n
    \end{aligned}
\]
where the last integral is just the standard integral in $\R^n$.

Is this definition independent of choice of coordinates? Let's check.
Let $(V,\psi)$ be another coordinate chart containing $K$. Then, the integral
with respect to this coordinate system is
\[
    \int_K\omega = \int_{\psi(K)}g\circ\psi^{-1}dy^1\wedge\ldots\wedge dy^n
\]
for $g$ defined as
\[
    \omega = hdy^1\wedge\ldots\wedge dy^n
\]
with coordinate functions $y^i$. The claim is that these integrals are equal.

Consider the change-of-coordinates map $\psi\circ\phi^{-1}$ from the $x^i$ to
the $y^i$ coordinate system. Since $K$ is in both $U$ and $V$, its image
$\phi(K)$ lies in the domain of $\psi\circ\phi^{-1}$.

All that remains is to apply the change of variables to the integrals. Recall
that if one has a diffeomorphism $F:\Omega_1\to\Omega_2$ for compact $\Omega_i$,
one has that
\[
    \int_{\Omega_2}fdy^1\ldots dy^n = \int_{\Omega_1}f\circ F|J_F|dx^1\ldots
    dx^n
\]
where $|J_F|$ is the determinant of the Jacobian matrix for $F$.
\begin{hw}
    Check that the two integrals claimed to be equal are actually equal.
\end{hw}

Now we have an idea for how to integrate $\omega$ on a single chart, let's
extend this. Let $(\eta_i,U_i)$ be a partition of unity of $M$ where each $U_i$
is contained in a single chart on $M$. Then,
\[
    \omega = \sum\omega\eta_i
\]
and we can integrate by extending linearly
\[
    \int_K\omega = \sum\int_{K}\omega\eta_i
\]
where the right hand side has integrals over functions supported in a single
chart, and is well-defined. But is this independent of the choice of partition
of unity? Short answer: yes (Optional homework).

\subsection{Integration on an Orientable Smooth Riemannian Manifold}
Recall that a Riemannian manifold has a volume form
\[
    dvol = \sqrt{|g_{ij}|}dx^1\wedge\ldots\wedge dx^n
\]
which is obtained by taking an orthonormal frame $e_i$ and considering the dual
frame $\omega^i$ defined as
\[
    \omega^ie_j = \delta^i_j
\]
and letting
\[
    dvol = \omega^1\wedge\ldots\wedge\omega^n
\]
This construction is independent of choice of orthonormal frame.
\begin{proof}
Let $\epsilon_i$ be another orthonormal frame with dual frame $\alpha^i$. Then,
$\epsilon_i = a^j_ie_j$ and $\alpha^i = b^i_j\omega^j$ and so
    \[
        \begin{aligned}
            \alpha^1\wedge\ldots\wedge\alpha^n &=
            b^1_{j_1}\omega^{j_1}\wedge\ldots\wedge b^n_{j_n}\omega^{j_n}\\
            &= \sum_{\sigma\in S_n} b^1_{\sigma(1)}\ldots
            b^n_{\sigma(n)}sgn(\sigma)\omega^1\wedge\ldots\wedge\omega^n\\
            &= |b|\omega^1\wedge\ldots\wedge\omega^n\\
            &= \omega^1\wedge\ldots\wedge\omega^n\\
        \end{aligned}
    \]
    where the last line was obtained from the fact that $b$ is the orthogonal
    change-of-basis matrix from $e$ to $\epsilon$.
\end{proof}

Then, we define
\[
    \Vol(K) = \int_Kdvol
\]

\subsection{Integrating a Non-Orientable Manifold}
How do we integrate a manifold that is not orientable? The previous construction
was coordinate-independent only because we chose positive oriented
coordinates...

Let $K\subset U$ be a compact set in a single chart on the manifold. Then, we
can define
\[
    \Vol(K) = \int_K\sqrt{|g_{ij}|}dx^n
\]
Now, this is independent of choice of coordinates, since if $K$ lies in the
intersection of two charts, we can use the Jacobian change-of-variables formula
to show that the two calculations of the volume are equal.

The problem is that $dy^n = det(J_{x\to y})dx^n$ depends also on the sign of
the determinant of the Jacobian.

On an orientable Manifold, we have $dvol\in \Omega^n(M)$ (i.e.
$dvol\in\Gamma(\Lambda^nT^*M)$), and in fact a manifold is orientable if and
only if it admits a nowhere-vanishing top degree form.
\begin{hw}
    Prove that a manifold is orientable if and only if it admits a
    nowhere-vanishing top degree form.
\end{hw}

\subsection{Existence of Metrics}

\begin{theorem}
    On each smooth manifold $M$ there exists smooth Riemannian metrics.
\end{theorem}

\begin{proof}
    Let $(U_i,\phi_i)$ be an atlas of $M$, and $\eta_j$ be a partition of unity
    subordinate to it.
    Then, on each $U_i$ we have a smooth Riemannian metric given by
    \[
        g_i = dx_i^1\otimes dx_i^1+\ldots+dx_i^n\otimes dx_i^n
    \]
    Then, we define
    \[
        g = \sum\eta_ig_i
    \]
\end{proof}

\subsection{Lower-Dimensional Integration on Riemannian Manifolds}
Suppose we want to find the arc length of a curve $\gamma:I\to M$. We can define
the length of $\gamma$ to be
\[
    L(\gamma) = \int_I|\gamma'|dt
\]
where $|\gamma'|$ is the length of the tangent vector with respect to the
metric.

\begin{defn}
    Let $p,q\in M$ be points in a connected manifold $M$. We define the distance
    between $p$ and $q$ to be
    \[
        \inf_{\gamma\in C^{\infty}(I,M)}\{L(\gamma)\ |\ \gamma(0)=p,\gamma(1)=q\}
    \]
\end{defn}
Note that we can relax the condition that $\gamma$ be smooth to $\gamma$ being
only piecewise smooth, since any piecewise smooth curve is uniformly
approximated by smooth curves.

This distance, denoted $d(p,q)$, turns out to metrize the manifold.
\begin{theorem}
    $d(\cdot,\cdot)$ is a metric on $M$, and the metric topology generated by
    $d$ coincides with the topology of $M$.
\end{theorem}

\begin{proof}
    First, we show that $d$ is a metric. Symmetry of $d$ should be obvious,
    since $L(\gamma)=L(-\gamma)$ and the curves from $p$ to $q$ directly
    coincide with curves from $q$ to $p$ via the map $\gamma\mapsto-\gamma$. 

    Now, $d$ is also clearly positive-definite, since the length functional is
    positive-definite.

    It should also be clear that $d(p,q)=0$ if and only if $p=q$. Clearly, if
    $p=q$, then the constant curve $\gamma(t)=p$ has length zero, so $d(p,p) =
    0$. Now, if $p\neq q$, then since $M$ is Hausdorff, they must have positive
    distance from each other. This follows from the second claim that the
    topologies coincide.

    The triangle inequality follows from the fact that given three points
    $p,q,m$, the curve going from $p$ to $m$, and then from $m$ to $q$, is a
    curve from $p$ to $q$, and so $d(p,q)\leq d(p,m)+d(m,q)$ (since it is part
    of the infimum).

    Now, we show that the topologies coincide...
\end{proof}
\begin{hw}
Show that the topology on $M$ coincides with the metric topology from $d$.
\end{hw}

\begin{hw}
    Show that for $(\R^n,g_{\R^n})$, $d(p,q) = \|p-q\|$.
\end{hw}

\subsection{Connections on a Riemannian Manifold}
Let $(M^n,g)$ be a smooth Riemannian Manifold, $X\in\mathfrak{X}(M)$. We wish to
take the derivative of this vector field. Recall that the Lie derivative allows
us to take the derivative of $X$ along another vector field $Y$, however this
operation is not linear with respect to the module of smooth functions. That is,
\[
    L_X(fY) = fL_XY + (Xf)Y
\]
Also, the Lie derivative is not defined for a single point, since it takes into
account the motion of $X$ around any particular point.

What we really want is $\nabla_v$, the covariant derivative.

\begin{defn}
    A {\em Connection} is a map
    \[
        \begin{aligned}
            \nabla: &\mathfrak{X}(M)\times \mathfrak{X}(M)\to\mathfrak{X}(M)
            &(X,Y)\mapsto \nabla_XY
        \end{aligned}
    \]
    such that $\nabla_XY$ is linear in both $X$ with respect to the module
    $C^{\infty}(M)$, scalar linear in $Y$ and satisfies the Leibniz rule
    \[
        \nabla_X(fY) = (Xf)Y + f\nabla_XY
    \]
\end{defn}
\begin{defn}
    A {\em Connection} is the following: for each $p\in M$, we have a map
    $\nabla: T_pM\times C^{\infty}(TM)\to T_pM$ that sends $(v,Y)$ to
    $\nabla_vY$. Such that $\nabla$ is linear in $v$, linear in $Y$, and
    satisfies the Leibniz rule
    \[
        \nabla_v(fY) = (vf)Y_p + f(p)\nabla_v(Y)
    \]
    and, for all $X,Y$ in $\mathfrak{X}(M)$, $\nabla_XY\in\mathfrak{X}(M)$ where
    \[
        (\nabla_XY)_p = \nabla_{X_p}Y
    \]
\end{defn}

Interpreting $\nabla$ as an operator from $\mathfrak{X}(M)$, we see that it
actually adds a covariant index. That is,
\[
    \nabla_{\mu}v^{\nu}
\]
takes in a vector, and outputs a $(1,1)$ tensor.

\begin{ex}
    The directional derivative in $\R^n$ yields a connection. For $v\in
    T_x\R^n$, and $X$ a smooth vector field on $\R^n$, we have
    \[
        D_{(x,v)}X = \partial_tX(x+tv)|_{t=0}
    \]
    and we define $\nabla_vX = (x,D_{(x,v)}X)$
\end{ex}

Now, on $TM$ for a general Riemannian manifold, there are many different
connections. However, given a metric, we have a unique metric compatible,
torsion-free connection called the {\em Levi-Civita Connection}.

\begin{theorem}
    For $M$ a smooth Riemannian manifold, then there exists a unique connection
    $\nabla$ on $TM$ such that
    \begin{itemize}
        \item $\nabla$ is symmetric i.e.
            \[
                \nabla_XY -\nabla_YX = [X,Y]
            \]
            (The Christoffel symbols are symmetric in lower indices)
        \item $\nabla$ is metric-compatible. That is,
            \[
                \nabla g = 0
            \]
            or
            \[
                \nabla_{\gamma}g_{\mu\nu}v^{\nu} =
                g_{\mu\nu}\nabla_{\gamma}v^{\nu}
            \]
            or
            \[
                Xg(Y,Z) = g(\nabla_XY,Z) + g(Y,\nabla_XZ)
            \]
    \end{itemize}
\end{theorem}
\begin{proof}
    See Carroll (p.99) for an explicit construction of the torsion free, metric
    compatible connection in terms of the components of $g$.
    The formula is
    \[
        \Gamma_{\mu\nu}^{\gamma} = \frac{1}{2}g^{\gamma\rho}
        (\partial_{\mu}g_{\nu\rho} +
        \partial_{\nu}g_{\mu\rho}-\partial_{\rho}g_{\mu\nu})
    \]
\end{proof}
\begin{hw}
    Prove that the resulting connection is indeed a connection.
\end{hw}

Now to prove that the two definitions of a connection coincide.

From the local to the global definition is trivial, so we wish to prove that we
can localize the global definition.

\begin{proof}
    Consider a smooth connection $\nabla$ on $M$. Let $U\subset M$ be open, and
    $Y$ a smooth vector field on $M$, $X$ a smooth vector field on $U$. 

    Now, for $p\in U$, choose a smooth function $\eta$ on $M$ such that $\eta=1$
    in a neighborhood $V_1$ of $p$, and $\eta=0$ on $M\setminus V_2$ with
    $\overline{V_1}\subset V_2$, $\overline{V_2}\subset U$ and $\overline{V_i}$
    compact.

    \begin{hw}
        Construct a one-dimensional smooth bump function on $\R$
    \end{hw}

    Now, set $\tilde{X}=\eta X$, which is defined globally on $M$.
    We can now define
    \[
        \nabla_XY|_{V_1} = \nabla_{\tilde{X}}Y|_{V_1}
    \]
    and we can do this for every point $p\in M$. Now, we must show that such a
    construction is unique. 
    
    Suppose instead that we chose a different $V_1',V_2',\eta'$. We have a new
    globally-defined vector field $X'=\eta' X$, and we wish to show that
    $\nabla_{\tilde{X}}Y = \nabla_{X'}Y$ at $p$.

    So, we construct
    \[
        \begin{aligned}
            \nabla_{\tilde{X}}(Y) - \nabla_{X'}(Y) &= \nabla_{\tilde{X}-X'}Y\\
        \end{aligned}
    \]
    Now, we know that $\tilde{X}-X'$ is zero at (and nearby) $p$, so
    \[
        \tilde{X} -X' = \zeta(\tilde{X}-X')
    \]
    \begin{hw}
        Construct $\zeta$.
    \end{hw}
    So, we have that
    \[
        \begin{aligned}
            \nabla_{\tilde{X}-X'} &= \nabla_{\zeta(\tilde{X}-X')}\\
            &= \zeta\nabla_{\tilde{X}-X'}\\
            &=0
        \end{aligned}
    \]
    and so they agree around $p$.

    Next, consider $p\in M$, with $Y$ a smooth vector field. Choose a coordinate
    chart $(U,\phi)$ around $p$, with $v\in T_pM$, $v = v^i\partial_i$.

    Then, we set $\nabla_v Y = \nabla_{v^i\partial_i}Y =
    v^i\nabla_{\partial_i}Y$, where we have already defined what
    $\nabla_{\partial_i}$ should be, since $\partial_i$ is a locally defined
    vector field.

    Now, we need to show this is independent of coordinate charts. Let
    $(V,\psi)$ be another coordinate chart, with $v = v^j\partial'_j$ for
    coordinate field $\partial'_i$. The claim is that
    \[
        v^i\nabla_{\partial_i}Y = v^j\nabla_{\partial'_j}Y
    \]
    which is easily verified, since $J(\partial\to\partial')\nabla_{\partial_i}
    = \nabla_{\partial'_j}$, and so
    \[
        v^j\nabla_{\partial'_j} = v^j\nabla_{J(\partial\to\partial')^j_i\partial_j}
    \]
    but $v^i = J^i_jb^j$, and so they agree.
\end{proof}

\subsection{The Levi-Cevita Connection}
Recall that we have a unique torsion-free, metric compatible connection $\nabla$
for any Riemannian manifold. We wish to localize this $\nabla$ further.

\begin{defn}
    Let $\gamma$ be a smooth curve in $M$. A vector field $X$ {\em along
    $\gamma$} is an assignment $X:I\to TM$ with $X(t) \in T_{\gamma(t)}M$ where
    $X$ is called smooth if its coordinate decomposition
    \[
        X = \xi^i(t)\partial_i
    \]
    is smooth in each component.
\end{defn}

\begin{defn}
    $\nabla_{\partial_t}X$ is define along $\gamma$ as follows: Let $I_{t_0}$ be an open
    interval around $t_0$, which maps into chart $(U,\phi)$. Then,
    \[
        \begin{aligned}
            \nabla_{\partial_t}X &= \nabla_{\partial_t}\xi^i(t)\partial_i\\
                        &= \partial_t\xi^i(t)\partial_i +
                        \xi^i(t)\nabla_{\partial_t}\partial_i\\
                        &= \partial_t\xi^i(t)\partial_i +
                        \xi^i(t)\nabla_{\partial_t\gamma}\partial_i\\
        \end{aligned}
    \]
    which is already defined.
\end{defn}

The second term in this expansion turns into
\[
    \xi^i(t)\nabla_{\partial_t\gamma}\partial_i
    = \xi^i\partial_tx^j \nabla_{\partial_j}\partial_i
\]
and we define
\[
    \Gamma^k_{ij}\partial_k = \nabla_{\partial_j}\partial_i
\]
Where $\Gamma^k_{ij}$ is the Christoffel symbol (of the first kind) for the
connection.

\begin{hw}
    Show that for the Levi-Civita connection,
    \[
        \Gamma^k_{ij} = \frac{1}{2}g^{kl}(g_{il,j} + g_{lj,i} - g_{ij,l})
    \]
\end{hw}

\subsection{The Connection in Local Coordinates}

\begin{defn}
    The connection forms of the connection $\omega_i^j$ associated with an
    orthonormal frame $e_i$ is defined as
    \[
        \nabla e_i = \omega_i^j\otimes e_j
    \]
\end{defn}

Knowing that the frame is orthonormal and the connection is metric compatible,
we get
\[
    \begin{aligned}
        \langle e_i, e_j\rangle &= \delta_{ij}\\
        \langle \nabla_Xe_i,e_j\rangle + \langle e_i,\nabla_Xe_j\rangle &=0\\
        \langle \omega_i^k(X)e_k, e_j\rangle + \langle e_i,
        \omega_j^l(X)e_l\rangle &= 0\\
        \omega_i^j(X) + \omega^i_j(X) &= 0\\
    \end{aligned}
\]
and so $\omega^i_j$ is antisymmetric.

\begin{theorem}
    The following holds for the connection forms:
    \begin{itemize}
        \item $\omega$ is antisymmetric
        \item $d\omega^i = \omega_j^i\wedge\omega^j$
    \end{itemize}
\end{theorem}

To prove this, we can use the identity
\[
    d\alpha(X,Y) = X\alpha(Y) -Y\alpha(X) - \alpha([X,Y])
\]
for one-forms $\alpha$.

\subsection{Parallel Transport}
Let $X$ be a vector field on $M$ along $\gamma$.
\begin{defn}
    $X$ is called parallel if
    \[
        \nabla_{\partial_t}X=0
    \]
\end{defn}

\begin{theorem}
    For each $v\in T_{\gamma(0)}M$, there is a unique solution to the initial
    value problem
    \[
        \begin{aligned}
            \nabla_{\partial_t}X = 0\\
            X(0)=v
        \end{aligned}
    \]
\end{theorem}
\begin{proof}
    Let $(U,\phi)$ be a coordinate chart around $\gamma(0)$. Then, in $U$,
    \[
        \nabla_{\partial_t}X = 0
    \]
    is the same as
    \[
        \partial_t\xi^k + \Gamma^k_{ij}\xi^i\partial_tx^j
    \]
    which is a first order linear ODE with smooth coefficients, and so it has
    unique solutions for the initial value $X(0)=v$, or $\xi^i(0) = v^i$.
\end{proof}
Now, since the ODE is linear, there is a linear map between initial values and
solutions, that is we have a linear map from $T_{\gamma(0)}M$ to
$T_{\gamma(1)}M$ by evaluating $X$ at $1$. This map is the parallel transport
map, and it is invertible by running the curve backwards. Thus, this map is an
isomorphism. Even better...
\begin{prop}
    The parallel transport map is an isometry.
\end{prop}

\begin{hw}
Prove that
    \[
        \partial_t g(X,Y) = g(\nabla_{\partial_t}X,Y) +
        g(X,\nabla_{\partial_t}Y)
    \]
\end{hw}

\end{document}
