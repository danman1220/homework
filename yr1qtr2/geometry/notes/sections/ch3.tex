\documentclass[../main.tex]{subfiles}
\begin{document}

\section{Geodesics and Curvature}

\subsection{Geodesics}
\begin{defn}
Let $(M^n,g)$ be a Riemannian manifold, and let $\gamma:I\to M$ a smooth curve.
$\gamma$ is called a {\em geodesic} if its second derivative vanishes. That is,
if it solves the geodesic equation
\[
    \nabla_{\partial_t}\partial_t\gamma = 0
\]
\end{defn}

Now, let's examine the geodesic equation further. In local coordinates, we have
\[
    \begin{aligned}
        \nabla_{\partial_t}\partial_t\gamma &=
        \nabla_{\partial_t}\partial_tx^i\partial_i\\
        &=\partial_t\partial_t x^k\partial_k +
        \partial_tx^k\nabla_{\partial_t}\partial_k\\
        &=(\partial_t\partial_t x^k +
        \Gamma^k_{ij}\partial_tx^i\partial_tx^j)\partial_k\\
    \end{aligned}
\]
and so the local coordinate version of the differential equation is the system
of equations
\[
    (\partial_t)^2x^k + \gamma^k_{ij}\partial_tx^i\partial_tx^j = 0
\]
which are guaranteed local unique solutions for initial conditions of $\gamma$
and $\gamma'$.

Let's look at properties of geodesics. In particular, we can look at
\[
    \partial_t|\gamma'|^2 = \partial_t(g(\gamma',\gamma'))
    = 2g(\nabla_{\partial_t}\gamma',\gamma') = 0
\]
and so the velocity of the geodesic does not change.

\subsection{The Exponential Map}
Let $p\in M$. We can define an exponential map $\exp:T_pM\to M$ via
the following:

\begin{defn}
    The {\em exponential map} $\exp:T_pM\to M$ is defined as
    $\exp(v) = \gamma(1)$ where $\gamma$ is a geodesic with $\gamma(0)=p$ and
    $\gamma'(0)=v$. 
\end{defn}
Why do we insist that $\exp_p(v) = \gamma(1)$? Consider
\[
    \exp_p(tv) = \gamma_{tv}(1) =\gamma_v(t)
\]
where $t\in \R$. The last equality is obtained in the following way:

\begin{lemma}
    $\gamma_{tv}(1) = \gamma_v(t)$ for all $t$. 
\end{lemma}
\begin{proof}
    Consider $\gamma(t) = \gamma_{sv}(t)$. This is the geodesic such that
    $\gamma(0)=p$ and $\gamma'(0)=sv$. Now, notice that $\tilde{\gamma}(t) =
    \gamma_v(st)$ is defined so that $\tilde{\gamma}(0)=p$ and
    $\tilde{\gamma}'(0) = \partial_t\gamma_v(st) =
    \gamma'_v(0)\partial_t(st)|_{t=0} = sv$ and by uniqueness of geodesics,
    $\gamma = \tilde{\gamma}$ as desired.
\end{proof}

Let's examine the domain for the exponential map. With no assumptions on the
structure of the manifold, what can we say about solutions to the geodesic
equation?

Recall the escape lemma for flows along vector fields. If $\gamma$ is a maximal
integral curve of a vector field $X$ whose domain $J$ has a least upper bound
$b$, then for each $t_0<b$, $\gamma([t_0,b))$ is not contained in any compact
subset of the manifold. That is, if $\gamma$ goes into a compact subset of the
manifold, it will not die in the interior of the compact subset.

We also have the uniform time lemma, which guarantees that for $U$ open with
compact closure, any $K>0$, there is some $\epsilon>0$ such that the geodesic
$\gamma(t)$ with $\gamma(t_0)=p$ $\gamma'(t_0)=v$ exists for
$t\in[t_0-\epsilon,t_0+\epsilon]$ and the map
\[
    \begin{aligned}
        \gamma&:U^*\times (t_0-\epsilon,t_0+\epsilon)\to M\\
        \gamma(v,t) &= \gamma(t)
    \end{aligned}
\]
and here $U^* = \{v\in TM, \|v\|<K,\pi(v)\in U\}$.

Now, we can see that $\exp_p$ is defined on a closed ball
$\overline{B}_{\epsilon}(0)\subset T_pM$ for some $\epsilon>0$, and furthermore for
any compact set $K$, there is some $\epsilon>0$ such that $\exp_p$ is defined on
$\overline{B}_{\epsilon}(0)$ for all $p\in K$.

In $\R^n$, we have geodesics as linear affinely parameterized curves i.e.
$\gamma(t) = \vec{a}t+\vec{b}$.

\begin{ex}
    Consider the sphere $S^n$ with the induced metric from $\R^{n+1}$. Then, the
    Levi-Civita connection is given as
    \[
        \nabla^{S^n}_{\partial_t}\gamma' =
        \left(\nabla^{\R^{n+1}}_{\partial_t}\gamma'\right)^T
    \]
    Where $T$ is the tangential projection onto $S^n$.
\end{ex}

More generally, for $N$ a submanifold of $M$, and $X$ a vector field on $N$, we
can extend $X$ to a neighborhood in $M$, and take $\nabla_{\partial_t}^NX =
(\nabla_{\partial_t}^MX)^T$.

\subsection{Further properties of geodesics}
In $\R^n$, it is clear that the straight line geodesic between two points is the path of
shortest length between them. Is this the case in general? Do geodesics exist
between any two points?

Obviously there are not geodesics between arbitrary points in a general
connected Riemannian manifold (motivating example: Schwarzschield geometry).
However, if a Riemannian manifold is complete, then it has geodesics between all
points.

Does there always exist a geodesic of minimal length? And are such geodesics
unique? No. On $S^2$, we have an infinite number of geodesics from the north to
the south pole. Suppose instead, however, that we restrict to anything but the
south pole. Then, there exist unique geodesics of minimal length from the north
pole to any point. This has to do with the fact that the geodesics from the
north pole do not cross until the south pole.

We can prove that {\em locally}, points are connected by a minimal geodesic, and
that open balls around a point correspond to exponential projections of open
balls in the tangent space.

Now, an important lemma:
\begin{lemma}
    {\em Gauss Lemma:} Let $M$ be a manifold, and $p\in M$ with exponential
    $\exp_p$. We wish to understand $(d\exp_p)_v:T_pM\to T_{\exp_p(v)}M$. It is
    true that
    \[
        g(v,w) = g(d(\exp_p)_v(v),d(\exp_p)_v(w))
    \]
\end{lemma}

\begin{proof}
    We begin by calculating $(d\exp_p)_v(V)$. Note that here, the $v$ in the
    parentheses is actually in $T_vT_pM$, which is canonically identified with
    $T_pM$.

    Specifically, we wish to show $\|(d\exp_p)_v(v)\| = \|v\|$, or that
    $g((d\exp_p)_v(v),(d\exp_p)_v(v)) = g(v,v)$.

    To see this, consider the geodesic $c(t) = \exp_p(tv)$. Now, $c$ is affinely
    parameterized, so it has constant speed (magnitude of tangent vector). Now,
    $c'(0)=v$, and $c'(1) = (d\exp_p)_v(\partial_t(tv)|_{t=1}) =
    (d\exp_p)_v(v)$ and since $c$ is affinely parameterized, these two have the
    same magnitude.

    Now, let $w\in T_pM$ such that $w$ is perpendicular to $v$. We can choose a
    path $\tau(s) = v+sw$ such that $\tau(0) = v$ and $\tau'(0) = w$.
    Consider
    \[
        F(t,s) = \exp_p(t(v+sw))
    \]
    where, by varying $s$, we get a family of geodesics from the tangent vectors
    $v+sw$. Now, for $t\in [0,1]$ (actually $(-\epsilon, 1+\epsilon)$), $s\in
    (-\epsilon,\epsilon)$, we have a smooth map $F:[0,1]\times
    (-\epsilon,\epsilon) \to M$.

    \begin{lemma}
        For a smooth map $F:[a,b]\times [c,d]\to M$ with first coordinate $t$
        and second coordinate $s$,
        \[
            \nabla_{\partial_s}\partial_t F = \nabla_{\partial_t}\partial_s F
        \]
    \end{lemma}
    \begin{hw}
        Prove this lemma, using the fact that $[\partial_s,\partial_t]=0$.
    \end{hw}

    Now, we have that
    \[
        \partial_t F(t,0) = c'(t)
    \]
    since $F(t,0) = \exp_p(tv) = c(t)$. We also have
    \[
        \partial_s F(1,0) = (d\exp_p)_v(\partial_s(t(v+sw))|_{t=1,s=0}) =
        (d\exp_p)_v(w)
    \]
    Now, we wish to show that
    \[
        g((d\exp_p)_v(v),(d\exp_p)_v(w)) = 0
    \]
    which is clear, since
    \[
        \begin{aligned}
            g((d\exp_p)_v(v),(d\exp_p)_v(w))
            &= g(\partial_t F(1,0),\partial_s F(1,0))\\
            &= g(\partial_t F,\partial_s F)|_{t=1,s=0}
        \end{aligned}
    \]
    Now,
    \[
        \begin{aligned}
            \partial_t g(\partial_tF,\partial_sF)|_{s=0} &= g(\nabla_{\partial_t}\partial_t
        F,\partial_s F) + g(\partial_t F,\nabla_{\partial_t}\partial_s F)\\
            &= g(\partial_t F,\nabla_{\partial_t}\partial_s F) &\text{since $F$
            is along a geodesic, second derivatives vanish}\\
            &= g(\partial_t F,\nabla_{\partial_s}\partial_t F)\\
            &= \frac{1}{2}\partial_s g(\partial_t F,\partial_tF) &\text{By
            symmetry of the metric}
        \end{aligned}
    \]
    Now, suppose instead that we use a circular arc in $T_pM$ between $v$ and
    $w$ so that $\|\partial_t F\|$ is independent of $s$. Then, it follows that
    $\partial_s g(\partial_t F,\partial_t F) = 0$ as desired.

    Now, let's calculate $g(\partial_t F,\partial_sF)|_{t=0,s=0}$.
    We have
    \[
        \begin{aligned}
            \partial_t F|_{t=0,s=0} &= c'(0) = v\\
            \partial_s F|_{t=0,s=0} &=\partial_s\exp_p(t\tau(s))|_{t=0,s=0}=0
        \end{aligned}
    \]
    and so $g((d\exp_p)_v(v),(d\exp_p)_v(w)) = 0$.

    These two facts then prove the Gauss lemma by writing $u = \alpha v+\beta
    w$ for $w$ perpendicular to $v$. 
\end{proof}

If we were to follow the proof through using the straight line in $T_pM$ instead
of a circular arc, we would need to use the following lemma
\begin{lemma}
    $(d\exp_p)_0 = id$
\end{lemma}
\begin{proof}
    Let $v\in T_pM$, and let $\gamma(t) = tv$ be a curve in $T_pM$ with tangent
    vector $v$ at zero. Then,
    \[
        (d\exp_p)_0(v) = \partial_t\exp_p(tv) = v
    \]
    as desired.
\end{proof}

\begin{prop}
    For $U_p$ an open set in $T_pM$, $0\in U_p$, the exponential map
    $\exp_p|_{U_p}$ is a diffeomorphism onto its image, and $\exp_p(U_p)$ is
    open in $M$.
\end{prop}
$B_r(0)\subset T_pM$ is called a normal ball if $\exp_p$ restricts to a
diffeomorphism from $B_r(0)$ to its image.

\begin{theorem}
    Let $B_{r_0}(0)$ be a normal ball. Then, for each $v\in B_{r_0}(0)$, the
    radial geodesic $c(t) = \exp_p(tv)$ for $t\in[0,1]$ is the unique shortest
    smooth curve up to reparameterization from $p$ to $\exp_p(v)$.
\end{theorem}

A corollary of this is that $\exp_p(B_r(0)) = B_r(p)$.

\begin{proof}
    Let $v\in B_{r_0}(0)$ as described in the hypothesis. Let $c(t) =
    \exp_p(v)$, with $c(0)=p$ and $c(1)=q$. Furthermore, let $\gamma$ be any
    curve from $p$ to $q$.

    Suppose $\gamma$ leaves $\exp_p(B_{\|v\|}(0))$ at some time$t_1$. That is,
    $\gamma([0,t_1))\subset\exp(B_{\|v\|})$ and $\gamma(t_1)$ is in the
    boundary. Then, we know that
    \[
        L_{\gamma}\geq L_{\gamma|_{[0,t_1)}}
    \]
    so all we need to show is that
    \[
        L_c\leq L_{\gamma|_{[0,t_1)}}
    \]

    Now, this reduces to the second case. Namely, suppose $\gamma$ is entirely
    contained in $\exp(B_{\|v\|}(0))$, and $\gamma(1)=q_1$ is on the boundary.
    Let $\tilde{\gamma}(t) = \exp_p|_{B_{\|v\|}(0)}^{-1}\circ\gamma(t)$ be the
    corresponding curve in $T_pM$. Now, all we have to do is calculate the
    length of $\gamma$.
    \[
        L_{\gamma} = \int_Ig(\gamma',\gamma')dt
    \]
    Now,
    $\gamma'(t) = (d\exp_p)_{\tilde{\gamma}(t)}(\tilde{\gamma}'(t))$
    and if we assume $\tilde{\gamma}(t)$ is not zero, we can calculate the
    magnitude of $\gamma'(t)$. Let $\tilde{\gamma}'(t)$ be decomposed into a
    radial and normal part $r(t)$ and $n(t)$ with
    respect to the vector $\tilde{\gamma}$. Then,
    \[
        \begin{aligned}
        g((d\exp_p)_{\tilde{\gamma}(t)}(r(t)+n(t)),(d\exp_p)_{\tilde{\gamma}(t)}(r+n))
        &= \|r(t)\|^2 + \|d\exp_p(n(t))\|^2\\
            &\geq \|r(t)\|^2
        \end{aligned}
    \]
\begin{hw}
    prove that equality is met in the previous inequality if and only if
    $\tilde{\gamma}(t)$ is radial.
\end{hw}
    Then,
    \[
        \begin{aligned}
            L_{\gamma} &= \int_{0+}^1 \|\gamma'\|dt\\
            &\geq \int_{0+}^1\|r(t)\|dt\\
        \end{aligned}
    \]
    switching to polar coordinates, we denote $R(v) = \|v\|$, and we can
    calculate 
    \[
        \begin{aligned}
        \partial_t R(\tilde{\gamma}(t)) &= \nabla (R)\cdot\tilde{\gamma}'(t)\\
            &=
            \frac{\tilde{\gamma}(t)}{\|\tilde{\gamma}(t)\|}\cdot\tilde{\gamma}'(t)\\
            &= r(t)
        \end{aligned}
    \]
    and so
    \[
        \begin{aligned}
            L_{\gamma} &\geq \int_{0+}^1\|r(t)\|dt\\
                        &= \int_{0+}^1\partial_tR(\tilde{\gamma}(t))dt
                        &=L(c)
        \end{aligned}
    \]
    as desired.
\end{proof}

The corollary follows immediately.

\begin{proof}
    Let $U_r = \exp_p(B_r(0))$. It should be clear that $U_r\subset B_r(p)$
    since the radial geodesic $\exp_p(tv)$ for $v\in B_r(0), t\in[0,1]$ has
    length $\|v\| <r$. Since this geodesic is the minimal path from $p$ to
    $\exp_p(v)$, it follows that $d(p,\exp_p(v))<r$ as well.

    Now, let $q\in B_r(p)$. That is, $d(p,q)<r$. Hence, we can find a smooth
    curve $c:I\to M$ with $L(c)<r$ and $c(0)=p,c(1)=q$.
    Let $t_1$ be such that $c([0,t_1))\subset U_r$ (since $c$ is continuous, and
    $c(0)\subset U_r$, and $U_r$ is open).
    Thus, for all $t\in[0,t_1)$, we have $c(t) = \exp_p(v(t))$ for some $v(t)\in
    B_r(0)$. In particular, we can find an $r_1<r$ such that $v(t)\in
    B_{r_1}(0)$. 

    Thus, we know that
    \[
        L(c|_{[0,t_1)})\leq L(c)\leq r_1
    \]
    and so
    \[
        L(\exp_p(sv(t))|_{s\in[0,1]})=\|v(t)\|\leq L(c) <r_1
    \]

    Now, consider the supremum of such $t_1$. We claim that $\sup t_1 = 1$,
    which implies that $L(c)\leq r_1<r$ as desired. To see that $\sup t_1 = 1$,
    suppose instead that $\sup t_1 = T < 1$. This means that for all $t<T$,
    $c(t) = \exp_p(v(t)), t\in[0,T)$. However, taking a limit of such $v(t)$
    yields some $V = v(T)$ for which $c(T) = \exp_p(V)$. However, this means
    that $c(T)\in U_r$ as well, and thus there is some $t' > T$ for which $c(t)
    = \exp_p(v(t)),t\in[0,t')$, which contradicts $T$ being the supremum.

    Thus, $c(t) = \exp_p(v(t)),t\in I$, and so $q\in U_r$ as desired.
\end{proof}

\begin{hw}
    Find a counterexample to $\exp_p(B_r(0)) = B_r(p)$ for arbitrary $r$.
\end{hw}
(Note, for compact Riemannian manifolds, this is actually true! So $S^n$ or
$T^n$ won't be a good counterexample...)


\begin{cor}
    If a piecewise differentiable curve $\gamma(t)$ affinely parameterized
    minimizes the length between $\gamma(0)$ and $\gamma(1)$ (that is,
    $L(\gamma) = d(\gamma(0),\gamma(1))$), then $\gamma$ is a smooth geodesic.
\end{cor}

\begin{proof}
    (Easy proof) Apply variational calculus to the arc length formula to see
    that minimal paths satisfy the geodesic equation, and apply uniqueness.
\end{proof}

\end{document}
