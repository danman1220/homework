\documentclass[../main.tex]{subfiles}
\begin{document}

\section{Geodesics and Curvature}

\subsection{Geodesics}
\begin{defn}
Let $(M^n,g)$ be a Riemannian manifold, and let $\gamma:I\to M$ a smooth curve.
$\gamma$ is called a {\em geodesic} if its second derivative vanishes. That is,
if it solves the geodesic equation
\[
    \nabla_{\partial_t}\partial_t\gamma = 0
\]
\end{defn}

Now, let's examine the geodesic equation further. In local coordinates, we have
\[
    \begin{aligned}
        \nabla_{\partial_t}\partial_t\gamma &=
        \nabla_{\partial_t}\partial_tx^i\partial_i\\
        &=\partial_t\partial_t x^k\partial_k +
        \partial_tx^k\nabla_{\partial_t}\partial_k\\
        &=(\partial_t\partial_t x^k +
        \Gamma^k_{ij}\partial_tx^i\partial_tx^j)\partial_k\\
    \end{aligned}
\]
and so the local coordinate version of the differential equation is the system
of equations
\[
    (\partial_t)^2x^k + \gamma^k_{ij}\partial_tx^i\partial_tx^j = 0
\]
which are guaranteed local unique solutions for initial conditions of $\gamma$
and $\gamma'$.

Let's look at properties of geodesics. In particular, we can look at
\[
    \partial_t|\gamma'|^2 = \partial_t(g(\gamma',\gamma'))
    = 2g(\nabla_{\partial_t}\gamma',\gamma') = 0
\]
and so the velocity of the geodesic does not change.

\subsection{The Exponential Map}
Let $p\in M$. We can define an exponential map $\exp:T_pM\to M$ via
the following:

\begin{defn}
    The {\em exponential map} $\exp:T_pM\to M$ is defined as
    $\exp(v) = \gamma(1)$ where $\gamma$ is a geodesic with $\gamma(0)=p$ and
    $\gamma'(0)=v$. 
\end{defn}
Why do we insist that $\exp_p(v) = \gamma(1)$? Consider
\[
    \exp_p(tv) = \gamma_{tv}(1) =\gamma_v(t)
\]
where $t\in \R$. The last equality is obtained in the following way:

\begin{lemma}
    $\gamma_{tv}(1) = \gamma_v(t)$ for all $t$. 
\end{lemma}
\begin{proof}
    Consider $\gamma(t) = \gamma_{sv}(t)$. This is the geodesic such that
    $\gamma(0)=p$ and $\gamma'(0)=sv$. Now, notice that $\tilde{\gamma}(t) =
    \gamma_v(st)$ is defined so that $\tilde{\gamma}(0)=p$ and
    $\tilde{\gamma}'(0) = \partial_t\gamma_v(st) =
    \gamma'_v(0)\partial_t(st)|_{t=0} = sv$ and by uniqueness of geodesics,
    $\gamma = \tilde{\gamma}$ as desired.
\end{proof}

Let's examine the domain for the exponential map. With no assumptions on the
structure of the manifold, what can we say about solutions to the geodesic
equation?

Recall the escape lemma for flows along vector fields. If $\gamma$ is a maximal
integral curve of a vector field $X$ whose domain $J$ has a least upper bound
$b$, then for each $t_0<b$, $\gamma([t_0,b))$ is not contained in any compact
subset of the manifold. That is, if $\gamma$ goes into a compact subset of the
manifold, it will not die in the interior of the compact subset.


\end{document}
