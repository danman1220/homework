\documentclass[../main.tex]{subfiles}
\begin{document}

\section{Geodesics and Curvature}

\subsection{Geodesics}
\begin{defn}
Let $(M^n,g)$ be a Riemannian manifold, and let $\gamma:I\to M$ a smooth curve.
$\gamma$ is called a {\em geodesic} if its second derivative vanishes. That is,
if it solves the geodesic equation
\[
    \nabla_{\partial_t}\partial_t\gamma = 0
\]
\end{defn}

Now, let's examine the geodesic equation further. In local coordinates, we have
\[
    \begin{aligned}
        \nabla_{\partial_t}\partial_t\gamma &=
        \nabla_{\partial_t}\partial_tx^i\partial_i\\
        &=\partial_t\partial_t x^k\partial_k +
        \partial_tx^k\nabla_{\partial_t}\partial_k\\
        &=(\partial_t\partial_t x^k +
        \Gamma^k_{ij}\partial_tx^i\partial_tx^j)\partial_k\\
    \end{aligned}
\]
and so the local coordinate version of the differential equation is the system
of equations
\[
    (\partial_t)^2x^k + \gamma^k_{ij}\partial_tx^i\partial_tx^j = 0
\]
which are guaranteed local unique solutions for initial conditions of $\gamma$
and $\gamma'$.

Let's look at properties of geodesics. In particular, we can look at
\[
    \partial_t|\gamma'|^2 = \partial_t(g(\gamma',\gamma'))
    = 2g(\nabla_{\partial_t}\gamma',\gamma') = 0
\]
and so the velocity of the geodesic does not change.

\subsection{The Exponential Map}
Let $p\in M$. We can define an exponential map $\exp:T_pM\to M$ via
the following:

\begin{defn}
    The {\em exponential map} $\exp:T_pM\to M$ is defined as
    $\exp(v) = \gamma(1)$ where $\gamma$ is a geodesic with $\gamma(0)=p$ and
    $\gamma'(0)=v$. 
\end{defn}
Why do we insist that $\exp_p(v) = \gamma(1)$? Consider
\[
    \exp_p(tv) = \gamma_{tv}(1) =\gamma_v(t)
\]
where $t\in \R$. The last equality is obtained in the following way:

\begin{lemma}
    $\gamma_{tv}(1) = \gamma_v(t)$ for all $t$. 
\end{lemma}
\begin{proof}
    Consider $\gamma(t) = \gamma_{sv}(t)$. This is the geodesic such that
    $\gamma(0)=p$ and $\gamma'(0)=sv$. Now, notice that $\tilde{\gamma}(t) =
    \gamma_v(st)$ is defined so that $\tilde{\gamma}(0)=p$ and
    $\tilde{\gamma}'(0) = \partial_t\gamma_v(st) =
    \gamma'_v(0)\partial_t(st)|_{t=0} = sv$ and by uniqueness of geodesics,
    $\gamma = \tilde{\gamma}$ as desired.
\end{proof}

Let's examine the domain for the exponential map. With no assumptions on the
structure of the manifold, what can we say about solutions to the geodesic
equation?

Recall the escape lemma for flows along vector fields. If $\gamma$ is a maximal
integral curve of a vector field $X$ whose domain $J$ has a least upper bound
$b$, then for each $t_0<b$, $\gamma([t_0,b))$ is not contained in any compact
subset of the manifold. That is, if $\gamma$ goes into a compact subset of the
manifold, it will not die in the interior of the compact subset.

We also have the uniform time lemma, which guarantees that for $U$ open with
compact closure, any $K>0$, there is some $\epsilon>0$ such that the geodesic
$\gamma(t)$ with $\gamma(t_0)=p$ $\gamma'(t_0)=v$ exists for
$t\in[t_0-\epsilon,t_0+\epsilon]$ and the map
\[
    \begin{aligned}
        \gamma&:U^*\times (t_0-\epsilon,t_0+\epsilon)\to M\\
        \gamma(v,t) &= \gamma(t)
    \end{aligned}
\]
and here $U^* = \{v\in TM, \|v\|<K,\pi(v)\in U\}$.

Now, we can see that $\exp_p$ is defined on a closed ball
$\overline{B}_{\epsilon}(0)\subset T_pM$ for some $\epsilon>0$, and furthermore for
any compact set $K$, there is some $\epsilon>0$ such that $\exp_p$ is defined on
$\overline{B}_{\epsilon}(0)$ for all $p\in K$.

In $\R^n$, we have geodesics as linear affinely parameterized curves i.e.
$\gamma(t) = \vec{a}t+\vec{b}$.

\begin{ex}
    Consider the sphere $S^n$ with the induced metric from $\R^{n+1}$. Then, the
    Levi-Civita connection is given as
    \[
        \nabla^{S^n}_{\partial_t}\gamma' =
        \left(\nabla^{\R^{n+1}}_{\partial_t}\gamma'\right)^T
    \]
    Where $T$ is the tangential projection onto $S^n$.
\end{ex}

More generally, for $N$ a submanifold of $M$, and $X$ a vector field on $N$, we
can extend $X$ to a neighborhood in $M$, and take $\nabla_{\partial_t}^NX =
(\nabla_{\partial_t}^MX)^T$.

\subsection{Further properties of geodesics}
In $\R^n$, it is clear that the straight line geodesic between two points is the path of
shortest length between them. Is this the case in general? Do geodesics exist
between any two points?

Obviously there are not geodesics between arbitrary points in a general
connected Riemannian manifold (motivating example: Schwarzschield geometry).
However, if a Riemannian manifold is complete, then it has geodesics between all
points.

Does there always exist a geodesic of minimal length? And are such geodesics
unique? No. On $S^2$, we have an infinite number of geodesics from the north to
the south pole. Suppose instead, however, that we restrict to anything but the
south pole. Then, there exist unique geodesics of minimal length from the north
pole to any point. This has to do with the fact that the geodesics from the
north pole do not cross until the south pole.

We can prove that {\em locally}, points are connected by a minimal geodesic, and
that open balls around a point correspond to exponential projections of open
balls in the tangent space.

Now, an important lemma:
\begin{lemma}
    {\em Gauss Lemma:} Let $M$ be a manifold, and $p\in M$ with exponential
    $\exp_p$. We wish to understand $(d\exp_p)_v:T_pM\to T_{\exp_p(v)}M$. It is
    true that
    \[
        g(v,w) = g(d(\exp_p)_v(v),d(\exp_p)_v(w))
    \]
\end{lemma}


\end{document}
