\documentclass[../main.tex]{subfiles}
\begin{document}

\section{Manifolds of Constant Curvature}
Suppose $M$ is a Riemannian manifold with constant sectional curvature $\kappa$.
Again, we note that $\exp_p:T_pM\to M$ is a diffeomorphism if and only if
$d\exp_p|_v$ is an isomorphism for all $v$ in the domain. This means that $p$
and $\exp_p(v)$ are not conjugate points.

\begin{prop}
    Let $\gamma$ be a path in a manifold $M$, with $K\leq 0$ along $\gamma$.
    Then, no point of $\gamma$ is conjugate to $\gamma(0)$.
\end{prop}

\begin{proof}
    Let $J$ be a Jacobi field along $\gamma$. Suppose $J(0)=J(1)=0$ (so that
    $\gamma(1)$ is a candidate for conjugacy). Observe that $J$ is normal to
    $\gamma$, since the function $f(t) = g(J(t),\gamma'(t))$ has derivative
    \[
        f' = g(J',\gamma')
    \]
    and second derivative
    \[
    f'' = g(J'',\gamma') = -g(R(\gamma',J)\gamma',\gamma') = 0
    \]
    and so $f$ is linear with $f(0)=f(1)=0$ and so $f=0$. The last equality
    follows from the Jacobi equation.

    Thus, by integrating $J''\cdot Jd\gamma$ we find $\|J\|^2=0$ everywhere, and
    $\gamma(1)$ is not a conjugate point.
\end{proof}

As a result, for $M$ a manifold with nonpositive sectional curvature, $\exp_p$
is a local diffeomorphism for each $p$ on its maximal domain. Furthermore, if
$M$ is complete, then $\exp_p$ is a diffeomorphism from $T_pM$ to $M$. It turns
out this becomes a covering map from $T_pM$ to $M$ as well. (Covered later).

Suppose $K$ is positive and bounded above by $\kappa$. Then, $p$ does not have a conjugate
point at (for unit speed parameterization of $\gamma$) $\gamma(b)$ for all $b <
\frac{\pi}{\sqrt{\kappa}}$.
\end{document}
