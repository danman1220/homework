\documentclass[../main.tex]{subfiles}
\begin{document}
\section{Jacobi Fields}
Consider $M$ a Riemannian manifold, $p\in M$. Let $\Omega$ be the maximal domain
of the exponential map $\exp_p$. We wish to understand what happens to the image
of a ball under the exponential map (which is a diffeo for small balls, but what
about larger ones?). In essence, where is the exponential map a local
diffeomorphism? Or, where is $d\exp_p$ an isomorphism? For what $u$ is
$(d\exp_p)_u:T_uT_pM \cong T_pM \to T_{\exp_p(u)}M$ an isomorphism? This is true
if and only if the kernel is trivial.

Suppose $v$ is such that $d\exp_p|_u(v) = 0$. We know that 
\[
    d\exp_p|_u(v) = \partial_s\exp_p(u+sv)|_{s=0}
\]
Now, we have a family of radial geodesics $\exp(t(u+sv))$ (familial wrt $s$)
which connect $\exp(tu)$ to $\exp(t(u+v))$, call the family $\gamma(t,s) =
\exp_p(t(u+sv))$. $\gamma$ is a smooth map from a disk $D$ (or a rectangle) to
$M$. (Horizontal slices in the rectangle are geodesics). We have two important
vector fields: $\partial_s\gamma$ and $\partial_t\gamma$. $\partial_s\gamma$ is
the rate of change of the deformation of the starting curve, the deformation
vector field. Let $J=\partial_s\gamma$. Do we have an equation for $J$? What
happens if we differentiate $J$?
\[
    \begin{aligned}
    \nabla_{\partial_t}\nabla_{\partial_t}\partial_s\gamma  &=
    \nabla_{\partial_t}\nabla_{\partial_s}\partial_t\gamma &\text{commutativity
    not proven in class}\\
    &= \nabla_t\nabla_s\partial_t\gamma - \nabla_s\nabla_t\partial_t\gamma +
    \nabla_as\nabla_t\partial_t\gamma +
    \nabla_{[\partial_s,\partial_t]}\partial_t\gamma\\
    &=R(\partial_s\gamma,\partial_t\gamma)\partial_t\gamma &\text{since
    $\nabla_t\partial_t\gamma$ vanishes (geodesic)}
\end{aligned}
\]
which gives us a formula for $J$ as
\[
    J'' + R(\gamma_0',J)\gamma_0 = 0
\]

\begin{theorem}
    The linear map $F(w) = Jw$ is an isomorphism.
\end{theorem}
\begin{proof}
    If $Jw = 0$, then $J'w= 0$ so $w = J'w(0) = 0$. So $F$ is injective. Now, to
    show surjectivity, we let $J$ be a Jacobi field along $\gamma$ such that
    $J(0) = 0$, $J(1)=0$, and set $w=J'(0)$. Then $Jw = J$ and $F$ is
    surjective.
\end{proof}

\begin{theorem}
    $d\exp_p|_v$ is an isomorphism if and only if $1$ is not a conjugate time
    for $\gamma(t) = \exp_p(tv)$ i.e. $\exp_p(v)$ is not conjugate to $p$ along
    $\gamma$.
\end{theorem}
Here, $\exp_p(v)$ is conjugate to $p$ if the Jacobi field along $\gamma$
vanishes at both $p$ and $\exp_p(v)$.

\subsection{Jacobi Fields on Manifolds With Constant Curvature}
Recall the Jacobi equation
\begin{equation}
    J'' + R_m(\gamma',J')\gamma' = 0
\end{equation}
First assume $J = f\gamma'$. Then, the jacobi equation yields
\begin{equation}
    J'' + fR(\gamma',\gamma')\gamma' = 0
\end{equation}
and thus $J''=0$ so $f = at+b$, and so $J(t) = (at+b)\gamma'$. With the initial
condition $J(0)=0$, we have $J(t) = at\gamma'(t)$, which will never be zero for
positive time and $a\neq 0$. So, $J$ as a tangential field cannot form conjugate
points.

Now, let's assume $J$ is normal to $\gamma'(t)$. Now, let's choose a parallel
frame along $\gamma$ the geodesic such that $e_1$ is parallel to $\gamma'(t)$.
Then, $J = f^ie_i$ which yields the system of differential equations
\begin{equation}
    f''^ie_i + f^iR_m(\gamma',e_i)\gamma' = 0
\end{equation}
If we assume that our manifold has constant curvature, we have
\begin{equation}
    R(X,Y)Z = \kappa(g(Z,X)Y - g(Z,Y)X)
\end{equation}
which yields the equations
\begin{equation}
    f''^ie_i + f^i\kappa(\|\gamma'\|^2e_i = 0
\end{equation}
or
\begin{equation}
    f''^i + f^i\kappa\|\gamma'\|^2 = 0
\end{equation}

Furthermore, if we assume $\gamma$ is unit  parameterized, we have the
equations
\begin{equation}
    f''^i + f^i\kappa = 0
\end{equation}

Which has solutions (for $J(0)=0$)
\begin{table}
    \centering
    \begin{tabular}{l l}
        $\kappa = 0$ & $J^i(t) = a^it$\\
        $\kappa < 0$ & $J^i(t) = a^i\sinh(\sqrt{-\kappa}t)$\\
        $\kappa > 0$ & $J^i(t) = a^i\sin(\sqrt{\kappa}t)$
    \end{tabular}
\end{table}

Suppose instead we wish to solve this without an orthonormal frame. Let $w =
J'(0)$ with $J(0)=0$ the initial data. Let $w(t)$ be the parallel vector field
of $w$ along $\gamma$. We assume the solution has the form $J(t) = f(t)w(t)$.
Then, we have
$f(0) = 0$ and $f'(0)=1$ initial conditions. Then, we have the equation
\begin{equation}
    f''w + f\kappa(\|\gamma'\|^2w - (\gamma'\cdot w)\gamma')=0
\end{equation}
Now, we have already taken care of the tangential part, so let's assume $w$ is
orthogonal to $\gamma'$. Then, we have the solutions
\begin{table}
    \centering
    \begin{tabular}{l l}
        $\kappa=0$ & $J(t) = tw(t)$\\
    $\kappa<0$ & $J(t) = \frac{\sinh(\sqrt{-\kappa} t)}{\sqrt{-\kappa}}$\\
    $\kappa>0$ & $J(t) = \frac{\sin(\sqrt{\kappa}t)}{\sqrt{\kappa}}$\\
    \end{tabular}
\end{table}

Note that in the case $\kappa > 0$, we know that $J(t)$ has a conjugate point
when $\sin(\sqrt{\kappa}t) = 0$, or $\sqrt{\kappa}t = n\pi$. The first of these
occurs at $t = \frac{\pi}{\sqrt{\kappa}}$. Note that these are exactly the
antipodal points on the sphere. Note that this implies the exponential map is
injective up until the conjugate time!

\end{document}
