\documentclass[../main.tex]{subfiles}
\begin{document}

\section{Isometric Immersions}
\subsection{Gauss Curvature Equation}

\begin{theorem}
    Let $u,v\in T_pM$ with $p\in M$, $\|u\|=\|v\| = 1$ and $u\cdot v=0$. Then
    \begin{equation}
        K(u,v) = \bar{K}(u,v) + B(u,u)\cdot B(v,v) - \|B(u,v)\|^2
        \label{Gauss Curvature Equation}
    \end{equation}
    Where $\bar{K}$ is the sectional curvature for the ambient space, and $B$ is
    defined as
    \begin{equation}
        B(X,Y) = \bar{\nabla}_{\bar{X}}\bar{Y} - \nabla_XY
    \end{equation}
\end{theorem}
The proof of this is found in Do Carmo\ldots

\begin{ex}
    Let's calculate the curvature of $S^n\subset \R^{n+1}$.

    Let $u,v$ be orthogonal vectors on $S^n$. Now, since we are in the ambient
    space $\R^{n+1}$, $\bar{K}=0$ everywhere. So, let's calculate the second
    fundamental form of the inclusion map $i:S^n\to \R^{n+1}$.
    \[
    B(u,v) = (\bar{\nabla}_{\bar{X}}\bar{Y})^N
    \]
    for $\bar{X},\bar{Y}$ extensions of $u$ and $v$ into the ambient space.

    So,
    \[
\begin{aligned}
B(u,v) &= (\bar{\nabla}_{\bar{X}}\bar{Y}\cdot \nu)\nu &\text{$\nu$ is unit
normal away from $S^n$.}\\
&=(-\bar{Y}\cdot\bar{\nabla}_{\bar{X}}\nu)\nu\\
&= (-v\cdot \bar{\nabla}_u v)\nu\\
&= (-v\cdot u)\nu
\end{aligned}
    \]
    and so using the Gauss curvature equation, we find that 
    \[
\begin{aligned}
    K(u,v) &= (-u\cdot u)\nu\cdot(-v\cdot v)\nu -\|(-v\cdot u)\|^2\\
    &= (-1)(-1)-0 = 1
\end{aligned}
    \]
    as desired.
\end{ex}

We can also define the mean curvature vector as $H = \frac{1}{2}\text{tr}(B)$
which is just
\[
    H = \frac{1}{2}(\sum_iB(E_i,E_i))
\]

For a $2$-dimensional subspace of $\R^3$, the Gauss curvature and the sectional
curvature are the same.

Of course, this theorem generalizes.

\begin{defn}
    An isometric immersion $f:M\to \bar{M}$ is called totally geodesic if the
    second fundamental form $B$ vanishes everywhere. If $B=0$ at a point $p$,
    then we say $M$ is geodesic at $p$.
\end{defn}

\begin{theorem}
    $f:M\to \bar{M}$ (think of an embedded submanifold) is totally geodesic if
    and only if all geodesics of $M$ are also geodesics of $\bar{M}$.
\end{theorem}

\begin{proof}
    ($\implies$) Suppose $M$ is totally geodesic. Then, 
    \[
        \bar{\nabla}_{\bar{X}}\bar{Y} = \nabla_XY +B(X,Y)
    \]
    and so the connections agree, and geodesics in $M$ are automatically
    geodesics in $\bar{M}$
\end{proof}
\begin{hw}
    prove the reverse implication.
\end{hw}

As a consequence, if a submanifold is totally geodesic, then the sectional
curvature of the submanifold is the same as the sectional curvature of the
submanifold with respect to the ambient space.

Note that if you take a slice of a Riemannian normal coordinate frame at a
point, the submanifold is geodesic at that point.
\end{document}
