\documentclass[../main.tex]{subfiles}
\begin{document}

\section{Curvature}

Let's just straight-up define the curvature: 

\begin{defn}
    Consider a Riemannian manifold $(M,g)$, with smooth vector fields
    $X,Y,Z\in\mathfrak{X}(M)$. We define
    \[
        R_m(X,Y)Z = -\nabla_X\nabla_YZ + \nabla_Y\nabla_XZ+\nabla_{[X,Y]}Z
    \]

    Alternately,
    \[
        R_{abc}^d\omega_d = \nabla_a\nabla_b\omega_c -
        \nabla_b\nabla_a\omega_c
    \]
    (Wald, p. 37)
\end{defn}

Now, we need to establish that this is a tensor by showing it is function linear
in each component.

Observe that
\[
    \begin{aligned}
        R_m(X,Y)fZ &= -\nabla_X\nabla_YfZ  + \nabla_Y\nabla_XfZ +
        \nabla_{[X,Y]}fZ\\
        &= -X(Yf)Z - (Yf)\nabla_XZ - (Xf)\nabla_YZ - f\nabla_X\nabla_YZ + Y(Xf)Z
        + (Xf)\nabla_YZ + Yf\nabla_XZ +f\nabla_Y\nabla_XZ + [X,Y]fZ +
        f\nabla_{[X,Y]}Z\\
        &= -f\nabla_X\nabla_YZ + f\nabla_Y\nabla_XZ + f\nabla_{[X,Y]}Z
    \end{aligned}
\]
as desired
\begin{hw}
    Show this is function-linear in other components.
\end{hw}

Note you can lower the contravariant index by applying $g_{ab}$ i.e.
\[
    R_{abcd} = g_{dd'}R_{abc}^{d'}
\]

\subsection*{Calculating Curvature}

We can calculate the Riemann curvature tensor in coordinates by using the
definitions of the covariant derivative.
\[
    \R_{abc}^d = \partial_b\Gamma^d_{ac} - \partial_a\Gamma^d_{bc}
    +\sum_{\alpha} (\Gamma^{\alpha}_{ac}\Gamma^d_{\alpha b} -
    \Gamma^{\alpha}_{bc}\Gamma^d_{\alpha a})
\]

To make things easier, we can use local Riemannian normal coordinates by pushing
the coordinates from $T_pM$ to $M$ via the exponential map.
\begin{hw}
    Show that in Riemannian normal coordinates,
    \[
        \Gamma^k_{ij} = 0 \text{ at }p
    \]
    and
    \[
        \partial_kg_{ij} = 0 \text{ at }p
    \]
\end{hw}

\begin{defn}
    an orthonormal frame $\{e_i\}$ on an open neighborhood of a point $p\in M$
    is called normal around $p$ if
    \[
        \nabla_ae_i = 0
    \]
    at $p$.
\end{defn}

The curvature follows the Bianchi Identity
\[
    R(X,Y)Z + R(Y,Z)X + R(Z,X)Y = 0
\]

In general, we have four important properties of the metric:
\begin{itemize}
    \item $R_{abc}^d = R_{[ab]c}^d$ antiymmetry of the first two components
    \item $R_{[abc]}^d = 0$ the Bianchi identity
    \item $R_{abcd} = R_{ab[cd]}$ antiymmetry of the second two components
    \item $R_{abcd} = R_{cdab}$ symmetry in the first and second half
        components.
\end{itemize}
Note that item $4$ can be derived from the other three.

An important concept not covered in Do Carmo:
\begin{defn}
    Given a finite dimensional vector space (over $\R$) $V$, consider the tensor $C$
    of rank $(0,4)$ (4 covariant indices). $C$ is called an algebraic curvature
    tensor on $V$ if it satisfies the above four properties (with appropriate
    index lowering).
\end{defn}

\subsection*{Sectional Curvature}

Let $p\in M$ and let $\sigma$ be a $2-$dimensional subspace of $T_pM$.
\begin{defn}
    The sectional curvature $K(\sigma)$ is defined to be
    \[
        K(\sigma) = R_m(e_1,e_2,e_1,e_2)
    \]
    for $\left\{ e_1,e_2 \right\}$ an orthonormal basis for $\sigma$.
\end{defn}

This definition is independent of choice of orthonormal basis by exploiting
linearity of $R_m$.

This can also be expressed in an arbitrary basis $u,v$ by
\begin{equation}
    K(\sigma) = \frac{R_m\left( u,v,u,v \right)}{\|u\wedge v\|^2}
\end{equation}
Where $\|u\wedge v\|^2$ is calculated from the inner product induced by the
metric. That is, for $\left\{ e_i \right\}$ an orthonormal basis for $V$, we
declare $\left\{ e_i\wedge e_j \right\}i<j$ to be orthonormal.

\begin{hw}
    Show that the induced inner product is independent of choice of orthonormal
    basis.
\end{hw}

\begin{lemma}
    Let $V$ be a vector space (finite dimensional, real) of dimension at least
    2 with an inner product. Consider two algebraic curvature tensors $C_1$ and
    $C_2$. Let $K_1,K_2$ denote the sectional curvatures of $C_1$ and $C_2$.
    $K_1 = K_2$ if and only if $C_1 = C_2$.
\end{lemma}

Suppose $C$ is such that $K(\sigma)=\kappa$ for all $\sigma$. Then,
\begin{equation}
    C(x,y,z,w) = \kappa\left( g(x,z)g(y,w) -g(x,w)g(y,z) \right)
\end{equation}

\subsection*{Ricci Curvature}
Let $R_m$ be a Riemannian curvature tensor, with components $R_{abc}^d$.
We can take the trace over the first and third components to get
\begin{equation}
    R_{ac} = R{abc}^b
\end{equation}

Geometrically, this is defined as
\begin{defn}
    $R_{C_p}(u,w) = \trace(R_{m_p}(u,\cdot)w)$.
\end{defn}

In an orthonormal frame with $g(e_j,e_k) = \delta_{jk}$, we have
\begin{equation}
    R_{ij} = R_{ikj}^k = R_{ikjk}
\end{equation}

We can also define the Ricci scalar
\begin{defn}
    $R = R_c(u,u)$ for unit vector $u$.
\end{defn}
This can be given in coordinates as
\begin{equation}
    R = R^i_i
\end{equation}



\end{document}
