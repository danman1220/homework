\documentclass[../main.tex]{subfiles}
\begin{document}

\section{Curvature}

Let's just straight-up define the curvature: 

\begin{defn}
    Consider a Riemannian manifold $(M,g)$, with smooth vector fields
    $X,Y,Z\in\mathfrak{X}(M)$. We define
    \[
        R_m(X,Y)Z = -\nabla_X\nabla_YZ + \nabla_Y\nabla_XZ+\nabla_{[X,Y]}Z
    \]

    Alternately,
    \[
        R_{abc}^d\omega_d = \nabla_a\nabla_b\omega_c -
        \nabla_b\nabla_a\omega_c
    \]
    (Wald, p. 37)
\end{defn}

Now, we need to establish that this is a tensor by showing it is function linear
in each component.

Observe that
\[
    \begin{aligned}
        R_m(X,Y)fZ &= -\nabla_X\nabla_YfZ  + \nabla_Y\nabla_XfZ +
        \nabla_{[X,Y]}fZ\\
        &= -X(Yf)Z - (Yf)\nabla_XZ - (Xf)\nabla_YZ - f\nabla_X\nabla_YZ + Y(Xf)Z
        + (Xf)\nabla_YZ + Yf\nabla_XZ +f\nabla_Y\nabla_XZ + [X,Y]fZ +
        f\nabla_{[X,Y]}Z\\
        &= -f\nabla_X\nabla_YZ + f\nabla_Y\nabla_XZ + f\nabla_{[X,Y]}Z
    \end{aligned}
\]
as desired
\begin{hw}
    Show this is function-linear in other components.
\end{hw}

Note you can lower the contravariant index by applying $g_{ab}$ i.e.
\[
    R_{abcd} = g_{dd'}R_{abc}^{d'}
\]

\subsection*{Calculating Curvature}

We can calculate the Riemann curvature tensor in coordinates by using the
definitions of the covariant derivative.
\[
    \R_{abc}^d = \partial_b\Gamma^d_{ac} - \partial_a\Gamma^d_{bc}
    +\sum_{\alpha} (\Gamma^{\alpha}_{ac}\Gamma^d_{\alpha b} -
    \Gamma^{\alpha}_{bc}\Gamma^d_{\alpha a})
\]

To make things easier, we can use local Riemannian normal coordinates by pushing
the coordinates from $T_pM$ to $M$ via the exponential map.
\begin{hw}
    Show that in Riemannian normal coordinates,
    \[
        \Gamma^k_{ij} = 0 \text{ at }p
    \]
    and
    \[
        \partial_kg_{ij} = 0 \text{ at }p
    \]
\end{hw}

\begin{defn}
    an orthonormal frame $\{e_i\}$ on an open neighborhood of a point $p\in M$
    is called normal around $p$ if
    \[
        \nabla_ae_i = 0
    \]
    at $p$.
\end{defn}

The curvature follows the Bianchi Identity
\[
    R(X,Y)Z + R(Y,Z)X + R(Z,X)Y = 0
\]

In general, we have four important properties of the metric:
\begin{itemize}
    \item $R_{abc}^d = R_{[ab]c}^d$ antiymmetry of the first two components
    \item $R_{[abc]}^d = 0$ the Bianchi identity
    \item $R_{abcd} = R_{ab[cd]}$ antiymmetry of the second two components
    \item $R_{abcd} = R_{cdab}$ symmetry in the first and second half
        components.
\end{itemize}

\end{document}
