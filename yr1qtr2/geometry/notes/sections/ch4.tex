\documentclass[../main.tex]{subfiles}
\begin{document}

\section{Curvature}

Let's just straight-up define the curvature: 

\begin{defn}
    Consider a Riemannian manifold $(M,g)$, with smooth vector fields
    $X,Y,Z\in\mathfrak{X}(M)$. We define
    \[
        R_m(X,Y)Z = -\nabla_X\nabla_YZ + \nabla_Y\nabla_XZ+\nabla_{[X,Y]}Z
    \]

    Alternately,
    \[
        R_{abc}^d\omega_d = \nabla_a\nabla_b\omega_c -
        \nabla_b\nabla_a\omega_c
    \]
    (Wald, p. 37)
\end{defn}

Now, we need to establish that this is a tensor by showing it is function linear
in each component.

Observe that
\[
    \begin{aligned}
        R_m(X,Y)fZ &= -\nabla_X\nabla_YfZ  + \nabla_Y\nabla_XfZ +
        \nabla_{[X,Y]}fZ\\
        &= -X(Yf)Z - (Yf)\nabla_XZ - (Xf)\nabla_YZ - f\nabla_X\nabla_YZ + Y(Xf)Z
        + (Xf)\nabla_YZ + Yf\nabla_XZ +f\nabla_Y\nabla_XZ + [X,Y]fZ +
        f\nabla_{[X,Y]}Z\\
        &= -f\nabla_X\nabla_YZ + f\nabla_Y\nabla_XZ + f\nabla_{[X,Y]}Z
    \end{aligned}
\]
as desired
\begin{hw}
    Show this is function-linear in other components.
\end{hw}

Note you can lower the contravariant index by applying $g_{ab}$ i.e.
\[
    R_{abcd} = g_{dd'}R_{abc}^{d'}
\]

\end{document}
