%%%%%%%%%%%%%%%%%%%%%%%%%%%%%%%%%%%%%%%%%
% Short Sectioned Assignment
% LaTeX Template
% Version 1.0 (5/5/12)
%
% This template has been downloaded from:
% http://www.LaTeXTemplates.com
%
% Original author:
% Frits Wenneker (http://www.howtotex.com)
%
% License:
% CC BY-NC-SA 3.0 (http://creativecommons.org/licenses/by-nc-sa/3.0/)
%
%%%%%%%%%%%%%%%%%%%%%%%%%%%%%%%%%%%%%%%%%

%----------------------------------------------------------------------------------------
%	PACKAGES AND OTHER DOCUMENT CONFIGURATIONS
%----------------------------------------------------------------------------------------

\documentclass[fontsize=11pt]{scrartcl} % 11pt font size

\usepackage[T1]{fontenc} % Use 8-bit encoding that has 256 glyphs
\usepackage[english]{babel} % English language/hyphenation
\usepackage{amsmath,amsfonts,amsthm} % Math packages
\usepackage{mathrsfs}

\usepackage[margin=1in]{geometry}

\usepackage{sectsty} % Allows customizing section commands
\allsectionsfont{\centering \normalfont\scshape} % Make all sections centered, the default font and small caps

\usepackage{fancyhdr} % Custom headers and footers
\pagestyle{fancyplain} % Makes all pages in the document conform to the custom headers and footers
\fancyhead{} % No page header - if you want one, create it in the same way as the footers below
\fancyfoot[L]{} % Empty left footer
\fancyfoot[C]{} % Empty center footer
\fancyfoot[R]{\thepage} % Page numbering for right footer
\renewcommand{\headrulewidth}{0pt} % Remove header underlines
\renewcommand{\footrulewidth}{0pt} % Remove footer underlines
\setlength{\headheight}{13.6pt} % Customize the height of the header

\numberwithin{equation}{section} % Number equations within sections (i.e. 1.1, 1.2, 2.1, 2.2 instead of 1, 2, 3, 4)
\numberwithin{figure}{section} % Number figures within sections (i.e. 1.1, 1.2, 2.1, 2.2 instead of 1, 2, 3, 4)
\numberwithin{table}{section} % Number tables within sections (i.e. 1.1, 1.2, 2.1, 2.2 instead of 1, 2, 3, 4)

\newcommand{\R}{\mathbb{R}}
\newcommand{\Q}{\mathbb{Q}}
\newcommand{\N}{\mathbb{N}}
\newcommand{\C}{\mathbb{C}}

\newtheorem{lemma}{Lemma}
%----------------------------------------------------------------------------------------
%	TITLE SECTION
%----------------------------------------------------------------------------------------

\newcommand{\horrule}[1]{\rule{\linewidth}{#1}} % Create horizontal rule command with 1 argument of height

\title{	
\normalfont \normalsize 
\textsc{<<CLASS>>} \\ [25pt] % Your university, school and/or department name(s)
\horrule{0.5pt} \\[0.4cm] % Thin top horizontal rule
\huge <<TITLE>> \\ % The assignment title
\horrule{2pt} \\[0.5cm] % Thick bottom horizontal rule
}

\author{Daniel Halmrast} % Your name

\date{\normalsize\today} % Today's date or a custom date

\begin{document}

\maketitle % Print the title

% Problems
\section*{Problem 1}
Show that hyperbolic space $H^n$ is complete.
\\
\\
\begin{proof}
    We will first show that $H^n$ is homogeneous, and then appeal to the next
    problem to conclude $H^n$ is complete.

    To see that $H^n$ is homogeneous, we consider two families of isometries.
    For simplicity, we will write points in $H^n$ as $(x,y)$ with $x\in
    \R^{n-1}$ the first $n-1$ coordinates, and $y\in\R$ the last coordinate.
    The first isometry we consider is
    \[
        \begin{aligned}
        T_a&:H^n\to H^n\\
        (x,y)&\mapsto(x+a,y)
    \end{aligned}
    \]
    for any $a\in\R^{n-1}$. To see this is an isometry, we just need to compute
    $dT_a$ and show it preserves the metric. So, let $v\in T_pH^n$ for some
    $p\in H^n$, $p=(x_p,y_p)$, and take $\gamma(t) = p+vt = (x_p+v_xt,y_p+v_yt)$
    a curve in $H^n$.  Note that $\gamma'(0) = v$. Now, we have that
    \[
        \begin{aligned}
        dT_a(v) &= dT_a(\gamma'(0))\\
        &= \partial_t T_a(\gamma(t))|_{t=0}\\
        &=\partial_t (x_p +v_xt+a,y_p+v_yt)|_{t=0}\\
        &=(v_x,v_y) = v
    \end{aligned}
    \]
    Thus, $dT_a(v)=v$. Furthermore, since the metric at $(x+a,y)$ is the same as
    at $(x,y)$ (since the scaling factor only depends on $y$) we have that for
    $u,v\in T_pM$,
    \[
\begin{aligned}
    g(u,v)_{(x,y)} &= g(dT_au,dT_av)_{(x+a,y)}
\end{aligned}
    \]
    and thus $T_a$ is an isometry (I suppose you'd have to check that $T_a$ is a
        diffeomorphism as well, but this is obvious. Clearly $T_a$ is smooth,
    and it has a smooth inverse $T_{-a}$).

    Secondly, we consider the isometry
    \[
        \begin{aligned}
        M_{\alpha}&:H^n\to H^n\\
        (x,y)&\mapsto (\alpha x, \alpha y)
    \end{aligned}
    \]
    for $\alpha>0$. This maps $H^n$ into $H^n$, since it keeps the $y$
    coordinate positive. Furthermore, it is a diffeomorphism (it is clearly
    smooth, and $M_{\frac{1}{\alpha}}$ acts as an inverse). I also claim it is
    an isometry. Again letting $\gamma = (x_p + v_x,y_p+v_y)$ for $(v_x,v_y)\in
    T_{(x,y)}H^n$ we note that
    \[
        \begin{aligned}
            dM_{\alpha}(v) &= dM_{\alpha}(\gamma'(0))\\
            &= \partial_t M_{\alpha}(\gamma(t))|_{t=0}\\
            &= \partial_t (\alpha(x_p + v_x),\alpha(y_p+v_y))|_{t=0}\\
            &= \alpha V
        \end{aligned}
    \]
    Finally, we compute the metric
    \[
\begin{aligned}
    g(u,v)_(x,y) &= g_{ab}u^av^b\\
    &=\frac{1}{y^2}u_bv^b\\
    \\
    g(dM_{\alpha}u,dM_{\alpha}v)_{(\alpha x,\alpha y)} &= g_{ab}\alpha u^a\alpha
    v^b\\
    &= \frac{1}{(\alpha y)^2}\alpha^2 u_bv^b\\
    &= \frac{1}{y^2}u_bv^b
\end{aligned}
    \]
    Where $u_b = \eta_{ab}u^a$ and so $u_bv^b$ is the standard inner product on
    $\R^n$. Thus, $M_{\alpha}$ is an isometry.

    I assert that the action of these two isometries is transitive. Indeed,
    given $(x,y)$ and $(x',y')$ in $H^n$, we construct the isometry as follows.
    First, apply $T_{-x}$ to map $(x,y)$ to $(0,y)$. Then, apply
    $M_{\frac{y'}{y}}$ to map $(0,y)$ to $(0,y')$. Finally, apply
    $T_{x'}$ to map $(0,y')$ to $(x',y')$.

    Thus, for any two points $(x,y)$ and $(x',y')$ in $H^n$, there is an
    isometry connecting them. Thus, by the result of the next problem, $H^n$ is
    complete.
\end{proof}

\newpage

\section*{Problem 2}
Show that a homogeneous space is complete.
\\
\\
\begin{proof}
    Let $M$ be a homogeneous manifold. We will show that $M$ is geodesically
    complete.

    Let $\varepsilon$ be such that $B_{\varepsilon}(p)\subset M$ is a normal
    ball at $p\in M$. Since $M$ is homogeneous, this implies that
    $B_{\varepsilon}(q)$ is a normal ball at $q\in M$ for any other $q$. 
    To see this, we note that for $\phi$ the isometry sending $p$ to $q$,
    \[
        \phi\circ\exp_p\circ d\phi^{-1}
    \]
    defines a diffeomorphism between $B_{\varepsilon}(0)\subset T_qM$ and the
    image $B_{\varepsilon}(q)$. This is well-defined, since $\phi$ is an
    isometry, so $\|v\| = \|d\phi^{-1}v\|$. Furthermore, we can see that
    $\exp_q=\phi\circ\exp_p\circ d\phi^{-1}$. Obesrve that $\gamma(t) =
    \exp_q(tv)$ is the unique geodesic through $q$ with tangent vector $v$.
    However,
    \[
        \tilde{\gamma}(t) = \phi\circ\exp_p\circ d\phi^{-1}(tv)
    \]
    has the same properties. Namely $\tilde{\gamma}(0) = \phi(p)=q$, and
    $\tilde{\gamma}'(0)= d\phi(d\phi^{-1}(v)) = v$. Thus, $\tilde{\gamma}(t) =
    \gamma(t)$ for all $t\in[0,1]$, and so $\exp_q$ and $\phi\circ\exp_p\circ
    d\phi^{-1}$ agree at all points in the normal ball. Thus,
    $B_{\varepsilon}(q)$ is a normal ball, as desired.

    Recall that in a normal ball at $p$, any geodesic going through $p$ can be
    extended throughout the entire normal ball. This follows from the fact that
    if $\gamma$ is a geodesic passing through $p$ at some time $t_p$ with
    $\gamma'(t_p)=v$, it is the unique geodesic (up to reparameterization) with
    $\gamma(t_p)=p$ and $\gamma'(t_p)=v$. Now, since radial geodesics through
    $p$ are defined on the entire normal ball, the radial geodesic starting at
    $p$ with tangent vector $v$ is defined throughout the normal ball, and is an
    extension of $\gamma$. Thus, $\gamma$ can be extended through the normal
    ball.

    It follows immediately, then, that any geodesic $\gamma$ (with unit speed,
    without loss of generality) defined on some
    interval $(a,b)$ can be extended to a geodesic defined on
    $(a,b+\frac{\varepsilon}{2})$ by observing that $\gamma$ passes through
    $\gamma(b-\frac{\varepsilon}{2})$, and since
    $\gamma(b-\frac{\varepsilon}{2})$ has a normal ball of radius $\varepsilon$
    around it, we know that $\gamma$ can be extended through this normal ball to
    be defined on $(a,b-\frac{\varepsilon}{2} + \varepsilon) =
    (a,b+\frac{\varepsilon}{2})$.

    Thus, it follows immediately that geodesics can be extended indefinitely
    (the symmetric argument works to show $\gamma$ can be extended the other
    way) and thus $M$ is geodesically complete.
\end{proof}

\newpage

\section*{Problem 3}
\subsection*{Part a}
Let $v$ be a linear field on $\R^n$. That is, $v$ is a vector field, and $v$ is
linear when thought of as a map from $\R^n$ to $\R^n$. Show that a linear field
given by a matrix $A$ is a killing field if and only if $A$ is antisymmetric.
\\
\\
\begin{proof}
    Let $X$ be a linear vector field. Then, $X$ is expressible as a matrix $A$.
    That is, $X(f(x_1,\dots,x_n)) = Af(x_1,\dots,x_n)$. In order for $X$ to be a
    killing field, we must have that its local flow around each point is an
    isometry. That is, for $\phi:(-\varepsilon,\varepsilon)\times U\to M$ the
    flow of $X$ around a point $p$, $d\phi(t,\cdot)$ preserves inner products.

    Now, the flow of $X$ is the solution to
    \[
        \partial_t\phi^a = A\phi^a
    \]
    which is solved by setting $\phi = \exp(At)$. Now, let's calculate the
    differential. For $v\in T_p\R^n$, let $\gamma(s) = p+sv$. Then
    \[
        \begin{aligned}
            d\phi(v) &= \partial_s\phi(\gamma(s))|_0\\
            &= \partial_s\exp(At)(p+sv)\\
            &=\exp(At)v
        \end{aligned}
    \]
    and so $d\phi = \phi$. We require that
    \[
        \langle u,v\rangle = \langle \exp(At)u,\exp(At)v\rangle
    \]
    which amounts to requiring
    \[
        \langle u,v\rangle = \langle u,\exp(A^Tt)\exp(At)v\rangle
    \]
    Now, this happens for all $u,v$ if and only if $\exp(A^Tt)\exp(At)=I$, which
    holds for all $t$ if and only if $A^T=-A$. Thus, in order for $X$ to be a
    killing field, $A$ must be antisymmetric, and vice versa.
\end{proof}

\subsection*{Part b}
Let $X$ be a killing field on $M$ with $p\in M$, and let $U$ be a normal
neighborhood of $p$ in $M$. Assume that $p$ is a unique point of $U$ with
$X_p=0$. Show that in $U$, $X$ is tangent to the geodesic spheres centered at
$p$.
\\
\\
\begin{proof}
    Let $\phi_q:(-\varepsilon,\varepsilon)\times V_q\to M$ denote the local flow
    of $X$ around any point $q$. Since $X_p = 0$, we know that $\phi(t,p) = p$.
    That is, $p$ is fixed by the flow of $X$. 

    Now, let $q$ be any point in $U$ the normal neighborhood of $p$. We know
    that there is a unique radial geodesic from $p$ to $q$ defined as $\gamma(t)
    = \exp_p(tv)$ for some $v$. Now, $\phi$ is defined
    across all of $\gamma(t)$ for $t\in[0,1]$ since $\gamma([0,1])$ is a compact
    set, and thus can be covered by a finite number of sets $V_q$ on which the
    flow is defined.

    Now, since $\phi(t,\cdot)$ is an isometry, it maps geodesics to geodesics.
    Thus, the image $\phi(t,\gamma([0,1]))$ is a geodesic from $\phi(t,p)=p$ to
    $\phi(t,q)$. Furthermore, this geodesic is defined by $\tilde{\gamma(t)} =
    \exp_p(tu)$ for some $u$. Now, we know that
    \[
\begin{aligned}
    d\phi(t,v) &= d\phi(t,\gamma'(0))\\
    &= \partial_s\phi(t,\gamma(s))\\
    &= \partial_s(\tilde{\gamma}(s))\\
    &= u
\end{aligned}
    \]
    Thus $u$ and $v$ have the same norm, and so $\gamma(1)=q$ and
    $\tilde{\gamma}(1)=\phi(t,q)$ are the same distance from $q$.

    Thus, $\phi$ moves points along the geodesic spheres, and so $X$ is
    tangential to the geodesic spheres, as desired.
\end{proof}

\subsection*{Part c}
Let $X$ be a smooth vector field on $M$ and let $f:M\to N$ be an isometry. Let
$Y$ be a vector field on $N$ defined by $Y(f(p))=df_p(X(p))$. Prove that $Y$ is
a killing field if and only if $X$ is.
\\
\\
\begin{proof}
    Suppose $X$ is a killing field. That is, the local flow $\phi$ is an
    isometry. Now, we can push forward a local flow on $X$ to a local flow on
    $Y$. That is,
    \[
        \psi(t,x) = f(\phi(t,f^{-1}(x)))
    \]
    defines a flow on $Y$. This is clear, since
    \[
        \begin{aligned}
        \partial_t\psi(t,x) &= \partial_t f(\phi(t,f^{-1}(x)))\\
        &= df(\partial_t\phi(t,f^{-1}(x)))\\
        &= df(X(\phi(t,f^{-1}(x))))\\
        &= Y(f(\phi(t,f^{-1}(x))))\\
        &= Y(\psi(t,x))
    \end{aligned}
    \]
    as desired. Now, for any fixed $t$, $\psi(t,\cdot) =
    f\circ\phi(t,\cdot)\circ f^{-1}$ is the composition of isometries, and is
    therefore an isometry as desired. Thus, $Y$ is a killing field if $X$ is.

    By symmetry of the problem, this implies that $Y$ is a killing field if and
    only if $X$ is.
\end{proof}

\subsection*{Part d}
Show that $X$ is a killing field if and only if
\[
    g(\nabla_YX,Z) + g(\nabla_ZX,Y)=0
\]
for all $X,Y,Z$. 
\\
\\
\begin{proof}
    Recall the definition of the Lie derivative of a tensor field $T$ along a
    vector field $v$ with flow $\phi$
    \[
        \mathfrak{L}_v(T)(p) = \lim_{t\to 0}\left\{
        \frac{\phi^*(-t,T(\phi(t,p)))-T(p)}{t} \right\}
    \]
    We first establish that the Lie derivative along a vector field $v$ of the
    metric $g$ is zero if and only if $v$ is a killing field.
    To see this, let $p\in M$, and choose a coordinate system $x_i$ around $p$
    that is compatible with the flow of $v$. That is, 
    \[
        \phi(t,(x_1,\dots,x_n)) = \phi(x_1+t,x_2,\dots,x_n)
    \]
    Which can be done by setting $x_1$ so that $v = \partial_1$ around $p$. (I
        suppose we are implicitly assuming $v\neq 0$ around and at $p$. We will
    handle the case $v=0$ around $p$ later).
    
    In this coordinate system, we have
    \[
        \mathfrak{L}_v(g)(p) = \partial_1(g)|_p
    \]
    which is zero if and only if $g$ does not vary along the flow $\phi$. In
    particular, this means that $v$ is a killing field at $p$ (its flow is a
    local isometry) if and only if $\mathfrak{L}_v(g)(p) = 0$.

    This condition is exactly the killing equation. To see this, we first
    establish a different form of the Lie derivative. Recall that for any vector
    field $u$ we have
    \[
        \mathfrak{L}_v(u) = [v,u]
    \]
    and for any function $f$ we have
    \[
        \mathfrak{L}_v(f) = v(f)
    \]
    These conditions along with the Leibniz rule allow us to characterize
    $\mathfrak{L}_v$ for more arbitrary tensors. Let $\omega$ be a one-form, and
    $u$ a vector field. We have
    \[
\begin{aligned}
    \mathfrak{L}_v(\omega_iu^i) &= v(\omega_iu^i)
\end{aligned}
    \]
    and
    \[
        \mathfrak{L}_v(\omega_iu^i) = u^i\mathfrak{L}_v(\omega_i) +
        \omega_i[v,u]^i
    \]
    setting these equal, we have
    \[
        \begin{aligned}
            v(\omega_iu^i) &= u^i\mathfrak{L}_v(\omega_i) + \omega_i[v,u]^i\\
            \nabla_v(\omega_iu^i) &= u^i\mathfrak{L}_v(\omega_i) +
            \omega_i[v,u]^i\\
            u^i\nabla_v\omega_i + \omega_i\nabla_vu^i
            &= u^i\mathfrak{L}_v(\omega_i) + \omega_i(\nabla_vu^i -
            \nabla_uv^i)\\
            u^i\nabla_v\omega_i &= u^i\mathfrak{L}_v(\omega_i) -
            \omega_i\nabla_uv^i\\
            \mathfrak{L}_v(\omega_i)u^i &= u^iv^j\nabla_j\omega_i +
            \omega_iu^j\nabla_jv^i\\
            \mathfrak{L}_v(\omega_i) &= v^j\nabla_j\omega_i +
            \omega_j\nabla_iv^j\\
        \end{aligned}
    \]
    and we extend this definition inductively to get
    \[
        \mathfrak{L}_v(g_{ab}) = v^c\nabla_cg_{ab} + g_{ac}\nabla_b v^c +
        g_{cb}\nabla_a v^c
    \]
    and since $\nabla$ is metric compatible, $\nabla_cg_{ab} = 0$. Then,
    \[
        \mathfrak{L}_v(g_{ab}) = g_{ac}\nabla_bv^c+g_{cb}\nabla_av^c
        = \nabla_bv_a + \nabla_av_b
    \]
    Thus, $v$ is a killing field if and only if
    \[
        \nabla_bv_a + \nabla_av_b = 0
    \]
    which is the local coordinate version of
    \[
        g(\nabla_Yv,Z) + g(Y,\nabla_Zv) = 0
    \]
    To see this, set $Y=\partial_a$, $Z=\partial_b$, and compute
    \[
        \begin{aligned}
        g(\nabla_Yv,Z) + g(Y,\nabla_Zv)
        &= g_{cd}\nabla_av^c(\partial_b)^d + g_{cd}(\partial_a)^c,\nabla_bv^d\\
        &=g_{cb}\nabla_av^c + g_{ad}\nabla_bv^d\\
        &=\nabla_av_b + \nabla_bv_a
    \end{aligned}
    \]
    as desired. Thus, $v$ is a killing field if and only if it satisfies the
    killing equation.

    Note that in the case $v=0$ around $p$, then $v$ trivially satisfies the
    killing equation, and $v$ is a killing field around $p$, since  the flow of
    $v$ is stationary.
\end{proof}

\subsection*{Part e}
Let $X$ be a killing field with $X(q)\neq 0$  for some $q\in M$. Prove that
there exist coordinates around $q$ for which the coefficients $g_{ij}$ of the
metric do not depend on one of the coordinates.
\\
\\
\begin{proof}
    We assumed this in the previous part, but we will prove it more rigorously
    now. Let $v=X(q)$, and form a coordinate system $\partial_i$ with
    $\partial_n = v$ in $T_qM$. This can be done via a coordinate mapping
    $\psi:M\to\R^n$ which sends the integral curve of $X$ through $q$ to the
    $n$th coordinate axis.

    Now, since $X$ is a killing field, its flow is a local isometry. So, take a
    small neighborhood $U$ of $q$ such that our coordinate system is defined on
    $U$ and the flow of $X$ is an isometry on $U$. Now, by our choice of
    coordinate system, the flow $\phi$ of $X$ is defined in these coordinates as
    \[
        \phi(t,(x_1,\dots,x_n)) = (x_1,\dots,x_n+t)
    \]
    Thus, since $\phi$ is an isometry, $g$ at $(x_1,\dots,x_n)$ is the same as
    $g$ at $(x_1,\dots,x_n+t)$ for $t\in (-\varepsilon,\varepsilon)$ as desired.
\end{proof}

\newpage

\section*{Problem 4}
We can define a metric on $TM$ by the following: Let $(p,v)\in TM$ and $W,V\in
T_vTM$. Let $\alpha,\beta$ be curves in $TM$ with $\alpha(0)=\beta(0)=(p,v)$ and
$\alpha'(0)=V,\beta'(0)=W$.
Define the metric on $TM$ as
\[
    g_T(V,W)_{(p,v)} = g(d\pi(V),d\pi(W))_p +
    g(\nabla_t\alpha|_0,\nabla_s\beta|_0)_p
\]
where by $\nabla_t\alpha$ we mean the covariant derivative along $\pi(\alpha)$
of the vector field $\alpha(t)$.

\subsection{Part a}
Prove that this metric is well-defined as a Riemannian metric.
\\
\\
\begin{proof}
    We first show that this definition is independent of choice of curves
    $\alpha,\beta$. This follows from the fact that $\nabla_t\alpha|_0$ only
    depends on $\alpha(0)$ and $\alpha'(0)$, which follows immediately from the
    definition of $\nabla_t$.

    Now, all we need to show is that this is a metric. That is, we need to show
    $g_T$ is smooth, symmetric, bilinear, and positive-definite.

    Clearly, $g_T$ is smooth as the composition of a bunch of smooth functions
    ($\pi:TM\to M$, $g$, and $\nabla$ are all smooth, and $g_T$ is the sum of
    compositions of these)

    Furthermore, $g_T$ is symmetric, since each component in its sum is
    symmetric.

    $g_T$ is bilinear, since $g$ is bilinear, and $d\pi$ and $\nabla_t$ are both
    linear.

    Finally, we observe that $g_T$ is positive definite. We know that $g_T$ is
    always positive (or zero) since it is the sum of positive terms. So, we only
    have to check that
    \[
        g_T(V,V)=0
    \]
    implies that $V=0$. Note that if $g_T(V,V)=0$, we know that
    \[
\begin{aligned}
    g(d\pi(V),d\pi(V)) = 0 \implies d\pi(V)=0\\
    g(\nabla_t\alpha,\nabla_t\alpha) = 0\implies \nabla_t\alpha|_0 = 0\\
\end{aligned}
    \]
    The first statement means that $\partial_t\pi(\alpha(t))=\partial_t(p(t))$
    is zero, and so $V$ must be a vertical vector. However, the second statement
    implies that the vertical part of $V$ is zero, and so $V$ itself must be
    zero. Thus, $g_T$ is positive-definite as desired.

    Thus, $g_T$ is a well-defined metric on $TM$.
\end{proof}

\subsection*{Part b}
Prove that the curve $t\mapsto (p(t),v(t))$ is horizontal if and only if $v(t)$
is parallel along $p(t)$ in $M$.
\\
\\
\begin{proof}
    Let $\gamma(t) = (p(t),v(t))$ be a horizontal curve. Since this is a
    horizontal curve, we know that $\gamma'(t)$ is orthogonal to the fiber
    $\pi^{-1}(p(t))$. That is, we know that for any vertical vector $W$, we have
    \[
        g_T(\gamma'(t),W) = 0
    \]
    Since $W$ is vertical, this implies that $d\pi(W) = 0$ and thus
    \[
        g_T(\gamma'(t),W) = g(\nabla_t\gamma(0),\nabla_s\beta(0))
    \]
    for some curve $\beta(s)$ with $\beta(0) = \gamma(0)$ and $\beta'(0)=W$.

    Now, since $W$ is vertical and nonzero, we know that $\nabla_s\beta(0)$ is
    nonzero (this follows from positive-definiteness of $g_T$ and by the fact
        that $d\pi(W) = 0$, which means that $g_T(W) =
    g(\nabla_s\beta(0),\nabla_s\beta(0))$ which is nonzero).

    Thus, the only way that $g(\nabla_t\gamma(0),\nabla_s\beta(0))$ is zero for
    all $t$ is if $\nabla_t\gamma(0)$ is zero. However, this is just the
    statement that the vector field $v(t)$ is parallel along $p(t)$ as desired.

    Conversely, let $\gamma(t)=(p(t),v(t))$ be such that $v(t)$ is parallel along
    $p(t)$, and let $W$ be a vertical vector at $p(t)$. Then, we have
    \[
        g_T(\gamma'(t),W) = g(d\pi(\gamma'(t)),d\pi(W)) +
        g(\nabla_t\gamma(0),\nabla_s\beta(0))
    \]
    where $\beta$ is defined in the same way as before. Since $W$ is vertical,
    $d\pi(W)=0$, and since $v(t)$ is parallel to $p(t)$, $\nabla_t\gamma(0)=0$.
    Thus,
    \[
        g_T(\gamma'(t),W) = 0
    \]
    and $\gamma$ is a horizontal curve as desired.
\end{proof}

\subsection{Part c}
Prove that the geodesic field is a horizontal vector field.
\\
\\
\begin{proof}
    Let $G$ be the geodesic field on $TM$. That is, $G$ has trajectories
    $t\mapsto (\gamma(t),\gamma'(t))$ for geodesics $\gamma$.
    We will prove that this field is parallel. However, this follows almost
    immediately from the definition of a geodesic. By definition, $\gamma$ is a
    geodesic if and only if $\nabla_t\gamma'(s) = 0$ for all $s$. This means
    that $\gamma'(t)$ is parallel along $\gamma$, and by the result in part b,
    we know that $G$ is a horizontal vector field.
\end{proof}

\subsection*{Part d}
Prove that the trajectories of the geodesic field are geodesics on $TM$ with the
metric $g_T$.
\\
\\
\begin{proof}
    Let $\alpha(t) = (\gamma(t),\gamma'(t))$ be an integral curve of $G$. We
    wish to show this is a geodesic with the metric $g_T$. This amounts to
    showing
    \[
        \nabla_t\alpha' = 0
    \]
    under the induced metric. Now, since $\alpha$ is an integral curve of $G$,
    by part c we know that $\alpha$ is a horizontal curve. This implies that
    \[
        g_T(\cdot,\alpha'(t)) = g(d\pi(\cdot),d\pi(\alpha'(t)))
    \]
    (since the vertical part is zero). Thus, along $\alpha$, $g_T = \pi^*(g)$.
    
    More formally, $\alpha$ lies in the submanifold of $TM$ consisting of all
    points horizontal to $\alpha(0)$. On this submanifold, the metric looks like
    $g_T = \pi^*(g)$, which is well-defined, since $\pi$ is bijective on this
    submanifold. Thus, geodesics in this submanifold are precisely the image
    under $\pi^{-1}$ of geodesics in $M$. Since $\alpha = \pi^{-1}(\gamma)$
    which is a geodesic, it follows that $\alpha$ is a geodesic as well.
\end{proof}

\newpage

\section*{Problem 5}
Let $M$ be a Riemannian manifold of dimension $2$. Let $B_{\delta}(p)$ be a
normal ball around $p\in M$, and consider the parameterized surface
\[
    f(\rho,\theta) = \exp_p(\rho v(\theta))
\]
where $v(\theta)$ is a circle in $B_{\delta}(0)$ parameterized by the central
angle $\theta$. 

\subsection*{Part a}
Show that $(\rho,\theta)$ are coordinates in an open set $U\subset M$ formed by
the open ball $B_{\delta}(p)$ minus the ray $\exp_p(-\rho v(0))$ for
$\rho\in(0,\delta)$.
\\
\\
\begin{proof}
    The fact that this is a coordinate system is clear. Since $U$ is contained
    in a normal ball, $\exp_p$ is a diffeomorphism, and thus the inverse
    $\exp_p^{-1}$ is a diffeomorphism into $B_{\delta}(0)\subset T_pM\cong \R^2$.

    Since $\rho,\theta$ define a coordinate system on $B_{\delta}(0)$ (as polar
    coordinates in $\R^2$), their image under $\exp_p$ is also a coordinate
    system of $M$, as desired.
\end{proof}

\subsection*{Part b}
Show that the coefficients of the metric $g_{ij}$ are
\[
    \begin{aligned}
        g_{12}=0,& &g_{11}=\|\partial_{\rho}f\|^2 = \|v(\theta)\|^2=1,&
        &g_{22}=\|\partial_{\theta}\|^2
    \end{aligned}
\]

\begin{proof}
    We first observe that $\partial_{\rho}$ is a geodesic, and so
    $\nabla_{\rho}\partial_{\rho} = 0$. In particular, this means that
    \[
        \partial_{\rho}g(\partial_\rho,\partial_{\rho}) =
        2g(\partial_{\rho},\nabla_{\rho}\partial_{\rho}) = 0
    \]
    and so $g_{\rho,\rho}$ does not change along $\rho$. Since $\|\partial_r\| =
    g_{\rho,\rho} = 1$ at the origin, this implies that $g_{\rho,\rho}=1$
    everywhere.

    Note also that by the Gauss lemma, geodesics from $p$ are orthogonal to the
    geodesic spheres centered at $p$, and in
    particular this means that $g(\partial_{\rho},\partial_{\theta}) = 0$. Thus,
    $g_{12}=g_{21}=0$ as desired.

    Finally, we wish to compute $g(\partial_{\theta},\partial_{\theta})$.
    Since $\exp_p$ is an isometry around $p$, we can pull back $g$ to find
    $g(\partial_{\theta},\partial_{\theta}) =
    f^*(g(\partial_{\theta},\partial_{\theta}))$ which yields
        \[
            f^*(g(\partial_{\theta},\partial_{\theta})) =
            g(\partial_{\theta}f,\partial_{\theta}f) = \|\partial_{theta}f\|^2
        \]
    as desired.
\end{proof}

\subsection*{Part c}
Show that along the geodesic $f(\rho,0)$ we have
\[
\begin{aligned}
    (\sqrt{g_{22}})_{\rho\rho} = -K(p)\rho + R(\rho)
\end{aligned}
\]
where $R(\rho) = 0 + O(\rho^2)$.
\\
\\
\begin{proof}
    
\end{proof}<++>


\end{document}
