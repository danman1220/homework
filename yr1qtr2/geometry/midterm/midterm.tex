%%%%%%%%%%%%%%%%%%%%%%%%%%%%%%%%%%%%%%%%%
% Short Sectioned Assignment
% LaTeX Template
% Version 1.0 (5/5/12)
%
% This template has been downloaded from:
% http://www.LaTeXTemplates.com
%
% Original author:
% Frits Wenneker (http://www.howtotex.com)
%
% License:
% CC BY-NC-SA 3.0 (http://creativecommons.org/licenses/by-nc-sa/3.0/)
%
%%%%%%%%%%%%%%%%%%%%%%%%%%%%%%%%%%%%%%%%%

%----------------------------------------------------------------------------------------
%	PACKAGES AND OTHER DOCUMENT CONFIGURATIONS
%----------------------------------------------------------------------------------------

\documentclass[fontsize=11pt]{scrartcl} % 11pt font size

\usepackage[T1]{fontenc} % Use 8-bit encoding that has 256 glyphs
\usepackage[english]{babel} % English language/hyphenation
\usepackage{amsmath,amsfonts,amsthm} % Math packages
\usepackage{mathrsfs}

\usepackage[margin=1in]{geometry}

\usepackage{sectsty} % Allows customizing section commands
\allsectionsfont{\centering \normalfont\scshape} % Make all sections centered, the default font and small caps

\usepackage{fancyhdr} % Custom headers and footers
\pagestyle{fancyplain} % Makes all pages in the document conform to the custom headers and footers
\fancyhead{} % No page header - if you want one, create it in the same way as the footers below
\fancyfoot[L]{} % Empty left footer
\fancyfoot[C]{} % Empty center footer
\fancyfoot[R]{\thepage} % Page numbering for right footer
\renewcommand{\headrulewidth}{0pt} % Remove header underlines
\renewcommand{\footrulewidth}{0pt} % Remove footer underlines
\setlength{\headheight}{13.6pt} % Customize the height of the header

\numberwithin{equation}{section} % Number equations within sections (i.e. 1.1, 1.2, 2.1, 2.2 instead of 1, 2, 3, 4)
\numberwithin{figure}{section} % Number figures within sections (i.e. 1.1, 1.2, 2.1, 2.2 instead of 1, 2, 3, 4)
\numberwithin{table}{section} % Number tables within sections (i.e. 1.1, 1.2, 2.1, 2.2 instead of 1, 2, 3, 4)

\newcommand{\R}{\mathbb{R}}
\newcommand{\Q}{\mathbb{Q}}
\newcommand{\N}{\mathbb{N}}
\newcommand{\C}{\mathbb{C}}

\newtheorem{lemma}{Lemma}
%----------------------------------------------------------------------------------------
%	TITLE SECTION
%----------------------------------------------------------------------------------------

\newcommand{\horrule}[1]{\rule{\linewidth}{#1}} % Create horizontal rule command with 1 argument of height

\title{	
\normalfont \normalsize 
\textsc{Geometry} \\ [25pt] % Your university, school and/or department name(s)
\horrule{0.5pt} \\[0.4cm] % Thin top horizontal rule
\huge Midterm \\ % The assignment title
\horrule{2pt} \\[0.5cm] % Thick bottom horizontal rule
}

\author{Daniel Halmrast} % Your name

\date{\normalsize\today} % Today's date or a custom date

\begin{document}

\maketitle % Print the title

% Problems
\section*{Problem 1}
Consider a Riemannian manifold $(M,g)$ of dimension $n$.Let $\{e_i\}$ be a local
orthonormal frame on $u\subset M$, and let $\{\omega_i\}$ be the dual basis.
Prove that there is a unique set of smooth $1$-forms $\omega^j_i$ such that
\[
    d\omega^i = \omega^k\wedge\omega_k^i
\]
and
\[
    \omega_i^j + \omega^j_i = 0
\]
without appealing to the Levi-Civita connection.
\\
\\
\begin{proof}
    We note first that the set $\{\omega^i\wedge\omega^j\}_{i<j}$ forms a basis
    for the second exterior power $\Lambda^2U$. Thus, we can express
    \[
        d\omega^i = \frac{1}{2}a_{\alpha\beta}{}^i\omega^{\alpha}\wedge\omega^{\beta}
    \]
    For convenience in notation, we will let $\alpha,\beta$ both run up to $n$,
    and require that $a_{\alpha\beta}{}^i$ be antisymmetric in its lower two
    indices. That is,
    \[
        \begin{aligned}
            d\omega^i &= \sum_{\alpha<\beta}
            (\frac{1}{2}a_{\alpha\beta}{}^i -
            \frac{1}{2}a_{\beta\alpha}{}^i)\omega^{\alpha}\wedge\omega^{\beta}\\
        &= \sum_{\alpha<\beta}
        a_{\alpha\beta}{}^i\omega^{\alpha}\wedge\omega^{\beta}
    \end{aligned}
    \]

    Now, we define $\omega^j_i$ as follows.
    \[
        \omega_{ij} = -\frac{1}{2}(a_{ij\alpha} + a_{i\alpha j} - a_{j\alpha
        i})\omega^{\alpha}
    \]
    Now, $a$ is antisymmetric in its first two indices, so
    \[
        \begin{aligned}
        \omega_{ij} &= -\frac{1}{2}(a_{ij\alpha} + a_{i\alpha j} - a_{j\alpha
        i})\omega^{\alpha}\\
        &= -\frac{1}{2}(-a_{ji\alpha} - a_{j\alpha i} + a_{i\alpha
        j})\omega^{\alpha}\\
        &= \frac{1}{2}(a_{ji\alpha} + a_{j\alpha i} -a_{i\alpha
        j})\omega^{\alpha}\\
        &= -\omega_{ji}
    \end{aligned}
    \]
    and so the family of one-forms $\omega_{ij}$ is antisymmetric in its
    indices.

    Next, we wish to show that the equation
    \[
        d\omega^i = \omega^k\wedge\omega_k^i
    \]
    holds for this definition of $\omega_k^i$. Allowing for somewhat haphazard
    placement of indices, we know that
    \[
        \begin{aligned}
            \omega^j\wedge\omega_{ij} &=
            -\frac{1}{2}a_{ij\alpha}\omega^j\wedge\omega^{\alpha} -
            \frac{1}{2}a_{i\alpha j}\omega^j\wedge\omega^{\alpha} +
            \frac{1}{2}a_{j\alpha i}\omega^j\wedge\omega^{\alpha}\\
            &= 
            -\frac{1}{2}a_{ij\alpha}\omega^j\wedge\omega^{\alpha} +
            \frac{1}{2}a_{i\alpha j}\omega^{\alpha}\wedge\omega^j +
            \frac{1}{2}a_{j\alpha i}\omega^j\wedge\omega^{\alpha}\\
            &=
            \frac{1}{2}a_{j\alpha i}\omega^j\wedge\omega^{\alpha}\\
            &=
            \frac{1}{2}a_{j\alpha}{}^i\omega^j\wedge\omega^{\alpha}\\
        \end{aligned}
    \]
    which, by the definition of $a_{j\alpha}{}^i$, is equal to $d\omega^i$, as
    desired.

    Now, to see that these are unique, we can simply count the degrees of
    freedom. To start with, $\omega_i^j$ specifies $n^2$ one-forms, each with
    $n$ degrees of freedom, for a total of $n^3$ degrees of freedom. However,
    requiring $\omega_i^j$ to be antisymmetric means that we only need to
    specify $\frac{n(n-1)}{2}$ one-forms, leading to $\frac{n^2(n-1)}{2}$
    degrees of freedom.

    Now, the condition that $d\omega^i = \omega^k\wedge\omega_k^i$ for all $i$
    imposes more restrictions. We expand the left hand side as
    \[
        d\omega^i =
        \sum_{\alpha<\beta}a_{\alpha\beta}{}^i\omega^{\alpha}\wedge\omega^{\beta}
    \]
    and let 
    \[
        \omega^j_i = b_{i\alpha}^j\omega^{\alpha}
    \]
    Then, we can expand the right hand side as
    \[
\begin{aligned}
    \omega^k\wedge\omega^i_k &= \omega^k\wedge(b^i_{k\alpha}\omega^{\alpha})\\
    &= b^i_{k\alpha}\omega^k\wedge\omega^{\alpha}\\
    &=\sum_{k<\alpha}(b^i_{k\alpha}-b^i_{\alpha k})\omega^k\wedge\omega^{\alpha}
\end{aligned}
    \]
    equating both sides to each other for each of the $\frac{n(n-1)}{2}$ terms
    in the sum yields $\frac{n(n-1)}{2}$ equations for each $i$. Furthermore,
    since this equality must hold for all $i$, we have a total of
    $\frac{n^2(n-1)}{2}$ constraints.

    Thus, there are no degrees of freedom for $\omega^j_i$ satisfying the
    equations, and the solution we found is unique.
\end{proof}

\newpage

\section*{Problem 2}
Use the curvature form method to calculate the sectional curvature for the
$n$-sphere $S^n$ with the induced metric from $\R^{n+1}$.
\\
\\
\begin{proof}
    We will induct on the dimension $n$.
    The base case will be $n=2$.

    Now, for $n=2$ we choose the orthonormal coframe $\omega^{\theta} =
    d\theta$ and $\omega^{\phi} = \sin(\theta)d\phi$ so that
    \[
        ds^2 = d\theta^2 + \sin^2(\theta)d\phi^2 = (\omega^{\theta})^2 +
        (\omega^{\phi})^2
    \]
    (which verifies that these form an orthonormal coframe).

    Now, we can calculate the connection $1$-forms. By antisymmetry,
    $\omega^{\theta}_{\theta} = \omega^{\phi}_{\phi} = 0$. Furthermore,
    \[
        d\omega^{\theta} = d^2\theta = 0
    \]
    implies that
    \[
        \omega^{\theta}_{\phi} = f\omega^{\phi}
    \]
    for some scalar function $f$.

    We also know that
    \[
        d\omega^{\phi} = \cos(\theta)d\theta\wedge d\phi
    \]
    and so $\omega^{\phi}_{\theta} = \cos(\theta)d\phi$, forcing
    $\omega^{\theta}_{\phi} = -\cos(\theta)d\phi$. This completely specifies the
    connection $1$-forms.

    Now, let's calculate the curvature $2$-forms. By definition,
    \[
        \Omega^i_j = -(d\omega^i_j + \omega^i_k\wedge\omega^k_j)
    \]
    Clearly, $\Omega$ is antisymmetric, so we only have to find
    $\Omega^{\phi}_{\theta}$.
    \[
        \begin{aligned}
            \Omega^{\phi}_{\theta} &= -(d\omega^{\phi}_{\theta} +
            \omega^{\phi}_{\theta}\wedge\omega^{\theta}_{\phi})\\
            &= -(d(\cos(\theta)d\phi) -
                \omega^{\phi}_{\theta}\wedge\omega^{\phi}_{\theta}\\
                &= -(d(\cos(\theta)d\phi) + 0)\\
                &= \sin(\theta)d\theta\wedge d\phi\\
                &= \omega^{\theta}\wedge\omega^{\phi}\\
                &=-\omega^{\phi}\wedge\omega^{\theta}\\
        \end{aligned}
    \]
    and so $\Omega^i_j = -\omega^i\wedge\omega^j$. 
    
    Finally, we observe that
    \[
        R(X,Y)e_i = \Omega^j_i(X,Y)e_j
    \]
    and so, letting $Z=\xi^ie_i$, we have
    \[
        \begin{aligned}
            R(X,Y)Z &= R(X,Y)e_i\xi^i\\
            &= \Omega^j_i(X,Y)e_j\xi^i\\
            &= (-\omega^j\wedge\omega^i)(X,Y)e_j\xi^i\\
            &= (\xi^i\omega^i\wedge\omega^j)(X,Y)e_j\\
            &= (Z^{\flat}\wedge\omega^j)(X,Y)e_j\\
            &= (Z^{\flat}\wedge(\text{Id}))(X,Y)\\
            &= g(Z,X)Y - g(Z,Y)X
        \end{aligned}
    \]
    Recalling that for a space of constant sectional curvature, $R(X,Y)Z =
    \kappa(g(Z,X)Y - g(Z,Y)X)$ we observe that $\kappa = 1$ and so the sphere
    $S^2$ has constant sectional curvature $1$.
    \\
    \\
    Now, let's prove the inductive step. Suppose that the sphere $S^{n-1}$ has
    a local orthonormal frame such that $\Omega^i_j = -\omega^i\wedge\omega^j$. 
    We will show that there is a local orthonormal frame on $S^n$ such that
    $\Omega^i_j = -\omega^i\wedge\omega^j$ as well.

    For ease of notation, we will denote all objects from $S^{n-1}$ with tildes
    (e.g. the local orthonormal frame is $\{\tilde{\omega}^i\}$).

    Recall that for spherical coordinates, the metric is given inductively as
    \[
        ds^2 = d\chi^2 + \sin^2(\chi)d\Phi^2
    \]
    where $d\Phi^2$ is the metric of $S^{n-1}$. Thus, we can choose a local
    orthonormal frame as
    \[
    \omega^{\chi} = d\chi
    \]
    and
    \[
        \omega^i = \sin(\chi)\tilde{\omega}^i
    \]
    for all $\tilde{\omega}^i$.

    Now, let's calculate the connection forms. For this calculation, greek
    letters will be used to denote indices coming from the orthonormal frame on
    $S^{n-1}$.
    \[
\begin{aligned}
    d\omega^{\alpha} &= \cos(\chi)d\chi\wedge\tilde{\omega}^{\alpha} +
    \sin(\chi)d\tilde{\omega}^{\alpha}\\
    &= \cot(\chi)\sin(\chi)d\chi\wedge\tilde{\omega}^{\alpha} +
    \sin(\chi)(\tilde{\omega}^{\beta}\wedge\tilde{\omega}^{\alpha}_{\beta})\\
    &= \cot(\chi)d\chi\wedge\omega^{\alpha} +
    \omega^{\beta}\wedge\tilde{\omega}^{\alpha}_{\beta}\\
    &= \cot(\chi)\omega^{\chi}\wedge\omega^{\alpha} +
    \omega^{\beta}\wedge\tilde{\omega}^{\alpha}_{\beta}
\end{aligned}
    \]
    and so it follows that
    \[
\begin{aligned}
    \omega^{\alpha}_{\chi} &= -\omega^{\chi}_{\alpha} =
    \cot(\chi)\omega^{\alpha}\\
    \omega^{\alpha}_{\beta} &= \tilde{\omega}^{\alpha}_{\beta}
\end{aligned}
    \]
    these are easily verifed to solve the constraint equations for $\omega^i_j$.

    Now, we calculate the curvature $2$-forms, using the inductive hypothsesis
    $\tilde{\Omega}^{\alpha}_{\beta} =-
    \tilde{\omega}^{\alpha}\wedge\tilde{\omega}^{\beta}$ to get
    \[
        \begin{aligned}
            -\Omega^{\alpha}_{\beta} &= d\omega^{\alpha}_{\beta} +
            \omega^{\alpha}_k\wedge\omega^k_{\beta}\\
            &= -\tilde{\Omega}^{\alpha}_{\beta} +
            \omega^{\alpha}_{\chi}\wedge\omega^{\chi}_{\beta}\\
            &= \tilde{\omega}^{\alpha}\wedge\tilde{\omega}^{\beta} +
            (\cot(\chi)\omega^{\alpha})\wedge(-\cot(\chi)\omega^{\beta})\\
            &=\csc^2(\chi)\omega^{\alpha}\wedge\omega^{\beta} -
            \cot^2(\chi)\omega^{\alpha}\wedge\omega^{\beta}\\
            &=\omega^{\alpha}\wedge\omega^{\beta}
    \end{aligned}
    \]
    as desired.

    All that remains is to calculate $\Omega^{\alpha}_{\chi}$ and show that it
    is equal to $-\omega^{\alpha}\wedge\omega^{\chi}$.

    \[
        \begin{aligned}
            -\Omega^{\alpha}_{\chi} &= d\omega^{\alpha}_{\chi} +
            \omega^{\alpha}_k\wedge\omega^k_{\chi}\\
            &= d(\cot(\chi)\omega^{\alpha}) +
            \omega^{\alpha}_k\wedge\omega^k_{\chi}\\
            &= -\csc^2(\chi)d\chi\wedge\omega^{\alpha} +
            \cot(\chi)d\omega^{\alpha} +
            \tilde{\omega}^{\alpha}_{\beta}\wedge(\cot(\chi)\omega^{\beta})\\
            &= -\csc^2(\chi)\omega^{\chi}\wedge\omega^{\alpha} +
            \cot(\chi)(\omega^{\chi}\wedge(\cot(\chi)\omega^{\alpha}) +
                \omega^{\beta}\wedge\tilde{\omega}^{\alpha}_{\beta}) -
            (\cot(\chi)\omega^{\beta})\wedge\tilde{\omega}^{\alpha}_{\beta}\\
            &= -\csc^2(\chi)\omega^{\chi}\wedge\omega^{\alpha} +
            \cot(\chi)\omega^{\chi}\wedge(\cot(\chi)\omega^{\alpha}) +
            \cot(\chi)\omega^{\beta}\wedge\tilde{\omega}^{\alpha}_{\beta} -
            \cot(\chi)\omega^{\beta}\wedge\tilde{\omega}^{\alpha}_{\beta}\\
            &= -\csc^2(\chi)\omega^{\chi}\wedge\omega^{\alpha} +
            \cot^2(\chi)\omega^{\chi}\wedge\omega^{\alpha})+
            0\\
            &= -\omega^{\chi}\wedge\omega^{\alpha}\\
            &=\omega^{\alpha}\wedge\omega^{\chi}
        \end{aligned}
    \]
    as desired.

    Thus, following the exact same argument made in the case of $S^2$, we see
    that the sectional curvature on the sphere is $1$.
\end{proof}

\newpage

\section*{Problem 3}
Let $c$ be an arbitrary parallel of latitude on $S^2$, with $V_0$ a tangent
vector to $S^2$ at some point on $c$. Describe geometrically the parallel
transport of $V_0$ along $c$.
\\
\\
\begin{proof}
    We will show that parallel transport along $c$ in $S^2$ is the same as
    parallel transport along $c$ thought of as a curve in the cone $C$ that lies
    tangent to $S^2$ at $c$.

    In particular, note that the tangent spaces of $S^2$ and $C$ coincide on
    $c$. This means that projection of a vector on $c$ in $\R^3$ is the same
    whether it goes to $TS^2$ or $TC$. Furthermore, since the covariant
    derivative of a vector on $c$ is equal to the ordinary partial derivative in
    $\R^3$ followed by projection into the tangent space, it follows that the
    covariant derivative of $V_0$ along $c$ is the same whether taken in $S^2$
    or $C$. Thus, since parallel transport is defined in terms of the covariant
    derivative, the parallel transport of $V_0$ on $S^2$ coincides with the
    parallel transport of $V_0$ on $C$.

    Now, we note that $C$ is actually flat: by making a suitable radial cut, one
    may flatten $C$ so that it forms a disk with a slice missing, with the
    boundary of the disk coinciding with $c$. Here, parallel transport of $V_0$
    along $c$ is just ordinary translation in $C\subset \R^2$.

    Thus, we have a complete description of the parallel transport of $V_0$
    along $c$. We form the cone $C$ tangential to $c$, and make a cut so that
    $C$ can be isometrically embedded as a subset of $\R^2$. Then, identifying
    $V_0$ with its corresponding tangent vector on $c \subset \partial C$, we
    apply ordinary translation (parallel transport in $\R^2$) to $V_0$ along
    $c$. The result is the parallel vector field $V(t)$ along $c$ in $\R^2$,
    which is identified with the parallel vector field $V(t)\subset TC$.
    Finally, noting that $TC$ and $TS^2$ coincide on $c$, we see that
    $V(t)\subset TS^2$ is the parallel vector field of $V_0$ on $c$.
\end{proof}

\newpage

\section*{Problem 4}
Suppose $M$ has the following property: given any two points $p,q$ in $M$, the
parallel transport from $p$ to $q$ does not depend on the path chosen. Show that
$M$ is flat ($R$ is identically zero).
\\
\\
\begin{proof}
    We follow the hint outlined in the problem. Let
    $f:(0-\varepsilon,1+\varepsilon)^2\to M$ parameterize a surface in $M$ with
    $f(s,0) = f(0,0)$ for all $s$. Let $V_0$ be an arbitrary vector at $f(0,0)$.
    Now, define a vector field $V$ on the surface where $V(s,0) = V_0$ and
    $V(s,t)$ is the parallel transport of $V_0$ along $t\mapsto f(s,t)$.

    Now, lemma 4.1 states that
    \[
        [\nabla_{\partial_t},\nabla_{\partial_s}]V =
        R(\partial_sf,\partial_tf)V
    \]
    (for ease of notation, we denote $\nabla_{\partial_t}$ as $\nabla_t$ and
    likewise for $s$).

    Now, we know that $\nabla_s\nabla_tV=0$, since $V$ is a parallel vector
    field along $t$. Thus,
    \[
        R(\partial_sf,\partial_tf)V - \nabla_t\nabla_sV = 0
    \]

    However, by the hypothesis of the problem, we know that $V(s,1)$ can be
    thought of as the parallel transport of $V(0,1)$ along $s\mapsto f(s,1)$.
    This follows, since $V(s,1)$ is the parallel transport of $V_0$ along the
    path of constant $s$, and thus $V(s,1)$ can be thought of as the parallel
    transport of $V(0,1)$ backwards to $V(0,0)$, then forwards to $V(s,1)$.
    Since parallel transport does not depend on paths chosen, it follows that
    $V(s,1)$ is also the parallel transport of $V(0,1)$ along the curve
    $s\mapsto f(s,1)$.

    Thus, $\nabla_sV(s,1) = 0$ and so we have that $\nabla_t\nabla_sV(s,1)=0$.
    In particular, for $s=0$ we have
    \[
        R_{f(0,1)}(\partial_sf(0,1),\partial_tf(0,1))V(0,1) = 0
    \]
    Now, $f$ and $V$ were arbitrary, and in particular for any vector fields
    $X,Y,Z$ we can construct an $f$ and $V_0$ for which $X_{f(0,1)}=
    \partial_sf(0,1)$ and $Y_{f(0,1)} = \partial_tf(0,1)$ and $Z_{f(0,1)} =
    V(0,1)$. To see this, note that we can choose an $f$ that satisfies
    conditions for $X$ and $Y$ easily (since $X$ and $Y$ only specify how the
    parameterization should behave at $(0,1)$) and by setting $V_0$ as the
    parallel transport of $Z_{f(0,1)}$ along $f(0,t)$ to $f(0,0)$, we obtain
    that the parallel transport of $V_0$ is $Z_{f(0,1)}$ as desired.

    Thus, the Riemann curvature tensor $R(X,Y)Z$ vanishes at every point for
    every vector field $X,Y,Z$ as desired.
\end{proof}

\end{document}
