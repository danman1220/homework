\documentclass[12pt,reqno]{amsart}
\usepackage{amssymb}
\usepackage{amscd}
\usepackage{amsxtra}
\usepackage[mathscr]{eucal}

\setlength{\oddsidemargin}{0cm}
\setlength{\evensidemargin}{0in}
\setlength{\textwidth}{16.5cm}
\setlength{\topmargin}{0.35cm}
\setlength{\textheight}{8.5in}

\renewcommand{\baselinestretch}{1.33}

\newcommand{\N}{\mathbb{N}}
\newcommand{\Z}{\mathbb{Z}}
\newcommand{\R}{\mathbb{R}}

\newcommand{\Hom}{\textrm{Hom}}

\newtheorem*{theorem}{Theorem}
\pagestyle{plain}

\begin{document}
\title[]{Math 220B: Final Examination\\March 20, 2018\\Daniel Halmrast}
\maketitle
\large

\section*{Problem 1}
\subsection*{Part i}
Prove that the nilpotent elements of a ring $R$ form an ideal $N$.
\\
\\
\begin{proof}
    Let $N$ be the collection of all nilpotent elements of $R$. We will show
    this collection is closed under addition, and is stable with respect to
    multiplication in $R$.

    First, suppose $a,b\in N$. That is, there exist integers $m,n\geq 1$ such
    that $a^m = b^n=0$. We wish to show that $a+b$ is nilpotent. This is clear,
    however, since
    \[
        (a+b)^{nm} = \sum_{k=1}^n\binom{n}{k}a^{nm-k}b^k
    \]
    and each term in the sequence has either $nm-k > m$ or $k>n$, and so
    $a^{nm-k}$ or $b^k$ is zero. Thus, $(a+b)^{nm}=0$ and $a+b$ is nilpotent as
    desired.

    Now, we show that $N$ is stable with respect to multiplication. That is, for
    any $r\in R$, $rN= N$. This amounts to showing that $ra\in N$ for any $a\in
    N$. So, let $a$ be such that $a^n=0$. Then,
    \[
        (ra)^n = r^na^n = 0
    \]
    as well, so $ra\in N$ as desired.

    Thus, $N$ is an ideal.
\end{proof}

\subsection*{Part ii}
Show further that if $R$ is Noetherian, then $N$ is a nilpotent ideal.
\\
\\
\begin{proof}
    Since $R$ is Noetherian, we know that all ideals of $R$ are finitely
    generated. In particular, $N$ is finitely generated by some finite set $S =
    \{n_1,\cdots,n_k\}$. Then, every element $x\in N$ is expressible as
    \[
        x = \sum_{i=1}^ka_in_i
    \]
    Now, let $m_1,\cdots,m_k$ be such that $n_i^{m_i} = 0$ (since $n_i$ is
    nilpotent). I assert that $N^{m_1m_2\ldots m_k}=\{0\}$. This follows
    from the fact that for $x_i\in N$,
    \[
        \begin{aligned}
            \prod_{i=1}^{m_1m_2\ldots m_k}x_i 
            &= \prod_{i=1}^{m_1m_2\ldots m_k}\left( \sum_{l=1}^k a_{il}n_i
            \right)\\
        &=\sum_j c_jn_1^{p_{1_j}}n_2^{p_{2_j}}\cdots n_k^{p_{k_j}}
    \end{aligned}
    \]
    (for some constants $c_j$)
    where for each $j$, $\sum_{i=1}^kp_{i_j} = m_1m_2\cdots m_k$. This implies
    that in each term, at least one exponent $p_{i_j}$ is greater than $m_i$,
    and so $n_i^{p_{i_j}}=0$, and thus the term is zero. Since each term in the
    expansion is zero, this implies that
    \[
            \prod_{i=1}^{m_1m_2\ldots m_k}x_i =0
    \]
    In particular, this holds for all $x_i\in N$. This implies that
    $N^{m_1m_2\ldots m_k} = \{0\}$ as desired.
\end{proof}

\subsection*{Part iii}
Give an example of a ring $R$ for which $N$ is not a niplotent ideal.
\\
\\
\begin{proof}
    Let
    \[
        G = \langle a_i|a_i^i = 0\rangle
    \]
    be the (infinitely) presented group, and let $R = \Z[G]$ be the group ring.
    Note that for each $i$, $a_i\in N$ the nilpotent ideal. However, for any
    fixed integer $m>0$, $N^m$ contains the element $a_{m+1}^m$ which is not
    zero. Thus, for any $m$, $N^m\neq \{0\}$ and $N$ is not nilpotent, as
    desired.
\end{proof}

\newpage

\section*{Problem 2}
\subsection*{Part i}
State the Hilbert Basis Theorem.
\\
\\
\begin{theorem}
    For $R$ a Noetherian ring, $R[x_1,..,x_k]$ is Noetherian.
\end{theorem}

\subsection*{Part ii}
let $k$ be a field. Show that every algebraic set can be defined by a finite
system of equations.
\\
\\
\begin{proof}
    Let $S$ be an algebraic set of $k^n$. That is, $S$ is the set of all points
    $\alpha$ such that $f_{\lambda}(\alpha)= 0$ for each $f_{\lambda}$ in a
    family $\mathscr{P}=\{f_{\lambda}\ |\ f_{\lambda}\in k[x_1,\cdots,x_n]\}$ of polynomials
    (indexed by $\lambda$).

    Consider the subspace generated by $\mathscr{P}$. That is, consider
    \[
        V = \text{span}(\{f_{\lambda}\ |\ f_{\lambda}\in \mathscr{P}\})
    \]
    This is (by definition) a linear subspace of $k[x_1,\cdots,x_n]$. Since $k$
    is Noetherian, it follows that $k[x_1,\cdots,x_n]$ is Noetherian as well.
    Thus, $V$ is finitely generated as a submodule of the Noetherian module
    $k[x_1,\cdots,x_n]$. Let $g_1,\cdots,g_m$ be the generators of $V$. I claim
    that these define the algebraic set $S$.

    Let $S'$ be the set of all $\alpha\in k^n$ for which $g_i(\alpha)=0$ for all
    generators $g_i$. That is, $S'$ is the algebraic set corresponding to
    $g_1,\cdots,g_m$. 
    
    First observe that $S'\subseteq S$. To see this, suppose
    $\alpha\in S'$. For $f_{\lambda}\in\mathscr{P}$, we know that
    \[
        f_{\lambda} = \sum_{i=1}^m a_ig_i
    \]
    since $g_1,\cdots,g_n$ generate $V$. Thus,
    \[
        f_{\lambda}(\alpha) = \sum_{i=1}^na_ig_i(\alpha) = \sum_{i=1}^na_i(0)=0
    \]
    and so $\alpha\in S$ as well.

    Next, we observe that $S\subseteq S'$. To see this, let $\alpha\in S$. We
    note first that for any $f\in V$, $f(\alpha) = 0$. This follows from the
    fact that $f$ is in the span of $\mathscr{P}$, and so
    \[
        f = \sum_{i=1}^k a_if_{\lambda_i}
    \]
    for $f_{\lambda_i}\in\mathscr{P}$. Thus,
    \[
        f(\alpha) = \sum_{i=1}^k a_if_{\lambda_i}(\alpha) = 0
    \]
    In particular, since $V$ is generated by $g_1,\cdots,g_m$, we know that
    $g_i\in V$ for all $i$, and so $g_i(\alpha)=0$ as well. Thus $\alpha\in S'$,
    and $S\subseteq S'$

    We have shown that $S'=S$, and so $S$ is determined by the finite set of
    polynomials $\{g_1,\cdots,g_n\}$.
\end{proof}


\newpage

\section*{Problem 3}
Give examples to show each of the following might occur.

\subsection*{Part a}
$M\otimes_R N\neq M\otimes_{\Z} N$.
\\
\\
\begin{proof}
    Let $M=N=\Z^2$, and $R=\Z^2$. Then
    \[
        \Z^2\otimes_{\Z^2}\Z^2\cong \Z^2
    \]
    This is clear, since for any simple tensor $a\otimes_{\Z^2}b$, we have
    \[
        a\otimes b = a(1\otimes b) = ab(1\otimes 1)
    \]
    Thus, every simple tensor is a multiple of $1\otimes 1$, and thus every
    tensor is a multiple of $1\otimes 1$. So, we can define an isomorphism
    $\Phi:\Z^2\to \Z^2\otimes_{\Z^2}\Z^{2}$ as
    \[
        \Phi(a,b) = ab(1\otimes 1)
    \]
    with inverse $\Phi^{-1}(a\otimes b) = ab$ extended linearly.

    However,
    \[
        \Z^2\otimes_{\Z}\Z^2 = (\Z\times \Z)\otimes(\Z\times\Z)\cong \Z^4
    \]
    (easily verified using the property $M\otimes(N_1\times N_2) = (M\otimes
    N_1)\times(M\otimes N_2)$). Since $\Z^2\neq \Z^4$, the two tensor products
    are not equal, as desired.
\end{proof}

\subsection*{Part b}
$u\in M\otimes_R N$ but $u\neq m\otimes_R n $ for any $m\in M$ and $n\in N$.
\\
\\
\begin{proof}
    Let $M=N=\R^n$, with $R=\R$, and $n\geq 3$. Recall that for
    finite-dimensional vector spaces, $\Hom(V,\R)=V^*\cong V$ (an elementary
    result not proven here). Recall also that
    $\Hom(V,\cdot)$ is right-adjoint to $\cdot\otimes V$. The proof for this is
    easy: let $X,Z$ be real vector spaces, and let $f\in\Hom(X\otimes_{\R}V,Z)$.
    Then, we can define an isomorphism $\Phi$ as
    \[
        \Phi(f)(x) = \tilde{f}(x) = (v\mapsto f(x\otimes v))
    \]
    clearly, this is linear in $x$ and $v$, and so $\tilde{f}$ is an element of
    $\Hom(X,\Hom(V,Z))$. Furthermore, $\Phi$ itself is linear, since
    \[
        \Phi(f+g)(x) = (v\mapsto f(x\otimes v)+g(x\otimes v)) =
        \Phi(f)(x)+\Phi(g)(x)
    \]
    and
    \[
        \begin{aligned}
        \Phi(\alpha f)(x) &= (v\mapsto \alpha f(x\otimes v))\\
        &=\alpha (v\mapsto f(x\otimes v)) = \alpha\Phi(f)(x)
    \end{aligned}
    \]
    as desired.

    $\Phi$ also has trivial kernel, since if $\Phi(f)(x) = 0$ for all $x$, then
    \[
        (v\mapsto f(x\otimes v)) = 0
    \]
    which implies that $f(x\otimes v)=0$ for all $x$ and $v$, and so $f=0$.

    Finally, we note that $\Phi$ is surjective, since for any
    $g\in\Hom(X,\Hom(V,Z))$, we can define $\tilde{g}$ to be the map
    \[
        \tilde{g}(x\otimes v) = g(x)(v)
    \]
    and 
    \[
        \Phi(\tilde{g})(x) = (v\mapsto \tilde{g}(x\otimes v)) = v\mapsto
    g(x)(v) = g(x)
    \]
    thus, $\Phi$ is an isomorphism as desired.
    \\
    \\
    We now combine these facts to show that for $V=\R^n$, $V\otimes
    V\cong\Hom(V,V)$. Note that
    \[
        \begin{aligned}
        V\otimes V &\cong \Hom(V\otimes V,\R)\\
        &\cong \Hom(V,\Hom(V,\R))\\
        &\cong \Hom(V,V)
    \end{aligned}
    \]
    as desired.

    But $\Hom(\R^n,\R^n)$ is the set of all linear maps from $\R^n$ to itself,
    which is the set of all $n\times n$ matrices over $\R$. This is a vector
    space of dimension $n^2$, whereas $\R^n\times \R^n$ is of dimension $2n$.
    Thus, for $n\geq 3$, the tensor product map $\otimes:
    \R^n\times\R^n\to\R^n\otimes \R^n$ cannot be surjective, and thus there
    exists an element $u$ of $\R^n\otimes \R^n$ that is not in the image of
    $\otimes$. That is, $u$ is not $v\otimes w$ for some $v,w\in \R^n$. This
    completes the example.
\end{proof}

\subsection*{Part c}
$m\otimes_{\Z} n = m_1\otimes_{\Z}n_1$ but $m\neq m_1$ and $n\neq n_1$.
\\
\\
\begin{proof}
    Let $M=N=\Z$. Then
    \[
        2\otimes 1 = 1\otimes 2
    \]
    but $2\neq 1$. Equality of the tensors follows immediately from bilinearity,
    since
    \[
        2\otimes 1 = 2(1\otimes 1) = 1\otimes 2
    \]
\end{proof}

\subsection*{Part d}
$M\otimes_R N = \{0\}$ but $M\neq \{0\}$ and $N\neq\{0\}$.
\\
\\
\begin{proof}
    Let $M=2\Z$, $N=\Z/{2\Z}$, and $R=\Z$. Then let $2k\otimes j$ be a simple
    tensor in $M\otimes_R N$. We have that
    \[
\begin{aligned}
    2k\otimes j = k\otimes 2j = k\otimes 0 = 0
\end{aligned}
    \]
    Thus, all simple tensors are zero, and it follows that all tensors in the
    product are zero. Thus, $M\otimes_R N = \{0\}$ as desired.
\end{proof}

\newpage

\section*{Problem 4}
Let $\mathbb{F}_7 = \Z/{7\Z}$ be the field of integers modulo seven, and let $R$
be any commutative ring containing $\mathbb{F}_7$ together with an element
$i\not\in \mathbb{F}_7$ for which $i^2=-1$. 

\subsection*{Part i}
Show that the map $\alpha:R\to R$ given by $x\mapsto x^7 + (2+i)x$ is an
endomorphism of $R$ as an $\mathbb{F}_7$ module.
\\
\\
\begin{proof}
    We wish to show $\alpha$ is linear with respect to the field $\mathbb{F}_7$.
    So, let $x,y\in R$. We can compute $\alpha(x+y)$ as
    \[
\begin{aligned}
    \alpha(x+y) &= (x+y)^7 + (2+i)(x+y)\\
    &= x^7+7x^6y+21x^5y^2+35x^4y^3+35x^3y^4+21x^2y^5+7xy^6+y^7\\
    &+(2+i)x+(2+i)y\\
    &= x^7 + y^7 + (2+i)x+(2+i)y\\
    &=\alpha(x) + \alpha(y)
\end{aligned}
    \]
    where we used the fact that $7=0$ in this module. Furthermore, for
    $r\in\mathbb{F}_7$, we have
    \[
\begin{aligned}
    \alpha(rx) &= (rx)^7 + (2+i)rx\\
    &= r^7x^7 + r(2+i)x\\
    &= rx^7 + r(2+i)x\\
    &=r\alpha(x)
\end{aligned}
    \]
    where we used the fact that $r^7 = r$ modulo $7$ for $r\in \mathbb{F}_7$
    (verified easily by computation). Thus $\alpha$ is linear over
    $\mathbb{F}_7$ as desired.
\end{proof}

\subsection*{Part ii}
Compute the effect of $\alpha$ on the elements $1$ and $i$ of $R$ and on the
element $1\wedge i$ of $\wedge^2R$.
\\
\\
\begin{proof}
    We compute directly.
    \[
        \begin{aligned}
            \alpha(1) &= 1^7 + (2+i)(1) = 3+i\\
            \alpha(i) &= (i)^7 + (2+i)(i)\\
            &= -i + 2i-1 = -1 +i
    \end{aligned}
    \]
    Furthermore, we can calculate $\alpha(1\wedge i)$ as $\alpha(1)\wedge
    \alpha(i)$ by
    \[
\begin{aligned}
    \alpha(1)\wedge\alpha(i) &= (3+i)\wedge(-1+i)\\
    &=(3+i)\wedge(-1) + (3+i)\wedge(i)\\
    &= 3\wedge(-1) + i\wedge(-1) + 3\wedge i + i\wedge i\\
    &= -3(1\wedge 1) -i\wedge 1 + 3(1\wedge i) + i\wedge i\\
    &= 0 + 1\wedge i + 3(1\wedge i) + 0\\
    &= 4(1\wedge i)
\end{aligned}
    \]
\end{proof}

\subsection*{Part iii}
Suppose $R$ has order $49$. Deduce the value of $\det(\alpha)$.
\\
\\
\begin{proof}
    Note first that if $R$ has order $49$, it must be that $R =
    \text{span}(1,i)$. This is true, since we know that $R$ must contain all
    elements of the form $a+bi$ for $a$ and $b$ in $\mathbb{F}_7$, since $R$ is
    an $\mathbb{F}_7$ module with $i$ not in $\mathbb{F}_7$. However, there are
    $49$ such elements. Thus, $R$ must be equal to the set of elements of the
    form $a+bi$.

    Clearly, then, $R$ is two-dimensional. Thus, the determinant $\det(\alpha)$
    can be calculated as the scalar $k$ such that
    \[
        \alpha(a\wedge b) = k a\wedge b
    \]
    But we already calculated that $\alpha(1\wedge i) = 4(1\wedge i)$, and so
    $\det(\alpha) = 4$.
\end{proof}

\end{document}
