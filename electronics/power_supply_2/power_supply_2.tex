\documentclass[12pt]{article}

\title{Power Supply Homework 2: The Zemer Diode}

\author{Daniel Halmrast}

\usepackage{amsmath, circuitikz}

\begin{document}

\maketitle

Using The parameters $V_{out}, i_{load}, \langle P_z \rangle, I_{zmin}, V_p, f$, solve for the parameters
$R, C, \Delta V$ in the following circuit.\\

\begin{center}
\begin{circuitikz}\draw

(0,0) to [short, *-] (2,0)
      to [capacitor, l=$C$] (2,4)
      to [short, -*] (0,4)

(2,0) to [short, -] (6,0)
      to [empty diode, l=$D_z$] (6,4)
      to [R, l=$R$] (4,4)
      to [short, -] (2,4)

(6,0) to [short, -] (8,0)
      to [R, l=$R_{load}$] (8,4)
      to [short, -] (6,4)
;
\draw[dashed] (3,5) -- (3,-1);
\end{circuitikz}
\end{center}

Firstly, we concern ourselves with the ripple voltage. We know from earlier analysis that the ripple voltage is given as
\[
\Delta V = V_p \frac{1}{CR_{eq}f}
\]
It is clear from analysis of the right side that the Thevevin equivalent resistance is just $R$. Thus, we have
\begin{equation}
\boxed{\Delta V = V_p \frac{1}{CRf}}
\end{equation}
\\

Next,we will analyze the average power requirement. It can be approximated that the average power through the Zemer diode is
\[
\langle P_z \rangle = \overline I \overline V
\]
Thus,
\[
\begin{aligned}
\overline I & = \frac{\langle P_z \rangle}{V_{out}}\\
i_{zmax} - \frac{\Delta i_z}{2} & = \frac{\langle P_z \rangle}{V_{out}}
\end{aligned}
\]

Since $i_z = i_R -i_l$, and $i_R = \frac{v_r}{R}$, it follows that
\[
\begin{aligned}
\frac{V_p - V_{out}}{R} - i_l - \frac{\Delta i_z}{2} & = \frac{\langle P_z \rangle}{V_{out}}\\
\frac{V_p - V_{out}}{R} - i_l - \frac{\Delta V}{2R} & = \frac{\langle P_z \rangle}{V_{out}}\\
V_p - V_{out} - i_lR - \frac{\Delta V}{2} & = \frac{\langle P_z \rangle}{V_{out}}R
\end{aligned}
\]

Thus,

\begin{equation}
\boxed{V_p - V_{out} - \frac{\Delta V}{2} = \Big(\frac{\langle P_z \rangle}{V_{out}} - i_l\Big)R}
\end{equation}
\\

Finally, we will analyze the minimum current requirement. Following a similar process to equation (2),
\[
\begin{aligned}
i_{zmin}  & = i_{zmax} - \Delta i\\
          & = i_{Rmax} - i_l -\Delta i\\
          & = \frac{V_p - V_{out}}{R} - i_l - \frac{\Delta V}{R}\\
i_{zmin}R & = V_p - V_{out} - i_lR - \Delta V
\end{aligned}
\]

This leads to the final relation
\begin{equation}
\boxed{V_p - V_{out} - \Delta V = (i_{zmin} + i_l)R}
\end{equation}

\end{document}
