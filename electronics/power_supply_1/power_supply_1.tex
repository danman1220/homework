\documentclass[12pt]{article}

\title{Power Supply Homework 1}

\author{Daniel Halmrast}

\usepackage{amsmath, amsfonts, circuitikz}


\begin{document}

\maketitle

\section{Problem 1}

Design an unregulated power supply that will output $12.0$VDC with a maximum ripple of $1\%$
at an output current of $200$mA.\\
\\

\subsection{Half-Wave Rectifier}
The basic design for a half-wave rectifier is
\begin{center}
\begin{circuitikz} \draw

(0,0) to [short, *-*] (8,0)
(0,0) to [open, l=$V_{in}$] (0,4)
to [empty diode, *-] (4,4)
to [short, -*] (8,4)

(6,4) to [R, l=$R_{load}$] (6,0)
(4,4) to [capacitor, l=$C$] (4,0)

%(8,0) to [open, l=$V_{out}$, right] (8,4)
;
\end{circuitikz}
\end{center}

This will output at a peak voltage equal to $V_{in} - 0.6$V.
It will have a fractional ripple of $\frac{1}{CR_Lf}$.
With a desired fractional ripple of $0.01$ on a frequency of $60$Hz at $200$mA,
$R_L$ will be $\frac{V_{out}}{i_out}=\frac{12.0}{0.2}=60$Ohms.
\\
For a fractional ripple of $1\%$, we have
\[
\begin{aligned}
    0.01 & = \frac{1}{C*60*60}\\
         & \implies C=\frac{1}{0.01*60*60}
         & = 0.0278F
\end{aligned}
\]
\\
Thus, with a capacitor of greater value than $C=0.0278F$, the power supply will have the desired properties.
Since the voltage sees a $0.6V$ drop, an input voltage of $12.6VAC$ is required.
\subsection{Full-Wave Rectifier Using Center Tap}

In this design, the capacitor sees a similar waveform to the half-wave rectifier, but at double frequency.
Using this approximation, it is easy to see that the capacitor will only be required to have half the
capacitance as the first. That is, $C=0.0139F$. Furthermore, the input voltage still only sees one
diode drop, but the output voltage will be relative to the center of the inductor,
which will be half of the full $V_p2p$ voltage expected out of the transformer.\\

That is, the transformer should output $12.6V$ on each side, for a total of $25.2VAC$
expected peak-to-peak voltage.

\subsection{Full-Wave Bridge Rectifier}

This power supply has the same output as the center-tapped full rectifier, but with two diode drops.
That is, $V_{in}=12.0+1.2=13.2VAC$, and $C=0.0139F$.

\section{Problem 2}

Make a power supply with $+5.0V$ and $-5.0V$ output using a center-tapped transformer.\\
\\
It will be easiest to do this by using a half-wave rectifier on each half of the transformer,
with the positive end recitfying in the normal way, and the negative end rectifying in reverse polarity.
\\
Grounding the center tap, the completed circuit design is

\begin{center}
\begin{circuitikz}\draw
(0,0) node [anchor=east]{$V_{in}-$}
(0,8) node [anchor=east]{$V_{in}+$}

(0,8) to [empty diode, *-] (3,8)
(3,0) to [empty diode, -*] (0,0)

(3,8) to [short, -] (3,7)
      to [short, -] (2,7)
      to [capacitor, l=$C$] (2,5)
      to [short, -] (3,5)

(3,7) to [short, -] (4,7)
      to [R, l=$R+_{load}$] (4,5)
      to [short, -] (3,5)

(3,0) to [short, -] (3,1)
      to [short, -] (2,1)
(2,3) to [capacitor, l=$C$] (2,1)
(2,3) to [short, -] (3,3)

(3,1) to [short, -] (4,1)
(4,3) to [R, l=$R-_{load}$] (4,1)
(4,3) to [short, -] (3,3)

(3,3) to [short,-] (3,5)
(3,4) to [short,-*] (5,4)
(5,4) node[ground] {}
;
\end{circuitikz}
\end{center}

By symmetry, the value of $C$ will be the same for each side. That value will be determined in a
similar way to Problem 1.1.
\[
\begin{aligned}
V_{ripple} & =\frac{1}{C*R_L *f}\\
    0.002 & = \frac{1}{C*(\frac{5V}{0.5A})*60Hz}\\
    C & = \frac{1}{.002*10*60}\\
        & = 0.833F
\end{aligned}
\]


\end{document}
