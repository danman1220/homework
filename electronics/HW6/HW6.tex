\documentclass[12pt]{article}
\usepackage{amsmath,amsthm,amssymb,amsfonts}
\usepackage{graphicx}

\graphicspath{ {images/} }

\newtheorem*{problem}{Problem}

\theoremstyle{definition}
\newtheorem*{solution}{Solution}

\begin{document}

\title{Electronics Homework 6}
\author{Daniel Halmrast}
\maketitle

%--PROBLEM 1--%
\begin{problem}[2.9]
A $1H$ inductor carries a current of $500mA$. The wire breaks, and in $10^{-3}s$, the current goes to zero. What happens? 
\end{problem}[2.9]

\begin{solution}[2.9]

Since $V_L = L\frac{dI}{dt}$, and $\frac{dI}{dt} \approx \frac{\Delta I}{\Delta t}$
\[
\begin{aligned}
\frac{dI}{dt} & \approx \frac{500mA}{10^{-3}s}\\
V_L & \approx 1H \frac{500mA}{10^{-3}s}
V_L & \approx 50V
\end{aligned}
\]
So, once the wire breaks, there will be a voltage difference across the inductor of $50V$.
\end{solution}


%--PROBLEM 2--%
\begin{problem}[2.10]
Calculate the impedance for a RC circuit in series and in parallel.
\end{problem}

\begin{solution}[2.10]
In series, the impedances add together and we get
\[
\begin{aligned}
Z_{tot} & = Z_R + Z_C\\
        & = R + \frac{-j}{C\omega}\\
	& = R - \frac{1}{C\omega}j\\
	& = \sqrt{R^2+(\frac{1}{Cw})^2}e^{\arctan(\frac{1}{RC\omega})}
\end{aligned}
\]

In parallel, the impedances follow the parallel law
\[
\begin{aligned}
Z_{tot} & = \frac{Z_RZ_C}{Z_R+Z_C}\\
        & = \frac{R\frac{-j}{C\omega}}{R + \frac{-j}{C\omega}}\\
	& = \frac{R\frac{-j}{C\omega}}{\sqrt{R^2+(\frac{1}{Cw})^2}e^{\arctan(\frac{1}{RC\omega})j}}\\
	& = \tfrac{\frac{R}{C\omega}}{\sqrt{R^2+(\frac{1}{Cw})^2}}e^{(\frac{3\pi}{2} - \arctan(\frac{1}{RC\omega}))j }\\
	& = \tfrac{\frac{R}{C\omega}}{\sqrt{R^2+(\frac{1}{Cw})^2}}\Big[\cos(\frac{3\pi}{2} - \arctan(\frac{1}{RC\omega})) + j\sin(\frac{3\pi}{2} - \arctan(\frac{1}{RC\omega}))\Big]
\end{aligned}
\]
\end{solution}

%--PROBLEM 3--%
\begin{problem}[2.11]
Calculate the impedance for an LRC series circuit and an RL parallel circuit.
\end{problem}

\begin{solution}[2.11]

Again, we will use the series law to write
\[
\begin{aligned}
Z_{tot} & = Z_R + Z_L + Z_C\\
	& = R + (L\omega - \frac{1}{C\omega})j\\
	& = \sqrt{R^2+(L\omega - \frac{1}{Cw})^2}e^{\arctan(\frac{L\omega - \frac{1}{Cw}}{R})}
\end{aligned}
\]

For the parallel combination, we see that
\[
\begin{aligned}
Z_{tot} & = \frac{Z_RZ_L}{Z_R+Z_L}\\
        & = \frac{RL\omega j}{R + L\omega j}\\
	& = \tfrac{RL\omega}{\sqrt{R^2 + (L\omega)^2}}e^{(\frac{\pi}{2} - \arctan{\frac{L\omega}{R}})}\\
	& = \tfrac{RL\omega}{\sqrt{R^2 + (L\omega)^2}}\Big[\cos(\frac{\pi}{2} - \arctan{\frac{L\omega}{R}}) + j\sin(\frac{\pi}{2} - \arctan{\frac{L\omega}{R}})\Big]
\end{aligned}
\]

\end{solution}

%--PROBLEM 4--%
\begin{problem}[2.12]
Calculate the impedence for a $C||(R+L)$ circuit.
\end{problem}

\begin{solution}[2.12]
The impedance of $R+L$ is 
\[
\begin{aligned}
Z_{RL} 	& = Z_R + Z_L\\
	& = R + L\omega j\\
	& = \sqrt{R^2 + (L\omega)^2}e^{\arctan(\frac{L\omega}{R})}
\end{aligned}
\]
so 
\[
\begin{aligned}
Z_{tot} & = \frac{Z_{RL}Z_C}{Z_{RL} + Z_C}\\
	& = \frac{(R+L\omega j)(\frac{-j}{C\omega})}{R + (L\omega - \frac{1}{C\omega})j}\\
	& = \frac{\frac{L}{C} - \frac{R}{C\omega}j}{R + (L\omega - \frac{1}{C\omega})j}\\
	& = \frac{R}{(C\omega)^2} + \Big[\frac{L-R^2C-CL^2\omega^2}{C^2\omega}\Big]j
\end{aligned}
\]
\end{solution}



\end{document}
