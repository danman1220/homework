%%%%%%%%%%%%%%%%%%%%%%%%%%%%%%%%%%%%%%%%%
% Short Sectioned Assignment
% LaTeX Template
% Version 1.0 (5/5/12)
%
% This template has been downloaded from:
% http://www.LaTeXTemplates.com
%
% Original author:
% Frits Wenneker (http://www.howtotex.com)
%
% License:
% CC BY-NC-SA 3.0 (http://creativecommons.org/licenses/by-nc-sa/3.0/)
%
%%%%%%%%%%%%%%%%%%%%%%%%%%%%%%%%%%%%%%%%%

%----------------------------------------------------------------------------------------
%	PACKAGES AND OTHER DOCUMENT CONFIGURATIONS
%----------------------------------------------------------------------------------------

\documentclass[fontsize=11pt]{scrartcl} % 11pt font size

\usepackage[T1]{fontenc} % Use 8-bit encoding that has 256 glyphs
\usepackage[english]{babel} % English language/hyphenation
\usepackage{amsmath,amsfonts,amsthm} % Math packages
\usepackage{mathrsfs}

\usepackage[margin=1in]{geometry}

\usepackage{sectsty} % Allows customizing section commands
\allsectionsfont{\centering \normalfont\scshape} % Make all sections centered, the default font and small caps

\usepackage{fancyhdr} % Custom headers and footers
\pagestyle{fancyplain} % Makes all pages in the document conform to the custom headers and footers
\fancyhead{} % No page header - if you want one, create it in the same way as the footers below
\fancyfoot[L]{} % Empty left footer
\fancyfoot[C]{} % Empty center footer
\fancyfoot[R]{\thepage} % Page numbering for right footer
\renewcommand{\headrulewidth}{0pt} % Remove header underlines
\renewcommand{\footrulewidth}{0pt} % Remove footer underlines
\setlength{\headheight}{13.6pt} % Customize the height of the header

\numberwithin{equation}{section} % Number equations within sections (i.e. 1.1, 1.2, 2.1, 2.2 instead of 1, 2, 3, 4)
\numberwithin{figure}{section} % Number figures within sections (i.e. 1.1, 1.2, 2.1, 2.2 instead of 1, 2, 3, 4)
\numberwithin{table}{section} % Number tables within sections (i.e. 1.1, 1.2, 2.1, 2.2 instead of 1, 2, 3, 4)

\newcommand{\R}{\mathbb{R}}
\newcommand{\Q}{\mathbb{Q}}
\newcommand{\C}{\mathbb{C}}
\newcommand{\A}{\mathscr{A}}

%----------------------------------------------------------------------------------------
%	TITLE SECTION
%----------------------------------------------------------------------------------------

\newcommand{\horrule}[1]{\rule{\linewidth}{#1}} % Create horizontal rule command with 1 argument of height

\title{	
\normalfont \normalsize 
\textsc{Analysis} \\ [25pt] % Your university, school and/or department name(s)
\horrule{0.5pt} \\[0.4cm] % Thin top horizontal rule
\huge Problem Set 2\\ % The assignment title
\horrule{2pt} \\[0.5cm] % Thick bottom horizontal rule
}

\author{Daniel Halmrast} % Your name

\date{\normalsize\today} % Today's date or a custom date

\begin{document}

\maketitle % Print the title

%----------------------------------------------------------------------------------------
%	PROBLEM 1
%----------------------------------------------------------------------------------------
\section*{Problem 1}
Prove that the push-forward measure is a measure.
\\
\begin{proof}
Let $(\A_1,\Omega_1)$ and $(\A_2,\Omega_2)$ be measurable spaces, and let $\mu:\A_1\to\R^+$
be a measure on $\A_1$. Furthermore, let $f:\Omega_1\to\Omega_2$ be a measurable function.
Then, the \em push forward measure \em $f_*\mu$ on $\A_2$ is defined as:
\[
f_*\mu(E) = \mu(f^{-1}(E)) \textrm{for } E\in\A_2
\]

Now, let's verify that such a construction is actually a measure. To do this, we will
check the nullity of the empty set with respect to the measure, and the 
$\sigma$-additivity of the measure.

To check the nullity of the empty set, we can directly compute its measure.
\[
\begin{aligned}
f_*\mu(\emptyset)   &= \mu(f^{-1}(\emptyset))\\
                    &= \mu(\emptyset)\\
                    &=0
\end{aligned}
\]
Thus, the measure respects the nullity of the empty set.

To check the $\sigma$-additivity of the measure, let $\{E_i\}$ be a countable
collection of disjoint measurable subsets of $\Omega_2$ (That is, $E_i\in\A_2\forall i$).
Then, the measure of the union can be calculated directly:
\[
\begin{aligned}
f_*\mu\left(\bigcup_{i=1}^{\infty}E_i\right) &= \mu\left(f^{-1}\left(\bigcup_{i=1}^{\infty} E_i\right)\right)\\
        &=\mu\left(\bigcup_{i=1}^{\infty} f^{-1}(E_i)\right)\\
        &=\bigcup_{i=1}^{\infty} \mu\left(f^{-1}(E_i)\right)\\
        &=\bigcup_{i=1}^{\infty} f_*\mu(E_i)\\
\end{aligned}
\] 
In this calculation, we used two important facts about inverse images. First, the inverse
image preserves unions, which let us pass the union through the inverse image in line 2.
Second, the inverse image preserves intersections, which guaranteed that the collection
$\{f^{-1}(E_i)\}$ remained disjoint to establish equality in line 3.

Thus, the push-forward measure satisfies the two measure axioms, and is a measure as
desired.
\end{proof}
%----------------------------------------------------------------------------------------
%----------------------------------------------------------------------------------------
%	PROBLEM 2
%----------------------------------------------------------------------------------------
\section*{Problem 2}
Calculate, for measurable sets $A$ and $B$ with measure $\mu$, the measures:
\[
\begin{aligned}
\mu(A\setminus B)\\
\mu(A\Delta B)
\end{aligned}
\]
\\
\begin{proof}
First, let's find $\mu(A\setminus B)$. This can be calculated directly using the identity
$(A\setminus B) \cup B = A\cup B$ (this identity is easily seen to be true by chasing
basic set theory definitions). Thus,
\[
\begin{aligned}
\mu((A\setminus B) \cup B &= \mu(A\cup B)\\
\mu(A\setminus B) + \mu(B) &= \mu(A\cup B)\\
\mu(A\setminus B) &= \mu(A\cup B) - \mu(B)
\end{aligned}
\]
Here, we use the fact that $A\setminus B$ and $B$ are disjoint from each other to
split the measure on line 2. Note that this identity only holds if $\mu(A\cup B)$ and
$\mu(B)$ are not both of infinite measure, so that subtraction can be performed. 

Second, let's calculate $\mu(A\Delta B)$. For this, we will use the identity
$A\Delta B = (A\setminus B)\cup (B\setminus A)$. It should be observed that the sets
$A\setminus B$ and $B\setminus A$ are disjoint from each other. Now, by direct calculation:
\[
\begin{aligned}
\mu(A\Delta B)  &= \mu((A\setminus B)\cup (B\setminus A))\\
                &= \mu(A\setminus B) + \mu(B\setminus A)\\
                &= \mu(A\cup B) - \mu(B) + \mu(A\cup B) - \mu(A)\\
                &= 2\mu(A\cup B) - \mu(B) - \mu(A)
\end{aligned}
\]
\end{proof}

%----------------------------------------------------------------------------------------
%----------------------------------------------------------------------------------------
%	PROBLEM 3
%----------------------------------------------------------------------------------------
\section*{Problem 3}
For a measure $\mu$ on $\A$, show that the countable union of null sets is a null set.
\\
\begin{proof}
Let $\{E_i\}$ be a countable collection of null sets (that is, $\mu(E_i)=0\ \forall i$).
Then we have the following:
\[
\begin{aligned}
\mu\left(\bigcup_{i=1}^{\infty} E_i\right) &\leq \sum{i=1}^{\infty}\mu(E_i)\\
        &= \sum_{i=1}^{\infty}0\\
        &= 0
\end{aligned}
\]
Furthermore, since the measure is positive, we have
$\mu\left(\bigcup_{i=1}^{\infty} E_i\right) \geq 0$.
Thus, the measure of the union is bounded above and below by zero, so it must be equal
to zero, as desired.
\end{proof}
%----------------------------------------------------------------------------------------
%----------------------------------------------------------------------------------------
%	PROBLEM 4
%----------------------------------------------------------------------------------------
\section*{Problem 4}
For $\mu$ a Borel measure on $\R$, with $\mu(\R)<\infty$, define the function
\[
\begin{aligned}
f_{\mu}&: \R\to\R\\
f_{\mu}(t)&=\mu((-\infty,t]))
\end{aligned}
\]

\subsection*{Part a}
Prove that $f_{\mu}$ is nondecreasing.
\\
\begin{proof}
Let $t\in\R$, and let $\Delta t$ be some positive number. Then:
\[
\begin{aligned}
f_{\mu}(t+\Delta t) &= \mu((-\infty,t+\Delta t])\\
                    &\geq \mu((-\infty,t])\\
                    &= f_{\mu}(t)
\end{aligned}
\]
as desired. Here we used the fact that $(-\infty,t] \subset (-\infty,t+\Delta t]$ and that,
for sets $A$ and $B$ such that $A\subset B$, the inequality
\[
\mu(A) \leq \mu(B)
\]
holds.
\end{proof}

\subsection*{Part b}
Find $f_{\mu}(a) - f_{\mu}(b)$.
\\
\begin{proof}
Without loss of generality, let $a>b$. (Note that $f_{\mu}(a) - f_{\mu}(b) = -(f_{\mu}(b) - f_{\mu}(a))$,
so in the case $a<b$, the result will be the negative of the case for $a>b$).

With this assumption, we have:
\[
\begin{aligned}
f_{\mu}(a) - f_{\mu}(b) &= \mu((-\infty, a]) - \mu((-\infty, b])\\
                        &= \mu((-\infty, a]\cup(-\infty,b]) - \mu((-\infty,b])\\
                        &= \mu((-\infty, a]\setminus (-\infty,b])\\
                        &= \mu((b,a])
\end{aligned}
\]
Here, line 3 was obtained by using the identity for the measure of $A\setminus B$ from
problem 2 above.
\end{proof}

\subsection*{Part c}
Prove that $f_{\mu}$ is continuous from the right.
\\
\begin{proof}
Let $\{x_i\}$ be a monotonically decreasing sequence with limit $x_i\to t$.
In particular, since $(-\infty,t]\subset (-\infty,x_i]$ for all $i$, the monotonicity
of the measure guarantees that $f_{\mu}(t)\leq f_{\mu}(x_i)$ for all $i$. Thus, the
sequence $\{f_{\mu}(x_i)\}$ is bounded below by $f_{\mu}(t)$.

Now, let $\epsilon > 0$  be arbitrary, and choose an $N$ such that for all $n>N$, $x_n < t+\epsilon$.
Since $(-\infty,x_n]\subset(-\infty,t+\epsilon]$, the monotonicity of the measure gives us
the fact that $f_{\mu}(x_n)\leq f_{\mu}(t)$.

Thus, we have that for all $\epsilon > 0$ and for all $n>N$, the inequality
\[
f_{\mu}(t)\leq f_{\mu}(x_n)\leq f_{\mu}(t+\epsilon)
\]
Thus, $f_{\mu}(x_i)\to f_{\mu}(t)$ as desired.
\end{proof}

\subsection*{Part d}
Give an example of a $\mu$ for which $f_{\mu}$ is not continuous.
\\
\begin{proof} %TODO proof or example?
Consider the $\delta$-measure centered at $0$ on $\R$. Then for all $\epsilon > 0$,
$f_{\mu}(0-\epsilon) = \mu((-\infty,-\epsilon]) = 0$, but $f_{\mu}(0) = \mu((-\infty,0])=1$.
Clearly, $f_{\mu}$ is not continuous, as desired.
\end{proof}

\subsection*{Part e}
Find $\lim_{t\to -\infty}f_{\mu}(t)$ and $\lim_{t\to\infty}f_{\mu}(t)$.
\\
\begin{proof}
For this proof, we will use the fact that every subsequence of a convergent sequence is
also convergent and converges to the same limit. In particular, for a sequence $\{t_i\}$
diverging to $+\infty$ such that $f_{\mu}(t_i)\to C$ for some $C$, there exists
a monotonically increasing subsequence $\{t_{i_j}\}$ such that $f_{\mu}(t_{i_j})\to C$ as
well. Without loss of generality, then, we will let $t$ monotonically increase when 
computing the limits.

First, we compute $\lim_{t\to -\infty}f_{\mu}(t)$. This can be formulated as
\[
\begin{aligned}
\lim_{t\to \infty}f_{\mu}(-t) &= lim_{t\to \infty}\mu((-\infty,-t])\\
        &= \mu\left(\lim_{t\to\infty}(-\infty,-t]\right)\\
        &= \mu(\emptyset)\\
        &= 0
\end{aligned}
\]
Here, we used the fact that the measure preserves limits (Notes part 1, section 23) to
establish equality in line 2, also observing that the sets $A_t = (-\infty,-t]$ are monotonic
($A_{t+\alpha}\subset A_t$ for positive $\alpha$). It is also clear that the limit is 
the empty set, since for each $x\in\R$, there is some $t<x$, and $x\not\in A_t$ which
implies $x$ is not in the limit.

Now, lets compute $\lim_{t\to\infty}f_{\mu}(t)$. This will use the same fact about measures
(preserving limits).
\[
\begin{aligned}
\lim_{t\to\infty}f_{\mu}(t) &= \lim_{t\to\infty}\mu((-\infty, t])\\
        &= \mu(\lim_{t\to\infty}(-\infty,t])\\
        &= \mu(\R)
\end{aligned}
\]
It is easy to see the sequence of sets $A_t = (-\infty,t]$ is a monotonically increasing
sequence, so the limit preservation property of measures holds. Furthermore, their limit
is easily seen to be all of $\R$, since for each $x\in\R$, there is some $t>x$, and thus
$x\in A_t$, so $x$ is also in the limit.
\end{proof}

%----------------------------------------------------------------------------------------
%----------------------------------------------------------------------------------------
%	PROBLEM 5
%----------------------------------------------------------------------------------------
\section*{Problem 5}
Show that the ``infinity-detecting'' function defined as
\[
\mu(E) = 
\begin{cases}
0 &\textrm{if } E \textrm{ is finite}\\
\infty &\textrm{else}
\end{cases}
\]
on a countable set $\Omega$ is additive, but not $\sigma$-additive.
\\
\begin{proof}
To show additivity, let $\{E_i\}_{i=1}^n$ be a finite collection of disjoint sets of $\Omega$.
If all $E_i$ are finite, then their union is finite, and the function $\mu$ is trivially
additive.
\[
\begin{aligned}
\mu\left(\bigcup_{i=1}^n E_i\right) &= 0\\
                                    &= \sum_{i=1}^n 0\\
                                    &= \sum_{i=1}^n \mu(E_i)
\end{aligned}
\]
Now suppose there is some $E_j$ infinite. Then,
\[
\begin{aligned}
\mu\left(\bigcup_{i=1}^n E_i\right) &= \infty\\
                                    &= \mu(E_j) + \sum_{i\neq j}^n \mu(E_i)\\
                                    &= \sum_{i=1}^n \mu(E_i)
\end{aligned}
\]
Where here we used the convention that $\infty + C = \infty$ for any $C$ finite or $C=\infty$.

However, this function is not $\sigma$-additive. To see this, note that
\[
\mu(\Omega) = \infty
\]
But 
\[
\Omega = \bigcup_{x\in\Omega}\{x\}
\]
and 
\[
\sum_{x\in\Omega}\mu(\{x\}) = \sum 0 = 0
\]
Thus,
\[
\begin{aligned}
\mu\left(\bigcup_{x\in\Omega}\{x\}\right)   &= \infty\\
                                            &\neq 0\\
                                            &= \sum_{x\in\Omega}\mu(\{x\})
\end{aligned}
\]
and $\mu$ is not $\sigma$-additive.
\end{proof}
%----------------------------------------------------------------------------------------
%----------------------------------------------------------------------------------------
%	PROBLEM 6
%----------------------------------------------------------------------------------------
\section*{Problem 6}
Prove that the conditional probability measure, induced by some measure $\mu$ on $\Omega$
with $\mu(\Omega) = 1$ and for some $\Gamma$, $\mu(\Gamma)$ positive and finite, given by
\[
\nu(E) = \frac{\mu(E\cap\Gamma)}{\mu(\Gamma)}
\]
is a measure.
\\
\begin{proof}
To show this is a measure, we will show it respects the nullity of the empty set, and that
it is $\sigma$-additive.

To verify the nullity of the empty set, we compute $\nu(\emptyset)$ directly.
\[
\begin{aligned}
\nu(\emptyset)  &= \frac{\mu(\emptyset\cap\Gamma)}{\mu(\Gamma)}\\
                &= \frac{\mu(\emptyset)}{\mu(\Gamma)}\\
                &= \frac{0}{\mu(\Gamma)}\\
                &= 0
\end{aligned}
\]

Now, let's verify $\sigma$-additivity. Let $\{E_i\}$ be a countable collection
of disjoint subsets of $\Omega$. Then, we can directly verify that
\[
\begin{aligned}
\nu\left(\bigcup_{i=1}^{\infty} E_i\right)  &= \frac{\mu\left(\left(\bigcup_{i=1}^{\infty} E_i\right)\cap\Gamma\right)}{\mu(\Gamma)}\\
            &=\frac{\mu\left(\left(\bigcup_{i=1}^{\infty} E_i\cap\Gamma\right)\right)}{\mu(\Gamma)}\\
            &=\frac{\sum_{i=1}^{\infty} \mu(E_i\cap\Gamma)}{\mu(\Gamma)}\\
            &=\sum_{i=1}^{\infty}\nu(E_i)
\end{aligned}
\]
Thus, $\nu$ is $\sigma$-additive, and is a measure as desired.
\end{proof}

%----------------------------------------------------------------------------------------
%----------------------------------------------------------------------------------------
%	PROBLEM 7
%----------------------------------------------------------------------------------------
\section*{Problem 7}
Let $\Omega$ be an uncountable set, and let the $\sigma$-algebra $\A$ be the set of all
subsets $E\subset\Omega$ such that either $E$ or $E^c$ is at most countable, and define
$\mu(E) = 0$ if $E$ is countable, and $\mu(E) = 1$ if $E^c$ is countable. Prove that $\mu$
is a measure.
\\
\begin{proof}
To show this is a measure, we will show it respects the nullity of the empty set, and that
it is $\sigma$-additive.

By the definition of the measure (along with the convention that $\emptyset$ is a finite
set), $\mu(\emptyset) = 0$

Now, let $\{E_i\}$ be a disjoint collection of measurable subsets of $\Omega$. Before
proceeding, it is key to note that at most one of these will be uncountable. To see this,
suppose both $E_i$ and $E_j$ are uncountable (with countable complement). In particular,
$E_i\not\subset (E_j)^c$ and vice versa, since $(E_j)^c$ is countable, but $E_i$ is
uncountable. Thus, they cannot be disjoint.

So, suppose none of $E_i$ are uncountable. Then,
\[
\begin{aligned}
\mu\left(\bigcup_{i=1}^{\infty} E_i\right)  &= 0\\
                                            &= \sum_{i=1}^{\infty} 0\\
                                            &= \sum_{i=1}^{\infty} \mu(E_i)\\
\end{aligned}
\]
as desired.

Instead, suppose some $E_j$ is uncountable. Then,
\[
\begin{aligned}
\mu\left(\bigcup_{i=1}^{\infty} E_i\right)  &= 1\\
                                            &= 1 + \sum_{i\neq j}^{\infty} 0
                                            &= \mu(E_j) + \sum_{i\neq j}^{\infty} \mu(E_i)\\
                                            &= \sum_{i=1}^{\infty} \mu(E_i)
\end{aligned}
\]
as desired.

\end{proof}
%----------------------------------------------------------------------------------------
\end{document}
