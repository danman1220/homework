%%%%%%%%%%%%%%%%%%%%%%%%%%%%%%%%%%%%%%%%%
% Short Sectioned Assignment
% LaTeX Template
% Version 1.0 (5/5/12)
%
% This template has been downloaded from:
% http://www.LaTeXTemplates.com
%
% Original author:
% Frits Wenneker (http://www.howtotex.com)
%
% License:
% CC BY-NC-SA 3.0 (http://creativecommons.org/licenses/by-nc-sa/3.0/)
%
%%%%%%%%%%%%%%%%%%%%%%%%%%%%%%%%%%%%%%%%%

%----------------------------------------------------------------------------------------
%	PACKAGES AND OTHER DOCUMENT CONFIGURATIONS
%----------------------------------------------------------------------------------------

\documentclass[fontsize=11pt]{scrartcl} % 11pt font size

\usepackage[T1]{fontenc} % Use 8-bit encoding that has 256 glyphs
\usepackage[english]{babel} % English language/hyphenation
\usepackage{amsmath,amsfonts,amsthm} % Math packages
\usepackage{mathrsfs}

\usepackage[margin=1in]{geometry}

\usepackage{sectsty} % Allows customizing section commands
\allsectionsfont{\centering \normalfont\scshape} % Make all sections centered, the default font and small caps

\usepackage{fancyhdr} % Custom headers and footers
\pagestyle{fancyplain} % Makes all pages in the document conform to the custom headers and footers
\fancyhead{} % No page header - if you want one, create it in the same way as the footers below
\fancyfoot[L]{} % Empty left footer
\fancyfoot[C]{} % Empty center footer
\fancyfoot[R]{\thepage} % Page numbering for right footer
\renewcommand{\headrulewidth}{0pt} % Remove header underlines
\renewcommand{\footrulewidth}{0pt} % Remove footer underlines
\setlength{\headheight}{13.6pt} % Customize the height of the header

\numberwithin{equation}{section} % Number equations within sections (i.e. 1.1, 1.2, 2.1, 2.2 instead of 1, 2, 3, 4)
\numberwithin{figure}{section} % Number figures within sections (i.e. 1.1, 1.2, 2.1, 2.2 instead of 1, 2, 3, 4)
\numberwithin{table}{section} % Number tables within sections (i.e. 1.1, 1.2, 2.1, 2.2 instead of 1, 2, 3, 4)

\newcommand{\R}{\mathbb{R}}
\newcommand{\Q}{\mathbb{Q}}
\newcommand{\N}{\mathbb{N}}
\newcommand{\C}{\mathbb{C}}

\newtheorem*{lemma}{Lemma}
%----------------------------------------------------------------------------------------
%	TITLE SECTION
%----------------------------------------------------------------------------------------

\newcommand{\horrule}[1]{\rule{\linewidth}{#1}} % Create horizontal rule command with 1 argument of height

\title{	
\normalfont \normalsize 
\textsc{Analysis} \\ [25pt] % Your university, school and/or department name(s)
\horrule{0.5pt} \\[0.4cm] % Thin top horizontal rule
\huge Final Exam \\ % The assignment title
\horrule{2pt} \\[0.5cm] % Thick bottom horizontal rule
}

\author{Daniel Halmrast} % Your name

\date{\normalsize\today} % Today's date or a custom date

\begin{document}

\maketitle % Print the title

Signature:
\\
\\
\\

\section*{Problem 1}
For every $n\in\N$, let $\mu_n$ be a measure on $(\Omega,\mathscr{A})$ with
$\mu_n(\Omega)=1$. For every $E\in\mathscr{A}$, define
\[
    \mu(E) = \sum_{n=1}^{\infty}\frac{\mu_n(E)}{2^n}
\]
Give a careful proof that $\mu$ is a measure on $(\Omega,\mathscr{A})$ with
$\mu(\Omega)=1$.
\\
\begin{proof}
    We wish to prove that $\mu$ is a measure on $(\Omega,\mathscr{A})$. That is,
    we wish to show that that $\mu(\emptyset) = 0$, that $\mu(E)\geq 0$ for all
    $E\in\mathscr{A}$, and that for a countable collection of disjoint sets
    $\{E_j\}_{j=1}^{\infty}$ for which $E_j\in\mathscr{A}$ for all $j$,
    \[
        \mu\left(\bigcup_{j=1}^{\infty}E_j\right) = \sum_{j=1}^{\infty}\mu(E_j)
    \]

    To begin with, we note that since each $\mu_n$ is a measure, we have that
    $\mu_n(\emptyset) = 0$.
    Thus,
    \[
        \begin{aligned}
            \mu(\emptyset)  &= \sum_{n=1}^{\infty}\frac{\mu_n(\emptyset)}{2^n}\\
                            &= \sum_{n=1}^{\infty}\frac{0}{2^n}\\
                            &=0
        \end{aligned}
    \]
    as desired.
    \\
    \\
    Next, we note that since each $\mu_n$ is a measure, $\mu_n(E)\geq 0$ for all
    $E\in\mathscr{A}$. Thus, since both $\mu_n(E)$ and $2^n$ are greater than
    zero for each $n$, it must be that 
    \[
        \mu(E) = \sum_{n=1}^{\infty}\frac{\mu_n(E)}{2^n} \geq 0
    \]
    as desired.
    \\
    \\
    To show that $\mu$ is countably additive, we first prove the following lemma:
    \begin{lemma}
        For a doubly indexed sequence $\{a_{ij}\}$ of positive numbers,
        \[
            \sum_{i=1}^{\infty}\sum_{j=1}^{\infty}a_{ij} =
            \sum_{j=1}^{\infty}\sum_{i=1}^{\infty}a_{ij}
        \]
    \end{lemma}
    \begin{proof}
        We note first that $a_{ij}$ can be thought of as a function from
        $\N\times \N$ to $\R$. 
        
        Now, Tonelli's theorem tells us that for any positive function
        $f:\Omega\times\Sigma\to\R$ on the product space $\Omega\times\Sigma$
        of $\sigma$-finite measure spaces $(\Omega,\mathscr{A},\mu)$ and
        $(\Sigma,\mathscr{B},\nu)$ such that $f$ is measurable with respect to
        $\mathscr{A}\otimes\mathscr{B}$, we have that
        \[
            \int_{\Omega}\left(\int_{\Sigma}f(x,y)d\nu(y)\right)d\mu(x) =
            \int_{\Sigma}\left(\int_{\Omega}f(x,y)d\mu(x)\right)d\nu(y)
        \]

        Now, consider the case where $\Omega = \Sigma = \N$, $\mathscr{A} =
        \mathscr{B} = 2^{\N}$, and $\mu = \nu = \mu_c$ the counting measure.
        The function $a_{ij}$ from $\N\times\N\to\R$ is positive (by
        hypothesis), and is measurable on $2^{\N}\otimes 2^{\N} =
        2^{\N\times\N}$, since every function is measurable with respect to this
        $\sigma$-algebra. Thus, applying Tonelli's theorem yields
        \[
            \begin{aligned}
                \sum_{i=1}^{\infty}\left(\sum_{j=1}^{\infty}a_{ij}\right) &=
                \int_{\N}\left(\int_{\N}a_{ij}d\mu_c(j)\right)d\mu_c(i)\\
                &= \int_{\N}\left(\int_{\N}a_{ij}d\mu_c(i)\right)d\mu_c(j)\\
                &=\sum_{j=1}^{\infty}\left(\sum_{i=1}^{\infty}a_{ij}\right)
            \end{aligned}
        \]
        as desired.
    \end{proof}
    Equipped with this result, we now prove that $\mu$ is countably additive.
    To do so, let $\{E_j\}_{j=1}^{\infty}$ be a countable collection of disjoint
    measurable sets. Now, we know by the fact that each $\mu_n$ is a measure
    that
    \[
        \mu_n\left(\bigcup_{j=1}^{\infty}E_j\right) = \sum_{j=1}^{\infty}\mu_n(E_j)
    \]
    Thus, we have
    \[
        \begin{aligned}
            \mu\left(\bigcup_{j=1}^{\infty}E_j\right) &=
            \sum_{n=1}^{\infty}\frac{1}{2^n}\mu_n\left(\bigcup_{j=1}^{\infty}E_j\right)\\
            &=\sum_{n=1}^{\infty}\sum_{j=1}^{\infty}\frac{1}{2^n}\mu_n(E_j)
        \end{aligned}
    \]
    We apply the above lemma to get
    \[
        \begin{aligned}
            \sum_{n=1}^{\infty}\sum_{j=1}^{\infty}\frac{1}{2^n}\mu_n(E_j)
            &= \sum_{j=1}^{\infty}\sum_{n=1}^{\infty}\frac{1}{2^n}\mu_n(E_j)\\
            &=\sum_{j=1}^{\infty}\mu(E_j)
        \end{aligned}
    \]
    as desired.
    \\
    \\
    Finally, we wish to show that $\mu(\Omega) = 1$. This follows from direct
    computation (observing that $\mu_n(\Omega)=1$ for all $n$):
    \[
        \begin{aligned}
            \mu(\Omega) &= \sum_{n=1}^{\infty}\frac{\mu_n(\Omega)}{2^n}\\
            &= \sum_{n=1}^{\infty}\frac{1}{2^n}\\
            &= \frac{1}{1-\frac{1}{2}}-1\\
            &=1
        \end{aligned}
    \]
    as desired. Here, we used the standard formula for a geometric series
    \[
        \sum_{n=1}^{\infty}a^n = \frac{1}{1-a}-1
    \]
    for $0<a<1$.
\end{proof}

\newpage

\section*{Problem 2}
Suppose $\mu(\Omega)<\infty$. Prove that
\[
    \lim_{p\to\infty} \|f\|_{L^p} = \|f\|_{L^{\infty}}
\]

\begin{proof}
    We note first that the trivial case of $\|f\|_{L^{\infty}} = 0$ is clear,
    since 
    \[
        \begin {aligned}
        \|f\|_{L^{\infty}}=0 &\implies f = 0\ \mu-\textrm{almost everywhere}\\
                            &\implies \|f\|_{L^p}=0\ \forall p\\
                            &\implies \lim_{p\to\infty}\|f\|_{L^p} = 0
\end{aligned}
    \]
    Therefore, for the rest of this problem, it is assumed that
    $\|f\|_{L^{\infty}}>0$.


    Suppose first that $\|f\|_{L^{\infty}} < \infty$. Then, we are free to scale
    $f$ so that $\|f\|_{L^{\infty}}=1$. (This is clear, since
    \[
        \lim_{p\to\infty}\|cf\|_{L^p} = c\lim_{p\to\infty}\|f\|_{L^p}
    \]
    so
    \[
        \lim_{p\to\infty}\|cf\|_{L^p} = \|cf\|_{L^{\infty}}
        \iff
        \lim_{p\to\infty}\|f\|_{L^p} = \|f\|_{L^{\infty}}
    \]
    and so multiplying $f$ by a constant will not change the equality.)

    So, without loss of generality, let $\|f\|_{L^{\infty}} = 1$.
    We will show first that
    \[
        \lim_{p\to\infty}\|f\|_{L^p} \leq 1
    \]
    and then that
    \[
        \lim_{p\to\infty}\|f\|_{L^p} \geq 1
    \]

    First, we prove that $\lim_{p\to\infty}\|f\|_{L^p} \leq 1$. To do so, we
    consider the altered function
    \[
        \tilde{f}(x) = \begin{cases}
            f(x), &\textrm{if } f(x) \leq \|f\|_{L^\infty}\\
            0, &\textrm{if } f(x) > \|f\|_{L^\infty}
        \end{cases}
    \]
    Now, we know that $\mu\{|f| > \|f\|_{L^{\infty}}\} = 0$ by the definition of
    the $L^{\infty}$ norm, so it follows that $\tilde{f}$ and $f$ differ only on
    a set of measure zero, and thus are in the same equivalence class in $L^p$
    for all $p$.

    So, we have that $\tilde{f} \leq \|f\|_{L^{\infty}} = 1$, and thus
    $\tilde{f}^p\leq 1$ for all $p\geq 1$. Therefore,
    \[
        \begin{aligned}
            \int_{\Omega}|\tilde{f}(x)|^pd\mu &\leq \int_{\Omega}1d\mu\\
                                            &=\mu(\Omega)\\
        \end{aligned}
    \]
    which implies that
    \[
        \begin{aligned}
        \|f\|_{L^p} &=
        \left(\int_{\Omega}|\tilde{f}(x)|^pd\mu\right)^{\frac{1}{p}}\\
            &\leq (\mu(\Omega))^{\frac{1}{p}}
        \end{aligned}
    \]
    and for $\mu(\Omega)<\infty$, we have that
    $\lim_{p\to\infty}(\mu(\Omega))^{\frac{1}{p}} = 1$.
    Thus,
    \[
        \lim_{p\to\infty}\|f\|_{L^p} \leq 1
    \]
    as desired.
    \\
    \\
    Next, we prove that
    \[
        \lim_{p\to\infty}\|f\|_{L^p} \geq 1
    \]
    To do so, we consider the set $\{|f|>1-\epsilon\}$, which has positive
    measure for every $\epsilon>0$ by the fact that $\|f\|_{L^{\infty}} = 1$.
    Thus, we know that
    \[
        \begin{aligned}
            \|f\|_{L^p} &=\left(\int_{\Omega}|f|^pd\mu\right)^{\frac{1}{p}}\\
                        &\geq
                        \left(\int_{\{|f|>1-\epsilon\}}|1-\epsilon|^pd\mu\right)^{\frac{1}{p}}\\
                        &=((1-\epsilon)^p\mu(\{|f|>1-\epsilon\}))^{\frac{1}{p}}\\
                        &=(1-\epsilon)(\mu(\{|f|>1-\epsilon\}))^{\frac{1}{p}}
        \end{aligned}
    \]
    Since $\lim_{p\to\infty}(\mu(\{|f|>1-\epsilon\}))^{\frac{1}{p}}
    = 1$, it follows that
    \[
        \lim_{p\to\infty}\|f\|_{L^p} \geq (1-\varepsilon)(1)
    \]
    and since this holds for any $\epsilon>0$, it follows that
    \[
        \lim_{p\to\infty}\|f\|_{L^p} \geq 1
    \]
    as desired.

    Thus, for $\|f\|_{L^{\infty}} < \infty$, we have that
    \[
        \lim_{p\to\infty}\|f\|_{L^p} = \|f\|_{L^{\infty}}
    \]
    as desired.

    Suppose instead that $\|f\|_{L^{\infty}} = \infty$. That is, for each $M>0$,
    $\mu(\{|f|>M\})>0$. It follows that
    \[
        \begin{aligned}
            \|f\|_{L^p} &= \left(\int_{\Omega}|f|^pd\mu\right)^{\frac{1}{p}}\\
                        &\geq \left(\int_{|f|>M}M^pd\mu\right)^{\frac{1}{p}}\\
                        &= \left(M^p\mu(\{f>M\})\right)^{\frac{1}{p}}\\
                        &= M(\mu(\{|f|>M\}))^{\frac{1}{p}}
        \end{aligned}
    \]
    We know already that $\lim_{p\to\infty}(\mu(\{|f|>M\}))^{\frac{1}{p}} = 1$,
    so it follows that
    \[
        \lim_{p\to\infty}\|f\|_{L^p} \geq M(1) = M
    \]
    Since $M$ was arbitrary, it follows that
    \[
        \lim_{p\to\infty}\|f\|_{L^p} = \infty
    \]
    as desired.
\end{proof}

\newpage

\section*{Problem 3}
For $n\in\N$, define the $n^{th}$ truncation of a positive measurable function
$f$ to be
\[
    f_n(x) = \begin{cases}
        f(x), &f(x)\leq n\\
        n, &f(x)\geq n
    \end{cases}
\]
Prove that 
\[
    f\in L^1 \iff \sup_n \|f_n\|_{L^1} < \infty
\]

\begin{proof}
We observe first that $f_n$ converges pointwise to $f$. This is clear, since
    for each $x$ such that $f(x)<\infty$, there is some $N$ for which $f(x)<N$, and
    thus $\forall m>N, f_m(x) = f(x)$ by the definition of $f_n(x)$.
    Furthermore, for each $x$ such that $f(x)=\infty$, it follows that $\forall n, f(x)>n$ and
    thus $f_n(x) = n$, which tends to infinity as $n$ goes to infinity. Thus,
    \[
        \lim_{n\to\infty}f_n(x) = f(x)\ \forall x
    \]

    The sequence $\{f_n\}$ is also monotonically increasing. To see
    this, we consider three cases:
    \\
    \\
    First, suppose $x$ is such that $f(x) \leq n$. In that case,
    \[
        f_n(x) =
        f_{n+1}(x) = f(x)
    \]
    and thus $f_n(x)\leq f_{n+1}(x)$.
    \\
    \\
    Second, suppose $x$ is such that $n<f(x)\leq n+1$. In this case, we have
    that
    \[
        \begin{aligned}
            f_n(x) &= n\\
                    &<f(x)\\
                    &=f_{n+1}(x)
        \end{aligned}
    \]
    and so $f_n(x)\leq f_{n+1}(x)$.
    \\
    \\
    Finally, suppose $x$ is such that $n+1<f(x)$. In this case, we have that
    \[
        \begin{aligned}
            f_n(x) &= n\\
                    &< n+1\\
                    &=f_{n+1}(x)
        \end{aligned}
    \]
    and so $f_n(x)<f_{n+1}(x)$ as desired.

    It follows, then, that
    \[
        \|f_n\|_{L^1}\leq\|f_{n+1}\|_{L^1}
    \]
    for all $n$, since the $L^1$ norm (the integral) preserves orders on
    positive functions. Therefore,
    \[
        \sup_n\|f_n\|_{L^1} = \lim_{n\to\infty}\|f_n\|_{L^1}
        =\lim_{n\to\infty}\int_{\Omega}f_nd\mu
    \]
    applying the monotone convergence theorem yields
    \[
        \begin{aligned}
            \sup_n\|f_n\|_{L^1} &= \lim_{n\to\infty}\int_{\Omega}f_nd\mu\\
                                &=\int_{\Omega}\lim_{n\to\infty}f_nd\mu\\
                                &=\int_{\Omega}fd\mu\\
                                &=\|f\|_{L^1}
        \end{aligned}
    \]
    as desired. Note that this holds even if $\|f\|_{L^1} = \infty$ by the
    monotone convergence theorem.

    Thus, $f\in L^1$ precisely when $\sup_n\|f_n\|_{L^1}<\infty$, as desired.
\end{proof}

\newpage

\section*{Problem 4}
Let $f$ be a measurable function on $(\Omega,\mathscr{A},\mu)$ which is finite
$\mu$-almost everywhere, and let $\mu(\Omega)<\infty$. Define
\[
    E_n = \{x\in\Omega\ |\ n-1\leq |f(x)|< n\}
\]
for $n\in\N$.

Prove that $f\in L^1 \iff \sum_{n=1}^{\infty}n\mu(E_n)<\infty$.
\\
\\
\begin{proof}
    To begin with, we alter the function $f$ on the set $\{x\ |\ f(x)=\infty\}$
    to be identically zero on that set. Thus, the new altered function (which we
    will continue to denote as $f$) is finite everywhere. Since $f$ was only
    altered on a set of measure zero, this will not affect either side of the
    equality we are trying to prove.

    We begin by showing that $\|f\|_{L^1}\leq \sum_{n=1}^{\infty}n\mu(E_n)$.
    To do so, we construct the function
    \[
        \phi(x) = \sum_{n=1}^{\infty}n\chi_{E_n}(x)
    \]
    Now, it is clear by the definition of $E_n$ that each $x\in\Omega$ belongs to
    exactly one $E_n$. Furthermore, for $x\in E_n$,
    \[
        |f(x)| < n = \phi(x)
    \]
    Thus, $|f(x)| < \phi(x)$ for all $x$.

    It follows (since integrals preserve inequalities of positive functions)
    that
    \[
        \begin{aligned}
            \|f\|_{L^1} &= \int_{\Omega}|f|d\mu\\
                                &< \int_{\Omega}\phi d\mu\\
                                &=\int_{\Omega}\sum_{n=1}^{\infty}n\chi_{E_n}
                                d\mu\\
                                &=\sum_{n=1}^{\infty}n\mu(E_n)
        \end{aligned}
    \]
    Thus, $\|f\|_{L^1} < \sum_{n=1}^{\infty}n\mu(E_n)$, and so
    \[
        \sum_{n=1}^{\infty}n\mu(E_n) <\infty \implies \|f\|_{L^1}<\infty
    \]
    as desired.
    \\
    \\
    Now, we will show that $\|f\|_{L^1} \geq \sum_{n=1}^{\infty}(n-1)\mu(E_n)$.
    To do so, we define
    \[
        \psi(x) = \sum_{n=1}^{\infty}(n-1)\chi_{E_n}(x)
    \]
    Clearly, we have that for $x\in E_n$,
    \[
        |f(x)| \geq n-1 = \psi(x)
    \]
    and since this holds for all $x$, we have that $|f| \geq \psi$. Thus, since
    integrals preserve inequalities of positive functions,
    \[
        \begin{aligned}
            \|f\|_{L^1} &= \int_{\Omega}|f|d\mu\\
                        &\geq \int_{\Omega}\psi d\mu\\
                        &=\sum_{n=1}^{\infty}(n-1)\mu(E_n)\\
                        &=\sum_{n=1}^{\infty}n\mu(E_n) -
                        \sum_{n=1}^{\infty}\mu(E_n)
        \end{aligned}
    \]
    Now, since the sets $E_n$ partition $\Omega$, we have that
    \[
        \begin{aligned}
            \sum_{n=1}^{\infty}\mu(E_n) &=
            \mu\left(\bigcup_{n=1}^{\infty}E_n\right)\\
            &=\mu(\Omega)
        \end{aligned}
    \]
    so it follows that
    \[
        \begin{aligned}
            \|f\|_{L^1} &\geq \sum_{n=1}^{\infty}n\mu(E_n) - \mu(\Omega)\\
            \implies \|f\|_{L^1}+\mu(\Omega) &\geq \sum_{n=1}^{\infty}n\mu(E_n)
        \end{aligned}
    \]
    Now, since $\mu(\Omega)$ is finite, it is clear that
    \[
        \|f\|_{L^1} < \infty \implies \sum_{n=1}^{\infty}n\mu(E_n) < \infty
    \]
    as desired.

    Thus, $f\in L^1 \iff \sum_{n=1}^{\infty}n\mu(E_n) < \infty$ as desired.
\end{proof}

\newpage

\section*{Problem 5}
Let $D_A$ be a dense subset of a normed space $X$ with norm $\|\cdot\|_1$. Let
$Y$ be a Banach space with norm $\|\cdot\|_2$, and let $A$ be a linear operator
with domain $D_A$ and codomain $Y$ such that for some $C<\infty$,
\[
    \|Ax\|_2 \leq C\|x\|_1\ \forall x\in D_A 
\]
Prove that there exists a unique linear bounded operator $\tilde{A}:X\to Y$ such
that $\tilde{A}|_{D_A} = A$. Give an estimate for $\|\tilde{A}\|$.
\\
\\
\begin{proof}
    To begin with, we note that for each $x\in X$, there exists
    some sequence $\{x_k\}$ in $D_A$ that converges to $x$ (since $D_A$ is dense
    in $X$). So, define a candidate function $\tilde{A}$ to be
    \[
        \tilde{A}(x) = \lim_{k\to\infty}A(x_k)
    \]
    where $\{x_k\}$ is some sequence converging to $x$.

    Now, we must prove that such an $\tilde{A}$ is well-defined, linear,
    bounded, and unique.
    \\
    \\
    To show that $\tilde{A}$ is well-defined, we need to show that
    $\lim_{k\to\infty}A(x_k)$ exists for every convergent sequence $x_k\to x$,
    and that such a limit does not depend upon the choice of convergent sequence
    chosen.

    So, let $\{x_k\}$ be a sequence in $D_A$ such that $x_k\to x$ for $x\in X$.
    Now, we know that $\{x_k\}$ is Cauchy, since it converges. Since $A$ is
    bounded, it follows that the sequence $A(x_k)$ is Cauchy as well. To see
    this, fix $\epsilon>0$, and choose an $N$ such that
    $\|x_m-x_n\|_1<\frac{\epsilon}{C}$ for all $m,n >N$. Then, we have that
    for all $m,n >N$,
    \[
        \begin{aligned}
            \|Ax_m -Ax_n\|_2 &= \|A(x_m-x_n)\|_2\\
                            &\leq C\|x_m-x_n\|_1\\
                            &<C\frac{\epsilon}{C}\\
                            &=\epsilon
        \end{aligned}
    \]
    Thus, $\|Ax_m-Ax_n\|_2 < \epsilon$ for all $m,n>N$, so $\{Ax_k\}$ is a Cauchy
    sequence. Since $Y$ is complete, this sequence must have a limit, and so the
    limit $\lim_{k\to\infty}Ax_k$ is well defined.

    Suppose, then, that both $\{x_k\}$ and $\{y_k\}$ both converge to $x$.
    Then, we have that
    \[
        \begin{aligned}
            \|Ax_k-Ay_k\|_2 &= \|A(x_k-y_k)\|_2\\
                        &\leq C\|x_k-y_k\|_1\\
                        &= C\|(x_k-x) + (x-y_k)\|_1\\
                        &\leq C(\|x_k-x\|_1 + \|y_k-x\|_1)
        \end{aligned}
    \]
    and since both $\|x_k-x\|_1$ and $\|y_k-x\|_1$ go to zero as $k$ goes to
    infinity, it follows that $\|Ax_k -Ay_k\|_2$ goes to zero as well.
    Thus,
    \[
        \lim_{k\to\infty}Ax_k = \lim_{k\to\infty}Ay_k
    \]
    and so $\tilde{A}(x)$ is well defined.
    \\
    \\
    Next, we wish to show $\tilde{A}$ is linear.
    To do so, consider $\tilde{A}(\alpha x + \beta y)$ for $\alpha,\beta\in\R$
    and $x,y\in X$.
    Now, if $\{x_k\}$ is a sequence converging to $x$, and $\{y_k\}$ is a
    sequence converging to $y$, we can define the sequence $\{\alpha x_k + \beta
    y_k\}$, which converges to $\alpha x + \beta y$ by algebraic limit theorems.
    Thus, we have that
    \[
        \begin{aligned}
            \tilde{A}(\alpha x+\beta y) &= \lim_{k\to\infty}A(\alpha x_k +\beta
            y_k)\\
            &=\lim_{k\to\infty}(\alpha Ax_k + \beta Ay_k)\\
            &=\alpha\lim_{k\to\infty}Ax_k + \beta\lim_{k\to\infty}Ay_k\\
            &=\alpha\tilde{A}x + \beta\tilde{A}y
        \end{aligned}
    \]
    and so $\tilde{A}$ is linear.
    \\
    \\
    Next, we wish to show that $\tilde{A}$ is bounded. To do so, we note two
    important properties of limits. First, limits preserve orderings (by the
    Order Limit Theorem), and second, limits pass through norms (by the
    continuity of the norm).
    
    Then, we have that
    \[
        \begin{aligned}
            \|\tilde{A}x\|_2 &= \|\lim_{k\to\infty}Ax_k\|_2\\
                    &=\lim_{k\to\infty}\|Ax_k\|_2\\
                    &\leq\lim_{k\to\infty}C\|x_k\|_1\  &(\textrm{since
                    }\|Ax_k\|_2\leq C\|x_k\|_1)\\
                    &=C\|\lim_{k\to\infty}x_k\|_1\\
                    &=C\|x\|_1
        \end{aligned}
    \]
    and so $\|\tilde{A}x\|_2\leq C\|x\|_1$ for all $x$, and $\tilde{A}$ is bounded,
    as desired.
    \\
    \\
    Finally, we show that such a $\tilde{A}$ is unique. Suppose that $A'$ were
    some other bounded linear extension of $A$. Now, since both $\tilde{A}$ and
    $A'$ are linear and bounded, they are continuous. Namely, they preserve
    limits. So, for each $x\in X$, let $\{x_k\}$ be a sequence in $D_A$
    converging to $x$. Now, since $A'$ and $\tilde{A}$ agree on $D_A$, we have
    that
    \[
        \tilde{A}x_k = A'x_k
    \]
    for all $x_k$. Then, taking the limit of both sides yields
    \[
        \tilde{A}x = \lim_{k\to\infty}\tilde{A}x_k = \lim_{k\to\infty}A'x_k =
        A'x
    \]
    and so $\tilde{A}$ and $A'$ agree for all $x\in X$, and thus $\tilde{A}=A'$.
    Thus, such an extension is unique.
    \\
    \\
    Finally, we observed in the proof of boundedness that $\|Ax\|_2\leq C\|x\|_1$
    for all $x$. Thus, it follows immediately by the definition of operator norm
    that
    \[
        \|A\|_{X\to Y} \leq C
    \]
    which provides an estimate for the operator norm of $A$.
\end{proof}

\end{document}
