%%%%%%%%%%%%%%%%%%%%%%%%%%%%%%%%%%%%%%%%%
% Short Sectioned Assignment
% LaTeX Template
% Version 1.0 (5/5/12)
%
% This template has been downloaded from:
% http://www.LaTeXTemplates.com
%
% Original author:
% Frits Wenneker (http://www.howtotex.com)
%
% License:
% CC BY-NC-SA 3.0 (http://creativecommons.org/licenses/by-nc-sa/3.0/)
%
%%%%%%%%%%%%%%%%%%%%%%%%%%%%%%%%%%%%%%%%%

%----------------------------------------------------------------------------------------
%	PACKAGES AND OTHER DOCUMENT CONFIGURATIONS
%----------------------------------------------------------------------------------------

\documentclass[fontsize=11pt]{scrartcl} % 11pt font size

\usepackage[T1]{fontenc} % Use 8-bit encoding that has 256 glyphs
\usepackage[english]{babel} % English language/hyphenation
\usepackage{amsmath,amsfonts,amsthm} % Math packages
\usepackage{mathrsfs}
\usepackage{tikz-cd}

\usepackage[margin=1in]{geometry}

\usepackage{sectsty} % Allows customizing section commands
\allsectionsfont{\centering \normalfont\scshape} % Make all sections centered, the default font and small caps

\usepackage{fancyhdr} % Custom headers and footers
\pagestyle{fancyplain} % Makes all pages in the document conform to the custom headers and footers
\fancyhead{} % No page header - if you want one, create it in the same way as the footers below
\fancyfoot[L]{} % Empty left footer
\fancyfoot[C]{} % Empty center footer
\fancyfoot[R]{\thepage} % Page numbering for right footer
\renewcommand{\headrulewidth}{0pt} % Remove header underlines
\renewcommand{\footrulewidth}{0pt} % Remove footer underlines
\setlength{\headheight}{13.6pt} % Customize the height of the header

\numberwithin{equation}{section} % Number equations within sections (i.e. 1.1, 1.2, 2.1, 2.2 instead of 1, 2, 3, 4)
\numberwithin{figure}{section} % Number figures within sections (i.e. 1.1, 1.2, 2.1, 2.2 instead of 1, 2, 3, 4)
\numberwithin{table}{section} % Number tables within sections (i.e. 1.1, 1.2, 2.1, 2.2 instead of 1, 2, 3, 4)

\newcommand{\R}{\mathbb{R}}
\newcommand{\Q}{\mathbb{Q}}
\newcommand{\N}{\mathbb{N}}
\newcommand{\C}{\mathbb{C}}

\newcommand{\im}{\textrm{im}}

\newtheorem{lemma}{Lemma}
%----------------------------------------------------------------------------------------
%	TITLE SECTION
%----------------------------------------------------------------------------------------

\newcommand{\horrule}[1]{\rule{\linewidth}{#1}} % Create horizontal rule command with 1 argument of height

\title{	
\normalfont \normalsize 
\textsc{Analysis} \\ [25pt] % Your university, school and/or department name(s)
\horrule{0.5pt} \\[0.4cm] % Thin top horizontal rule
\huge Problem Set 7 \\ % The assignment title
\horrule{2pt} \\[0.5cm] % Thick bottom horizontal rule
}

\author{Daniel Halmrast} % Your name

\date{\normalsize\today} % Today's date or a custom date

\begin{document}

\maketitle % Print the title

% Problems
\section*{Problem 1}
Show that for a subset $S$ of $V$, and $x\in V$, $d(x,S)=0 \iff
x\in\overline{V}$.
\\
\begin{proof}
    ($\implies$)
    Suppose that $d(x,S)=0$. This means that for any $\varepsilon > 0$, there is
    some $s\in S$ such that $d(x,s)<\varepsilon$.
    So, for any $\varepsilon$-ball centered at $x$, there is some $s\in S$ in
    it. Thus, every neighborhood of $x$ intersects $S$, and $x$ is in the
    closure of $S$.
    \\
    \\
    ($\impliedby$)
    Suppose that $x\in \overline{S}$. Then, we know that for every $\varepsilon
    >0$, the $\varepsilon$-ball around $x$ intersects $S$. In other words, for
    every $\varepsilon >0$, there is some $s\in S$ with $d(x,s)<\varepsilon$.
    It follows immediately, then, that $d(x,S)=\inf_{s\in S}d(x,s)=0$ as
    desired.
\end{proof}

\newpage

\section*{Problem 2}
Show that the $\ell^1$ norm is equivalent to the norm
\[
    \|x\| = 2\left|\sum_{n=1}^{\infty}x_n\right| +
    \sum_{n=2}^{\infty}\left(1+\frac{1}{n}\right)|x_n|
\]
\\
\begin{proof}
    For the first bound, we note that
\[
    \begin{aligned}
        \|x\| &= 2\left|\sum_{n=1}^{\infty}x_n\right| +
        \sum_{n=2}^{\infty}\left(1+\frac{1}{n}\right)|x_n|\\
        &\leq 2\sum_{n=1}^{\infty}|x_n| + \sum_{n=1}^{\infty}2|x_n|\\
        &= 2\|x_n\|_1 + 2\|x_n\|_1\\
        &= 4\|x_n\|_1
\end{aligned}
\]
so the new norm is bounded above by the $\ell^1$ norm.

For the other direction, we take a bit more care. We note that
\[
    \begin{aligned}
        \|x\| &= 2\left|\sum_{n=1}^{\infty}x_n\right| +
    \sum_{n=2}^{\infty}\left(1+\frac{1}{n}\right)|x_n|\\
            &\geq 2\left|\sum_{n=1}^{\infty}x_n\right| +
        \sum_{n=2}^{\infty}|x_n|\\
            &\geq \frac{1}{2}\left|\sum_{n=1}^{\infty}x_n\right| +
        \sum_{n=2}^{\infty} |x_n|\\
            &\geq \frac{1}{2}\left||x_1|-\sum_{n=2}^{\infty}|x_n|\right| +
        \sum_{n=2}^{\infty} |x_n|\\
            &\geq \frac{1}{2}|x_1| -\frac{1}{2}\sum_{n=2}^{\infty}|x_n| +
            \sum_{n=2}^{\infty}|x_n|\\
            &= \frac{1}{2}|x_1| + \frac{1}{2}\sum_{n=2}^{\infty}|x_n|\\
            &=\frac{1}{2}\|x\|_1
    \end{aligned}
\]
where the inequality from line 3 to line 4 is obtained by using the reverse
    triangle inequality on the infinite sum.

So the new norm is bounded below by the $\ell^1$ norm, and thus the norms
    are equivalent.
\end{proof}

\newpage

\section*{Problem 3}
Let $V,W$ be normed vector spaces, with $V$ Banach. Let $A\in \mathscr{B}(V,W)$.
Show that $V/{\ker A}$ is Banach with the quotient norm.
\\
\begin{proof}
    To begin with, we observe that the short exact sequence of continuous linear
    functions
    \[
        \begin{tikzcd}[column sep=small]
            0\arrow{r} &\ker A \arrow[hook]{r}{i} &V\arrow[two heads]{r}{\pi} &V/{\ker
            A}\arrow{r} &0 
        \end{tikzcd}
    \]
    exists. This is clearly exact, since for any $a\in \ker A$, $\pi(a) =
    [0]$, and furthermore, for any $v\in V$ with $\pi(v) = [0]$, we have that
    $v\in 0+\ker A = \ker A$. Thus, $\im i = \ker \pi$, and the sequence is
    exact.

    Furthermore, this sequence splits. This follows from the general result that
    any short exact sequence of vector spaces splits, but a proof for this case
    will be replicated.

    We construct a retract $r:V\to \ker A$ such that $r\circ i=id$. This is done
    by first fixing a basis $\mathscr{B}$ for $\ker A$. Then, a basis
    $\mathscr{B}'$ for $V$ can be chosen so that $\mathscr{B}' =
    \mathscr{B}\coprod\mathscr{C}$ for some set of vectors $\mathscr{C}$.
    Then, $r$ can be defined as
    \[
        r\left(\sum_{b\in\mathscr{B}}v_b b + \sum_{c\in\mathscr{C}}v_c c\right)
        = \sum_{b\in\mathscr{B}} v_b b
    \]
    which is clearly a linear continuous function. Furthermore, it is clear that
    $r\circ i = id$ by the definition of $r$. Thus, $r$ defines a retract of
    $i$, and the sequence splits.

    So, by the splitting lemma, we have that
    \[
        \begin{tikzcd}[column sep=small]
            0\arrow{r} &\ker A \arrow[hook,shift left]{r}{i} &V\arrow[two
            heads, shift left]{r}{\pi}\arrow[two heads, shift left]{l}{r} &V/{\ker
            A}\arrow{r}\arrow[hook', shift left]{l}{s} &0 
        \end{tikzcd}
    \]
    where $s$ is a section of $\pi$, and $r$ is a retract of $i$. Thus, we have
    the continuous injective map $s:V/{\ker A}\to V$.

    Now, let $[x]_n$ be a Cauchy sequence in $V/{\ker A}$. This sequence lifts
    along $s$ to the sequence $s([x]_n)$ in $V$. Now, since $V$ is complete,
    and $s$ is continuous, it follows that $s([x]_n)$ is Cauchy in $V$, and
    we have that $s([x]_n)\to x$ for some $x\in V$. 
    
    I now assert that $[x]_n\to\pi(x)$. To see this, we consider (since $\pi$ is
    continuous) the fact that $\pi(s([x]_n))\to \pi(x)$.

    But since $s$ is a section of $\pi$, it must be that $\pi\circ s=id$, so
    $\pi(s([x]_n)) = [x]_n \to\pi(x)$ as desired.
\end{proof}

\newpage

\section*{Problem 4}
Suppose $V=X_1\oplus X_2$ and that $X_1$ is finite dimensional. Show that
the projection operator $P_1:V\to X_1$ is in $\mathscr{B}(V,X_1)$.
\\
\begin{proof}
We will first show that $P_1$ is bounded by using the fact that all norms on
    finite dimensional vector spaces are equivalent. Equipped with this, and the
    fact that $X_1\cong V/{X_2}$, we see that
    \[
        \begin{aligned}
            \|P_1x\| &\leq C\|[x]\|_{Q}\\
                    &= C\inf_{x_2\in X_2}\|x-x_2\|\\
                    &\leq C\|x-0\|\\
                    &=C\|x\|
        \end{aligned}
    \]
    Thus, $P_1$ is bounded, and is in $\mathscr{B}(V,X_1)$.

    Now, since $P_2 = I-P_1$, we have that $\|P_2\| = \|I-P_1\|\leq
    \|I\|+\|P_1\|$, and since $I$ and $P_1$ are bounded, so is $P_2$.
\end{proof}

\section*{Problem 5}
For $V$ a normed space, $Y$ a closed subspace of $V$ with finite codimension,
show that for $\phi\in V'$ with $\phi$ continuous on $Y$, $\phi\in V^*$.
\\
\begin{proof}
    Consider the decomposition $V=Y\oplus V/{Y}$, and let $v=y+x\in V$. Now,
    we have that
    \[
        \begin{aligned}
            \|\phi(y+x)\| &= \|\phi(y)+\phi(x)\|\\
                            &\leq \|\phi(y)\|+\|\phi(x)\|
        \end{aligned}
    \]
    now, since $\phi$ is continuous on $Y$, it is bounded on $Y$, so
    $\|\phi(y)\|$ is bounded. Furthermore, since $V/Y$ is of finite dimension,
    it follows that $\phi$ is bounded on $V/Y$ as well, so $\|\phi(x)\|$ is
    bounded too.

    So, we have that
    \[
        \begin{aligned}
            \|\phi(v)\| &\leq \|\phi(y)\| + \|\phi(x)\|\\
                        &\leq C_1\|y\| + C_2\|x\|\\
                        &= C_1\|P_1(v)\| + C_2\|P_2(v)\|\\
                        &\leq (C_1K_1 + C_2K_2)\|v\|
        \end{aligned}
    \]
    where $K_1$ and $K_2$ are the bounds of the projection operators, which
    exist as a result of problem 4.

    Thus, $\|\phi(x+y)\|$ is bounded as well, and $\phi\in V^*$ as desired.
\end{proof}

\newpage

\section*{Problem 6}
Let $V_1,V_2$ be subspaces of some vector space $L$. Prove that
$(V_1+V_2)/{V_2}\cong V_1/{V_1\cap V_2}$. Furthermore, prove that if
$\dim L/{V_1} \leq n_1$ and $\dim L/{V_2}\leq n_2$, then
$\dim L/{V_1\cap V_2}\leq n_1+n_2$.
\\
\begin{proof}
The first result is a direct restatement of the second isomorphism theorem for
    Abelian groups. We replicate the proof here:
    \\
    Consider the surjective homomorphism $\phi:V_1+V_2\to V_1/{V_1\cap V_2}$
    as $\phi(v_1+v_2) = [v_1]$ for $v_i\in V_i$. The kernel of this homomorphism
    is any vector that gets sent to $[0]$, which is precisely the vectors
    $0+v_2$. It is clear that $0+V_2\subset \ker\phi$. Furthermore, suppose
    $v=v_1+v_2$ for some $v_1\neq 0$ ($v_1\not\in V_2$). Then,
    $\phi(v) = [v_1] \neq 0$ since $v_1\not\in V_2$. So, $\ker\phi=V_2$.

    Thus, by the first isomorphism theorem, we have that
    \[
        V_1+V_2/{V_2} \cong V_1/{V_1\cap V_2}
    \]
    as desired.
    \\
    \\
    We observe first that since $L/{V_2}$ is finite dimensional, so is
    $V_1+V_2/{V_2}$ since it is a subspace of $L/{V_2}$. And, if $\dim
    L/{V_2}\leq n_2$, then $\dim V_1+V_2/{V_2}\leq n_2$ as well.

    Now, consider that by the third isomorphism theorem, we have that
    \[
        L/{V_1} \cong \frac{L/{(V_1\cap V_2)}}{V_1/{(V_1\cap V_2)}}
    \]
    and since both $L/{V_1}$ and $V_1/{(V_1\cap V_2)}\cong V_1+V_2/{V_2}$ are
    finite dimensional, it must be that $L/{(V_1\cap V_2)}$ is as well.
    Furthermore, we have that
    \[
        \begin{aligned}
            \dim L/{V_1} &= \dim L/{(V_1\cap V_2)} - \dim V_1/{(V_1\cap V_2)}\\
            \dim L/{(V_1\cap V_2)} &= \dim L/{V_1} + \dim V_1/{(V_1\cap V_2)}\\
                                   &\leq n_1 + n_2
        \end{aligned}
    \]
    as desired.
\end{proof}

\end{document}
