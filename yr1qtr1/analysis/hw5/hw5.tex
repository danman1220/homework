%%%%%%%%%%%%%%%%%%%%%%%%%%%%%%%%%%%%%%%%%
% Short Sectioned Assignment
% LaTeX Template
% Version 1.0 (5/5/12)
%
% This template has been downloaded from:
% http://www.LaTeXTemplates.com
%
% Original author:
% Frits Wenneker (http://www.howtotex.com)
%
% License:
% CC BY-NC-SA 3.0 (http://creativecommons.org/licenses/by-nc-sa/3.0/)
%
%%%%%%%%%%%%%%%%%%%%%%%%%%%%%%%%%%%%%%%%%

%----------------------------------------------------------------------------------------
%	PACKAGES AND OTHER DOCUMENT CONFIGURATIONS
%----------------------------------------------------------------------------------------

\documentclass[fontsize=11pt]{scrartcl} % 11pt font size

\usepackage[T1]{fontenc} % Use 8-bit encoding that has 256 glyphs
\usepackage[english]{babel} % English language/hyphenation
\usepackage{amsmath,amsfonts,amsthm} % Math packages
\usepackage{mathrsfs}

\usepackage[margin=1in]{geometry}

\usepackage{sectsty} % Allows customizing section commands
\allsectionsfont{\centering \normalfont\scshape} % Make all sections centered, the default font and small caps

\usepackage{fancyhdr} % Custom headers and footers
\pagestyle{fancyplain} % Makes all pages in the document conform to the custom headers and footers
\fancyhead{} % No page header - if you want one, create it in the same way as the footers below
\fancyfoot[L]{} % Empty left footer
\fancyfoot[C]{} % Empty center footer
\fancyfoot[R]{\thepage} % Page numbering for right footer
\renewcommand{\headrulewidth}{0pt} % Remove header underlines
\renewcommand{\footrulewidth}{0pt} % Remove footer underlines
\setlength{\headheight}{13.6pt} % Customize the height of the header

\numberwithin{equation}{section} % Number equations within sections (i.e. 1.1, 1.2, 2.1, 2.2 instead of 1, 2, 3, 4)
\numberwithin{figure}{section} % Number figures within sections (i.e. 1.1, 1.2, 2.1, 2.2 instead of 1, 2, 3, 4)
\numberwithin{table}{section} % Number tables within sections (i.e. 1.1, 1.2, 2.1, 2.2 instead of 1, 2, 3, 4)

\newcommand{\R}{\mathbb{R}}
\newcommand{\Q}{\mathbb{Q}}
\newcommand{\N}{\mathbb{N}}
\newcommand{\C}{\mathbb{C}}

\theoremstyle{definition}
\newtheorem{lemma}{Lemma}
%----------------------------------------------------------------------------------------
%	TITLE SECTION
%----------------------------------------------------------------------------------------

\newcommand{\horrule}[1]{\rule{\linewidth}{#1}} % Create horizontal rule command with 1 argument of height

\title{	
\normalfont \normalsize 
\textsc{Analysis} \\ [25pt] % Your university, school and/or department name(s)
\horrule{0.5pt} \\[0.4cm] % Thin top horizontal rule
\huge Problem Set 5\\ % The assignment title
\horrule{2pt} \\[0.5cm] % Thick bottom horizontal rule
}

\author{Daniel Halmrast} % Your name

\date{\normalsize\today} % Today's date or a custom date

\begin{document}

\maketitle % Print the title

%----------------------------------------------------------------------------------------
%	PROBLEM 1
%----------------------------------------------------------------------------------------
\section*{Problem 1}
Prove the linearity of the general Lebesgue integral for complex-valued functions.
\\
\begin{proof}
To begin with, we prove a sequence of lemmas concerning the linearity of easier functions.
\begin{lemma}[Linearity of Positive Real Functions]
For $f,g$ positive, real-valued measurable functions from a measure space $(\Omega,\mu)$,
the integral is linear with respect to pointwise addition and positive scalar
multiplication. That is, for $\alpha, \beta \geq 0$,
\[
\int_{\Omega} \alpha f + \beta g d\mu = \alpha\int_{\Omega} fd\mu + \beta\int_{\Omega} gd\mu
\]
\end{lemma}
\begin{proof}
This has been proven in the notes, and will not be replicated here.
\end{proof}

\begin{lemma}
For $f$ a positive real-valued measurable function from a measure space $(\Omega,\mu)$,
the integral respects complex scalar multiplication. That is, for $\alpha\in\C$, 
\[
\int_{\Omega}\alpha fd\mu = \alpha\int_{\Omega} fd\mu
\]
\end{lemma}
\begin{proof}
We note first that for $\alpha=a+ib$, we have
\[
\begin{aligned}
\int_{\Omega}\alpha fd\mu   &= \int_{\Omega}(a+ib)fd\mu\\
                            &= \int_{\Omega}(af +ibf)d\mu\\
                            &= \int_{\Omega}afd\mu + i\int_{\Omega}bfd\mu
\end{aligned}
\]
So, it suffices to show that the integral of a real-valued function respects real
scalar multiplication. Without loss of generality, let $\alpha$ be real.

This lemma has already been proven for $\alpha\geq 0$. So, suppose $\alpha < 0$.
In particular, $-\alpha > 0$, and thus by straightforward application of the definition of
the Lebesgue integral along with lemma 1, we have
\[
\begin{aligned}
\int_{\Omega}\alpha fd\mu   &= \int_{\Omega}-(-\alpha f)d\mu\\
                            &= -\int_{\Omega}-\alpha fd\mu\\
                            &= -(-\alpha)\int_{\Omega} fd\mu\\
                            &= \alpha\int_{\Omega} fd\mu
\end{aligned}
\]
which is the desired result.
\end{proof}

\begin{lemma}
For $f_1,f_2,g_1,g_2$ positive real-valued measurable functions from a measure space
$(\Omega,\mu)$ such that $f_1-f_2 = g_1-g_2$,
\[
\int_{\Omega} f_1d\mu - \int_{\Omega} f_2d\mu = \int_{\Omega} g_1d\mu - \int_{\Omega}g_2d\mu
\]
\end{lemma}

\begin{proof}
This follows immediately from integrating $f_1-f_2$ and $g_1-g_2$, and applying lemma 2
and lemma 1.
\end{proof}
We are now ready to begin the proof of the linearity of the Lebesgue integral.

To do this proof, we will consider two cases. The first case will handle additivity,
and the second case will handle scalar multiplication.

For the first case, let $f=u_f+iv_f$, and $g=u_g+iv_g$ be complex-valued measurable functions.
Now, it follows immediately that
\[
\begin{aligned}
\int_{\Omega} (f+g)d\mu &= \int_{\Omega} ((u_f+iv_f) + (u_g+iv_g))d\mu\\
                        &= \int_{\Omega} (u_f + u_g)d\mu + i\int_{\Omega}(v_f+v_g)d\mu
\end{aligned}
\]
for real-valued functions $u_f,v_f,u_g,v_g$. So, it suffices to show the integral is
additive for real-valued functions, and the complex case will follow immediately.

So, assume without loss of generality that $f$ and $g$ are real. It follows directly that,
since $f=f^+-f^-$ and $g=g^+-g^-$, the equalities
\[
\begin{aligned}
f+g &= (f+g)^+ - (f+g)^-\\
    &= f^+ + g^+ - f^- - g^-
\end{aligned}
\]
hold. Thus, lemma 3 guarantees that
\[
\int_{\Omega}(f+g)^+d\mu - \int_{\Omega}(f+g)^-d\mu = \int_{\Omega}(f^+ + g^+)d\mu -\int_{\Omega}(f^-+g^-)d\mu
\]
Putting it all together, we have
\[
\begin{aligned}
\int_{\Omega}(f+g)d\mu  &= \int_{\Omega}(f+g)^+d\mu - \int_{\Omega}(f+g)^-d\mu\\
                        &= \int_{\Omega}(f^++g^+)d\mu - \int_{\Omega}(f^-+g^-)d\mu\\
                        &= \int_{\Omega}f^+d\mu + \int_{\Omega}g^+d\mu - \int_{\Omega}f^-d\mu-\int_{\Omega}g^-d\mu\\
                        &= \int_{\Omega}fd\mu + \int_{\Omega}gd\mu
\end{aligned}
\]
where the equality from the second to the third line is done using lemma 1.
Thus, the integral respects additivity of complex-valued functions.

For the second case, let $f$ be a complex-valued measurable function from a measure space
$(\Omega,\mu)$, and let $\alpha\in\C$. We wish to evaluate
\[
\int_{\Omega}\alpha fd\mu = \int_{\Omega}\alpha\Re(f)d\mu + i\int_{\Omega}\alpha\Im(f)d\mu
\]
for real-valued functions $\Re(f)$ and $\Im(f)$.

By lemma 2, we have
\[
\begin{aligned}
\int_{\Omega}\alpha fd\mu &= \int_{\Omega}\alpha\Re(f)d\mu + i\int_{\Omega}\alpha\Im(f)d\mu\\
                            &=\alpha\left(\int_{\Omega}\Re(f)d\mu + i\int_{\Omega}\Im(f)d\mu\right)\\
                            &=\alpha\int_{\Omega}fd\mu
\end{aligned}
\]

Thus, the integral is linear, as desired.
\end{proof}

%----------------------------------------------------------------------------------------
\newpage
%----------------------------------------------------------------------------------------
%	PROBLEM 2
%----------------------------------------------------------------------------------------
\section*{Problem 2}
\subsection*{Part 1}
Prove the statement in part 3.
\\
\begin{proof}
In particular, we have $\xi_k$ a sequence of partition refinements for which
\[
\phi_k = \sum_{I_j}\sup_{x\in I_j}f(x)\chi_{I_j}
\]
and
\[
\psi_k = \sum_{I_j}\inf_{x\in I_j}f(x)\chi_{I_j}
\]
(where $I_j$ is the $j^{th}$ interval in the partition $\xi_k$) converge to the
upper and lower sums (respectively) with respect to the partition $\xi_k$ as $k\to\infty$.
We also define $P=\bigcup_{k=1}^{\infty}\xi_k$ to be the set of all endpoints of the partitions $\xi_k$.

Now, we need to show that $f$ continuous at $x_0 \iff \phi(x_0) = f(x_0) = \psi(x_0)$ where
$\phi_k\to\phi$ and $\psi_k\to\psi$.

($\implies$)
To see the first implication, we observe that $f$ being continuous at $x_0$ means that
\[
\lim_{\delta\to 0}\sup_{x\in V_{\delta}(x_0)}f(x) - \inf_{x\in(V_{\delta}(x_0)}f(x) = 0
\]

Now, fix $\epsilon > 0$. Since $f$ is continuous, we can choose some $\delta$ such that
on the interval $(x_0-\delta,x_0+\delta)$, $\sup f -\inf f < \epsilon$. Furthermore,
we can choose some $k$ for which $x_0\in I_j$ for some interval in the partition $\xi_k$, and
such that $I_j\subseteq (x_0-\delta,x_0+\delta)$.

It follows immediately, then, that $\phi_k(x_0)-\psi_k(x_0)<\epsilon$, and thus as $k$
goes to infinity, we have $\phi(x_0)=\psi(x_0)$. And since $\psi_k \leq f\leq \phi_k\ \forall k$, 
we have $f(x_0) = \psi(x_0) = \phi(x_0)$ as desired.

($\impliedby$)
For the other direction, assume $f(x_0) = \psi(x_0) = \phi(x_0)$.
Now, fix $\epsilon >0$. In particular, we can choose an $N>0$ such that for all $k>N$,
$\phi(x_0)-\psi(x_0) <\epsilon$.

By the definition of $\phi_k$ and $\psi_k$, this means that
\[
\sup_{I_j}f - \inf_{I_j} < \epsilon
\]
where $I_j$ is the interval of $\xi_k$ that contains $x_0$. So, let $V$ be a $\delta$-neighborhood
of $x_0$ contained in $I_j$. Then, the variation of $f$ on $V$ is bounded by $\epsilon$, and
$f$ is continuous at $x_0$.
\end{proof}

\subsection*{Part 2}
Prove the statement in part 5.
\\
\begin{proof}
In particular, we need to prove that for a bounded function $f$ with a set of discontinuities
$E$ with measure zero, $f$ is both Riemann-integrable and Lebesgue-integrable, and that the
integrals coincide.

Notice that the previous step in the notes establishes that for $f$ a bounded Riemann-integrable function,
$f$ is also Lebesgue-integrable, and the integrals coincide. Thus, we only have to show
that $f$ is Riemann-integrable.

To do so, let $M$ be such that $|f|<M$, and let $E$ be the set of discontinuities of $f$.
Since $\lambda^1(E) = 0$, we can choose a set of disjoint open sets $\{U_i\}$ that cover $E$
with total measure $\lambda^1(\bigcup_i U_i) < \frac{\epsilon}{4M}$ (This follows from
the Borel-regularity of the Lebesgue measure). In particular, since the region of integration is
compact, we can make $\{U_i\}$ finite.

Now, let's attempt to estimate the difference in the upper and lower sums of $\int f$. 
\[
\begin{aligned}
\mathcal{U}(f,P) -\mathcal{L}(f,P)  &= \sum_{i=1}^n (\sup_{x\in I_i}f -\inf_{x\in I_i}f)\Delta x_i\\
\end{aligned}
\]
where the sum is being taken over the subintervals $I_i$ of the partition $P$.

Now, let's split the sum into the intervals that do not intersect $E$, and the intervals
that do. In particular, we take as our initial partition the set of intervals $U_i$, along
with the intervals in between them, so that the $U_i$ part of the partition contains every
element of $E$, and the other part does not.
\[
\begin{aligned}
\mathcal{U}(f,P) -\mathcal{L}(f,P)  &= \sum_{i=1}^m (\sup_{x\in I_i}f -\inf_{x\in I_i}f)\Delta x_i + \sum_{i=1}^{m'}(\sup_{x\in U_i}f -\inf_{x\in U_i}f)\Delta x_i\\
\end{aligned}
\]
Since $f$ is continuous in the first sum, we can refine the partition so that the upper and
lower sums are within $\frac{\epsilon}{2}$ of each other. Then, using the fact that
the measure of the union of the $\{U_i\}$ is less than $\frac{\epsilon}{4M}$, we obtain the bound
\[
\begin{aligned}
\mathcal{U}(f,P) -\mathcal{L}(f,P)  &\leq \frac{\epsilon}{2} + 2M\frac{\epsilon}{4M} = \epsilon\\
\end{aligned}
\]
Thus, the upper and lower sums converge to each other, and $f$ is Riemann-integrable as desired.
\end{proof}
%----------------------------------------------------------------------------------------
\newpage
%----------------------------------------------------------------------------------------
%	PROBLEM 3
%----------------------------------------------------------------------------------------
\section*{Problem 3}
Prove that the support of $f$ for $f\in L^1(\Omega,\mu)$ is $\sigma$-finite.
\\
\begin{proof}
We shall observe that the sets $E_n = \{f>\frac{1}{n}\}$ are of finite measure, and
union to the support of $f$.

The fact that $\bigcup_n E_n = \textrm{supp}(f)$ is immediate. All that remains is to
show that the measure of each of these is finite.

Suppose for a contradiction that for some $n$, $E_n$ has infinite measure. Then,
we can establish a lower bound for the integral
\[
\int_{\Omega}|f|d\mu \geq \int_{E_n}\frac{1}{n}d\mu
\]
but since $\mu(E_n)=\infty$, we have that
\[
\int_{E_n}\frac{1}{n}d\mu = \frac{1}{n}\mu(E_n) = \infty
\]
which cannot be a lower bound for $\int |f|$, since $f\in L^1$.

Thus, each $E_n$ is finite, and the support of $f$ is $\sigma$-finite as desired.
\end{proof}
%----------------------------------------------------------------------------------------
%----------------------------------------------------------------------------------------
%	PROBLEM 4
%----------------------------------------------------------------------------------------
\section*{Problem 4}
Show that if $f\in L^1$, then $f$ is finite almost everywhere. Provide an example where
the converse fails.
\\
\begin{proof}
Suppose for a contradiction that there was a set $E$ of positive measure for which $f=\infty$
on $E$. Then,
\[
\begin{aligned}
\int_{\Omega}|f|d\mu  &= \int_E|f|d\mu + \int_{E^c}|f|d\mu\\
                    &= \infty\mu(E) + \int_{E^c}|f|d\mu\\
                    &=\infty
\end{aligned}
\]
which is a clear contradiction.

However, the function $f(x) = 1$ is finite everywhere, but is not in $L^1$. Hence, the contrapositive
fails.
\end{proof}
%----------------------------------------------------------------------------------------
\newpage
%----------------------------------------------------------------------------------------
%	PROBLEM 5
%----------------------------------------------------------------------------------------
\section*{Problem 5}
Compute
\[
\lim_{n\to\infty}\left(\int_0^n (1+\frac{x}{n})^n\exp(-2x)dx\right)
\]

\begin{proof}
For this proof, we will first show that the doubly-indexed limit
\[
\lim_{n\to\infty}\lim_{i\to\infty}\left(\int_0^n (1+\frac{x}{i})^i\exp(-2x)dx\right)
\]
converges. This follows by first observing that inside the integral, we have
a sequence of functions approaching $\exp(x)\exp(-2x)=\exp(-x)$, which is clearly (eventually)
dominated by $\frac{1}{x^2}$. Thus, the dominated convergence theorem kicks in, and
\[
\begin{aligned}
\lim_{n\to\infty}\lim_{i\to\infty}\left(\int_0^n (1+\frac{x}{i})^i\exp(-2x)dx\right) &=
\lim_{n\to\infty}\left(\int_0^n \lim_{i\to\infty}(1+\frac{x}{i})^i\exp(-2x)dx\right)\\
                &= \lim_{n\to\infty}\left(\int_0^n\exp(-x)dx\right)\\
                &=\lim_{n\to\infty}(1-\exp(-n))\\
                &= 1
\end{aligned}
\]
Now, since this doubly-indexed limit converges, it must be the case that the singly-indexed
limit converges to the same thing. To see this, we note that any doubly-indexed sequence
$(n,i)$ has the diagonal $(n,n)$ as a subsequence. Thus, since every doubly-indexed sequence
converges, every singly-indexed sequence converges, so the limit converges as well.
\end{proof}
%----------------------------------------------------------------------------------------
\newpage
%----------------------------------------------------------------------------------------
%	PROBLEM 6
%----------------------------------------------------------------------------------------
\section*{Problem 6}
Suppose $f_n\geq0$ is measurable for all $n\in\N$. Furthermore, assume that $f_1\leq f_2\leq\ldots\leq 0$,
and $f_n(x)\to f(x)$ for each $x\in\Omega$, and that $f_1\in L^1$.
Prove that
\[
\lim_{n\to\infty}\int_{\Omega}f_nd\mu = \int_{\Omega}fd\mu
\]
and show that if the condition $f_1\in L^1$ is dropped, that the theorem does not hold in general.
\\
\begin{proof}
We note that every $f_n$ is dominated by $f_1\in L^1$, so it follows immediately from the
dominated convergence theorem that
\[
\lim_{n\to\infty}\int_{\Omega}f_nd\mu = \int_{\Omega}fd\mu
\]
as desired.

If one drops the condition that $f_1\in L^1$, then the sequence
$f_n = \frac{1}{n}$ does not satisfy this theorem.

In particular,
\[
\int_{\R}f_nd\mu = \frac{1}{n}\mu(\R)=\infty
\]
but
\[
f_n(x)\to0\ \forall x
\]
and thus
\[
\int_{\R}\lim_{n\to\infty} f_nd\mu = \int_{\R}0d\mu = 0
\]
which does not agree with $\lim_{n\to\infty}\int_{\R}f_nd\mu=\infty$.
\end{proof}
%----------------------------------------------------------------------------------------
%----------------------------------------------------------------------------------------
%	PROBLEM 7
%----------------------------------------------------------------------------------------

%----------------------------------------------------------------------------------------

\end{document}
