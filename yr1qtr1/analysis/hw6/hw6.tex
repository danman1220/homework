%%%%%%%%%%%%%%%%%%%%%%%%%%%%%%%%%%%%%%%%%
% Short Sectioned Assignment
% LaTeX Template
% Version 1.0 (5/5/12)
%
% This template has been downloaded from:
% http://www.LaTeXTemplates.com
%
% Original author:
% Frits Wenneker (http://www.howtotex.com)
%
% License:
% CC BY-NC-SA 3.0 (http://creativecommons.org/licenses/by-nc-sa/3.0/)
%
%%%%%%%%%%%%%%%%%%%%%%%%%%%%%%%%%%%%%%%%%

%----------------------------------------------------------------------------------------
%	PACKAGES AND OTHER DOCUMENT CONFIGURATIONS
%----------------------------------------------------------------------------------------

\documentclass[fontsize=11pt]{scrartcl} % 11pt font size

\usepackage[T1]{fontenc} % Use 8-bit encoding that has 256 glyphs
\usepackage[english]{babel} % English language/hyphenation
\usepackage{amsmath,amsfonts,amsthm} % Math packages
\usepackage{mathrsfs}

\usepackage[margin=1in]{geometry}

\usepackage{sectsty} % Allows customizing section commands
\allsectionsfont{\centering \normalfont\scshape} % Make all sections centered, the default font and small caps

\usepackage{fancyhdr} % Custom headers and footers
\pagestyle{fancyplain} % Makes all pages in the document conform to the custom headers and footers
\fancyhead{} % No page header - if you want one, create it in the same way as the footers below
\fancyfoot[L]{} % Empty left footer
\fancyfoot[C]{} % Empty center footer
\fancyfoot[R]{\thepage} % Page numbering for right footer
\renewcommand{\headrulewidth}{0pt} % Remove header underlines
\renewcommand{\footrulewidth}{0pt} % Remove footer underlines
\setlength{\headheight}{13.6pt} % Customize the height of the header

\numberwithin{equation}{section} % Number equations within sections (i.e. 1.1, 1.2, 2.1, 2.2 instead of 1, 2, 3, 4)
\numberwithin{figure}{section} % Number figures within sections (i.e. 1.1, 1.2, 2.1, 2.2 instead of 1, 2, 3, 4)
\numberwithin{table}{section} % Number tables within sections (i.e. 1.1, 1.2, 2.1, 2.2 instead of 1, 2, 3, 4)

\newcommand{\R}{\mathbb{R}}
\newcommand{\Q}{\mathbb{Q}}
\newcommand{\N}{\mathbb{N}}
\newcommand{\C}{\mathbb{C}}

\newtheorem{lemma}{Lemma}
%----------------------------------------------------------------------------------------
%	TITLE SECTION
%----------------------------------------------------------------------------------------

\newcommand{\horrule}[1]{\rule{\linewidth}{#1}} % Create horizontal rule command with 1 argument of height

\title{	
\normalfont \normalsize 
\textsc{Analysis} \\ [25pt] % Your university, school and/or department name(s)
\horrule{0.5pt} \\[0.4cm] % Thin top horizontal rule
\huge Problem Set 6\\ % The assignment title
\horrule{2pt} \\[0.5cm] % Thick bottom horizontal rule
}

\author{Daniel Halmrast} % Your name

\date{\normalsize\today} % Today's date or a custom date

\begin{document}

\maketitle % Print the title

%----------------------------------------------------------------------------------------
%	PROBLEM 1
%----------------------------------------------------------------------------------------
\section*{Problem 1}
Prove that in a normed space, a sequence can have at least one strong limit.
Prove that a strongly convergent sequence is Cauchy.
\\
\begin{proof}
    To begin with, we note that every normed space is necessarily Hausdorff,
    and that every Hausdorff space has the property that sequences have at most
    one limit. 

    To see that a normed space is Hausdorff, consider an arbitrary normed space
    $X$, and two distinct points $x,y$. Now, by the definition of a norm, it
    must be that $||x-y||>0$. So, let
    \[
        ||x-y||=\varepsilon
    \]
    Then, the open sets $B(x,\frac{\varepsilon}{2})$ and
    $B(Y,\frac{\varepsilon}{2})$ separate $x$ and $y$. Thus, normed spaces are
    Hausdorff.

    Furthermore, it is clear that a sequence converges to at most one limit in
    Hausdorff spaces. To see this, we recall the topological definition of
    convergence, which states that a sequence $(x_n)$ converges to $x$ if and
    only if every neighborhood of $x$ eventually contains the sequence.

    Suppose for a contradiction that $(x_n)$ had two limits, $x$, and $y$.
    Since $x$ and $y$ are distinct points in a Hausdorff space, there must be
    some neighborhood $V_x$ of $x$ and $V_y$ of $y$ such that $V_x\cap
    V_y=\emptyset$.

    However, this contradicts the sequence converging to both $x$ and $y$, since
    for $(x_n)$ to converge to $x$, it must eventually be in $V_x$, which means
    it is eventually outside $V_y$, and thus cannot converge to $y$ as well.

    Now, let $(x_n)$ be a convergent sequence in a normed space $X$, and let $x$
    be its limit. We will show that this sequence is Cauchy.

    To do this, let $\varepsilon>0$ be arbitrary. Now, we know that there exists
    some $N$ such that for any $n>N$,
    \[
        ||x_n - x||<\frac{\varepsilon}{2}
    \]
    
    Furthermore, for any $m,n>N$ we have that
    \[
        \begin{aligned}
            ||x_m-x_n|| &= ||x_m-x+x-x_n||\\
                        &\leq ||x_m-x|| + ||x_n-x||\\
                        &\leq \frac{\varepsilon}{2}+\frac{\varepsilon}{2}\\
                        &=\varepsilon
        \end{aligned}
    \]
    and thus the sequence is Cauchy.
\end{proof}
%----------------------------------------------------------------------------------------
\newpage
%----------------------------------------------------------------------------------------
%	PROBLEM 2
%----------------------------------------------------------------------------------------
\section*{Problem 2}
Show that the closure of the ball $B(a,r)$ is the closed ball $\overline{B}(a,r)=\{x\ |\ |x-a|\leq r\}$
\\
\begin{proof}
Suppose $x$ is in $\overline{B}(a,r)$. Then, consider the sequence
\[
(x_n) = (x-a)(1-\frac{1}{n}) + a
\]
Clearly, this sequence converges to $x$, and each term is in $B(a,r)$.
To see this, we observe that
\[
\begin{aligned}
||x_n-a|| &= ||(x-a)(1-\frac{1}{n}) + a - a||\\
          &= ||x-a||(1-\frac{1}{n}\\
            &\leq r(1-\frac{1}{n}\\
            &\leq r
\end{aligned}
\]
Thus, each point in $\overline{B}(a,r)$ is a limit point of $B(a,r)$, and since
$\overline{B}(a,r)$ is closed, it follows that it is the closure of $B(a,r)$ (since the 
closure of $B(a,r)$ is the smallest closed set containing it.)
\end{proof}

%----------------------------------------------------------------------------------------
\newpage
%----------------------------------------------------------------------------------------
%	PROBLEM 3
%----------------------------------------------------------------------------------------
\section*{Problem 3}
Prove that for a linear operator $A$ between normed spaces $V,W$, the following
are equivalent:
\begin{enumerate}
    \item $A$ is continuous at every $p\in V$.
    \item $A$ is continuous at $0_V$.
    \item $A$ is bounded in the sense that 
        \[
            \sup_{||x||=1} ||Ax|| < \infty
        \]
    \item $A$ is bounded in the sense that for some $M>0$,
        \[
            ||Ax||\leq M||x||
        \]
        for all $x\in V$.
\end{enumerate}
Furthermore, prove that the set $\mathscr{B}(V,W)$ of the bounded linear
operators from $V$ to $W$ is a normed space.
\\
\begin{proof}
    ($1\implies 2$)
    This follows immediately from the statement of $1$.
    \\
    \\
    ($2\implies 3$)
    Suppose $A$ is continuous at zero. Then, choose $\varepsilon=1$. We must
    have some $\delta>0$ such that $||x||\leq\delta$ implies $||Ax||<\varepsilon=1$.
    Then, we have
    \[
        ||Ax|| = ||A\left(\frac{||x||}{\delta}\frac{x\delta}{||x||}\right)||
                = \frac{||x||}{\delta}||A\left(\frac{x\delta}{||x||}\right)||
                \leq \frac{||x||}{\delta}(1)
    \]
    and thus $A$ is bounded by $\frac{1}{\delta}$. In particular,
    \[
        \sup \frac{||Ax||}{||x||} \leq \frac{\frac{||x||}{\delta}}{||x||} =
        \frac{1}{\delta} <\infty
    \]
    as desired.
    \\
    \\  
    ($3\implies 4$)
    Suppose $A$ is bounded in the sense of statement $3$. In particular, let
    \[
        \sup_{x\neq 0} \frac{||Ax||}{||x||} = M
    \]
    for some positive $M$.
    Then, for all $x\in V$ with $x\neq 0$,
    \[
        \begin{aligned}
        \frac{||Ax||}{||x||} &\leq M\\
            \implies ||Ax|| &\leq M||x||
        \end{aligned}
    \]
    as desired. Note that if $x=0$, then $Ax=0$ as well and the statement is
    vacuously true.
    \\
    \\
    ($4\implies 1$)
    Suppose $A$ is bounded in the sense that there is some $M>0$ such that
    \[
        ||Ax||\leq M||x||
    \]
    for all $x$ in $V$. Now, let $\varepsilon >0$ be arbitrary. Then,
    the bound $\delta = \frac{\varepsilon}{M}$ on $||x-p||$ forces
    \[
        \begin{aligned}
            ||Ax-Ap|| = ||A(x-p)|| &\leq M||x-p||\\
                        &\leq M\frac{\varepsilon}{M}= \varepsilon
        \end{aligned}
    \]
    Thus, $A$ is continuous at $p$.
    \\
    \\
    Finally, we will prove that the space $\mathscr{B}(V,W)$ is a normed vector
    space under the operator norm $||A|| = \sup_{||x||=1}||Ax||$.

    To see this, we need to check that the norm is positive definite and
    satisfies the triangle inequality.

    The norm is clearly positive definite, since it is taken as the sup of a set
    of nonnegative numbers, and if $||A||=0$, then (by the
    definition of operator norm from statement 4)
    \[
        ||Ax||\leq (0)||x||\ \forall x
    \]
    which forces $||Ax||=0$ for all $x\in V$. Since the vector norm is positive
    definite, it follows that $Ax=0$ for all $x\in V$, and thus $A=0$.

    Now, we need to show that
    \[
        ||A+B|| \leq ||A||+||B||
    \]

    To see this, let $||A||=M_A$ and $||B||=M_B$. Then, we have that
    \[
        \begin{aligned}
            ||(A+B)(x)|| &= ||Ax+Bx||\\
                        &\leq ||Ax||+||Bx||\\
        \end{aligned}
    \]
    Since this holds for all $x$, it follows that
    \[
        ||A+B||\leq ||A||+||B||
    \]
    as desired. Thus, $\mathscr{B}(V,W)$ is a normed vector space.
\end{proof}

%----------------------------------------------------------------------------------------
\newpage
%----------------------------------------------------------------------------------------
%	PROBLEM 4
%----------------------------------------------------------------------------------------
\section*{Problem 4}
Suppose that $V,W$ are normed vector spaces, and that $W$ is Banach. Prove that
$\mathscr{B}(V,W)$ is Banach as well.
\\
\begin{proof}
    To begin with, let $(A_n)$ be a Cauchy sequence in $\mathscr{B}(V,W)$.
    In particular, for any $\varepsilon >0$, we can find an $N>0$ such that
    for all $n,m>N$,
    \[
        ||A_n-A_m|| < \varepsilon
    \]

    So, let $x\in V$, and consider the sequence $(A_nx)$. We will show this
    sequence is Cauchy, and by completeness of $W$, has a limit.

    To see how this sequence is Cauchy, let $\varepsilon >0$, and choose an
    $N$ such that for all $n,m>N$, $||A_n-A_m||<\frac{\varepsilon}{||x||}$.
    Then, we have
    \[
        \begin{aligned}
            ||(A_n-A_m)(x)|| &=||A_nx-A_mx||\\
                            &< \frac{\varepsilon}{||x||}||x||\\
                            &=\varepsilon
        \end{aligned}
    \]
    Thus, the sequence is Cauchy, and has a limit. Since this can be done
    for every $x\in V$, we can define $Ax = \lim_{n\to\infty}A_nx$.

    Clearly, $A$ is linear, since the limit respects scalar multiplication and
    addition. Furthermore, it can be seen that $A$ is bounded. To see this, we
    observe that
    \[
        \begin{aligned}
            ||Ax|| &= ||\lim_{n\to\infty}A_nx||\\
                    &=\lim_{n\to\infty}||A_nx||\\
                    &\leq\lim_{n\to\infty}||A_n||||x||
        \end{aligned}
    \]
    Now, $\lim_{n\to\infty}||A_n||$ is a sequence of real numbers, and can be
    seen to be Cauchy, since
    \[
    |||A_n||-||A_m||| \leq ||A_n-A_m||
    \]
    and since $A_n$ is Cauchy in the operator norm, $||A_n-A_m||$ can be
    bounded, and thus $||A_n||$ is a Cauchy sequence of real numbers, which
    converges to some $K>0$.

    So, we have
    \[
        \begin{aligned}
            ||Ax|| &\leq \lim_{n\to\infty}||A_n||||x||\\
                    &= K||x||
        \end{aligned}
    \]
    and thus $A$ is bounded.

    Now, all we have to show is that $\lim A_n = A$. To do this,
    we fix $\varepsilon > 0$, and consider
    \[
        \begin{aligned}
            ||A-A_n|| &= \sup_{||x||=1} ||(A-A_n)x||\\
        \end{aligned}
    \]
    Now, we will show that $||(A-A_n)x||$ is bounded above by $\varepsilon$ when
    $n$ is sufficiently large. To see this, let $\varepsilon > 0$, and choose
    $N$ large enough so that $\forall n,m>N$, $||A_m-A_n||<\frac{\varepsilon}{2}$.
    Now, we have 
    \[
        \begin{aligned}
            ||(A-A_n)x|| &\leq ||(A-A_m)x|| + ||(A_m-A_n)x||\\
                        &\leq \frac{\varepsilon}{2} + \frac{\varepsilon}{2}\\
                        &=\varepsilon
        \end{aligned}
    \]
    where $m$ was chosen via pointwise convergence to make
    $||(A-A_m)x||<\frac{\varepsilon}{2}$.

    Thus, $\lim_{n\to\infty}A_n = A$ which is in $\mathscr{B}(V,W)$.
\end{proof}
%----------------------------------------------------------------------------------------
\newpage
%----------------------------------------------------------------------------------------
%	PROBLEM 5
%----------------------------------------------------------------------------------------
\section*{Problem 5}
Show that a subspace $W$ of a Banach space $V$ is Banach if and only if its
closed.
\\
\begin{proof}
    ($\implies$)
    Suppose $W$ is a Banach subspace of $V$. In particular, this means that $W$
    contains its limits, since any convergent sequence in $W$ is Cauchy, and
    every Cauchy sequence in $W$ converges in $W$. Then, it follows immediately
    that $W$ is closed.
    \\
    \\
    ($\impliedby$)
    Suppose $W$ is a closed subspace of $V$. In particular, this means that
    every convergent sequence in $W$ converges in $W$. So, for any Cauchy
    sequence in $W$, that sequence converges (since $V$ is Banach), and since
    $W$ is closed, it must converge in $W$. Thus, $W$ is a Banach space.
\end{proof}
%----------------------------------------------------------------------------------------
\newpage
%----------------------------------------------------------------------------------------
%	PROBLEM 6
%----------------------------------------------------------------------------------------
\section*{Problem 6}
Prove that $\ell^1,\ell^{\infty},c_0$ are all Banach spaces.
\\
\begin{proof}
    We know by Riesz-Fischer that any $L^p$ space is a Banach space. Thus,
    $\ell^1=L^1(\N,\mu_c)$ and $\ell^{\infty} = L^{\infty}(\N,\mu_c)$ are
    Banach spaces.

    Now, $c_0 \subset \ell^{\infty}$, so if we show that $c_0$ is closed in
    $\ell^{\infty}$, then $c_0$ is Banach.

    So, let $(f_n)$ be a sequence of points in $c_0$ that converge in
    $\ell^{\infty}$, with $f = \lim f_n$.

    In particular, we can choose $N_1>0$ such that for all $i>N_1$, $\sup
    |f_i(n) - f(n)| < \frac{\varepsilon}{2}$. Furthermore, we can choose $N_2>0$
    such that for all $m>N_2$, $|f_{N_1}(m)|<\frac{\varepsilon}{2}$. Then
    \[
        \begin{aligned}
            |f(m)| &= |f(m)-f_{N_1}(m)+f_{N_1}(m)|\\
                    &\leq |f(m) - f_{N_1}(m)| + |f_{N_1}(m)|\\
                    &\leq \frac{\varepsilon}{2}+\frac{\varepsilon}{2}\\
                    &=\varepsilon
        \end{aligned}
    \]
    Thus, $f$ decays to zero, and therefore $f\in c_0$ as desired.
\end{proof}
%----------------------------------------------------------------------------------------
\newpage
%----------------------------------------------------------------------------------------
%	PROBLEM 7
%----------------------------------------------------------------------------------------
\section*{Problem 7}
For $K$ a compact subset of $\R^n$, prove that $C(K)$ under the $\sup$ norm is a
Banach space.
\\
\begin{proof}
    We note first that the sum of two $C(K)$ functions is again $C(K)$, and that
    the scalar multiple of a $C(K)$ function is still $C(K)$. This follows
    almost immediately from basic undergraduate Real analysis arguments.

    Since $C(K)$ is a subset of $L^{\infty}(K)$ (continuous functions on a
    compact domain are bounded), it follows that the $\sup$ norm, which
    coincides with the $L^{\infty}$ norm, is a norm on $C(K)$.

    All that is left to show is completeness. For this, we will simply show that
    the $L^{\infty}$ limit of $C(K)$ functions is again a $C(K)$ function.

    This is clear, however, since (by basic undergraduate Real analysis) the
    uniform limit of continuous functions is continuous. Since the $\sup$ norm
    defines uniform convergence, if a sequence $f_n$ of $C(K)$ functions
    converges in $L^{\infty}$ to $f$, it follows that $f_n$ converges uniformly
    to $f$, and thus if $f_n\in C(K)$, then $f\in C(K)$ as well.

    Thus, $C(K)$ contains its limits, and is a closed subspace of
    $L^{\infty}(K)$. So, $C(K)$ is Banach.
\end{proof}
%----------------------------------------------------------------------------------------
\newpage
%----------------------------------------------------------------------------------------
%	PROBLEM 8
%----------------------------------------------------------------------------------------
\section*{Problem 8}
Let $S$ be the set of all simple functions in the $L^1$ norm $||\cdot||_1$.
Prove that $S$ is a normed space. Prove that $\overline{S}=L^1$. Prove that for
$\R^n$ with the Lebesgue measure, $S$ is not complete. Give an example of a
measure space for which $S$ is complete.
\\
\begin{proof}
We first show that $S$ is a normed space. Since the $L^1$ norm already satisfies
    the axioms of a norm, it suffices to show that $S$ is a vector subspace of
    $L^1$. This is clear, since scaling a simple function by a constant yields
    another simple function, and addition of two simple functions is again a
    simple function.

    Now, we wish to show that $\overline{S}=L^1$. To see this, we will show that
    every $f\in L^1$ is the limit of simple functions with respect to the $L^1$
    norm.

    So, let $f\in L^1$, and let $f^+, f^-$ be its positive and negative parts.
    Now, a basic theorem tells us that $f^{\pm} = \lim \phi^{\pm}_n$ for a sequence of
    monotonically increasing simple functions $\phi_n$, and furthermore
    \[
        \lim_{n\to\infty}\int\phi^{\pm}_n = \int f^{\pm}
    \]
    Thus, we have that
    \[
        \begin{aligned}
            f &= f^+ - f^-\\
                &=\lim_{n\to\infty}(\phi^+_n - \phi^-_n)
        \end{aligned}
    \]
    where the limit is taken pointwise. So, let $\phi_n = \phi^+-\phi^-$. Then
    consider
    \[
        \begin{aligned}
            \int(|f-\phi_n|) &= \int(|f^+-f^-+\phi^+-\phi^-|)\\
                            &\leq \int |f^+ - \phi^+_n| -\int |f^- - \phi^-_n|\\
                            &= \int (f^+-\phi^+_n) - \int (f^- - \phi^-_n)\\
                            &\to 0-0 = 0
        \end{aligned}
    \]
    where the second to last line was obtained by observing that $f^{\pm}\geq
    \phi^{\pm}_n\ \forall n$, and the last line was obtained by the monotone
    convergence theorem on the monotonic sequences $\phi^{\pm}_n$.
    
    Thus, $||f-\phi_n||_1\to 0$, and thus $f$ is the limit of simple functions.
    Therefore, each $f\in L^1$ is the limit of simple functions, and since $L^1$
    is complete, it follows that $\overline{S}=L^1$.
    \\
    \\
    However, for $L^1(\R^n)$, $S\neq L^1$, since the function $f(x) =
    \exp(-|x|^2)$ (for $x\in\R^n$)is in $L^1$, but is not a simple function. Thus,
    $\overline{S}\neq S$, and $S$ is not complete.
    \\
    \\
    For an example of a measure space for which $S=L^1$, consider the one-point
    space $\Omega = \{\bullet\}$ with the counting measure $\mu_c$. Every
    function from $\Omega$ to $\C$ is just a choice of scalar in $\C$, and is a
    simple function. Thus, $L^1(\Omega,\mu_c)\subset \textrm{Hom}(\Omega,C) =
    S$, and thus $S=L^1$, and $S$ is complete.
\end{proof}
%----------------------------------------------------------------------------------------
\newpage
%----------------------------------------------------------------------------------------
%	PROBLEM 9
%----------------------------------------------------------------------------------------
\section*{Problem 9}
Find the completion of $C_0(\R)$.
\\
\begin{proof}
    We will show that 
    \[
        \overline{C_0(\R)} = S \stackrel{\textrm{def}}{=} \{f\in C(\R)\ |\ \lim_{|x|\to\infty}f(x) = 0\}
    \]

    To do so, we show that every $f\in S$ is the limit of functions $f_n\in
    C_0$. So, let $f\in S$, and consider the sequence of functions
    \[
        f_n(x) = 
        \begin{cases}
            f(x), &\textrm{if }x\in[-n,n]\\
            f(n)(n+1-x), &\textrm{if }x\in[n,n+1]\\
            f(-n)(x+n+1), &\textrm{ if}x\in[-n-1,-n]\\
            0, &\textrm{else}
        \end{cases}
    \]
    which agree with $f$ on the interval $[-n,n]$, and on $[n,n+1]$ and
    $[-n-1,n]$ decrease linearly to zero.

    Clearly, each $f_n\in C_0$, since their support is contained in
    $[-n-1,n+1]$. Thus, it suffices to show that $f_n(x)$ converges to $f(x)$
    uniformly.

    So, let $\varepsilon >0$, and choose $n$ such that $|f(x)|
    <\frac{\varepsilon}{2}$ for
    all $|x| > n$.

    Now, consider 
    \[
        ||f-f_n||_{\infty} = \sup_x |f_n(x)-f(x)|
    \]
    In particular, inside $[-n,n]$, the functions agree and $|f_n(x)-f(x)|=0$.
    Outside $[-n,n]$ we know that $|f_n(x)| \leq \max(|f(n)|,|f(-n)|)$ since
    $f_n$ decays monotonically to zero from $f(n)$ and $f(-n)$. By our choice of
    $n$, we have for $x\not\in[-n,n]$, $|f_n(x)|\leq \frac{\varepsilon}{2}$.
    To conclude, we have
    \[
    \begin{aligned}
        |f_n(x) - f(x)| &\leq |f_n(x)|+|f(x)|\\
                        &\leq \frac{\varepsilon}{2} + \frac{\varepsilon}{2}
    \end{aligned}
    \]
    Thus, $\sup_x |f_n(x) - f(x)|$ is bounded by $\varepsilon$, and must decay
    to zero. So, $f_n$ converges to $f$.

    So, we know that $S\subset \overline{C_0}$. Now, we must show that $S$
    itself is complete.

    To see this, we observe the basic fact that the uniform limit of functions
    that decay to zero at infinity is also a function that decays to zero at
    infinity.

    So, let $f_n$ be a Cauchy sequence of functions in $S$ that converge to some
    function $f$. We just need to show that $f\in S$. This is clear, however,
    since 
    \[
        \begin{aligned}
            |f(x)| &= |f(x)-f_n(x)+f_n(x)|\\
                    &\leq |f(x)-f_n(x)| + |f_n(x)|
        \end{aligned}
    \]
    Which goes to zero as $|x|\to\infty$ since $|f(x)-f_n(x)|\to 0$ as $n\to\infty$, and
    $|f_n(x)|\to 0 $ as $|x|\to\infty$.

    Thus $S$ is complete, and so $S$ is the completion of $C_0$. 
\end{proof}
%----------------------------------------------------------------------------------------
\newpage
%----------------------------------------------------------------------------------------
%	PROBLEM 10
%----------------------------------------------------------------------------------------
\section*{Problem 10}
Show that $\ell^1$ is not complete in the $\ell^{\infty}$ norm.
\\
\begin{proof}
We will show that the sequence $(x_m)_k = (\frac{1}{m^{1+\frac{1}{k}}})$ converges to
the function $x_m = \frac{1}{m}$, which is not in $\ell^1$, even though each term in the
sequence is in $\ell^1$.

To show convergence, we wish to show that the functions
\[
f_k(n) = \frac{1}{n^{1+\frac{1}{k}}} 
\]
converges uniformly to $f(n)=\frac{1}{n}$.

To do so, we will consider instead the extended functions
\[
f_k(x) = x^{1+\frac{1}{k}},\ x\in[0,1]
\]
which clearly converges pointwise to $f(x) = x$. Now, since $f$ is defined on a compact
domain, it must also uniformly converge to $f(x) = x$. Thus, the restriction $f_k(x)|_{\{\frac{1}{n}\}}$
also converges uniformly to $f(n)=\frac{1}{n}$.

Thus,the sequence of sequences $(x_m)_k$ converges uniformly (in $\ell^{\infty}$) to $(x_m)$ as
desired. Furthermore, since each $(x_m)_k$ is in $\ell^1$, but $(x_m)$ is not, it follows that
$\ell^1$ is not complete in the $\ell^{\infty}$ norm.
\end{proof}

%----------------------------------------------------------------------------------------

\end{document}
