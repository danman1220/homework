%%%%%%%%%%%%%%%%%%%%%%%%%%%%%%%%%%%%%%%%%
% Short Sectioned Assignment
% LaTeX Template
% Version 1.0 (5/5/12)
%
% This template has been downloaded from:
% http://www.LaTeXTemplates.com
%
% Original author:
% Frits Wenneker (http://www.howtotex.com)
%
% License:
% CC BY-NC-SA 3.0 (http://creativecommons.org/licenses/by-nc-sa/3.0/)
%
%%%%%%%%%%%%%%%%%%%%%%%%%%%%%%%%%%%%%%%%%

%----------------------------------------------------------------------------------------
%	PACKAGES AND OTHER DOCUMENT CONFIGURATIONS
%----------------------------------------------------------------------------------------

\documentclass[fontsize=11pt]{scrartcl} % A4 paper and 11pt font size

\usepackage[T1]{fontenc} % Use 8-bit encoding that has 256 glyphs
\usepackage[english]{babel} % English language/hyphenation
\usepackage{amsmath,amsfonts,amsthm} % Math packages
\usepackage{mathrsfs}
\usepackage[margin=1in]{geometry}

\usepackage{sectsty} % Allows customizing section commands
\allsectionsfont{\centering \normalfont\scshape} % Make all sections centered, the default font and small caps

\usepackage{fancyhdr} % Custom headers and footers
\pagestyle{fancyplain} % Makes all pages in the document conform to the custom headers and footers
\fancyhead{} % No page header - if you want one, create it in the same way as the footers below
\fancyfoot[L]{} % Empty left footer
\fancyfoot[C]{} % Empty center footer
\fancyfoot[R]{\thepage} % Page numbering for right footer
\renewcommand{\headrulewidth}{0pt} % Remove header underlines
\renewcommand{\footrulewidth}{0pt} % Remove footer underlines
\setlength{\headheight}{13.6pt} % Customize the height of the header

\numberwithin{equation}{section} % Number equations within sections (i.e. 1.1, 1.2, 2.1, 2.2 instead of 1, 2, 3, 4)
\numberwithin{figure}{section} % Number figures within sections (i.e. 1.1, 1.2, 2.1, 2.2 instead of 1, 2, 3, 4)
\numberwithin{table}{section} % Number tables within sections (i.e. 1.1, 1.2, 2.1, 2.2 instead of 1, 2, 3, 4)


%----------------------------------------------------------------------------------------
%	TITLE SECTION
%----------------------------------------------------------------------------------------

\newcommand{\horrule}[1]{\rule{\linewidth}{#1}} % Create horizontal rule command with 1 argument of height

\title{	
\normalfont \normalsize 
\textsc{Topology} \\ [25pt] % Your university, school and/or department name(s)
\horrule{0.5pt} \\[0.4cm] % Thin top horizontal rule
\huge Problem Set 1 \\ % The assignment title
\horrule{2pt} \\[0.5cm] % Thick bottom horizontal rule
}

\author{Daniel Halmrast} % Your name

\date{\normalsize\today} % Today's date or a custom date

\begin{document}

\maketitle % Print the title

%----------------------------------------------------------------------------------------
%	PROBLEM 1
%----------------------------------------------------------------------------------------
\section*{Problem 1}
Prove that $U$ is open if and only if for all $x$ in $U$, there exists some neighborhood
$U_x$ such that $x\in U_x\subseteq U$.
\\
\begin{proof}
(=>)
Assume $U$ is open. If $U=\emptyset$, the proposition is vacuously true. So, suppose $U$ is
nonempty. Then, for each $x\in U$, the open set $U$ satisfies $x\in U\subseteq U$. Thus,
this half of the proof is complete.

(<=)
Now, assume that for each $x\in U$, there exists $U_x$ open such that $x\in U_x\subseteq U$.
Then, $U$ can be written as the union
\[
U = \bigcup_{x\in U} U_x
\]
This is easily seen by verifying that, since each $x$ is in some $U_x$, $U\subseteq \bigcup_x U_x$
and since each $U_x\subseteq U$, $\bigcup_x U_x \subseteq U$.

Thus, $U$ is the union of open sets, and is open.
\end{proof}

%----------------------------------------------------------------------------------------

%----------------------------------------------------------------------------------------
%	PROBLEM 2
%----------------------------------------------------------------------------------------
\section*{Problem 2}
Prove that, for a product space $X\times Y$, the projection $\pi_x:X\times Y\to X$ is
an open map.
\\
\begin{proof}
Let $U = \bigcup_{\alpha\in I} V_{\alpha}\times W_{\alpha}$ be an arbitrary open set
in $X\times Y$, with $V_{\alpha}$ open in $X$, and $W_{\alpha}$ open in $Y$.

Then, 
\[
\begin{aligned}
\pi_x(U) &= \pi_x(\cup_{\alpha}V_{\alpha}\times W_{\alpha}\\
        &= \cup_{\alpha}\pi_x(V_{\alpha}\times W_{\alpha})\\
        &= \cup_{\alpha}V_{\alpha}
\end{aligned}
\]
which is the union of open sets, and is open in $X$.
\end{proof}


%----------------------------------------------------------------------------------------
%	PROBLEM 3
%----------------------------------------------------------------------------------------
\section*{Problem 3}
Show that, for $A\subset X$ and $B\subset Y$, the identity $\overline{A\times B} = \overline{A}\times\overline{B}$
holds.
\\
\begin{proof}
($\subseteq$)
Let $(a,b)\in\overline{A\times B}$ be an element of the closure. Then, by the definition of
closure, there exists some net $(a_{\alpha},b_{\alpha})$ that converges to $(a,b)$ with each
$(a_{\alpha},b_{\alpha})\in A\times B$. Now, since the projection maps $\pi_x$ and $\pi_y$
are continuous, they preserve nets. Thus, we have
\[
\begin{aligned}
\pi_x(a_{\alpha},b_{\alpha})&\to\pi_x(a,b)\\
a_{\alpha}&\to a
\end{aligned}
\]
and since each $a_{\alpha}\in A$, $a\in \overline{A}$. Similarly in the second coordinate, we have
$b\in \overline{B}$. Thus, $(a,b)\in\overline{A}\times\overline{B}$.

($\supseteq$) 
Let $(a,b)\in \overline{A}\times\overline{B}$. In particular, this means that $a\in\overline{A}$ and
$b\in\overline{B}$. Then, by definition of closure, there exists a net $a_{\alpha}\to a$ and $b_{\beta}\to b$
such that for all $\alpha$, $a_{\alpha}\in A$ and for all $\beta$, $b_{\beta}\in B$.
Then, the net $(a_{\alpha},b_{\beta})\to (a,b)$ is such that each $(a_{\alpha},b_{\beta})\in A\times B$,
so $(a,b)\in\overline{A\times B}$.

(Note that the net $(a_{\alpha},b_{\beta})$ is still a net, if the indexing space is taken to be
the product of the indexing space of $\alpha$ and $\beta$ with the directed order 
\[
(\alpha_1,\beta_1) > (\alpha_2,\beta_2)\iff \alpha_1 > \alpha_2 \textrm{and } \beta_1 > \beta_2
\]
.)
\end{proof}

%----------------------------------------------------------------------------------------
%	PROBLEM 4
%----------------------------------------------------------------------------------------
\section*{Problem 4}
Prove that, for $A,B\subset X$, then $\overline{A\cup B} = \overline{A}\cup\overline{B}$.
\\
\begin{proof}
($\subseteq$)
Let $x\in \overline{A\cup B}$. Then, there is some net $x_{\alpha}\to x$ for $x_{\alpha}\in A\cup B$.
Now, this net is either frequently in $A$, or frequently in $B$ (or possibly both). Suppose
 $x_{\alpha}$ frequently be in $A$. Then, consider the subnet $x_{\alpha_{\beta}}$
of elements of $x_{\alpha}$ that are in $A$. This subnet also converges to $x$, so $x\in\overline{A}$.
Similarly, if $x_{\alpha}$ is frequently in $B$, then $x\in\overline{B}$. Thus, $x$ is either
in $\overline{A}$ or $\overline{B}$, that is, $x\in\overline{A}\cup\overline{B}$.

($\supseteq$)
Let $x\in\overline{A}\cup\overline{B}$. Suppose $x\in\overline{A}$. Then, there exists a net
$x_{\alpha}\to x$ such that $x_{\alpha}\in A$ for all $\alpha$. In particular, $x_{\alpha}\in A\cup B$
for all $\alpha$, so $x\in\overline{A\cup B}$, since it is the limit of a net in $A\cup B$. 
By exactly the same argument, if $x\in\overline{B}$, then $x\in\overline{A\cup B}$.
\end{proof}


%----------------------------------------------------------------------------------------
%	PROBLEM 5
%----------------------------------------------------------------------------------------
\section*{Problem 5}
Show that for a collection $A_{\alpha}$ of subsets of $X$, then 
$\bigcup_{\alpha}\overline{A_{\alpha}}\subseteq\overline{\bigcup_{\alpha}A_{\alpha}}$,
but equality does not necessarily hold.
\\
\begin{proof}
First, let $x\in\bigcup_{\alpha}\overline{A_{\alpha}}$. In particular, $x$ is in some $\overline{A_{x}}$.
So, let $x_{\gamma}\to x$ be a net in $A_x$. Then, $x_{\gamma}\in \bigcup_{\alpha}A_{\alpha}$ for all $\gamma$. So,
since $x_{\gamma}\to x$, $x\in\overline{\bigcup_{\alpha}A_{\alpha}}$.
\\
\\
Now, for a counterexample to equality, consider the collection
\[
A_n = \{\frac{1}{n}\} \subset \mathbb{R}
\]
Now, $\overline{A_n} = A_n$ for each $n$, since they are all closed, so $\bigcup_n \overline{A_n} = \bigcup_n A_n$.
However, $0$ is a limit point of $\bigcup_n A_n$, so it is in the closure $\overline{\bigcup_n A_n}$. But $0$
is not in $\bigcup_n \overline{A_n}$. Thus, 
$\bigcup_{\alpha}\overline{A_{\alpha}}\neq\overline{\bigcup_{\alpha}A_{\alpha}}$,

\end{proof}


%----------------------------------------------------------------------------------------
%	PROBLEM 6
%----------------------------------------------------------------------------------------
\section*{Problem 6}
Show that the product $X\times Y$ of two Hausdorff spaces $X$ and $Y$ is Hausdorff.
\\
\begin{proof}
For this proof, we will use the unique limits definition of the Hausdorff condition.
That is, a space is Hausdorff if and only if convergent nets in the space have unique limits.

Suppose for a contradiction that $X\times Y$ is not Hausdorff. Then, there exists some
net $(x_{\alpha},y_{\alpha})$ that has two or more limits. Let $(x_1,y_1),(x_2,y_2)$ be two
such distinct limits.

Since these points are distinct, either $x_1\neq x_2$, or $y_1\neq y_2$. Suppose $x_1\neq x_2$.
Then, the projection map gives us
\[
\begin{aligned}
\pi_x(x_{\alpha},y_{\alpha})&\to(x_1,y_1)
\implies x_{\alpha}&\to x_1
\end{aligned}
\]
However, by a similar argument, it can be shown that $x_{\alpha}\to x_2$. Since $x_1 \neq x_2$,
the net $x_{\alpha}$ does not have unique limits in $X$, which is a contradiction to $X$ being Hausdorff.

If $x_1 = x_2$, then it must be that $y_1\neq y_2$, and by a similar argument to the one above,
one reaches a contradiction on $Y$ being Hausdorff.
\end{proof}

%----------------------------------------------------------------------------------------
%	PROBLEM 7
%----------------------------------------------------------------------------------------
\section*{Problem 7}
Show that $X$ is Hausdorff if and only if the diagonal in $X\times X$ is closed.
\\
\begin{proof}
For this proof, we will use the separation definition of the Hausdorff condition.

(=>)
Suppose $X$ is Hausdorff, and let $x,y\in X$ be distinct points. Now, let $U,V\subset X$ open
such that $x\in U$, $y\in V$, and $U\cap V = \emptyset$. Then, the open set
$U\times V\subset X\times X$ does not intersect the diagonal, but contains $(x,y)$.

Since this can be done for any point $(x,y)$ not on the diagonal, the complement of the diagonal
is open (by the result from problem 1) and thus the diagonal is closed.

(<=)
Suppose $X$ is such that the diagonal is closed. In particular, this means that the
complement to the diagonal is open. Let $x,y\in X$ with $x\neq y$. Then, the point
$(x,y)$ is not on the diagonal, and there must be some basic open set $U\times V$ containing
$(x,y)$ that does not intersect the diagonal. In particular, this means that $U$ and $V$
are disjoint, since if they shared an element $x_0\in U$ and $x_0\in V$, then $(x_0,x_0)\in U\times V$
which is a point on the diagonal contained in $U\times V$, a contradiction.

Since $(x,y)\in U\times V$, this means that $x\in U$ and $y\in V$ for disjoint open $U$ and $V$.

This separation can be done for any pair of points in $X$, so $X$ is Hausdorff.
\end{proof}


%----------------------------------------------------------------------------------------
%	PROBLEM 8
%----------------------------------------------------------------------------------------
\section*{Problem 8}
Define $\partial A = \overline{A}\cap\overline{A^c}$. Show that $\overline{A} = \textrm{Int}(A)\cup\partial A$.
\\
\begin{proof}
This proof will be done by simple equation chasing.
\[
\begin{aligned}
\textrm{Int}(A)\cup\partial A &= \textrm{Int}(A)\cup(\overline{A}\cap\overline{A^c})\\
        &=(\textrm{Int}(A) \cup \overline{A})\cap(\textrm{Int}(A) \cup\overline{A^c})\\
        &=\overline{A}\cap X\\
        &= \overline{A}
\end{aligned}
\]
Now, the second line is obtained by simple set theory logic.
The third line equality can be seen to be true by an easy argument. First, observe
that $\textrm{Int}(A) \subset \overline{A}$ since $\textrm{Int}(A)\subset A \subset\overline{A}$ by
definition. Thus, 
\[
\textrm{Int}(A)\cup\overline{A} = \overline{A}
\]
For the second half,
observe that 
\[
\textrm{Int}(A)^c = (\bigcup_{U\subset A} U)^c = \bigcap_{A^c\subset U^c}U^c = \overline{A^c}
\]
so that 
\[
\textrm{Int}(A)\cup\overline{A^c} = \overline{A^c}^c\cup\overline{A^c} = X
\]
The result follows immediately.
\end{proof}

\end{document}
