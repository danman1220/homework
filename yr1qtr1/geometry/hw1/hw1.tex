%%%%%%%%%%%%%%%%%%%%%%%%%%%%%%%%%%%%%%%%%
% Short Sectioned Assignment
% LaTeX Template
% Version 1.0 (5/5/12)
%
% This template has been downloaded from:
% http://www.LaTeXTemplates.com
%
% Original author:
% Frits Wenneker (http://www.howtotex.com)
%
% License:
% CC BY-NC-SA 3.0 (http://creativecommons.org/licenses/by-nc-sa/3.0/)
%
%%%%%%%%%%%%%%%%%%%%%%%%%%%%%%%%%%%%%%%%%

%----------------------------------------------------------------------------------------
%	PACKAGES AND OTHER DOCUMENT CONFIGURATIONS
%----------------------------------------------------------------------------------------

\documentclass[fontsize=11pt]{scrartcl} % A4 paper and 11pt font size

\usepackage[T1]{fontenc} % Use 8-bit encoding that has 256 glyphs
\usepackage[english]{babel} % English language/hyphenation
\usepackage{amsmath,amsfonts,amsthm} % Math packages
\usepackage{mathrsfs}

\usepackage{sectsty} % Allows customizing section commands
\allsectionsfont{\centering \normalfont\scshape} % Make all sections centered, the default font and small caps

\usepackage{fancyhdr} % Custom headers and footers
\pagestyle{fancyplain} % Makes all pages in the document conform to the custom headers and footers
\fancyhead{} % No page header - if you want one, create it in the same way as the footers below
\fancyfoot[L]{} % Empty left footer
\fancyfoot[C]{} % Empty center footer
\fancyfoot[R]{\thepage} % Page numbering for right footer
\renewcommand{\headrulewidth}{0pt} % Remove header underlines
\renewcommand{\footrulewidth}{0pt} % Remove footer underlines
\setlength{\headheight}{13.6pt} % Customize the height of the header

\numberwithin{equation}{section} % Number equations within sections (i.e. 1.1, 1.2, 2.1, 2.2 instead of 1, 2, 3, 4)
\numberwithin{figure}{section} % Number figures within sections (i.e. 1.1, 1.2, 2.1, 2.2 instead of 1, 2, 3, 4)
\numberwithin{table}{section} % Number tables within sections (i.e. 1.1, 1.2, 2.1, 2.2 instead of 1, 2, 3, 4)


%----------------------------------------------------------------------------------------
%	TITLE SECTION
%----------------------------------------------------------------------------------------

\newcommand{\horrule}[1]{\rule{\linewidth}{#1}} % Create horizontal rule command with 1 argument of height

\title{	
\normalfont \normalsize 
\textsc{Differential Geometry} \\ [25pt] % Your university, school and/or department name(s)
\horrule{0.5pt} \\[0.4cm] % Thin top horizontal rule
\huge Problem Set 1\\ % The assignment title
\horrule{2pt} \\[0.5cm] % Thick bottom horizontal rule
}

\author{Daniel Halmrast} % Your name

\date{\normalsize\today} % Today's date or a custom date

\begin{document}

\maketitle % Print the title

%----------------------------------------------------------------------------------------
%	PROBLEM 1
%----------------------------------------------------------------------------------------

\section*{Problem 1}
Show that the line with two origins is not Hausdorff, but is locally Euclidean and is
second countable.
\\
\begin{proof}
The line with two origins is defined as the quotient of $\mathbb{R}_a\coprod\mathbb{R}_b$ by
the equivalence relation $(x, a) \sim (x, b)$ for $x \neq 0$ (the notation $(x,a)$ indicates,
for example, the point $x$ on the line $\mathbb{R}_a$. For convenience, an interval will be
represented similarly as, for example, $((c,d),a)$ for the interval $(c,d)$ in $\mathbb{R}_a$).

To show this space is locally Euclidean, consider any point $[x]$ on the line with two
origins. if $[x] \neq 0_a, 0_b$, then $[x]$ has a neighborhood $[((x-\epsilon,
x+\epsilon),a)] \cong (x-\epsilon, x+\epsilon)$ which does not intersect the origins, and
is homeomorphic to the same interval in $\mathbb{R}$.  If $[x]$ is equal to $0_a$ or $0_b$
(without loss of generality, let $[x] = 0_a$), then the neighborhood $[(-1,1),a)] \cong
(-1,1)$ is homeomorphic to that interval in $\mathbb{R}$.  Thus, each point has a
neighborhood homeomorphic to an open interval in $\mathbb{R}$ and the line with two
origins is locally Euclidean.

To see this space is second countable, consider the canonical projection map $\pi :
\mathbb{R}\coprod\mathbb{R} \to \mathbb{R}\coprod\mathbb{R}/{\sim}$ given by $\pi(x) =
[x]$. This projection is an open map. To see this, consider a general basic open set
$((c,d),a)$ or $((c,d),b)$. For $(c,d)$ such that $0\not\in(c,d)$, the interval projects
to $(c,d)$, which is open. Now, without loss of generality, consider an interval
$((c,d),a)$ containing $0$. This projects to the set $(c,0)\cup\{0_a\}\cup(0,d)$. This set
can easily be seen to be open in the quotient by checking its inverse image is open. Thus,
since $\pi$ maps basic open sets to open sets, $\pi$ is an open map. Noting also that
$\pi$ is a surjection, we can apply Lemma 1 for quotients (defined in class) to show that
the quotient is second countable.

Finally, to see this space is not Hausdorff, consider the two origins $0_a$ and $0_b$.
Each open neighborhood of $0_a$ contains some interval
$(-\epsilon_a,0)\cup\{0_a\}\cup(0,\epsilon_a)$. Similarly, each neighborhood of $0_b$
contains some interval $(-\epsilon_b,0)\cup\{0_b\}\cup(0,\epsilon_b)$. These intervals
intersect, so the neighborhoods of $0_a$ and $0_b$ intersect as well.  Since this holds
true for all neighborhoods of the origins, the two origins are not seperable by open sets,
and the line with two origins is not Hausdorff.
\end{proof}

%----------------------------------------------------------------------------------------


%----------------------------------------------------------------------------------------
%	PROBLEM 2
%----------------------------------------------------------------------------------------

\section*{Problem 2}
Show that the coproduct of $\mathbb{R}_{\alpha}$ for $\alpha\in I$ with $I$ uncountable
is locally Euclidean and Hausdorff, but is not second countable.
\\
\begin{proof}
To see that this space is locally Euclidean, consider a point 
\[
x \in \coprod_{\alpha\in I}\mathbb{R}_{\alpha}
\]

Since $x$ is in the coproduct, it is in exactly one of the $\mathbb{R}_{\alpha}$.
Thus, the neighborhood $\mathbb{R}_{\alpha}$ of $x$ is homeomorphic to $\mathbb{R}$.
Since this holds for any $x$, the space is locally Euclidean.

Furthermore, consider two points $x,y$ distinct from each other in the coproduct.
In the case $x\in\mathbb{R}_{\alpha}$ and $y\in\mathbb{R}_{\beta}$ for $\alpha\neq\beta$,
the open sets $\mathbb{R}_{\alpha}$ and $\mathbb{R}_{\beta}$ separate $x$ and $y$.
If instead $x$ and $y$ are from the same copy $\mathbb{R}_{\alpha}$, then they are seperable
in the subspace $\mathbb{R}_{\alpha}$ (since $\mathbb{R}$ is Hausdorff) and are seperable
in the coproduct.

However, the coproduct is not second countable. Let $\mathscr{B}$ be a basis for the coproduct.
Then, consider the set of disjoint open sets
\[
    \{\mathbb{R}_\alpha | \alpha\in I\}
\]
By the definition of a basis, each $\mathbb{R}_{\alpha}$ must contain at least one basis
element. Furthermore, since each are disjoint from the rest, the basis element is unique.
Thus, for each $\alpha \in I$, $\mathscr{B}$ must have at least one distinct basis element
$B_{\alpha}\subset \mathbb{R}_{\alpha}$. Thus, $\mathscr{B}$ must have uncountably many
elements. Therefore, the coproduct does not admit a countable basis, and is not second
countable. 
\end{proof}

%----------------------------------------------------------------------------------------


%----------------------------------------------------------------------------------------
%	PROBLEM 3
%----------------------------------------------------------------------------------------
\section*{Problem 7}
Let $\sigma:S^n\setminus\{N\}\to \mathbb{R}^n$ be given as
\[
\sigma(x^1, x^2,\ldots,x^n) = \frac{(x^1,\ldots,x^n)}{1-x^{n+1}}
\]
And similarly define $\widetilde{\sigma}(x) = -\sigma(-x)$ for $x\in S^n\setminus\{S\}$.

\subsection*{part a}
Show that $\sigma(x)$ sends a point to the intersection of the $\mathbb{R}^n$ subspace
and the line from the north pole $N$ through the point $x$.
\\
\begin{proof}
Let $x\in S^n\setminus\{N\}$, and consider the line through $N$ and $x$ parameterized as
\[
f(t) = tx + (1-t)N
\]
This function intersects the $\mathbb{R}^n$ subspace when the $n+1$ coordinate is zero.
Now, the $n+1$ coordinate is parameterized as
\[
f^{n+1}(t) = tx^{n+1} + (1-t)(1)
\]
This is zero when $t=i=t_0=\frac{1}{1-x^{n+1}}$. At this point, the other $n$ coordinates of
$x$ are given by
\[
f^i(t) = tx^i + (1-t)(0)\\
       = tx^i
\]
so
\[
f^i(t_0) = \frac{x^i}{1-x^{n+1}}
\]

Thus, we have recovered $\sigma(x)$ as desired.

Following similar logic for the $\widetilde{\sigma}(x)$ map, we see that the line connecting
$S$ and $x$ is given by the equation
\[
f(t) = tx + (1-t)S
\]
Similarly, the point where the $n+1$ coordinate is zero in this equation is given by
\[
0=f^{n+1}(t)=tx^{n+1} + (1-t)(-1)
\]
which is zero when $t=t_0=\frac{1}{x+1}$. Thus, the point of intersection is
\[
f^i(t_0) = \frac{x^i}{1+x^{n+1}}
\]

Since $\sigma_i(x) = \frac{x^i}{1-x^{n+1}}$, 
\[
-\sigma^i(-x) = -\frac{-x^i}{1-(-x^{n+1})} = \frac{x^i}{1+x^{n+1}}
\]
which is just $\widetilde{\sigma}(x)$ as desired.

\end{proof}

\subsection*{Part b}
Show the map $\sigma$ is bijective and compute its inverse.
\\
\begin{proof}
To see that $\sigma$ is injective, observe that $\sigma$ was defined
qualitatively as the intersection of the target subspace $\mathbb{R}^n$
with the line uniquely determined by the points $x$ and $N$.
So, let $x_1$ and $x_2$ be elements of $S^n$ such that $\sigma(x_1) = \sigma(x_2)$.
Then, consider the line defined through the points $\sigma(x_1)$ and $N$.
This line crosses the punctured sphere $S^n\setminus\{N\}$ in exactly one location,
which must be $x_1$, since $\sigma(x_1)$ determined the line, but it also must be $x_2$
by a similar argument. Thus, $x_1$ = $x_2$.

Surjectivity will be proved by showing that the inverse map has no domain restrictions
on $\mathbb{R}^n$.

The inverse is given as
\[
\sigma^{-1}(u^1,u^2,\ldots,u^n) = \frac{(2u^1,2u^2,\ldots,2u^n,|u|^2-1)}{|u|^2+1}
\]
This is easily seen to be the inverse of $\sigma$, since
\[
\begin{aligned}
\sigma(\sigma^{-1}(u^1,\ldots,u^n)) &= \sigma( \frac{(2u^1,2u^2,\ldots,2u^n,|u|^2-1)}{|u|^2+1})\\
        &=\left(\frac{1}{|u|^2+1}\frac{1}{1-\frac{|u|^2+1}{|u|^2-1}}\right)(2u^1,\ldots,2u^n)\\
        &=(u^1,\ldots,u^n)
\end{aligned}
\]
Which has no domain restrictions, so $\sigma$ is surjective.

\end{proof}

\subsection*{Part c}
Compute the transition map, and verify it is $C^{\infty}$.
\\
\begin{proof}
The transition map $\widetilde{\sigma}\circ\sigma^{-1}$ can be seen to be
\[
\begin{aligned}
\widetilde{\sigma}\circ\sigma^{-1} &= \widetilde{\sigma}\left( \frac{(2u^1,2u^2,\ldots,2u^n,|u|^2-1)}{|u|^2+1}\right)\\
        &= -\sigma\left(\frac{(-2u^1,-2u^2,\ldots,-2u^n,-|u|^2+1)}{|u|^2+1}\right)\\
        &= -\frac{1}{1-\frac{-|u|^2+1}{|u|^2+1}}\left(-2u^1,\ldots,-2u^n\right)
\end{aligned}
\]
which is clearly smooth, since it is a quotient of smooth functions.
\end{proof}

\subsection*{Part d}
Show this structure is compatible with the projection structure defined in the book.
\\
\begin{proof}
To show this, consider the compatibility maps
\[
\phi_i\circ\sigma^{-1} = \frac{(2u^1,2u^2,\ldots\hat{u_i},\ldots,2u^n,|u|^2-1)}{|u|^2+1}
\]
which is clearly smooth for any $i$. The reverse $\sigma\circ\phi_i^{-1}$ is also clearly
smooth, so the charts are compatible with each other (Repeat the argument for the south
pole stereographic projection to attain full compatibility at all points, but the argument
is the same).
\end{proof}

\section*{Problem ``Batch 1 Extra"}
Show that the graph of $|x|$ with the subspace topology from $\mathbb{R}^2$ is a smooth manifold.
\\
\begin{proof}
Consider the projection mapping:
\[
\phi(x,|x|) = x
\]

This is clearly defined for all points in the graph, and is easily seen to be a homeomorphism 
of the graph with $\mathbb{R}$. Furthermore, since this is a global atlas, no transition maps
need to be calculated, and the manifold is trivially smooth.
\end{proof}

\section*{Batch 2 Problem 1}
Show that a manifold $M$ with a smooth structure has infinitely many such structures.
\\
\begin{proof}
Let $(U_{\alpha}, \phi_{\alpha})$ be an atlas of $M$ such that for all $U_{\alpha},U_{\beta}$
distinct pairs, the coordinate charts on the intersection $U_{\alpha}\cap U_{\beta}$ do not
contain the origin of $\mathbb{R}^n$, Let there also be some $U_0$ such that the coordinate
chart on $U_0$ does contain the origin. Furthermore, we require that each $U_{\alpha}$ map
by $\phi_{\alpha}$ into the unit ball $\mathbb{B}^n$.

With this construction in place, we construct for each $s>0$ a new differentiable structure
on $M$ by composing the coordinate charts with the homeomorphism
\[
F_s(x) = |x|^{s-1}x
\]
from $\mathbb{R}^n$ to itself. Clearly, this is a homeomorphism, but at the origin it fails
to be smooth in the standard differential structure on $\mathbb{R}^n$.

Now, consider the new atlas $(U_{\alpha},F_s\circ\phi_{\alpha})$. The transition map
between two charts is $F_s\circ\phi_{\beta}\circ\phi_{\alpha}^{-1}\circ F_s^{-1}$, which is
smooth away from the origin. Since the intersections were constructed to avoid the origin, the
transition map is smooth.

However, checking compatibility with some other $t\neq s$ will fail on $U_0$, since the compatibility map
$F_t\circ\phi_0\circ\phi_0^{-1}\circ F_s^{-1}$ which is not smooth on $U_0$.

Thus, for each $s>0$, we get a distinct differentiable structure on $M$ as desired.


\end{proof}

\section*{Batch 2 Problem 2}
Show that $\mathbb{C}P^n$ is a compact $2n$-manifold.
\\
\begin{proof}
Observe that $\mathbb{C}P^n$ can be constructed as the quotient $S^{2n+1}/\sim$
with $\sim$ identifying antipodal points (i.e. $x\sim y \iff x = \lambda y$ for some $\lambda\in\mathbb{C}$). 
Now, the quotient map from $S^{2n+1}$
(which is compact) to $\mathbb{C}P^n$ is continuous, so the projective space
$\mathbb{C}P^n$ is compact as well.

Now, let's push forward the differential structure from $S^{2n+1}$ along the quotient map $q$.
Since $q$ is a local homeomorphism, we can choose a collection of open sets $V_{\alpha}$ in $S^{2n+1}$
such that $q|_{V_{\alpha}}$ is a homeomorphism, $V_{\alpha}\subset U_{\alpha}$ for some $U_{\alpha}$ in the
chart $(U_{\alpha},\phi_{\alpha})$, and $q(V_{\alpha})$ covers $\mathbb{C}P^n$.
Then, the charts on $\mathbb{C}P^n$ is the collection $(q(V_{\alpha}),\phi_{\alpha}\circ q|_{V_{\alpha}}^{-1})$,
which is clearly a differential structure, as it inherits compatibility from the differential structure on $S^{2n+1}$.
\end{proof}

\section*{Batch 2 Problem 3}
Show the map $\widetilde{F}: \mathbb{R}P^n\to\mathbb{R}P^k$ given by $\widetilde{F}([x]) = [F(x)]$ for a
homogeneous map $F:\mathbb{R}^{n+1}\to\mathbb{R}^{k+1}$ is well-defined and smooth.
\\
\begin{proof}
To show well-definedness of $\widetilde{F}$, consider $[x] = [cx]$ for some nonzero $C$.
Now, $\widetilde{F}([x]) = [F(x)] = [c^dF(x)] = [F(cx)] = \widetilde{F}([cx])$ as desired.

Now, let's show that $\widetilde{F}$ is smooth. Now, the standard smooth structure on projective
space is given by $\phi_i([x^1,\ldots,x^{n+1}]) = \frac{1}{x^i}(x^1,\ldots,\hat{x^i},\ldots,x^{n+1})$.

The compatibility map $\phi_i\circ\widetilde{F}\circ\phi_i^{-1}$ is given as
\[
\phi_i([F(u^1,\ldots,u^{i-1},1,u^{i+1},\ldots,u^n)])
\]
which is the composition of smooth functions, and is easily seen to be smooth.
\end{proof}

\section*{Batch 2 Problem 4}
For $\mathbb{R}$ the standard real line, and $\widetilde{\mathbb{R}}$ the real line
with the structure given by the map $t\mapsto t^3$, $f:\mathbb{R}\to\mathbb{R}$ a smooth
function:
\subsection*{Part a}
Show that $f:\mathbb{R}\to\widetilde{\mathbb{R}}$ is smooth.
\\
\begin{proof}
To show $f$ is smooth, we consider the charts $\phi$ (the standard chart on $\mathbb{R}$ given
by the identity) and $\psi$ the cubic chart.
Now, we check that the compatibility map $\psi\circ f\circ\phi^{-1}$ is smooth.
However, this is just $\psi\circ f\circ\phi^{-1}(t) = (f(t))^3$, which, for $f$ smooth, is a
smooth function. Thus, $f$ is smooth from $\mathbb{R}\to\widetilde{\mathbb{R}}$.
\end{proof}

\subsection*{Part b}
Show that for $f$ such that $f^{(n)}(0) = 0$ for integer $n$ not a multiple of $3$,
the map $f:\widetilde{\mathbb{R}}\to\mathbb{R}$ is smooth.
\\
\begin{proof}
To show this, we must show that the compatibility map $\phi\circ f\circ\psi^{-1}$ is
a smooth map, where $\phi$ and $\psi$ are the same as in part a.

This map is
\[
F(x) = \phi\circ f\circ\psi^{-1}(x) = f(x^{\frac{1}{3}})
\]
for a smooth function $f$. Now, this is differentiable away from zero without issue,
so consider the derivatives at zero.
$F'(0) = f'(0)*\frac{1}{3}x^{\frac{-2}{3}}|_{x=0}$ which has problems at zero in the
second term. But, since $f'(0)=0$, the problems can be ignored. Similarly for higher
derivatives, terms with an $n^{th}$ derivative of $x^{\frac{1}{3}}$ for $n$ not a multiple
of three will have problems, but those terms are the ones where $f^{(n)}(0) = 0$.
So, the overall function $F$ is smooth.

Note that the arguments above show that this derivative condition characterizes the
functions that are smooth from $\widetilde{\mathbb{R}}\to\mathbb{R}$ as desired.
\end{proof}

\end{document}
