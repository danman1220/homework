%%%%%%%%%%%%%%%%%%%%%%%%%%%%%%%%%%%%%%%%%
% Short Sectioned Assignment
% LaTeX Template
% Version 1.0 (5/5/12)
%
% This template has been downloaded from:
% http://www.LaTeXTemplates.com
%
% Original author:
% Frits Wenneker (http://www.howtotex.com)
%
% License:
% CC BY-NC-SA 3.0 (http://creativecommons.org/licenses/by-nc-sa/3.0/)
%
%%%%%%%%%%%%%%%%%%%%%%%%%%%%%%%%%%%%%%%%%

%----------------------------------------------------------------------------------------
%	PACKAGES AND OTHER DOCUMENT CONFIGURATIONS
%----------------------------------------------------------------------------------------

\documentclass[fontsize=11pt]{scrartcl} % 11pt font size

\usepackage[T1]{fontenc} % Use 8-bit encoding that has 256 glyphs
\usepackage[english]{babel} % English language/hyphenation
\usepackage{amsmath,amsfonts,amsthm, amssymb} % Math packages
\usepackage{mathrsfs}
\usepackage{tikz-cd}

\usepackage[margin=1in]{geometry}

\usepackage{sectsty} % Allows customizing section commands
\allsectionsfont{\centering \normalfont\scshape} % Make all sections centered, the default font and small caps

\usepackage{fancyhdr} % Custom headers and footers
\pagestyle{fancyplain} % Makes all pages in the document conform to the custom headers and footers
\fancyhead{} % No page header - if you want one, create it in the same way as the footers below
\fancyfoot[L]{} % Empty left footer
\fancyfoot[C]{} % Empty center footer
\fancyfoot[R]{\thepage} % Page numbering for right footer
\renewcommand{\headrulewidth}{0pt} % Remove header underlines
\renewcommand{\footrulewidth}{0pt} % Remove footer underlines
\setlength{\headheight}{13.6pt} % Customize the height of the header

\numberwithin{equation}{section} % Number equations within sections (i.e. 1.1, 1.2, 2.1, 2.2 instead of 1, 2, 3, 4)
\numberwithin{figure}{section} % Number figures within sections (i.e. 1.1, 1.2, 2.1, 2.2 instead of 1, 2, 3, 4)
\numberwithin{table}{section} % Number tables within sections (i.e. 1.1, 1.2, 2.1, 2.2 instead of 1, 2, 3, 4)

\newcommand{\R}{\mathbb{R}}
\newcommand{\Q}{\mathbb{Q}}
\newcommand{\N}{\mathbb{N}}
\newcommand{\C}{\mathbb{C}}

\newtheorem*{lemma}{Lemma}
%----------------------------------------------------------------------------------------
%	TITLE SECTION
%----------------------------------------------------------------------------------------

\newcommand{\horrule}[1]{\rule{\linewidth}{#1}} % Create horizontal rule command with 1 argument of height

\title{	
\normalfont \normalsize 
\textsc{Geometry} \\ [25pt] % Your university, school and/or department name(s)
\horrule{0.5pt} \\[0.4cm] % Thin top horizontal rule
\huge Midterm\\ % The assignment title
\horrule{2pt} \\[0.5cm] % Thick bottom horizontal rule
}

\author{Daniel Halmrast} % Your name

\date{\normalsize\today} % Today's date or a custom date

\begin{document}

\maketitle % Print the title

%----------------------------------------------------------------------------------------
%	PROBLEM 1
%----------------------------------------------------------------------------------------
\section*{Problem 1}
\subsection*{Part a}
Use the standard charts on $S^n$ to calculate the matrix representation of $di:T_pS^n\to T_p\R^{n+1}$,
and show that $di$ is injective, and thus $i$ is an embedding.
\\
\begin{proof}
For this calculation, we will use the chart given by hemisphere projection. That is, the
domains for the charts will be the open sets $U_i^{\pm}=\{(x^1\ldots,x^{n+1})\ |\ x^i>0 (x^i<0\textrm{resp.})\}$
with maps
\[
\phi_i^{\pm}(x^1,\ldots,x^{n+1}) = (x^1,\ldots,\hat{x^i},\ldots,x^{n+1})
\]
Where a hat denotes omission of the variable.

Now, suppose $p\in U_i^+$ (without loss of generality, we take the positive hemisphere
of $x^i$, but the argument can be repeated exactly with the negative hemisphere as well.)
and let the coordinate representation of $p$ be
\[
\phi(p) = (x^1,\ldots,\hat{x^i},\ldots,x^{n+1})
\]
Then, the inclusion map looks like
\[
\begin{aligned}
i\circ\phi^{-1}((x^1,\ldots,\hat{x^i},\ldots,x^{n+1}) &=(y^1,\ldots,y^{n+1})\\
                        &=(x^1,\ldots,x^{i-1},\sqrt{1-x^ax_a},\ldots,x^{n+1})
\end{aligned}
\]
and the Jacobian $di$ can be calculated directly using the identity $di_j^k=\partial_j(y^k)$.
Which gives the matrix (for $j=1,\ldots,i-1,i+1,\ldots,n+1$ and $k=1,\ldots,n+1$)
\[
\partial_j(y^k) = \delta_j^k -\frac{1}{\sqrt{1-x^ax_a}}\delta^{ik} x_j
\]
Which is clearly injective, since the rows $k\neq i$ are the basis covectors for $\R^n$, and
thus $\partial_j(y^k)$ has rank $n$ as desired.
Note that the $i$ index is the (fixed) priveleged index relative to which the chart was taken,
and is not summed over.

Furthermore, since $S^n$ is compact, and the inclusion is injective, it follows that
$i$ defines an embedding of $S^n$ into $\R^{n+1}$.
\end{proof}

\subsection*{Part b}
Show that $T_pS^n$, when identified with $di(T_pS^n)$ is the subspace of $\R^{n+1}$ consisting
of all vectors perpendicular to the radial vector to $p$.
\\
\begin{proof}
This follows by direct calculation. To see this, let $v\in T_pS^n$. Then,
\[
\begin{aligned}
di(v) = \partial_jy^kv^j &= \delta_j^kv^j-\frac{1}{\sqrt{1-x^ax_a}}\delta^{ik}x_jv^j\\
                &= (1-\delta^{ik})v^k - \frac{x^av_a}{\sqrt{1-x^ax_a}}\delta^{ik}
\end{aligned}
\]

Recalling earlier that the embedding sends 
\[
p = (x^1,\ldots,\hat{x^i},\ldots,x^{n+1})
\]
to 
\[
(y^1,\ldots,y^{n+1}) = (x^1,\ldots,x^{i-1},\sqrt{1-x^ax_a},\ldots,x^{n+1})
\]
we can compute the inner product $g_{jk}v^jy^k$ directly.
\[
\begin{aligned}
g_{jk}v^jy^k &=v^kx_k + \delta^{ik}_{ij}v^jy_k\\
            &=v^kx_k + v^iy_i\\
            &=v^kx_k -\frac{v^ax_a}{\sqrt{1-x^ax_a}}\sqrt{1-x^ax_a}\\
            &=v^kx_k - v^ax_a\\
            &= 0 
\end{aligned}
\]
Where the contraction $v^kx_k$ omits the $i\th$ index (since it is not defined for 
$v$ or $x$).

Thus, $di(v)$ is perpendicular to $p$, as desired.
\end{proof}

\subsection*{part c}
For $F$ a smooth map from $\R^{n+1}$ to $\R^{m+1}$ such that $F(S^n)\subset S^m$, show
that $d(F|_{S^n}) = dF|_{T_pS^n}$.
\\
\begin{proof}
To begin with, let $\gamma$ be a curve in $S^n$ such that
$\gamma(0)=p$ and $\gamma'(0)=v$. Then,
\[
\begin{aligned}
d(F|_{S^n})(\gamma'(0)) &= \partial_t|_0 F|_{S^n}(\gamma(t))\\
                        &= \partial_t|_0 F(\gamma(t))\\
                        &= dF(\gamma'(0))\\
                        &= dF(\gamma'(0))|_{T_pS^n}
\end{aligned}
\]
Here, the equality from line 1 to line 2 comes from the fact that $F$ maps $S^n$ into $S^m$,
and the equality from line 3 to line 4 comes from the fact that $\gamma'(0)$ started in
$T_pS^n$ to begin with.
\end{proof}
%----------------------------------------------------------------------------------------
\newpage
%----------------------------------------------------------------------------------------
%	PROBLEM 2
%----------------------------------------------------------------------------------------
\section*{Problem 2}
Show that the tangent bundle $TM$ is always orientable.
\\
\begin{proof}
For this problem, we will use that fact that a manifold is orientable if there exist
charts such that the coordinate transition maps have a Jacobian of positive determinant.

Before proceeding further, we prove the following lemma:

\begin{lemma}
Suppose $M$ and $N$ are smooth manifolds. In particular, their product $M\times N$ is a
smooth manifold. Furthermore, for pairs of coordinates $x_m,y_m$ on $U_m\subset M$ and
$x_n,y_n$ on $U_n\subset N$, the product coordinates $x=(x_m,x_n)$ and $y=(y_m,y_n)$ are
smooth coordinates on $U_m\times U_n$, and the Jacobian $J(x\to y)$ is given componentwise.
That is,
\[
J(x\to y) = (J(x_m\to y_m),J(x_n\to y_n))
\]
\end{lemma}
\begin{proof}
That $M\times N$ is a smooth manifold follows almost immediately by taking products of
coordinate charts on $M$ and $N$. Now, we have the following diagram:
\[
\begin{tikzcd}[column sep=tiny]
 &                  &M\times N                  &       &\\
 &                  &U_m\times U_n\arrow[hook]{u}\arrow[dashed, shift right]{ddd}[swap]{x}\arrow[dashed, shift left]{ddd}{y}\arrow[two heads]{dl}\arrow[two heads]{dr}&     &\\
M& U_m\arrow[hook']{l}\arrow[hook, shift right]{d}[swap]{x_m} \arrow[hook, shift left]{d}{y_m} && U_n\arrow[hook]{r} \arrow[hook, shift right]{d}[swap]{x_n}\arrow[hook,shift left]{d}{y_n} &N\\
 & \R^m\arrow[hook]{dr} &&\R^n\arrow[hook']{dl} &\\
    &&\R^{m+n} &&
\end{tikzcd}
\]
which implies that the induced coordinates $x$ and $y$ are smooth.
Now, let's expand the lower half of the commutative diagram to get the transition maps
$x_m\to y_m$ and $x_n\to y_n$:
\[
\begin{tikzcd}[column sep=tiny]
\R^m\arrow{dd}[swap]{x_m\to y_m}\arrow[hook]{dr} &&\R^n\arrow{dd}{x_n\to y_n}\arrow[hook]{dl}\\ 
&\R^{m+n}\arrow[dashed]{dd}[description]{x\to y}\arrow[two heads]{dl}\arrow[two heads]{dr}&\\
\R^m\arrow[hook]{dr}&&\R^n\arrow[hook']{dl}\\
&\R^{m+n}
\end{tikzcd}
\]

Differentiating this diagram (applying the differential functor) yields:
\[
\begin{tikzcd}[column sep=small]
\R^m\arrow{dd}[swap]{J(x_m\to y_m)}\arrow[hook]{dr} &&\R^n\arrow{dd}{J(x_n\to y_n)}\arrow[hook]{dl}\\ 
&\R^{m+n}\arrow[dashed]{dd}[description]{J(x\to y)}\arrow[two heads]{dl}\arrow[two heads]{dr}&\\
\R^m\arrow[hook]{dr}&&\R^n\arrow[hook']{dl}\\
&\R^{m+n}
\end{tikzcd}
\]
Thus, $J(x\to y)=(J(x_m\to y_m),J(x_n\to y_n))$ as desired.
\end{proof}
We are now ready to prove the general result.

So, let $M$ be a smooth manifold with tangent bundle $TM$. Furthermore, for a point $p$,
suppose there are two coordinate charts $x^i$ and $y^i$ on a neighborhood of $p$. We wish
to calculate the Jacobian of the induced coordinate transformations on the tangent bundle.

To do so, we first appeal to the fact that $TM$ is locally trivializable. That is, on
some neighborhood $U$ containing $p$, $\pi^{-1}(U)\cong M\times T_pM$, where $\pi$ is the
canonical projection of $TM$ onto $M$. In particular, this means that in $\pi^{-1}(U)$,
we have the coordinate charts $x^i\times dx^i$ and $y^i\times dy^i$. 

Thus, the transition map is just $(x\to y,dx\to dy)$, where $x\to y$ is the transition map
from the $x$ coordinate system to the $y$ coordinate system, and $dx\to dy$ is the transition
map from the $\partial_{x}|_p$ coordinate system to the $\partial_y|_p$ coordinate system.

Recall that $dx\to dy$ is simply the Jacobian of the original coordinate transform. That is,
$dx\to dy = J(x\to y)$.
Now, from the above lemma, 
\[
J(x\to y,dx\to dy) = J(x\to y,J(x\to y)) = (J(x\to y),J(J(x\to y)))
\]
It should be clear that $J^2=J$, since the Jacobian of a transformation is linear. Thus,
we have that
\[
J(x\to y, dx\to dy) = (J(x\to y),J(x\to y))
\]

To calculate the determinant of this, we appeal to the fact that the determinant of
a linear transformation of the form $(A,B)$ is the product of the determinants of $A$ and $B$.
Thus,
\[
\begin{aligned}
\det J(x\to y,dx\to dy) &= \det(J(x\to y))\det(J(x\to y))\\
                        &= \left(\det(J(x\to y))\right)^2
\end{aligned}
\]
Which is always positive.

Since this can be done at any point $p$ in the manifold, we have an atlas for $TM$
where the determinant of the coordinate transforms is always positive, and thus
$TM$ is orientable.
\end{proof}
%----------------------------------------------------------------------------------------
\newpage
%----------------------------------------------------------------------------------------
%	PROBLEM 3
%----------------------------------------------------------------------------------------
\section*{Problem 3}
Show that for $M$ a smooth manifold, and $S\subset M$ a smooth submanifold, $S$
is embedded if and only if for every $f\in C^{\infty}(S)$, $f$ has a smooth
extension to a neighborhood of $S$ in $M$.
\\
\begin{proof}
    ($\implies$)
    Suppose $S$ is embedded. In particular, this means that each point $s\in S$
    has a neighborhood $U_s$ on which there is a slice chart $\phi$ of $S$ on
    $M$. In particular, if $\dim(M)=m$ and $\dim(S)=k$, and $x^i$ are the
    coordinate functions of $\phi$, then there is a chart on $U_s$ centered at
    $s$ for which $(x^1,\ldots,x^m)|_S = (x^1,\ldots,x^k,0,\ldots,0)$.

    Now, let $f\in C^{\infty}(S)$. Since $U_s$ has a slice chart, it is possible
    to extend $f$ locally to a function $\tilde{f}_s\in C^{\infty}(U_s)$ such
    that $\tilde{f}_s|_S = f$. This is possible by making the function constant
across the orthogonal complements to the slice chart. That is, given that the
function is defined on $\R^k\subset \R^m$, extend it to the whole space
by defining $\tilde{f}(p)$ to be the value of $f$ at the orthogonal projection of
$p$ onto the subspace $\R^k$.

    (Note that if $\dim(S)=\dim(M)$, then such extensions cannot be well defined. However,
    Proposition 5.1 guarantees that $S$ is open in $M$, and $S$ itself is an open
neighborhood on which $f$ is defined, which is what we wanted to begin with).

    Now that we have local extensions of the function, we use a partition of unity
to extend them globally. So, choose a countable number of such $U_s$, call them $U_i$, that cover
    $S$, and let $\Psi = \sum_{i=1}^{\infty}\psi_i$ be a partition of unity
    subordinate to $\{U_i\}$. Then, the function
    \[
        \tilde{f}(x) = \sum_{i=1}^{\infty}\psi_i(x)\tilde{f}_i(x)
    \]
    is a smooth extension of $f$ that restricts to $f$. 

    ($\impliedby$)
    Suppose for the converse that each $f\in C^{\infty}(S)$ had a smooth
    extension to a neighborhood $U$ of $S$. Now, in particular for each open set
    $U$ in $S$, we can construct a smooth function $f$ on $S$ for which
    $\textrm{supp}(f) = U$. This function has a smooth extension $\tilde{f}$ on some
    $U'\subset M$ for which $\tilde{f}|_U=f$. Furthermore,
    $\textrm{supp}(\tilde{f})$ is open in $M$, and the intersection
    \[
        \textrm{supp}(\tilde{f})\cap S = \textrm{supp}(f)=U
    \]
    Thus, each open set in $S$ is the intersection of $S$ with an open set in
    $M$, and $S$ has the subspace topology, making $S$ an embedding.
\end{proof}
%----------------------------------------------------------------------------------------
\newpage
%----------------------------------------------------------------------------------------
%	PROBLEM 4
%----------------------------------------------------------------------------------------
\section*{Problem 4}
Let $M$ be a manifold, and $p\in M$.
\subsection*{Part a}
Show that $I_p = \{f\in C^{\infty}(M)\ |\ f(p)=0\}$ is a maximal ideal in $C^{\infty}(M)$.
\\
\begin{proof}
Suppose $I_p\subsetneq I$ for some ideal $I$. In particular, this means that
    there is some $f\in I$ with $f(p)\neq 0$. Without loss of generality, let
    $f(p)>0$.

    Now, by theorem 2.29 of Lee, there exists a nonnegative function $g$ for which
    $g^{-1}(0)=\{p\}$. Since this function vanishes at $p$, it is in the
    ideal, as well as the function $f+g$. In particular, $f+g > 0$ at every
    point.

    Thus, the function $\frac{1}{f+g}$ is well-defined, and by the
    multiplicatively absorbing property of ideals, the function
    $\frac{1}{f+g}(f+g) = 1$ is in the ideal as well. Since $1$ is the unit,
    it follows that $I=C^{\infty}(M)$. Thus, $I_p$ is maximal.
\end{proof}

\subsection*{Part b}
Show that if $M$ is compact, any maximal ideal in $C^{\infty}(M)$ is of this
form.
\\
\begin{proof}
Suppose for a contradiction that some maximal ideal $I$ such that at each point
$p\in M$, there is some $f_p$ such that $f_p\neq 0$.

In particular, this means that there is a neighborhood $U_p$ of $p$ for which
    $f_p$ is nonzero. The neighborhoods $U_p$ form an open cover of $M$, of
    which there is a finite subcover $\{U_i\}_{i=1}^n$. Then, the function
    $F = \sum_{i=1}^n(f_i)^2$ is everywhere nonzero, and in the ideal since $I$ is
    additively closed. Thus, the function $\frac{F}{F}=1$ is well-defined
    everywhere, and in the ideal $I$. Thus, $I=C^{\infty}(M)$, which contradicts
    $I$ being (nontrivially) maximal.

    Thus, for each ideal $I$ in $C^{\infty}(M)$, $I$ must have all functions
    vanish at at least one point $p$. Considering only the maximal ideals, it
    must follow that for a maximal ideal $I$, the functions in $I$ vanish at
    exactly one point $p$, so $I=I_p$. This follows by observing that if
    functions in $I$ vanished at two points $p$ and $q$, then $I$ would be
    contained in $I_p$ and $I_q$, and would not be maximal.

    Thus, every maximal ideal in $C^{\infty}(M)$ is of the form $I_p$.
\end{proof}
%----------------------------------------------------------------------------------------
\newpage
%----------------------------------------------------------------------------------------
%	PROBLEM 5
%----------------------------------------------------------------------------------------
\section*{Problem 5}
\subsection*{Part a}
Show that, for a quaternion $q=a+bi+cj+dk$, $|q|^2 = q\bar{q}$ is equal to
$a^2+b^2+c^2+d^2$. Conclude that $S^3$ can be identified with the unit
quaternions.
\\
\begin{proof}
This follows from direct computation.
    \[
        \begin{aligned}
            q\bar{q} &= (a+bi+cj+dk)(a-bi-cj-dk)\\
                    &=a^2-abi-acj-adk+abi-(bi)^2-bcij-bdik\\
                    &+acj-bcji-(cj)^2-cdjk+adk-bdki-cdkj-(dk)^2\\
                    &=a^2+b^2+c^2+d^2-bcij-bcji-bdik-bdki-cdjk-cdkj-bdki-bdik\\
                    &=a^2+b^2+c^2+d^2
        \end{aligned}
    \]
    where the last equality was obtained by observing that $ij=-ji$ and
    $jk=-kj$.

    Thus, the norm on $\mathbb{H}$ coincides with the norm on $\R^4$, and so
    the topologies agree. Thus, $S^3$ can be identified with the unit
    quaternions in an isometric way.
\end{proof}

\subsection*{Part b}
Show that $S^3$ is a Lie group with quaternion multiplication.
\\
\begin{proof}
We will first show that the operation of quaternion multiplication is smooth
    in the ambient space $\R^4\setminus\{0\}\cong\mathbb{H}\setminus\{0\}$, and
    conclude that since $S^3$ is a closed embedded subgroup of
    $\mathbb{H}\setminus\{0\}$, it is a Lie group under the same operation.

    We will show that the operation of multiplication is smooth by directly
    calculating the multiplication in the standard coordinates on $\R^4$.

    So, let $q_1 = (x^1,x^2,x^3,x^4)$ and $q^2=(y^1,y^2,y^3,y^4)$. Then,
    \[
        \begin{aligned}
            q_1q_2 &= (x^1+x^2i+x^3j+x^4k)(y^1+y^2i+y^3j+y^4k)\\
                    &=x^1y^1 + x^1y^2i + x^1y^3j + x^1y^4k\\
                    &+x^2y^1i + x^2y^2ii + x^2y^3ij + x^2y^4ik\\
                    &+x^3y^1j + x^3y^2ji + x^3y^3jj + x^3y^4jk\\
                    &+x^4y^1k + x^4y^2ki + x^4y^3kj + x^4y^4kk\\
                    &= x^1y^1 - x^2y^2 - x^3y^3 - x^4y^4\\
                    &+(x^1y^2 + x^2y^1 + x^3x^4 - x^4x^3)i\\
                    &+(x^1y^3 - x^2y^4 + x^3y^1 + x^4y^2)j\\
                    &+(x^1y^4 + x^2y^3 - x^3y^2 + x^4y^1)k
        \end{aligned}
    \]
    which is clearly a smooth operation. Thus, $S^3$ is a Lie group under
    quaternion multiplication.
\end{proof}
%----------------------------------------------------------------------------------------

\end{document}
