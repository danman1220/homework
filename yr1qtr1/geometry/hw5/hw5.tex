%%%%%%%%%%%%%%%%%%%%%%%%%%%%%%%%%%%%%%%%%
% Short Sectioned Assignment
% LaTeX Template
% Version 1.0 (5/5/12)
%
% This template has been downloaded from:
% http://www.LaTeXTemplates.com
%
% Original author:
% Frits Wenneker (http://www.howtotex.com)
%
% License:
% CC BY-NC-SA 3.0 (http://creativecommons.org/licenses/by-nc-sa/3.0/)
%
%%%%%%%%%%%%%%%%%%%%%%%%%%%%%%%%%%%%%%%%%

%----------------------------------------------------------------------------------------
%	PACKAGES AND OTHER DOCUMENT CONFIGURATIONS
%----------------------------------------------------------------------------------------

\documentclass[fontsize=11pt]{scrartcl} % 11pt font size

\usepackage[T1]{fontenc} % Use 8-bit encoding that has 256 glyphs
\usepackage[english]{babel} % English language/hyphenation
\usepackage{amsmath,amsfonts,amsthm} % Math packages
\usepackage{mathrsfs}
\usepackage{tikz-cd}

\usepackage[margin=1in]{geometry}

\usepackage{sectsty} % Allows customizing section commands
\allsectionsfont{\centering \normalfont\scshape} % Make all sections centered, the default font and small caps

\usepackage{fancyhdr} % Custom headers and footers
\pagestyle{fancyplain} % Makes all pages in the document conform to the custom headers and footers
\fancyhead{} % No page header - if you want one, create it in the same way as the footers below
\fancyfoot[L]{} % Empty left footer
\fancyfoot[C]{} % Empty center footer
\fancyfoot[R]{\thepage} % Page numbering for right footer
\renewcommand{\headrulewidth}{0pt} % Remove header underlines
\renewcommand{\footrulewidth}{0pt} % Remove footer underlines
\setlength{\headheight}{13.6pt} % Customize the height of the header

\numberwithin{equation}{section} % Number equations within sections (i.e. 1.1, 1.2, 2.1, 2.2 instead of 1, 2, 3, 4)
\numberwithin{figure}{section} % Number figures within sections (i.e. 1.1, 1.2, 2.1, 2.2 instead of 1, 2, 3, 4)
\numberwithin{table}{section} % Number tables within sections (i.e. 1.1, 1.2, 2.1, 2.2 instead of 1, 2, 3, 4)

\newcommand{\R}{\mathbb{R}}
\newcommand{\Q}{\mathbb{Q}}
\newcommand{\N}{\mathbb{N}}
\newcommand{\C}{\mathbb{C}}

\newtheorem{lemma}{Lemma}
%----------------------------------------------------------------------------------------
%	TITLE SECTION
%----------------------------------------------------------------------------------------

\newcommand{\horrule}[1]{\rule{\linewidth}{#1}} % Create horizontal rule command with 1 argument of height

\title{	
\normalfont \normalsize 
\textsc{Geometry} \\ [25pt] % Your university, school and/or department name(s)
\horrule{0.5pt} \\[0.4cm] % Thin top horizontal rule
\huge Problem Set 5 \\ % The assignment title
\horrule{2pt} \\[0.5cm] % Thick bottom horizontal rule
}

\author{Daniel Halmrast} % Your name

\date{\normalsize\today} % Today's date or a custom date

\begin{document}

\maketitle % Print the title


\section*{Problem 1} %Lee 8-10
Let $M$ be the open submanifold of $\R^2$ with both coordinates positive, and
define $F:M\to M$ as $F(x,y) = (xy,\frac{y}{x})$. Show that $F$ is a
diffeomorphism, and compute $dFX$ and $dFY$ for
\[
    \begin{aligned}
        X &= x\partial_x + y\partial_y\\
        Y &= y\partial_x
    \end{aligned}
\]

\begin{proof}
    To begin with, we observe that so long as $x\neq 0$, $F$ is in fact smooth.
    Furthermore, it has an inverse
    \[
        F^{-1}(x,y) = (\sqrt{\frac{x}{y}},\sqrt{xy})
    \]
    which is also smooth on $M$, and defined for all of $M$.

    Furthermore, we can calculate its Jacobian:
    \[
        J(F) = dF =
        \begin{bmatrix}
            y & x\\
            \frac{-y}{x^2} & \frac{1}{x}
        \end{bmatrix}
    \]
    which expresses $dF$ in the coordinates $\partial_x,\partial_y$ at every
    point.

    Now, we calculate $dF(X)$:
    \[
        \begin{aligned}
            dF(X) &= dF(x\partial_x + y\partial_y)\\
                &= xdF(\partial_x) + ydF(\partial_y)\\
                &= x(y\partial_x - \frac{y}{x^2}\partial_y) + y(x\partial_x +
                \frac{1}{x}\partial_y)\\
                &= xy\partial_x - \frac{y}{x}\partial_y + yx\partial_x
                +\frac{y}{x}\partial_y\\
                &= 2xy\partial_x
        \end{aligned}
    \]
    and $dF(Y)$:
    \[
        \begin{aligned}
            dF(Y) &= dF(x\partial_y)\\
                  &= xdF(\partial_y)\\
                  &= x(x\partial_x + \frac{1}{x}\partial_y)\\
                  &=x^2\partial_x + \partial_y
        \end{aligned}
    \]
\end{proof}

\section*{Problem 2} %Lee 8-15
Let $M$ be a smooth manifold, $S\subseteq M$ an embedded submanifold. Given
$X\in\mathcal{X}(S)$, show that there is a smooth vector field $Y$ on a
neighborhood of $S$ in $M$ such that $X=Y|_S$. Show that every such vector field
extends to all of $M$ if and only if $S$ is properly embedded.
\\
\begin{proof}
    Recall that a vector field $X\in\mathcal(X)(S)$ is a linear derivation of
    the algebra $C^{\infty}(M)$ over $\R$. That is, $X$ is a linear map
    $X:C^{\infty}(M)\to C^{\infty}(M)$ such that
    \[
        X(fg) = X(f)g + fX(g)
    \]

    Now, from an earlier assignment, we know that for $S\subseteq M$ an embedded
    submanifold of a manifold $M$, we have the existence of extensions of
    $C^{\infty}$ functions on $S$ to $C^{\infty}$ functions on the neighborhood
    $U\subseteq M$ of $S$. That is, the restriction function $r:C^{\infty}(U)\to
    C^{\infty}(S)$ has a section $e:C^{\infty}(S)\to C^{\infty}(U)$ such that
    $r\circ e = id$. Thus, we have the following diagram:
    \[
        \begin{tikzcd}
            C^{\infty}(S)\arrow{r}{X} &C^{\infty}(S)\arrow{d}{e}\\
            C^{\infty}(U)\arrow{u}{r} &C^{\infty}(U)
        \end{tikzcd}
    \]
    Now, we define $Y$ to be the linear map $Y:C^{\infty}(U)\to C^{\infty}(U)$
    that makes the diagram commute. That is:
    \[
        \begin{tikzcd}
            C^{\infty}(S)\arrow{r}{X} &C^{\infty}(S)\arrow{d}{e}\\
            C^{\infty}(U)\arrow{u}{r}\arrow[dashed]{r}{Y} &C^{\infty}(U)
        \end{tikzcd}
    \]
    We note that such a $Y$ is not unique, since the extension $e$ is not
    uniquely defined. In fact, all extensions differ by an element of
    $C^{\infty}(U)/{C^{\infty}(S)}$.

    $Y$ is clearly linear, since it is the composition of linear arrows, so all
    that we need to show is that $Y$ is a derivation. First, we observer that
    since $e$ is a section of $r$, it follows that $X\circ r = r\circ Y$. That
    is, the diagram
    \[
        \begin{tikzcd}
            C^{\infty}(S)\arrow{r}{X} &C^{\infty}(S)\arrow[shift left]{d}{e}\\
            C^{\infty}(U)\arrow{u}{r}\arrow[dashed]{r}{Y}
            &C^{\infty}(U)\arrow[shift left]{u}{r}
        \end{tikzcd}
    \]
    commutes. We note also that the restriction map $r$ is multiplicative. That
    is, $r(fg) = r(f)r(g)$. 

    Let $f,g\in
    C^{\infty}(U)$. We calculate
    \[
        \begin{aligned}
            Y(fg) &= e\circ X\circ r(fg)\\
                    &= e\circ X(r(f)r(g))\\
                    &= e(X(r(f))r(g) + r(f)X(r(g)))\\
                    &= e(r(Y(f))r(g) +r(f)r(Y(g))\\
                    &= (e\circ r)(Y(f)g + fY(g))
        \end{aligned}
    \]
    so if $e\circ r$ is the identity on $Y(f)g + fY(g)$, then $Y$ is a
    derivation. However, we recall that $e$ is only unique up to a factor of
    $C^{\infty}(U)/{C^{\infty}(S)}$. So, for each $f\in C^{\infty}(U)$, we
    define $e_f$... %TODO finish
\end{proof}

\section*{Problem 3} %Lee 8-18
%TODO write this

\section*{Problem 4} %Lee 8-19
Show that $\R^3$ is a Lie algebra with the cross product.
\\
\begin{proof}
    The cross product is, by definition, bilinear, so it suffices to check that the
    cross product satisfies the Jacobi identity.

    We proceed to calculate the Jacobi identity directly. Now, we know that
    \[
        ((A\times B)\times C)^i = \epsilon^i_{jk}\epsilon^j_{mn}A^mB^nC^k
    \]
    where $\epsilon^i_{jk}$ is the Levi-Civita symbol.
    So
    \[
        \begin{aligned}
            ((A\times B)\times C + (B\times C)\times A + (C\times A)\times B)^i &=
            \epsilon^i_{jk}\epsilon^j_{mn}(A^kB^mC^n + B^kC^mA^n + C^kA^mB^n)\\
            &=T^i_{kmn}V^{kmn}
        \end{aligned}
    \]
    where $T^i_{kmn}= \epsilon^i_{jk}\epsilon^j_{mn}$ and $V^{kmn} = A^kB^mC^n +
    B^kC^mA^n + C^kA^mB^n$.

    Now, $\epsilon^i_{jk}\epsilon^j_{mn} = T^i_{kmn}$is a symbol of rank $(1,3)$ whose
    components can be calculated directly. It is easy to see that this symbol is
    antisymmetric in the first and last two components. Furthermore, the only
    nonzero terms (up to antisymmetry) is $T^i_{kki} = 1$. Thus
    \[
        \begin{aligned}
            ((A\times B)\times C + (B\times C)\times A + (C\times A)\times B)^i &=
            T^i_{kmn}V^{kmn}\\
            &=T^i_{kki}V^{kki} + T^i_{kik}V^{kik} &\textrm{for fixed }i,k\\
            &=T^i_{kki}V^{kki} - T^i_{kki}V^{kki} &\textrm{by antisymmetry of
            }T\\
            &=0
        \end{aligned}
    \]
    as desired.

    Thus, the cross product is a bilinear map that satisfies the Jacobi
    identity, and $\R^3$ with this product is a Lie algebra.
\end{proof}

\section*{Problem 5} %Lee 8-20
Let $A\subseteq \mathcal{X}(\R^3)$ be the subspace spanned by the vector fields
\[
    \begin{aligned}
        X &= y\partial_z - z\partial_y\\
        Y &= z\partial_x - x\partial_z\\
        Z &= x\partial_y - y\partial_x\\
    \end{aligned}
\]
Show that $A$ is a Lie subalgebra of $\mathcal{X}(\R^3)$.
\\
\begin{proof}
    We note first that this defines a two-dimensional distribution on $\R^3$,
    since for all points $(x,y,z)$ for which $z\neq 0$, we have that
    $Z = \frac{x}{z}X + \frac{y}{z}Y$, and similarly if $z=0$, we have that
    $-X = \frac{y}{x}Y + \frac{z}{x}Z$. 

    So, all we need to check is that this distribution is involutive.
    Furthermore, since $[X,X]=0$ for all $X$, it suffices to only check that for
    $X,Y$ that span the distribution, $[X,Y]$ is in the distribution.

    So, since $X$ and $Y$ are linearly independent for $z\neq 0$, they span the
    distribution for all $z\neq 0$. Now, we can calculate
    \[
        \begin{aligned}
            {[}X,Y{]}^i &= X^j\partial_jY^i - Y^j\partial_jX^i\\
            [X,Y]^x &= X^j\partial_jY^x - Y^j\partial_jX^x\\
                    &= X^j\partial_jz - 0\\
                    &= X^z = y\\
                    \\
            [X,Y]^y &= X^j\partial_jY^y - Y^j\partial_jX^y\\
                    &= 0 - Y^j\partial_j(-z)\\
                    &= Y^z = -x\\
                    \\
            [X,Y]^z &= X^j\partial_jY^z - Y^j\partial_jX^z\\
                    &= X^j\partial_j(-x) - Y^j\partial_j(y)\\
                    &=-X^x - Y^y = 0
        \end{aligned}
    \]
    Thus, ${[}X,Y{]} = y\partial_x - x\partial_y = -Z$, which is in the
    distribution, so the distribution is involutive for $z\neq 0$. However, the
    exact same calculation on ${[}Y,Z{]}$ by permuting the indices reveals that
    ${[}Y,Z{]} = -X$, and so the distribution is involutive.

    Thus, the subspace $A$ is closed under the Lie bracket, and is a Lie
    subalgebra of $\mathcal{X}(\R^3)$
\end{proof}

\section*{Problem 6}
Construct a non-vanishing vector field on a general odd-dimensional sphere.
\\
\\
\begin{proof}
    Let $S^{2n-1}$ be embedded in $\C^n$.
\end{proof}

\end{document}
