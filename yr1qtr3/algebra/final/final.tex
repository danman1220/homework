\documentclass[12pt,reqno]{amsart}
\usepackage{amssymb}
\usepackage{amscd}
\usepackage{amsxtra}
\usepackage[mathscr]{eucal}

\setlength{\oddsidemargin}{0cm}
\setlength{\evensidemargin}{0in}
\setlength{\textwidth}{16.5cm}
\setlength{\topmargin}{0.35cm}
\setlength{\textheight}{8.5in}
\renewcommand{\baselinestretch}{1.33}
\pagestyle{plain}

\newcommand{\Z}{\mathbb{Z}}
\newcommand{\C}{\mathbb{C}}
\newcommand{\R}{\mathbb{R}}
\newcommand{\N}{\mathbb{N}}
\newcommand{\Q}{\mathbb{Q}}
\newcommand{\F}{\mathbb{F}}

\newcommand{\Hom}{\text{Hom}}
\newcommand{\im}{\text{im}}

\newtheorem*{defn}{Definition}
\newtheorem*{lemma}{Lemma}
\newtheorem*{theorem}{Theorem}

\begin{document}
\title[]{Math 220C: Final Examination\\June 4, 2018\\Daniel Halmrast}
\maketitle
\large

\section*{Problem 1}
Let $K/k$ be a finite extension of fields, with $k$ infinite.

\subsection*{Part a}
Define what it means for $K/k$ to be separable.

\begin{defn}
    Let $L$ be an algebraic closure of $K$. We say that $K/k$ is separable if
    the number of $k$-embeddings of $K$ into $L$ is equal to the degree $[K:k]$
    of the extension.
\end{defn}

\subsection*{Part b}
Suppose $K/k$ is separable. Prove that $K/k$ is simple, i.e. that there exists
$\alpha\in K$ such that $K=k(\alpha)$.
\\
\\
\begin{proof}
    Since $K/k$ is a finite separable extension, it is built up from simple
    extensions.  That is, there is a tower
    \[
        F_0 = k \subset F_1\subset \cdots \subset F_n = K
    \]
    with $F_{i+1} = F_i(\alpha_i)$ where the extension $F_i(\alpha_i)$ is
    separable. We will prove that a separable extension of the form
    $k(\alpha_1,\alpha_2)/k$ is simple, and use induction to show that $K/k$ is
    simple.

    So, let $k(\alpha_1,\alpha_2)/k$ be a separable extension, with $L$ an
    algebraic closure of $k$. In particular, we know that there are
    $n=[k(\alpha_1,\alpha_2):k]$ distinct $k$-embeddings of
    $k(\alpha_1,\alpha_2)$ into $L$. Denote these embeddings as $\sigma_i$ for
    $1\leq i\leq n$.

    Now, define a polynomial $f$ as
    \[
        f(x) = \prod_{i\neq j}((\sigma_i(\alpha_1) - \sigma_j(\alpha_1)) +
        (\sigma_i(\alpha_2) -\sigma_j(\alpha_2))x)
    \]

    Since each $\sigma_i$ is a distinct $k$-embedding of $k(\alpha_1,\alpha_2)$,
    it follows that at least one of
    \[
        \begin{aligned}
            \sigma_i(\alpha_1)-\sigma_j(\alpha_1)\\
            \sigma_i(\alpha_2)-\sigma_j(\alpha_2)\\
        \end{aligned}
    \]
    is nonzero for each $i\neq j$. Thus, $f$ is not the zero polynomial. So, let
    $a$ be such that $f(a)\neq 0$. Then, it follows that each linear term of
    $f(a)$ is nonzero as well. That is, for every $i\neq j$,
    \[
        ((\sigma_i(\alpha_1) - \sigma_j(\alpha_1)) +
        (\sigma_i(\alpha_2) -\sigma_j(\alpha_2))a)\neq 0
    \]
    or
    \[
        \sigma_i(\alpha_1) + \sigma_i(\alpha_2)a \neq
        \sigma_j(\alpha_1)+\sigma_j(\alpha_2)a
    \]
    which implies (by linearity of $\sigma$) that
    \[
        \sigma_i(\alpha_1 + a\alpha_2) \neq \sigma_j(\alpha_1 + a\alpha_2)
    \]
    for all $i\neq j$. Thus, each $\sigma_i$ sends $\alpha_1 + a\alpha_2$ to a
    different element of $k(\alpha_1,\alpha_2)$, and so the separability degree
    of $k(\alpha_1 + a\alpha_2)$
    (the number of distinct $k$-embeddings of $k(\alpha_1 + a\alpha_2)$ into an
    algebraic closure $L$) is
    \[
        [k(\alpha_1 + a\alpha_2):k]_s\geq n
    \]
    and since $\alpha_1 +a\alpha_2\in k(\alpha_1,\alpha_2)$, the extension
    $k(\alpha_1+a\alpha_2)$ is separable. Thus the separability degree equals
    the degree of the extension. However, since $k(\alpha_1 +a\alpha_2)\subset
    k(\alpha_1,\alpha_2)$, we know that
    \[
        [k(\alpha_1+a\alpha_2):k]\leq [k(\alpha_1,\alpha_2):k]=n
    \]
    and thus $[k(\alpha_1+a\alpha_2):k]=n$. Therefore, the extension
    \[
        [k(\alpha_1,\alpha_2):k(\alpha_1+a\alpha_2)]=1
    \]
    and so $k(\alpha_1,\alpha_2) = k(\alpha_1+a\alpha_2)$ and
    $k(\alpha_1,\alpha_2)$ is indeed simple.

    Finally, we argue by induction that every finite separable extension is
    simple. Let $F_i$ be as defined earlier in the proof. We will induct on the
    length of the tower $i+1$. The base case for $i=1$ implies that $K =
    k(\alpha)$ which is already a simple extension.

    So, suppose $K$ has a separable simple tower $F_i$ of length $j+1$, and
    assume that every separable extension with a separable simple tower $F_i$ of
    length $j$ is simple. Now,
    \[
        K = F_j(\alpha_j)
    \]
    By the inductive hypothesis, $F_j$ is simple. Thus, $F_j = k(\beta)$ for
    some $\beta$. Then, $K = k(\alpha_j,\beta)$ where $k(\alpha_j,\beta)$ is a
    separable extension. By the earlier result of this proof,
    $k(\alpha_j,\beta)$ is in fact a simple extension, and so
    $K=k(\alpha_j,\beta)$ is a simple extension as well, as desired.
\end{proof}

\subsection*{Part c}
Define what it means for $K/k$ to be normal.

\begin{defn}
    A finite extension $K/k$ is called normal if $K$ is the splitting field for
    some polynomial in $k[X]$.
\end{defn}

\subsection*{Part d}
Suppose $K/k$ is normal, and let $L/K$ be a finite normal extension. Is the
extension $L/k$ necessarily normal?
\\
\\
\begin{proof}
    We illustrate with a counterexample that $L/k$ is not necessarily normal.
    Consider the tower $\Q\subset \Q(\sqrt{2})\subset \Q(\sqrt[4]{2})$. Now,
    $\Q(\sqrt{2})/\Q$ is normal, since $\Q(\sqrt{2})$ is the splitting field for
    $f(X) = X^2-2$. Furthermore, $\Q(\sqrt[4]{2})/\Q(\sqrt{2})$ is normal, as it
    is the splitting field for $g(X) = X^2 - \sqrt{2}$.

    However, $\Q(\sqrt[4]{2})/\Q$ is not normal. This is clear, since the
    minimal polynomial for $\sqrt[4]{2}$ over $\Q$ is $h(X) = X^4-2$, which has
    complex roots. Thus, $\Q(\sqrt[4]{2})$ is not the splitting field for any
    polynomial over $\Q$, and is not normal over $\Q$ as desired.
\end{proof}

\newpage

\section*{Problem 2}
Let $\alpha_1,\alpha_2,\alpha_3,\alpha_4$ be the roots in $\C$ of the
polynomial
\[
    f(x) = x^4 +4x+2
\]
and let $K = \Q(\alpha_1,\alpha_2,\alpha_3,\alpha_4)$.

\subsection*{Part a}
Determine the number of real and non-real roots of $f(x)$.
\\
\\
\begin{proof}
    Consider the derivative
    \[
        f'(x) = 4x^3+4
    \]
    which is zero precisely when $x^3+1=0$. This has one root at $-1$, and
    dividing out the factor $(x+1)$ splits $f'(x)$ into
    \[
        f'(x) = 4(x+1)(x^2 -x+1)
    \]
    and so the other two roots of $f'(x)$ are
    \[
        \alpha_{\pm} = \frac{1\pm\sqrt{1-4}}{2}
    \]
    which are both complex. Thus, there is only one real root of $f'(x)$ at
    $x=-1$. We calculate the second derivative
    \[
        f''(x) = 12x^2
    \]
    which is always non-negative, and positive for $x\neq 0$. Thus, $f(x)$ is
    always concave up. Finally, we note that
    \[
        f(-1) = 1-4+2 = -1
    \]
    is less than zero, but since $f$ is everywhere concave up, $f$ is increasing
    after $x=-1$ and decreasing before $x=-1$, and so there are exactly two real
    roots of $f$, one before $x=-1$ and one after $x=-1$. The other two roots of
    $f$ must then be complex.
\end{proof}

\subsection*{Part b}
Explain why the number $\beta_1 = \alpha_1\alpha_2 + \alpha_3\alpha_4$ is the
root of a cubic polynomial $g(x)$ over $\Q$. Determine the number of real and
non-real roots of $g$.
\\
\\
\begin{proof}
    We define all three roots of a cubic polynomial as
    \[
        \begin{aligned}
            \beta_1 = \alpha_1\alpha_2 + \alpha_3\alpha_4\\
            \beta_2 = \alpha_1\alpha_3 + \alpha_2\alpha_4\\
            \beta_3 = \alpha_1\alpha_4 + \alpha_2\alpha_3
        \end{aligned}
    \]
    Now, we will establish that the monic polynomial with these three roots is
    in fact a polynomial over $\Q$. We calculate
    \[
        \begin{aligned}
            g(x) &= \prod_{i=1}^3(x-\beta_i)\\
                &= x^3 -\beta_1x^2 -\beta_2x^2-\beta_3x^2 + \beta_1\beta_2x +
                \beta_1\beta_3x+\beta_2\beta_3x - \beta_1\beta_2\beta_3\\
                &= x^3 - (\beta_1+\beta_2+\beta_3)x^2 + (\beta_1\beta_2 +
                \beta_1\beta_3 + \beta_2\beta_3)x - \beta_1\beta_2\beta_3
        \end{aligned}
    \]
    So, all we have to show is that 
    \[
        \begin{aligned}
            a_2 &= -(\beta_1+\beta_2+\beta_3)\\
            a_3 &= \beta_1\beta_2 + \beta_1\beta_3 + \beta_2\beta_3\\
            a_4 &= -\beta_1\beta_2\beta_3
        \end{aligned}
    \]
    are in $\Q$. We will do this by showing that $a_2,a_3,a_4$ are all fixed
    by the action of the Galois group $G(K/\Q)$.

    Now, observe that any permutation of the roots $\alpha_i$ will also permute
    the set $\{\beta_i\}$. This follows from the fact that
    $\beta_1,\beta_2,\beta_3$ represent the three ways to partition the roots
    into two disjoint sets. Thus, any permutation of the roots $\alpha_i$ will
    send one partition $\beta_i$ to another partition $\beta_j$.  For $\sigma$ a
    permutation of the roots, we will denote $\beta_{\sigma(i)}$ to be
    $\sigma(\beta_i)$ the result of permuting the roots by $\sigma$. That is,
    \[
        \beta_{\sigma(1)} = \sigma(\alpha_1)\sigma(\alpha_2) +
        \sigma(\alpha_3)\sigma(\alpha_4)
    \]
    and so on.

    Observe that
    \[
        \sigma(a_2) = -(\beta_{\sigma(1)} + \beta_{\sigma(2)} +
        \beta_{\sigma(3)})
    \]
    which, by rearranging, is just
    \[
        \sigma(a_2) = -(\beta_1 + \beta_2 + \beta_3) =a_2
    \]
    and so $a_2$ is fixed by each $\sigma\in G(K/\Q)$.

    We observe similarly that $a_3$ and $a_4$ are fixed under permutations of
    $\{\beta_i\}$, and are thus also fixed for each $\sigma\in G(K/\Q)$.
    Therefore, each coefficient is in the fixed field of $G(K/\Q)$, and thus are
    all in $\Q$ as desired. 

    So, $g(x)$ is a polynomial over $\Q$ with $\beta_1$ as a root.

    Finally, we show that $g(x)$ has one real root and two complex roots. To see
    this, recall that $g(x)$ has $\beta_i$ as its roots. Now, without loss of
    generality, let $\alpha_2=\overline{\alpha_1}$ be the two complex roots of
    $f$. Then, we observe that
    \[
        \begin{aligned}
            \beta_1 &= \alpha_1\overline{\alpha_1} + \alpha_3\alpha_4\\
            &= \|\alpha_1\|^2 + \alpha_3\alpha_4
        \end{aligned}
    \]
    is the sum of real numbers, and is therefore real. However, writing
    $\alpha_1$ as $a+bi$, we see that
    \[
        \begin{aligned}
            \beta_2 &= \alpha_1\alpha_3 + \overline{\alpha_1}\alpha_4\\
            &= \alpha_3a + \alpha_3 bi + \alpha_4a - \alpha_4bi\\
            &= (\alpha_3+\alpha_4)a +(\alpha_3-\alpha_4)bi
        \end{aligned}
    \]
    and since $\alpha_3\neq \alpha_4$, this number is non-real. Finally, this
    implies that $\overline{\beta_2}=\beta_3$ is the other complex root, and so
    $g$ has one real root, and two complex roots, as desired.
\end{proof}

\subsection*{Part c}
Show that $K$ contains a splitting field $L$ for $g(x)$ over $\Q$.
\\
\\
\begin{proof}
    Recall from the previous part that the roots of $g(x)$ are
    \[
        \begin{aligned}
            \beta_1 = \alpha_1\alpha_2 + \alpha_3\alpha_4\\
            \beta_2 = \alpha_1\alpha_3 + \alpha_2\alpha_4\\
            \beta_3 = \alpha_1\alpha_4 + \alpha_2\alpha_3
        \end{aligned}
    \]
    which are all in $K$. Thus, $\Q(\beta_1,\beta_2,\beta_3)\subset K$, and
    since $L = \Q(\beta_1,\beta_2,\beta_3)$ is the splitting field for $g(x)$
    over $\Q$, $L\subset K$ as desired.
\end{proof}

\subsection*{Part d}
Show that the Galois group of $K$ over $L$ is a subgroup of the Klein four
group.
\\
\\
\begin{proof}
    The Galois group of $K$ over $L$ must fix $L$, so in particular it must fix
    $\beta_1$. We also know it is a subgroup of the Galois group of $K$ over
    $\Q$, and thus must be a permutation of the roots $\alpha_i$. However, the
    only permutations that fix $\beta_1$ are (in cycle notation)
    $\{e,(12),(34),(12)(34)\}$ which is the Klein four group. Thus, the Galois
    group of $K$ over $L$ must be a subgroup of the Klein four group as desired.
\end{proof}

\newpage

\section*{Problem 3}
Suppose $G$ is a finite group, and $K$ is an algebraically closed field of
characteristic zero.

\subsection*{Part a}
State Schur's lemma concerning irreducible $KG$-modules.
\begin{lemma}
    Suppose $U$, $W$ are irreducible $KG$-modules, and $f:U\to W$ is a
    $KG$-module homomorphism. Then, $f$ is either an isomorphism or the zero
    map. Furthermore, if $g:U\to U$ is a $KG$-module homomorphism, then $g$ is a
    multiple of the identity.
\end{lemma}

\subsection*{Part b}
Suppose that $G$ is abelian. Show that each irreducible $KG$-module has
dimension $1$ as a $K$-vector space.
\\
\\
\begin{proof}
    Since $G$ is abelian, each element is its own conjugacy class. Thus, there
    are exactly $|G|$ conjugacy classes in $G$. This implies that there are
    exactly $|G|$ irreducible representations of $G$. Now, the first column of
    the character table for $G$ shows the dimensions of the irreducible
    representations, and the normality condition on the first column of the
    character table says that the sum of the squares of the dimensions of the
    irreducible representations have to equal the order of the group. Now, since
    there are exactly $|G|$ irreducible representations, each with dimension
    $\geq 1$, they must all have dimension $1$ to square-sum to $|G|$. Thus, the
    dimension of each irreducible representation is $1$, as desired.
\end{proof}

\subsection*{Part c}
Suppose $G$ is abelian, and let $\chi$ be any irreducible $K$-valued character
of $G$. Evaluate $\sum_{g\in G}\chi(g)$.
\\
\\
\begin{proof}
    Suppose first that $\chi$ is the trivial character. Then, $\chi(g) = 1$ for
    all $g\in G$, and
    \[
        \sum_{g\in G}\chi(g) = |G|
    \]
    
    So, assume $\chi$ is not the trivial character.
    We know from the orthogonality relation on the rows of the character table
    of $G$ that
    \[
        |G|\langle \chi,\chi^{(1)}\rangle = 0
    \]
    we evaluate the left hand side to
    \[
        \begin{aligned}
            |G|\langle \chi,\chi^{(1)}\rangle 
            &= \sum_{g\in G}\chi(g)\overline{\chi^{(1)}(g)}\\
            &= \sum_{g\in G}\chi(g)(1)\\
            &= \sum_{g\in G} \chi(g)
        \end{aligned}
    \]
    and so $\sum_{g\in G}\chi(g) = 0$.
\end{proof}

\newpage

\section*{Problem 4}
Let $G$ denote the permutation group $S_4$ on four letters.

\subsection*{Part a}
Calculate the size of each conjugacy class of $G$, and write down a
representative of each conjugacy class.
\\
\\
\begin{proof}
    We do so by brute-force calculation. The results are tabulated below.
    Observe that
    \[
        \begin{aligned}
            S_4 = \{&(1),(12),(13),(14),(23),(24),(34),\\
                &(123),(132),(124),(142),(134),(143),(234),(243),\\
                &(12)(34),(13)(24),(14)(23),\\
            &(1234),(1432),(1243),(1342),(1324),(1423)\}
            \end{aligned}
    \]
    and so we compute the conjugacy classes:
    \begin{center}
        \begin{tabular}{l|l}
            \textbf{Conjugacy Class} &\textbf{Size}\\
            \hline
            $\{(1)\}$ &$1$\\
            $\{(12),(13),(14),(23),(24),(34)\}$ &$6$\\
            $\{(123),(132),(124),(142),(134),(143),(234),(243)\}$ & $8$\\
            $\{(12)(34),(13)(24),(14)(23)\}$ & $3$\\
            $\{(1234),(1432),(1243),(1342),(1324),(1423)\}$ &$6$\\
        \end{tabular}
    \end{center}

    We obtained this table by observing that conjugation in $S_4$ amounts to a
    relabeling of the symbols, and thus cannot change the cycle type.
    Furthermore, $S_4$ acts $4$-transitively on the set of four letters, and so
    for any two elements with the same cycle type, there is a relabeling which
    takes one to the other. That is, suppose $(abcd)$ and $(efgh)$ are of the
    same cycle type. Then, there is an element $\sigma$ of $S_4$ which takes
    $(a,b,c,d)$ to $(e,f,g,h) = (\sigma(a),\sigma(b),\sigma(c),\sigma(d))$.
    Thus,
    \[
        \sigma (abcd) \sigma^{-1} = (\sigma(a)\sigma(b)\sigma(c)\sigma(d))
        =(efgh)
    \]
    and similarly for other cycle types. 
    
    From here on out, the conjugacy classes will be represented by
    \begin{center}
        \begin{tabular}{l|l}
            \textbf{Conjugacy Class} &\textbf{Size}\\
            $(1)$ &$1$\\
            $(12)$ &$6$\\
            $(123)$ & $8$\\
            $(12)(34)$ & $3$\\
            $(1234)$ &$6$\\
        \end{tabular}
    \end{center}
\end{proof}

\subsection*{Part b}
Let $V$ be a $\C$-vector space with basis $\{e_1,e_2,e_3,e_4\}$. Suppose that
$G$ acts on $V$ by permuting the elements of this basis in the obvious way, and
let $\chi_V$ be the corresponding character of $V$. Calculate the value of
$\chi_V$ on each conjugacy class of $V$, and determine whether or not $\chi_V$
is irreducible.
\\
\\
\begin{proof}
    Note that the matrix representation of an element $g\in G$ acting on $V$ in
    the basis $\{e_i\}$ is a permutation matrix, whose trace is just the number
    of basis vectors fixed by $g$. This follows from the fact that a permutation
    matrix has columns of all zeros except one $1$, and if column $j$ has a $1$
    in the $j$th row (on the diagonal), then the matrix sends the $j$th basis
    vector to itself. Thus, the trace (the sum of the diagonal elements) is just
    the number of basis vectors fixed by $g$.

    We calculate the value of $\chi_V$ on each conjugacy class directly
    \begin{center}
        \begin{tabular}{l| l l l l l}
            $S_4$    & $(1)$ & $(12)$ & $(123)$ & $(12)(34)$ & $(1234)$\\
            $|c_i|$  & $1$   & $6$    & $8$     & $3$        & $6$\\
            \hline
            $\chi_V$ & $4$   & $2$    & $1$     & $0$        & $0$\\
        \end{tabular}
    \end{center}
    Now, if $\chi_V$ were irreducible, we must have that
    \[
        \langle \chi_V,\chi_V\rangle = 1
    \]
    so, we calculate
    \[
        \begin{aligned}
            |G|\langle \chi_V,\chi_V\rangle &= \sum_{g\in
            G}\chi_V(g)\overline{\chi_v(g)}\\
            &= \sum_{i=1}^5|c_i|\chi_V(g_i)\overline{\chi_V(g_i)} &g_i\in c_i\\
            &= 1(4)(4) + 6(2)(2) + 8(1)(1) + 3(0)(0) + 6(0)(0)\\
            &= 16 + 24 + 8 = 48
        \end{aligned}
    \]
    since $|G| = 4! = 24$, it follows that $\chi_V$ is not irreducible.
\end{proof}

\subsection*{Part c}
Let $\chi^{(1)}$ denote the trivial character. By considering
$\chi_V-\chi^{(1)}$, show that $G$ has an irreducible character $\psi$ of degree
three.
\\
\\
\begin{proof}
    Let $\psi = \chi_V-\chi^{(1)}$. First, we calculate the values of $\psi$ on
    the conjugacy classes.
    \begin{center}
        \begin{tabular}{l| l l l l l}
            $S_4$       & $(1)$ & $(12)$ & $(123)$ & $(12)(34)$ & $(1234)$\\
            $|c_i|$     & $1$   & $6$    & $8$     & $3$        & $6$\\
            \hline
            $\chi^{(1)}$& $1$   & $1$    & $1$     & $1$        & $1$\\
            $\chi_V$    & $4$   & $2$    & $1$     & $0$        & $0$\\
            $\psi$      & $3$   & $1$    & $0$     & $-1$       & $-1$\\
        \end{tabular}
    \end{center}

    Now, $\psi$ is irreducible if and only if $\langle \psi,\psi\rangle  =1$.
    So, we calculate
    \[
        \begin{aligned}
            |G|\langle \psi, \psi\rangle &=
            \sum_{i=1}^5|c_i|\psi(g_i)\overline{\psi(g_i)}\\
            &= 1(3)(3) + 6(1)(1) + 8(0)(0) + 3(-1)(-1) + 6(-1)(-1)\\
            &= 9 + 6 + 3 + 6 = 24 = |G|
        \end{aligned}
    \]
    and so $\langle \psi,\psi\rangle = 1$ and $\psi$ is irreducible.

    Furthermore, since $\psi(1) = 3$, $\psi$ has degree three, as desired.
\end{proof}

\subsection*{Part d}
Determine the degrees of all the irreducible characters of $G$.
\\
\\
\begin{proof}
    By the normality condition on the first column of the character table for
    $G$, the sum of the squares of the degrees of the irreducible characters of
    $G$ must be equal to $|G|$. Furthermore, since there are $5$ conjugacy
    classes in $G$, there are $5$ irreducible characters on $G$. Let $n_i$ for
    $1\leq i\leq 5$ be the degrees of the characters.

    We know two characters already. The trivial character $\chi^{(1)}$ has
    degree $1$, so $n_1=1$. The character $\psi$ found in the last part (which
    we will call the second character) has degree $3$, so $n_2 = 3$. Thus,
    \[
        |G| = 24 = n_1^2 + n_2^2 + n_3^2 + n_4^2 + n_5^2 = 1 + 3^2 + n_3^2 + n_4^2 + n_5^2
    \]
    and so
    \[
        n_3^2 + n_4^2 + n_5^2 = 24-9-1 = 14
    \]
    The only integer solution to this is (up to relabeling)
    \[
        \begin{aligned}
            n_3 = 3\\
            n_4 = 2\\
            n_5 = 1
        \end{aligned}
    \]
    and so the degrees of all the irreducible characters of $G$ are
    \[
        \begin{aligned}
            n_1 = 1\\
            n_2 = 3\\
            n_3 = 3\\
            n_4 = 2\\
            n_5 = 1
        \end{aligned}
    \]
\end{proof}

\end{document}
