%%%%%%%%%%%%%%%%%%%%%%%%%%%%%%%%%%%%%%%%%
% Short Sectioned Assignment
% LaTeX Template
% Version 1.0 (5/5/12)
%
% This template has been downloaded from:
% http://www.LaTeXTemplates.com
%
% Original author:
% Frits Wenneker (http://www.howtotex.com)
%
% License:
% CC BY-NC-SA 3.0 (http://creativecommons.org/licenses/by-nc-sa/3.0/)
%
%%%%%%%%%%%%%%%%%%%%%%%%%%%%%%%%%%%%%%%%%

%----------------------------------------------------------------------------------------
%	PACKAGES AND OTHER DOCUMENT CONFIGURATIONS
%----------------------------------------------------------------------------------------

\documentclass[fontsize=11pt]{scrartcl} % 11pt font size

\usepackage[T1]{fontenc} % Use 8-bit encoding that has 256 glyphs
\usepackage[english]{babel} % English language/hyphenation
\usepackage{amsmath,amsfonts,amsthm} % Math packages
\usepackage{mathrsfs}

\usepackage[margin=1in]{geometry}

\usepackage{sectsty} % Allows customizing section commands
\allsectionsfont{\centering \normalfont\scshape} % Make all sections centered, the default font and small caps

\usepackage{fancyhdr} % Custom headers and footers
\pagestyle{fancyplain} % Makes all pages in the document conform to the custom headers and footers
\fancyhead{} % No page header - if you want one, create it in the same way as the footers below
\fancyfoot[L]{} % Empty left footer
\fancyfoot[C]{} % Empty center footer
\fancyfoot[R]{\thepage} % Page numbering for right footer
\renewcommand{\headrulewidth}{0pt} % Remove header underlines
\renewcommand{\footrulewidth}{0pt} % Remove footer underlines
\setlength{\headheight}{13.6pt} % Customize the height of the header

\numberwithin{equation}{section} % Number equations within sections (i.e. 1.1, 1.2, 2.1, 2.2 instead of 1, 2, 3, 4)
\numberwithin{figure}{section} % Number figures within sections (i.e. 1.1, 1.2, 2.1, 2.2 instead of 1, 2, 3, 4)
\numberwithin{table}{section} % Number tables within sections (i.e. 1.1, 1.2, 2.1, 2.2 instead of 1, 2, 3, 4)

\newcommand{\R}{\mathbb{R}}
\newcommand{\Q}{\mathbb{Q}}
\newcommand{\N}{\mathbb{N}}
\newcommand{\C}{\mathbb{C}}

\newtheorem{lemma}{Lemma}
%----------------------------------------------------------------------------------------
%	TITLE SECTION
%----------------------------------------------------------------------------------------

\newcommand{\horrule}[1]{\rule{\linewidth}{#1}} % Create horizontal rule command with 1 argument of height

\title{	
\normalfont \normalsize 
\textsc{Differential Topology} \\ [25pt] % Your university, school and/or department name(s)
\horrule{0.5pt} \\[0.4cm] % Thin top horizontal rule
\huge Homework 1 \\ % The assignment title
\horrule{2pt} \\[0.5cm] % Thick bottom horizontal rule
}

\author{Daniel Halmrast} % Your name

\date{\normalsize\today} % Today's date or a custom date

\begin{document}

\maketitle % Print the title

% Problems
\section*{Problem 1}
Let $f:X\to X'$ and $g:Y\to Y'$ be smooth maps. Prove that the composite map
$f\times g:X\times Y\to X'\times Y'$ is smooth.
\\
\\
\begin{proof}
    Recall the definition of a smooth map. $f:X\to Y$ is called smooth if for
    every chart $\phi$ on $X$ and $\psi$ on $Y$, the composite $\psi f\phi^{-1}$
    is smooth. Recall also that for manifolds $M,N$, the product manifold
    $M\times N$ is defined as the cartesian product $M\times N$ with the
    differentiable structure generated by products of charts on $M$ and $N$.

    With that aside, we proceed with the proof. Let $(x,y)\in X\times Y$. We
    will show that $f\times g$ is smooth at $(x,y)$. Let $\phi$ be a chart
    around $x\in X$, $\phi'$ a chart around $f(x)$, $\psi$ a chart around $y$,
    and $\psi'$ around $g(y)$. Then, $\phi\times \psi$ is a chart around
    $(x,y)$, and $\phi'\times \psi'$ is a chart around $(f\times g)(x,y)$.

    Now, the composite 
    \[
        (\phi'\times \psi')\circ (f\times g)\circ (\phi\times
        \psi)^{-1} = (\phi'\circ f\circ \phi^{-1})\times(\psi'\circ g\circ
        \psi^{-1})
    \]
    is the product of smooth functions on Euclidean space, which is trivially
    seen to be smooth. Thus, $f\times g$ is smooth at every point, as desired.
\end{proof}

\newpage

\section*{Problem 2}
Prove that the projection map $\pi_x:X\times Y\to X$ is smooth.
\\
\\
\begin{proof}
    Let $(x,y)\in X\times Y$, and let $\phi$ be a chart around $x\in X$, $\psi$
    a chart around $y\in Y$. Now, the composite
    \[
        \phi\circ\pi_x\circ(\phi\times \psi)^{-1}(\phi(x),\psi(y)) = \phi(x)
    \]
    is just the standard projection operator on Euclidean space, which we know
    to be smooth. Thus, $\pi_x$ is smooth at every point, as desired.
\end{proof}

\newpage

\section*{Problem 3}
Let $U\subset X$ be open. Prove that for all $p\in U$, $T_pU = T_pX$.
\\
\\
\begin{proof}
    Recall that for $M$ a differentiable manifold, $p\in M$, the tangent space
    $T_pM$ at $p$ is defined to be the span of 
\end{proof}<++>
\end{document}
