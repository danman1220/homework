%%%%%%%%%%%%%%%%%%%%%%%%%%%%%%%%%%%%%%%%%
% Short Sectioned Assignment
% LaTeX Template
% Version 1.0 (5/5/12)
%
% This template has been downloaded from:
% http://www.LaTeXTemplates.com
%
% Original author:
% Frits Wenneker (http://www.howtotex.com)
%
% License:
% CC BY-NC-SA 3.0 (http://creativecommons.org/licenses/by-nc-sa/3.0/)
%
%%%%%%%%%%%%%%%%%%%%%%%%%%%%%%%%%%%%%%%%%

%----------------------------------------------------------------------------------------
%	PACKAGES AND OTHER DOCUMENT CONFIGURATIONS
%----------------------------------------------------------------------------------------

\documentclass[fontsize=11pt]{scrartcl} % 11pt font size

\usepackage[T1]{fontenc} % Use 8-bit encoding that has 256 glyphs
\usepackage[english]{babel} % English language/hyphenation
\usepackage{amsmath,amsfonts,amsthm} % Math packages
\usepackage{mathrsfs}
\usepackage{bbm}

\usepackage[margin=1in]{geometry}

\usepackage{sectsty} % Allows customizing section commands
\allsectionsfont{\centering \normalfont\scshape} % Make all sections centered, the default font and small caps

\usepackage{fancyhdr} % Custom headers and footers
\pagestyle{fancyplain} % Makes all pages in the document conform to the custom headers and footers
\fancyhead{} % No page header - if you want one, create it in the same way as the footers below
\fancyfoot[L]{} % Empty left footer
\fancyfoot[C]{} % Empty center footer
\fancyfoot[R]{\thepage} % Page numbering for right footer
\renewcommand{\headrulewidth}{0pt} % Remove header underlines
\renewcommand{\footrulewidth}{0pt} % Remove footer underlines
\setlength{\headheight}{13.6pt} % Customize the height of the header

\numberwithin{equation}{section} % Number equations within sections (i.e. 1.1, 1.2, 2.1, 2.2 instead of 1, 2, 3, 4)
\numberwithin{figure}{section} % Number figures within sections (i.e. 1.1, 1.2, 2.1, 2.2 instead of 1, 2, 3, 4)
\numberwithin{table}{section} % Number tables within sections (i.e. 1.1, 1.2, 2.1, 2.2 instead of 1, 2, 3, 4)

\newcommand{\R}{\mathbb{R}}
\newcommand{\Q}{\mathbb{Q}}
\newcommand{\N}{\mathbb{N}}
\newcommand{\C}{\mathbb{C}}

\newtheorem{lemma}{Lemma}
%----------------------------------------------------------------------------------------
%	TITLE SECTION
%----------------------------------------------------------------------------------------

\newcommand{\horrule}[1]{\rule{\linewidth}{#1}} % Create horizontal rule command with 1 argument of height

\title{	
\normalfont \normalsize 
\textsc{Differential Topology} \\ [25pt] % Your university, school and/or department name(s)
\horrule{0.5pt} \\[0.4cm] % Thin top horizontal rule
\huge Homework 1 \\ % The assignment title
\horrule{2pt} \\[0.5cm] % Thick bottom horizontal rule
}

\author{Daniel Halmrast} % Your name

\date{\normalsize\today} % Today's date or a custom date

\begin{document}

\maketitle % Print the title

% Problems
\section*{Problem 1}
Let $f:X\to X'$ and $g:Y\to Y'$ be smooth maps. Prove that the composite map
$f\times g:X\times Y\to X'\times Y'$ is smooth.
\\
\\
\begin{proof}
    Recall the definition of a smooth map. $f:X\to Y$ is called smooth if for
    every chart $\phi$ on $X$ and $\psi$ on $Y$, the composite $\psi f\phi^{-1}$
    is smooth. Recall also that for manifolds $M,N$, the product manifold
    $M\times N$ is defined as the cartesian product $M\times N$ with the
    differentiable structure generated by products of charts on $M$ and $N$.

    With that aside, we proceed with the proof. Let $(x,y)\in X\times Y$. We
    will show that $f\times g$ is smooth at $(x,y)$. Let $\phi$ be a chart
    around $x\in X$, $\phi'$ a chart around $f(x)$, $\psi$ a chart around $y$,
    and $\psi'$ around $g(y)$. Then, $\phi\times \psi$ is a chart around
    $(x,y)$, and $\phi'\times \psi'$ is a chart around $(f\times g)(x,y)$.

    Now, the composite 
    \[
        (\phi'\times \psi')\circ (f\times g)\circ (\phi\times
        \psi)^{-1} = (\phi'\circ f\circ \phi^{-1})\times(\psi'\circ g\circ
        \psi^{-1})
    \]
    is the product of smooth functions on Euclidean space, which is trivially
    seen to be smooth. Thus, $f\times g$ is smooth at every point, as desired.
\end{proof}

\newpage

\section*{Problem 2}
Prove that the projection map $\pi_x:X\times Y\to X$ is smooth.
\\
\\
\begin{proof}
    Let $(x,y)\in X\times Y$, and let $\phi$ be a chart around $x\in X$, $\psi$
    a chart around $y\in Y$. Now, the composite
    \[
        \phi\circ\pi_x\circ(\phi\times \psi)^{-1}(\phi(x),\psi(y)) = \phi(x)
    \]
    is just the standard projection operator on Euclidean space, which we know
    to be smooth. Thus, $\pi_x$ is smooth at every point, as desired.
\end{proof}

\newpage

\section*{Problem 3}
Let $U\subset X$ be open. Prove that for all $p\in U$, $T_pU = T_pX$.
\\
\\
\begin{proof}
    First, I assert that $U$ has a manifold structure given by $\{(V\cap
    U,\phi)\ |\ (V,\phi) \text{a chart in $X$}\}$. This works because $V\cap U$
    is open, and thus $\phi|_{V\cap U}$ is a coordinate chart. Compatibility of
    the charts follows from the manifold structure on $X$ itself, which
    guarantees the charts are compatible.

    Let $p\in U$, and let $(V,\phi)$ be a chart around $p\in X$. Let's also
    require that $V\subset U$. We know that
    $T_pX$ is the image $d\phi^{-1}(\phi(V))$ of the derivative of $\phi^{-1}$
    on its domain. Furthermore, we know that $(V,\phi) = (V\cap U,\phi)$ is also
    a chart for $U$ around $p$. Thus, at $p$, $\phi$ works as both a chart on
    $X$ and a chart on $U$, and the tangent space (which is defined entirely
    with respect to the chart) must be the same. That is, $T_pU = T_pX$ as
    desired.
\end{proof}

\newpage

\section*{Problem 4}
Prove that if $f:X\to Y$ is a diffeomorphism, then $df_x$ is an isomorphism for
all $x\in X$.
\\
\\
\begin{proof}
    Recall that the differential is functorial. That is, $d(f\circ g) = df\circ
    dg$ and $d(\mathbbm{1}) = \mathbbm{1}$ (this follows from the chain rule).
    Then, as a consequence, we know that $d(f^{-1}) = (df)^{-1}$.

    Now, since $f$ is a diffeomorphism, it has an inverse $f^{-1}$ such that
    $f\circ f^{-1} = f^{-1}\circ f = \mathbbm{1}$.
    Thus, we know that $df_x$ has both a right and left inverse as $df^{-1}_x$.
    However, any linear map with both a left and right inverse is necessarily an
    isomorphism. Thus, $df_x$ is an isomorphism for all $x$, as desired.
\end{proof}

\newpage

\section*{Problem 5}
Show that $T_pX$ is the set of velocity vectors of curves through $p$.
\\
\\
\begin{proof}
    We first show that any $v\in T_pX$ is the velocity vector of some curve
    through $p$.

    To see this, let $v\in T_pX$, and choose a coordinate system $(U,\phi)$
    centered at $p$ such that $v = d\phi^{-1}(\partial_1)$ where $\partial_1$ is
    the first basis vector for $T_0\R^n$ (i.e. $\partial_1 = (1,0,\dots,0)$ if
    we identify $T_0\R^n$ with $\R^n$.)

    Then, consider the curve $\gamma:(-\varepsilon,\varepsilon)\to X$ defined as
    \[
        \gamma(t) = \phi^{-1}(t,0,\dots,0)
    \]
    This has derivative
    \[
        \begin{aligned}
        \gamma'(0) &= \partial_t\phi^{-1}(t,0,\dots,0)\\
        &= \partial_t (t)\partial_1 + \partial_t (0)\partial_2 + \dots +
        \partial_t(0)\partial_n\\
        &= \partial_1 = v
    \end{aligned}
    \]
    Thus, every $v\in T_pX$ is the derivative of some curve.

    Next, we show that every velocity vector is in the tangent space. This is
    clear, since if $\gamma:[-1,1]\to X$ is a curve with $\gamma(0)=p$, we can
    fix a coordinate system $(U,\phi)$ around $p$ with coordinate functions
    $x^i$, and calculate
    \[
        \gamma'(0) = \partial_t\gamma^i(t)|_0\partial_i\\
    \]
    where $\gamma^i(t)$ is $x^i(\gamma(t))$, and $\partial_i$ is the basis for
    $T_pX$ generated by the $x^i$ functions. Thus,
    $\gamma'(0)\in T_pX$ as desired.
\end{proof}
\newpage
\section*{Problem 6}
Prove that if $f:X\to Y$ is a submersion, and $U\subset X$ is open, then
$f(U)\subset Y$ is open.
\\
\\
\begin{proof}
    Suppose $y\in f(U)$. Now, for any $x\in U$ with $f(x) = y$, we can find a
    neighborhood $V$ of $y$ for which there is a smooth section $\sigma:V\to X$
    with $\sigma(y)=x$. Then, for each $z\in \sigma^{-1}(U)$, we have
    $z=f(\sigma(z))\in f(U)$. So, $\sigma^{-1}(U)$ is an open neighborhood of
    $y$ contained in $f(U)$. Thus, since we can do this for all $y\in f(U)$, we
    see that $f(U)$ is open, as desired.
\end{proof}

\end{document}
