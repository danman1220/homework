%%%%%%%%%%%%%%%%%%%%%%%%%%%%%%%%%%%%%%%%%
% Short Sectioned Assignment
% LaTeX Template
% Version 1.0 (5/5/12)
%
% This template has been downloaded from:
% http://www.LaTeXTemplates.com
%
% Original author:
% Frits Wenneker (http://www.howtotex.com)
%
% License:
% CC BY-NC-SA 3.0 (http://creativecommons.org/licenses/by-nc-sa/3.0/)
%
%%%%%%%%%%%%%%%%%%%%%%%%%%%%%%%%%%%%%%%%%

%----------------------------------------------------------------------------------------
%	PACKAGES AND OTHER DOCUMENT CONFIGURATIONS
%----------------------------------------------------------------------------------------

\documentclass[fontsize=11pt]{scrartcl} % 11pt font size

\usepackage[T1]{fontenc} % Use 8-bit encoding that has 256 glyphs
\usepackage[english]{babel} % English language/hyphenation
\usepackage{amsmath,amsfonts,amsthm} % Math packages
\usepackage{mathrsfs}
\usepackage{tikz-cd}

\usepackage[margin=1in]{geometry}

\usepackage{sectsty} % Allows customizing section commands
\allsectionsfont{\centering \normalfont\scshape} % Make all sections centered, the default font and small caps

\usepackage{fancyhdr} % Custom headers and footers
\pagestyle{fancyplain} % Makes all pages in the document conform to the custom headers and footers
\fancyhead{} % No page header - if you want one, create it in the same way as the footers below
\fancyfoot[L]{} % Empty left footer
\fancyfoot[C]{} % Empty center footer
\fancyfoot[R]{\thepage} % Page numbering for right footer
\renewcommand{\headrulewidth}{0pt} % Remove header underlines
\renewcommand{\footrulewidth}{0pt} % Remove footer underlines
\setlength{\headheight}{13.6pt} % Customize the height of the header

\numberwithin{equation}{section} % Number equations within sections (i.e. 1.1, 1.2, 2.1, 2.2 instead of 1, 2, 3, 4)
\numberwithin{figure}{section} % Number figures within sections (i.e. 1.1, 1.2, 2.1, 2.2 instead of 1, 2, 3, 4)
\numberwithin{table}{section} % Number tables within sections (i.e. 1.1, 1.2, 2.1, 2.2 instead of 1, 2, 3, 4)

\newcommand{\R}{\mathbb{R}}
\newcommand{\Q}{\mathbb{Q}}
\newcommand{\N}{\mathbb{N}}
\newcommand{\C}{\mathbb{C}}
\newcommand{\Z}{\mathbb{Z}}

\newcommand{\sgn}{\text{sgn}}

\newtheorem{lemma}{Lemma}
%----------------------------------------------------------------------------------------
%	TITLE SECTION
%----------------------------------------------------------------------------------------

\newcommand{\horrule}[1]{\rule{\linewidth}{#1}} % Create horizontal rule command with 1 argument of height

\title{	
\normalfont \normalsize 
\textsc{Topology} \\ [25pt] % Your university, school and/or department name(s)
\horrule{0.5pt} \\[0.4cm] % Thin top horizontal rule
\huge Final \\ % The assignment title
\horrule{2pt} \\[0.5cm] % Thick bottom horizontal rule
}

\author{Daniel Halmrast} % Your name

\date{\normalsize\today} % Today's date or a custom date

\begin{document}

\maketitle % Print the title

% Problems
\section*{Problem 1}
\subsection*{Part i}
Give the definition of degree for a smooth map $f:A\to B$ between closed
oriented manifolds of the same dimension.

Show that if $g:B\to C$ is another such map, then
\[
    \deg(g\circ f) = \deg(f)\deg(g)
\]

\begin{proof}
    We assume here that the manifolds $A$, $B$, and $C$ are all connected.

    Consider a regular value $y\in B$ of $f$. The inverse image $f^{-1}(\{y\})$
    is a finite set of points (since $A$ is compact, and $f^{-1}(\{y\})$ is of
    dimension zero). For each point $x\in f^{-1}(\{y\})$, we say the sign of
    $df_x$ at $x$ (denoted $\sgn(df_x)$) is $+1$ if $df_x$ preserves
    orientation, and $-1$ if $df_x$ reverses orientation. Then, the degree of
    $f$ is defined as the sum
    \[
        \deg(f) = \sum_{x\in f^{-1}(\{y\})}\sgn(df_x)
    \]
    Recall that this definition is well-defined, as it is independent of choice
    of regular value.

    Now, we turn our attention to the composition $g\circ f$. Recall that $y\in
    C$ is called a regular value of $g\circ f$ if for every $x\in (g\circ
    f)^{-1}(\{y\})$, the differential $d(g\circ f)_x$ is surjective. By
    Sard's theorem, the set of critical values for $g\circ f$ has measure zero,
    as well as the set of critical values for $g$, in $C$. Therefore, on any
    chart $(U,\phi)$ in $C$, there must exist a point which is regular for both
    $g$ and $g\circ f$. To see this, let $R$ denote the set of regular values of
    $g\circ f$ in $U$, and $R'$ the set of regular values of $g$ in $U$. If $R$
    and $R'$ were disjoint, then we would have
    \[
        \begin{aligned}
            \mu(R) &= \mu(U) = \mu(R') &\text{By Sard's Theorem}\\
            \mu(R\cup R') &= \mu(R)+\mu(R')=2\mu(U)>\mu(U)
        \end{aligned}
    \]
    which is a contradiction. From here on out, let $y\in C$ be a regular value
    of both $f\circ g$ and $g$.

    Now, we will show that for all $x\in (g\circ f)^{-1}(\{y\})$, $f(x)$ is a
    regular value for $f$. so, let $x$ be as specified.
     This means that the differential $d(g\circ f)_x
    = dg_{f(x)}\circ df_x$ is surjective. In particular, since the dimensions of
    $T_xA$ and $T_yC$ are equal, $d(g\circ f)_x$ is an isomorphism. Furthermore,
    since $T_{f(x)}B$ also has the same dimension, it must be that $df_x$ and
    $dg_{f(x)}$ are both isomorphisms as well. This follows from the fact that
    $dg_{f(x)}\circ df_x$ is surjective, so $dg_{f(x)}$ is surjective onto a
    space of the same dimension, and hence is an isomorphism. Similarly,
    $dg_{f(x)}\circ df_x$ is injective, so $df_x$ is injective into a space of
    the same dimension, and is thus an isomorphism. From all this, we conclude
    that $df_x$ is surjective for all $x\in (g\circ f)^{-1}(\{y\})$. In
    particular, since $f^{-1}(\{f(x)\})\subset (g\circ f)^{-1}(\{y\})$, we have
    that $df_x$ is surjective for all $x$ in the preimage of $f(x)$, and so
    $f(x)$ is a regular value of $f$.

    Finally, we show that $\deg(g\circ f) = \deg(g)\deg(f)$. Suppose $f$ is
    surjective. %TODO handle the non-surjective case
    Then, let $z\in C$ be a regular value of $g\circ f$ and $g$, as before.
    Now, let $P = (g\circ f)^{-1}(\{y\})$, and partition $P$ into disjoint sets
    $P_y = \{x\in P:\ f(x)=y\}$ of points which map to the same point by $f$.
    Then, we compute
    \[
        \begin{aligned}
            \deg(g\circ f) &= \sum_{x\in P}\sgn(d(g\circ f)_x)\\
            &=\sum_{y\in g^{-1}(\{z\})}\sum_{x\in P_y}\sgn(d(g\circ f)_x)\\
            &=\sum_{y\in g^{-1}(\{z\})}\sum_{x\in P_y}\sgn(dg_y\circ df_x)\\
        &=\sum_{y\in g^{-1}(\{z\})}\sum_{x\in P_y}\sgn(dg_y)\sgn(df_x)\\
        &=\sum_{y\in g^{-1}(\{z\})}\sgn(dg_y)\sum_{x\in P_y}\sgn(df_x)\\
        &=\sum_{y\in g^{-1}(\{z\})}\sgn(dg_y)\deg(f)\\
        &=\deg(g)\deg(f)
        \end{aligned}
    \]
    as desired. Here, we used the fact that each $y$ is a regular value for $f$,
    and $\deg(f)$ is independent of choice of $y$.
\end{proof}

\newpage

\subsection*{Part ii}
Suppose $f:A\to B$ as in part i has degree $\deg(f)=1$ and that the index
$[\pi_1(B):f_*\pi_1(A)]<\infty]$. Prove that $f_*:\pi_1(A)\to\pi_1(B)$ is
surjective.

\begin{proof}
    Let $(C,p)$ be the covering space associated to the subgroup $H =
    f_*(\pi_1(A))$ over $B$. That is, $p_*(\pi_1(C)) =H$. Since $H$
    has finite index, $(C,p)$ is a finite-sheeted covering space, and is thus
    compact. Furthermore, $C$ can be given a manifold structure from $B$
    via pulling the differentiable structure back across $p$, and such a
    structure will be orientable, since $B$ is orientable. 

    Since $f_*(\pi_1(A)) = p_*(\pi_1(C)) =H$, the criteria for lifting $f$ to
    $C$ is satisfied, and so we have that the diagram
    \[
        \begin{tikzcd}
            &C\arrow{d}{p}\\
            A\arrow{r}{f}\arrow[dashed]{ur}{\tilde{f}}&B
        \end{tikzcd}
    \]
    commutes. Thus, $f = p\circ \tilde{f}$.

    Now, by construction $p$ preserves orientation, and each point in $B$ is a
    regular value of $p$ with exactly $n$ points in its preimage, where
    $n=[\pi_1(B):H]$. Recall this comes from the fact that the number of sheets
    (sheetedness?) of $(C,P)$ is exactly the index of $p_*(\pi_1(C))=H$ in
    $\pi_1(B)$. Thus, it follows that $p$ has degree $n$.

    By the first part of this problem, we know that
    \[
        1 = \deg(f) = \deg(p\circ \tilde{f}) = \deg(p)\deg(\tilde{f})
    \]
    Since the degree of a map is always an integer, it follows that
    $\deg(p)=\deg(\tilde{f}) = 1$. Thus,
    \[
        [\pi_1(B):H]=1
    \]
    as desired.

\end{proof}

\newpage

\section*{Problem 2}
\subsection*{Part i}
Prove Euler's identity for homogeneous polynomials
\[
    \sum_j x_j\partial_jp = mp(x_1,\dots,x_n)
\]

\begin{proof}
    Since $p$ is homogeneous, we have that
    \[
        p(tx_1,\dots,tx_n) = t^mp(x_1,\dots,x_n)
    \]
    Now, if we differentiate the left-hand side with respect to $t$, we get
    \[
\begin{aligned}
    \partial_tp(tx_1,\dots,tx_n) &= \sum_i\frac{\partial tx_i}{\partial
    t}\frac{\partial p(tx_1,\dots,tx_n)}{\partial tx_i}\\
    &=\sum_ix_i\frac{\partial p(tx_1,\dots,tx_n)}{\partial tx_i}\\
\end{aligned}
    \]
    Differentiating the right-hand side yields
    \[
        \partial_t (t^mp(x_1,\dots,x_n)) = mt^{m-1}p(x_1,\dots,x_n)
    \]
    and equating the two and setting $t=1$ yields
    \[
        \sum_i x_i\partial_ip(x_1,\dots,x_n) = mp(x_1,\dots,x_n)
    \]
    as desired.
\end{proof}

\subsection*{Part ii}
Show that $SL(n,\R)$ is a Lie group.

\begin{proof}
    We know already that $SL(n,\R)$ is a subgroup of $GL(n,\R)$, so all we need
    to show is that it is a submanifold of $GL(n,\R)$. To do so, we will show it
    is the level set of a regular value of a smooth function. Specifically, we
    recall that $SL(n,\R)$ is defined to be the set of all matrices in
    $GL(n,\R)$ with determinant $1$.

    First, I claim the determinant map $\det:GL(n,\R)\to \R$ is homogeneous. To
    see this, recall that similar matrices have the same determinant, and that
    every matrix is similar to a matrix in Jordan form. So, without loss of
    generality, we will assume our matrices are in Jordan form. Recall that in
    Jordan form, the determinant is given by the product of the diagonal
    elements. Thus, for a matrix $A$,
    \[
        \det(tA) = \prod_{i=1}^n tA_{ii} = t^n\prod_{i=1}^n A_{ii} = t^n\det(A)
    \]
    and if we think of $A$ as a tuple of numbers, we see that the determinant
    map is a homogeneous polynomial.

    Now, we will show that $1$ is a regular value for this polynomial (call it
    $p$). For this, we have to show that the differential $dp_x$ at each point
    $x\in p^{-1}(\{1\})$ is surjective. Since $\R$ is one-dimensional, this
    amounts to showing that there is some $v\in T_x(GL(n,\R))$ for which
    $dp_x(v)\neq 0$.

    So, let $x\in p^{-1}(\{1\})$. Since $GL(n,\R)$ has its differential
    structure from $\R^{n^2}$, $T_x(GL(n,\R))$ can be identified with $M(n,\R)$
    the set of all real $n\times n$ matrices. In particular, $x\in
    T_x(GL(n,\R))$. So, we calculate
    \[
\begin{aligned}
    dp_x(x) = x^a(dp_x)_a = x^a\partial_ap|_x = mp(x) = m
\end{aligned}
    \]
    by using Euler's identity from part i, and the fact that $p(x)=1$. Thus,
    $dp_x(x)$ is nontrivial, and so $dp_x$ is surjective for all $x\in
    p^{-1}(\{1\})$. Since $SL(n,\R)=p^{-1}(\{1\})$, we see that $SL(n,\R)$ is a
    submanifold of $GL(n,\R)$ and is thus a Lie subgroup of $GL(n,\R)$ as
    desired.
\end{proof}

\newpage

\section*{Problem 3}
\subsection*{Part i}
Suppose $M$ is simply-connected. Prove it is orientable.
\\
\\
\begin{proof}
    Suppose for a contradiction that $M$ were not orientable. Then, $M$ would
    have an oriented double cover $(\tilde{M},\tilde{p})$.
    However, since $M$ is simply-connected, $M$ is its own universal cover,
    which would imply that $M$ covers $\tilde{M}$ in such a way that the diagram
    \[
\begin{tikzcd}
    M\arrow{r}{p}\arrow[bend right]{rr}{id} &\tilde{M}\arrow{r}{\tilde{p}} &M
\end{tikzcd}
    \]
    commutes. However, this cannot be, since $\tilde{p}$ is a two-sheeted cover
    of $M$, and is thus not invertible, so $\tilde{p}\circ p$ cannot be the
    identity. Thus, $M$ is orientable.
\end{proof}

\subsection*{Part ii}
Prove that every Lie group is orientable.

\begin{proof}
    Let $G$ be a Lie group, with identity $e$. Let $(U,\phi)$ be a coordinate
    chart around $e$. We define an oriented chart on $G$ as follows. Recall that
    left multiplication is a smooth diffeomorphism of $G$ to itself, and so for
    each $g\in G$, we let $(L_g(U),\phi\circ L_g^{-1})$ be a chart around $g$.
    We check that these charts are compatible. Suppose $g$ and $h$ are elements
    so that $L_g(U)\cap L_h(U)$ is nonempty. Then, the transition map for the
    intersection is given as
    \[
        \begin{aligned}
            (\phi\circ L_g^{-1})\circ(\phi\circ L_h^{-1})^{-1}&:U\subset\R^n\to \R^n\\
            (\phi\circ L_g^{-1})\circ(\phi\circ L_h^{-1})^{-1}
            &= (\phi\circ L_g^{-1})\circ(L_h\circ \phi^{-1})\\
            &=\phi\circ L_{g^{-1}h}\circ\phi^{-1}
        \end{aligned}
    \]
    which is smooth, since $L_{g^{-1}h}$ is a smooth diffeomorphism. Finally, I
    assert that this atlas is oriented. We now calculate the differential of the
    change of coordinates from $L_g(U)$ to $L_h(U)$ for $x\in L_g(U)\cap
    L_h(U)$.

    Let $x_i$ be the coordinate functions $x_i:U\to \R$ around the identity.
    Then, $x_i\circ L_g^{-1}$ are the coordinate functions for $L_g(U)$ around
    $g$. In particular, $\partial_i$ forms a basis for $T_uG$ for $u\in U$, and
    $dL_g(\partial_i)$ forms a basis for $T_pG$ for $p\in L_g(U)$. Thus, we
    calculate in coordinates
    \[
        \begin{aligned}
        d((\phi\circ L_h^{-1})^{-1}\circ (\phi\circ
        L_g^{-1}))_x(v)
        &= d(L_h)_{L_{g^{-1}(x)}}\circ d(L_g^{-1})_x(v^idL_g\partial_i)\\
        &=v^idL_h\partial_i = v
        \end{aligned}
    \]
    and thus the change-of-coordinates map is the identity, which has positive
    determinant. Thus, this atlas is indeed an orientation on $G$.

\end{proof}

\subsection*{Part iii}
Show that every Lie group has Euler characteristic zero.

\begin{proof}
    To show this, we will show that the identity function on $G$ has Lefshetz
    number zero, which implies automatically that $G$ has Euler characteristic
    zero.

    To see this, we will construct a smooth function on $G$ with no fixed
    points that is homotopic to the identity. Consider a nonzero vector $v\in
    \mathfrak{g}$. This generates a nowhere-vanishing vector field $V$ by
    \[
        V(p) = d(L_p)_e(v)
    \]
    In particular, this vector field generates a flow
    \[
\begin{aligned}
    \theta:(-\varepsilon,\varepsilon)\times G\to G\\
\end{aligned}
    \]
    which has no fixed points in positive time, since the vector field it was
    generated from did not vanish anywhere. Now, if we vary
    $t\in(-\varepsilon,\varepsilon)$, we get a homotopy from $\theta(0,\cdot) =
    id$ to $\theta(\frac{\varepsilon}{2},\cdot)$, which is the map with no fixed
    points.

    Thus, the Lefschetz number for $id$ is the Lefschetz number for
    $\theta_{\frac{\varepsilon}{2}}$, which, since
    $\theta_{\frac{\varepsilon}{2}}$ has no fixed points, is zero. Thus, since
    the Lefschetz number of the identity is equal to the Euler characteristic of
    $G$, it follows that $\chi(G)=0$ as desired.
\end{proof}

\newpage

\section*{Problem 4}
Give a sketch of the construction of Pontraygin manifolds. State the three main
theorems and give an outline of the proof of one of them.

\begin{proof}
    Let $M$ be a manifold (smooth, compact, orientable), and $f$ a
    smooth map from $M$ to $S^p$. Choose a regular value $y\in S^p$ for $f$, and
    define the Pontraygin manifold associated to $f$ as the manifold
    $f^{-1}(\{y\})$. We define a framing on this manifold as follows. Let
    $\{v_i\}$ be a basis for $T_yS^p$. Then, the pullback of this basis along
    $f$ defines a trivialization of the normal bundle to $f^{-1}(\{y\})$ (since
        the tangent bundle to $f^{-1}(\{y\})$ is the nullspace of $df$, the
    normal bundle gets mapped isomorphically into $T_yS^p$).
    
    The first main result is that the Pontraygin manifolds, in a sense, do not
    depend on choice of regular value. That is, if $y,z$ are regular values for
    $f$ and $N_1$ is the Pontraygin
    manifold from $f$ at $y$, $N_2$ the Pontraygin manifold from $f$ at $z$,
    then $N_1$ is framed cobordant to $N_2$. 

    The second main result is that if two Pontraygin manifolds $f^{-1}(\{y\})$
    and $g^{-1}(\{y\})$ are equal, then $f$ is homotopic to $g$.

    I offer a proof sketch of the first result. Let $y,z$ be regular values for
    $f$ on $S^p$, and let $R_t$ be a one-parameter family of rotations with
    $R_0=id$ and $R_1$ taking $y$ to $z$. Now, consider the map
    \[
        H(t,x) = R_t\circ f(x)
    \]
    which has $H(0,x)=f(x)$ and for $p$ such that $f(p) = y$, $H(1,p) = z$. This
    defines a framed cobordism via $H^{-1}(y)$ from $f^{-1}(y)$ to $(R_1\circ
    f)^{-1}(y) = f^{-1}(z)$ as desired.

\end{proof}



\end{document}
