%%%%%%%%%%%%%%%%%%%%%%%%%%%%%%%%%%%%%%%%%
% Short Sectioned Assignment
% LaTeX Template
% Version 1.0 (5/5/12)
%
% This template has been downloaded from:
% http://www.LaTeXTemplates.com
%
% Original author:
% Frits Wenneker (http://www.howtotex.com)
%
% License:
% CC BY-NC-SA 3.0 (http://creativecommons.org/licenses/by-nc-sa/3.0/)
%
%%%%%%%%%%%%%%%%%%%%%%%%%%%%%%%%%%%%%%%%%

%----------------------------------------------------------------------------------------
%	PACKAGES AND OTHER DOCUMENT CONFIGURATIONS
%----------------------------------------------------------------------------------------

\documentclass[fontsize=11pt]{scrartcl} % 11pt font size

\usepackage[T1]{fontenc} % Use 8-bit encoding that has 256 glyphs
\usepackage[english]{babel} % English language/hyphenation
\usepackage{amsmath,amsfonts,amsthm} % Math packages
\usepackage{mathrsfs}

\usepackage[margin=1in]{geometry}

\usepackage{sectsty} % Allows customizing section commands
\allsectionsfont{\centering \normalfont\scshape} % Make all sections centered, the default font and small caps

\usepackage{fancyhdr} % Custom headers and footers
\pagestyle{fancyplain} % Makes all pages in the document conform to the custom headers and footers
\fancyhead{} % No page header - if you want one, create it in the same way as the footers below
\fancyfoot[L]{} % Empty left footer
\fancyfoot[C]{} % Empty center footer
\fancyfoot[R]{\thepage} % Page numbering for right footer
\renewcommand{\headrulewidth}{0pt} % Remove header underlines
\renewcommand{\footrulewidth}{0pt} % Remove footer underlines
\setlength{\headheight}{13.6pt} % Customize the height of the header

\numberwithin{equation}{section} % Number equations within sections (i.e. 1.1, 1.2, 2.1, 2.2 instead of 1, 2, 3, 4)
\numberwithin{figure}{section} % Number figures within sections (i.e. 1.1, 1.2, 2.1, 2.2 instead of 1, 2, 3, 4)
\numberwithin{table}{section} % Number tables within sections (i.e. 1.1, 1.2, 2.1, 2.2 instead of 1, 2, 3, 4)

\newcommand{\R}{\mathbb{R}}
\newcommand{\Q}{\mathbb{Q}}
\newcommand{\N}{\mathbb{N}}
\newcommand{\C}{\mathbb{C}}
\newcommand{\Z}{\mathbb{Z}}

\newcommand{\sgn}{\text{sgn}}

\newtheorem{lemma}{Lemma}
%----------------------------------------------------------------------------------------
%	TITLE SECTION
%----------------------------------------------------------------------------------------

\newcommand{\horrule}[1]{\rule{\linewidth}{#1}} % Create horizontal rule command with 1 argument of height

\title{	
\normalfont \normalsize 
\textsc{Topology} \\ [25pt] % Your university, school and/or department name(s)
\horrule{0.5pt} \\[0.4cm] % Thin top horizontal rule
\huge Final \\ % The assignment title
\horrule{2pt} \\[0.5cm] % Thick bottom horizontal rule
}

\author{Daniel Halmrast} % Your name

\date{\normalsize\today} % Today's date or a custom date

\begin{document}

\maketitle % Print the title

% Problems
\section*{Problem 1}
\subsection*{Part i}
Give the definition of degree for a smooth map $f:A\to B$ between closed
oriented manifolds of the same dimension.

Show that if $g:B\to C$ is another such map, then
\[
    \deg(g\circ f) = \deg(f)\deg(g)
\]

\begin{proof}
    We assume here that the manifolds $A$, $B$, and $C$ are all connected.

    Consider a regular value $y\in B$ of $f$. The inverse image $f^{-1}(\{y\})$
    is a finite set of points (since $A$ is compact, and $f^{-1}(\{y\})$ is of
    dimension zero). For each point $x\in f^{-1}(\{y\})$, we say the sign of
    $df_x$ at $x$ (denoted $\sgn(df_x)$) is $+1$ if $df_x$ preserves
    orientation, and $-1$ if $df_x$ reverses orientation. Then, the degree of
    $f$ is defined as the sum
    \[
        \deg(f) = \sum_{x\in f^{-1}(\{y\})}\sgn(df_x)
    \]
    Recall that this definition is well-defined, as it is independent of choice
    of regular value.

    Now, we turn our attention to the composition $g\circ f$. Recall that $y\in
    C$ is called a regular value of $g\circ f$ if for every $x\in (g\circ
    f)^{-1}(\{y\})$, the differential $d(g\circ f)_x$ is surjective. By
    Sard's theorem, the set of critical values for $g\circ f$ has measure zero,
    as well as the set of critical values for $g$, in $C$. Therefore, on any
    chart $(U,\phi)$ in $C$, there must exist a point which is regular for both
    $g$ and $g\circ f$. To see this, let $R$ denote the set of regular values of
    $g\circ f$ in $U$, and $R'$ the set of regular values of $g$ in $U$. If $R$
    and $R'$ were disjoint, then we would have
    \[
        \begin{aligned}
            \mu(R) &= \mu(U) = \mu(R') &\text{By Sard's Theorem}\\
            \mu(R\cup R') &= \mu(R)+\mu(R')=2\mu(U)>\mu(U)
        \end{aligned}
    \]
    which is a contradiction. From here on out, let $y\in C$ be a regular value
    of both $f\circ g$ and $g$.

    Now, we will show that for all $x\in (g\circ f)^{-1}(\{y\})$, $f(x)$ is a
    regular value for $f$. so, let $x$ be as specified.
     This means that the differential $d(g\circ f)_x
    = dg_{f(x)}\circ df_x$ is surjective. In particular, since the dimensions of
    $T_xA$ and $T_yC$ are equal, $d(g\circ f)_x$ is an isomorphism. Furthermore,
    since $T_{f(x)}B$ also has the same dimension, it must be that $df_x$ and
    $dg_{f(x)}$ are both isomorphisms as well. This follows from the fact that
    $dg_{f(x)}\circ df_x$ is surjective, so $dg_{f(x)}$ is surjective onto a
    space of the same dimension, and hence is an isomorphism. Similarly,
    $dg_{f(x)}\circ df_x$ is injective, so $df_x$ is injective into a space of
    the same dimension, and is thus an isomorphism. From all this, we conclude
    that $df_x$ is surjective for all $x\in (g\circ f)^{-1}(\{y\})$. In
    particular, since $f^{-1}(\{f(x)\})\subset (g\circ f)^{-1}(\{y\})$, we have
    that $df_x$ is surjective for all $x$ in the preimage of $f(x)$, and so
    $f(x)$ is a regular value of $f$.

    Finally, we show that $\deg(g\circ f) = \deg(g)\deg(f)$. Trivially, if $f$
    is not surjective, then there exists a point $y\in B$ with $f^{-1}(\{y\}) =
    \emptyset$, and so $y$ is trivially a regular value, and $\deg(f)=0$.
    Furthermore, if $f$ is not surjective, then $g\circ f$ cannot be surjective
    either, 

\end{proof}<++>

\end{document}
