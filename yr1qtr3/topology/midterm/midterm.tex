%%%%%%%%%%%%%%%%%%%%%%%%%%%%%%%%%%%%%%%%%
% Short Sectioned Assignment
% LaTeX Template
% Version 1.0 (5/5/12)
%
% This template has been downloaded from:
% http://www.LaTeXTemplates.com
%
% Original author:
% Frits Wenneker (http://www.howtotex.com)
%
% License:
% CC BY-NC-SA 3.0 (http://creativecommons.org/licenses/by-nc-sa/3.0/)
%
%%%%%%%%%%%%%%%%%%%%%%%%%%%%%%%%%%%%%%%%%

%----------------------------------------------------------------------------------------
%	PACKAGES AND OTHER DOCUMENT CONFIGURATIONS
%----------------------------------------------------------------------------------------

\documentclass[fontsize=11pt]{scrartcl} % 11pt font size

\usepackage[T1]{fontenc} % Use 8-bit encoding that has 256 glyphs
\usepackage[english]{babel} % English language/hyphenation
\usepackage{amsmath,amsfonts,amsthm} % Math packages
\usepackage{mathrsfs}

\usepackage[margin=1in]{geometry}

\usepackage{sectsty} % Allows customizing section commands
\allsectionsfont{\centering \normalfont\scshape} % Make all sections centered, the default font and small caps

\usepackage{fancyhdr} % Custom headers and footers
\pagestyle{fancyplain} % Makes all pages in the document conform to the custom headers and footers
\fancyhead{} % No page header - if you want one, create it in the same way as the footers below
\fancyfoot[L]{} % Empty left footer
\fancyfoot[C]{} % Empty center footer
\fancyfoot[R]{\thepage} % Page numbering for right footer
\renewcommand{\headrulewidth}{0pt} % Remove header underlines
\renewcommand{\footrulewidth}{0pt} % Remove footer underlines
\setlength{\headheight}{13.6pt} % Customize the height of the header

\numberwithin{equation}{section} % Number equations within sections (i.e. 1.1, 1.2, 2.1, 2.2 instead of 1, 2, 3, 4)
\numberwithin{figure}{section} % Number figures within sections (i.e. 1.1, 1.2, 2.1, 2.2 instead of 1, 2, 3, 4)
\numberwithin{table}{section} % Number tables within sections (i.e. 1.1, 1.2, 2.1, 2.2 instead of 1, 2, 3, 4)

\newcommand{\R}{\mathbb{R}}
\newcommand{\Q}{\mathbb{Q}}
\newcommand{\N}{\mathbb{N}}
\newcommand{\C}{\mathbb{C}}
\newcommand{\Z}{\mathbb{Z}}

\newtheorem{lemma}{Lemma}
%----------------------------------------------------------------------------------------
%	TITLE SECTION
%----------------------------------------------------------------------------------------

\newcommand{\horrule}[1]{\rule{\linewidth}{#1}} % Create horizontal rule command with 1 argument of height

\title{	
\normalfont \normalsize 
\textsc{Topology} \\ [25pt] % Your university, school and/or department name(s)
\horrule{0.5pt} \\[0.4cm] % Thin top horizontal rule
\huge Midterm \\ % The assignment title
\horrule{2pt} \\[0.5cm] % Thick bottom horizontal rule
}

\author{Daniel Halmrast} % Your name

\date{\normalsize\today} % Today's date or a custom date

\begin{document}

\maketitle % Print the title

% Problems
\section*{Problem 1}
Define carefully what it means for a map $f:X\to Y$ to be transverse to a
submanifold $Z$ of $Y$.

Suppose that $X$ and $Z$ are smooth and transverse submanifolds of $Y$. Prove
that if $y\in X\cap Z$ then
\[
    T_y(X\cap Z) = T_y(X)\cap T_y(Z)
\]

\begin{proof}
    We begin by defining transversality. Suppose $f:X\to Y$ is a smooth map, and
    $Z\subset Y$ a submanifold of $Y$. We say that $f$ is transverse to $Z$ if
    \[
        T_{f(x)}(Y) = T_f(x)(Z) + df_x(T_x(X))
    \]
    for all $x\in f^{-1}(Z)$. That is, the tangent space at $f(x)$ in $Y$ is
    spanned by the tangent space of $Z$ and the push-forward of the tangent
    space of $X$.
    \\
    \\
    Now, suppose $X$ and $Z$ are smooth and transverse submanifolds of $Y$. That
    is, $T_y(Y) = T_y(X)+T_y(Z)$ for all $y\in X\cap Z$. Suppose first that
    $v\in T_y(X\cap Z)$. In particular, this means there is a curve
    $\gamma_v:I\to X\cap Z$ with $\gamma_v(0)=y$ and $\gamma_v'(0)=v$. Clearly,
    $\gamma_v$ is also a curve in $X$ and in $Z$, and so $v\in T_y(X)$ and $v\in
    T_y(Z)$ as desired. Thus, $T_y(X\cap Z)\subset T_y(X)\cap T_y(Z)$. Call this
    inclusion $\Phi$. Clearly, $\Phi$ is injective. Thus, all we need to show is
    that $\dim(T_y(X)\cap T_y(Z)) = \dim(T_y(X\cap Z))$ to establish equality.

    So, let $(U,\phi)$ be a slice chart of $Z$ at $y$. That is, $U$ is a
    neighborhood of $Y$, and $\phi:U\to\R^n$ is a coordinate chart such that
    $\phi(Z)\subset \R^k\times \{0\}^{n-k}$. In particular, we consider the
    augmented ``height'' function $\psi:U\to \R^{n-k}$ for which $\psi(Z) =
    \{0\}$. Thus, $Z=\psi^{-1}(\{0\})$. Let $i$ be the inclusion of $X$ into
    $Y$, and observe that $X\cap Z = (\psi\circ i)^{-1}(\{0\})$. We will show
    that $\{0\}$ is a regular value for $\psi\circ i$.

    To that end, we wish to show that $d(\psi\circ i)_y$ is surjective. So, let
    $v\in\R^{n-k}$. Now, since $\psi$ is part of a coordinate chart, $d\psi_y$ is
    surjective, and its kernel is $T_y(Z)$. Write a generic element of the fiber
    of $v$ as $w + v_z$ for $v_z\in T_y(Z)$. Since this is in $Y$, and $X$ and
    $Z$ are transverse, $w+v_z = v_x+v_z'$ for some $v_x\in T_y(X)$ and $v_z'\in
    T_y(Z)$. Absorbing $v_z'$ into $v_z$, we see that $w+v_z = v_x$. So,
    thinking of $v_x$ as an element of $T_y(X)$, we see that
    \[
        d(\psi\circ i)_y(v_x) = d\psi_{i(y)}\circ di_y(v_x) = d\psi_y(v_x) = v
    \]
    and so $d(\psi\circ i)$ is surjective as desired.

    Thus, the codimension of $X\cap Z$ in $X$ is $n-k$, which is the codimension
    of $Z$ in $Y$. That is,
    \[
        \dim(T_y(X)) - \dim(T_y(X\cap Z)) = \dim(T_y(Y))-\dim(T_y(Z))
    \]
    or
    \[
        \dim(T_y(X\cap Z)) = \dim(T_y(X)) + \dim(T_y(Z))-\dim(T_y(Y))
    \]

    However, since $T_y(Y) = T_y(X) + T_y(Z)$, we know that
    \[
    \dim(T_y(Y)) = \dim(T_y(X)) + \dim(T_y(Z))-\dim(T_y(X)\cap T_y(Z))
    \]
    or
    \[
        \dim(T_y(X)\cap T_y(Z)) = \dim(T_y(X)) + \dim (T_y(Z)) - \dim(T_y(Y))
    \]
    and so
    \[
        \dim(T_y(X\cap Z)) = \dim(T_y(X)\cap T_y(Z))
    \]
    and since the inclusion $\Phi:T_y(X\cap Z)\to T_y(X)\cap T_y(Z)$ is
    injective, the spaces are equal, as desired.
\end{proof}

\newpage

\section*{Problem 2}
Suppose $f:X\to Y$ is a smooth map between compact manifolds of the same
dimension. Suppose $y\in Y$ is a regular value of $f$.

\subsection*{Part i}
Prove that $f^{-1}(\{y\})$ is a finite set.
\\
\\
\begin{proof}
    Since $y$ is a regular value, we know that $f^{-1}(\{y\})$ is a submanifold
    of $X$ with dimension
    \[
        \dim(f^{-1}(\{y\}) = \dim(X)-\dim(Y) = 0
    \]
    Since the only manifolds of dimension zero are countable discrete sets,
    $f^{-1}(\{y\})$ is an (at most) countable collection of points with the
    discrete topology. Since $Y$ is compact, this automatically implies that
    $f^{-1}(\{y\})$ is finite. This follows from the fact that every infinite
    set in a compact space has an accumulation point, and discrete sets have no
    accumulation points.
\end{proof}

\subsection*{Part ii}
%TODO this problem sucks

\newpage

\section*{Problem 3}
Show that the set of rank 1 matrices in $M(2,\R)$ is a 3-dimensional submanifold
of $M(2,\R)$.
\\
\\
\begin{proof}
    A $2\times 2$ rank-1 matrix is a nonzero matrix with nontrivial kernel. This
    set is exactly specified as the set of all nonzero $2\times 2$ matrices with
    determinant zero. That is, letting $R$ denote the set of rank-1 matrices,
    \[
        R = {\det}^{-1}(0)\setminus \{0\}
    \]
    In particular, since $M(2,\R)\setminus\{0\}$ is an open subset of $M(2,\R)$,
    it is a manifold, and so we only need to consider ${\det}^{-1}(\{0\})$ in
    $M(2,\R)\setminus\{0\}$. To show this is a manifold, we will show it is a
    submanifold of $M(2,\R)\setminus\{0\}$ by showing $0$ is a regular value of
    $\det$.

    To show this, we need to show that for every nonzero matrix
    $A\in{\det}^{-1}(\{0\})$, $d(\det)_A$ is surjective. Since the codomain of
    $\det$ is $\R$, it suffices to show that there is at least one vector in
    $T_A(R(2,\R)\setminus\{0\})$ which does not map to zero.

    Observe first that $A$ is always of the form
    \[
        A =
        \begin{bmatrix}
            a &\lambda a\\
            b &\lambda b
        \end{bmatrix}
    \]
    for real numbers $a,b,\lambda$ such that they are not all identically zero
    (I suppose this is up to a similarity transformation, but the determinant is
    invariant under similarity transformations, so we won't worry about it).

    Let $\gamma(t) = A +tI$ so that $\gamma(0)=A$ and $\gamma'(0)=I$. We will
    show that $d(\det)_A(I) \neq 0$. We calculate
    \[
        \begin{aligned}
            d(\det)_A(I) &= \partial_t(\det(\gamma(t)))|_0\\
            &=\partial_t(\det(A+tI))|_0\\
            &=\partial_t\left(\det\left(
                    \begin{bmatrix}
                        a &\lambda a\\
                        b &\lambda b
                    \end{bmatrix}
                    + t
                    \begin{bmatrix}
                        1&0\\
                        0&1
            \end{bmatrix}\right)\right)|_0\\
            &=\partial_t\left( (a+t)(\lambda b+t)-\lambda ab \right)|_0\\
            &=\partial_t\left( t^2 + at + \lambda bt + \lambda ab - \lambda ab
            \right)|_0\\
            &= a+\lambda b
        \end{aligned}
    \]
    Thus for all $A = \begin{bmatrix}a&\lambda a\\b&\lambda b\end{bmatrix}$ with
    $a\neq -\lambda b$, $d(\det)_A(I)\neq 0$.

    Defining $B=\begin{bmatrix}-1&0\\0&1\end{bmatrix}$, and repeating the
    calculation for $d(\det)_A(B)$, we see
    that $d(\det)_A(B) = a - \lambda b$ which is nonzero for $a\neq \lambda b$.

    Finally, defining $C = \begin{bmatrix}0&1\\1&0\end{bmatrix}$, we calculate
    $d(\det)_A(C) = -b-\lambda a$, which is nonzero when $b\neq -\lambda a$.
    This exhausts all possible forms for $A$, and so the determinant is
    surjective at each point $A\in {\det}^{-1}(\{0\})$, as desired.

    In particular, this means that the codimension of $R$ is $1$, making it a 3
    dimensional submanifold of $M(2,\R)$ as desired.
\end{proof}

\end{document}
