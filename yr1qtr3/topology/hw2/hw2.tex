%%%%%%%%%%%%%%%%%%%%%%%%%%%%%%%%%%%%%%%%%
% Short Sectioned Assignment
% LaTeX Template
% Version 1.0 (5/5/12)
%
% This template has been downloaded from:
% http://www.LaTeXTemplates.com
%
% Original author:
% Frits Wenneker (http://www.howtotex.com)
%
% License:
% CC BY-NC-SA 3.0 (http://creativecommons.org/licenses/by-nc-sa/3.0/)
%
%%%%%%%%%%%%%%%%%%%%%%%%%%%%%%%%%%%%%%%%%

%----------------------------------------------------------------------------------------
%	PACKAGES AND OTHER DOCUMENT CONFIGURATIONS
%----------------------------------------------------------------------------------------

\documentclass[fontsize=11pt]{scrartcl} % 11pt font size

\usepackage[T1]{fontenc} % Use 8-bit encoding that has 256 glyphs
\usepackage[english]{babel} % English language/hyphenation
\usepackage{amsmath,amsfonts,amsthm} % Math packages
\usepackage{mathrsfs}

\usepackage[margin=1in]{geometry}

\usepackage{sectsty} % Allows customizing section commands
\allsectionsfont{\centering \normalfont\scshape} % Make all sections centered, the default font and small caps

\usepackage{fancyhdr} % Custom headers and footers
\pagestyle{fancyplain} % Makes all pages in the document conform to the custom headers and footers
\fancyhead{} % No page header - if you want one, create it in the same way as the footers below
\fancyfoot[L]{} % Empty left footer
\fancyfoot[C]{} % Empty center footer
\fancyfoot[R]{\thepage} % Page numbering for right footer
\renewcommand{\headrulewidth}{0pt} % Remove header underlines
\renewcommand{\footrulewidth}{0pt} % Remove footer underlines
\setlength{\headheight}{13.6pt} % Customize the height of the header

\numberwithin{equation}{section} % Number equations within sections (i.e. 1.1, 1.2, 2.1, 2.2 instead of 1, 2, 3, 4)
\numberwithin{figure}{section} % Number figures within sections (i.e. 1.1, 1.2, 2.1, 2.2 instead of 1, 2, 3, 4)
\numberwithin{table}{section} % Number tables within sections (i.e. 1.1, 1.2, 2.1, 2.2 instead of 1, 2, 3, 4)

\newcommand{\R}{\mathbb{R}}
\newcommand{\Q}{\mathbb{Q}}
\newcommand{\N}{\mathbb{N}}
\newcommand{\C}{\mathbb{C}}

\newcommand{\Diff}{\text{Diff}}

\newtheorem{lemma}{Lemma}
%----------------------------------------------------------------------------------------
%	TITLE SECTION
%----------------------------------------------------------------------------------------

\newcommand{\horrule}[1]{\rule{\linewidth}{#1}} % Create horizontal rule command with 1 argument of height

\title{	
\normalfont \normalsize 
\textsc{Topology} \\ [25pt] % Your university, school and/or department name(s)
\horrule{0.5pt} \\[0.4cm] % Thin top horizontal rule
\huge Homework 2 \\ % The assignment title
\horrule{2pt} \\[0.5cm] % Thick bottom horizontal rule
}

\author{Daniel Halmrast} % Your name

\date{\normalsize\today} % Today's date or a custom date

\begin{document}

\maketitle % Print the title

% Problems
\section*{Problem 1}
Suppose $f:X\to Y$ is a submersion. Prove that if $X$ is compact and $Y$ is
connected, then $f$ is surjective.
\\
\\
\begin{proof}
    Recall from the earlier homework that $f$ is an open map. Thus, the image
    $f(X)$ is open. Furthermore, since $X$ is compact, $f(X)$ is compact as
    well. Since $Y$ is Hausdorff, $f(X)$ is closed, and so $f(X)$ is a nonempty
    clopen set. Since $Y$ is connected, $f(X)=Y$ as desired.
\end{proof}<++>

\section*{Problem 2}
\subsection*{Part a}
Calculate the Lie algebra of $SO(n)$.
\\
\\
\begin{proof}
    Consider the smooth function
    \[
        \begin{aligned}
            \Phi:GL(n)\to GL(n)\\
            \Phi(A) = AA^T
        \end{aligned}
    \]
    Now, $O(n)$ is defined to be $\Phi^{-1}(I)$. We calculate the differential
    directly. Let $B\in T_A(O(n))$, and let $\gamma:[0,1]\to O(n)$ be
    \[
        \gamma(t) = A+tB
    \]
    Then,
    \[
        \begin{aligned}
            d\Phi_A(B) &= d\Phi_A(\gamma'(0))\\
            &=\partial_t(\Phi(\gamma(t)))|_0\\
            &=\partial_t((A+tB)(A+tB)^T)|_0\\
            &=\partial_t(AA^T + tAB^T + tBA^T + t^2BB^T)|_0\\
            &=AB^T + BA^T
    \end{aligned}
    \]
    Now, to show that $I$ is a regular value, we need to show that for all $A\in
    O(n)$, and for all $C\in T_{\Phi(A)}(GL(n))$, there is some $B\in T_A(O(n))$
    with $d\Phi(B)=C$.

    Take $B=\frac{1}{2}CA$, we see that
    \[
        d\Phi_A(B) = \frac{1}{2}(A(CA)^T + CAA^T) = C
    \]
    as desired. Thus, $I$ is a regular value.

    We next appeal to the fact that the tangent space of a level curve is the
    kernel of the differential. Thus, $T_I(O(n))$ is the set of all matrices for
    which
    \[
        d\Phi_I(B) = B+B^T = 0
    \]
    which is exactly $\mathfrak{o}(n)$ the set of all skew-symmetric matrices.

    Now, we will observe that $SO(n)$ is open. Consider the determinant map, a
    continuous map from $O(n)$ to the two-point set $\{-1,1\}$ with the discrete
    topology. This defines a separation of $O(n)$ into connected components.
    Specifically, the inverse image of $1$ is $SO(n)$, and thus $SO(n)$ is both
    open and closed.

    Thus, since $SO(n)$ is open, $T_I(SO(n)) = T_I(O(n)) = \mathfrak{o}(n)$ as
    desired.
\end{proof}

\subsection*{Part b}
Show $SO(n)$ is compact.
\\
\\
\begin{proof}
    To show $SO(n)$ is compact, we will show it is a closed subspace of $O(n)$,
    and show that $O(n)$ is compact. We have already observed before that
    $SO(n)$ is clopen in $O(n)$, so all we need to show is that $O(n)$ is
    compact. First, we recall that $O(n)$ is a level set, and thus is closed.
    Next, we show it is bounded. Recall that in finite-dimensional normed
    spaces, all norms are equivalent. So, we just need to show $O(n)$ is bounded
    with respect to some norm.

    Take the operator norm on $M(n)$. Then, for any $A\in O(n)$,
    \[
        \|Ax\|^2 = g(Ax,Ax) = g(A^TAx,x) = g(x,x) =\|x\|^2
    \]
    and so $\|A\|=1$. Thus, $O(n)$ is bounded by $1$ in operator norm. So,
    $O(n)$ is compact. Since $SO(n)$ is a closed subgroup of $O(n)$, $SO(n)$ is
    compact as well, as desired.
\end{proof}

\section*{Problem 3}
\subsection*{Part a}
Let $G$ be a subgroup of $\Diff(M)$, and suppose $p$ is fixed by $G$.
Show the map
\[
    g\mapsto dg_p
\]
is a group homomorphism $G\to GL(T_pM)$.
\\
\\
\begin{proof}
    We just need to show that this map respects the group operation. That is, we
    need to show that
    \[
        d(gh)_p = dg_p\circ dh_p
    \]
    but this is just a restatement of the functoriality of the differential,
    which has already been proven.
\end{proof}

\subsection*{Part b}
Find a basis for $\mathfrak{su}(2)$ and hence compute the dimension of $SU(2)$.
Prove that for $x,y\in\mathfrak{su}(2)$,
\[
    \text{trace}(x^*y)
\]
is a nondegenerate inner product. Deduce that there is a homomorphism
\[
    \pi:SU(2)\to SO(3)
\]

\begin{proof}
    We first calculate $\mathfrak{u}(2)$. This is just the level set of $I$
    under the function
    \[
        \Phi(A) = AA^*
    \]
    and thus $\mathfrak{u}(2)$ is the kernel of $d\Phi_I$. However, we've
    already calculated what $d\Phi_I$ does in problem 2, so
    \[
        d\Phi_I(A) = A + A^*
    \]
    which has kernel $\mathfrak{u}(2) = \{A\in M(2,\C)\ |\ A^*=-A\}$.

    Now, we observe that $SU(2)$ is the level set of $1$ under the determinant
    map. Now, we observe that
    \[
        d(\text{det}_I(A) = \text{tr}(A)
    \]
    
We begin by noting that the determinant can be expressed as
    \[
        \det(I+tA) = \sum_{\sigma\in
        S_n}\epsilon(\sigma)\prod_{i=1}^{n}(I+tA)_i^{\sigma(i)}
    \]
    Now, if we single out the linear term in the product by multiplying by $tA$
    once and then by $I$ the rest of the time, we end up with
    \[
        \begin{aligned}
            \textrm{lin}(\det(I+tA)) &= \sum_{\sigma\in S_n}
        \epsilon(\sigma)\sum_{i=1}^m(\prod_{j\neq
        i}I_j^{\sigma(j)})A_i^{\sigma(i)}t\\
            &= \sum_{i=1}^nA_i^it\\
            &=t\textrm{tr}(A)
        \end{aligned}
    \]
    and thus, the derivative at zero is $\textrm{tr}(A)$, as desired.
    Here, the equality from line 1 to line 2 is made by observing that
    $I_j^{\sigma(j)}$ is nonzero only when $\sigma(j)=j$, or when $\sigma=id$.

    Thus, $\mathfrak{su}(2)$ is the subspace of $\mathfrak{u}(2)$ such that
    $\text{tr}(A)=0$.

    Next, we compute a basis. Representing an arbitrary matrix as
    \[
\begin{bmatrix}
    a+bi &c+di\\
    f+gi &h+ki
\end{bmatrix}
    \]
    the trace-free requirement says that $a=-h$ and $b=-k$, and skew-symmetry
    says that $a=h=0$, $c=-f$ and  $d=g$. Thus, a typical matrix in
    $\mathfrak{su}(2)$ is
    \[
\begin{bmatrix}
    bi & c+di\\
    -c+di &-bi
\end{bmatrix}
    \]
    which has basis
    \[
        \left\{ 
            \begin{bmatrix}
                0&1\\
                -1&0
            \end{bmatrix}
            \begin{bmatrix}
                i&0\\
                0&i
            \end{bmatrix}
            \begin{bmatrix}
                0&i\\
                i&0
            \end{bmatrix}
        \right\}
    \]
    and thus has dimension $3$.
    \\
    \\
    Next, we show that $\text{trace}(x^*y)$ is an inner product on this space.
    First, we show this is symmetric. To see this, we calculate
    \[
        \text{trace}(x^*y) = \text{trace}(-xy) = \text{trace}(y(-x)) =
        \text{trace}(-yx) = \text{trace}(y^*x)
    \]
    and so this is symmetric. Next, we show it is linear in the first term. This
    follows directly: let $x,y,z\in \mathfrak{su}(2)$. Then,
    \[
        \text{trace}((x+z)^*y) = \text{trace}((-x-z)y) = \text{trace}(x^*y) +
        \text{trace}(z^*y)
    \]
    and for $\alpha\in \R$,
    \[
        \text{trace}((\alpha x)^*y) = \text{trace}(\alpha x^*y) =
        \alpha\text{trace}(x^*y)
    \]
    and thus this form is linear in the first term.

    Finally, we need to show that this is a nondegenerate form. For nonzero
    $x\in \mathfrak{su}(2)$, let
    \[
        x=
\begin{bmatrix}
    bi & c+di\\
    -c+di &-bi
\end{bmatrix}
    \]
    Then,
    \[
        \text{trace}(x^*x) = 2(b^2+c^2+d^2) \geq 0
    \]
    with equality if and only if $x=0$.

    Finally, we deduce that there is a homomorphism
    \[
        \pi:SU(2)\to SO(3)
    \]
    which is the well-known double cover of $SO(3)$.
\end{proof}

\newpage

\section*{Problem 4}
\subsection*{Part a}
Let $p$ be a homogeneous polynomial. Prove that any $a\neq 0$ is a regular value
of $p$.
\\
\\
\begin{proof}
    We calculate the differential directly.
    \[
        (dp)_aX^a = X^a\nabla_ap = X^a\partial_ap
    \]
    Now, let $\beta\in p^{-1}(a)$. We wish to show $dp_{\beta}$ is surjective.
    Since its codomain has dimension one, we just need to show it has nontrivial
    image. So, we see that
    \[
        ((dp)_{\beta})_a\beta^a = \beta^a\partial_ap|_{\beta} = ma
    \]
    by Euler's identity for homogeneous polynomials. Thus, if $a\neq 0$, then
    $a$ is a regular value, as desired.
\end{proof}

\subsection*{Part b}
Deduce that $SL(n,\R)$ is a Lie group.
\\
\\
\begin{proof}
    Observe that $SL(n)$ is the inverse image of $1$ under the determinant map.
    Now, for arbitrary $n\times n$ matrices, the determinant is a homogeneous
    polynomial of $n^2$ variables (the entries of the matrix), and thus $1$ is a
    regular value of the determinant. Thus, $SL(n)$ is a submanifold of $GL(n)$,
    and is a Lie group, as desired.
\end{proof}


\newpage

\section*{Problem 5}
Suppose $f:X\to Y$ is smooth, $X,Y$ are compact with the same dimension, and
$y\in Y$ is a regular value.
Show that $f^{-1}(y)$ is a finite set $\{x_i\}$ and that there is an open
neighborhood $U$ of $y$ for which $f^{-1}(U)$ is a finite disjoint union of open
sets $\{V_i\}$ such that $V_i$ is a neighborhood of $x_i$ and each $V_i$ is
mapped diffeomorphically onto $U$.
\\
\\
\begin{proof}
    We first show $f^{-1}(y)$ is a finite set. Since $y$ is a regular value,
    $f^{-1}(y)$ is a submanifold of $X$ of dimension $0$ (since it has
    codimension $\text{dim}(Y) = \text{dim}(X)$). Now, the only submanifolds of
    dimension zero are discrete sets, so $f^{-1}(y)$ is a closed discrete subset
    of $X$. However, since $X$ is compact, $f^{-1}(y)$ must be finite (since if
        $f^{-1}(y)$ were infinite, each point by itself is open, and thus
        $\{x_i\}$ forms an open cover of $f^{-1}(y)$ with no finite subcover, a
    contradiction).

    Now, since $df_{x_i}$ is surjective for each $x_i$ (definition of $y$ being
    a regular value), and $\text{dim}(X) = \text{dim}(Y)$, it follows that
    $df_{x_i}$ is an isomorphism for each $x_i$. Thus there are neighborhoods
    $W_i$ around each $x_i$ such that $f|_{W_i}$ is a diffemorphism onto its
    image. Without loss of generality, we take the $W_i$ to be disjoint from
    each other; since $X$ is Hausdorff, we can find open sets separating $x_i$
    and $x_j$ for $i\neq j$, and intersect them with $W_i$ to get disjoint open
    sets on which $f$ is a diffeomorphism onto.

    Take $U = \bigcap_{x_i}f(W_i)$ an open neighborhood of $y$, and take $V_i =
    f|_{W_i}^{-1}(U)$. Then, $V_i\subset W_i$ open and so $f$ is a
    diffeomorphism on $V_i$, with each $V_i$ a neighborhood of $x_i$. Finally,
    noting that $(\cup V_i)^c$ is closed, and $f$ is a closed map (taking
compact sets to compact sets), we see that $f((\cup V_i)^c)$ is closed and
avoids $y$. So, if we let $U' = f((\cup V_i)^c)^c$, we see that $U\cap U'$ is
the desired neighborhood of $y$ that is evenly covered. Let $V'_i$ be the
neighborhood of $x_i$ that maps diffeomorphically to $U\cap U'$. Then, by
construction we have that each $V_i'$ is a neighborhood of $x_i$, all disjoin
from each other, and furthermore for any $z\in (\cup V_i')^c$, either $z\in
(\cup V_i^c)$ in which case $f(z)\not\in U\cap U'$, or $z$ is in some $V_i$ but
not $V_i'$ for some $i$, in which case $f(z)\not\in U\cap U'$ since $f$ is a
diffeomorphism from $V_i$ to $U$.
\end{proof}

\end{document}
