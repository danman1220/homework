%%%%%%%%%%%%%%%%%%%%%%%%%%%%%%%%%%%%%%%%%
% Short Sectioned Assignment
% LaTeX Template
% Version 1.0 (5/5/12)
%
% This template has been downloaded from:
% http://www.LaTeXTemplates.com
%
% Original author:
% Frits Wenneker (http://www.howtotex.com)
%
% License:
% CC BY-NC-SA 3.0 (http://creativecommons.org/licenses/by-nc-sa/3.0/)
%
%%%%%%%%%%%%%%%%%%%%%%%%%%%%%%%%%%%%%%%%%

%----------------------------------------------------------------------------------------
%	PACKAGES AND OTHER DOCUMENT CONFIGURATIONS
%----------------------------------------------------------------------------------------

\documentclass[fontsize=11pt]{scrartcl} % 11pt font size

\usepackage[T1]{fontenc} % Use 8-bit encoding that has 256 glyphs
\usepackage[english]{babel} % English language/hyphenation
\usepackage{amsmath,amsfonts,amsthm} % Math packages
\usepackage{mathrsfs}
\usepackage{bbm}

\usepackage[margin=1in]{geometry}

\usepackage{sectsty} % Allows customizing section commands
\allsectionsfont{\centering \normalfont\scshape} % Make all sections centered, the default font and small caps

\usepackage{fancyhdr} % Custom headers and footers
\pagestyle{fancyplain} % Makes all pages in the document conform to the custom headers and footers
\fancyhead{} % No page header - if you want one, create it in the same way as the footers below
\fancyfoot[L]{} % Empty left footer
\fancyfoot[C]{} % Empty center footer
\fancyfoot[R]{\thepage} % Page numbering for right footer
\renewcommand{\headrulewidth}{0pt} % Remove header underlines
\renewcommand{\footrulewidth}{0pt} % Remove footer underlines
\setlength{\headheight}{13.6pt} % Customize the height of the header

\numberwithin{equation}{section} % Number equations within sections (i.e. 1.1, 1.2, 2.1, 2.2 instead of 1, 2, 3, 4)
\numberwithin{figure}{section} % Number figures within sections (i.e. 1.1, 1.2, 2.1, 2.2 instead of 1, 2, 3, 4)
\numberwithin{table}{section} % Number tables within sections (i.e. 1.1, 1.2, 2.1, 2.2 instead of 1, 2, 3, 4)

\newcommand{\R}{\mathbb{R}}
\newcommand{\Q}{\mathbb{Q}}
\newcommand{\N}{\mathbb{N}}
\newcommand{\C}{\mathbb{C}}
\newcommand{\Z}{\mathbb{Z}}

\newcommand{\ev}{\textrm{ev}}

\newtheorem{lemma}{Lemma}
%----------------------------------------------------------------------------------------
%	TITLE SECTION
%----------------------------------------------------------------------------------------

\newcommand{\horrule}[1]{\rule{\linewidth}{#1}} % Create horizontal rule command with 1 argument of height

\title{	
\normalfont \normalsize 
\textsc{analysis} \\ [25pt] % Your university, school and/or department name(s)
\horrule{0.5pt} \\[0.4cm] % Thin top horizontal rule
\huge Homework 7 \\ % The assignment title
\horrule{2pt} \\[0.5cm] % Thick bottom horizontal rule
}

\author{Daniel Halmrast} % Your name

\date{\normalsize\today} % Today's date or a custom date

\begin{document}

\maketitle % Print the title

% Problems
\section*{Problem 1}
Suppose $G$ is a locally compact abelian group, with $\mu$ the Haar measure on
$G$. If $G$ is not discrete and $\varepsilon>0$ is fixed, find a compact
neighborhood $K$ of the identity $0$ with $\mu(K)<\varepsilon$.

\begin{proof}
    We first observe that $\{0\}$ has measure zero. This follows from the
    observation that every open set in $G$ has infinitely many points, and that
    the measure $\mu$ is translation-invariant. First, we show that every open
    set has infinitely many points.

    Suppose for a contradiction that there is an open set $U\subset G$ with
    finitely many points $\{x_i\}_{i=1}^n$. Then, since $G$ is Hausdorff, we can
    choose neighborhoods $V_i$ of $x_1$ such that $V_i\subset U$ and $x_i\not\in
    V_i$. Taking the intersection of all $V_i$ yields an open set with only
    $x_1$ in it, which contradicts $G$ being non-discrete (since we could
        translate the open set $\{x_1\}$ to open sets around any singleton, and
    so every singleton would be open).

    Now, if $\{0\}$ has positive measure, then every singleton would have
    positive measure, and since every open set is infinite, each open set would
    have infinite measure, which cannot happen since $\mu$ is a Borel-regular
    measure.

    Thus, $\mu(\{0\})=0$. Since $\mu$ is outer-regular, we know that
    \[
        \mu(\{0\}) = 0 = \inf_{U\supset \{0\} \text{ open}}\mu(U)
    \]
    and so for each $\varepsilon>0$, we can find a $U$ open around $0$ for which
    $\mu(U)<\varepsilon$.

    Since $\mu$ is also inner-regular with respect to open sets, we know that
    \[
        \mu(U) = \sup_{K\subset U \text{ compact}}\mu(K)
    \]
    and so we can find a compact neighborhood $K$ of $\{0\}$ with $\mu(K)>0$.
    Since $K\subset U$, we know $\mu(K)\leq\mu(U)<\varepsilon$ as desired.
\end{proof}

\newpage

\section*{Problem 2}
Suppose $G$ is a locally compact abelian group. If $G$ is not discrete,
$\varepsilon>0$ and $G$ is Lindelof, prove there is an open neighborhood $U$ of
$0$ for which $U+U=G$ and $\mu(U)<\varepsilon$.

\begin{proof}
    For ease of notation (and without loss of generality),we will find a
    neighborhood $U$ of measure less than $2\varepsilon$ instead.

    Let $U_0$ be a neighborhood of $0$ as in the previous problem with
    $\mu(U)<\varepsilon$. Then we note that
    \[
        G = \bigcup_{g\in G} g+U_0
    \]
    and so the set $\{g+U_0\}$ is an open cover of $G$. Thus, it has a finite
    subcover $\{g_i +U_0\}_{i=1}^{\infty}$.

    For each $g_i$, let $U_i$ be constructed as in problem 1 with
    $\mu(U_i)<\frac{\varepsilon}{2^{i}}$. I assert that the open set $U =
    U_0\cup\bigcup_{=1}^{\infty}U_i$ is the open neighborhood of $0$ we desire.

    First, notice that
    \[
        \begin{aligned}
            \mu(U)&\leq \mu(U_0) + \sum_{i=1}^{\infty}\mu(U_i)\\
            &\leq \varepsilon + \sum_{i=1}^{\infty}\frac{\varepsilon}{2^i}\\
            &= 2\varepsilon
        \end{aligned}
    \]
    as desired. Next, note that from before
    \[
        G = \bigcup_{i=1}^{\infty}g_i + U_0
    \]
    and so for all $g\in G$, we have that there is some $i$ and some $u\in U_0$
    for which
    \[
        g = g_i + u
    \]
    and since $g_i,u\in U$, we have that $g\in U+U$ as desired.

    Thus, $U+U=G$.
\end{proof}

\newpage

\section*{Problem 3}
    (For this problem, I collaborated with Ashwin Trisal, Daniel Epelbaum,
    Aaron Bagheri, Micah Pedrick, and Andre Matrins.)

Suppose $G$ is a locally compact abelian group which is also compact, and let
$F\subset G$ open such that $F+F=G$. Consider the action of $G$ on $L^2(G)$ by
\[
    g(f(x)) = f(g+x)
\]
for $g,x\in G, f\in L^2(G)$.

\subsection*{Part a}
Show that this action is unitary.

\begin{proof}
    To show this action is unitary, we need to show it is invertible, and that
    it preserves inner products.

    Observe first that this group action is indeed an action, in the sense that
    it respects the group operation. This follows immediately from the
    definition, since
    \[
        (g+h)(f(x)) = f(g+h+x) = g\circ h(f(x)))
    \]
    So, the action of $g$ is invertible by $-g$, since
    \[
        (g+(-g))(f(x)) = 0(f(x)) = f(x)
    \]

    Now, we need to show that the action of $g$ preserves inner products. So,
    let $f,h\in L^2(G)$, and let $g\in G$. We calculate
    \[
\begin{aligned}
    \langle g(f),g(h)\rangle &= \int_Gf(g+x)\overline{h(g+x)}d\mu(x)\\
    &=\int_Gf(x)\overline{h(x)}d\mu(x-g)\\
    &=\int_Gf(x)\overline{h(x)}d\mu(x) = \langle f,h\rangle
\end{aligned}
    \]
    as desired. Here, we used the fact that the measure is
    translation-invariant, and so $\mu(x-g)=\mu(x)$.
\end{proof}

\subsection*{Part b}
Let $p$ be the projection in $B(L^2(G))$ onto the subspace $L^2(F)$. Let
\[
    \begin{aligned}
        \phi:G\to B(L^2(G))\\
        g\mapsto p\circ U_g\circ p
    \end{aligned}
\]
where $U_g$ is the unitary operator defined by the action of $g$.

\subsubsection*{Part i}
is $\phi(G)$ a subgroup of $B(L^2(G))$?

\begin{proof}
    If $F=G$, then $\phi(g) = U_g$ and $\phi$ is indeed a group homomorphism, so
    $\phi(G)$ is a subgroup as desired. So, assume $F\neq G$.

    I assert $\phi(G)$ is not a subgroup of $B(L^2(G))$. This follows from the
    fact that for $g=0$, we have $\phi(0) = p\circ \mathbbm{1}\circ p =p$ is not
    invertible. Thus, $\phi(G)$ contains a non-invertible element, and cannot be
    a group.
\end{proof}

\subsubsection*{Part ii}
Is $\phi(G)$ compact in any of the topologies studied in class?

\begin{proof}
    I assert that $\phi(G)$ is compact in the weak operator topology (WOT) as
    well as the strong operator topology (SOT). 

    To see this, we will prove that $\phi$ is continuous into $B(L^2(G))$ in the
    SOT.

    Recall that the SOT is the initial topology with respect to the evaluation
    functions
    \[
\begin{aligned}
    \ev_f:B(L^2(G))\to L^2(G)\\
    A\mapsto A(f)
\end{aligned}
    \]
    for all $f\in L^2(G)$. So, by the definition of the initial topology, $\phi$
    is continuous if and only if $\ev_f\circ\phi$ is continuous for all $f$.
    
    Now, 
    \[
        \ev_f\circ\phi(g) = p\circ U_g\circ p(f)
    \]

    Consider the function
    \[
        \begin{aligned}
            \varphi_f:G\to L^2(G)\\
            g\mapsto U_g\circ p(f)\\
        \end{aligned}
    \]
    Now, $p(f)$ is a (fixed) function in $L^2(G)$, and we know that the map
    $g\mapsto U_g$ is continuous (Rudin's Fourier Analysis on Groups proves
    thisi, theorem 1.1.5), so $\varphi_f$ is continuous as well.

    Now, $\ev_f\circ\phi = p\circ \varphi_f$ is the composition of continuous
    functions (as projections are continuous), and so $\ev_f\circ\phi$ is
    continuous for all $f$. Thus, $\phi$ is continuous in the SOT, and $\phi(G)$
    as the image of a compact set is compact.

    Since the WOT is strictly weaker than the SOT, this also implies that $\phi$
    is continuous in the WOT, and so $\phi(G)$ is compact in the WOT as well.
\end{proof}

\subsubsection*{Part iii}
Is $\phi$ injective?

\begin{proof}

    Suppose that $F$ is such that for all $g\in G$, there is some $f\in F$ for
    which $g+f\in F$. Then, I assert $\phi$ is injective.

    To see this, suppose $g_1,g_2\in G$. Now, there is an open set $U\subset F$
    containing $f$ as specified above (so that $g_1+f\in F$) with $g_1+U\subset
    F$ and $g_2+U\cap g_1+U = \emptyset$. This follows from $G$ being
    Hausdorff.

    Now, consider $\chi_U\in L^2(G)$. We have that 
    \[
        \begin{aligned}
            \phi(g_1)(\chi_U) = p\circ U_g(\chi_U) =
            p(\chi_{g_1+U})=\chi_{F\cap g_1+U}\\
            phi(g_2)(\chi(U)) = \chi_{F\cap g_2+U}
        \end{aligned}
    \]
    but $g_1+U$ does not intersect $g_2+U$, and $F\cap g_1+U = g_1+U\neq
    \emptyset$, so $\phi(g_1)\neq \phi(g_2)$ for all $g_1,g_2\in G$, and $\phi$
    is injective.
\end{proof}

\end{document}
