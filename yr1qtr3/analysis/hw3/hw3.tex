%%%%%%%%%%%%%%%%%%%%%%%%%%%%%%%%%%%%%%%%%
% Short Sectioned Assignment
% LaTeX Template
% Version 1.0 (5/5/12)
%
% This template has been downloaded from:
% http://www.LaTeXTemplates.com
%
% Original author:
% Frits Wenneker (http://www.howtotex.com)
%
% License:
% CC BY-NC-SA 3.0 (http://creativecommons.org/licenses/by-nc-sa/3.0/)
%
%%%%%%%%%%%%%%%%%%%%%%%%%%%%%%%%%%%%%%%%%

%----------------------------------------------------------------------------------------
%	PACKAGES AND OTHER DOCUMENT CONFIGURATIONS
%----------------------------------------------------------------------------------------

\documentclass[fontsize=11pt]{scrartcl} % 11pt font size

\usepackage[T1]{fontenc} % Use 8-bit encoding that has 256 glyphs
\usepackage[english]{babel} % English language/hyphenation
\usepackage{amsmath,amsfonts,amsthm} % Math packages
\usepackage{mathrsfs}

\usepackage[margin=1in]{geometry}

\usepackage{sectsty} % Allows customizing section commands
\allsectionsfont{\centering \normalfont\scshape} % Make all sections centered, the default font and small caps

\usepackage{fancyhdr} % Custom headers and footers
\pagestyle{fancyplain} % Makes all pages in the document conform to the custom headers and footers
\fancyhead{} % No page header - if you want one, create it in the same way as the footers below
\fancyfoot[L]{} % Empty left footer
\fancyfoot[C]{} % Empty center footer
\fancyfoot[R]{\thepage} % Page numbering for right footer
\renewcommand{\headrulewidth}{0pt} % Remove header underlines
\renewcommand{\footrulewidth}{0pt} % Remove footer underlines
\setlength{\headheight}{13.6pt} % Customize the height of the header

\numberwithin{equation}{section} % Number equations within sections (i.e. 1.1, 1.2, 2.1, 2.2 instead of 1, 2, 3, 4)
\numberwithin{figure}{section} % Number figures within sections (i.e. 1.1, 1.2, 2.1, 2.2 instead of 1, 2, 3, 4)
\numberwithin{table}{section} % Number tables within sections (i.e. 1.1, 1.2, 2.1, 2.2 instead of 1, 2, 3, 4)

\newcommand{\R}{\mathbb{R}}
\newcommand{\Q}{\mathbb{Q}}
\newcommand{\N}{\mathbb{N}}
\newcommand{\C}{\mathbb{C}}

\newcommand{\la}{\langle}
\newcommand{\ra}{\rangle}
\newcommand{\ev}{\text{ev}}
\newcommand{\im}{\text{im}}

\newtheorem{lemma}{Lemma}
%----------------------------------------------------------------------------------------
%	TITLE SECTION
%----------------------------------------------------------------------------------------

\newcommand{\horrule}[1]{\rule{\linewidth}{#1}} % Create horizontal rule command with 1 argument of height

\title{	
\normalfont \normalsize 
\textsc{Analysis} \\ [25pt] % Your university, school and/or department name(s)
\horrule{0.5pt} \\[0.4cm] % Thin top horizontal rule
\huge Homework 3 \\ % The assignment title
\horrule{2pt} \\[0.5cm] % Thick bottom horizontal rule
}

\author{Daniel Halmrast} % Your name

\date{\normalsize\today} % Today's date or a custom date

\begin{document}

\maketitle % Print the title

% Problems
\section*{Problem 1}
Prove that for $X$ a compact metric space, the multiplicative linear functionals
on $C(X)$ are exactly the point evaluation functionals
\[
    \delta_x(f) = f(x)
\]

\begin{proof}
    We first establish the following result:
    \begin{lemma}
        For every multiplicative linear functional $\phi$ on a unital Banach
        algebra $\mathscr{A}$, the kernel of $\phi$ is a maximal ideal in
        $\mathscr{A}$. Conversely, every maximal ideal in $\mathscr{A}$ is the
        kernel of some multiplicative linear functional.
    \end{lemma}
    \begin{proof}
        Let $\phi$ be a multiplicative linear functional on $\mathscr{A}$. We
        know that $\ker(\phi)$ is a closed ideal in $\mathscr{A}$, since it is
        the kernel of an algebra homomorphism. Furthermore, this ideal is
        maximal. This follows from the fact that $\im(\phi) = \C \cong
        \mathscr{A}/{\ker(\phi)}$, which has dimension one (here, we used the
            fact that $\phi\neq 0$, since the zero functional is not
        multiplicative since it has to send $I$ to $1$).

        That is, we have shown that for $\phi$ a multiplicative linear
        functional on $\mathscr{A}$, $\ker(\phi)$ is a maximal ideal in
        $\mathscr{A}$.

        Conversely, suppose $\mathscr{M}$ is a maximal ideal of $\mathscr{A}$.
        We examine the space $\mathscr{A}/{\mathscr{M}}$. Specifically, we show
        that for each nonzero $X+\mathscr{M}\in\mathscr{A}/{\mathscr{M}}$,
        $X+\mathscr{M}$ is invertible. This follows from the fact that the ideal
        \[
            \mathscr{J}_X = \{AX+Y\ |\ A\in\mathscr{A},Y\in\mathscr{M}\}
        \]
        properly contains
        \[
            \mathscr{M} = \{0X+Y\ |\ Y\in\mathscr{M}\}
        \]
        and so $\mathscr{J}_X=\mathscr{A}$ by maximality of $\mathscr{M}$. Thus,
        there is some $A\in\mathscr{A}$ and $Y\in\mathscr{M}$ with
        \[
            AX+Y=I
        \]
        and so $X+\mathscr{M}$ is invertible. We finally observe that this
        implies that $\mathscr{A}/{\mathscr{M}}\cong \C$ isometrically. This can
        be seen directly. For ease of notation, we denote $X:=
        X+\mathscr{M}\in\mathscr{A}/{\mathscr{M}}$. Now, we know that
        \[
            \sigma(X)\neq \emptyset
        \]
        However, since each $X\in\mathscr{A}/{\mathscr{M}}$ that is nonzero is
        invertible, the spectrum can contain at most one element. This is
        because at most one of
        \[
\begin{aligned}
    X-\lambda_1 I\\
    X-\lambda_2 I
\end{aligned}
        \]
        is zero, and the other must be invertible. Thus, $\sigma(X)=\{\lambda\}$
        for some $\lambda\in \C$. The map $\Phi:\mathscr{A}/{\mathscr{M}}\to \C$
        given by
        \[
            \Phi(X) = \lambda\in\sigma(X)
        \]
        is easily seen to be a bijective multiplicative linear isometry.

        Putting it all together, let $q:\mathscr{A}\to\mathscr{A}/{\mathscr{M}}$
        be the canonical quotient map. Then, the map
        \[
            \Phi\circ q:\mathscr{A}\to \C
        \]
        is a multiplicative linear functional with kernel $\mathscr{M}$, as
        desired.
    \end{proof}

    With this lemma, the problem is easy. To characterize the multiplicative
    linear functionals on $C(X)$, we just need to characterize its maximal
    ideals. Specifically, we will show that the maximal ideals of $C(X)$ are
    \[
        \mathscr{M}_x = \{f\in C(X)\ |\ f(x)=0\}
    \]
    That is, $\mathscr{M}_x$ is the set of functions that vanish at $x$.

    We first show that $\mathscr{M}_x$ is maximal (the fact that it is an ideal
    is clear). To see this, suppose $\mathscr{I}$ is another ideal containing
    $\mathscr{M}_x$ with $\mathscr{I}\neq \mathscr{M}_x$. Then, there is some
    $f\in \mathscr{I}$ with $f(x)> 0$. Now, we also know that there is some
    $g\in C(X)$ with $g^{-1}(\{0\}) = \{x\}$. That is, $g$ vanishes only at $x$.
    We can also force $g(y)>0$ for all $y\neq x$.

    Thus $f,g\in\mathscr{I}$, and thus so is $f+g$. Furthermore, by construction
    $f+g\neq 0$, and so $\frac{1}{f+g}$ is well-defined. Thus, $f+g$ has an
    inverse in $C(X)$, and since $f+g\in\mathscr{I}$, $\mathscr{I}=C(X)$ and
    $\mathscr{M}_x$ is a maximal ideal, as desired.

    We can also show that these are the only maximal ideals. Suppose
    $\mathscr{I}$ is a maximal ideal such that for each $x\in X$, there is some
    $f_x\in \mathscr{I}$ with $f_x(x)=0$. Since each $f_x$ is continuous, there
    is a neighborhood $U_x$ around $x$ for which $f_x$ is nonzero in $U_x$. This
    forms an open cover of $X$, which has a finite subcover indexed by $x_i$.
    Now, take the function
    \[
        F = \sum_{i=1}^n (f_{x_i})^2
    \]
    which is a finite sum and product of things in $\mathscr{I}$, and is thus in
    $\mathscr{I}$. However, $F(y)\neq 0$ for all $y\in X$, and so $F(y)$ is
    invertible. Thus, $\mathscr{I} = C(X)$.

    Thus, all maximal ideals of $C(X)$ are of the form $\mathscr{M}_x$. Each
    multiplicative linear functional $\phi_x$, then, has kernel $\mathscr{M}_x$
    and is thus of the form
    \[
    \phi_x(f) = f(x)
    \]
    as desired.
\end{proof}

\newpage

\section*{Problem 2}
Prove that these functionals are exactly the extreme points of $K$, the positive
part of the unit ball in $C(X)^*$.
\\
\\
\begin{proof}
    We first show that these are extreme points of $K$. To see this, suppose
    $\psi_1,\psi_2\in K$ with
    \[
        \ev_x = \phi_x = t\psi_1 + (1-t)\psi_2
    \]
    We wish to show $\psi_1 = \psi_2 = \phi_x$.
    To do so, we invoke the Riesz-Markov theorem to translate into a statement
    about measures. That is, the statement above is equivalent to
    \[
        \delta_x = t\mu_1 + (1-t)\mu_2
    \]
    where we know that $\|\mu_1(X)\|=\|\mu_2(X)\|=1$. However, this means that
    for all $E\subset X$,
    \[
        \delta_x(E) = t\mu_1(E) + (1-t)\mu_2(E)
    \]
    which, when considering the cases $x\in E$ and $x\not\in E$, we see that
    $\mu_1=\mu_2=\delta_x$, and thus $\ev_x$ is an extreme point.

    Next, we show that these are all the extreme points. To see this, suppose
    $\mu\in C(X)^*$ with $\mu\neq \delta_x$ for any $x$. In particular, we know
    that we can find $S_1,S_2\subset X$ such that $X=S_1\cup S_2$, $S_1\cap
    S_2=\emptyset$, and $\mu(S_1),\mu(S_2)>0$. Then, we have
    \[
        \mu =
        \frac{\mu(S_1)}{\mu(X)}\left(\frac{\mu(X)}{\mu(S_1)}\chi_{S_1}\mu\right) +
        \frac{\mu(S_2)}{\mu(X)}\left(\frac{\mu(X)}{\mu(S_2)}\chi_{S_2}\mu\right)
    \]
    where $\frac{\mu(S_1)}{\mu(X)} + \frac{\mu(S_2)}{\mu(X)} = 1$, and each
    term in the convex linear combination is in $K$. Thus, $\mu$ is not an
    extreme point, as desired.
\end{proof}

\newpage

\section*{Problem 3}
Find all the two-dimensional faces of $K$.
\\
\\
\begin{proof}
    I assert that the two-dimensional faces of $K$ are the convex linear
    combinations of any three extreme points.

    First, we observe that
    \[
        F = \{a\delta_x + b\delta_y + c\delta_z\ |\ a+b+c=1\}
    \]
    is indeed a face. To see this, note that by definition $F$ is convex, and
    with three degrees of freedom and one constraint, it is two-dimensional.
    Now, we just need to show it is closed under linear interpolation. So,
    suppose
    \[
        a\delta_x + b\delta_y + c\delta_z = t\phi + (1-t)\psi
    \]
    for $\phi,\psi\in K$. By utilizing Riesz-Markov theorem, we know that this
    equation must hold for the induced measures as well. So, we have that for
    all measurable $E$,
    \[
        a\delta_x(E) + b\delta_y(E) + c\delta_z(E) = t\phi(E) + (1-t)\psi(E)
    \]
    Let $E$ be such that $x,y,z\not\in E$. Then,
    \[
        0+0+0 = t\phi(E) + (1-t)\psi(E)
    \]
    which forces $\phi(E)=\psi(E) = 0$. Finally, observing that since
    $\phi,\psi\in K$, we must have $|\phi| = |\psi| = 1$ and so $\phi$ and
    $\psi$ must be linear combinations of $\delta_x,\delta_y,\delta_z$ whose
    coefficients add up to $1$, and are thus in $F$ as desired.
    Thus, $F$ is a face.

    Now, we claim that these are the only faces. This follows from the
    Krein-Milman theorem, which states that a compact and convex set is the
    convex hull of its extreme points. Now, under the weak-$*$ topology, the
    unit ball is compact, and since $K$ is a closed subset of the unit ball, it
    is compact as well. Furthermore, any closed face will also be compact. 
    
    So, suppose $F$ is a closed face of $K$. Then, $F$ is a convex hull of its
    extreme points, which will be some collection of point-mass measures.
    However, if this collection has $n>3$ measures, the resulting
    space will be $n-1$ dimensional, which is greater than two. Thus, the only
    two-dimensional faces of $K$ are the convex hulls of three extreme points of
    $K$.

\end{proof}

\newpage

\section*{Problem 4}
Let $B_1^+$ be the set $\{f\in L^1(\R)\ | f(x)\geq 0 \forall x, \int_{\R}f
=1\}$.
Find the extreme points of $B_1^+$.
\\
\\
\begin{proof}
    I claim that there are no extreme points of $B_1^+$. To see this, suppose
    $f\in B_1^+$. In particular, this means that for some positive-measure set
    $E$, $f|_E > 0$. Now, split $E$ into two sets $E_1, E_2$ with equal positive
    measure. Set $\varepsilon>0$ such that $f-\varepsilon >0$ on $E$.

    Define
    \[
\begin{aligned}
    g_{\pm}(x) =
    \begin{cases}
        f(x), &x\in E^c\\
        f(x)\pm\varepsilon, &x\in E_1\\
        f(x)\mp\varepsilon, &x\in E_2\\
    \end{cases}
\end{aligned}
    \]
    clearly, $\int g_{\pm} = 1$, and furthermore,
    \[
        f(x) = \frac{1}{2}g_+(x) + \frac{1}{2}g_-(x)
    \]
    so $f$ is not an extreme point.
\end{proof}

\newpage

\section*{Problem 5}
Find the extreme points of $F_1^+$, the set of all positive $n\times n$
self-adjoint complex matrices with trace $1$.
\\
\\
\begin{proof}
    I assert that all the extreme points of $F_1^+$ are the one-dimensional
    projection operators.

    First, observe that $F_1^+\subset M_1^+$, where $M_1^+$ from last homework
    is the set of all positive, self-adjoint $n\times n$ matrices less than $I$.
    This is clear, since for $M\in F_1^+$, we know that $\sigma(M)\subset[0,1]$
    since the trace of $M$ (the sum of the eigenvalues) is $1$, and since $M$ is
    positive, it has all positive eigenvalues. Thus, its eigenvalues are
    positive and sum to one, and thus must be between zero and one. This is the
    condition necessary to be in $M_1^+$, as desired.

    Thus, if $M$ is an extreme point of $M_1^+$, and $M\in F_1^+$, then $M$ is
    an extreme point of $F_1^+$ as well. We noted that the projections are
    extreme points in $M_1^+$, and specifically the one-dimensional projections
    are in $F_1^+$. Thus, they are extreme points.

    We next observe that these are the only extreme points. To see this, suppose
    $M\in F_1^+$ with $M$ not a one-dimensional projection. Since the
    eigenvalues of $M$ add up to one, and $M$ is not a one-dimensional
    projection, $M$ must have at least two eigenvalues less than one. So, write
    $M$ as
    \[
        M = \lambda_1|e_1\ra\la e_1| + \lambda_2 |e_2\ra\la e_2| +
        \sum_{i=3}^n \lambda_i |e_i\ra\la e_i|
    \]
    and let $\varepsilon>0$ be such that $0 <\lambda_1\pm \varepsilon <1$ and
    $0<\lambda_2\pm\varepsilon < 1$. Then, define
    \[
        M_{\pm} = (\lambda_1\pm\varepsilon)|e_1\ra\la e_1| +
        (\lambda_2\mp\varepsilon) |e_2\ra\la e_2| + 
        \sum_{i=3}^n \lambda_i |e_i\ra\la e_i|
    \]
    which are clearly in $F_1^+$. We finally observe that
    \[
        M = \frac{1}{2}M_+ + \frac{1}{2}M_-
    \]
    and so $M$ is not an extreme point, as desired.
\end{proof}

\end{document}
