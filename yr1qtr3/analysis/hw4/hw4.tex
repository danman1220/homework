%%%%%%%%%%%%%%%%%%%%%%%%%%%%%%%%%%%%%%%%%
% Short Sectioned Assignment
% LaTeX Template
% Version 1.0 (5/5/12)
%
% This template has been downloaded from:
% http://www.LaTeXTemplates.com
%
% Original author:
% Frits Wenneker (http://www.howtotex.com)
%
% License:
% CC BY-NC-SA 3.0 (http://creativecommons.org/licenses/by-nc-sa/3.0/)
%
%%%%%%%%%%%%%%%%%%%%%%%%%%%%%%%%%%%%%%%%%

%----------------------------------------------------------------------------------------
%	PACKAGES AND OTHER DOCUMENT CONFIGURATIONS
%----------------------------------------------------------------------------------------

\documentclass[fontsize=11pt]{scrartcl} % 11pt font size

\usepackage[T1]{fontenc} % Use 8-bit encoding that has 256 glyphs
\usepackage[english]{babel} % English language/hyphenation
\usepackage{amsmath,amsfonts,amsthm} % Math packages
\usepackage{mathrsfs}

\usepackage[margin=1in]{geometry}

\usepackage{sectsty} % Allows customizing section commands
\allsectionsfont{\centering \normalfont\scshape} % Make all sections centered, the default font and small caps

\usepackage{fancyhdr} % Custom headers and footers
\pagestyle{fancyplain} % Makes all pages in the document conform to the custom headers and footers
\fancyhead{} % No page header - if you want one, create it in the same way as the footers below
\fancyfoot[L]{} % Empty left footer
\fancyfoot[C]{} % Empty center footer
\fancyfoot[R]{\thepage} % Page numbering for right footer
\renewcommand{\headrulewidth}{0pt} % Remove header underlines
\renewcommand{\footrulewidth}{0pt} % Remove footer underlines
\setlength{\headheight}{13.6pt} % Customize the height of the header

\numberwithin{equation}{section} % Number equations within sections (i.e. 1.1, 1.2, 2.1, 2.2 instead of 1, 2, 3, 4)
\numberwithin{figure}{section} % Number figures within sections (i.e. 1.1, 1.2, 2.1, 2.2 instead of 1, 2, 3, 4)
\numberwithin{table}{section} % Number tables within sections (i.e. 1.1, 1.2, 2.1, 2.2 instead of 1, 2, 3, 4)

\newcommand{\R}{\mathbb{R}}
\newcommand{\Q}{\mathbb{Q}}
\newcommand{\N}{\mathbb{N}}
\newcommand{\C}{\mathbb{C}}

\newcommand{\Hom}{\text{Hom}}
\newcommand{\la}{\langle}
\newcommand{\ra}{\rangle}

\newtheorem{lemma}{Lemma}
%----------------------------------------------------------------------------------------
%	TITLE SECTION
%----------------------------------------------------------------------------------------

\newcommand{\horrule}[1]{\rule{\linewidth}{#1}} % Create horizontal rule command with 1 argument of height

\title{	
\normalfont \normalsize 
\textsc{Analysis} \\ [25pt] % Your university, school and/or department name(s)
\horrule{0.5pt} \\[0.4cm] % Thin top horizontal rule
\huge Homework 4 \\ % The assignment title
\horrule{2pt} \\[0.5cm] % Thick bottom horizontal rule
}

\author{Daniel Halmrast} % Your name

\date{\normalsize\today} % Today's date or a custom date

\begin{document}

\maketitle % Print the title

% Problems
\section*{Problem 1}
Let $\{H_{\gamma}\ |\ \gamma\in\Gamma\}$ be a family of Hilbert spaces, and let
$H$ be the vector space of sections of $\cup_{\gamma\in\Gamma}H_{\gamma}$ over
$\Gamma$ with
\[
\begin{aligned}
    f\in \Gamma(\cup H_{\gamma},\Gamma)\\
    \sum_{\gamma\in\Gamma}\|f(\gamma)\|^2 < \infty
\end{aligned}
\]
Show that
\[
    \|f\| = \left( \sum_{\gamma\in\Gamma}\|f(\gamma)\|^2
    \right)^{\frac{1}{2}}
\]
is a norm on $H$, and that with this norm $H$ is a Euclidean space. Is $H$
necessarily a Hilbert space?
\\
\\
\begin{proof}
    We immediately recognize this construction as the direct integral
    \[
        \int_{\Gamma}^{\oplus}H_{\gamma}d\mu
    \]
    where $\mu=\mu_c$ is the counting measure on $\Gamma$. This space is defined
    to be the set of all sections
    $\Gamma(\coprod_{\gamma\in\Gamma}H_{\gamma},\Gamma)$ over $\Gamma$ with the
    property that if $f$ is a section, then its composition
    \[
        g_{\gamma}(f(\gamma),f(\gamma)) = \|f(\gamma)\|^2\in L^2(\Gamma,\mu_c)
    \]
    (with $g_{\gamma}$ the metric on $H_{\gamma}$)
    is required to be an $L^2$ function. 
    For the remainder of this problem, we will let $(\Gamma,\mu)$ be an
    arbitrary measure space, and prove the more general result that
    \[
        H = \int_{\Gamma}^{\oplus}H_{\gamma}d\mu(\gamma)
    \]
    is a Hilbert space. Note that in this more general setting, $H$ is actually
    a set of equivalence classes of sections where $s\sim t \iff
    \|s(\gamma)\|_{\gamma} \sim \|t(\gamma)\|_{\gamma}$ as functions in
    $L^2(\Gamma,\mu)$.
    That is, $s$ is equivalent to $t$ if and only if $s(\gamma)$ and
    $t(\gamma)$ differ on a set of Hilbert spaces of measure zero.
    
    (Furthermore, for more general measure spaces, we might run into
    measurability issues, but since this example is against the counting
measure, we will not worry about such technicalities.)

    An inner product on the direct integral can be defined as
    \[
        \la s|t\ra := \int_{\Gamma}\la s(\gamma)|t(\gamma)\ra d\mu(\gamma)
    \]
    where it is understood that $\la s(\gamma)|t(\gamma)\ra =
    g_{\gamma}(s(\gamma),t(\gamma))$.
    We need to show that this inner product is indeed a well-defined inner
    product on $\int_{\Gamma}^{\oplus}H_{\gamma}d\mu$.


    We first show that this is well-defined. That is, we need to show that the
    integral is finite. To see this, let $s,t\in H$. In particular, this means
    that
    \[
        \begin{aligned}
            \|s(\gamma)\|_{\gamma}\in L^2(\Gamma,\mu)\\
            \|t(\gamma)\|_{\gamma}\in L^2(\Gamma,\mu)
    \end{aligned}
    \]
    We calculate the inner product as
    \[
\begin{aligned}
    |\la s|t\ra| &= |\int_{\Gamma}\la s(\gamma)|t(\gamma)\ra d\mu(\gamma)|\\
    &\leq \int_{\Gamma}|\la s(\gamma)|t(\gamma)\ra| d\mu(\gamma)\\
    &\leq \int_{\Gamma}\|s(\gamma)\|_{\gamma}\|t(\gamma)\|_{\gamma}
    d\mu(\gamma)&\text{ By Cauchy-Schwarz inequality}\\
    &=\|\left(\|s(\gamma)\|_{\gamma}\right)\left(\|t(\gamma)\|_{\gamma}\right)\|_1
    &\text{ By definiton of $L^1$ norm}\\
    &\leq
    \big(\|\left(\|s(\gamma)\|_{\gamma}\right)\|_2\big)
    \big(\|\left(\|t(\gamma)\|_{\gamma}\right)\|_2\big)
    &\text{ By Holder's inequality with $p=q=2$}\\
    &< \infty &\text{ Since $\|s(\gamma)\|_{\gamma}$ and
    $\|t(\gamma)\|_{\gamma}$ are in $L^2(\Gamma,\mu)$}
\end{aligned}
    \]
    and thus, the proposed inner product is well-defined.

    Next, we show sesquilinearity. We adopt the mathematics convention that the
    inner product $\la s|t\ra$ is linear in the first term, and conjugate linear
    in the second term. Let $s,t\in H$, and let $\alpha\in \C$.
    Then,
    \[
        \begin{aligned}
        \la \alpha s|t\ra &= \int_{\Gamma}\la \alpha
        s(\gamma)|t(\gamma)\ra d\mu(\gamma)\\
        &= \int_{\Gamma}\alpha\la s(\gamma)|t(\gamma)\ra d\mu(\gamma)\\
        &= \alpha\int_{\Gamma}\la s(\gamma)|t(\gamma)\ra d\mu(\gamma)\\
        &=\alpha\la s|t\ra
    \end{aligned}
    \]
    Furthermore, with $r\in H$ as well, we have
    \[
\begin{aligned}
    \la r+s|t\ra &= \int_{\Gamma}\la r(\gamma)+s(\gamma)|t(\gamma)\ra
    d\mu(\gamma)\\
    &= \int_{\Gamma}\la r(\gamma)|t(\gamma)\ra + \la s(\gamma)|t(\gamma)\ra
    d\mu(\gamma)\\
    &=\int_{\Gamma}\la r(\gamma)|t(\gamma)\ra d\mu(\gamma) + \int_{\Gamma}\la
    s(\gamma)|t(\gamma)\ra d\mu(\gamma)\\
    &=\la r|t\ra + \la s|t\ra
\end{aligned}
    \]
    and so the proposed inner product is linear in the first term. Furthermore,
    we see that
    \[
\begin{aligned}
    \la s|t\ra &= \int_{\Gamma}\la s(\gamma)|t(\gamma)\ra d\mu(\gamma)\\
    &= \int_{\Gamma}\overline{\la t(\gamma)|s(\gamma)\ra} d\mu(\gamma)\\
    &= \overline{\int_{\Gamma}\la t(\gamma)|s(\gamma)\ra d\mu(\gamma)}\\
    &= \overline{\la t|s\ra}
\end{aligned}
    \]
    and so the proposed inner product is conjugate linear in the second term.

    Finally, we need to show this inner product is positive-definite. That is,
    we need to show
    \[
        \la s|s\ra \geq 0
    \]
    with equality if and only if $s=0$.
    So, let $s\in H$. Trivially, if $s=0$, then
    \[
        \la s|s\ra = \int_{\Gamma}\la s(\gamma)|s(\gamma)\ra d\mu(\gamma) =
        \int_{\Gamma}0 = 0
    \]
    So, let $s\neq 0$. Then, in particular, $\|s(\gamma)\|_{\gamma}$ differs
    from zero on a set $E\subset\Gamma$ of positive measure. Thus,
    \[
        \begin{aligned}
            \la s|s\ra &= \int_{\Gamma} \la s(\gamma)|s(\gamma)\ra
            d\mu(\gamma)\\
            &= \int_{\Gamma} \|s(\gamma)\|_{\gamma}^2 d\mu(\gamma)\\
            &\geq \int_E \|s(\gamma)\|_{\gamma}^2d\mu(\gamma)\\
            &> 0
    \end{aligned}
    \]
    as desired. Thus, this proposed inner product is indeed an inner product on
    $H$.

    Observe that this completes the first two parts of this problem. We defined
    \[
        \|s\|^2 = \la s|s\ra = \int_{\Gamma}\|s(\gamma)\|_{\gamma}^2
        d\mu(\gamma)
    \]
    and setting $\mu = \mu_c$ the counting measure, we get
    \[
        \|s\|^2 = \sum_{\gamma\in\Gamma}\|s(\gamma)\|_{\gamma}^2
    \]
    which is a norm, since it is induced by an inner product.

    Furthermore, we have defined an inner product on $H$, which makes it a
    Euclidean space. Now, we just have to show that $H$ is actually a Hilbert
    space.


    We need to show that $H$ is complete with respect to its norm. So, suppose
    $s_n\in H$ is a sequence of elements (sections) in $H$ such that $\{s_n\}$
    is a Cauchy sequence with respect to the norm.

    In particular, this means that for $\varepsilon>0$ there is an $N$ such that
    for all $n,m>N$,
    \[
        \|s_n-s_m\|^2 < \varepsilon
    \]
    which translates to
    \[
        \begin{aligned}
            \|s_n-s_m\|^2 = \int_{\Gamma}\|s_n(\gamma) - s_m(\gamma)\|_{\gamma}^2
            d\mu(\gamma) <
            \varepsilon
    \end{aligned}
    \]
    and so $\mu$-almost every term $\|s_n(\gamma)-s_m(\gamma)\|_{\gamma}^2 <
    \varepsilon$ as well (modulo some constant proportionality). Thus, each term
    is Cauchy, and converges to some limit $s(\gamma)$ pointwise $\mu$-a.e.(note
    that this convergence is uniform over $\gamma$). It is quickly verified that
    $s(\gamma)\in H$, since
    \[
        \begin{aligned}
        \|s\|^2 &= \int_{\Gamma}\|s(\gamma)\|_{\gamma}^2d\mu(\gamma)\\
        &=
        \int_{\Gamma}\|\lim_{n\to\infty}s_n(\gamma)\|_{\gamma}^2d\mu(\gamma)\\
        &= \lim_{n\to\infty}\int_{\Gamma}\|s_n(\gamma)\|_{\gamma}^2
        d\mu(\gamma) &\text{ by continuity of $\|\cdot\|_{\gamma}$ and
        $s_n(\gamma)\to s(\gamma)$ uniformly}\\
        &= \lim_{n\to\infty}\|s_n\|^2 < \infty
        \end{aligned}
    \]
    Here, the limit $\lim_n\|s_n\|^2 <\infty$ follows from the fact that the
    sequence $\|s_n\|^2$ is a Cauchy sequence of real numbers, and thus has a
    limit.

    Now, we need to show that $s_n$ converges to $s$ in norm.
    To show this, we just need to show that
    \[
        \|s_n-s\|^2 = \int_{\Gamma}\|s_n(\gamma)- s(\gamma)\|^2_{\gamma}d\mu(\gamma)
        \to 0
    \]
    However, this follows immediately from the uniform convergence observed
    earlier. In particular, we consider the sequence $f_n(\gamma) =
    \|s_n(\gamma) - s(\gamma)\|^2_{\gamma}$, which tends uniformly to zero.
    Then, we have that
    \[
        \lim_{n\to\infty}\int_{\Gamma}f_n(\gamma)d\mu(\gamma) =
        \int_{\Gamma}0d\mu(\gamma) = 0
    \]
    and so $\lim_{n\to\infty}\|s_n-s\| = 0$ as desired. Thus, $H$ is complete.


    (This construction was first presented to me in Hall's {\em Quantum Theory
        for Mathematicians}, and much of the proof technique mirrors what I
    remember from the text.)
\end{proof}

\newpage

\section*{Problem 2}
Prove that if $f\in C(\mathbb{T})$, then
\[
    \frac{1}{2\pi}\int_{\mathbb{T}}f(t)e^{-int}dt \to 0
\]
as $|n|\to \infty$.
\\
\\
\begin{proof}
    This result follows immediately from the earlier observation that the set
    $\{|e^{int}\ra\}_{n=-\infty}^{\infty}$ forms an orthonormal basis for
    $L^2(\mathbb{T})$. In particular, this means that
    \[
        \sum_n |e^{int}\ra\la e^{int}| = I
    \]
    (It is the sum of projections, so it is a projection itself. If it did have
        some subspace on which it was not the identity, a unit vector in that
        subspace would be orthogonal to each $|e^{int}\ra$, and thus they would
    not form an orthonormal basis).

    Applying this to $f$, we see that
    \[
        I|f\ra = |f\ra = \sum_n \la e^{int}|f\ra|e^{int}\ra
    \]
    but since $f\in L^2(\mathbb{T})$, $\|f\|^2 < \infty$, and thus
    \[
        \|f\|^2 = \la f|f\ra = \sum_n \la e^{int}|f\ra^2 < \infty
    \]
    Thus, it must be that
    \[
        \la e^{int}|f\ra^2 = \int_{\mathbb{T}}e^{-int}f(t)dt \to 0
    \]
    to keep the sum finite, as desired.
\end{proof}

\newpage

\section*{Problem 3}
Show that if $T\in B(H)$ is hermitian, then $\exp(iT)$ is unitary.
\\
\\
\begin{proof}
    We only need to show that $\exp(iT)^* = \exp(iT)^{-1}$. To do so, we will
    make use of the {\em Continuous Functional Calculus} given by the Gelfand
    transform. Recall that there exists a unique isometric $*$-isomorphism $\gamma:
    C^*(A)\to C(\sigma(A))$ between the unital $C$-$*$ algebra generated by a
    normal operator $A$ and the algebra of continuous functions on the spectrum
    of $A$ under the sup norm.

    The construction of this isomorphism is done in two steps. First, recall
    that we have already shown that there is a unique isometric $*$-isomorphism
    between $C^*(A)$ and $C(\mathscr{M}_A)$ the continuous functions on the
    multiplicative linear functionals on $C^*(A)$ given by the Gelfand map $\Gamma$
    defined as
    \[
        \Gamma(A)(\phi) = \phi(A)
    \]
    
    Now, we next apply the fact that $\sigma(A) = \{\phi(A)\ |\ \phi\in
    \mathscr{M}_A\}$. This follows from an easy argument: suppose
    $\lambda\in\sigma(A)$. Then, $A-\lambda I$ is not invertible, and generates
    a proper ideal in $C^*(A)$. This is contained in a maximal ideal, and is
    thus in the kernel of some multiplicative linear functional
    $\phi_{\lambda}$. This implies that
    \[
        \phi_{\lambda}(A-\lambda I) = 0
    \]
    and since $\phi_{\lambda}$ is linear, and $\phi_{\lambda}(I)=1$ (true of
    every multiplicative linear functional), we have that
    \[
        \phi_{\lambda}(A) = \lambda
    \]
    as desired. Conversely, if $\lambda\not\in\sigma(A)$, then there is some
    $B\in C^*(A)$ for which $(A-\lambda I)B =I$. Applying an arbitrary
    nontrivial multiplicative linear functional yields
    \[
        \phi(A-\lambda I)\phi(B) = 1
    \]
    and so $\phi(A-\lambda I)\neq 0$ as desired.

    With this result in hand, we proceed with the construction of the continuous
    functional calculus. Define $\tau:\mathscr{M}_A\to \sigma(A)$ by
    \[
        \tau(\phi) = \phi(A)
    \]
    This is a continuous bijection between $\mathscr{M}_A$ and $\sigma(A)$, and
    since $\mathscr{M}_A$ is compact (in the weak-$*$ topology, as a closed
    subset of the unit ball) and $\sigma(A)$ is Hausdorff (as a subset of $\C$ a
    Hausdorff space) $\tau$ is a homeomorphism. Thus, it induces an algebra
    $*$-isomorphism $h$ between $C(\mathscr{M}_A)$ and $C(\sigma(A))$ given by
    \[
        h(f) = f\circ\tau^{-1}
    \]
    the pullback of $f$ along $\tau^{-1}$.

    Composing these two yields the desired $*$-isomorphism
    \[
        \begin{aligned}
    \gamma&:C^*(A)\to C(\sigma(A))\\
    \gamma &= h\circ \Gamma
\end{aligned}
    \]
    Now, it is easy to see that $\gamma(A) = z$ the identity on $\C$. This
    follows by direct calculation:
    \[
        \begin{aligned}
            \gamma(A)(\lambda) &= h\circ\Gamma(A)(\lambda)\\
            &= h(\text{ev}_A)(\lambda)\\
            &= \text{ev}_A(\tau^{-1}(\lambda))\\
            &= \text{ev}_A(\phi)\\
            &=\phi(A) = \lambda
    \end{aligned}
    \]
    Since this map is an algebra isomorphism, we have immediately that if $p$ is
    a polynomial in $A$, then $\gamma(p(A)) = p(z)$. Taking closures, we see
    that if $f$ is a continuous function (approximated by polynomials), then
    \[
        f(A) = \gamma^{-1}(f)
    \]

    With all this machinery, we are ready to prove that $\exp(iT)$ is unitary.
    Let $\gamma$ be the unique $*$-isomorphism between $C^*(T)$ and
    $C(\sigma(T))$. Then, we have that
    \[
        \begin{aligned}
        \gamma(\exp(iT)) &= \exp(it)\\
        \gamma((\exp(iT))^*) &= \overline{(\exp(it))}=\exp(-it) &\text{ since
        $\sigma(T)\subset \R$}
    \end{aligned}
    \]
    Thus,
    \[
        \gamma(\exp(iT)(\exp(iT))^*) = \exp(it)\exp(-it) = 1
    \]
    and since $\gamma^{-1}(1)=I$, we have that
    \[
        \exp(iT)(\exp(iT))^* = I
    \]
    as desired.
\end{proof}

\newpage

\section*{Problem 4}
Show that every nonempty compact subset $K\subset\C$ is the spectrum of some
operator.
\\
\\
\begin{proof}
    Let $\{\lambda_n\}$ be a countable dense subset of $K$. Define an operator
    $T\in B(\ell^2)$ as
    \[
        T(e_n) = \lambda_ne_n
    \]
    and extend linearly. This is a bounded linear operator, since each
    $\lambda_n$ is bounded as an element of a compact set. Furthermore, each
    $\lambda_n\in\sigma(T)$ by construction. Now, since the spectrum $\sigma(T)$
    is closed, we have that
    \[
        \overline{\{\lambda_n\}} \subset K
    \]

    Now for the converse direction, suppose $\lambda\in \C\setminus K$. Since
    $K$ is closed, there is some $\varepsilon$ for which $\|\lambda - x\| >
    \varepsilon$ for all $x\in K$. Then, it is easy to see that $T-\lambda I$ is
    invertible. First, observe that it is bounded below, since
    \[
        (T-\lambda I)e_n = (\lambda_n-\lambda)e_n > \varepsilon e_n
    \]
    and thus $\|T-\lambda I\| > \varepsilon$. Secondly, notice that it has dense
    range, since
    \[
        (T-\lambda I)(\frac{1}{\lambda_n-\lambda})e_n = e_n
    \]
    and $\frac{1}{\lambda_n - \lambda}$ never diverges. Thus, $\lambda$ is not
    in the spectrum of $T$. Therefore, the spectrum of $T$ is precisely the
    closure of $\{\lambda_n\}$, which is $K$ by construction.
\end{proof}

\newpage

\section*{Problem 5}
Let $M$ be a maximal ideal of a complex unital Banach algebra $B$. Show that
$B/{M}$ is also a complex unital Banach algebra.
\\
\\
\begin{proof}
    Recall the norm we give to a quotient space is
    \[
        \|b + M\| = \inf_{m\in M}\|b-m\|
    \]
    Now, this relies on $M$ being closed, but every maximal ideal is closed,
    since its closure is a proper ideal containing it.

    Now, $B/{M}$ is a Banach space under this norm (proven last quarter: the
        quotient of a Banach space by a closed linear subspace is again a Banach
    space), so all we have to show is that the norm is submultiplicative, and
    that the quotient contains the identity. 

    Recall that the multiplication on $B/{M}$ is given by
    \[
        (a+M)(b+M) = ab+M
    \]
which is easily verified to be well-defined, since 
\[
    (a+m+M)(b+n+M) = (ab + mb + an+mn +M) = ab+M
\]

    To see that the norm is submultiplicative, we observe that for $a_1,a_2\in
    B$ and $m_1,m_2\in M$ we have
    \[
        \begin{aligned}
            \|(a_1+m_1)(a_2+m_2)\|\leq \|a_1+m_1\|\|a_2+m_2\|
        \end{aligned}
    \]
    and taking $\inf$ over choices of $m_1$ and $m_2$ yields the desired result.

    Now, the identity in $B/{M}$ is just $1+M$. This is not zero, since
    $1\not\in M$ (otherwise $M$ would be forced to be equal to $B$).
    Furthermore, for any $b+M\in B/{M}$, we have
    \[
        (1+M)(b+M) = b+M = (b+M)(1+M)
    \]
    as desired.

\end{proof}

\newpage

\section*{Problem 6}
Prove that if $M$ is a maximal ideal of $A$ a commutative complex unital Banach
algebra, then $A/M$ is isometrically isomorphic to $\C$.
\\
\\
\begin{proof}
    Recall that the quotient of a commutative ring by a maximal ideal is a
    field. Thus, $A/{M}$ is a field. Furthermore, from the previous problem, we
    know that $A/{M}$ is a complex unital Banach algebra. Applying the
    Gelfand-Mazur theorem (Theorem 10, Chapter 12 of Bolabas) we see that
    $A/{M}$ must be (isometrically isomorphic to) $\C$.
\end{proof}

\end{document}
