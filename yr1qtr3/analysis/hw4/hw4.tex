%%%%%%%%%%%%%%%%%%%%%%%%%%%%%%%%%%%%%%%%%
% Short Sectioned Assignment
% LaTeX Template
% Version 1.0 (5/5/12)
%
% This template has been downloaded from:
% http://www.LaTeXTemplates.com
%
% Original author:
% Frits Wenneker (http://www.howtotex.com)
%
% License:
% CC BY-NC-SA 3.0 (http://creativecommons.org/licenses/by-nc-sa/3.0/)
%
%%%%%%%%%%%%%%%%%%%%%%%%%%%%%%%%%%%%%%%%%

%----------------------------------------------------------------------------------------
%	PACKAGES AND OTHER DOCUMENT CONFIGURATIONS
%----------------------------------------------------------------------------------------

\documentclass[fontsize=11pt]{scrartcl} % 11pt font size

\usepackage[T1]{fontenc} % Use 8-bit encoding that has 256 glyphs
\usepackage[english]{babel} % English language/hyphenation
\usepackage{amsmath,amsfonts,amsthm} % Math packages
\usepackage{mathrsfs}

\usepackage[margin=1in]{geometry}

\usepackage{sectsty} % Allows customizing section commands
\allsectionsfont{\centering \normalfont\scshape} % Make all sections centered, the default font and small caps

\usepackage{fancyhdr} % Custom headers and footers
\pagestyle{fancyplain} % Makes all pages in the document conform to the custom headers and footers
\fancyhead{} % No page header - if you want one, create it in the same way as the footers below
\fancyfoot[L]{} % Empty left footer
\fancyfoot[C]{} % Empty center footer
\fancyfoot[R]{\thepage} % Page numbering for right footer
\renewcommand{\headrulewidth}{0pt} % Remove header underlines
\renewcommand{\footrulewidth}{0pt} % Remove footer underlines
\setlength{\headheight}{13.6pt} % Customize the height of the header

\numberwithin{equation}{section} % Number equations within sections (i.e. 1.1, 1.2, 2.1, 2.2 instead of 1, 2, 3, 4)
\numberwithin{figure}{section} % Number figures within sections (i.e. 1.1, 1.2, 2.1, 2.2 instead of 1, 2, 3, 4)
\numberwithin{table}{section} % Number tables within sections (i.e. 1.1, 1.2, 2.1, 2.2 instead of 1, 2, 3, 4)

\newcommand{\R}{\mathbb{R}}
\newcommand{\Q}{\mathbb{Q}}
\newcommand{\N}{\mathbb{N}}
\newcommand{\C}{\mathbb{C}}

\newcommand{\Hom}{\text{Hom}}
\newcommand{\la}{\langle}
\newcommand{\ra}{\rangle}

\newtheorem{lemma}{Lemma}
%----------------------------------------------------------------------------------------
%	TITLE SECTION
%----------------------------------------------------------------------------------------

\newcommand{\horrule}[1]{\rule{\linewidth}{#1}} % Create horizontal rule command with 1 argument of height

\title{	
\normalfont \normalsize 
\textsc{Analysis} \\ [25pt] % Your university, school and/or department name(s)
\horrule{0.5pt} \\[0.4cm] % Thin top horizontal rule
\huge Homework 4 \\ % The assignment title
\horrule{2pt} \\[0.5cm] % Thick bottom horizontal rule
}

\author{Daniel Halmrast} % Your name

\date{\normalsize\today} % Today's date or a custom date

\begin{document}

\maketitle % Print the title

% Problems
\section*{Problem 1}
Let $\{H_{\gamma}\ |\ \gamma\in\Gamma\}$ be a family of Hilbert spaces, and let
$H$ be the vector space of sections of $\cup_{\gamma\in\Gamma}H_{\gamma}$ over
$\Gamma$ with
\[
\begin{aligned}
    f\in \Gamma(\cup H_{\gamma},\Gamma)\\
    \sum_{\gamma\in\Gamma}\|f(\gamma)\|^2 < \infty
\end{aligned}
\]
Show that
\[
    \|f\| = \left( \sum_{\gamma\in\Gamma}\|f(\gamma)\|^2
    \right)^{\frac{1}{2}}
\]
is a norm on $H$, and that with this norm $H$ is a Euclidean space. Is $H$
necessarily a Hilbert space?
\\
\\
\begin{proof}
    We immediately recognize this construction as the direct integral
    \[
        \int_{\Gamma}^{\oplus}H_{\gamma}d\mu
    \]
    where $\mu=\mu_c$ is the counting measure on $\Gamma$. This space is defined
    to be the set of all sections
    $\Gamma(\coprod_{\gamma\in\Gamma}H_{\gamma},\Gamma)$ over $\Gamma$ with the
    property that if $f$ is a section, then its composition
    \[
        g_{\gamma}(f(\gamma),f(\gamma)) = \|f(\gamma)\|^2\in L^2(\Gamma,\mu_c)
    \]
    (with $g_{\gamma}$ the metric on $H_{\gamma}$)
    is required to be an $L^2$ function. In fact, it is clear that this is the
    exact scenario described in the hypothesis of the problem.
    So, for the remainder of this problem, we will let $(\Gamma,\mu)$ be an
    arbitrary measure space, and prove the more general result that
    \[
        H = \int_{\Gamma}^{\oplus}H_{\gamma}d\mu(\gamma)
    \]
    is a Hilbert space. Note that in this more general setting, $H$ is actually
    a set of equivalence classes of sections where $s~t \iff
    \|s(\gamma)\|_{\gamma} ~ \|t(\gamma)\|_{\gamma}$ as functions in
    $L^2(\Gamma,\mu)$.
    That is, $s$ is equivalent to $t$ if and only if $s(\gamma)$ and
    $t(\gamma)$ differ on a set of Hilbert spaces of measure zero.
    
    Furthermore, for more general measure spaces, we might run into
    measurability issues, but since this example is against the counting
    measure, we will not worry about such technicalities.

    An inner product on the direct integral can be defined as
    \[
        \la s|t\ra := \int_{\Gamma}\la s(\gamma)|t(\gamma)\ra d\mu(\gamma)
    \]
    where it is understood that $\la s(\gamma)|t(\gamma)\ra =
    g_{\gamma}(s(\gamma),t(\gamma))$.
    We need to show that this inner product is indeed a well-defined inner
    product on $\int_{\Gamma}^{\oplus}H_{\gamma}d\mu$.


    We first show that this is well-defined. That is, we need to show that the
    integral is finite. To see this, let $s,t\in H$. In particular, this means
    that
    \[
        \begin{aligned}
            \|s(\gamma)\|_{\gamma}\in L^2(\Gamma,\mu)\\
            \|t(\gamma)\|_{\gamma}\in L^2(\Gamma,\mu)
    \end{aligned}
    \]
    We calculate the inner product as
    \[
\begin{aligned}
    |\la s|t\ra| &= |\int_{\Gamma}\la s(\gamma)|t(\gamma)\ra d\mu(\gamma)|\\
    &\leq \int_{\Gamma}|\la s(\gamma)|t(\gamma)\ra| d\mu(\gamma)\\
    &\leq \int_{\Gamma}\|s(\gamma)\|_{\gamma}\|t(\gamma)\|_{\gamma}
    d\mu(\gamma)&\text{ By Cauchy-Schwarz inequality}\\
    &=\|\left(\|s(\gamma)\|_{\gamma}\right)\left(\|t(\gamma)\|_{\gamma}\right)\|_1
    &\text{ By definiton of $L^1$ norm}\\
    &\leq
    \|\left(\|s(\gamma)\|_{\gamma}\right)\|_2\|\left(\|t(\gamma)\|_{\gamma}\right)\|_2
    &\text{ By Holder's inequality with $p=q=2$}\\
    &\leq \infty &\text{ Since $\|s(\gamma)\|_{\gamma}$ and
    $\|t(\gamma)\|_{\gamma}$ are in $L^2(\Gamma,\mu)$}
\end{aligned}
    \]
    and thus, the proposed inner product is well-defined.

    Next, we show sesquilinearity. We adopt the mathematics convention that the
    inner product $\la s|t\ra$ is linear in the first term, and conjugate linear
    in the second term. Let $s,t\in H$, and let $\alpha\in \C$.
    Then,
    \[
        \begin{aligned}
        \la \alpha s|t\ra &= \int_{\Gamma}\la \alpha
        s(\gamma)|t(\gamma)\ra d\mu(\gamma)\\
        &= \int_{\Gamma}\alpha\la s(\gamma)|t(\gamma)\ra d\mu(\gamma)\\
        &= \alpha\int_{\Gamma}\la s(\gamma)|t(\gamma)\ra d\mu(\gamma)\\
        &=\alpha\la s|t\ra
    \end{aligned}
    \]
    Furthermore, with $r\in H$ as well, we have
    \[
\begin{aligned}
    \la r+s|t\ra &= \int_{\Gamma}\la r(\gamma)+s(\gamma)|t(\gamma)\ra
    d\mu(\gamma)\\
    &= \int_{\Gamma}\la r(\gamma)|t(\gamma)\ra + \la s(\gamma)|t(\gamma)\ra
    d\mu(\gamma)\\
    &=\int_{\Gamma}\la r(\gamma)|t(\gamma)\ra d\mu(\gamma) + \int_{\Gamma}\la
    s(\gamma)|t(\gamma)\ra d\mu(\gamma)\\
    &=\la r|t\ra + \la s|t\ra
\end{aligned}
    \]
    and so the proposed inner product is linear in the first term. Furthermore,
    we see that
    \[
\begin{aligned}
    \la s|t\ra &= \int_{\Gamma}\la s(\gamma)|t(\gamma)\ra d\mu(\gamma)\\
    &= \int_{\Gamma}\overline{\la t(\gamma)|s(\gamma)\ra} d\mu(\gamma)\\
    &= \overline{\int_{\Gamma}\la t(\gamma)|s(\gamma)\ra d\mu(\gamma)}\\
    &= \overline{\la t|s\ra}
\end{aligned}
    \]
    and so the proposed inner product is conjugate linear in the second term.

    Finally, we need to show this inner product is positive-definite. That is,
    we need to show
    \[
        \la s|s\ra \geq 0
    \]
    with equality if and only if $s=0$.
    So, let $s\in H$. Trivially, if $s=0$, then
    \[
        \la s|s\ra = \int_{\Gamma}\la s(\gamma)|s(\gamma)\ra d\mu(\gamma) =
        \int_{\Gamma}0 = 0
    \]
    So, let $s\neq 0$. Then, in particular, $\|s(\gamma)\|_{\gamma}$ differs
    from zero on a set $E\subset\Gamma$ of positive measure. Thus,
    \[
        \begin{aligned}
            \la s|s\ra &= \int_{\Gamma} \la s(\gamma)|s(\gamma)\ra
            d\mu(\gamma)\\
            &= \int_{\Gamma} \|s(\gamma)\|_{\gamma}^2 d\mu(\gamma)\\
            &\geq \int_E \|s(\gamma)\|_{\gamma}^2d\mu(\gamma)\\
            &\geq 0
    \end{aligned}
    \]
    as desired. Thus, this proposed inner product is indeed an inner product on
    $H$.

    Observe that this completes the first two parts of this problem. We defined
    \[
        \|s\|^2 = \la s|s\ra = \int_{\Gamma}\|s(\gamma)\|_{\gamma}^2
        d\mu(\gamma)
    \]
    and setting $\mu = \mu_c$ the counting measure, we get
    \[
        \|s\|^2 = \sum_{\gamma\in\Gamma}\|s(\gamma)\|_{\gamma}^2
    \]
    which is a norm, since it is induced by an inner product.

    Furthermore, we have defined an inner product on $H$, which makes it a
    Euclidean space. Now, we just have to show that $H$ is actually a Hilbert
    space.


    We need to show that $H$ is complete with respect to its norm. So, suppose
    $s_n\in H$ is a sequence of elements (sections) in $H$ such that $\{s_n\}$
    is a Cauchy sequence with respect to the norm.

    In particular, this means that for $\varepsilon>0$ there is an $N$ such that
    for all $n,m>N$,
    \[
        \|s_n-s_m\|^2 < \varepsilon
    \]
    which translates to
    \[
        \begin{aligned}
            \|s_n-s_m\|^2 = \int_{\Gamma}\|s_n(\gamma) - s_m(\gamma)\|_{\gamma}^2
            d\mu(\gamma) <
            \varepsilon
    \end{aligned}
    \]
    and so each term $\|s_n(\gamma)-s_m(\gamma)\|_{\gamma}^2 < \varepsilon$ as
    well. Thus, each term is Cauchy, and converges to some limit $s(\gamma)$
    pointwise. We now wish to show this converges in norm. 

\end{proof}

\end{document}
