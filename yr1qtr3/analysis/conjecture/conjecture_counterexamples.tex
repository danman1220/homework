\documentclass[12pt]{article}

\usepackage{amsmath, amsthm, amsfonts}
\usepackage{mathrsfs}

\newcommand{\Q}{\mathbb{Q}}
\newcommand{\N}{\mathbb{N}}

\theoremstyle{definition}

\newtheorem{example}{Example}

\begin{document}
\section*{Some Counterexamples to Chuck's Conjecture}

Perhaps the simplest counterexample to Chuck's Conjecture is the following:

\begin{example}
    Let $X$ be a compact metric space, equipped with the Borel $\sigma$-algebra,
    and fix $x\in X$. Suppose $\mu$ is a probability measure on
    $(X,\mathscr{B}(X))$ for which $\{x\}$ is the only atom, with
    $\mu(\{x\})=0.1$. We define a second measure $\nu$ on $X$ as
    \[
        \begin{aligned}
            \nu(\{x\}) = 0.05\\
            \nu(A) = \frac{0.95}{0.9}\mu(A)
        \end{aligned}
    \]
    for all $A\in\mathscr{B}(X)$ measurable subsets of $X$ with $x\not\in A$.
    For example, we can set $X=I$, $\mu = 0.9\lambda^1 + 0.1\delta_x$ and $\nu =
    0.95\lambda^1 + 0.05\delta_x$ so that $\mu(X)=\nu(X)=1$. 

    With such measures $\mu$ and $\nu$, there is no perfect sharing. That is,
    there is no measurable subset $S\in\mathscr{B}(X)$ with
    $\mu(S)=\nu(S)=\frac{1}{2}$. 

    To see this, suppose there did exist a perfect sharing set $S$, and without
    loss of generality take $x\in S$ (if $x\not\in S$, we consider $S^c$ which
    is also a perfect sharing set containing $x$). Now, let $A=S\setminus\{x\}$.
    Then,
    \[
        \frac{1}{2} = \mu(S) = \mu(A) + \mu(\{x\}) = \mu(A) + 0.1
    \]
    and so
    \[
        \mu(A) = 0.4
    \]

    However, we also know that
    \[
        \frac{1}{2} = \nu(S) = \nu(A) + \nu(\{x\}) = \nu(A) + 0.05
    \]
    and so it must be that
    \[
        \nu(A) = 0.45
    \]

    By definition of $\nu$, we have that
    \[
        \nu(A) = \frac{0.95}{0.9}\mu(A) = \frac{0.95}{0.9}(0.4) \approx 0.42\neq
        0.45
    \]
    so no such perfect sharing can exist for these two measures.

\end{example}

The previous example generalizes easily to arbitrary measures which satisfy the
hypotheses.
\begin{example}
    Let $X$ be as before, and let $\mu$ be a probability measure on $X$ which
    satisfies the hypotheses for the conjecture, with atoms
    $P=\{x_i\}_{i=1}^{\infty}$ and $\mu(P)<\frac{1}{2}$. 
    
    Fix $\varepsilon$ so that $0<\varepsilon<0.1$, and that $\mu(P) + \varepsilon <
    \frac{1}{2}$. Fix some $x\in P^c$, and define $\nu$ another probability
    measure on $X$ as 
    \[
        \nu(S) = (1-\varepsilon)\mu(S) + \varepsilon \delta_x(S)
    \]
    for all measurable subsets $S$,
    so that $\nu(X) = 1$. This satisfies the hypotheses for the conjecture, but
    does not allow a perfect sharing.

    To see this, suppose $S$ is a perfect sharing set of $X$, and without loss
    of generality let $x\in S$, and let $A=S\setminus\{x\}$. Then, we have that
    \[
        \frac{1}{2} = \mu(S) = \mu(A) + \mu(\{x\}) = \mu(A)
    \]
    and so 
    \[
        \mu(A) = \frac{1}{2}
    \]

    Similarly, for $\nu$ we calculate
    \[
        \frac{1}{2} = \nu(S) = \nu(A) + \nu(\{x\}) = \nu(A) + \varepsilon
    \]
    and so
    \[
        \nu(A) = \frac{1}{2} - \varepsilon
    \]

    However, we defined $\nu(A)$ to be
    \[
        \nu(A) = (1-\varepsilon)\mu(A) + \varepsilon\delta_x(A) =
        (1-\varepsilon)\mu(A) = \frac{1}{2} - \frac{\varepsilon}{2}\neq
        \frac{1}{2}-\varepsilon
    \]
    and so such a sharing cannot exist.
\end{example}

Finally, we consider a specific counterexample with infinitely many atoms in one
measure that are not in the other.

\begin{example}
    This example comes from Ethan Robinett.

    Let $\lambda$ be the Lebesgue measure on $I^2$. Let $f:\Q^2\cap
    I^2\to\N_{\geq 2}$ be a bijection of the rational points in $I^2$ to the
    natural numbers above 1. Define
    $c = \sum_{i=2}^{\infty}\frac{2}{10^i} = 0.0\bar{2}$, and for any Lebesgue
    measurable set $S\subset I^2$, define
    \[
        \mu(S) = (1-c)\lambda(S) + \sum_{q\in \Q^2\cap S}\frac{2}{10^{f(q)}}
    \]
    Now, clearly $\mu$ is also a Lebesgue measure on $I^2$ with each element of
    $\Q^2\cap I^2$ an atom for $\mu$. Furthermore, each atom $q$ has mass
    $\frac{2}{10^{f(q)}}<0.1$, and the total mass of the atoms is $\mu(\Q^2) =
    c < 0.5$. Thus, $\mu$ also satisfies the hypotheses for the conjecture.
    However, there is no perfect sharing between $\mu$ and $\lambda$.

    Suppose there was a perfect sharing set $S$. Then, we would have
    \[
        \frac{1}{2} = \mu(S) = (1-c)\frac{1}{2} + \sum_{q\in \Q^2\cap
        S}\frac{2}{10^{f(q)}}
    \]
    which forces
    \[
        \sum_{q\in\Q^2\cap S}\frac{2}{10^{f(q)}} = \frac{c}{2} = 0.0\bar{1}
    \]
    which cannot be attained, since the decimal expansion of the left-hand side
    is made up of only $0$ and $2$, whereas the right-hand side has only $0$ and
    $1$ in its expansion. Thus, no such perfect sharing can exist.
\end{example}

\end{document}
