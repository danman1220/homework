\documentclass[12pt]{article}

\usepackage{amsmath, amsthm}
\usepackage{mathrsfs}

\newcommand{\R}{\mathbb{R}}

\theoremstyle{definition}

\newtheorem{example}{Example}

\begin{document}
\section*{A Counterexample to Chuck's Conjecture}

Perhaps the simplest counterexample to Chuck's Conjecture is the following:

\begin{example}
    Let $X$ be a compact metric space, equipped with the Borel $\sigma$-algebra,
    and fix $x\in X$. Suppose $\mu$ is a probability measure on
    $(X,\mathscr{B}(X))$ for which $\{x\}$ is the only atom, with
    $\mu(\{x\})=0.1$. We define a second measure $\nu$ on $X$ as
    \[
        \begin{aligned}
            \nu(\{x\}) = 0.05\\
            \nu(A) = \frac{0.9}{0.95}\mu(A)
        \end{aligned}
    \]
    for all $A\in\mathscr{B}(X)$ measurable subsets of $X$ with $x\not\in A$.
    For example, we can set $X=I$, $\mu = 0.9\lambda^1 + 0.1\delta_x$ and $\nu =
    0.95\lambda^1 + 0.05\delta_x$ so that $\mu(X)=\nu(X)=1$. 

    With such measures $\mu$ and $\nu$, there is no perfect sharing. That is,
    there is no measurable subset $S\in\mathscr{B}(X)$ with
    $\mu(S)=\nu(S)=\frac{1}{2}$. 

    To see this, suppose there did exist a perfect sharing set $S$, and without
    loss of generality take $x\in S$ (if $x\not\in S$, we consider $S^c$ which
    is also a perfect sharing set containing $x$). Now, let $A=S\setminus\{x\}$.
    Then,
    \[
        \frac{1}{2} = \mu(S) = \mu(A) + \mu(\{x\}) = \mu(A) + 0.1
    \]
    and so
    \[
        \mu(A) = 0.4
    \]

    However, we also know that
    \[
        \frac{1}{2} = \nu(S) = \nu(A) + \nu(\{x\}) = \nu(A) + 0.05
    \]
    and so it must be that
    \[
        \nu(A) = 0.45
    \]

    By definition of $\nu$, we have that
    \[
        \nu(A) = \frac{0.9}{0.95}\mu(A) = \frac{0.9}{0.95}(0.4) \approx 0.38\neq
        0.45
    \]
    so no such perfect sharing can exist for these two measures.

\end{example}

The previous example generalizes easily to measures with a finite number of
atoms.

\begin{example}
    let $X$ be as before, and let $\mu$ be a probability measure on $X$ with
    finitely many atoms $P = \{x_i\}_{i=1}^n$ which satisfies the hypotheses of
    the conjecture. Let $c=\mu(P)$, observing that $c<\frac{1}{2}$. Define $\nu$
    a second probability measure on $X$ with
    \[
        \nu(\{x_i\}) = \frac{1}{2}\mu(\{x_i\})
    \]
    so that $\nu(P) = \frac{c}{2}$. Then, for all $A\in\mathscr{B}(X)$ with
    $A\cap P = \emptyset$, define
    \[
        \nu(A) = \frac{1-\frac{c}{2}}{1-c}\mu(A)
    \]
    so that $\nu$ is a constant proportion of $\mu$ on $X\setminus P$, and
    $\nu(X) = 1$. Again, there is no perfect sharing set on $X$ with respect to
    these two measures.

    Suppose such a perfect sharing $S$ existed, and without loss of generality
    suppose $P\cap S\neq \emptyset$. Like before, we set $A=S\setminus P$, and
    calculate
    \[
        \frac{1}{2} = \mu(S) = \mu(A) + \mu(P) = \mu(A) + c
    \]
    and so
    \[
        \mu(A) = \frac{1}{2} - c
    \]

    However, we calculate
    \[
        \frac{1}{2} = \nu(S) = \nu(A) - \nu(P) = \nu(A) - \frac{c}{2}
    \]
    and so it must be that
    \[
        \nu(A) = \frac{1-c}{2}
    \]
    
    But by definition of $\nu$, we have
    \[
        \nu(A) = \frac{1-\frac{c}{2}}{1-c}\mu(A) =
        \frac{1-\frac{c}{2}}{1-c}(\frac{1}{2}-c) =
        \frac{(2-c)(1-2c)}{4(1-c)}
    \]
    which can only be equal to $\frac{1-c}{2}$ for $c=0$. So, for positive $c$
    (which corresponds to $\mu$ having positive-mass singletons) there is no
    perfect sharing between $\mu$ and $\nu$.
\end{example}





\end{document}
