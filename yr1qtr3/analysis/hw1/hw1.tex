%%%%%%%%%%%%%%%%%%%%%%%%%%%%%%%%%%%%%%%%%
% Short Sectioned Assignment
% LaTeX Template
% Version 1.0 (5/5/12)
%
% This template has been downloaded from:
% http://www.LaTeXTemplates.com
%
% Original author:
% Frits Wenneker (http://www.howtotex.com)
%
% License:
% CC BY-NC-SA 3.0 (http://creativecommons.org/licenses/by-nc-sa/3.0/)
%
%%%%%%%%%%%%%%%%%%%%%%%%%%%%%%%%%%%%%%%%%

%----------------------------------------------------------------------------------------
%	PACKAGES AND OTHER DOCUMENT CONFIGURATIONS
%----------------------------------------------------------------------------------------

\documentclass[fontsize=11pt]{scrartcl} % 11pt font size

\usepackage[T1]{fontenc} % Use 8-bit encoding that has 256 glyphs
\usepackage[english]{babel} % English language/hyphenation
\usepackage{amsmath,amsfonts,amsthm} % Math packages
\usepackage{mathrsfs}

\usepackage[margin=1in]{geometry}

\usepackage{sectsty} % Allows customizing section commands
\allsectionsfont{\centering \normalfont\scshape} % Make all sections centered, the default font and small caps

\usepackage{fancyhdr} % Custom headers and footers
\pagestyle{fancyplain} % Makes all pages in the document conform to the custom headers and footers
\fancyhead{} % No page header - if you want one, create it in the same way as the footers below
\fancyfoot[L]{} % Empty left footer
\fancyfoot[C]{} % Empty center footer
\fancyfoot[R]{\thepage} % Page numbering for right footer
\renewcommand{\headrulewidth}{0pt} % Remove header underlines
\renewcommand{\footrulewidth}{0pt} % Remove footer underlines
\setlength{\headheight}{13.6pt} % Customize the height of the header

\numberwithin{equation}{section} % Number equations within sections (i.e. 1.1, 1.2, 2.1, 2.2 instead of 1, 2, 3, 4)
\numberwithin{figure}{section} % Number figures within sections (i.e. 1.1, 1.2, 2.1, 2.2 instead of 1, 2, 3, 4)
\numberwithin{table}{section} % Number tables within sections (i.e. 1.1, 1.2, 2.1, 2.2 instead of 1, 2, 3, 4)

\newcommand{\R}{\mathbb{R}}
\newcommand{\Q}{\mathbb{Q}}
\newcommand{\N}{\mathbb{N}}
\newcommand{\C}{\mathbb{C}}

\newcommand{\im}{\textrm{im}}

\newtheorem{lemma}{Lemma}
%----------------------------------------------------------------------------------------
%	TITLE SECTION
%----------------------------------------------------------------------------------------

\newcommand{\horrule}[1]{\rule{\linewidth}{#1}} % Create horizontal rule command with 1 argument of height

\title{	
\normalfont \normalsize 
\textsc{Analysis} \\ [25pt] % Your university, school and/or department name(s)
\horrule{0.5pt} \\[0.4cm] % Thin top horizontal rule
\huge Homework 1 \\ % The assignment title
\horrule{2pt} \\[0.5cm] % Thick bottom horizontal rule
}

\author{Daniel Halmrast} % Your name

\date{\normalsize\today} % Today's date or a custom date

\begin{document}

\maketitle % Print the title

% Problems

\section*{Problem A}
Let $V$ be the real vector space $\text{Set}(\R,\R)$ of all functions from $\R$
to itself, and let $K = \{f\in V\ |\ \im(f)\subset[0,1]\}$. Prove that $K$ is
convex and find all extreme points and finite-dimensional faces.
\\
\\
\begin{proof}
    To start with, we show that $K$ is convex. Let $f,g\in K$ be arbitrary. We
    will show that the function $h=\lambda f + (1-\lambda)g$ is in $K$ for all
    $\lambda\in [0,1]$. This is clear, however, since for all $x\in \R$,
    \[
        \begin{aligned}
        h(x) &=\lambda f(x) + (1-\lambda)g(x)\\
        &\leq \lambda(1)+ (1-\lambda)(1)\\
        &= 1\\
        \\
        h(x) &=\lambda f(x) + (1-\lambda)g(x)\\
        &\geq \lambda (0) + (1-\lambda)(0)\\
        &=0
    \end{aligned}
    \]
    and so $\im(h)\subset [0,1]$ as desired.

    Next, we wish to find the finite dimensional faces of $K$. I assert that the
    finite dimensional faces of $K$ are defined as follows. First, partition
    $\R$ into three sets $A,N,P$ such that $\|A\|<\infty$. Then, define a face
    \[
        F = \{f\in K\ |\ f^{-1}(\{0\})\supset N, f^{-1}(\{1\})\supset P\}
    \]

    To see that this is a face, we have to check that it is convex, and that it
    contains its linear interpolations.

    So, let $f,g\in F$, and let $\lambda\in [0,1]$. We need to show that
    \[
        h(x) = \lambda f(x) + (1-\lambda)g(x)
    \]
    is in $F$. Clearly $h\in K$ as a convex linear combination of elements of
    $K$, so we only need to examine two cases: $x\in N$ and $x\in P$.

    If $x\in N$, then
    \[
        h(x) = \lambda (0) + (1-\lambda)(0) = 0
    \]
    and so $h^{-1}(\{0\})\supset N$.

    If $x\in P$, then
    \[
        h(x) = \lambda(1) _ (1-\lambda)(1) = 1
    \]
    and so $h^{-1}(\{1\})\supset P$.

    Finally, we show that $F$ contains all its linear interpolations. That is,
    for any $h\in F$, if there exists $f,g\in K$ and $t\in (0,1)$ with
    $h=tf+(1-t)g$, then $f,g\in F$ as well.
    So, suppose $h\in F$ and $f,g$ and $t$ are as described. We examine again
    two cases.

    If $x\in N$, then $h(x)=0$ and so
    \[
        0=tf(x)+(1-t)g(x)
    \]
    but $t\in (0,1)$ and $f,g\geq 0$, so it must be that $f(x) = g(x) = 0$ as
    desired. Thus, $f^{-1}(\{0\})\supset N$ (and similarly for $g$).

    If $x\in P$, then $h(x) = 1$ and so
    \[
        1= tf(x) + (1-t)g(x)
    \]
    but $t\in(0,1)$ and $f(x),g(x)\leq 1$, so it must be that $f(x)=g(x)=1$ as
    desired. Thus $f^{-1}(\{1\})\supset P$ (and similarly for $g$).

    Thus, we have shown that the set $F$ defined this way is a face. Next, we
    show it is finite-dimensional. In particular, we show that the dimension of
    $F$ is $\|A\|$.

    Recall that the dimension of a face is defined as the dimension of
    $\text{span}\{g-f\ |\ g\in F\}$ for some fixed $f\in F$. 
    So, let $f = \chi_P$. I assert that the span of $\{g-f\ |\ g\in F\}$ has a
    basis given by $g_i = \chi_{\{a_i\}}$ for $a_i\in A$.

    First, observe that $\{g_i\}$ is clearly a linearly independent set. Next,
    we observe that any function of the form $g-f$ for $g\in F$ can be written
    as a finite linear combination of the $g_i$ basis functions. This is clear,
    since for any $g\in F$, we know that
    \[
        (g-f)(x) =
        \begin{cases}
            f(x), &x\in A\\
            0, &x\not\in A
        \end{cases}
    \]
    and so
    \[
        (g-f)(x) = \sum_{a_i\in A}f(a_i)g_i(x)
    \]
    as desired.

    I now assert that this describes all the finite-dimensional faces. To see
    this, let $G$ be a face that cannot be described using the construction
    above. In particular, there is some $g\in G$ and some subset $E\subset\R$
    with $\|E\|=\infty$ and $g(E)\in (0,1)$. Since $G$ is a face, this means
    that $G$ contains all functions which agree with $g$ outside $E$.

    Now, fix
    \[
        f(x) =
        \begin{cases}
            g(x), &x\not\in E\\
            0, &x\in E
        \end{cases}
    \]
    Then, for every $e\in E$, the function $h_e = \chi_{\{e\}}$ is in $\{h-f\ |\
    h\in G\}$. Moreover, the collection $\{h_e\}$ is clearly linearly
    independent. Thus, we have found an infinite linearly independent subset of
    $\{h-f\ |\ h\in G\}$, and so $G$ is infinite dimensional.

    Finally, we observe that the extreme points of $K$ are simply the faces
    defined as above with $A=\emptyset$. That is, the extreme points of $K$ are
    the functions $f\in K$ with $f(\R)\in\partial[0,1]$.
\end{proof}

\section*{Problem B}
Do the same, replacing the domain $\R$ with $\N$.
\\
\\
\begin{proof}
    Note that the construction for part $A$ generalizes to functions from
    arbitrary sets into $\R$, and so the finite-dimensional faces and extreme
    points are described in exactly the same way.
\end{proof}

\newpage

\section*{Problem C}
Let $X$ be the real vector space $\text{Set}(\N,\C)$ and let $E=\{f\in X\ |\
|f|(\N)\in[0,1]\}$. Prove $E$ is convex, and find all extreme points and
finite-dimensional faces.
\\
\\
\begin{proof}
    Convexity of $E$ follows almost immediately from the fact that the unit disk
    is convex. That is, for $f,g\in E$ and $\lambda\in [0,1]$, we know that
    \[
\begin{aligned}
        \|\lambda f(x) + (1-\lambda)g(x)\|
        &\leq \lambda \|f(x)\| + (1-\lambda)\|g(x)\|\\
        &\leq \lambda(1) + (1-\lambda)(1)\\
        &=1
\end{aligned}
    \]
    as desired.

    Let a face $F$ be defined as follows. Partition $\N$ into two sets $A,P$
    with $\|A\|<\infty$. Then, fix a function $g_0:\N\to \partial\mathbb{D}$ and
    define
    \[
        F = \{f\in E\ |\ f(x)=g_0(x), x\in P\}
    \]
    I assert that $F$ is a face. To see this, note that $F$ is clearly convex,
    since for any $f,g\in F$ and $\lambda\in[0,1]$ the function
    \[
        h(x) = \lambda f(x) + (1-\lambda)g(x)
    \]
    is such that for any $x\in P$
    \[
        \begin{aligned}
        h(x) &= \lambda f(x) + (1-\lambda)g(x)\\
        &= (\lambda + 1-\lambda)g_0(x)\\
        &=g_0(x)
    \end{aligned}
    \]
    as desired.

    Next, we show $E$ contains things that linearly interpolate to it. Suppose
    $h\in F$ with $f,g\in E$ and $t\in (0,1)$ with $h=tf+(1-t)g$. Then, we know
    that for $x\in P$, $h(x)=g_0(x)$ is in $\partial\mathbb{D}$. In particular,
    $g_0(x)$ is an extreme point, which forces $f(x)=g(x)=g_0(x)$ as desired.

    Thus, $F$ is a face. Next, we observe it is finite-dimensional. To see this,
    we fix $f\in F$ as
    \[
        f(x) = 
        \begin{cases}
            g_0(x), &x\in P\\
            0, &x\not\in P
        \end{cases}
    \]
    and note that the functions
    \[
        g_i(x) =
        \begin{cases}
            1, &x=a_i\\
            0 &x\neq a_i
        \end{cases}
    \]
    for each $a_i\in A$ form the set $\{g_i, ig_i\}$ of linearly independent
    functions that span the set
    \[
        \text{span}(\{g-f\ |\ g\in F\})
    \]
    Thus, $F$ has dimension $2\|A\|$ as desired.

    By a similar argument to the one used in part A, any face not defined this
    way must be infinite-dimensional. Finally, we note the extreme points are
    found by setting $A=\emptyset$.
\end{proof}

\newpage

\section*{Problem D}
Let $Y$ be the real vector space $\text{Top}([0,1],\R)$ and let $D=\{f\in Y\ |\
f([0,1])\subset [0,1]\}$. Prove that $D$ is convex, and find all extreme
points.
\\
\\
\begin{proof}
    We begin by proving that $D$ is convex. So, let $f,g\in D$, and let
    $\lambda\in[0,1]$. Then, we know from part A that $h(x) = \lambda
    f(x)+(1-\lambda)g(x)$ is such that $h([0,1])\subset [0,1]$. Furthermore,
    since $h$ is a linear combination of continuous functions, it is continuous
    as well. Thus, $D$ is convex.

    I assert that the only extreme points of $D$ are the constant functions
    $f_1(x) = 1$ and $f_0(x) = 0$. Now, these are clearly extreme points, since
    $1$ and $0$ are extreme points of $[0,1]$. That is, if $f,g$ are such that
    for some $t\in (0,1)$, $f_1(x)=1 = tf(x) + (1-t)g(x)$, then $f(x) = g(x) =
    1$ for all $x$ (and similarly for $f_0$).

    Now, suppose $f\in D$ with $f$ not equal to the constant functions $f_0$ and
    $f_1$. Then, there is some $x$ for which $f(x)\neq 1$ and $f(x)\neq 0$.
    Furthermore, by continuity there is some interval $(a,b)$ around $x$ for
    which $f$ is not $1$ or $0$. Clearly, there exists functions $g,h\in D$
    and $f(x) = tg(x) + (1-t)h(x)$, with $g,h\neq f$ (take, for example,
        functions $g$ and $h$ that agree with $f$ outside $(a,b)$, and vary an
        equal distance away from $f$ in the positive and negative directions.
        Perhaps one could take $\varepsilon$ such that $f(x)\pm \varepsilon\in
        (0,1)$ for all $x\in (a,b)$, and take 
        \[
            g(x) = f(x)+\varepsilon\sin(\frac{\pi}{b-a}x)
        \]
        and 
        \[
            h(x) = f(x)-\varepsilon\sin(\frac{\pi}{b-a}x)
        \]
    for $x\in (a,b)$).
    
    Thus, $f$ is not an
    extreme point, as desired.
\end{proof}

\newpage

\section*{Problem E}
Let $U$ be the real vector space $\text{Top}([0,1],\C)$ and let $F = \{f\in U\
|\ |f|([0,1])\subset [0,1]\}$. Prove $F$ is convex, and find all extreme points.
\\
\\
\begin{proof}
    The fact that $F$ is convex follows from the triangle inequality. That is,
    for any $f,g\in F$ and $\lambda\in [0,1]$ we know that for $h(x) = \lambda
    f(x) + (1-\lambda)g(x)$ we have
    \[
\begin{aligned}
    \|h(x)\| &= \|\lambda f(x) + (1-\lambda)g(x)\|\\
    &\leq \lambda\|f(x)\| + (1-\lambda)\|g(x)\|\\
    &\leq \lambda (1) + (1-\lambda)(1)\\
    &=1
\end{aligned}
    \]
    as desired. Furthermore, $h$ is indeed continuous since it is a linear
    combination of continuous functions.

    Furthermore, I assert that the extreme points are the functions in $F$ whose
    image lies entirely in $\partial\mathbb{D}$. To see these are extreme
    points, observe that if $h\in F$ is such that the image of $h$ lies inside
    $\partial\mathbb{D}$, then for each $x\in[0,1]$, $h(x)$ is an extreme point
    of $\mathbb{D}$. Thus, if
    \[
        h(x) = tf(x) + (1-t)g(x)
    \]
    for some $f,g\in F$ and $t\in (0,1)$, then $f(x)=g(x)=h(x)$ (since $h(x)$ is
    an extreme point). Thus, $h$ is an extreme point of $F$.

    Furthermore, these exhaust the extreme points. To see this, suppose $h$ is
    such that $h(x_0)\not\in\partial\mathbb{D}$ for some $x_0$. By continuity, we
    know that $h(x)\not\in\partial\mathbb{D}$ for all $x$ in some interval
    $(a,b)$ around $x_0$. Thus, by a similar construction to the one in part D,
    we can find $f$ and $g$ in $F$ which linearly interpolate to $h$. Thus, $h$
    cannot be an extreme point.
\end{proof}

\end{document}
