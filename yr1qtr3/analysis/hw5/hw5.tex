%%%%%%%%%%%%%%%%%%%%%%%%%%%%%%%%%%%%%%%%%
% Short Sectioned Assignment
% LaTeX Template
% Version 1.0 (5/5/12)
%
% This template has been downloaded from:
% http://www.LaTeXTemplates.com
%
% Original author:
% Frits Wenneker (http://www.howtotex.com)
%
% License:
% CC BY-NC-SA 3.0 (http://creativecommons.org/licenses/by-nc-sa/3.0/)
%
%%%%%%%%%%%%%%%%%%%%%%%%%%%%%%%%%%%%%%%%%

%----------------------------------------------------------------------------------------
%	PACKAGES AND OTHER DOCUMENT CONFIGURATIONS
%----------------------------------------------------------------------------------------

\documentclass[fontsize=11pt]{scrartcl} % 11pt font size

\usepackage[T1]{fontenc} % Use 8-bit encoding that has 256 glyphs
\usepackage[english]{babel} % English language/hyphenation
\usepackage{amsmath,amsfonts,amsthm} % Math packages
\usepackage{mathrsfs}

\usepackage[margin=1in]{geometry}

\usepackage{sectsty} % Allows customizing section commands
\allsectionsfont{\centering \normalfont\scshape} % Make all sections centered, the default font and small caps

\usepackage{fancyhdr} % Custom headers and footers
\pagestyle{fancyplain} % Makes all pages in the document conform to the custom headers and footers
\fancyhead{} % No page header - if you want one, create it in the same way as the footers below
\fancyfoot[L]{} % Empty left footer
\fancyfoot[C]{} % Empty center footer
\fancyfoot[R]{\thepage} % Page numbering for right footer
\renewcommand{\headrulewidth}{0pt} % Remove header underlines
\renewcommand{\footrulewidth}{0pt} % Remove footer underlines
\setlength{\headheight}{13.6pt} % Customize the height of the header

\numberwithin{equation}{section} % Number equations within sections (i.e. 1.1, 1.2, 2.1, 2.2 instead of 1, 2, 3, 4)
\numberwithin{figure}{section} % Number figures within sections (i.e. 1.1, 1.2, 2.1, 2.2 instead of 1, 2, 3, 4)
\numberwithin{table}{section} % Number tables within sections (i.e. 1.1, 1.2, 2.1, 2.2 instead of 1, 2, 3, 4)

\newcommand{\R}{\mathbb{R}}
\newcommand{\Q}{\mathbb{Q}}
\newcommand{\N}{\mathbb{N}}
\newcommand{\C}{\mathbb{C}}

\newtheorem{lemma}{Lemma}
%----------------------------------------------------------------------------------------
%	TITLE SECTION
%----------------------------------------------------------------------------------------

\newcommand{\horrule}[1]{\rule{\linewidth}{#1}} % Create horizontal rule command with 1 argument of height

\title{	
\normalfont \normalsize 
\textsc{Analysis} \\ [25pt] % Your university, school and/or department name(s)
\horrule{0.5pt} \\[0.4cm] % Thin top horizontal rule
\huge Homework 5 \\ % The assignment title
\horrule{2pt} \\[0.5cm] % Thick bottom horizontal rule
}

\author{Daniel Halmrast} % Your name

\date{\normalsize\today} % Today's date or a custom date

\begin{document}

\maketitle % Print the title

% Problems
\section*{Problem 1}
Let $E,F$ be closed subspaces of a Hilbert space. Prove that $P_EP_F=P_E$ if and
only if $E\subseteq F$.
\\
\\
\begin{proof}
    Suppose first that $P_EP_F=P_E$. Then, in particular,
    \[
        P_E^* = (P_EP_F)^* = P_F^*P_E^* = P_FP_E = P_E
    \]
    by self-adjointness of projections. Thus, $P_EP_F=P_FP_E=P_E$, and the
    projections commute. Thus, the von Neumann algebra $W^*(P_E,P_F,I)$ is
    Abelian, and is isometrically $*$-isomorphic to $L^{\infty}(X,\mu)$ for some
    measure space $(X,\mu)$. In particular, the projections $P_E,P_F$ get sent
    to self-adjoint idempotents $P_E\mapsto M_{\chi_S}$ and $P_F\mapsto
    M_{\chi_{S'}}$ for some measurable subsets $S,S'\subset X$.

    Now, the requirement $P_FP_E=P_E$ corresponds to the requirement
    \[
        M_{\chi_{S'}}M_{\chi_S}=M_{\chi_S}
    \]
    which means that $S\subset S'$. This, in turn, implies that $E\subset F$.

    Indeed, $E$ is the subspace of $H$ on which $P_E$ is the identity, which
    corresponds to the subspace
    \[
        \tilde{E} = \int_S^{\oplus}H(x)d\mu(x)
    \]
    on which $M_{\chi_S}$ is the identity. Similarly,
    \[
        \tilde{F} = \int_{S'}^{\oplus}H(x)d\mu(x)
    \]
    Clearly, $\tilde{E}\subset \tilde{F}$ (since $S\subset S'$) and so $E\subset
    F$ as well.
    \\
    \\
    For the converse direction, assume that $E\subset F$. Then, on $F$, $P_F =
    I_F$, and $P_FP_E = I_FP_E = P_E$. Furthermore, on $F^{\perp}$, $P_FP_E = 0
    = P_E$. Thus, on all of $H=F\oplus F^{\perp}$, $P_FP_E=P_E$ as desired.
\end{proof}

\newpage

\section*{Problem 2}
Characterize the closed subspaces $E,F$ of a Hilbert space $H$ satisfy
$P_FP_E=P_EP_F$.
\\
\\
\begin{proof}
    I assert that $P_E$ and $P_F$ commute if and only if $H$ can be decomposed
    into the four orthogonal components
    \[
        H = E\cap F\oplus E\cap F^{\perp}\oplus E^{\perp}\cap F\oplus
        E^{\perp}\cap F^{\perp}
    \]

    To see this, suppose first that $P_E$ and $P_F$ commute. Then, consider the
    Abelian von Neumann algebra $W^*(P_E,P_F,I)$. Like before, we use the Borel
    functional calculus to identify $W^*(P_E,P_F,I)$ with $L^{\infty}(X,\mu)$
    acting on $\tilde{H} = \int_X^{\oplus}H(x)d\mu(x)$, which is unitarily
    equivalent to $H$ (in the sense that there is a unitary transformation $U$
    such that $\tilde{H}=UH$ and $UW^*(P_E,P_F,I)U^*=L^{\infty}(X,\mu)$).

    Under this identification, $P_E\mapsto M_{\chi_S}$, and $P_F\mapsto
    M_{\chi_{S'}}$ for some measurable subsets $S,S'\subset X$. In particular,
    this decomposes $\tilde{H}$ into four orthogonal components
    \[
        \begin{aligned}
            \tilde{H} &= \int_X^{\oplus}H(x)d\mu(x)\\
            &= \int_{S\cap S'}^{\oplus}H(x)d\mu(x)\oplus 
            \int_{S\cap S'^c}^{\oplus}H(x)d\mu(x)\oplus
            \int_{S'\cap S^c}^{\oplus}H(x)d\mu(x)\oplus
            \int_{S^c\cap S'^c}^{\oplus}H(x)d\mu(x)
        \end{aligned}
    \]
    which translates into the decomposition on $H$ as
    \[
        H = E\cap F\oplus E\cap F^{\perp}\oplus E^{\perp}\cap F\oplus
        E^{\perp}\cap F^{\perp}
    \]
    as desired.
    \\
    \\
    Conversely, suppose $H$ can be decomposed this way. Then, let $v\in H$ be
    decomposed as 
    \[
        v= v_1 + v_2 + v_3 + v_4
    \]
    where $v_1\in E\cap F$, $v_2\in E\cap F^{\perp}$, $v_3\in E^{\perp}\cap F$
    and $v_4\in E^{\perp}\cap F^{\perp}$.

    We compute the effect of $P_EP_F$ and $P_FP_E$ directly.
    \[
\begin{aligned}
    P_EP_F(v) &= P_EP_F(v_1+v_2+v_3+v_4)\\
    &= P_E(v_1 + v_3)\\
    &= v_1\\
    P_FP_E(v) &= P_FP_E(v_1+v_2+v_3+v_4)\\
    &=P_F(v_1+v_2)\\
    &=v_1
\end{aligned}
    \]
    and thus $P_EP_F=P_FP_E$ as desired.
\end{proof}

\newpage

\section*{Problem 3}
Let $E,F$ be closed subspaces of a Hilbert space $H$. An operator $U$ is said to
be a partial isometry from $E$ to $F$ if $U|_E$ is an isometry onto $F$, and
$U|_{E^{\perp}}=0$.
Prove that $U$ is a partial isometry $\iff$ $U^*U$ is a projection $\iff$ $UU^*$
is a projection.
\\
\\
\begin{proof}
    Suppose first that $U$ is a partial isometry from $E$ to $F$. I assert that
    $U^*U = P_E$. To see this, suppose $e\in E$, $v\in H$ and let $v=e'+v'$
    where $e'\in E$ and $v'\in E^{\perp}$. Then,
    \[
        \begin{aligned}
            \langle U^*Ue|v\rangle &= \langle U^*Ue|e'+v'\rangle\\
            &= \langle Ue|Ue'\rangle + \langle Ue|Uv'\rangle\\
            &= \langle e|e'\rangle + 0\\
            \implies \langle U^*Ue-e|v\rangle&=0
        \end{aligned}
    \]
    and since this holds for all $v\in H$, $U^*Ue-e=0$ and thus $U^*Ue=e$ and
    $U^*U$ is the identity on $E$.

    Furthermore, for $v'\in E^{\perp}$,
    \[
        U^*Uv'=U^*(0) = 0
    \]
    and so $U^*U$ is the zero map on $E^{\perp}$. Thus, $U^*U$ agrees with $P_E$
    at all points, so $U^*U=P_E$ as desired.
    \\
    \\
    Conversely, suppose $U^*U$ is a projection $P_E$ for some closed subspace
    $E$. Define $F=U(E)$. We will first show that $U$ is an isometry of $E$ onto
    $F$. To see this, suppose $e,e'\in E$. We calculate directly
    \[
\begin{aligned}
    \langle Ue|Ue'\rangle &= \langle U^*Ue|e'\rangle\\
    &=\langle P_Ee|e'\rangle\\
    &=\langle e|e'\rangle
\end{aligned}
    \]
    and thus $U$ is an isometry from $E$ to $F$. Note that this immediately
    implies that $F$ is a closed subspace. Finally, we show that
    $U|_{E^{\perp}}=0$. Let $v'\in E^{\perp}$, Then,
    \[
\begin{aligned}
    \|Uv'\| &=\langle Uv'|Uv'\rangle\\
    &=\langle U^*Uv'|v'\rangle\\
    &=\langle 0|v'\rangle = 0
\end{aligned}
    \]
    and so $U^*U|_{E^{\perp}}=0$ as desired. Thus, $U$ is a partial isometry.
    \\
    \\
    Finally, we show that if $U$ is a partial isometry, then $U^*$ is a partial
    isometry, which will complete the proof. So, suppose $U$ is a partial
    isometry. We will show that $U^*$ is a partial isometry from $F$ to $E$.
    First, we check that $U^*|_F$ is an isometry. Now, for each $f\in F$, there
    is some $e\in E$ with $Ue=f$. Thus, for $e,e'\in E$, $f,f'\in F$ with
    $Ue=f,Ue'=f'$,
    \[
        \begin{aligned}
            \langle U^*f|U^*f'\rangle &= \langle U^*Ue|U^*Ue\rangle\\
            &=\langle P_Ee|P_Ee'\rangle\\
            &=\langle e|e'\rangle\\
            &=\langle Ue|Ue'\rangle\\
            &=\langle f|f'\rangle
        \end{aligned}
    \]
    and so $U^*$ is an isometry from $F$ to $E$.
    Finally, we show that $U^*|_{F^{\perp}}=0$. This is immediate, since
    \[
        \begin{aligned}
            \ker U^* = U(H)^{\perp} = F^{\perp}
        \end{aligned}
    \]
    where we used the identity $\ker A^* = A(H)^{\perp}$ for all bounded
    operators $A$. 

    Note that this completes the proof. If $U$ is a partial isometry, then $U^*$
    is a partial isometry, which implies that $U^{**}U^*=UU^*$ is a projection.
    Similarly, if $UU^*=U^{**}U^*$ is a projection, then $U^*$ is a partial
    isometry, and thus $U^{**}=U$ is a partial isometry as well.
\end{proof}

\newpage

\section*{Problem 4}

\end{document}
