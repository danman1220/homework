%%%%%%%%%%%%%%%%%%%%%%%%%%%%%%%%%%%%%%%%%
% Short Sectioned Assignment
% LaTeX Template
% Version 1.0 (5/5/12)
%
% This template has been downloaded from:
% http://www.LaTeXTemplates.com
%
% Original author:
% Frits Wenneker (http://www.howtotex.com)
%
% License:
% CC BY-NC-SA 3.0 (http://creativecommons.org/licenses/by-nc-sa/3.0/)
%
%%%%%%%%%%%%%%%%%%%%%%%%%%%%%%%%%%%%%%%%%

%----------------------------------------------------------------------------------------
%	PACKAGES AND OTHER DOCUMENT CONFIGURATIONS
%----------------------------------------------------------------------------------------

\documentclass[fontsize=11pt]{scrartcl} % 11pt font size

\usepackage[T1]{fontenc} % Use 8-bit encoding that has 256 glyphs
\usepackage[english]{babel} % English language/hyphenation
\usepackage{amsmath,amsfonts,amsthm} % Math packages
\usepackage{mathrsfs}

\usepackage[margin=1in]{geometry}

\usepackage{sectsty} % Allows customizing section commands
\allsectionsfont{\centering \normalfont\scshape} % Make all sections centered, the default font and small caps

\usepackage{fancyhdr} % Custom headers and footers
\pagestyle{fancyplain} % Makes all pages in the document conform to the custom headers and footers
\fancyhead{} % No page header - if you want one, create it in the same way as the footers below
\fancyfoot[L]{} % Empty left footer
\fancyfoot[C]{} % Empty center footer
\fancyfoot[R]{\thepage} % Page numbering for right footer
\renewcommand{\headrulewidth}{0pt} % Remove header underlines
\renewcommand{\footrulewidth}{0pt} % Remove footer underlines
\setlength{\headheight}{13.6pt} % Customize the height of the header

\numberwithin{equation}{section} % Number equations within sections (i.e. 1.1, 1.2, 2.1, 2.2 instead of 1, 2, 3, 4)
\numberwithin{figure}{section} % Number figures within sections (i.e. 1.1, 1.2, 2.1, 2.2 instead of 1, 2, 3, 4)
\numberwithin{table}{section} % Number tables within sections (i.e. 1.1, 1.2, 2.1, 2.2 instead of 1, 2, 3, 4)

\newcommand{\R}{\mathbb{R}}
\newcommand{\Q}{\mathbb{Q}}
\newcommand{\N}{\mathbb{N}}
\newcommand{\C}{\mathbb{C}}

\newtheorem{lemma}{Lemma}
%----------------------------------------------------------------------------------------
%	TITLE SECTION
%----------------------------------------------------------------------------------------

\newcommand{\horrule}[1]{\rule{\linewidth}{#1}} % Create horizontal rule command with 1 argument of height

\title{	
\normalfont \normalsize 
\textsc{Analysis} \\ [25pt] % Your university, school and/or department name(s)
\horrule{0.5pt} \\[0.4cm] % Thin top horizontal rule
\huge Homework 5 \\ % The assignment title
\horrule{2pt} \\[0.5cm] % Thick bottom horizontal rule
}

\author{Daniel Halmrast} % Your name

\date{\normalsize\today} % Today's date or a custom date

\begin{document}

\maketitle % Print the title

% Problems
\section*{Problem 1}
Let $E,F$ be closed subspaces of a Hilbert space. Prove that $P_EP_F=P_E$ if and
only if $E\subseteq F$.
\\
\\
\begin{proof}
    Suppose first that $P_EP_F=P_E$. Then, in particular,
    \[
        P_E^* = (P_EP_F)^* = P_F^*P_E^* = P_FP_E = P_E
    \]
    by self-adjointness of projections. Thus, $P_EP_F=P_FP_E=P_E$, and the
    projections commute. Thus, the von Neumann algebra $W^*(P_E,P_F,I)$ is
    Abelian, and is isometrically $*$-isomorphic to $L^{\infty}(X,\mu)$ for some
    measure space $(X,\mu)$. In particular, the projections $P_E,P_F$ get sent
    to self-adjoint idempotents $P_E\mapsto M_{\chi_S}$ and $P_F\mapsto
    M_{\chi_{S'}}$ for some measurable subsets $S,S'\subset X$.

    Now, the requirement $P_FP_E=P_E$ corresponds to the requirement
    \[
        M_{\chi_{S'}}M_{\chi_S}=M_{\chi_S}
    \]
    which means that $S\subset S'$. This, in turn, implies that $E\subset F$.

    Indeed, $E$ is the subspace of $H$ on which $P_E$ is the identity, which
    corresponds to the subspace
    \[
        \tilde{E} = \int_S^{\oplus}H(x)d\mu(x)
    \]
    on which $M_{\chi_S}$ is the identity. Similarly,
    \[
        \tilde{F} = \int_{S'}^{\oplus}H(x)d\mu(x)
    \]
    Clearly, $\tilde{E}\subset \tilde{F}$ (since $S\subset S'$) and so $E\subset
    F$ as well.
    \\
    \\
    For the converse direction, assume that $E\subset F$. Then, on $F$, $P_F =
    I_F$, and $P_FP_E = I_FP_E = P_E$. Furthermore, on $F^{\perp}$, $P_FP_E = 0
    = P_E$. Thus, on all of $H=F\oplus F^{\perp}$, $P_FP_E=P_E$ as desired.
\end{proof}

\newpage

\section*{Problem 2}
Characterize the closed subspaces $E,F$ of a Hilbert space $H$ satisfy
$P_FP_E=P_EP_F$.
\\
\\
\begin{proof}
    I assert that $P_E$ and $P_F$ commute if and only if $H$ can be decomposed
    into the four orthogonal components
    \[
        H = E\cap F\oplus E\cap F^{\perp}\oplus E^{\perp}\cap F\oplus
        E^{\perp}\cap F^{\perp}
    \]

    To see this, suppose first that $P_E$ and $P_F$ commute. Then,
\end{proof}<++>


\end{document}
