%%%%%%%%%%%%%%%%%%%%%%%%%%%%%%%%%%%%%%%%%
% Short Sectioned Assignment
% LaTeX Template
% Version 1.0 (5/5/12)
%
% This template has been downloaded from:
% http://www.LaTeXTemplates.com
%
% Original author:
% Frits Wenneker (http://www.howtotex.com)
%
% License:
% CC BY-NC-SA 3.0 (http://creativecommons.org/licenses/by-nc-sa/3.0/)
%
%%%%%%%%%%%%%%%%%%%%%%%%%%%%%%%%%%%%%%%%%

%----------------------------------------------------------------------------------------
%	PACKAGES AND OTHER DOCUMENT CONFIGURATIONS
%----------------------------------------------------------------------------------------

\documentclass[fontsize=11pt]{scrartcl} % 11pt font size

\usepackage[T1]{fontenc} % Use 8-bit encoding that has 256 glyphs
\usepackage[english]{babel} % English language/hyphenation
\usepackage{amsmath,amsfonts,amsthm} % Math packages
\usepackage{mathrsfs}

\usepackage[margin=1in]{geometry}

\usepackage{sectsty} % Allows customizing section commands
\allsectionsfont{\centering \normalfont\scshape} % Make all sections centered, the default font and small caps

\usepackage{fancyhdr} % Custom headers and footers
\pagestyle{fancyplain} % Makes all pages in the document conform to the custom headers and footers
\fancyhead{} % No page header - if you want one, create it in the same way as the footers below
\fancyfoot[L]{} % Empty left footer
\fancyfoot[C]{} % Empty center footer
\fancyfoot[R]{\thepage} % Page numbering for right footer
\renewcommand{\headrulewidth}{0pt} % Remove header underlines
\renewcommand{\footrulewidth}{0pt} % Remove footer underlines
\setlength{\headheight}{13.6pt} % Customize the height of the header

\numberwithin{equation}{section} % Number equations within sections (i.e. 1.1, 1.2, 2.1, 2.2 instead of 1, 2, 3, 4)
\numberwithin{figure}{section} % Number figures within sections (i.e. 1.1, 1.2, 2.1, 2.2 instead of 1, 2, 3, 4)
\numberwithin{table}{section} % Number tables within sections (i.e. 1.1, 1.2, 2.1, 2.2 instead of 1, 2, 3, 4)

\newcommand{\R}{\mathbb{R}}
\newcommand{\Q}{\mathbb{Q}}
\newcommand{\N}{\mathbb{N}}
\newcommand{\C}{\mathbb{C}}

\newtheorem{lemma}{Lemma}
%----------------------------------------------------------------------------------------
%	TITLE SECTION
%----------------------------------------------------------------------------------------

\newcommand{\horrule}[1]{\rule{\linewidth}{#1}} % Create horizontal rule command with 1 argument of height

\title{	
\normalfont \normalsize 
\textsc{Geometry} \\ [25pt] % Your university, school and/or department name(s)
\horrule{0.5pt} \\[0.4cm] % Thin top horizontal rule
\huge Homework 1 \\ % The assignment title
\horrule{2pt} \\[0.5cm] % Thick bottom horizontal rule
}

\author{Daniel Halmrast} % Your name

\date{\normalsize\today} % Today's date or a custom date

\begin{document}

\maketitle % Print the title

% Problems
\section*{Problem 1} %Do Carmo 6.3
Suppose $M$ is a complete Riemannian manifold of dimension $n$, and suppose
there exist constants $a>0$ and $c\geq 0$ such that for all pairs of points and
all minimizing geodesics $\gamma(s)$ parameterized by arc length $s$, joining
these points, we have
\[
    R(\gamma'(s))\geq a + \partial_s f
\]
where $f$ is a function of $s$ and $|f(s)|\leq c$ along $\gamma$. Prove that $M$
is compact.
\\
\\
\begin{proof}
    To show $M$ is compact, we just need to show $M$ is bounded. That is, there
    is some number $N$ such that $d(p,q)<N$ for all $p,q\in M$.

    Let $\gamma$ be a minimizing geodesic connecting two points $p,q$ in $M$. We
    calculate the second variation in energy along $\gamma$ in a manner similar
    to the proof of the Bonnet-Meyers theorem in Do Carmo. In particular, we
    know
    \[
\begin{aligned}
    \frac{1}{2}E''(0) &= \int_0^1\sin^2(\pi t)((n-1)\pi^2 - (n-1)\ell^2
    R_{\gamma}(e_n(t)))dt\\
    &=(n-1)\left[ \frac{\pi^2}{2}- \int_0^1 \ell^2R_{\gamma}(e_n(t))\right]\\
    &\leq (n-1)\left[ \frac{\pi^2}{2} - \int_0^1\sin^2(\pi s)\ell^2 \partial_s f
    ds\right]\\
    &= (n-1)\left[ \frac{\pi^2}{2} - \int_0^1\sin^2(\pi s)\ell^2 (a+ \partial_s
        f) ds\right]\\
    &= (n-1)\left[ \frac{\pi^2-\ell^2a}{2} - \int_0^1\sin^2(\pi s)\ell^2\partial_s
    f ds\right]\\
    &= (n-1)\left[ \frac{\pi^2-\ell^2a}{2} + \int_0^1\pi\sin(\pi s)\ell^2
    f ds\right]\\
    &\leq (n-1)\left[ \frac{\pi^2-\ell^2a}{2} + \int_0^1\pi c|\sin(\pi s)|\ell^2
    ds\right]\\
    &= (n-1)\left[ \frac{\pi^2-\ell^2a}{2} + \pi c\ell^2(\frac{2}{\pi}\right]\\
    &= (n-1)\left[ \frac{\pi^2-\ell^2a}{2} + 2c\ell^2\right]\\
\end{aligned}
    \]
    which is negative for $\ell^2 \geq \frac{\pi^2}{a-4c}$. Since $\gamma$ is
    assumed to be minimal, this implies that $\ell^2 < \frac{\pi^2}{a-4c}$ and
    thus $M$ is bounded with the diameter of $M$ less than
    $\frac{\pi}{\sqrt{a-4c}}$ as desired.
\end{proof}

\newpage

\section*{Problem 2}
Let $M^n$ be an orientable Riemannian manifold with positive curvature, and even
dimension. Let $\gamma$ be a closed geodesic in $M^n$. Prove that $\gamma$ is
homotopic to closed curve whose length is strictly less than $\gamma$.
\\
\\
\begin{proof}
    We first prove the statement assumed in the book: namely, that the parallel
    transport map along $\gamma$ leaves some vector $v$ orthogonal to
    $\gamma'(0)$ invariant. To see this, note that the linear transformation
    \[
\begin{aligned}
    A&:T_{\gamma(0)}M\to T_{\gamma(0)}M\\
    A(v) &= P_{\gamma}(v)
\end{aligned}
    \]
    where $P_{\gamma}(v)$ is the parallel transport of $v$ along $\gamma$ back
    to itself. In particular, $A(\gamma'(0)) = 0$, and so the subspace of
    $T_{\gamma(0)}M$ perpendicular to $\gamma'(0)$ is invariant under $A$. Thus,
    $A$ restricts to a map on the orthogonal complement (of odd dimension). $A$
    is just parallel transport, so it is an isometry. Furthermore, $A$ preserves
    orientation (since $M$ is orientable), so $A$ is an orthogonal transformation
        with determinant $1$ on an odd-dimensional subspace, and thus must have
        an eigenvalue of $1$. Take $v$ to be the corresponding eigenvector.

    Now, let $V(t)$ be the parallel transport of $v$ along $\gamma$. We
    calculate $E''(0)$ directly. Using the fact that $\gamma(0)=\gamma(a)$ and
    for $f$ the variational field corresponding to $V$, $f(0,0)=f(0,a)$, we see
    \[
        \begin{aligned}
            \frac{1}{2}E''(0) &= \int_0^a g(V'.V') - g(R(\gamma',V)\gamma',V)dt
    \end{aligned}
    \]
    However, $V'(t) = \nabla_{\gamma(t)}V(t) = 0$ since $V$ is
    parallel-transported. Thus,
    \[
        \frac{1}{2}E''(0) = -\int_0^a g(R(\gamma',V)\gamma',V)dt
    \]
    and since the space has positive curvature, $g(R(\gamma',V)\gamma',V) =
    K(\gamma',V) >0$ always. Thus,
    \[
        \frac{1}{2}E''(0) < 0
    \]
    which implies there is another curve near to $\gamma$ with smaller length,
    as desired.
\end{proof}

\newpage

\section*{Problem 3}
Let $\tilde{M}$ be a complete simply-connected Riemannian manifold, with
curvature $K\leq 0$. Let $\gamma$ be a normalized geodesic, and let $p\in
\tilde{M}$ be a point which does not belong to $\gamma$. Let $d(s) =
d(p,\gamma(s))$.

\subsection*{Part a}
Consider the minimizing geodesic $\sigma_s$ joining $p$ to $\gamma(s)$, and
consider the variation $h(t,s) = \sigma_s(t)$. Show that
\[
    \frac{1}{2}E'(s) = g(\gamma'(s),\sigma_s'(d(s)))
\]
and
\[
    \frac{1}{2}E''(s_0) >0
\]
if $d'(s_0)=0$.
\\
\\

\begin{proof}
    We proceed in the same way Do Carmo does in the proof for the first
    variation of energy formula. Now, we know that
    \[
        \frac{1}{2}E'(s) = g(\partial_sh,\partial_th)|_0^{d(s)} -
        \int_0^{d(s)}g(\partial_s h,\nabla_t\partial_th)dt
    \]
    We calculate that
    \[
\begin{aligned}
    \partial_t h(t,s) &= \sigma_s'(t))\\
    \partial_s h(d(s),s) &= \gamma'(s)\\
    \partial_s h(0,s) &= 0
\end{aligned}
    \]
    which follows from the chain rule, the fact that $\sigma_s(d(s)) =
    \gamma(s)$, and the fact that $\sigma_s(0) = p$. Thus,
    \[
        \begin{aligned}
        \frac{1}{2}E'(s) &= g(\gamma'(s),\sigma_s'(d(s))) -
        \int_0^{d(s)}g(\partial_s h,\nabla_t\sigma_s'(t))dt\\
        &= g(\gamma'(s),\sigma_s'(d(s))) -
        \int_0^{d(s)}g(\partial_s h,0)dt\\
        &=g(\gamma'(s),\sigma_s'(d(s)))
    \end{aligned}
    \]
    as desired.
    
    Next, we examine the second variation of energy. In particular, we calculate
    \[
        \begin{aligned}
            \frac{1}{2}E''(s) &= \partial_s g(\gamma'(s),\sigma_s'(d(s)))\\
            &= g(\nabla_s\gamma'(s),\sigma_s'(d(s))) +
            g(\gamma'(s),\nabla_s\sigma_s'(d(s)))\\
            &= g(0,\sigma_s'(d(s))) +
            g(\gamma'(s),\nabla_s\sigma_s'(d(s)))\\
            &=g(\gamma'(s),\nabla_s\sigma_s'(d(s)))\\
            &= g(\gamma'(s), \gamma'(s))
        \end{aligned}
    \]
    where the last equality is obtained by assuming $d'(s) = 0$. This is always
    positive, as desired.
\end{proof}

\subsection*{Part b}
Show that $s_0$ is a critical point if and only if
$g(\gamma'(s_0),\sigma_s'(d(s_0))) = 0$, and conclude that $d$ has a unique
critical point, which is a minimum.
\\
\\
\begin{proof}
We conclude form the first equation that $s_0$ is a critical point ($E'(s_0)=0$)
if and only if $g(\gamma'(s_0),\sigma_s'(d(s_0))) = 0$ by reading directly off
the formula for $E'(s)$.

Furthermore, the second equation tells us that if $s_0$ is a critical point of
$d$ ($d'(s_0)=0$), then $E''(s_0)$ is positive, and thus the geodesics
connecting $p$ to points around $\gamma(s_0)$ get longer, and so $d(s_0)$ is a
local minimum. I assert this is a unique minimum. This, however, follows easily
from the fact that $d$ has no maxima, since each critical point is a minimum.
Thus, the critical point for which $d'(s_0)=0$ is unique. 
\end{proof}

\subsection*{Part c}
Provide examples where this fails if $M$ is not simply connected, or does not
have nonpositive curvature.
\\
\\
\begin{proof}
    Consider the flat torus, which has zero curvature, but is not simply
    connected. In particular, coordinatize the flat torus as the unit square
    with the ends identified. Let $\gamma$ be the geodesic running along
    $(\frac{1}{4},t)$ and $(\frac{3}{4},t)$, and let
    $p=(\frac{1}{2},\frac{1}{2})$. Then, there are two points on $\gamma$ that
    are minimally close to $p$, which contradicts the last conclusion of part
    $b$.

    On the other hand, consider the sphere $S^2$ which has positive curvature.
    Let $\gamma$ be the equator, and $p$ be the north pole. $p$ is equidistant
    from every point on $\gamma$, which also contradicts the conclusion in part
    $b$.
\end{proof}


\end{document}
