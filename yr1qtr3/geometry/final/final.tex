%%%%%%%%%%%%%%%%%%%%%%%%%%%%%%%%%%%%%%%%%
% Short Sectioned Assignment
% LaTeX Template
% Version 1.0 (5/5/12)
%
% This template has been downloaded from:
% http://www.LaTeXTemplates.com
%
% Original author:
% Frits Wenneker (http://www.howtotex.com)
%
% License:
% CC BY-NC-SA 3.0 (http://creativecommons.org/licenses/by-nc-sa/3.0/)
%
%%%%%%%%%%%%%%%%%%%%%%%%%%%%%%%%%%%%%%%%%

%----------------------------------------------------------------------------------------
%	PACKAGES AND OTHER DOCUMENT CONFIGURATIONS
%----------------------------------------------------------------------------------------

\documentclass[fontsize=11pt]{scrartcl} % 11pt font size

\usepackage[T1]{fontenc} % Use 8-bit encoding that has 256 glyphs
\usepackage[english]{babel} % English language/hyphenation
\usepackage{amsmath,amsfonts,amsthm} % Math packages
\usepackage{mathrsfs}

\usepackage[margin=1in]{geometry}

\usepackage{sectsty} % Allows customizing section commands
\allsectionsfont{\centering \normalfont\scshape} % Make all sections centered, the default font and small caps

\usepackage{fancyhdr} % Custom headers and footers
\pagestyle{fancyplain} % Makes all pages in the document conform to the custom headers and footers
\fancyhead{} % No page header - if you want one, create it in the same way as the footers below
\fancyfoot[L]{} % Empty left footer
\fancyfoot[C]{} % Empty center footer
\fancyfoot[R]{\thepage} % Page numbering for right footer
\renewcommand{\headrulewidth}{0pt} % Remove header underlines
\renewcommand{\footrulewidth}{0pt} % Remove footer underlines
\setlength{\headheight}{13.6pt} % Customize the height of the header

\numberwithin{equation}{section} % Number equations within sections (i.e. 1.1, 1.2, 2.1, 2.2 instead of 1, 2, 3, 4)
\numberwithin{figure}{section} % Number figures within sections (i.e. 1.1, 1.2, 2.1, 2.2 instead of 1, 2, 3, 4)
\numberwithin{table}{section} % Number tables within sections (i.e. 1.1, 1.2, 2.1, 2.2 instead of 1, 2, 3, 4)

\newcommand{\R}{\mathbb{R}}
\newcommand{\Q}{\mathbb{Q}}
\newcommand{\N}{\mathbb{N}}
\newcommand{\C}{\mathbb{C}}
\newcommand{\Z}{\mathbb{Z}}

\newtheorem{lemma}{Lemma}
%----------------------------------------------------------------------------------------
%	TITLE SECTION
%----------------------------------------------------------------------------------------

\newcommand{\horrule}[1]{\rule{\linewidth}{#1}} % Create horizontal rule command with 1 argument of height

\title{	
\normalfont \normalsize 
\textsc{geometry} \\ [25pt] % Your university, school and/or department name(s)
\horrule{0.5pt} \\[0.4cm] % Thin top horizontal rule
\huge Final Exam \\ % The assignment title
\horrule{2pt} \\[0.5cm] % Thick bottom horizontal rule
}

\author{Daniel Halmrast} % Your name

\date{\normalsize\today} % Today's date or a custom date

\begin{document}

\maketitle % Print the title

% Problems
\section*{Problem 1}
Let $M$ be a complete Riemannian manifold with sectional curvature $K, K\geq
k>0$. Let $\gamma$ be a nontrivial closed geodesic in $M$. Show that for any
$p\in M$,
\[
    d(p,\gamma)\leq \frac{\pi}{2\sqrt{k}}
\]

\begin{proof}
    For simplicity, we normalize our space so that $k=1$. Now, suppose for a
    contradiction that there is some $p\in M$ with
    $d(p,\gamma)>\frac{\pi}{2}$. Denote by $\sigma$ a minimizing geodesic from
    $p$ to a closest point $q$ on $\gamma$, which we know satisfies
    $\ell(\sigma)>\frac{\pi}{2}$. Finally, denote by $q'$ the point opposite $q$
    on $\gamma$ (that is, parameterizing $\gamma$ with $\gamma(0)=q$, let $q' =
    \gamma(\frac{\ell(\gamma)}{2})$).

    We split this theorem in two cases based on the length of $\gamma$.

    Suppose first that $\ell(\gamma)\geq \pi$. We now invoke the Toponogov
    comparison theorem on the triangle formed by the geodesics $\sigma$ from $p$
    to $q$ and $\gamma$ from $q$ to $q'$. We will label the edges of the
    triangle as:
    \begin{center}
        \begin{tabular}{l|l}
            Side & Label\\
            \hline
            $\overline{p q'}$ & $a$\\
            $\overline{p q} (=\sigma)$ & $b$\\
            $\overline{q q'}(=\gamma)$ & $c$
        \end{tabular}
    \end{center}

    Since $\sigma$ minimizes the distance from $p$ to $\gamma$, we know that the
    angle formed between $\sigma'$ and $\gamma'$ is exactly
    $\frac{\pi}{2}$. That is, $\sigma$ and $\gamma$ intersect orthogonally.

    Now, we form the associated triangle in $S^2$ with constant curvature $1$,
    labeling the sides in the same way as $\tilde{a},\tilde{b},\tilde{c}$, and
    requiring that $b = \tilde{b}$ and $c=\tilde{c}$ in length, and the angle
    opposite $a$ is equal to the angle opposite $\tilde{a}$. In
    particular, we know that $\tilde{b}>\frac{\pi}{2}$, and the angle opposite
    $\tilde{a}$ (labeled $\tilde{A}$) is exactly $\frac{\pi}{2}$. By simple
    spherical geometry, we know that
    \[
        \begin{aligned}
            \cos(\tilde{a}) &= \cos(\tilde{b})\cos(\tilde{c}) +
            \sin(\tilde{b})\sin(\tilde{c})\cos(\tilde{A})\\
            &= \cos(\tilde{b})\cos(\tilde{c}) + 0
        \end{aligned}
    \]
    Now, since $\tilde{b}>\frac{\pi}{2}$, $\cos(\tilde{b})<0$. Furthermore, we
    know that $\cos(\tilde{c})<0$ as well, since $\tilde{c}>\frac{\pi}{2}$ (by
        assumption, $\ell(\gamma)\geq \pi$, so
    $d(q,q')=\frac{\ell(\gamma)}{2}\geq \frac{\pi}{2}$).

    However, since both $\cos(\tilde{b})$ and $\cos(\tilde{c})$ are negative,
    their product is positive. This implies that $\cos(\tilde{a})$ is positive,
    and so $\tilde{a}<\frac{\pi}{2}$. By the Toponogov comparison theorem,
    $a\leq \tilde{a}<\frac{\pi}{2}$. However, this contradicts $q$ being a
    closest point to $p$, since $d(p,q')\leq a <\frac{\pi}{2}$ but
    $d(p,q)>\frac{\pi}{2}$. Thus, no such $p$ can exist.


\end{proof}

\newpage

\section*{Problem 2}
Let $M$ be a compact $n$-dimensional manifold of positive sectional curvature,
and $A$, $B$ two closed totally geodesic submanifolds. Show that $A$ and $B$
must intersect if $\dim(A) + \dim(B)\geq n$.

\begin{proof}
    Suppose for a contradiction that $A$ and $B$ do not intersect. Then, I claim
    there is a point $a\in A$ and $b\in B$ such that $d(a,b) = d(A,B)$ and a
    geodesic $\gamma$ from $a$ to $b$ which realizes this distance. Furthermore,
    $\gamma$ is orthogonal to $A$ and $B$.

    To see this, recall that for any closed submanifold $N$ of $M$, and point
    $p\not\in N$, there exists a point $q\in N$ with $d(p,q) = d(p,N)$ and a
    minimizing geodesic $\gamma$ from $p$ to $q$ which realizes this distance,
    and is orthogonal to $N$. Letting $p\in A$ and $N=B$, we define the function
    $f:A\to \R$ as $f(p) = d(p,B)$. Since $A$ is compact (as a closed subset of
    $M$ a compact space), this function achieves a minimum. Call the point which
    achieves such a minimum $a$. Clearly, $a$ and the corresponding close point
    $b\in B$ are such that $d(a,b) = d(A,B)$, and by construction the minimizing
    geodesic $\gamma$ is orthogonal to $B$. By symmetry, $\gamma$ is also
    orthogonal to $A$ as well.

    Now, I assert that the second variation of energy for any orthogonal
    variational field $V$ with associated variation $h(t,s)$ of $\gamma$ with
    $h(0,s)\in A$ and $h(l,s)\in B$ is given by
    \[
        \frac{1}{2}E''(0) = I_l(V,V) + \langle
        V(l),S^{(2)}_{\gamma'(l)}V(l)\rangle - \langle V(l),
        S^{(1)}_{\gamma'(0)}V(l)\rangle
    \]
    where $S^{(i)}_{\gamma'}$ is the linear map associated to the second
    fundamental form of $A,B$ in the direction of $\gamma'$.

    To see this, we calculate directly
    \[
        \begin{aligned}
            \frac{1}{2}E''(0) &= I_l(V,V) + \langle
            \nabla_sV,\gamma'\rangle(0,l) - \langle
            \nabla_sV,\gamma'\rangle(0,0)\\
            &= 
        \end{aligned}<++>
    \]
\end{proof}<++>

\end{document}
