%%%%%%%%%%%%%%%%%%%%%%%%%%%%%%%%%%%%%%%%%
% Short Sectioned Assignment
% LaTeX Template
% Version 1.0 (5/5/12)
%
% This template has been downloaded from:
% http://www.LaTeXTemplates.com
%
% Original author:
% Frits Wenneker (http://www.howtotex.com)
%
% License:
% CC BY-NC-SA 3.0 (http://creativecommons.org/licenses/by-nc-sa/3.0/)
%
%%%%%%%%%%%%%%%%%%%%%%%%%%%%%%%%%%%%%%%%%

%----------------------------------------------------------------------------------------
%	PACKAGES AND OTHER DOCUMENT CONFIGURATIONS
%----------------------------------------------------------------------------------------

\documentclass[fontsize=11pt]{scrartcl} % 11pt font size

\usepackage[T1]{fontenc} % Use 8-bit encoding that has 256 glyphs
\usepackage[english]{babel} % English language/hyphenation
\usepackage{amsmath,amsfonts,amsthm} % Math packages
\usepackage{mathrsfs}

\usepackage[margin=1in]{geometry}

\usepackage{sectsty} % Allows customizing section commands
\allsectionsfont{\centering \normalfont\scshape} % Make all sections centered, the default font and small caps

\usepackage{fancyhdr} % Custom headers and footers
\pagestyle{fancyplain} % Makes all pages in the document conform to the custom headers and footers
\fancyhead{} % No page header - if you want one, create it in the same way as the footers below
\fancyfoot[L]{} % Empty left footer
\fancyfoot[C]{} % Empty center footer
\fancyfoot[R]{\thepage} % Page numbering for right footer
\renewcommand{\headrulewidth}{0pt} % Remove header underlines
\renewcommand{\footrulewidth}{0pt} % Remove footer underlines
\setlength{\headheight}{13.6pt} % Customize the height of the header

\numberwithin{equation}{section} % Number equations within sections (i.e. 1.1, 1.2, 2.1, 2.2 instead of 1, 2, 3, 4)
\numberwithin{figure}{section} % Number figures within sections (i.e. 1.1, 1.2, 2.1, 2.2 instead of 1, 2, 3, 4)
\numberwithin{table}{section} % Number tables within sections (i.e. 1.1, 1.2, 2.1, 2.2 instead of 1, 2, 3, 4)

\newcommand{\R}{\mathbb{R}}
\newcommand{\Q}{\mathbb{Q}}
\newcommand{\N}{\mathbb{N}}
\newcommand{\C}{\mathbb{C}}
\newcommand{\Z}{\mathbb{Z}}

\newtheorem{lemma}{Lemma}
%----------------------------------------------------------------------------------------
%	TITLE SECTION
%----------------------------------------------------------------------------------------

\newcommand{\horrule}[1]{\rule{\linewidth}{#1}} % Create horizontal rule command with 1 argument of height

\title{	
\normalfont \normalsize 
\textsc{geometry} \\ [25pt] % Your university, school and/or department name(s)
\horrule{0.5pt} \\[0.4cm] % Thin top horizontal rule
\huge Final Exam \\ % The assignment title
\horrule{2pt} \\[0.5cm] % Thick bottom horizontal rule
}

\author{Daniel Halmrast} % Your name

\date{\normalsize\today} % Today's date or a custom date

\begin{document}

\maketitle % Print the title

% Problems
\section*{Problem 1}
Let $M$ be a complete Riemannian manifold with sectional curvature $K, K\geq
k>0$. Let $\gamma$ be a nontrivial closed geodesic in $M$. Show that for any
$p\in M$,
\[
    d(p,\gamma)\leq \frac{\pi}{2\sqrt{k}}
\]

\begin{proof}
    For simplicity, we normalize our space so that $k=1$. Now, suppose for a
    contradiction that there is some $p\in M$ with
    $d(p,\gamma)>\frac{\pi}{2}$. Denote by $\sigma$ a minimizing geodesic from
    $p$ to a closest point $q$ on $\gamma$, which we know satisfies
    $\ell(\sigma)>\frac{\pi}{2}$. Finally, denote by $q'$ the point opposite $q$
    on $\gamma$ (that is, parameterizing $\gamma$ with $\gamma(0)=q$, let $q' =
    \gamma(\frac{\ell(\gamma)}{2})$).

    We split this proof in two cases based on the length of $\gamma$.

    Suppose first that $\ell(\gamma)\geq \pi$. We now invoke the Toponogov
    comparison theorem on the triangle formed by the geodesics $\sigma$ from $p$
    to $q$ and $\gamma$ from $q$ to $q'$. We will label the edges of the
    triangle as:
    \begin{center}
        \begin{tabular}{l|l}
            Side & Label\\
            \hline
            $\overline{p q'}$ & $a$\\
            $\overline{p q} (=\sigma)$ & $b$\\
            $\overline{q q'}(=\gamma)$ & $c$
        \end{tabular}
    \end{center}

    Since $\sigma$ minimizes the distance from $p$ to $\gamma$, we know that the
    angle formed between $\sigma'$ and $\gamma'$ is exactly
    $\frac{\pi}{2}$. That is, $\sigma$ and $\gamma$ intersect orthogonally.

    Now, we form the associated triangle in $S^2$ with constant curvature $1$,
    labeling the sides in the same way as $\tilde{a},\tilde{b},\tilde{c}$, and
    requiring that $b = \tilde{b}$ and $c=\tilde{c}$ in length, and the angle
    opposite $a$ is equal to the angle opposite $\tilde{a}$. In
    particular, we know that $\tilde{b}>\frac{\pi}{2}$, and the angle opposite
    $\tilde{a}$ (labeled $\tilde{A}$) is exactly $\frac{\pi}{2}$. By simple
    spherical geometry, we know that
    \[
        \begin{aligned}
            \cos(\tilde{a}) &= \cos(\tilde{b})\cos(\tilde{c}) +
            \sin(\tilde{b})\sin(\tilde{c})\cos(\tilde{A})\\
            &= \cos(\tilde{b})\cos(\tilde{c}) + 0
        \end{aligned}
    \]
    Now, since $\tilde{b}>\frac{\pi}{2}$, $\cos(\tilde{b})<0$. Furthermore, we
    know that $\cos(\tilde{c})<0$ as well, since $\tilde{c}>\frac{\pi}{2}$ (by
        assumption, $\ell(\gamma)\geq \pi$, so
    $d(q,q')=\frac{\ell(\gamma)}{2}\geq \frac{\pi}{2}$).

    However, since both $\cos(\tilde{b})$ and $\cos(\tilde{c})$ are negative,
    their product is positive. This implies that $\cos(\tilde{a})$ is positive,
    and so $\tilde{a}<\frac{\pi}{2}$. By the Toponogov comparison theorem,
    $a\leq \tilde{a}<\frac{\pi}{2}$. However, this contradicts $q$ being a
    closest point to $p$, since $d(p,q')\leq a <\frac{\pi}{2}$ but
    $d(p,q)>\frac{\pi}{2}$. Thus, no such $p$ can exist.


    Now, suppose $\ell(\gamma)<\pi$. Since we assumed
    $d(p,\gamma)>\frac{\pi}{2}$, we know that the diameter $d(M)>\frac{\pi}{2}$
    as well\dots

    If we accept the hint in problem 5, this situation cannot occur, and the
    proof is finished.
\end{proof}

\newpage

\section*{Problem 2}
Let $M$ be a compact $n$-dimensional manifold of positive sectional curvature,
and $A$, $B$ two closed totally geodesic submanifolds. Show that $A$ and $B$
must intersect if $\dim(A) + \dim(B)\geq n$.

\begin{proof}
    Suppose for a contradiction that $A$ and $B$ do not intersect. Then, I claim
    there is a point $a\in A$ and $b\in B$ such that $d(a,b) = d(A,B)$ and a
    geodesic $\gamma$ from $a$ to $b$ which realizes this distance. Furthermore,
    $\gamma$ is orthogonal to $A$ and $B$. We will show that such a $\gamma$ has
    a variation whose second variation of energy is negative, leading to a
    contradiction of $\gamma$ being minimizing.

    To see that such a $\gamma$ exists, recall that for any closed submanifold
    $N$ of $M$, and point $p\not\in N$, there exists a point $q\in N$ with
    $d(p,q) = d(p,N)$ and a minimizing geodesic $\gamma$ from $p$ to $q$ which
    realizes this distance, and is orthogonal to $N$. Letting $p\in A$ and
    $N=B$, we define the function $f:A\to \R$ as $f(p) = d(p,B)$. Since $A$ is
    compact (as a closed subset of $M$ a compact space), this function achieves
    a minimum. Call the point which achieves such a minimum $a$. Clearly, $a$
    and the corresponding close point $b\in B$ are such that $d(a,b) = d(A,B)$,
    and by construction the minimizing geodesic $\gamma$ is orthogonal to $B$.
    By symmetry, $\gamma$ is also orthogonal to $A$ as well.

    Now, I assert that the second variation of energy for any orthogonal
    variational field $V$ with associated variation $h(t,s)$ of $\gamma$ with
    $h(0,s)\in A$ and $h(l,s)\in B$ is given by
    \[
        \frac{1}{2}E''(0) = I_l(V,V) + \langle
        V(l),S^{(2)}_{\gamma'(l)}V(l)\rangle - \langle V(l),
        S^{(1)}_{\gamma'(0)}V(l)\rangle
    \]
    where $S^{(i)}_{\gamma'}$ is the linear map associated to the second
    fundamental form of $A,B$ in the direction of $\gamma'$.

    To see this, we calculate directly
    \[
        \begin{aligned}
            \frac{1}{2}E''(0) &= I_l(V,V) + \langle
            \nabla_sV,\gamma'\rangle(0,l) - \langle
            \nabla_sV,\gamma'\rangle(0,0)\\
            &= I_l(V,V) + \langle
            B(V,V),\gamma'\rangle(0,l) - \langle
            B(V,V),\gamma'\rangle(0,0)\\
            &= I_l(V,V) + \langle
        V(l),S^{(2)}_{\gamma'(l)}V(l)\rangle - \langle V(l),
        S^{(1)}_{\gamma'(0)}V(l)\rangle
        \end{aligned}
    \]
    as desired.

    Now, since $A$ and $B$ are totally geodesic submanifolds, the second
    fundamental form vanishes, and we are left with
    \[
        \frac{1}{2}E''(0) = I_l(V,V)
    \]
    which will be important later.

    Let $\{a_i\}$ be a basis for $T_a(A)$. We can parallel transport this basis
    along $\gamma$ to obtain a set of $\dim(A)$ linearly independent vectors
    orthogonal to $\gamma$ at $T_b(M)$. Denote by $\{b_i\}$ a basis for
    $T_b(B)\subset T_b(M)$. Since $\dim(A)+\dim(B)\geq n$, there
    must be some linear combination of the parallel transports of $\{a_i\}$ that
    sum to some vector $v\in T_b(B)$ orthogonal to $\gamma$ (since the dimension
        of the subspace of $T_b(M)$ orthogonal to $\gamma$ is $n-1$, and there
        are $\dim(A)+\dim(B)\geq n>n-1$ vectors in the set $\{a_i\}\cup\{b_i\}$,
        this set cannot be linearly independent. Since each basis set $\{a_i\}$
        and $\{b_i\}$ are linearly independent, there must be some linear
        combination of $\{a_i\}$ that sums to some linear combination of
    $\{b_i\}$). That is to say, there is a vector $v\in T_a(A)$ such that
    the parallel transport of $v$ into $T_b(M)$ lies in $T_b(B)$.

    Consider a variational field generated by parallel transport of such a
    vector $v$ (call this variational field $V$). In particular, this variation
    satisfies the hypotheses for the calculation in the previous paragraph.
    So, we calculate:
    \[
        \begin{aligned}
            \frac{1}{2}E''(0) &= I_l(V,V)\\
            &= \int_0^l\langle V',V'\rangle - K(V,\gamma')dt\\
        \end{aligned}
    \]
    Now, since $V$ is parallel to $\gamma$, $V'=0$, and so
    \[
        \frac{1}{2}E''(0) = -\int_0^lK(V,\gamma')dt
    \]
    and since $K$ is always positive, the second variation of energy of $V$ is
    negative, contradicting $\gamma$ being a minimal geodesic from $a$ to $b$.
    Thus, it cannot be that $A$ and $B$ do not intersect, as desired.
\end{proof}

\newpage

\section*{Problem 3}

Let $M^2$ be a complete simply connected $2$-dimensional Riemannian manifold.
Suppose that for each point $p\in M$, the locus $C(p)$ of first conjugate points
to $p$ reduces to a unique $q\neq p$ and that $d(p,C(p))=\pi$. Prove that if the
sectional curvature $K$ of $M$ satisfies $K\leq 1$, then $M$ is isometric to the
sphere $S^2$ with $K=1$.

\begin{proof}
    Let $J$ be a Jacobi field along a normalized geodesic $\gamma$ joining $p$
    to $q$ with $J(0)=J(\pi)=0$ and $g(J,\gamma')=0$. Let $\{e_i,\gamma'\}$ be
    an orthonormal parallel frame to $\gamma$, and write
    \[
        J = a^ie_i
    \]
    Define $K(t) = K(\gamma',J)$. We calculate
    \[
\begin{aligned}
    0 = I_{\pi}(J,J) &= -\int_0^{\pi}g(J'' + R(\gamma',J)\gamma',J)dt\\
    &= -\int_0^{\pi}g(J'',J)dt -\int_0^{\pi}K(t)\|J\|^2dt\\
    &= -\int_0^{\pi}a''^ia_idt - \int_0^{\pi}K(t)a^ia_idt\\
    &= \int_0^{\pi}a'^ia'_idt - \int_0^{\pi}K(t)a^ia_idt &\text{using
    integration by parts}\\
    &\geq \int_0^{\pi}a^ia_idt - \int_0^{\pi}K(t)a^ia_idt &\text{by homework 3
    problem 1}\\
    &=\int_0^{\pi}a^ia_i(1-K(t))dt\geq 0
\end{aligned}
    \]
    and thus $K(t)=1$ for all $t$. Thus, in the interior of any geodesic
    connecting a point $p$ to its first conjugate point, the sectional curvature
    of the space spanned by $\gamma'$ and $J$ for $J$ a Jacobi field along
    $\gamma$ vanishing at the endpoints is identically $1$.

    Now, all that remains is to show every point on the manifold is an interior
    point of some geodesic as described above, and that for such a geodesic,
    there exists a Jacobi field vanishing at the endpoints in each direction
    orthogonal to $\gamma'$.

    We handle the second assertion first. To prove that there exist Jacobi
    fields vanishing at the endpoints of $\gamma$ in each direction orthogonal
    to $\gamma'$, we show that the multiplicity of $q$ as a conjugate point to
    $p$ is exactly $n-1$. We appeal to proposition 3.5 of chapter 5 of Do Carmo,
    which states that the multiplicity of $q$ is exactly the dimension of the
    kernel of $(d\exp_p)_{v_0}$ where $v_0 = \gamma'(0)$.

    Recall that for $w\in T_{V_0}(T_pM)$, we have
    \[
        (d\exp_p)_{v_0}(w) = \partial_t(\exp_p(v_0 + tw))|_{t=0}
    \]
    Suppose $w$ is orthogonal to $v_0$. Then, $v_0+tw$ lies approximately on the
    sphere of radius $\|v_0\|$ for small $t$. In particular,
    \[
        \begin{aligned}
            \|v_0+tw\|^2 &= \|v_0\|^2 + t^2\|w\|^2 + 2t\langle v_0,w\rangle\\
            &=\|v_0\|^2 + t^2\|w\|^{2}\\
            \|v_0+tw\| &\approx \|v_0\| + O(t^2)
        \end{aligned}
    \]
    and so the geodesic $\gamma_{tw}(s) = \exp_p(s(v_0+tw))$ has to first order the
    same speed as $\gamma(s) = \exp_p(sv_0)$. Now, the function $f_p:T_pM\to
    \R$ taking a vector $v$ and returning the distance along the geodesic
    generated by $v$ to its first conjugate point is continuous, so $f(v_0+tw)$
    is close to $f(v_0)$. In particular, since $q$ is the only conjugate point
    to $p$, it must be that $\gamma_{tw}(1+\varepsilon)$ for some small
    $\varepsilon$ lands at $q$. Thus,
    $\partial_t(\exp_p(v_0+tw))|_{t=0}=0$
    and $w$ is in the kernel of $(d\exp_p)_{v_0}$. Since we can do this for each
    $w$ orthogonal to $v_0$, it follows that the dimension of the kernel is
    $n-1$, and so the multiplicity of $q$ is $n-1$ as desired.

    Finally, we show that every point in $M$ is the interior point of some
    geodesic connecting two conjugate points. In fact, we only have to show that
    there is a dense subset of $M$ with this property, and since the sectional
    curvature is continuous with respect to points in $M$, it will hold that
    $K=1$ for all points in $M$ as well.

    So, let $p\in M$. Now, we know that the conjugate locus of $p$ is a single
    point $q$. This means that there is some geodesic $\exp_p(tv)$ for a unit
    vector $v$ in $T_pM$ for which $\exp_p(\pi v) = q$ and $q$ is conjugate to
    $p$ along this geodesic. Now, since the function $f_p$ defined above is
    continuous, and can only take values of $\pi$ and $\infty$ (by the
    hypothesis of the problem), it follows that $f(v) = \pi$ for all $v$.
    Furthermore, since $M$ is complete, there exists a minimizing geodesic from
    $p$ to any other point on the manifold. Thus,
    \[
        M = \exp_p(B_{\pi}(0))\cup\{q\}
    \]
    and thus every point except $p$ and $q$ lies in the interior of a geodesic
    on which $p$ is conjugate to $q$. Thus, $K$ is identically $1$ on $M$, and
    since $M$ is simply connected, it follows that $M=S^n$ for $n$ the dimension
    of $M$.
\end{proof}

\newpage

\section*{Problem 5}
Let $M$ be a compact Riemannian manifold of dimension $n$ with sectional
curvature $K\geq 1$. Suppose $p,q\in M$ with $d(p,q)=d(M)>\frac{\pi}{2}$.
Moreover, suppose $\gamma:[0,1]\to M$ is a geodesic with
$\gamma(0)=\gamma(1)=p$. Show that $\gamma$ has Morse index at least $n-1$.

\begin{proof}
    Suppose we accept the hint that $\ell(\gamma)>\pi$. Now, fix $c\in
    (1,\infty)$ a constant such that $\ell(\gamma)>\pi\sqrt{c}$.

    Now, in homework 3, we proved that if $\gamma:(-\infty,\infty)\to M$ is a
    normalized geodesic in $M$, then there exists a $t_0\in \R$ with $\gamma$
    restricted to $[-t_0,t_0]$ having Morse index at least $n-1$. The proof is
    replicated here:

    \begin{proof}
    Let $Y$ be a parallel field along $\gamma$ with $g(\gamma',Y)=0$ and
    $\|Y\|=1$. Set
    \[
        \phi_Y = g(R(\gamma',Y)\gamma',Y)
    \]
    and
    \[
        K(t) = \inf_Y\phi_Y(t)
    \]
    and let $a:\R\to\R$ be a differentiable function such that $0\leq a(t)\leq
    K(t)$ with $0<a(0)<K(0)$. Let $\phi$ be the solution to $\phi'' + a\phi = 0$
    with $\phi(0)=1,\phi'(0)=0$, with $-t_1,t_2$ the two zeroes of $\phi$ found
    in the previous problem. We consider the field $X = \phi Y$, and calculate
    \[
        \begin{aligned}
            I_{[-t_1,t_2]}(X,X) &= -\int_{-t_1}^{t_2}g(X'' +
            R(\gamma',X)\gamma',X)dt\\
            &= -\int_{-t_1}^{t_2}g(\phi''Y,\phi Y)dt - \int_{t_1}^{t_2}g(\phi
            R(\gamma',Y)\gamma',\phi Y)\\
            &= -\int_{-t_1}^{t_2}g(\phi''Y,\phi Y)dt -
            \int_{t_1}^{t_2}\phi^2\phi_Ydt\\
            &\leq -\int_{-t-1}^{t_2}g(\phi''Y,\phi Y)dt -
            \int_{-t_1}^{t_2}K(t)\phi^2(t)dt\\
            &=-\int_{-t_1}^{t_2}\phi(\phi'' + K(t)\phi)dt\\
            &<-\int_{-t_1}^{t_2}\phi(\phi'' + a(t)\phi)dt\\
            &=0
        \end{aligned}
    \]
    Thus, for all $Y$ perpendicular to $\gamma'$ (an $n-1$ dimensional subspace)
    the form $I_{[-t_1,t_2]}(Y,Y)$ is negative-definite, and so the index is
    greater than or equal to $n-1$. 
\end{proof}
    Recall that we required the field $X$ to vanish at the endpoints so that
    integration by parts works.

    Now, setting $a(t) = \frac{1}{c}$ (which satisfies the hypotheses for $a$,
    since $K\geq 1$, and $\frac{1}{c}<1$) we observe that the unique solution
    $\phi$ is
    \[
    \phi(t) = \cos(\frac{t}{\sqrt{c}})
    \]
    which has zeroes at $t = \pm\frac{\pi\sqrt{c}}{2}$. In particular, the
    distance between consecutive zeroes is $\pi\sqrt{c}<\ell(\gamma)$. So,
    reparameterizing $\gamma$ to be unit speed, we see that
    $\gamma:[-\frac{\pi\sqrt{c}}{2},\frac{\pi\sqrt{c}}{2}]\to M$ contains both
    zeros and does not intersect itself (since $\ell(\gamma)>\pi\sqrt{c}$).
    Thus, since the Morse index is an increasing function, it follows that the
    Morse index for one period of $\gamma$ is at least $n-1$ as desired.
\end{proof}

\end{document}
