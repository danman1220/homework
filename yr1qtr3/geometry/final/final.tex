%%%%%%%%%%%%%%%%%%%%%%%%%%%%%%%%%%%%%%%%%
% Short Sectioned Assignment
% LaTeX Template
% Version 1.0 (5/5/12)
%
% This template has been downloaded from:
% http://www.LaTeXTemplates.com
%
% Original author:
% Frits Wenneker (http://www.howtotex.com)
%
% License:
% CC BY-NC-SA 3.0 (http://creativecommons.org/licenses/by-nc-sa/3.0/)
%
%%%%%%%%%%%%%%%%%%%%%%%%%%%%%%%%%%%%%%%%%

%----------------------------------------------------------------------------------------
%	PACKAGES AND OTHER DOCUMENT CONFIGURATIONS
%----------------------------------------------------------------------------------------

\documentclass[fontsize=11pt]{scrartcl} % 11pt font size

\usepackage[T1]{fontenc} % Use 8-bit encoding that has 256 glyphs
\usepackage[english]{babel} % English language/hyphenation
\usepackage{amsmath,amsfonts,amsthm} % Math packages
\usepackage{mathrsfs}

\usepackage[margin=1in]{geometry}

\usepackage{sectsty} % Allows customizing section commands
\allsectionsfont{\centering \normalfont\scshape} % Make all sections centered, the default font and small caps

\usepackage{fancyhdr} % Custom headers and footers
\pagestyle{fancyplain} % Makes all pages in the document conform to the custom headers and footers
\fancyhead{} % No page header - if you want one, create it in the same way as the footers below
\fancyfoot[L]{} % Empty left footer
\fancyfoot[C]{} % Empty center footer
\fancyfoot[R]{\thepage} % Page numbering for right footer
\renewcommand{\headrulewidth}{0pt} % Remove header underlines
\renewcommand{\footrulewidth}{0pt} % Remove footer underlines
\setlength{\headheight}{13.6pt} % Customize the height of the header

\numberwithin{equation}{section} % Number equations within sections (i.e. 1.1, 1.2, 2.1, 2.2 instead of 1, 2, 3, 4)
\numberwithin{figure}{section} % Number figures within sections (i.e. 1.1, 1.2, 2.1, 2.2 instead of 1, 2, 3, 4)
\numberwithin{table}{section} % Number tables within sections (i.e. 1.1, 1.2, 2.1, 2.2 instead of 1, 2, 3, 4)

\newcommand{\R}{\mathbb{R}}
\newcommand{\Q}{\mathbb{Q}}
\newcommand{\N}{\mathbb{N}}
\newcommand{\C}{\mathbb{C}}
\newcommand{\Z}{\mathbb{Z}}

\newtheorem{lemma}{Lemma}
%----------------------------------------------------------------------------------------
%	TITLE SECTION
%----------------------------------------------------------------------------------------

\newcommand{\horrule}[1]{\rule{\linewidth}{#1}} % Create horizontal rule command with 1 argument of height

\title{	
\normalfont \normalsize 
\textsc{geometry} \\ [25pt] % Your university, school and/or department name(s)
\horrule{0.5pt} \\[0.4cm] % Thin top horizontal rule
\huge Final Exam \\ % The assignment title
\horrule{2pt} \\[0.5cm] % Thick bottom horizontal rule
}

\author{Daniel Halmrast} % Your name

\date{\normalsize\today} % Today's date or a custom date

\begin{document}

\maketitle % Print the title

% Problems
\section*{Problem 1}
Let $M$ be a complete Riemannian manifold with sectional curvature $K, K\geq
k>0$. Let $\gamma$ be a nontrivial closed geodesic in $M$. Show that for any
$p\in M$,
\[
    d(p,\gamma)\leq \frac{\pi}{2\sqrt{k}}
\]

\begin{proof}
    For simplicity, we normalize our space so that $k=1$. Now, suppose for a
    contradiction that there is some $p\in M$ with
    $d(p,\gamma)>\frac{\pi}{2}$. Denote by $\sigma$ a minimizing geodesic from
    $p$ to a closest point $q$ on $\gamma$, which we know satisfies
    $\ell(\sigma)>\frac{\pi}{2}$. Finally, denote by $q'$ the point opposite $q$
    on $\gamma$ (that is, parameterizing $\gamma$ with $\gamma(0)=q$, let $q' =
    \gamma(\frac{\ell(\gamma)}{2})$).

    We now invoke the Toponogov comparison theorem on the triangle formed by the
    geodesics $\sigma$ from $p$ to $q$ and $\gamma$ from $q$ to $q'$. We will
    label the edges of the triangle as:
    \begin{center}
        \begin{tabular}{l|l}
            Side & Label\\
            \hline
            $\overline{p q'}$ & $a$\\
            $\overline{p q} (=\sigma)$ & $b$\\
            $\overline{q q'}(=\gamma)$ & $c$
        \end{tabular}
    \end{center}

    Since $\sigma$ minimizes the distance from $p$ to $\gamma$, we know that the
    angle formed between $\sigma'$ and $\gamma'$ is exactly
    $\frac{\pi}{2}$. That is, $\sigma$ and $\gamma$ intersect orthogonally.

    Now, we form the associated triangle in $S^2$ with constant curvature $1$,
    labeling the sides in the same way as $\tilde{a},\tilde{b},\tilde{c}$, and
    requiring that $b = \tilde{b}$ and $c=\tilde{c}$ in length, and the angle
    opposite $a$ is equal to the angle opposite $\tilde{a}$. In
    particular, we know that $\tilde{b}>\frac{\pi}{2}$, and the angle opposite
    $\tilde{a}$ (labeled $\tilde{A}$) is exactly $\frac{\pi}{2}$. By simple
    spherical geometry, we know that
    \[
        \begin{aligned}
            \cos(\tilde{a}) &= \cos(\tilde{b})\cos(\tilde{c}) +
            \sin(\tilde{b})\sin(\tilde{c})\cos(\tilde{A})\\
            &= \cos(\tilde{b})\cos(\tilde{c}) + 0
        \end{aligned}
    \]
    Now, since $\tilde{b}>\frac{\pi}{2}$, $\cos(\tilde{b})<0$. Furthermore, we
    know that $\cos(\tilde{c})<0$ as well, since
    $\tilde{c}>\frac{\pi}{2}$.

\end{proof}<++>

\end{document}
