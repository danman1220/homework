%%%%%%%%%%%%%%%%%%%%%%%%%%%%%%%%%%%%%%%%%
% Short Sectioned Assignment
% LaTeX Template
% Version 1.0 (5/5/12)
%
% This template has been downloaded from:
% http://www.LaTeXTemplates.com
%
% Original author:
% Frits Wenneker (http://www.howtotex.com)
%
% License:
% CC BY-NC-SA 3.0 (http://creativecommons.org/licenses/by-nc-sa/3.0/)
%
%%%%%%%%%%%%%%%%%%%%%%%%%%%%%%%%%%%%%%%%%

%----------------------------------------------------------------------------------------
%	PACKAGES AND OTHER DOCUMENT CONFIGURATIONS
%----------------------------------------------------------------------------------------

\documentclass[fontsize=11pt]{scrartcl} % 11pt font size

\usepackage[T1]{fontenc} % Use 8-bit encoding that has 256 glyphs
\usepackage[english]{babel} % English language/hyphenation
\usepackage{amsmath,amsfonts,amsthm} % Math packages
\usepackage{mathrsfs}

\usepackage[margin=1in]{geometry}

\usepackage{sectsty} % Allows customizing section commands
\allsectionsfont{\centering \normalfont\scshape} % Make all sections centered, the default font and small caps

\usepackage{fancyhdr} % Custom headers and footers
\pagestyle{fancyplain} % Makes all pages in the document conform to the custom headers and footers
\fancyhead{} % No page header - if you want one, create it in the same way as the footers below
\fancyfoot[L]{} % Empty left footer
\fancyfoot[C]{} % Empty center footer
\fancyfoot[R]{\thepage} % Page numbering for right footer
\renewcommand{\headrulewidth}{0pt} % Remove header underlines
\renewcommand{\footrulewidth}{0pt} % Remove footer underlines
\setlength{\headheight}{13.6pt} % Customize the height of the header

\numberwithin{equation}{section} % Number equations within sections (i.e. 1.1, 1.2, 2.1, 2.2 instead of 1, 2, 3, 4)
\numberwithin{figure}{section} % Number figures within sections (i.e. 1.1, 1.2, 2.1, 2.2 instead of 1, 2, 3, 4)
\numberwithin{table}{section} % Number tables within sections (i.e. 1.1, 1.2, 2.1, 2.2 instead of 1, 2, 3, 4)

\newcommand{\R}{\mathbb{R}}
\newcommand{\Q}{\mathbb{Q}}
\newcommand{\N}{\mathbb{N}}
\newcommand{\C}{\mathbb{C}}
\newcommand{\Z}{\mathbb{Z}}

\newtheorem{lemma}{Lemma}
%----------------------------------------------------------------------------------------
%	TITLE SECTION
%----------------------------------------------------------------------------------------

\newcommand{\horrule}[1]{\rule{\linewidth}{#1}} % Create horizontal rule command with 1 argument of height

\title{	
\normalfont \normalsize 
\textsc{geometry} \\ [25pt] % Your university, school and/or department name(s)
\horrule{0.5pt} \\[0.4cm] % Thin top horizontal rule
\huge Homework 4 \\ % The assignment title
\horrule{2pt} \\[0.5cm] % Thick bottom horizontal rule
}

\author{Daniel Halmrast} % Your name

\date{\normalsize\today} % Today's date or a custom date

\begin{document}

\maketitle % Print the title

% Problems
\section*{Problem 1}
Find the cut locus of a general flat torus.

\newpage

\section*{Problem 2}
Let $M$ be a complete Riemannian manifold. Prove that $M$ is compact if and only
if there exists a point $p\in M$ such that every geodesic starting from $p$ has
a cut point.
\\
\\
\begin{proof}
    ($\implies$)
    
    Suppose first that $M$ is compact. In particular, we know that the diameter
    $d(M)$ of $M$ is finite. Suppose for a contradiction that every point $p$ in
    $M$ has some unit speed geodesic $\gamma$ starting at $p$ with no cut
    point. This means that $\gamma$ minimizes the distance from $p$ to
    $\gamma(t)$ for all $t$. So, $d(p,\gamma(d(M)+1)) = d(M)+1$, which
    contradicts $d(M)$ being the diameter of $M$. Thus, there exists a point in
    $M$ for which every geodesic starting at that point has a cut point.
    \\
    \\
    ($\impliedby$)

    This proof comes from Do Carmo, chapter 13 corollary 2.11.

    Suppose $M$ is such that there is a point $p\in M$ with every geodesic
    starting from $p$ having a cut point. Since $M$ is complete, we know that
    \[
        M = \bigcup_{\gamma}\{\gamma(t):\ t\leq f(p,\gamma'(0))\}
    \]
    where the union is taken over all geodesics $\gamma$ starting at $p$, and
    $f$ is the function which takes a geodesic $\gamma$ starting at a point, and
    returns the first $t_0$ for which $\gamma(t_0)$ is the cut point of
    $\gamma(0)$. This follows, since $M$ being complete implies that there
    exists a minimizing geodesic from $p$ to $q$ for each pair of points $p,q\in
    M$.

    Now, since $f$ is continuous, and each geodesic starting at $p$ has a cut
    point, it follows that $f$ is bounded. Therefore, $M$ is bounded, and thus
    $M$ is compact, as desired.
\end{proof}

\newpage

\section*{Problem 3}
Prove that for every compact, even-dimensional manifold $M$ with sectional
curvature $0<M\leq 1$, the injectivity radius $i(M)\geq \frac{\pi}{2}$.
\\
\\
\begin{proof}
    Suppose first $M$ is orientable. Then, by proposition 3.4 of chapter 13 of
    Do Carmo, we know that $i(M)\geq \pi$, and we are done.

    So, suppose instead $M$ is not orientable. We know from problem 12 of
    chapter 0 of Do Carmo that there exists an orientable double cover of $M$.
    Call such a cover $(\tilde{M},p)$. Now, since $p$ is a local isometry, the
    sectional curvature of $\tilde{M}$ is equal to the sectional curvature of
    $M$. Thus, $\tilde{M}$ satisfies the hypotheses for proposition 3.4 of
    chapter 13, and $i(\tilde{M})\geq \pi$.

    Now, we will show that this implies that $i(M)\geq \frac{\pi}{2}$. Suppose
    for a contradiction there existed a point $q\in M$ with a geodesic $\gamma$
    starting at $q$ (with unit speed) so that $\gamma(t_0)$ is a cut point of
    $q$ with $t_0<\frac{\pi}{2}$.

    Suppose first that $\gamma(t_0)$ is conjugate to $q$. Let $J$ be the Jacobi
    field which vanishes at $\gamma(0)$ and $\gamma(t_0)$ and is not everywhere
    zero. Now, fix $y_0\in p^{-1}(q)$ the basepoint of $\tilde{M}$, and lift
    $\gamma$ along $p$ to a geodesic $\tilde{\gamma}$ in $\tilde{M}$ starting at
    $y_0$. Since $p$ is a local isometry, $J$ also lifts along $p$ to a Jacobi
    field along $\tilde{\gamma}$ which vanishes at the endpoints. But this
    implies that $\tilde{\gamma}(0)$ and $\tilde{\gamma}(t_0)$ are conjugate to
    each other, which cannot happen since
    \[
        d(\tilde{\gamma}(0),\tilde{\gamma}(t_0))\leq t_0<\pi
    \]

    So, suppose instead there are two geodesics $\gamma,\sigma$ with
    $\gamma(0)=\sigma(0)=q$ and $\gamma(t_0)=\sigma(t_0)$. This forms a closed
    geodesic loop $\delta$ with $\ell(\delta) = 2t_0 < \pi$. This loop lifts to
    a geodesic $\tilde{\delta}$ in $\tilde{M}$. Now, either $\tilde{\delta}$ is
    a closed loop, or $\tilde{\delta}(0)=y_0$ and $\tilde{\delta}(2t_0) = y_1$
    with $y_1$ the other preimage of $q$.

    If $\tilde{\delta}$ is a loop, then it is the concatenation of
    $\tilde{\gamma}$ and $\tilde{\sigma}$, each of which have length $t_0$ and
    connect $y_0$ to $\tilde{\gamma}(t_0)$. Thus, $\tilde{\gamma}(t)$ is in
    the cut locus of $y_0$ for some $t\in (0,t_0]$, which is a contradiction
    since $d(y_0,\tilde{\gamma}(t))\leq t_0<\pi$.

    Suppose instead that $\tilde{\delta}$ is a path from $y_0$ to $y_1$. Now,
    let $x_0,x_1\in \tilde{M}$ be the preimages of $\gamma(t_0)$ in
    $\tilde{M}$. Since $\delta$ passes through $\gamma(t_0)$ exactly once,
    $\tilde{\delta}$ passes through $x_0$ and not $x_1$ (without loss of
    generality in labeling).

    However, consider a different lift of $\delta$ (say, $\hat{\delta}$) where
    $\delta(t_0)$ gets lifted to $x_1$ instead of $x_0$. This defines a path
    $\hat{\delta}$ in $\tilde{M}$ which does not pass through $x_0$, but does
    pass through $x_1$. Furthermore, $\hat{\delta}$ has $y_0$ and $y_1$ as its
    endpoints (if it were a loop, we could use the same argument as in the
    previous paragraph to establish a contradiction). Thus, there are two
    geodesics $\tilde{\delta}$ and $\hat{\delta}$ from $y_0$ to $y_1$ which are
    distinct, but have the same length $\ell(\tilde{\delta}) =
    \ell(\hat{\delta}) = 2t_0$. This implies that there is a $t\in (0,2t_0]$ for
    which $\tilde{\delta}(t)$ is a cut point for $y_0$. This is clearly a
    contradiction, since
    \[
        d(y_0,\tilde{\delta}(t))\leq t \leq 2t_0<\pi
    \]
    and $i(\tilde{M})\geq \pi$.

    Thus, $i(M)\geq \frac{\pi}{2}$ as desired.
\end{proof}

\end{document}
