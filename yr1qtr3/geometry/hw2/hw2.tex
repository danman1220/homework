%%%%%%%%%%%%%%%%%%%%%%%%%%%%%%%%%%%%%%%%%
% Short Sectioned Assignment
% LaTeX Template
% Version 1.0 (5/5/12)
%
% This template has been downloaded from:
% http://www.LaTeXTemplates.com
%
% Original author:
% Frits Wenneker (http://www.howtotex.com)
%
% License:
% CC BY-NC-SA 3.0 (http://creativecommons.org/licenses/by-nc-sa/3.0/)
%
%%%%%%%%%%%%%%%%%%%%%%%%%%%%%%%%%%%%%%%%%

%----------------------------------------------------------------------------------------
%	PACKAGES AND OTHER DOCUMENT CONFIGURATIONS
%----------------------------------------------------------------------------------------

\documentclass[fontsize=11pt]{scrartcl} % 11pt font size

\usepackage[T1]{fontenc} % Use 8-bit encoding that has 256 glyphs
\usepackage[english]{babel} % English language/hyphenation
\usepackage{amsmath,amsfonts,amsthm} % Math packages
\usepackage{mathrsfs}

\usepackage[margin=1in]{geometry}

\usepackage{sectsty} % Allows customizing section commands
\allsectionsfont{\centering \normalfont\scshape} % Make all sections centered, the default font and small caps

\usepackage{fancyhdr} % Custom headers and footers
\pagestyle{fancyplain} % Makes all pages in the document conform to the custom headers and footers
\fancyhead{} % No page header - if you want one, create it in the same way as the footers below
\fancyfoot[L]{} % Empty left footer
\fancyfoot[C]{} % Empty center footer
\fancyfoot[R]{\thepage} % Page numbering for right footer
\renewcommand{\headrulewidth}{0pt} % Remove header underlines
\renewcommand{\footrulewidth}{0pt} % Remove footer underlines
\setlength{\headheight}{13.6pt} % Customize the height of the header

\numberwithin{equation}{section} % Number equations within sections (i.e. 1.1, 1.2, 2.1, 2.2 instead of 1, 2, 3, 4)
\numberwithin{figure}{section} % Number figures within sections (i.e. 1.1, 1.2, 2.1, 2.2 instead of 1, 2, 3, 4)
\numberwithin{table}{section} % Number tables within sections (i.e. 1.1, 1.2, 2.1, 2.2 instead of 1, 2, 3, 4)

\newcommand{\R}{\mathbb{R}}
\newcommand{\Q}{\mathbb{Q}}
\newcommand{\N}{\mathbb{N}}
\newcommand{\C}{\mathbb{C}}

\newtheorem{lemma}{Lemma}
%----------------------------------------------------------------------------------------
%	TITLE SECTION
%----------------------------------------------------------------------------------------

\newcommand{\horrule}[1]{\rule{\linewidth}{#1}} % Create horizontal rule command with 1 argument of height

\title{	
\normalfont \normalsize 
\textsc{Geometry} \\ [25pt] % Your university, school and/or department name(s)
\horrule{0.5pt} \\[0.4cm] % Thin top horizontal rule
\huge Homework 2 \\ % The assignment title
\horrule{2pt} \\[0.5cm] % Thick bottom horizontal rule
}

\author{Daniel Halmrast} % Your name

\date{\normalsize\today} % Today's date or a custom date

\begin{document}

\maketitle % Print the title

% Problems
\section*{Problem 1}
Let $M$ be a complete Riemannian manifold with non-positive sectional curvature.
Prove that
\[
    |(d\exp_p)_v(w)|\geq |w|
\]
for all $p\in M$, $v\in T_pM$, and $w\in T_v(T_pM)$.
\\
\\
\begin{proof}
    We compare $M$ to Euclidean space $\R^n$, noting that $K = K_M \leq
    \tilde{K} = K_{\R^n} = 0$. Now, let $\gamma(t) = \exp_p(tv)$ be the geodesic
    generated by $v$ in $M$, and observe that
    \[
        (d\exp_p)_{tv}(tw)
    \]
    is a Jacobi field along $\gamma$ (in fact, this is the first Jacobi field Do
    Carmo studies in chapter 5). Since $M$ is complete, such a field is defined,
    as $\exp_p(tv)$ is defined.

    Now, in $\R^n$, we construct $\tilde{\gamma}(t) = \exp_0(tv) = tv$ and
    note that
    \[
        \tilde{J}(t) = (d\exp_0)_{tv}(tw) = tw
    \]
    Now, clearly
    \[
        \|\gamma'(t)\| = \|\tilde{\gamma}'(t)\| = v
    \]
    and
    \[
        J(0) = \tilde{J}(0) = 0
    \]
    and
    \[
        g(J'(0),\gamma'(0)) = g(\tilde{J}'(0),\tilde{\gamma}'(0)) = g(w,v)
    \]
    and
    \[
        \|J'(0)\| = \|\tilde{J}'(0)\| = \|w\|
    \]
    and since $M$ has non-positive sectional curvature, $\gamma$ has no
    conjugate points. Thus, the hypotheses for the Rauch comparison theorem are
    satisfied, and immediately we know that
    \[
        |\tilde{J}(t)| \leq |J(t)|
    \]
    plugging in $t=1$, we observe that
    \[
        |(d\exp_p)_{v}(w)| \geq |w|
    \]
    as desired.
\end{proof}

\newpage

\section*{Problem 2}
Let
\[
    f''(t) + K(t)f(t) = 0, f(0)=0, t\in [0,\ell]
\]
\[
    \tilde{f}''(t) + \tilde{K}(t)\tilde{f}(t) = 0, \tilde{f}(0)=0, t\in [0,\ell]
\]
be two ODEs. Suppose $\tilde{K}(t)\geq K(t)$ for $t\in[0,\ell]$ and that
$f'(0)=\tilde{f}'(0)=1$.

\subsection*{Part a}
Show that for all $t\in [0,\ell]$,
\[
    \begin{aligned}
    0 &= \int_0^t\left\{ \tilde{f}(f''+Kf) - f(\tilde{f}'' +
    \tilde{K}\tilde{f}) \right\}dt\\
        &= [\tilde{f}f' - f\tilde{f}']_0^t +
    \int_0^t(K-\tilde{K})f\tilde{f}dt
    \end{aligned}
\]
and conclude that the first zero of $f$ does not occur before the first zero of
$\tilde{f}$.
\\
\\
\begin{proof}
    The first equality follows immediately from the differential equation. That
    is,
    \[
     \int_0^t\left\{ \tilde{f}(f''+Kf) - f(\tilde{f}'' +
     \tilde{K}\tilde{f}) \right\}dt = \int_0^t\tilde{f}(0) - f(0) dt = 0
    \]
    for the second equality, we integrate by parts. Noticing that
    \[
        \int_0^t\tilde{f}f''dt = [\tilde{f}f']_0^t - \int_0^t\tilde{f}'f'dt
    \]
    we see that
    \[
        \begin{aligned}
     \int_0^t\left\{ \tilde{f}(f''+Kf) - f(\tilde{f}'' +
     \tilde{K}\tilde{f}) \right\}dt\\
        &= \int_0^t\tilde{f}f''dt - \int_0^tf\tilde{f}''dt +
        \int_0^tK\tilde{f}fdt - \int_0^t\tilde{K}\tilde{f}fdt\\
        &=[\tilde{f}f']_0^t - \int_0^t\tilde{f}'f'dt - [f\tilde{f}']_0^t +
        \int_0^t\tilde{f}'f'dt + \int_0^t(K-\tilde{K})f\tilde{f}dt\\
        &=[\tilde{f}f'-f\tilde{f}']_0^t + \int_0^t(K-\tilde{K})f\tilde{f}dt
        \end{aligned}
    \]
    as desired.

    Now, we know that $f(t)>0$ on a neighborhood of zero. Let $t_0$ be the first
    zero of $\tilde{f}$ (so $\tilde{f}(t)>0$ for $t<t_0$). Assume for
    contradiction that for some $t_1<t_0$, $f(t_1)=0$. Furthermore, suppose this
    is the first zero (that is $f(t)>0$ for $t< t_1$). Then, $f'(t_1)<0$,
    $\tilde{f}(t_1)>0$, and so
    \[
        \begin{aligned}
            \left[\tilde{f}f'-f\tilde{f}'\right]_0^{t_1} +
            \int_0^{t_1}(K-\tilde{K})f\tilde{f}dt
            &= \tilde{f}(t_1)f'(t_1)-f(t_1)\tilde{f}'(t_1) +
            \int_0^{t_1}(K-\tilde{K}f\tilde{f}dt\\
            &= \tilde{f}(t_1)f'(t_1) +
            \int_0^{t_1}(K-\tilde{K}f\tilde{f}dt\\
        \end{aligned}
    \]
    now $\tilde{f}(t_1)>0$, $f'(t_1)<0$, $(K-\tilde{K})<0$ and
    $f(t)\tilde{f}(t)>0$ for $t<t_1$. Thus,
    \[
            \tilde{f}(t_1)f'(t_1) +
            \int_0^{t_1}(K-\tilde{K}f\tilde{f}dt <0
    \]
    a contradiction.
\end{proof}

\subsection*{Part b}
Suppose $\tilde{f}(t)>0$ on $(0,\ell]$. Show that $f(t)\geq \tilde{f}(t)$ for
$t\in [0,\ell]$, and equality holds for $t=t_1$ if and only if
$K(t)=\tilde{K}(t)$ on $[0,t_1]$.
\\
\\
\begin{proof}
    From the first equality in part a, and that $\tilde{f}(t),f(t)>0$, we see
    that in order for
    \[
    0 = [\tilde{f}f' - f\tilde{f}']_0^t +
    \int_0^t(K-\tilde{K})f\tilde{f}dt
    \]
    is satisfied only when
    \[
        [\tilde{f}f' - f\tilde{f}']_0^t = \tilde{f}(t)f'(t) -
        f(t)\tilde{f}'(t) > 0
    \]
    or,
    \[
        \frac{f'}{f}\geq \frac{\tilde{f}'}{\tilde{f}}
    \]
    In other words,
    \[
        (\log f)'\geq (\log \tilde{f})'
    \]

    Integrating this from $t_0$ to $t$, we see that
    \[
        \log f(t) - \log f(t_0)\geq \log \tilde{f}(t) - \log\tilde{f}(t_0)
    \]
    which implies that
    \[
        \frac{f(t)}{\tilde{f}(t)}\geq \frac{f(t_0)}{\tilde{f}(t_0)}
    \]
    taking the limit as $t_0\to 0$ and noting that $f(0)=\tilde{f}(0)=0$ (and
    applying l'Hopital's rule), we see that
    \[
        \frac{f(t)}{\tilde{f}(t)}\geq \frac{f'(0)}{\tilde{f}'(0)} = 1
    \]
    and the desired result is achieved.
\end{proof}

\newpage



\end{document}
