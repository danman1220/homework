%%%%%%%%%%%%%%%%%%%%%%%%%%%%%%%%%%%%%%%%%
% Short Sectioned Assignment
% LaTeX Template
% Version 1.0 (5/5/12)
%
% This template has been downloaded from:
% http://www.LaTeXTemplates.com
%
% Original author:
% Frits Wenneker (http://www.howtotex.com)
%
% License:
% CC BY-NC-SA 3.0 (http://creativecommons.org/licenses/by-nc-sa/3.0/)
%
%%%%%%%%%%%%%%%%%%%%%%%%%%%%%%%%%%%%%%%%%

%----------------------------------------------------------------------------------------
%	PACKAGES AND OTHER DOCUMENT CONFIGURATIONS
%----------------------------------------------------------------------------------------

\documentclass[paper=a4, fontsize=11pt]{scrartcl} % A4 paper and 11pt font size

\usepackage[T1]{fontenc} % Use 8-bit encoding that has 256 glyphs
\usepackage[english]{babel} % English language/hyphenation
\usepackage{amsmath,amsfonts,amsthm} % Math packages
\usepackage{mathrsfs}

\usepackage{sectsty} % Allows customizing section commands
\allsectionsfont{\centering \normalfont\scshape} % Make all sections centered, the default font and small caps

\usepackage{fancyhdr} % Custom headers and footers
\pagestyle{fancyplain} % Makes all pages in the document conform to the custom headers and footers
\fancyhead{} % No page header - if you want one, create it in the same way as the footers below
\fancyfoot[L]{} % Empty left footer
\fancyfoot[C]{} % Empty center footer
\fancyfoot[R]{\thepage} % Page numbering for right footer
\renewcommand{\headrulewidth}{0pt} % Remove header underlines
\renewcommand{\footrulewidth}{0pt} % Remove footer underlines
\setlength{\headheight}{13.6pt} % Customize the height of the header

\numberwithin{equation}{section} % Number equations within sections (i.e. 1.1, 1.2, 2.1, 2.2 instead of 1, 2, 3, 4)
\numberwithin{figure}{section} % Number figures within sections (i.e. 1.1, 1.2, 2.1, 2.2 instead of 1, 2, 3, 4)
\numberwithin{table}{section} % Number tables within sections (i.e. 1.1, 1.2, 2.1, 2.2 instead of 1, 2, 3, 4)


%----------------------------------------------------------------------------------------
%	TITLE SECTION
%----------------------------------------------------------------------------------------

\newcommand{\horrule}[1]{\rule{\linewidth}{#1}} % Create horizontal rule command with 1 argument of height

\title{	
\normalfont \normalsize 
\textsc{Algebra} \\ [25pt] % Your university, school and/or department name(s)
\horrule{0.5pt} \\[0.4cm] % Thin top horizontal rule
\huge Problem Set 1\\ % The assignment title
\horrule{2pt} \\[0.5cm] % Thick bottom horizontal rule
}

\author{Daniel Halmrast} % Your name

\date{\normalsize\today} % Today's date or a custom date

\begin{document}

\maketitle % Print the title

%----------------------------------------------------------------------------------------
%	PROBLEM 1
%----------------------------------------------------------------------------------------
\section*{Problem 1}
Show that the cycle $(1,2,\ldots,n)$ can be written as a product of $n-1$ transpositions, but no
fewer.
\\
\begin{proof}
Trivially, this holds for $n=2$. So assume $n\geq 3$.

First, note that the decomposition
\[
(1,2,\ldots,n) = (1,n)(1,n-1),\ldots,(1,3)(1,2)
\]
is a product of $n-1$ transpositions as desired.

Now, consider an arbitrary decomposition of the cycle
\[
(1,2,\ldots,n) = \tau_1 \tau_2 \ldots \tau_m
\]
where each $\tau_i$ is a transposition. Observe that for any decomposition, each number
from $1$ to $n$ must appear in at least 1 transposition, since the permutation admits no
fixed points.

Next, let's count how many unique numbers appear in the decomposition.
$\tau_1$ will introduce two unique numbers, leaving $n-2$ left to appear. Since
the permutation has no disjoint subcycles, there must exist some $\tau_{i_1}$ that
is not disjoint to $\tau_1$. 
At most, $\tau_{i_1}$ will introduce one new unique number (since one is shared by $\tau_1$,
leaving at least $n-3$ numbers to be introduced by $m-2$ traspositions.) 
Similarly, there must be some $\tau_{i_2}$ that shares a number with either $\tau_1$ or
$\tau_{i_1}$, and will introduce at most one new number.

Following this process shows that each $\tau_{i_k}$ will introduce at most one new number
to the decomposition, so to introduce $n-2$ numbers requires at least $n-2$ transpositions.
Considering also $\tau_1$, we see that there must be at least $n-1$ total transpositions in
the decomposition.

\end{proof}

%----------------------------------------------------------------------------------------

%----------------------------------------------------------------------------------------
%	PROBLEM 2
%----------------------------------------------------------------------------------------
\section*{Problem 2}
Let $G\leq S_n$, with $G\not\subset A_n$. Show that exactly half of the permutations in
$G$ are even.
\\
\begin{proof}
We know that $e\in G$ and there is some $\sigma\in S_n\setminus A_n$ such that $\sigma\in G$.

Now, consider the cyclic subgroup $\langle \sigma \rangle$ generated by this element. Since
$G$ is finite, this subgroup has finite order, call it $m$. I assert that $m$ must be even, since
if $m$ were odd, it would imply that $\sigma^m$, which is an odd permutation, is equal to $e$. But
since $e$ is an even permutation, such a relation cannot exist. Thus, $m$ must be even. 

Furthermore, for each $m$, the parity of $m$ determines the parity of $\sigma^m$.
This is made clear by considering a decomposition of $\sigma = \tau_1\ldots\tau_k$ for some
odd $k$. Then, $\sigma^m = \prod_m \tau_1\ldots\tau_k$ before reduction has $m*k$ transpositions.
When reducing this word, if two adjacent transpositions are inverses, both are removed. Thus, the 
reduced word has an even number of symbols less than $m*k$. Since $m*k$ is even, the reduced word
is also even as desired.

Thus, the cyclic group $\langle \sigma \rangle$ has exactly half of its permutations as even permutations.

Observe that, for each coset $\omega\langle\sigma\rangle$ of $\langle\sigma\rangle$, either $\omega$ is even,
or it is odd. If $\omega$ is even, it will not change the parity of any permutation, and the coset
$\omega\langle\sigma\rangle$ has exactly half of its permutations as even permutations.
If $\omega$ is odd, on the other hand, it will flip the parity of each permutation, but $\omega\langle\sigma\rangle$
will still have exactly half of its permutations as even permutations.

Since $G$ is partitioned by cosets of the subgroup $\langle\sigma\rangle$, it can be written as
a union of the disjoint cosets. Each coset has exactly half of its elements as even permutations,
so the whole group $G$ has the same property.
\end{proof}

\section*{problem 3}
Show that a permutation is regular if and only if it is a power of a permutation with no fixed points.
\\
\begin{proof}
(=>)
Let $\sigma\in S_n$ be such that it can be written as a product of disjoint cycles $\omega_1\ldots\omega_m$
of the same length.
Now, since each number from $1$ to $n$ will appear exactly once in the decomposition,
the order of any $\omega_i$ is just $l = \frac{n}{m}$.
So, let 
\[
\begin{aligned}
\omega_1 &= (k_{1_1} k_{1_2}\ldots k_{1_l})\\
\omega_2 &= (k_{2_1} k_{2_2}\ldots k_{2_l})\\
\vdots &\\
\omega_m &= (k_{m_1} k_{m_2}\ldots k_{m_l})\\
\end{aligned}
\]

Then, it is easy to verify that
\[
\sigma = (k_{1_1} k_{2_1}\ldots k_{m_1} k_{1_2} \ldots k_{m_l})^m
\]
as desired.

(<=)
For the other direction, assume (without loss of generality) 
$\sigma = (1 2 \ldots n)^m$ for some $0<m<n$, which implies that the order of $\sigma$
is the smallest number $l$ such that $lm=0\ (\textrm{mod }n)$. 
Now, if $m$ is relatively prime to $n$, then the order of $\sigma$ is $n$,
and can be written as the single permutation
\[
\sigma = (1\ m\ 2m \ldots (n-1)m)
\]
where multiplication is taken modulo $n$.

Next, consider the case where $m$ divides $n$, so that the order of $\sigma$ is 
$l = \frac{n}{m}$. Then, each number $1\leq i \leq l-1$ moves in a disjoint subcycle,
\[
\omega_i = (i\ i+m\ i+2m\ldots i+(l-1)m)
\]
and thus $\sigma = \prod_i\omega_i$ as desired.
\end{proof}

\section*{Problem 4}
Show that a permutation group is regular if and only if each permutation
moves all points or no points.
\\
\begin{proof}
(=>) Note that regular permutations are characterized by being powers of a cycle
that moves all points. So, if a group is made up of regular permutations, it
must be made up of powers of cycles that move all points. Now, since a power of
a cycle also moves all points (unless it is equal to the identity, in which case
it moves no points), each element of a regular group moves all points, or moves no
points.

(<=)
Now, let $G\subset S_n$ be a permutation group such that each element of $G$ either moves
all points or moves no points. If $g\in G$ moves no points, then $g$ is the identity,
which is regular.

So, let $g\in G$ be an element that moves all points. For a contradiction, assume
that $g$ has a disjoint cycle decomposition with two elements $\sigma$, $\omega$ 
of differing order. Let $|\sigma| = j$ and $|\omega| = k$. Without loss of generality,
let $j<k$. Then, the element $g^j$ fixes all points in the $\sigma$ subcycle, but does
not fix all points in the $\omega$ subcycle (since the order of $\omega$ is not $j$).
This violates the assumption that all the elements of $G$ either fix all points or move
all points, since $g^j\in G$.

Therefore, every element of $G$ moves all points, or moves no points as desired.

\end{proof}

\section*{Problem 5}
Show that in any group, $ab$ and $ba$ have the same order.
\\
\begin{proof}
Suppose first that the order of $ab$ is finite. Then, for some $n$, $(ab)^n = e$.
Now, take the conjugate with $a$ to obtain
\[
\begin{aligned}
a^{-1}(ab)^na &= a^{-1}ea = e\\
(ba)^n &= e
\end{aligned}
\]
Thus, the order of $ba$ divides the order of $ab$, since $(ba)^n=e$.
However, reversing the roles of $a$ and $b$ shows that the order of $ab$ divides the
order of $ba$. Thus, since they both divide each other, they must be equal.

Suppose instead that the order of $ab$ is infinite, and for a contradiction assume that
$ba$ had finite order. Then, by the above argument, the order of $ab$ must equal the order
of $ba$, which is a contradiction.

Thus, the order of $ab$ is equal to the order of $ba$.
\end{proof}

\section*{Problem 6}
Let $G$ be any group, and $n>2$. Show that there are an even number of elements order $n$ in $G$.
\\
\begin{proof}
Consider the set $E_n$ of all elements of order $n>2$ in $G$, along with the equivalence relation $g\sim g^{-1}$.
Since $n>2$, it is known that $g^{-1}\neq g$. Furthermore, since $(g^{-1})^{-1} = g$, each equivalence class has
exactly two elements.

Thus, since $E_n$ is partitioned by $\sim$ into disjoint sets, each
with two elements, the size of $E_n$ must be a multiple of two. Hence, it must be even.
\end{proof}

\section*{Problem 7}
Show that any group of even order has an element of order 2.
\\
\begin{proof}
To prove this, we start by proving the case for Abelian groups,
and we will induct on the order of the group.

The base case of $|G|=2$ is trivial.

Now, let $G$ be an Abelian group of even order, and let $g\in G$. If $|g|$
is even, then the element $g^{\frac{|g|}{2}}$ has order 2, and the proof is complete.
So, suppose $|g|$ is odd. Then, consider the quotient $G/{\langle g\rangle}$. Since
$|G|$ is even and $|\langle g\rangle|$ is odd, Lagrange's theorem tells us that 
$\left|G/{\langle g\rangle}\right|$ is even, and has smaller order than $G$. By the
inductive hypothesis, we are done. 

Now, consider a more general group $G$.
By the class equation, we have
\[
|G| = |Z(G)| + \sum_i[G:C_G(x_i)]
\]
For $Z(G)$ the center of $G$, $C_G(x_i)$ the centralizer of $x_i$, and the sum ranging over
representatives of the different conjugacy classes in $G$ not in the center.

Now, if $|Z(G)|$ is even, then it contains an element of order 2. This is clear, since 
the center is Abelian, and by the previous result any Abelian group of even order has
an element of order 2.

Suppose, then, that $|Z(G)|$ is odd. Then, there must be at least one element $[G:C_G(x_j)]$
of the sum which is also odd.
However, by Lagrange's theorem, this means that $|C_G(x_j)|$, which is a proper subgroup of $G$,
has even order. Thus, by the inductive hypothesis, it contains an element of order 2.

Therefore by induction, any group of even order has an element of order 2.
\end{proof}

\section*{Problem 8}
Show that every finitely generated subgroup of $(\mathbb{Q},+)$ is cyclic.
\\
\begin{proof}
Let $S$ be a finite subset of $\mathbb{Q}$. In particular, the elements of $S$ have a
least common denominator $d$, so that for each $s_i\in S$, $s_i = k_i * \frac{1}{d}$ for
some integer $k_i$. Thus, $s_i\in\langle\frac{1}{d}\rangle$. 
Since each $s_i$ is in the subgroup $\langle\frac{1}{d}\rangle$, by the definition of a generated group,
\[
\langle S\rangle\leq\langle\frac{1}{d}\rangle
\]
Subgroups of cyclic groups are cyclic, so the subgroup $\langle S \rangle$ is cyclic.
\end{proof}


\end{document}
