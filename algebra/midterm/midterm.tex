\documentclass[12pt,reqno]{amsart}
\usepackage{amssymb}
\usepackage{amscd}
\usepackage{amsxtra}
\usepackage[mathscr]{eucal}

\setlength{\oddsidemargin}{0cm}
\setlength{\evensidemargin}{0in}
\setlength{\textwidth}{16.5cm}
\setlength{\topmargin}{0.35cm}
\setlength{\textheight}{8.5in}
\renewcommand{\baselinestretch}{1.33}

\pagestyle{plain}

\newcommand{\inv}{^{-1}}
\newcommand{\N}{\mathbb{N}}

\newtheorem*{statement}{Statement}

\begin{document}
\title[]{Math 220A: Midterm Examination\\
        November 2, 2017\\
        Daniel Halmrast\\
        Time taken: 9 hours}
\maketitle
\large
\section*{Problem 1}
Let $G$ be a non-abelian group of order 26, and let $H\leq G$ with $|H|=13$.
Prove that each element of $G$ which is not in $H$ is of order 2. Deduce that
if $a, b$ are elements of $G$ of orders $2$ and $13$ respectively, then
$aba=b\inv$.
\\
\begin{proof}
    We will first prove that any element in $G$ that is not in $H$ has
    order 2. We will do this by careful use of the class equation. In
    particular, letting $G$ act on itself by conjugation (the group action
    $(g,x)\mapsto gxg\inv$), we have that
    \[
        |G| = |Z(G)| + \sum|Orb_G(x)|
    \]
    Where $Z(G)$ is the center of $G$, and the sum is taken over
    representatives of the distinct conjugacy classes (orbits) of $G$.
    
    Since $G$ is non-abelian, we know that $|G|\neq|Z(G)|$. Furthermore, since
    $Z(G)\leq G$, we know that $|Z(G)|$ divides $|G|=26$. Thus, either
    $|Z(G)| = 13$, $|Z(G)| = 2$, or $|Z(G)| = 1$.

    Suppose that $|Z(G)| = 13$.
    Then, the class equation reads
    \[
        |G| = 26 = 13 + \sum|Orb_G(x)|
    \]
    Now, the orbit-stabilizer theorem tells us that $|Orb_G(x)| = [G:c_G(x)]$,
    and Lagrange's theorem tells us that this index divides the order of the
    group. Thus, we have that for each $x$, $|Orb_G(x)|$ is either $13$ or $2$.
    However, since the sum must equal 13, at least one term in the sum must be
    odd. Therefore, the only term in the sum is the single orbit of order $13$.

    This implies that for any $x\not\in Z(G)$ we have that $c_G(x)$ has index
    $13$, implying that $|c_G(x)| = 2$. This cannot be, however, since each
    element of the center $Z(G)$ clearly centralizes $x$, so $|c_G(x)|\geq 13$.

    Thus, $|Z(G)|$ cannot be $13$. So, assume $|Z(G)| = 2$.

    Now, we have a class equation that reads
    \[
    |G| = 26 = 2 + \sum|Orb_G(x)|
    \]
    In particular, $\sum|Orb_G(x)| = 24$. By the same argument as above, we
    must have that each orbit is of order $2$ or $13$. However, we cannot
    have an orbit of order $13$, since that would leave $24-13=11$ for the sum
    of the orders of the rest of the orbits, and each orbit must have order $2$
    or $13$, which clearly cannot sum to $11$.

    Thus, we have that each orbit has order $2$, with a stabilizer of index
    $2$. Now, a basic argument will show that, for $g\in G$, $\textrm{Stab}_G(g)
    = c_G(g)$, and for $g' = kgk\inv$ for any $k\in g$, $c_G(g') =
    kc_G(g')k\inv$.
    That is, $c_G(kgk\inv) = kc_G(g)k\inv$ for any $g,k\in G$.

    Since each orbit has only two elements in it (say, $g$ and $g'=kgk\inv$),
    it must be that the conjugcacy class of centralizers of $g$ has at most
    two elements. Since each centralizer has order $13$, it is conjugate to
    $H$, but since the number of conjugates of $H$ must be equal to $1\mod 13$,
    it must be that $H$ is the only element of this conjugacy class. Thus,
    for each $g$ in some orbit, we have that $c_G(g) = H$.

    Suppose that $x\in G$ is in some orbit (with elements $x$ and $x'$).
    Then, we have that the centralizer of $x$ is $H$. Now, let $k\not\in H$ be
    arbitrary. Since $k\not\in H$, $kxk\inv =x'$. And since $k$ doesn't
    centralize $x'$ either, we have that $kx'k\inv = k^2x(k\inv)^2 = x$.
    
    Now, the order of $k$ is either trivial, $2$, $13$, or $26$. If the order
    of $k$ is $1$, then $k=e\in H$ which contradicts $k\not\in H$. If the order
    of $k$ is $13$, then $\langle k \rangle$ has order $13$, and is then equal
    to $H$, which contradicts $k\not\in H$. If $k$ has order $26$, then $\langle
    k\rangle = G$, which cannot be since $G$ is non-abelian.
    
    Thus, $|k| = 2$ as desired.

    Suppose instead that $|Z(G)| = 1$. In this case, we must have some
    orbit of order $13$, along with some orbits of order $2$. In particular,
    applying the above argument to an orbit of order $2$ completes the argument
    that each $k\not\in H$ has order $2$ as desired.
    \\
    \\
    Now, let's use this result to prove that for elements $a,b$ in $G$ with
    $|a|=2$ and $|b|=13$, $aba=b\inv$.

    To do so, we observe that this is equivalent to showing that $(ab)^2 = e$.
    So, consider the element $ab$. Since $H$ is of prime order, each element of
    $H$ has order $13$. Thus, $a$ cannot be in $H$. In particular, this means
    that (since $H$ is cyclic), $ab\not\in H$ as well.

    By the above result, we have that $|ab| = 2$, which implies that
    $(ab)^2 = e$ as desired.
\end{proof}


\newpage

\section*{Problem 2}
For any $\sigma\in S_n$, we say $\sigma$ has a cycle type
$(r_1,\ldots,r_n)$ where each $r_i\in \N\cup\{0\}$, and $\sum_{i=1}^nir_i = n$,
if $\sigma = \sigma_1\ldots\sigma_k$ where each $\sigma_i$ is disjoint from the
others, and $r_i$ of them have length $i$.

\subsection*{Part i}
For $x\in G$, what is the cycle type of $\rho(X)$, where $\rho$ is the canonical
embedding of $G$ into $S_{|G|}$?
\\
\begin{proof}
    Let $|x| = r$, and let $|G| = rs$ for some $r$ and $s$. Now, we first
    observe that the map $x\mapsto l(x)\in S_{|G|}$ (where $l(x)$ is left-multiplication by
    $x$) has trivial kernel, and is thus an isomorphism onto its image.
    So, this group action is an embedding of $G$ into $S_{|G|}$, and we will
    identify $l(x)$ with $\rho(x)$.

    Furthermore, consider the subgroup $\langle x\rangle$ of order $r$.
    Clearly, if $l(x^n)g=g$, then $x^n = e$ and $n=r$. Therefore, for each
    $g\in G$, it is in a $\langle x\rangle$-orbit of order $r$ (formed by
    repeated application of $l(x)$). So, each orbit of $\langle x\rangle$ has
    order $r$. Observing that the orders partition $G$, we have that
    \[
        |G| = \sum|Orb(g)|
    \]
    where the sum is taken over representatives of distinct orbits. Since 
    each orbit has order $r$, and $G$ has order $rs$, we must have exactly $s$
    disjoint orbits.

    So, the cycle decomposition of $x$ has $S$ cycles of length $r$,
    and has a cycle type of $(0,\ldots,s_{(r)},\ldots,0)$. That is,
    $r_r = s$ and the rest are zero.
\end{proof}

\subsection*{Part ii}
Suppose that a group of order $2^rs$, with $s$ odd, contains an element $x$ of
order $2^r$. Show that for $r>0$, $G$ has a normal subgroup of index $2$.
\\
\\
\begin{proof}
    Since $x$ has order $2^r$, by the above argument it is represented by
    $s$ disjoint $2^r$-cycles, with signature 
    \[
        \varepsilon(\rho(x)) = \varepsilon(2^r\textrm{-cycle})^s
    \]
    Since the signature of a $2^r$-cycle is odd, and $s$ is odd, the signature
    of $\rho(x)$ is odd as well.

    Thus, the representation $\rho(x)$ is not an element of $A_{|G|}$, and by
    an argument made in homework 1, exactly half of the elements of $G$
    are in $A_{|G|}$. Therefore, the subgroup $G\cap A_{|G|}$ has order
    $\frac{|G|}{2}$, and has index $[G:G\cap A_{|G|}] = 2$ as desired.
    
    Furthermore, this subgroup is normal, since conjugation of $\tau\in
    A_{|G|}$ with any element of $G$ will not change the signature of $\tau$, so
    it will stay in $A_{|G|}$.
    \\
    \\
    (The argument for $G\cap A_{|G|}$ being exactly half of $G$ is recreated
    below).

    Let $G$ be a subset of $S_n$, with $G\not\subset A_n$. Then, there is
    at least one element $\sigma\in G$ that is odd. Now, consider the sets
    $T_1$ and $T_2$, containing the even and odd permutations of $G$
    respectively.
    
    Clearly, the map from $T_1$ to $T_2$ defined by $\tau\mapsto\sigma\tau$
    is a bijection of sets, since it is invertible by $\sigma\inv$.
    Thus, the two sets have the same cardinality, and exactly half the elements
    of $G$ are even, as desired.
\end{proof}

\newpage

\section*{Problem 3}
\subsection*{Part i}
State Sylow's theorems.
\\
\begin{statement}
Sylow's theorems are as follows:

    For a finite group $G$, and prime $p$ such that $p$ divides $|G|$ and
    $p^{\alpha}$ is the highest power of $p$ that divides $|G|$, the following
    hold:
    \begin{enumerate}
        \item There exists a subgroup of $G$ with order $p^{\alpha}$. This
            subgroup is called a Sylow-$p$ subgroup of $G$.
        \item Every Sylow-$p$ subgroup lies in the same conjugacy class. That
            is, given two Sylow-$p$ subgroups $P,S$, there exists $g\in G$ such
            that $P=gSg\inv$.
        \item The number of Sylow-$p$ subgroups of $G$ is equal to $1\mod p$.
    \end{enumerate}

\end{statement}
\subsection*{Part ii}
Let $G$ be a finite, simple, non-abelian group. Let $p$ be a prime number. Show
that if $p$ divides $|G|$, then $|G|$ divides $\frac{n_p!}{2}$, where $n_p$ is
the number of Sylow-$p$ subgroups of $G$.
\\
\begin{proof}
By Sylow's second theorem, we know that all of the Sylow-$p$ subgroups of $G$
    lie in the same conjugacy class. Now, consider the action of $G$ on this
    conjugacy class by conjugation
    \[
        (g,S)\mapsto gSg\inv
    \]
    for $g\in G$ and $S$ a Sylow-$p$ subgroup.

    This action defines a map $\phi:G\to S_{n_p}$.
    Now, the kernel $\ker(\phi)$ is normal in $G$, so by the fact that $G$ is
    simple, it must be trivial. (Since the orbit of the action is all of the
    conjugacy class, the kernel cannot be all of $G$, therefore, it must be just
    $\{e\}$.)

    Therefore, $\phi$ is injective. In particular, $G$ is isomorphic to a
    subgroup of $S_{n_p}$, which we will, for ease of notation, identify with
    $G$.
    
    Suppose that $G$ were not a subgroup of $A_{n_p}$. Then, we will show that
    the subgroup $G\cap A_{n_p}$ (which contains $\frac{|G|}{2}$ elements by the
    previous exercise and thus cannot be trivial or equal to $G$) is normal in
    $G$. To do so, let $\sigma\in A_{n_p}\cap G$. In particular, $\sigma$ is
    even.
    Now, for any $\tau\in G$, the element $\tau\sigma\tau\inv$ has even
    signature, since $\sigma$ is even, and the signature of $\tau$ is equal to
    the signature of $\tau\inv$. Thus, $\tau$ normalizes $\sigma$, and since
    $\sigma$ and $\tau$ were arbitrary, it follows that all of $G$ normalizes
    all of $A_{n_p}\cap G$, and thus $A_{n_p}\cap G$ is normal in $G$, which is 
    a contradiction.

    So, $G$ is a subgroup of $A_{n_p}$, and thus its order must divide the order
    of $A_{n_p}$.

    Thus, $|G|$ divides $\frac{n_p!}{2}$.
\end{proof}

\subsection*{Part iii}
Let $G$ be a group of order $48$. Show that $G$ is not simple.
\\
\begin{proof}
    To begin with, we observe that $48 = 2^4\times3$.

    Now, let's consider the Sylow-$2$ subgroups of $G$, along with the action
    of $G$ by conjugation. Now, Sylow's $2^{nd}$ theorem guarantees that all
    the Sylow-$2$ subgroups are in the same orbit. Thus, by the orbit-stabilizer
    theorem, we have that, for $P$ a Sylow-$2$ subgroup,
    \[
        \#(\textrm{Sylow-$2$ subgroups}) = |Orb_G(P)|=[G:N_G(P)]
    \]
    Furthermore, $N_G(P)$ contains $P$, so it has order at least $2^4$.
    If $|N_G(P)| = |G|$, then $P$ would be normal and $G$ would not be simple.

    So, it follows that $|N_G(P)| = 2^4$. Then $[G:N_G(P)] = 3$, and there are $3$ Sylow-$2$
    subgroups of $G$. Thus, by the argument from part ii, along with the
    observation that $48$ does not divide $\frac{3!}{2} = 3$, we know that $G$
    is not simple.
\end{proof}

\subsection*{Part iv}
Find a group of order 48 that has no normal Sylow-$2$ subgroup.
\\
\\
Consider the dihedral group $D_{24}$, which has $24\times 2=48$ elements. In
particular, $D_8$ is a subgroup of $D_{24}$ of order $8\times 2=16$, but is not
normal in $D_{24}$. This can be clearly illustrated by conjugation with the
smallest rotation in $D_{24}$. If one considers conjugating a reflection by this
rotation, it is clear that the rotation will set the octagon off-axis, and the
reflection will be performed off the symmetry axis of the octagon, leading to a
transformation not in $D_8$.

Thus, $D_8$ is not normal in $D_{24}$, as desired.

\newpage

\section*{Problem 4}
Let $G$ be a simple group of order $168=2^3\times3\times7$.

\subsection*{part i}
$G$ has just $8$ Sylow-$7$ subgroups, and the normalizer of each is of order
$21$.
\\
\begin{proof}
We know from before that the number of Sylow-$p$ subgroups is equal to the index
    of the normalizer of any particular Sylow-$p$ subgroup. Thus, the second
    statement follows immediately from the first.

    We note first that the normalizer of each Sylow-$7$ subgroup has order at
    least $7$, which means it has index at most $24$. Furthermore, we know that
    the number of Sylow-$7$ subgroups must be congruent to $1\mod 7$, which
    limits it to either $8$, $15$, or $22$.

    However, the only one of these that divides the group order is $8$, and
    since the index of the normalizer (which is the size of the conjugacy class)
    of a Sylow-$p$ subgroup must divide the order of the group, it must be that
    $G$ only has $8$ Sylow-$7$ subgroups.

    As stated above, this means that each normalizer has index $8$, and order
    $21$.
\end{proof}

\subsection*{Part ii}
Show that $G$ is isomorphic to a subgroup of $S_8$.
\\
\begin{proof}
Consider the action of $G$ by conjugation on the Sylow-$7$ subgroups of $G$.
    This defines a homomorphism $\phi:G\to S_8$. Furthermore, since $G$ is
    simple, and kernels are normal in their group, $\ker \phi$ must be either
    trivial or the whole group. However, since $G$ acts transitively, it must be
    that the kernel is trivial. Thus, $\phi$ is monic, and defines an
    isomorphism onto its image, which is a subgroup of $S_8$ as desired.
\end{proof}

\subsection*{Part iii}
Show that $G$ has no elements of order $14$ or $21$.
\\
\begin{proof}
    Suppose for a contradiction that $x^{14} = e$ for some $x\in G$.
    Now, since $x^2$ generates a subgroup of order $7$, $\langle x^2\rangle = P$
    is a Sylow-$7$ subgroup of $G$, and thus has a normalizer $N_G(P)$ with
    order $21$. But, since $\langle x\rangle$ also normalizes $P$, we have
    \[
        \langle x\rangle \leq N_G(P)
    \]
    which cannot be, since $|x|=14$ does not divide $|N_G(P)| = 21$.

    Suppose, then, that there is some $x\in G$ for which $x^{21}=e$. In
    particular, the element $x^7$ generates a Sylow-$3$ subgroup, and all of
    $\langle x\rangle$ normalizes it. But this contradicts part v (proven
    later without using this result), so such an $x$ cannot exist.
\end{proof}

\subsection*{Part iv}
Show that the normalizer of each Sylow-$7$ subgroup contains just $7$ Sylow-$3$
subgroups.
\\
\begin{proof}
    To begin with, we note that $|N_G(P_7)|=21$, where $P_7$ is a Sylow-$7$
    subgroup.

    Now, we know that from the perspective of $N_G(P_7)$, we must have for the
    Sylow-$3$ subgroup $P_3$
    \[
        \#(\textrm{Sylow-3 subgroups in $N_G(P_7)$}) = [N_G(P_7):N_G(P_3)]
    \]
    which must divide $21$, but must also be at least $3$. That is, the number
    of Sylow-$3$ subgroups of $N_G(P_7)$ must be either $3$, $7$, or $21$.

    Since we also know that the number of Sylow-$3$ subgroups is congruent
    to $1\mod 3$, our only option for the number of Sylow-$3$ subgroups is $7$,
    as desired.
\end{proof}

\subsection*{Part v}
Show that $G$ has $28$ Sylow-$3$ subgroups, and the normalizer of each has order
$6$.
\\
\begin{proof}
We know from the previous result that the number of Sylow-$3$ subgroups of $G$
    is at least $7$. We also note that the normalizer of each must be of order
    at least $3$, so the index of the normalizer (equal to the number of
    Sylow-$3$ subgroups) is at most $2^3\times 7=56$.
    
    We also know that this number is congruent to $1\mod 3$,
    which leaves us with the options $7+3n$ for $n\leq16$. Furthermore, this number
    must divide the order of the group, which leaves us with the options
    $7$ and $28$.

    The number of Sylow-$3$ subgroups of $G$ cannot be $7$. To see this, assume
    that there are only $7$ Sylow-$3$ subgroups. Then, for some Sylow-$7$
    subgroup $P_7$, we have that all of the Sylow-$3$ subgroups are in
    $N_G(P_7)$ by the above argument. This means that for any Sylow-$3$ subgroup
    $P_3$, any $g\in G$, there is some $k\in N_G(P_7)$ such that
    \[
        \begin{aligned}
            gP_3g\inv &= kP_3k\inv\\
            k\inv gP_3 g\inv k &= P_3
        \end{aligned}
    \]
    which implies that the element $k\inv g=n$ for some $n\in N_G(P_3)$, and
    $g=kn$, This means that $G$
    can be written as the product $N_G(P_3)N_G(P_7)$. However, this contradicts
    $G$ being simple.

    Thus, the number of Sylow-$3$ subgroups is $28$, which implies the index
    of each normalizer is $28$, and the order of each normalizer is
    $\frac{168}{28}=6$, as desired.
\end{proof}

\end{document}
