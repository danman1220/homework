\documentclass[12pt,reqno]{amsart}
\usepackage{amssymb}
\usepackage{amscd}
\usepackage{amsxtra}
\usepackage[mathscr]{eucal}

\setlength{\oddsidemargin}{0cm}
\setlength{\evensidemargin}{0in}
\setlength{\textwidth}{16.5cm}
\setlength{\topmargin}{0.35cm}
\setlength{\textheight}{8.5in}
\renewcommand{\baselinestretch}{1.33}

\pagestyle{plain}

\newcommand{\inv}{^{-1}}

\begin{document}
\title[]{Math 220A: Midterm Examination\\
        November 2, 2017\\
        Daniel Halmrast}
\maketitle
\large
\section*{Problem 1}
Let $G$ be a non-abelian group of order 26, and let $H\leq G$ with $|H|=13$.
Prove that each element of $G$ which is not in $H$ is of order 2. Deduce that
if $a, b$ are elements of $G$ of orders $2$ and $13$ respectively, then
$aba=b\inv$.
\\
\begin{proof}
    We will first prove that any element in $G$ that is not in $H$ has
    order 2. We will do this by careful use of the class equation. In
    particular, letting $G$ act on itself by conjugation (the group action
    $(g,x)\mapsto gxg\inv$), we have that
    \[
        |G| = |Z(G)| + \sum|Orb_G(x)|
    \]
    Where $Z(G)$ is the center of $G$, and the sum is taken over
    representatives of the distinct conjugacy classes (orbits) of $G$.
    
    Since $G$ is non-abelian, we know that $|G|\neq|Z(G)|$. Furthermore, since
    $Z(G)\leq G$, we know that $|Z(G)|$ divides $|G|=26$. Thus, either
    $|Z(G)| = 13$, $|Z(G)| = 2$,or $|Z(G)| = 1$.

    Suppose that $|Z(G)| = 13$. In particular, $Z(G)$ is a Sylow-13 subgroup
    of $G$. Furthermore, since $Z(G)$ is fixed by conjugation, the second Sylow
    theorem tells us that it is the only Sylow-13 subgroup of $G$. Thus,
    $Z(G) = H$.

    Now, the class equation reads
    \[
        |G| = 26 = 13 + \sum|Orb_G(x)|
    \]
    Now, the orbit-stabilizer theorem tells us that $|Orb_G(x)| = [G:c_G(x)]$,
    and Lagrange's theorem tells us that this index divides the order of the
    group. Thus, we have that for each $x$, $|Orb_G(x)|$ is either $13$ or $2$.
    However, since the sum must equal 13, at least one term in the sum must be
    odd. Therefore, the only term in the sum is the single orbit of order $13$.

    This implies that for any $x\not\in Z(G)$ we have that $c_G(x)$ has index
    $13$, implying that $|c_G(x)| = 2$. This cannot be, however, since each
    element of the center $Z(G)$ clearly centralizes $x$, so $|c_G(x)|\geq 13|$.

    Thus, $|Z(G)|$ cannot be $13$. So, assume $|Z(G)| = 2$.

    Now, we have a class equation that reads
    \[
    |G| = 26 = 2 + \sum|Orb_G(x)|
    \]
    In particular, $\sum|Orb_G(x)| = 24$. By the same argument as above, we
    must have that each orbit is of order $2$ or $13$. However, we cannot
    have an orbit of order $13$, since that would leave $24-13=11$ for the sum
    of the orders of the rest of the orbits, and each orbit must have order $2$
    or $13$, which clearly cannot sum to $11$.

    Thus, we have that each orbit has order $2$, with a stabilizer of index
    $2$. 
\end{proof}

\end{document}
