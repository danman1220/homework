%%%%%%%%%%%%%%%%%%%%%%%%%%%%%%%%%%%%%%%%%
% Short Sectioned Assignment
% LaTeX Template
% Version 1.0 (5/5/12)
%
% This template has been downloaded from:
% http://www.LaTeXTemplates.com
%
% Original author:
% Frits Wenneker (http://www.howtotex.com)
%
% License:
% CC BY-NC-SA 3.0 (http://creativecommons.org/licenses/by-nc-sa/3.0/)
%
%%%%%%%%%%%%%%%%%%%%%%%%%%%%%%%%%%%%%%%%%

%----------------------------------------------------------------------------------------
%	PACKAGES AND OTHER DOCUMENT CONFIGURATIONS
%----------------------------------------------------------------------------------------

\documentclass[paper=a4, fontsize=11pt]{scrartcl} % A4 paper and 11pt font size

\usepackage[T1]{fontenc} % Use 8-bit encoding that has 256 glyphs
\usepackage{fourier} % Use the Adobe Utopia font for the document - comment this line to return to the LaTeX default
\usepackage[english]{babel} % English language/hyphenation
\usepackage{amsmath,amsfonts,amsthm} % Math packages
\usepackage{mathrsfs}

\usepackage{lipsum} % Used for inserting dummy 'Lorem ipsum' text into the template

\usepackage{sectsty} % Allows customizing section commands
\allsectionsfont{\centering \normalfont\scshape} % Make all sections centered, the default font and small caps

\usepackage{fancyhdr} % Custom headers and footers
\pagestyle{fancyplain} % Makes all pages in the document conform to the custom headers and footers
\fancyhead{} % No page header - if you want one, create it in the same way as the footers below
\fancyfoot[L]{} % Empty left footer
\fancyfoot[C]{} % Empty center footer
\fancyfoot[R]{\thepage} % Page numbering for right footer
\renewcommand{\headrulewidth}{0pt} % Remove header underlines
\renewcommand{\footrulewidth}{0pt} % Remove footer underlines
\setlength{\headheight}{13.6pt} % Customize the height of the header

\numberwithin{equation}{section} % Number equations within sections (i.e. 1.1, 1.2, 2.1, 2.2 instead of 1, 2, 3, 4)
\numberwithin{figure}{section} % Number figures within sections (i.e. 1.1, 1.2, 2.1, 2.2 instead of 1, 2, 3, 4)
\numberwithin{table}{section} % Number tables within sections (i.e. 1.1, 1.2, 2.1, 2.2 instead of 1, 2, 3, 4)

\setlength\parindent{0pt} % Removes all indentation from paragraphs - comment this line for an assignment with lots of text

%----------------------------------------------------------------------------------------
%	TITLE SECTION
%----------------------------------------------------------------------------------------

\newcommand{\horrule}[1]{\rule{\linewidth}{#1}} % Create horizontal rule command with 1 argument of height

\title{	
\normalfont \normalsize 
\textsc{Thermal Physics} \\ [25pt] % Your university, school and/or department name(s)
\horrule{0.5pt} \\[0.4cm] % Thin top horizontal rule
\huge Problem Set 2: 1.17, 1.18, 1.21 \\ % The assignment title
\horrule{2pt} \\[0.5cm] % Thick bottom horizontal rule
}

\author{Daniel Halmrast} % Your name

\date{\normalsize\today} % Today's date or a custom date

\begin{document}

\maketitle % Print the title

%----------------------------------------------------------------------------------------
%	PROBLEM 1
%----------------------------------------------------------------------------------------

\section*{Problem 1.17}
\subsection*{Part A}
Problem: For each temperature, calculate the second term in the Virial expansion
$\frac{B(T)}{\frac{V}{n}}$ for $N_2$ at atmospheric pressure.
\\
\\
Solution:
\\
\begin{tabular}{l | l | l}
\textbf{T (K)}  & \textbf{B ($\frac{cm^3}{mol}$)} & \textbf{$\frac{B(T)}{\frac{V}{n}}$}\\
100             & -160          & $-0.0194$\\
200             & -35           & $-0.00213$\\
300             & -4.2          & $-1.71*10^{-4}$\\
400             & 9.0           & $2.74*10^{-4}$\\
500             & 16.9          & $4.119*10^{-4}$\\
600             & 21.3          & $4.33*10^{-4}$
\end{tabular}
\\
Since the expansion term is so small, this shows the ideal gas law is quite valid for high
temperature $N_2$.

\subsection*{Part B}
As the molecules get close to each other (low temperature), the intermolecular forces will
begin to cause attraction between the molecules. This attraction will \emph{lessen} the
pressure of the gas compared to the ideal gas law prediction.

\subsection*{Part C}
By direct computation:
\[
\begin{aligned}
(P+\frac{an^2}{V^2})(V-nb) &= nRT\\
(P+\frac{an^2}{V^2})(1-\frac{nb}{V})V &= nRT\\
(P+\frac{an^2}{V^2})V &= nRT\left(1-\frac{nb}{V}\right)^{-1}\\
(P+\frac{an^2}{V^2})V &= nRT\left(1-\frac{nb}{V} + b^2\frac{n^2}{V^2}\right)\\
PV + \frac{an^2}{V}   &= nRT\left(1-\frac{nb}{V} + b^2\frac{n^2}{V^2}\right)\\
PV                    &= nRT\left(1-\frac{nb}{V} + b^2\frac{n^2}{V^2} - \frac{an}{VRT}\right)\\
PV &= nRT\left(1-(b+\frac{a}{RT})\frac{n}{V} + b^2\frac{n^2}{V^2}\right)
\end{aligned}
\]

\subsection*{Part D}
Using values of $a=1.31*10^8$ and $b=48.5$ (Found using the data for temperatures 400 and 500 kelvin),
the graph of $B(T)$ looks like a $\frac{1}{x}$ graph that reaches valid ($|B(T)| < 100$) values past about
100 kelvin, so the ideal gas law will hold quite well for 100k and above tempearatures.

\section*{Problem 1.18}
We know that the rms speed of a molecule at a temperature is given by the equation
\[
\sqrt{\frac{3kT}{m}}
\]
Which, for molecular nitrogen ($N_2$, $m=4.7*10^{-26}kg$), yields
\[
v_{rms} = 514\frac{m}{s}
\]


\section*{Problem 1.21}
Since the overall speed of the hailstones is $15\frac{m}{s}$, and it hits at a $45^{\circ}$ angle,
the $v_x$ component of the velocity will be $\frac{15}{\sqrt{2}}\frac{m}{s}\approx 10.6$.

Thus, the momentum imparted to the window pane is $2p=2mv=2*0.002kg*10.6\frac{m}{s} = 0.042 \frac{kg*m}{s}$.
The force imparted to the window is $\frac{\Delta p}{\Delta t}$ where $\frac{1}{\Delta t}$ is the
frequency of arrival, at 30 times per second. Thus, $F = 0.042*30 = 1.26 N$, and
$P = \frac{F}{A} = 2.54 Pa$.

Compare this to the atmospheric pressure $P_{atm} = 1.013*10^5$, and you will see the hailstones
exert a miniscule ($0.002\%$) pressure on the window compared to the atmosphere.



%----------------------------------------------------------------------------------------

\end{document}
