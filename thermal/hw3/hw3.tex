%%%%%%%%%%%%%%%%%%%%%%%%%%%%%%%%%%%%%%%%%
% Short Sectioned Assignment
% LaTeX Template
% Version 1.0 (5/5/12)
%
% This template has been downloaded from:
% http://www.LaTeXTemplates.com
%
% Original author:
% Frits Wenneker (http://www.howtotex.com)
%
% License:
% CC BY-NC-SA 3.0 (http://creativecommons.org/licenses/by-nc-sa/3.0/)
%
%%%%%%%%%%%%%%%%%%%%%%%%%%%%%%%%%%%%%%%%%

%----------------------------------------------------------------------------------------
%	PACKAGES AND OTHER DOCUMENT CONFIGURATIONS
%----------------------------------------------------------------------------------------

\documentclass[paper=a4, fontsize=11pt]{scrartcl} % A4 paper and 11pt font size

\usepackage[T1]{fontenc} % Use 8-bit encoding that has 256 glyphs
\usepackage{fourier} % Use the Adobe Utopia font for the document - comment this line to return to the LaTeX default
\usepackage[english]{babel} % English language/hyphenation
\usepackage{amsmath,amsfonts,amsthm} % Math packages
\usepackage{mathrsfs}

\usepackage{lipsum} % Used for inserting dummy 'Lorem ipsum' text into the template

\usepackage{sectsty} % Allows customizing section commands
\allsectionsfont{\centering \normalfont\scshape} % Make all sections centered, the default font and small caps

\usepackage{fancyhdr} % Custom headers and footers
\pagestyle{fancyplain} % Makes all pages in the document conform to the custom headers and footers
\fancyhead{} % No page header - if you want one, create it in the same way as the footers below
\fancyfoot[L]{} % Empty left footer
\fancyfoot[C]{} % Empty center footer
\fancyfoot[R]{\thepage} % Page numbering for right footer
\renewcommand{\headrulewidth}{0pt} % Remove header underlines
\renewcommand{\footrulewidth}{0pt} % Remove footer underlines
\setlength{\headheight}{13.6pt} % Customize the height of the header

\numberwithin{equation}{section} % Number equations within sections (i.e. 1.1, 1.2, 2.1, 2.2 instead of 1, 2, 3, 4)
\numberwithin{figure}{section} % Number figures within sections (i.e. 1.1, 1.2, 2.1, 2.2 instead of 1, 2, 3, 4)
\numberwithin{table}{section} % Number tables within sections (i.e. 1.1, 1.2, 2.1, 2.2 instead of 1, 2, 3, 4)

\setlength\parindent{0pt} % Removes all indentation from paragraphs - comment this line for an assignment with lots of text

%----------------------------------------------------------------------------------------
%	TITLE SECTION
%----------------------------------------------------------------------------------------

\newcommand{\horrule}[1]{\rule{\linewidth}{#1}} % Create horizontal rule command with 1 argument of height

\title{	
\normalfont \normalsize 
\textsc{Thermal Physics} \\ [25pt] % Your university, school and/or department name(s)
\horrule{0.5pt} \\[0.4cm] % Thin top horizontal rule
\huge Problem Set 3: 1.23, 1.28, 1.31-34 \\ % The assignment title
\horrule{2pt} \\[0.5cm] % Thick bottom horizontal rule
}

\author{Daniel Halmrast} % Your name

\date{\normalsize\today} % Today's date or a custom date

\begin{document}

\maketitle % Print the title

%----------------------------------------------------------------------------------------
%	PROBLEM 1
%----------------------------------------------------------------------------------------

\section*{Problem 1.23}
Problem: Calculate the total thermal energy in a liter of helium at STP. Then, repeat for
a liter of air.
\\
\\
Solution:
Thermal energy is given as
\[
U_{thermal} = Nf\frac{1}{2}kT
\]
For a liter of helium at STP, $\frac{PV}{kT}=N=2.44*10^{22}$.
Then, $U_{thermal}=2.44*10^{22}*3*\frac{1}{2}k*300k=152J$.

For a liter of air, $N_{air}=N_{helium}$, but air has 5 DoF compared to helium's 3.
Thus, $U_{air}=\frac{5}{3}U_{helium}=253J$.

\section*{Problem 1.28}
Problem: Estimate how long it takes to bring a cup of water to boiling in a 600-Watt microwave,
where all the energy ends up in the water. Explain why no heat is involved in the process.
\\
\\
Solution:
Assuming the water starts at 20 celcius, and ends at 100 celcius, $\Delta T=80K$.
One cup of water weighs approximately 200 grams, so the energy needed to heat the cup of
water to boiling is
\[
\Delta U = 80K*200g*4.186\frac{J}{gK}
\Delta U \approx 67000
\]
Thus, for a 600W microwave, to deliver 67000 Joules of energy would take 110 seconds.

\section*{Problem 1.31}
Problem: Imagine a container of helium, initial volume 1 Liter, initial pressure 1 atm,
final volume 3 Liters, and the process is linear on a PV diagram.

\subsection*{Part a}
Sketch a PV diagram for this process:
\\
\\
\\
\\
\subsection*{Part b}
Calculate the work done on the gas during this process:
\\
\\
Solution: Work is defined as $W=-\int_{V-i}^{V_f}P(V)dV$. Furthermore, it is assumed
that $P(V)=V$ Thus,
\[
\begin{aligned}
W &= -\int_1^3 aVdV\\
  &= -\frac{1}{2}V^2|_1^3\\
  &= \frac{1}{2}(1-9)\\
  &= -4
\end{aligned}
\]
Where the work is in atmosphere liters. In Joules, this yields $W=-405J$.

\subsection*{Part c}
Calculate the change in the helium's energy content during the process.
\\
\\
Solution: We'll assume that 0.1 mol of helium is in the container, for an initial temperature
of 122 kelvin. Then, the final temperature is given as
\[
T = \frac{PV}{nR}
\]
which, for $P=3atm$, $V=3L$, $n=0.1mol$, yields $T_f = 1097K$, for a $\Delta T = 975$. Thus,
\[
\begin{aligned}
\Delta U &= (0.1 mol)*N_a*f*\frac{1}{2}*k*\Delta T\\
         &= 1216J
\end{aligned}
\]

\subsection*{Part d}
Calculate the amount of heat added to the helium during this process.
\\
\\
Solution: Since $\Delta U = Q+W$, $Q=\Delta U - W$, and plugging in values from parts b and c
yields
\[
Q=1621J
\]

\subsection*{Part e}
Describe how you would cause the pressure to rise in this process.
\\
\\
Solution: By heating the gas as it expands at an appropriately high rate, the pressure of the
gas will also increase.


\section*{Problem 1.32}
Sketch a PV diagram of compressing an amount of water to $99\%$ of its initial volume at a
pressure of 200atm.
\\
\\
\\
\\
Estimate the work required to compress a liter.
\\
\\
Solution:
For an initial pressure of 1atm, initial volume of 1L, final pressure of 200atm, final volume
of .99L, and a linear relationship between P and V, $P(V)=19901 -19900V$.
\[
\begin{aligned}
W &= -\int_{V_i}^{V_f} P(V)dV\\
W &= -\int_1^{0.99} (19901 -19900V)dV\\
W &= 1.005
\end{aligned}
\]
Which, in freedom units, is $W=101J$.
This is a bit surprising, as 200atm is quite the pressure to reach. However, since the
volume barely changes, it makes sense why the energy needed to compress it is (relatively) low.

\section*{Problem 1.33}
Determine the sign of the value specified in reference to the diagrams on page 23:
\\
\begin{tabular}{l | l | l | l}
\textbf{Path} & \textbf{work on gas} & \textbf{$\Delta U$ gas} & \textbf{Q to gas}\\
        A        & Negative     & Positive      & Positive \\
        B        & Zero         & Positive      & Positive \\
        C        & Positive     & Negative      & Negative \\
    Full Cycle   & Positive     & Zero          & Negative
\end{tabular}

\section*{Problem 1.34}
There is an ideal diatomic gas with vibrational modes frozen out ($f=5$). It undergoes
a process in a rectangular fasion from $P_1, V_1$ to $P_2,V_2$ and back.

\subsection*{Part a}

For each path, we will use the integral definition of work, and the equipartition theorem
combined with the ideal gas law for $\Delta U = PV\frac{f}{2}$.
\\
\begin{tabular}{l | l | l | l}
\textbf{Path} & \textbf{work on gas} & \textbf{$\Delta U$ gas} & \textbf{Q to gas}\\
        A        & Zero              & $(P_2-P_1)*V_1*\frac{5}{2}$        & $\Delta U$\\
        B        & $-(V_2-V_1)*P_2$  & $(V_2-V_1)*P_2*\frac{5}{2}$        & $\Delta U - W$ \\
        C        & Zero              & $(P_1-P_2)*V_2*\frac{5}{2}$        & $\Delta U$ \\
        D        & $-(V_1-V_2)*P_2$  & $(V_1-V_2)*P_1*\frac{5}{2}$        & $\Delta U - W$\\
\end{tabular}

\subsection*{Part b}
In part A, heat is added to the gas to increase its pressure. Then, in part B, the gas is
expanded and heat is added to keep pressure constant. C is the reversal of A, and D is the reversal
of B.

\subsection*{Part c}
Net work is calculated by finding the total area of the rectangle, which is clearly
$-(V_2-V_1)(P_2-P_1)$ (it's negative, since the higher pressure process expands the gas).
The net heat added to the gas will exactly cancel with the net work, since the end state is
the same (temperature) as the initial state, so $\Delta U_{net} = 0$.



%----------------------------------------------------------------------------------------

\end{document}
