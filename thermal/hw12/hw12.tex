%%%%%%%%%%%%%%%%%%%%%%%%%%%%%%%%%%%%%%%%%
% Short Sectioned Assignment
% LaTeX Template
% Version 1.0 (5/5/12)
%
% This template has been downloaded from:
% http://www.LaTeXTemplates.com
%
% Original author:
% Frits Wenneker (http://www.howtotex.com)
%
% License:
% CC BY-NC-SA 3.0 (http://creativecommons.org/licenses/by-nc-sa/3.0/)
%
%%%%%%%%%%%%%%%%%%%%%%%%%%%%%%%%%%%%%%%%%

%----------------------------------------------------------------------------------------
%	PACKAGES AND OTHER DOCUMENT CONFIGURATIONS
%----------------------------------------------------------------------------------------

\documentclass[paper=a4, fontsize=11pt]{scrartcl} % A4 paper and 11pt font size

\usepackage[T1]{fontenc} % Use 8-bit encoding that has 256 glyphs
\usepackage{fourier} % Use the Adobe Utopia font for the document - comment this line to return to the LaTeX default
\usepackage[english]{babel} % English language/hyphenation
\usepackage{amsmath,amsfonts,amsthm} % Math packages
\usepackage{mathrsfs}
\usepackage{graphicx}


\usepackage{sectsty} % Allows customizing section commands
\allsectionsfont{\centering \normalfont\scshape} % Make all sections centered, the default font and small caps

\usepackage{fancyhdr} % Custom headers and footers
\pagestyle{fancyplain} % Makes all pages in the document conform to the custom headers and footers
\fancyhead{} % No page header - if you want one, create it in the same way as the footers below
\fancyfoot[L]{} % Empty left footer
\fancyfoot[C]{} % Empty center footer
\fancyfoot[R]{\thepage} % Page numbering for right footer
\renewcommand{\headrulewidth}{0pt} % Remove header underlines
\renewcommand{\footrulewidth}{0pt} % Remove footer underlines
\setlength{\headheight}{13.6pt} % Customize the height of the header

\numberwithin{equation}{section} % Number equations within sections (i.e. 1.1, 1.2, 2.1, 2.2 instead of 1, 2, 3, 4)
\numberwithin{figure}{section} % Number figures within sections (i.e. 1.1, 1.2, 2.1, 2.2 instead of 1, 2, 3, 4)
\numberwithin{table}{section} % Number tables within sections (i.e. 1.1, 1.2, 2.1, 2.2 instead of 1, 2, 3, 4)

\setlength\parindent{0pt} % Removes all indentation from paragraphs - comment this line for an assignment with lots of text

%----------------------------------------------------------------------------------------
%	TITLE SECTION
%----------------------------------------------------------------------------------------

\newcommand{\horrule}[1]{\rule{\linewidth}{#1}} % Create horizontal rule command with 1 argument of height

\title{	
\normalfont \normalsize 
\textsc{Thermal Physics} \\ [25pt] % Your university, school and/or department name(s)
\horrule{0.5pt} \\[0.4cm] % Thin top horizontal rule
\huge Problem Set: 2.42 3.1 3.3 3.5 3.6\\ % The assignment title
\horrule{2pt} \\[0.5cm] % Thick bottom horizontal rule
}

\author{Daniel Halmrast} % Your name

\date{\normalsize\today} % Today's date or a custom date

\begin{document}

\maketitle % Print the title

%----------------------------------------------------------------------------------------
%	PROBLEMS
%----------------------------------------------------------------------------------------

\section*{Problem 2.42}
\subsection*{Part a}
The dimensions of G are $\frac{m^3}{kgs^2}$, so the units of $\frac{GM}{c^2}$ will be in meters.

For a black hole of one solar mass, the radius will be about $1500m$.

\subsection*{Part b}
We know that the entropy for most systems is proportional to the number of particles
in the system, the entropy of the black hole will be on the order of the number of particles
used to make it.

\subsection*{Part c}
We want the number of low-energy photons that sum to the mass of a black hole. That is, the
total energy of all photons will be $Mc^2$.

The energy of an individual photon with wavelength $\lambda$ is given as $E=\frac{hc}{\lambda}$.
Set $\lambda$ to $\frac{GM}{c^2}$ to get $E=\frac{hc^3}{GM}$. So, divide $Mc^2$ by the energy
per photon to get
\[
    \begin{aligned}
        n_{\gamma} &= \frac{Mc^2}{\frac{hc^3}{GM}}\\
                   &= \frac{GM^2}{hc}
    \end{aligned}
\]
Then, entropy will be
\[
    S = Nk = \frac{GM^2}{hc}k
\]

\subsection*{Part d}
The entropy of a one solar mass black hole is (via wolfram)
\[
    S = 1.45\times 10^{54} \frac{J}{K}
    \]
This is a very large value for entropy, which makes sense as the black hole is the
largest-entropy state for a gravitational system.


%----------------------------------------------------------------------------------------

\end{document}
