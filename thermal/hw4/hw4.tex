%%%%%%%%%%%%%%%%%%%%%%%%%%%%%%%%%%%%%%%%%
% Short Sectioned Assignment
% LaTeX Template
% Version 1.0 (5/5/12)
%
% This template has been downloaded from:
% http://www.LaTeXTemplates.com
%
% Original author:
% Frits Wenneker (http://www.howtotex.com)
%
% License:
% CC BY-NC-SA 3.0 (http://creativecommons.org/licenses/by-nc-sa/3.0/)
%
%%%%%%%%%%%%%%%%%%%%%%%%%%%%%%%%%%%%%%%%%

%----------------------------------------------------------------------------------------
%	PACKAGES AND OTHER DOCUMENT CONFIGURATIONS
%----------------------------------------------------------------------------------------

\documentclass[paper=a4, fontsize=11pt]{scrartcl} % A4 paper and 11pt font size

\usepackage[T1]{fontenc} % Use 8-bit encoding that has 256 glyphs
\usepackage{fourier} % Use the Adobe Utopia font for the document - comment this line to return to the LaTeX default
\usepackage[english]{babel} % English language/hyphenation
\usepackage{amsmath,amsfonts,amsthm} % Math packages
\usepackage{mathrsfs}

\usepackage{sectsty} % Allows customizing section commands
\allsectionsfont{\centering \normalfont\scshape} % Make all sections centered, the default font and small caps

\usepackage{fancyhdr} % Custom headers and footers
\pagestyle{fancyplain} % Makes all pages in the document conform to the custom headers and footers
\fancyhead{} % No page header - if you want one, create it in the same way as the footers below
\fancyfoot[L]{} % Empty left footer
\fancyfoot[C]{} % Empty center footer
\fancyfoot[R]{\thepage} % Page numbering for right footer
\renewcommand{\headrulewidth}{0pt} % Remove header underlines
\renewcommand{\footrulewidth}{0pt} % Remove footer underlines
\setlength{\headheight}{13.6pt} % Customize the height of the header

\numberwithin{equation}{section} % Number equations within sections (i.e. 1.1, 1.2, 2.1, 2.2 instead of 1, 2, 3, 4)
\numberwithin{figure}{section} % Number figures within sections (i.e. 1.1, 1.2, 2.1, 2.2 instead of 1, 2, 3, 4)
\numberwithin{table}{section} % Number tables within sections (i.e. 1.1, 1.2, 2.1, 2.2 instead of 1, 2, 3, 4)

\setlength\parindent{0pt} % Removes all indentation from paragraphs - comment this line for an assignment with lots of text

%----------------------------------------------------------------------------------------
%	TITLE SECTION
%----------------------------------------------------------------------------------------

\newcommand{\horrule}[1]{\rule{\linewidth}{#1}} % Create horizontal rule command with 1 argument of height

\title{	
\normalfont \normalsize 
\textsc{Thermal Physics} \\ [25pt] % Your university, school and/or department name(s)
\horrule{0.5pt} \\[0.4cm] % Thin top horizontal rule
\huge Problem Set 4: 1.49, 1.50, 1.51, 1.53, 1.55 \\ % The assignment title
\horrule{2pt} \\[0.5cm] % Thick bottom horizontal rule
}

\author{Daniel Halmrast} % Your name

\date{\normalsize\today} % Today's date or a custom date

\begin{document}

\maketitle % Print the title

%----------------------------------------------------------------------------------------
%	PROBLEM 1
%----------------------------------------------------------------------------------------

\section*{Problem 1.49}
Problem: Combust one mole of $H_2$ with $\frac{1}{2}$ mole of $O_2$. How much heat
comes from a decrease in internal energy, and how much comes from work done by the
atmosphere?
\\
\\
Solution:
We know the total amount of heat generated is 286kJ, so we need only calculate one of the
two quantities, and subtract it from the total to get the other.

The work done by the atmosphere collapsing on the gas (when it turns to a negligible-volume
liquid) is given by
\[
\begin{aligned}
W &= -\int_{V_i}^{V_f} PdV = -P\Delta V = PV_i = RT\\
  &= 2500J
\end{aligned}
\]

So, the work done by the atmosphere is 2500J, and the decrease in internal energy is
\[
\begin{aligned}
\Delta U &= \Delta H - W\\
         &= 286kJ - 2.5kJ\\
         &= 284.5kJ
\end{aligned}
\]

\section*{Problem 1.50}
Problem:
Consider the combustion reaction
\[
CH_4 + 2O_2 \to CO_2 + 2H_2O_{(g)}
\]
at 298K and $10^5$Pa both before and after the reaction.

\subsection*{Part a}
Problem: Imagine the process of converting a mole of methane into its constituents.
What is $\Delta H$?
\\
\\
Solution:
The book gives $\Delta H_f(CH_4) = -74.81$, so the formation process gives off $74.81$kJ of
heat, and the decomposition takes $74.81$kJ to do.

\subsection*{Part b}
Problem: Now form a mole of $CO_2$ and two moles of $H_2O_{(g)}$, what is $\Delta H$?
\\
\\
Solution:
Well, the book gives $\Delta H_f(CO_2) = -393.51$kJ, and $\Delta H_f(H_2O_{(g)}) = -241.82$kJ.

\subsection*{Part c}
Problem: What is the total $\Delta H$ for the reaction?
\\
\\
Solution:
It will take $74.81$kJ to break up the methane, but then we will get $393.51+241.82=635.33$kJ
out of the reaction, for a net total of $635.33-74.81=560.52$kJ emitted from the reaction.

\subsection*{Part d}
Problem: If no other work is done, how much heat is given off?
\\
\\
Solution: If no other work is done, then all the energy went into heat, so there was
$560.52$kJ of heat emitted.

\subsection*{Part e}
Problem: What is the change in the energy of the system?
\\
\\
Solution: Since the degrees of freedom and the temperature remain constant, $\Delta U=0$ for
the reaction. If instead the water had condensed to a liquid form, the degrees of freedom would
change and the system would either gain or lose energy to make the temperature constant.

\subsection*{Part f}
The sun has a mass of $2\times 10^{30}$kg and gives off energy at a rate of $3.9\times 10^{26}$
watts. If the source of the sun's energy were ordinary combustion, how long could it last?
\\
\\
Solution:
Using wolfram alpha, I find that $2\times 10^{30}$kg of methane is about $1.32\times 10^{32}$mol.
To emit $3.9\times 10^{26}$J of energy, it would take
$\frac{3.9\times 10^{26}J}{5.6\times 10^5\frac{J}{mol}} = 7.0\times 10^{30}mol$.
So, the sun would burn $7\times 10^{30}$ moles per second, lasting a grand total of
\[
\frac{1.32\times 10^{32}mol}{7\times 10^{30}\frac{mol}{s}} = 19s
\]
So, the sun would last about 19 seconds before it burnt through all its methane.


\section*{Problem 1.51}
Problem: Determine the $\Delta H$ for the reaction
\[
C_6H_{12}O_6 + 6O_2 \to 6CO_2 +6H_2O
\]
\\
\\
Solution:
It will take $1273$kJ to break up glucose, and we will gain $393.51$kJ for each mol of
$CO_2$ we form, and $285.83$kJ for each mol of water we form.

This results in a net enthalpy of
\[
\Delta H = 1273 - 6*(393.51+285.83) = -2803kJ
\]


\section*{Problem 1.53}
Problem: Find the enthalpy of formation of atomic hydrogen, and determine the energy needed to
dissociate a single $H_2$ molecule, in eV.
\\
\\
Solution:
$\Delta H_f(H) = 217.97$kJ is the energy to dissociate a half mole of $H_2$, so we divide by
one half Avogadro's number to get $\Delta H_{single molecule}(H) = 4.52eV$.


\section*{Problem 1.55}

\subsection*{Part a}
Consider two particles orbiting each other. Show the gravitational potential energy is
$-2$ times the kinetic energy.
\\
\\
Solution:
For a particle of mass $m$ orbiting a center-of-gravity, orbital mechanics tells us
that the potential of a particle is related to its velocity by
\[
\begin{aligned}
\frac{\partial V}{\partial r} &= \frac{v^2}{r}\\
r\frac{\partial V}{\partial r}&= v^2
\end{aligned}
\]
Since the potential is a $\frac{1}{r}$ potential, $r\frac{\partial V}{\partial r} = -V$, and
thus
\[
-V = v^2
\]
Therefore, the kinetic energy, given by $KE = \frac{1}{2}mv^2 = \frac{-1}{2}mV = \frac{-1}{2}U_p$.


\subsection*{Part b}
Problem: If you add energy to the system, then wait for it to equilibriate, does the
average total kinetic energy increase or decrease?
\\
\\
Solution:
Since the total energy of the system is $U = KE + V = KE - 2KE = -KE$,
adding energy will decrease the kinetic energy.

\subsection*{Part c}
Problem: Express the total energy of a star in terms of its average temperature, and
calculate the heat capacity.
\\
\\
Solution:
Since $KE = \frac{3}{2}kT$, and the total energy of the star is $U=-N*KE$,
$U = \frac{-3N}{2}kT$, and thus $C_p = \frac{-3}{2}kN$

\subsection*{Part d}
Problem: Argue that the potential energy of the star should be a constant multiple of
$\frac{-GM^2}{R}$.
\\
\\
Solution: The gravitational constant $G$ has dimensions that make $V = G\frac{M}{r}$ consistent.
Thus, the dimensions of $\frac{-GM^2}{R}$ are dimensions of energy.

\subsection*{Part e}
Problem: Estimate the average temperature of the sun, with a mass $2\times 10^{30}$kg with radius
$7\times 10^8$m.
\\
\\
Solution: From part d, the total potential energy of the sun will be
\[
V = -\frac{GM^2}{R} = -3.814\times 10^{41}J
\]
Which gives a total kinetic energy of $KE = -\frac{1}{2}U_p = 1.9\times 10^{41}J$
So, the kinetic energy per particle, if there are $1\times 10^{57}$ protons in the sun,
is $KE_{p} = 1.6\times 10^{-16}J$
So, $1.6\times 10^{-16} = \frac{3}{2}kT$. Thus,the average temperature is $7.7\times 10^6K$.



%----------------------------------------------------------------------------------------

\end{document}
